% File: mfpguide.tex
% A part of mfpic 1.10 2012/12/03
%
%  Tutorial on mfpic
\documentclass[letterpaper]{article}

\usepackage[chapters]{mfpdoc}

\renewcommand{\thefigure}{\thesection.\arabic{figure}}

\newenvironment{mfpfig}[1]
{\figure[htb] \centering \refstepcounter{figure} \label{#1}}
{\par\medskip Figure \thefigure. \endfigure}

\ifpdf
  \usepackage[pdftex,final]{graphics}
\else
  \usepackage[dvips,final]{graphics}
\fi

\usepackage[metapost]{mfpic}
\opengraphsfile{guide}

\dotsize1pt

\ifpdf
\expandafter\usepackage\expandafter
  [\mfpHyOpts,plainpages=false,naturalnames=true]{hyperref}
\expandafter\pdfstringdefDisableCommands\expandafter
  {\mfpHyDisable}

\fi

\def\theHfigure{\thefigure}

\begin{document}

\title{\Mfp: A Short Introduction}
\author{Daniel H. Luecking\thanks{luecking at uark dot edu}}
\date{\mfpfiledate}

\maketitle
\tableofcontents

\chapter{Introduction}\label{sec:intro}

As this document aims only to instruct the reader in the building of
figures with \mfp{}, we will not be too concerned with the intricacies
of running programs in various operating systems and \TeX{}
distributions. What will be described here is the simplest case: a
command-line system in which commands are typed at a keyboard. To
simplify things further, we will assume that \mfp{} is used with the
\opt{metapost} option, in a \LaTeX{} document, with \pdfLaTeX{} as the
compiler. An appendix will discuss some of the differences when these
assumptions are not satisfied.

We will start right out with the ``Hello, world'' of \mfp{}. Construct a
\LaTeX{} document by typing the following in a text editor and saving it
as \file{first.tex}.
\begin{verbatim}
% first.tex
\documentclass{article}
  \usepackage[metapost]{mfpic}
  \opengraphsfile{myfigs}
\begin{document}
  My first figure:
  \begin{mfpic}[72]{-1}{1}{-1}{1}
    \ellipse{(0,0),1,.5}
  \end{mfpic}
  \closegraphsfile
\end{document}
\end{verbatim}
Run the command
\begin{ex}
    \texttt{pdflatex first}
\end{ex}
which should create several files, the two most important being
\file{first.pdf} and \file{myfigs.mp}. You can go ahead and open
\file{first.pdf}. You should see a 2 inch by 2 inch square with
something similar to `\texttt{\#1}' in the lower left corner. This shows
where the picture will be when it has been created.

Now run the command
\begin{ex}
    \texttt{mpost myfigs}
\end{ex}
which should create the file \file{myfigs.1}. This is an EPS file
(Encapsulated PostScript) and can be opened in GhostScript or GSview or
similar Postscript viewing program to see an ellipse.

If you are viewing \file{first.pdf} in Acrobat Reader or Adobe Reader,
you will need to close it. Now repeat the \pdfLaTeX{} step:
\begin{ex}
    \texttt{pdflatex first}
\end{ex}
and then view the file \file{first.pdf}. You should see something very
close to figure~\ref{fig1}.
\begin{mfpfig}{fig1}
  My first figure:
  \begin{mfpic}[72]{-1}{1}{-1}{1}
    \ellipse{(0,0),1,.5}
  \end{mfpic}
\end{mfpfig}

What can go wrong? According to Murphy's Law: anything. If \mfp{} is not
properly installed, one could obtain messages of files not found. If
that happens, determine (from your \TeX{} system's documentation) where
\TeX{} input files should go and make sure that \file{mfpic.sty},
\file{mfpic.tex} and \file{mfpicdef.tex} reside there. Similarly, find
out where \MP{} inputs should go and make sure that \file{grafbase.mp}
and \file{dvipsnam.mp} reside there. Then run whatever command your
\TeX{} system might require to ``update the filename database''. You may
safely ignore the message from \mfp{} itself that \file{myfigs.1} is not
found (on the first run of \pdfLaTeX). This file should be created only
after running \texttt{mpost}.

If you get an error message from \LaTeX{}, carefully check your typing.
Also check whether an older version of \mfp{} might have been used
instead of the current version. If you get an error message from \MP{}
do the same, especially checking the typing within the \env{mfpic}
environment. If you get a message from \MP{} that ``Grafbase'' believes
your \mfp{} installation may be broken, check the log files
(\file{first.log} and \file{myfigs.log}) to find out the locations of
these input files:
\begin{ex}
    \file{mfpic.tex} and \file{grafbase.mp}
\end{ex}
and make sure that both these files are from the most recently installed
\mfp{} package. If you are only evaluating \mfp{} without committing to
installing it, just make sure all the files mentioned in the previous
paragraphs are in the current directory.

If \pdfLaTeX{} complains it can't write on the file \file{first.pdf},
unload \file{first.pdf} from your pdf viewer and try again.

If the figures look a little choppy in Acrobat Reader, turn on ``smooth
line art'' in the edit preferences dialogue.

I will assume that eventually all went well and you are now able to
obtain the ellipse of figure~\ref{fig1}. Each time you change an
\env{mfpic} environment or the options to the package, you potentially
change the file \file{myfigs.mp} produced and you should repeat the
sequence:
\begin{verbatim}
pdflatex first
mpost myfigs
pdflatex first
\end{verbatim}
to be sure of seeing the changes.

One thing you might notice about figure~\ref{fig1} is that the ellipse
is positioned quite a bit above the base line of the text. This is
because \mfp{} reserves the amount of space specified in the arguments
of the \env{mfpic} environment. These arguments were
\verb$[72]{-1}{1}{-1}{1}$, which means that each unit in the picture is
72 times the value of \cs{mfpicunit}, that is, about one inch. The first
pair of mandatory arguments, \verb${-1}{1}$, indicate the
$x$-coordinates run from $-1$ to $1$. Since these differ by 2, they
indicate a width of two inches. The second pair similarly represents a
height of two inches. But the ellipse is centered at $(0,0)$, which is
one inch above the bottom (bottom is at $y = -1$), and its vertical
radius is .5. So the lowest point on the ellipse should be 0.5 inches
above the bottom of the space reserved. \Mfp{} provides a way to fit the
space reserved to the actual extent of `ink' in the picture. That is by
the option \opt{truebbox}:
\begin{verbatim}
\usepackage[metapost,truebbox]{mfpic}
\end{verbatim}
This would then produce something like figure~\ref{fig2}.
From now on, this option will be in effect in our examples.
\usetruebbox
\begin{mfpfig}{fig2}
  My first figure:
  \begin{mfpic}[72]{-1}{1}{-1}{1}
    \ellipse{(0,0),1,.5}
  \end{mfpic}
\end{mfpfig}

Even though the arguments to the \env{mfpic} environment are ignored in
determining the size of the figure (under \opt{truebbox}), they are
still needed in order to establish the coordinate system that the
ordered pairs refer to (for example \texttt{(0,0)} in the \cs{ellipse}
arguments).

\chapter{Positioning text}\label{sec:text}

By now you are probably thinking: ``This so-called `Hello, world'
of \mfp{} doesn't say `Hello, world' anywhere!'' We correct that
with the following example:
\begin{verbatim}
\begin{mfpic}[72]{-1}{1}{-1}{1}
  \ellipse{(0,0),1,.5}
  \tlabel[cc](0,0){Hello, world.}
\end{mfpic}
\end{verbatim}
This should give you figure~\ref{fig3}.
\begin{mfpfig}{fig3}
  \begin{mfpic}[72]{-1}{1}{-1}{1}
    \ellipse{(0,0),1,.5}
    \tlabel[cc](0,0){Hello, world.}
  \end{mfpic}
\end{mfpfig}

The \cs{tlabel} command places the given text at the given position
($(0,0)$) adjusted according to the optional argument \texttt{[cc]},
which says to center the text (both vertically and horizontally) at that
location. The \texttt{[cc]} is optional. Without it, the text would have
the leftmost point of its baseline (the imaginary line that most letters
sit on) placed at $(0,0)$.

You are no doubt thinking: ``The ellipse doesn't really match the
text. What you need is some macro that measures the text and produces an
oval with similar dimensions.'' For that we have the \cs{tlabeloval}
command. The \cs{tlabeljustify} command in the example below is to
communicate to both the text placement and the curve generation
procedures that they are to be centered at the point $(0,0)$. (We'll see
an easier way to do this later.)
\begin{verbatim}
\begin{mfpic}[72]{-1}{1}{-1}{1}
  \tlabeljustify{cc}
  \tlabeloval(0,0){Hello, world.}
\end{mfpic}
\end{verbatim}
This produces figure~\ref{fig4}.
\begin{mfpfig}{fig4}
  \begin{mfpic}[72]{-1}{1}{-1}{1}
    \tlabeljustify{cc}
    \tlabeloval(0,0){Hello, world.}
  \end{mfpic}
\end{mfpfig}

This would be better still if a little space is left around the text so
the ellipse doesn't touch it. The \cs{tlpathsep} command can do that:
\begin{verbatim}
\begin{mfpic}[72]{-1}{1}{-1}{1}
  \tlabeljustify{cc}
  \tlpathsep{3pt}
  \tlabeloval(0,0){Hello, world.}
\end{mfpic}
\end{verbatim}
producing figure~\ref{fig5}.
\begin{mfpfig}{fig5}
\begin{mfpic}[72]{-1}{1}{-1}{1}
  \tlabeljustify{cc}
  \tlpathsep{3pt}
  \tlabeloval(0,0){Hello, world.}
\end{mfpic}
\end{mfpfig}

It would be nice to make the text pop out a bit with some color%
    \footnote{Colors are included in this document only to give
    examples of their use in \prog{mfpic}. I do not necessarily
    recommend any of them.}%
. You can do that by adding \verb$\gfill[yellow]$ in front of either
\verb$\ellipse$ or \verb$\tlabeloval$:
\begin{verbatim}
\begin{mfpic}[72]{-1}{1}{-1}{1}
  \tlpathsep{3pt}
  \tlabeljustify{cc}
  \gfill[yellow]\tlabeloval(0,0){Hello, world.}
\end{mfpic}
\end{verbatim}
This will produce figure~\ref{fig6}.
\begin{mfpfig}{fig6}
  \begin{mfpic}[72]{-1}{1}{-1}{1}
    \tlpathsep{3pt}
    \tlabeljustify{cc}
    \gfill[yellow]\tlabeloval(0,0){Hello, world.}
  \end{mfpic}
\end{mfpfig}

Notice that now the boundary of the oval has not been drawn. This is the
standard behavior of \mfp{}. A figure command alone will draw the
figure. If you want some other rendering than that, you must explicitly
provide all of it. To get the boundary back, simply add \verb$\draw$
before the \verb$\gfill$. You can draw the curve in a color other than
black with an optional argument. We can also make the line thicker with
the command \verb$\penwd$:
\begin{verbatim}
\begin{mfpic}[72]{-1}{1}{-1}{1}
  \penwd{1.5pt}
  \tlpathsep{3pt}
  \tlabeljustify{cc}
  \draw[blue]\gfill[yellow]\tlabeloval(0,0){Hello, world.}
\end{mfpic}
\end{verbatim}
This will produce figure~\ref{fig7}.
\begin{mfpfig}{fig7}
  \begin{mfpic}[72]{-1}{1}{-1}{1}
    \penwd{1.5pt}
    \tlpathsep{3pt}
    \tlabeljustify{cc}
    \draw[blue]\gfill[yellow]\tlabeloval(0,0){Hello, world.}
  \end{mfpic}
\end{mfpfig}

This last version doesn't look too bad, but it seems that the oval ought
to be a little fatter (slightly higher than it is now). By default,
\verb$\tlabeloval$ will make the ratio of width to height the same as
that of the text, or rather of the text plus the additional space
specified by \verb$\tlpathsep$. This can be changed with an optional
argument, a number that multiplies the width-to-height ratio. Decreasing
this ratio will decrease the width (slightly) and increase the height.
Here we have also omitted the \cs{tlabeljustify} command and shown that
\cs{tlabeloval} takes a second optional argument that can be used to
`justify' both the curve and the text. To use this, one must explicitly
include the first optional argument; if the default is intended, an
empty pair of brackets may be used.
\begin{verbatim}
\begin{mfpic}[72]{-1}{1}{-1}{1}
  \penwd{1.5pt}
  \tlpathsep{3pt}
  \draw[blue]\gfill[yellow]\tlabeloval[.8][cc](0,0){Hello, world.}
\end{mfpic}
\end{verbatim}
This will produce figure~\ref{fig8}.
\begin{mfpfig}{fig8}
  \begin{mfpic}[72]{-1}{1}{-1}{1}
    \penwd{1.5pt}
    \tlpathsep{3pt}
    \draw[blue]\gfill[yellow]\tlabeloval[.8][cc](0,0){Hello, world.}
  \end{mfpic}
\end{mfpfig}

The \cs{tlabeloval} command places the label last, after the action of
all the preceding macros; therefore the text ends up on top of
everything else. The \cs{tlabeloval} command also has a \mbox{`*-form'} that
does everything \emph{except} place the text. Finally, ovals are not the
only thing that can be used to surround text. See the manual
(\file{mfpic-doc.pdf}) and below for others.

Here is a more common use of \verb$tlabel$ commands: labeling a graph
and axes. In the following example we have given \cs{tlabel} the option
\texttt{[bl]} to place the bottom left corner of the text at the given
coordinates. However, we have used \verb$\tlpointsep{3pt}$, which has
the effect of shifting text away from its nominal location, to prevent the
text from colliding with the curve.%
    \footnote{One can also use \cs{tlabelsep}, which is equivalent to
    \cs{tlpathsep} plus \cs{tlpointsep}.}
The value set by \cs{tlpointsep} has an effect only if the point is on
the edge of the text, so there would be no shifting with the
\texttt{[cc]} placement used earlier.
\begin{verbatim}
\begin{mfpic}[72]{0}{2.5}{0}{1}
  \tlpointsep{3pt}
  \polyline{(0,.2),(.5,1),(1,.7),(1.5,0),(2,.3)}
  \tlabel[bl](.5,1){Max output}
  \dashed\polyline{(0,.2),(.5,.6),(1,.3),(1.5,.7),(2,.1)}
  \tlabel[bl](1.5,.7){Max input}
\end{mfpic}
\end{verbatim}
This will produce figure~\ref{fig9}.
\begin{mfpfig}{fig9}
  \begin{mfpic}[72]{0}{2.5}{0}{1}
    \tlpointsep{3pt}
    \polyline{(0,.2),(.5,1),(1,.7),(1.5,0),(2,.3)}
    \tlabel[bl](.5,1){Max output}
    \dashed\polyline{(0,.2),(.5,.6),(1,.3),(1.5,.7),(2,.1)}
    \tlabel[bl](1.5,.7){Max input}
  \end{mfpic}
\end{mfpfig}

Notice that \verb$\polyline$ alone produces a solid line while
\verb$\dashed\polyline$ makes a dashed line. Let us close this section
by dressing up this figure with axes, some fat dots marking the
keypoints, and hash marks on the axes:
\begin{verbatim}
\begin{mfpic}[72]{0}{2.5}{0}{1}
  \tlpointsep{3pt}
  \polyline{(0,.2),(.5,1),(1,.7),(1.5,0),(2,.3)}
  \point[3pt]{(0,.2),(.5,1),(1,.7),(1.5,0),(2,.3)}
  \tlabel[bl](.5,1){Max output}
  \dashed\polyline{(0,.2),(.5,.6),(1,.3),(1.5,.7),(2,.1)}
  \pointfillfalse
  \point[3pt]{(0,.2),(.5,.6),(1,.3),(1.5,.7),(2,.1)}
  \tlabel[bl](1.5,.7){Max input}
  \axes
  \xmarks{0,0.5,1,1.5,2}
  \axislabels x{{$50$} .5, {$100$} 1, {$150$} 1.5, {$200$} 2}
\end{mfpic}
\end{verbatim}
This will produce figure~\ref{fig10}.
\begin{mfpfig}{fig10}
  \begin{mfpic}[72]{0}{2.5}{0}{1}
    \tlpointsep{3pt}
    \polyline{(0,.2),(.5,1),(1,.7),(1.5,0),(2,.3)}
    \point[3pt]{(0,.2),(.5,1),(1,.7),(1.5,0),(2,.3)}
    \tlabel[bl](.5,1){Max output}
    \dashed\polyline{(0,.2),(.5,.6),(1,.3),(1.5,.7),(2,.1)}
    \pointfillfalse
    \point[3pt]{(0,.2),(.5,.6),(1,.3),(1.5,.7),(2,.1)}
    \tlabel[bl](1.5,.7){Max input}
    \axes
    \xmarks{0,0.5,1,1.5,2}
    \axislabels x{{$50$} .5, {$100$} 1, {$150$} 1.5, {$200$} 2}
  \end{mfpic}
\end{mfpfig}

The optional argument of \verb$\point$ specifies the diameter of the
points to draw. The command \verb$\pointfillfalse$ forces the points to
be drawn as open circles. The axes configure themselves to the size
specified in the argument of the \env{mfpic} environment. The
\verb$axislabels$ command takes as arguments a letter, to specify the
axis, and a comma separated list of labels, each of which is specified
by some text to place (in braces) and the x-coordinate to place it at.

\chapter{Drawing figures}

\Mfp{} has several predefined figures and commands to obtain essentially
any curve (provided one can obtain enough points on it with sufficient
precision). We've already seen \verb$\polyline$ and \verb$\ellipse$. The
former needs a list of points to connect with line segments and the
latter needs the center and radii of the ellipse. The \verb$\ellipse$
also takes an optional argument: the number of degrees to rotate the
ellipse. Here we list some of the more common such figures. Remember
that all of them will produce some sort of line drawing if used alone.
They can be preceded by \verb$\dashed$ to make the lines dashed or
\verb$\dotted$ to make them dotted. If the figure is a closed curve,
\verb$\gfill$ will fill it in.
\begin{verbatim}
\begin{mfpic}[72]{0}{4}{0}{1}
  \rect{(0,0),(1,.75)}
  \circle{(1.5,.5),.45}
  \arc[s]{(3,0),(2,1),45}
  \ellipse[20]{(3.5, 0.5), 0.6, 0.4}
\end{mfpic}
\end{verbatim}
This produces figure~\ref{fig11}. The \verb$\arc$ command has several
forms. The optional argument picks the form to use. This one specifies
the endpoints of the circular arc and the angle of the arc (the angle
between the radii from the center of the circle to those two points).
Other possibilities are a three-point form (option \texttt{[t]}), a
polar form (option \texttt{[p]}), and a center-point-sweep form (option
\texttt{[c]}, specify a center, starting point, and angle). See the
manual for details. The default (what would be assumed if no optional
argument is given) is \oarg{s} and is called the ``point-sweep'' form.
\begin{mfpfig}{fig11}
  \begin{mfpic}[72]{0}{4}{0}{1}
    \rect{(0,0),(1,.75)}
    \circle{(1.5,.5),.45}
    \arc[s]{(3,0),(2,1),45}
    \ellipse[20]{(3.5, 0.5), 0.6, 0.4}
  \end{mfpic}
\end{mfpfig}

The \cs{polyline} command draws straight lines connecting points. We can
also draw smooth curves. Lets take the same points from our
\cs{polyline} example (figure~\ref{fig9}), but change \cs{polyline} to
\cs{curve}, omit the text, and add the points from figure~\ref{fig10}:
\begin{verbatim}
\begin{mfpic}[72]{0}{2.5}{0}{1}
  \curve{(0,.2),(.5,1),(1,.7),(1.5,0),(2,.3)}
    \point[3pt]{(0,.2),(.5,1),(1,.7),(1.5,0),(2,.3)}
  \dashed\curve{(0,.2),(.5,.6),(1,.3),(1.5,.7),(2,.1)}
    \pointfillfalse
    \point[3pt]{(0,.2),(.5,.6),(1,.3),(1.5,.7),(2,.1)}
\end{mfpic}
\end{verbatim}
This should produce figure~\ref{fig12}.
\begin{mfpfig}{fig12}
  \begin{mfpic}[72]{0}{2.5}{0}{1}
    \curve{(0,.2),(.5,1),(1,.7),(1.5,0),(2,.3)}
      \point[3pt]{(0,.2),(.5,1),(1,.7),(1.5,0),(2,.3)}
    \dashed\curve{(0,.2),(.5,.6),(1,.3),(1.5,.7),(2,.1)}
      \pointfillfalse
      \point[3pt]{(0,.2),(.5,.6),(1,.3),(1.5,.7),(2,.1)}
  \end{mfpic}
\end{mfpfig}

This is somewhat unsatisfying. One could improve the result by selecting
more points, or by increasing the `tension' in the curve.

Roughly speaking, tension determines how straight the segments between
the points are, and how sharp the turns at each point. High tension
makes the curve look a little more like a polyline. The default tension
is 1, a tension of about 5 makes the result look somewhat like a
polyline with very slightly rounded corners, very high tensions make the
curve indistinguishable from a polyline. Tension must (almost) always be
greater than $0.75$.

Another effect of increased tension is to reduce the little wobbles we
can see in the first curve. Let's try a tension of 1.5, which can be
specified as an optional argument to \cs{curve}:
\begin{verbatim}
\begin{mfpic}[72]{0}{2.5}{0}{1}
  \curve[1.5]{(0,.2),(.5,1),(1,.7),(1.5,0),(2,.3)}
    \point[3pt]{(0,.2),(.5,1),(1,.7),(1.5,0),(2,.3)}
  \dashed\curve[1.5]{(0,.2),(.5,.6),(1,.3),(1.5,.7),(2,.1)}
    \pointfillfalse
    \point[3pt]{(0,.2),(.5,.6),(1,.3),(1.5,.7),(2,.1)}
\end{mfpic}
\end{verbatim}
This give figure~\ref{fig13}.
\begin{mfpfig}{fig13}
  \begin{mfpic}[72]{0}{2.5}{0}{1}
    \curve[1.5]{(0,.2),(.5,1),(1,.7),(1.5,0),(2,.3)}
      \point[3pt]{(0,.2),(.5,1),(1,.7),(1.5,0),(2,.3)}
    \dashed\curve[1.5]{(0,.2),(.5,.6),(1,.3),(1.5,.7),(2,.1)}
      \pointfillfalse
      \point[3pt]{(0,.2),(.5,.6),(1,.3),(1.5,.7),(2,.1)}
  \end{mfpic}
\end{mfpfig}

When we use \cs{curve}, there is no way \MP{} can tell if we are just
connecting points or if we are trying to graph a function.
It \emph{cannot} enforce the requirement, which every function must
satisfy, that the curve should travel left-to-right. The command
\cs{fcncurve} does enforce this (assuming the points to be connected are
listed in left-to-right order). This command also permits an optional
tension argument. The dotted line in figure~\ref{fig14} is produced with
\cs{curve}, the solid one with \cs{fcncurve}. One might conceivably want
to decrease the tension a bit here.
\begin{verbatim}
\begin{mfpic}[72]{0}{2.5}{0}{1}
  \dotted\curve{(0,.2),(.5,0),(.85,.5),(1,1),(1.5,0),(2,.3)}
  \fcncurve{(0,.2),(.5,0),(.85,.5),(1,1),(1.5,0),(2,.3)}
    \pointfillfalse
    \point[3pt]{(0,.2),(.5,0),(.85,.5),(1,1),(1.5,0),(2,.3)}
\end{mfpic}
\end{verbatim}
\begin{mfpfig}{fig14}
  \begin{mfpic}[72]{0}{2.5}{0}{1}
    \dotted\curve{(0,.2),(.5,0),(.85,.5),(1,1),(1.5,0),(2,.3)}
    \fcncurve{(0,.2),(.5,0),(.85,.5),(1,1),(1.5,0),(2,.3)}
      \pointfillfalse
      \point[3pt]{(0,.2),(.5,0),(.85,.5),(1,1),(1.5,0),(2,.3)}
  \end{mfpic}
\end{mfpfig}

Other figures available include\leftmargini=2\leftmargini
\begin{description}
\item[\cs{cyclic}] Used just like \cs{curve} but closes the path
    (connects the last point smoothly to the starting point).
\item[\cs{polygon}] Used just like \cs{polyline} except it connects the
    last point to the first with a straight line.
\item[\cs{sector}] Makes a wedge with two straight lines and an arc. The
    arguments are almost the same as \cs{arc}\oarg{p}, but the order is
    different: center, radius and two angles.
\end{description}

Here are some other curves that, like \cs{tlabeloval}, are proportioned
to fit given text. All have a \texttt{*}-form that draws the path
without placing the text.
\begin{description}
\item[\cs{tlabelrect}]  This produces a rectangle. It has the same usage
    as \cs{tlabeloval}, except the first optional argument specifies
    the radius of quarter-circles used to make rounded corners.
\item[\cs{tlabelellipse}] This is similar to \cs{tlabeloval}
    except that instead of modifying the width-to-height ratio, the
    first optional argument \emph{is} the width-to-height ratio. If that
    argument is 1 (the default) you get a circle.
\item[\cs{tlabelcircle}] This produces a circle, of course.
\end{description}


\chapter{Functions}

\CMP{} is able to calculate a number of functions natively, and still
more have been defined in \mfp{}. Also available are the usual
arithmetic operations. Any valid \MP{} expression, containing
one unknown \gbc{x} and producing a numerical result can be graphed.

Here is an example of the graphs of $y = x^2$ and $y = \pm\sqrt{x}$.
Note that exponentials are denoted by \texttt{**} and it is important to
note that it has the same precedence as multiplication (denoted by a
single \texttt{*}). That is, in a formula like \mfc{3*3**2}, the
operations are performed in order, left to right, producing $(3\cdot
3)^2 = 81$ and not $3\cdot 3^2 = 27$. Parentheses are needed if the
latter is intended: \mfc{3*(3**2)}.
\begin{verbatim}
\setlength{\mfpicunit}{1cm}
\begin{mfpic}{-2.5}{2.5}{-1.5}{4}
  \function{-2,2,.1}{x**2}
  \function{0,2,.1}{sqrt x}
  \function{0,2,.1}{-sqrt x}
  \axes
  \xmarks{-2,-1,1,2}
  \ymarks{-1,1,2,3}
  \tlpointsep{3pt}
  \axislabels x{{$-2$}-2,{$-1$}-1,{$1$}1,{$2$}2}
  \axislabels y{{$-1$}-1,{$1$}1,{$2$}2,{$3$}3}
\end{mfpic}
\end{verbatim}
This produces figure~\ref{fig15}.
\begin{mfpfig}{fig15}
  \setlength{\mfpicunit}{1cm}
  \begin{mfpic}{-2.5}{2.5}{-1.5}{4}
    \function{-2,2,.1}{x**2}
    \function{0,2,.1}{sqrt x}
    \function{0,2,.1}{-sqrt x}
    \axes
    \xmarks{-2,-1,1,2}
    \ymarks{-1,1,2,3}
    \tlpointsep{3pt}
    \axislabels x{{$-2$}-2,{$-1$}-1,{$1$}1,{$2$}2}
    \axislabels y{{$-1$}-1,{$1$}1,{$2$}2,{$3$}3}
  \end{mfpic}
\end{mfpfig}

The command \cs{function} has two arguments. The first contains the
starting and ending x-values of the desired graph, followed by a
\emph{step size}. Generally the smaller the steps the better the
accuracy, but \MP{} has a limit on the number of steps (usually about
2000). There is also an optional argument which can be \oarg{s}, the
default, which means the graph is to be smooth, or \oarg{p}, which means
the graph is constructed by connecting the calculated points with
straight lines. Here is the same example with larger step size to
emphasize the difference (see figure~\ref{fig16})
\begin{verbatim}
\setlength{\mfpicunit}{1cm}
\begin{mfpic}{-2.5}{2.5}{-1.5}{4}
  \function[p]{-2,2,.5}{x**2}
  \function[p]{0,2,.5}{sqrt x}
  \function[p]{0,2,.5}{-sqrt x}
  \axes
  \xmarks{-2,-1,1,2}
  \ymarks{-1,1,2,3}
  \tlpointsep{3pt}
  \axislabels x{{$-2$}-2,{$-1$}-1,{$1$}1,{$2$}2}
  \axislabels y{{$-1$}-1,{$1$}1,{$2$}2,{$3$}3}
\end{mfpic}
\end{verbatim}
\begin{mfpfig}{fig16}
  \setlength{\mfpicunit}{1cm}
  \begin{mfpic}{-2.5}{2.5}{-1.5}{4}
    \function[p]{-2,2,.5}{x**2}
    \function[p]{0,2,.5}{sqrt x}
    \function[p]{0,2,.5}{-sqrt x}
    \axes
    \xmarks{-2,-1,1,2}
    \ymarks{-1,1,2,3}
    \tlpointsep{3pt}
    \axislabels x{{$-2$}-2,{$-1$}-1,{$1$}1,{$2$}2}
    \axislabels y{{$-1$}-1,{$1$}1,{$2$}2,{$3$}3}
  \end{mfpic}
\end{mfpfig}

In addition, one can increase the tension in the curve drawn by putting
a tension value after the \gbc{s} in \oarg{s}. For a tension of 2.4:
\cs{function}\oarg{s2.4}\marg{\dots}.

The functions available include \gbc{sqrt} and all the trig functions:
\gbc{sin x} assumes \gbc{x} is an angle in radians, \mfc{sind x} assumes
it is in degrees, with a similar naming convention for the remaining
trig functions. The inverses are \gbc{asin x}, \gbc{acos x}, and
\gbc{atan x}, which produce angles in degrees, and \gbc{invsin x}, etc.,
which produce angles in radians. There is also \gbc{ln~x} or \gbc{log~x}
for the natural logarithm, \gbc{exp~x} for $e^x$, \gbc{logten}~\gbc{x}
for the base 10 logarithm, \gbc{logtwo}~\gbc{x} for base 2, and
\gbc{logbase} for other bases:
\gbc{logbase(16)}~\gbc{x} (for example) for base 16. The general syntax of
these functions is the following: if the argument is \gbc{x} alone or a
pure number alone or the particular case of a number followed by \gbc{x}
(no \gbc{*} in between!) then parentheses are not needed. Example:
\gbc{sin~2x}. For almost anything else, parentheses are required:
\gbc{sin(3*x)} or \gbc{sin(x**2)}.

Some other functions available are the hyperbolic functions, \gbc{sinh x},
\gbc{cosh x}, etc. (all 6 of them), and the inverses of three of them:
\gbc{asinh x}, \gbc{acosh x}, and \gbc{atanh x}.

These functions (or any \MP{} numeric expression) can also be used in
any of the coordinates of points in drawing commands like \cs{polyline}
(but not usually in text placement commands like \cs{tlabeloval}). For
example (from now on the value of \cs{mfpicunit} is set to
\texttt{1cm}):\setlength{\mfpicunit}{1cm}
\begin{verbatim}
\begin{mfpic}{-.5}{2.5}{-1.5}{1.5}
  \polyline{(2,-sqrt 2),(1,-1),(.5,- sqrt .5),(0,0),
      (.5,sqrt .5),(1,1),(2,sqrt 2)}
  \axes
  \xmarks{1,2}
  \ymarks{-1,1}
  \tlpointsep{3pt}
  \axislabels x{{$1$}1,{$2$}2}
  \axislabels y{{$-1$}-1,{$1$}1}
\end{mfpic}
\end{verbatim}
\begin{mfpfig}{fig17}
  \begin{mfpic}{-.5}{2.5}{-1.5}{1.5}
    \polyline{(2,-sqrt 2),(1,-1),(.5,- sqrt .5),(0,0),
        (.5,sqrt .5),(1,1),(2,sqrt 2)}
    \axes
    \xmarks{1,2}
    \ymarks{-1,1}
    \tlpointsep{3pt}
    \axislabels x{{$1$}1,{$2$}2}
    \axislabels y{{$-1$}-1,{$1$}1}
  \end{mfpic}
\end{mfpfig}

There are other types of functions: parametric functions, and polar
coordinate versions. \Mfp{} provides \cs{parafcn} and \cs{plrfcn} to
graph these. The \cs{parafcn} requires a starting value, and ending
value and a step size just as in \cs{function}, but in the second
argument there must be either a pair of expressions in the variable
\gbc{t}, separated by a comma and enclosed in parentheses, or a single
\emph{pair-valued} expression. \CMP{} and \mfp{} provide only a few
pair-valued functions; one is used below.

The second argument of \cs{plrfcn} must contain a single numeric
expression in the variable \gbc{t}, and indicates a function of $\theta$
to be graphed in polar coordinates: $r = f(\theta)$. In the following
example (figure~\ref{fig22}), we draw a portion of the graph of $x =
y^2$ by representing it as the graph of the parametric equations $x =
t^2$, $y = t$, and a portion of a circle of radius $1.5$ by representing
it as the graph of the pair-valued function \mfc{dir(t)}. The expression
\mfc{dir(t)} gives the point whose distance from $(0,0)$ is $1$ in the
direction given by the angle \mfc{t}.
\begin{verbatim}
\begin{mfpic}{-2}{4}{-2}{2}
  \parafcn{-2,2,.1}{(t**2,t)}
  \dotted\parafcn{45,315,5}{1.5*dir(t)}
\end{mfpic}
\end{verbatim}
\begin{mfpfig}{fig22}
  \begin{mfpic}{-2}{4}{-2}{2}
    \parafcn{-2,2,.1}{(t**2,t)}
    \dotted\parafcn{45,315,5}{1.5*dir(t)}
  \end{mfpic}
\end{mfpfig}


Here is an example of a graph of the polar coordinate function
$r=2\sin 3\theta$ (figure~\ref{fig23}). We use the degree version
\mfc{sind} in order to work with integers.
\begin{verbatim}
\begin{mfpic}{-2}{2}{-2}{2}
  \plrfcn{0,180,5}{2*sind 3t}
\end{mfpic}
\end{verbatim}
\begin{mfpfig}{fig23}
  \begin{mfpic}{-2}{2}{-2}{2}
    \plrfcn{0,180,5}{2*sind 3t}
  \end{mfpic}
\end{mfpfig}


\chapter{Transforming figures}

\CMP{} is capable of any affine transformation (things like shifting,
rotating, scaling, reflecting and slanting) of any path. The figures
we've been dealing with so far (\cs{ellipse}, \cs{curve}, \cs{function},
etc.) all produce, in the \MP{} code, the definition of some path (as
well as a drawing of that path). \Mfp{} provides for different methods
of `drawing' the path with \emph{prefix macros}. We've seen \cs{dashed},
\cs{dotted}, \cs{gfill} so far, in addition to the default \cs{draw}.
\Mfp{} also provides for modifying the shape and position of the path
with other prefixes. Here's a simple example.
\begin{verbatim}
\begin{mfpic}{-.5}{2.5}{-.5}{2.5}
  \rotatepath{(1,.5), 45}\rect{(0,0),(2,1)}
  \point{(1,.5)}
\end{mfpic}
\end{verbatim}
The command \cs{rotatepath} obviously rotates the path that follows, but
it needs to know what the center of rotation will be, and how much to
rotate. These are given in its mandatory argument, separated by a comma.
The example above (pictured in figure~\ref{fig18}) rotates 45 degrees
around the center of the rectangle.
\begin{mfpfig}{fig18}
  \begin{mfpic}{-.5}{2.5}{-.5}{2.5}
    \rotatepath{(1,.5), 45}\rect{(0,0),(2,1)}
    \point{(1,.5)}
  \end{mfpic}
\end{mfpfig}

Notice that we have no drawing prefix. A combination of
transformation-plus-figure is treated as a figure in its own right and
behaves the same. If we want the figure dashed, we could write
\begin{verbatim}
\dashed\rotatepath{(1,.5),45}\rect{(0,0),(2,1)}
\end{verbatim}
It may not be obvious, but we can also write a drawing macro between
the rotation and the figure, producing figure~\ref{fig19}
\begin{verbatim}
\begin{mfpic}{-.5}{2.5}{-.5}{2.5}
  \rotatepath{(1,.5), 45}\draw\rect{(0,0),(2,1)}
  \point{(1,.5)}
\end{mfpic}
\end{verbatim}
\begin{mfpfig}{fig19}
  \begin{mfpic}{-.5}{2.5}{-.5}{2.5}
    \rotatepath{(1,.5), 45}\draw\rect{(0,0),(2,1)}
    \point{(1,.5)}
  \end{mfpic}
\end{mfpfig}

This illustrates another property of \mfp{} macros: the combination of a
rendering prefix and a figure is also treated the same as a figure in
its own right: the same figure as the one that follows. In fact, the
only difference between \cs{rect} and \cs{draw}\cs{rect} in this
example is that the second one has a minor(!) side effect: the
rectangle is drawn.

Finally, try to guess what happens if we add another prefix at the
front:
\begin{verbatim}
\begin{mfpic}{-.5}{2.5}{-.5}{2.5}
  \dotted\rotatepath{(1,.5), 45}
    \draw\rect{(0,0),(2,1)}
\end{mfpic}
\end{verbatim}
and if we add another rotation in front of that.
\begin{verbatim}
\begin{mfpic}{-.5}{2.5}{-.5}{2.5}
  \rotatepath{(0,0),45}
    \dotted\rotatepath{(1,.5), 45}
      \draw\rect{(0,0),(2,1)}
\end{mfpic}
\end{verbatim}


Available transformations include
\begin{display}\raggedright
  \cs{scalepath}, \cs{shiftpath}, \cs{xscalepath}, \cs{yscalepath},
  \cs{slantpath}, and \cs{reflectpath}.
\end{display}
See the manual for a description of what arguments are required for
each. Here's a final example, producing figure~\ref{fig20}
\begin{verbatim}
\begin{mfpic}{-.5}{2.5}{-.5}{2.5}
  \shiftpath{(-1,1)}\draw[red]\slantpath{.5,1}\dotted
  \rotatepath{(0,0), 90}\dashed\rect{(0,0),(2,1)}
  \point{(0,0),(2,1)}
  \tlpointsep{2pt}
  \tlabel[tr](0,0){$(0,0)$}
  \tlabel[bl](2,1){$(2,1)$}
\end{mfpic}
\end{verbatim}
\begin{mfpfig}{fig20}
  \begin{mfpic}{-.5}{2.5}{-.5}{2.5}
    \shiftpath{(-1,1)}\draw[red]\slantpath{.5,1}\dotted
    \rotatepath{(0,0), 90}\dashed\rect{(0,0),(2,1)}
    \point{(0,0),(2,1)}
    \tlpointsep{2pt}
    \tlabel[tr](0,0){$(0,0)$}
    \tlabel[bl](2,1){$(2,1)$}
  \end{mfpic}
\end{mfpfig}

\chapter{Rendering figures}

\emph{Rendering} is the act of making a description of a figure visible.
Examples are: drawing a solid curve, drawing a dashed curve, or filling
its interior, For \mfp{} figure macros the default, in the absence of
explicit commands, is to use \cs{draw}. That is,
\begin{verbatim}
\rect{(0,0),(1,2)}
\end{verbatim}
has the same result as
\begin{verbatim}
\draw\rect{(0,0),(1,2)}
\end{verbatim}
The default rendering can be changed. Just say
\cs{setrender}\marg{\cs{dashed}}, and all figures afterward will be dashed
(see figure~\ref{fig21}).
\begin{verbatim}
\begin{mfpic}{0}{2}{0}{1}
\setrender{\dashed}
  \rect{(0,0),(1,1)}
  \circle{(1.5,.5),.5}
\end{mfpic}
\end{verbatim}
\begin{mfpfig}{fig21}
  \begin{mfpic}{0}{2}{0}{1}
  \setrender{\dashed}
    \rect{(0,0),(1,1)}
    \circle{(1.5,.5),.5}
  \end{mfpic}
\end{mfpfig}
The \cs{setrender} command can be inside an \env{mfpic} environment to
affect only later commands in that figure, or outside to affect all
later \mfp{} figures.

We give a few examples now of the renderings possible. These divide
more-or-less into those that trace a path and those that fill in a path.
In order to fill in a path, it must be a closed path, of course, but
\MP{} distinguishes between closed paths and those that merely happen to
end where they began. There is a good reason for this: \MP{} cannot,
without human aid, know if two points are the same, or merely
accidentally so close that the accuracy of the program sees them as the
same. It requires human aid in the form of an explicit request to
create a closed path. Of the \mfp{} macros we've seen so far,
\cs{ellipse}, \cs{circle}, \cs{rect}, \cs{polygon}, and \cs{cyclic}
produce closed paths, but \cs{polyline}, \cs{curve}, \cs{function},
\cs{parafcn}, and \cs{plrfcn} do not. Also producing closed paths are
\cs{tlabeloval} and its relatives.

The following example illustrates filling with a hatching pattern
(parallel lines) and an \emph{unfilling}. Clearing the interior of a
path may not seem like rendering, but it is treated in exactly the same way
(think of it as a negative rendering). We first hatch a rectangle, then
clear out a smaller rectangle with rounded corners to place our text
inside. The results are in figure~\ref{fig24}.
\begin{verbatim}
\begin{mfpic}{0}{2}{0}{2}
  \draw[red]\lhatch[2pt][blue]\rect{(0,0),(2,2)}
  \gclear\tlabelrect[6pt][cc](1,1){Hatching!}
\end{mfpic}
\end{verbatim}
\begin{mfpfig}{fig24}
  \begin{mfpic}{0}{2}{0}{2}
    \draw[red]\lhatch[2pt][blue]\rect{(0,0),(2,2)}
    \gclear\tlabelrect[6pt][cc](1,1){Hatching!}
  \end{mfpic}
\end{mfpfig}
This example illustrates that \cs{lhatch} fills with left slanting
lines. And that it takes two optional arguments. The first is the
distance between lines, and the second is the color to make the lines.
There are also \cs{rhatch} which slants the lines the other way,
\cs{xhatch} which uses both slants, and \cs{thatch} which can draw
the lines at any angle.

Here is another example of rendering (figure~\ref{fig26}). The new
macro is \cs{polkadot}. We've repeated this example twice to
show the effect of changing the order of the prefixes. Each prefix
applies its rendering to the result of everything to the right of it. In
the second example the hatching goes over the dots (and a bit of the
dashes as well). If the \cs{gfill} were first, it would cover almost
everything else.
\begin{verbatim}
\begin{mfpic}{0}{6}{0}{2}
  \penwd{2pt}
  \hatchwd{2pt}
  \drawcolor{blue}
  \hatchcolor{red}
  \fillcolor{green}
  \dashed\polkadot\rhatch[5pt]\gfill[yellow]\rect{(0,0),(2.8,1.8)}
  \rhatch[5pt]\dashed\polkadot\gfill[yellow]\rect{(3,0),(5.8,1.8)}
\end{mfpic}
\end{verbatim}
We've added a couple of other new features to this example. To emphasize
effects, we've increased the thickness of the drawing pen (\cs{penwd})
and the hatch lines (\cs{hatchwd}). We've also used the \cs{drawcolor}
macro and its relatives to set the colors to be used. The \cs{polkadot}
macro uses the color set by \cs{fillcolor}; so does \cs{gfill} if no
optional color is given.
\begin{mfpfig}{fig26}
  \begin{mfpic}{0}{6}{0}{2}
    \penwd{2pt}
    \hatchwd{2pt}
    \drawcolor{blue}
    \hatchcolor{red}
    \fillcolor{green}
    \dashed\polkadot\rhatch[5pt]\gfill[yellow]\rect{(0,0),(2.8,1.8)}
    \rhatch[5pt]\dashed\polkadot\gfill[yellow]\rect{(3,0),(5.8,1.8)}
  \end{mfpic}
\end{mfpfig}

If one wants to plot several curves in a single graph, they often need
to be rendered differently. The three methods we've seen so far,
\cs{draw}, \cs{dashed}, and \cs{dotted}, may not be enough. The
\cs{dashed} and \cs{dotted} commands permit an optional argument to
adjust the length of the dashes and spaces, and size of the dots. One
can also change the curve thickness with \cs{penwd}. But
that may not be `different' enough. \Mfp{} provides a few solutions.
When color is available, they may be drawn in different colors. When
not, there are two possibilities: \cs{gendashed} and \cs{plot}.

The first, \cs{gendashed}, is a generalized dashing macro. It takes one
mandatory argument, the name of a dashing pattern. Named dashing
patterns may be created with the \cs{dashpattern} command, as shown by
the following example (see figure~\ref{fig27}):
\begin{verbatim}
\begin{mfpic}{-3.5}{3.5}{-1.2}{1.2}
  \dashpattern{dotdash}{0pt,4pt,3pt,4pt}
  \gendashed{dotdash}\function{-pi,pi,.2}{sin 2x}
  \function{-pi,pi,.2}{cos 2x}
  \axes
\end{mfpic}
\end{verbatim}
\begin{mfpfig}{fig27}
  \begin{mfpic}{-3.5}{3.5}{-1.2}{1.2}
    \dashpattern{dotdash}{0pt,4pt,3pt,4pt}
    \gendashed{dotdash}\function{-pi,pi,.2}{sin 2x}
    \function{-pi,pi,.2}{cos 2x}
    \axes
  \end{mfpic}
\end{mfpfig}

The \cs{dashpattern} command takes a name and an even number of lengths.
The first, third, etc., lengths represent the lengths of dashes
(\texttt{0pt} means a dot), and the second, fourth, etc., represent
spaces. The given pattern is dot-space-dash-space. This pattern, when
used in a \cs{gendashed} command, is repeated for the length of the
curve.

This last example illustrates that the predefined \MP{} variable
\gbc{pi} (equal to 3.14159) can be used pretty much anywhere a number
can be used (except, often, in text label commands).

Another way to get more distinctive curves is to `dot' them with
something other than tiny dots. The \cs{plot} command does that. It
takes one mandatory argument, the name of a symbol to use instead of a
dot. Here are the same two curves \cs{plot}-ed (figure~\ref{fig28}):
\begin{verbatim}
\begin{mfpic}{-3.5}{3.5}{-1.2}{1.2}
  \setlength{\pointsize}{2.5pt}
  \plot{Triangle}\function{-pi,pi,.2}{sin 2x}
  \plot[2pt,6pt]{SolidCircle}\function{-pi,pi,.2}{cos 2x}
  \axes
\end{mfpic}
\end{verbatim}
\begin{mfpfig}{fig28}
  \begin{mfpic}{-3.5}{3.5}{-1.2}{1.2}
    \setlength{\pointsize}{2.5pt}
    \plot{Triangle}\function{-pi,pi,.2}{sin 2x}
    \plot[2pt,6pt]{SolidCircle}\function{-pi,pi,.2}{cos 2x}
    \axes
  \end{mfpic}
\end{mfpfig}

The \cs{plot} command takes an optional argument to specify the size of
the symbols and the spacing between them. The size of the symbols can
also be adjusted by changing the length command \cs{pointsize} (that also
adjusts the size of the dots placed with the \cs{point} command).

In this last example, \cs{plotnodes} is similar to \cs{plot}, except it
placed the symbols at the `nodes' defined by the path command. In the
case of \cs{function}, these are the points $(x_k, f(x_k))$ with $x_k$
stepping through all the $x$-values determined by the first argument of
\cs{function} (figure~\ref{fig29}).
\begin{verbatim}
\begin{mfpic}{-3.5}{3.5}{-1.2}{1.2}
  \plotnodes[2.5pt]{Square}\function{-pi,pi,pi/16}{sin 2x}
  \axes
\end{mfpic}
\end{verbatim}
\begin{mfpfig}{fig29}
  \begin{mfpic}{-3.5}{3.5}{-1.2}{1.2}
    \plotnodes[2.5pt]{Square}\function{-pi,pi,pi/16}{sin 2x}
    \axes
  \end{mfpic}
\end{mfpfig}

See the manual for the list of predefined symbols available to the
\cs{plot} and \cs{plotnodes} command.

\chapter{More on text}

The text positioning commands used so far in this guide are entirely
handled by \TeX{} or \LaTeX. This is why we have occasionally had to say
that certain things could be done ``except in text placement commands''.
It is possible for text positioning to be done within \MP, making many
things possible that couldn't be done otherwise. For example, text can
be rotated about the point of placement. You are probably thinking that
\LaTeX{} can rotate text, but it is not all that easy to arrange for the
point on the graph where we place the text to be the center of rotation.
Below are two examples, in which we attempt to place the text separated
from $(0,0)$ by \texttt{5pt} and rotated 45 degrees around $(0,0)$. In
the first we try to use \LaTeX's \cs{rotatebox} command, and in the
second we turn on \MP{} handling of labels and use a rotation option to
the \cs{tlabel} command.
\begin{verbatim}
\begin{mfpic}{0}{1}{0}{1}
  \point{(0,0)}
  \polyline{(0,0),(1,1)}
\tlabel[Bl](0,0){\rotatebox{45}{\hspace{5pt}Test text}}
\end{mfpic}
\end{verbatim}
\begin{verbatim}
\usemplabels
\begin{mfpic}{0}{1}{0}{1}
  \point{(0,0)}
  \polyline{(0,0),(1,1)}
  \tlpointsep{5pt}
  \tlabel[Bl45](0,0){Test text}
\end{mfpic}
\end{verbatim}
\begin{figure}[!hb]
  \centering
  \begin{minipage}{4cm}
    \centering
    \begin{mfpic}{0}{1}{0}{1}
      \point{(0,0)}
      \polyline{(0,0),(1,1)}
    \tlabel[Bl](0,0){\rotatebox{45}{\hspace{5pt}Test text}}
    \end{mfpic}\renewcommand\thefigure{\thesection.\arabic{figure}a}
    \refstepcounter{figure}\label{fig30a}
    \par\medskip (a)
  \end{minipage}\addtocounter{figure}{-1}\qquad
  \begin{minipage}{4cm}\usemplabels
    \centering
    \begin{mfpic}{0}{1}{0}{1}
      \point{(0,0)}
      \polyline{(0,0),(1,1)}
      \tlpointsep{5pt}
      \tlabel[Bl45](0,0){Test text}
    \end{mfpic}\renewcommand\thefigure{\thesection.\arabic{figure}b}
    \refstepcounter{figure}\label{fig30b}
    \par\medskip (b)
  \end{minipage}

  \medskip Figure \thesection.\arabic{figure}.
\end{figure}

The first produces figure~\ref{fig30a} and the second produces
figure~\ref{fig30b}. Our goal was to get the baseline of the text lined
up with the reference line drawn.

In the first example, \LaTeX's \cs{rotatebox} command produces the
following result, where we put a frame around both the unrotated text
and the rotated result to emphasize what \LaTeX{} sees as the
boundaries:
$$
  \hbox{\setlength\fboxsep{0pt}%
    \fbox{\rotatebox{45}{\fbox{\hspace{5pt}Test text}}}}
$$
This is then placed by the \cs{tlabel} command with the lower left
corner of the \emph{outer} box at $(0,0)$. But \LaTeX's axis of rotation
was at the lower left corner of the inner box. In the second case, \MP{}
placed the label. The command \cs{tlpointsep}\marg{5pt} and the parameter
\oarg{Bl45} explicitly request that the label be placed with its left
baseline 5 points from $(0,0)$ and rotated 45 degrees \emph{about the
point $(0,0)$}.

The \cs{usemplabels} command used above asks \MP{} to arrange for the
setting of labels. Adding the option \opt{mplabels} to the
\cs{usepackage} command that loads \mfp{} has the same effect for the
whole document.  There can be problems with using \MP{} to set labels.
One is that \MP{} has to call a \file{tex} program to do the actual
typesetting, and then one must either make arrangements that ensure
\MP{} will call \LaTeX, or never use any macros in the labels that are
not defined in plain\TeX. If one does arrange for \LaTeX{} to be used,
one needs to arrange that a \LaTeX{} preamble is prepended to the output
\file{.mp} file. The \cs{mfpverbtex} command can be used for this.

The command \cs{nomplabels} can be used to return to having labels set
at the document level. For the rest of this guide, we have
\opt{mplabels} in effect.

There are a few more commands that place text on the picture. All of
them pass the final responsability for text placement to \MP{} if
\opt{mplabels} is in effect. See the manual for more details.


\chapter{Arrows}

The command \cs{arrow} adds an arrowhead onto the \emph{end} of any path
that follows. For this to have predictable effects, you need to know
which part of a curve is the end, and which the start. Not surprisingly,
for the commands that connect a list of points in order the first point
in the list is the start point and the last point is the end. Except the
closed paths (\cs{cyclic}, \cs{polygon}, etc.); for them, the start and
the end points are the same, but the order of the points gives a
direction to the arrowhead. The default \cs{circle} has an anticlockwise
direction, but if the circle is defined by three points (for example)
the direction of the circle is determined by the order in which the
points are written.

Anyway, here are a few examples, illustrating the use of \cs{arrow}, and
some of its optional arguments.
\begin{verbatim}
\begin{mfpic}{0}{4}{0}{4}
  \arrow[r-5]\circle{(1,1),.5}
  \arrow[b4pt]\arrow\polyline{(3,2),(3,0)}
  \arrow[cred]\reverse\arrow\polyline{(0,3),(2,3)}
  \arrow[l 5pt]\rect{(4,2),(2,4)}
\end{mfpic}
\end{verbatim}
See figure~\ref{fig31} for the  results of this example. There are four
possible optional arguments, the first character inside the brackets
tells what option the rest of the argument applies to. The first example
above starts with `\texttt{r}', which stands for `rotate' and asks for
the arrowhead to be rotated $-5$ degrees (positive rotation means
anticlockwise, negative means clockwise). This is frequently useful for
arrows on curved paths, as the default direction (tangent to the path)
often just looks wrong). The second example starts with `\texttt{b}',
which stands for `backset' and it moves the head back \texttt{4pt} from
where it would otherwise be placed. In the example, this is used to put
a double arrowhead on the line. In the third example we put an arrow at
both ends by reversing the sense of the curve in between the two
\cs{arrow} prefixes. We also used the letter `\texttt{c}' in the
optional argument of one arrowhead. This stands for `color' and the
requested color is `\texttt{red}'. Finally, the \texttt{l} option
(that's a lowercase `ell', not the number `one') changes the length of
the arrowhead to 5 points (from the \verb$3pt$ default).%
    \footnote{I have put a space between the \texttt{l} and the
      \texttt{5pt} so it won't be mistaken for `$15$pt. Normally one
      should avoid spaces in \mfp{} optional arguments, but this is one
      case where it will cause no harm.}
\begin{mfpfig}{fig31}
  \begin{mfpic}{0}{4}{0}{4}
    \arrow[r-5]\circle{(1,1),.5}
    \arrow[b4pt]\arrow\polyline{(3,2),(3,0)}
    \arrow[cred]\reverse\arrow\polyline{(0,3),(2,3)}
    \arrow[l 5pt]\rect{(4,2),(2,4)}
  \end{mfpic}
\end{mfpfig}

The options can be combined in one command:
\cs{arrow}\oarg{cblue}\oarg{b4pt}\oarg{r25}\oarg{l6pt} would produce a
\texttt{6pt} long blue arrowhead rotated 25 degrees anticlockwise, set
back \texttt{4pt}. The setting back is done in the direction determined
\emph{after} rotation. The order of the options is not significant.

The shape of the arrowhead can be changed with the \cs{headshape}
command. The following example draws the arrowhead first normally, and
then after an instance of this command. We draw it a third time, exactly
like the second time, except we use the *-form. We have increased the
length of head and the thickness of the pen to emphasize the effects.
\begin{verbatim}
\begin{mfpic}{0}{4}{0}{4}
  \setlength{\headlen}{20pt}
  \penwd{3pt}
  \arrow\polyline{(0,3),(4,3)}
    \headshape{.5}{2}{true}
  \arrow\polyline{(0,2),(4,2)}
  \arrow*\polyline{(0,1),(4,1)}
\end{mfpic}
\end{verbatim}
The results are pictured in figure~\ref{fig32}. The first argument to
\cs{headshape} sets the ratio of width to height for the head. We have
cut it in half here. The second argument sets the tension in the curves
that form the sides of the head. This reduces the curvature in the
sides. The third argument can be only \texttt{true} or \texttt{false}
and determines whether the head is a solid shape, or only the two
`barbs'. The defaults correspond to \verb$\headshape{1}{1}{false}$. The
filled form does not draw the outline so what we see is the pointy
arrowhead on top of a thick line. The *-form tries to erase part of the
line so that one sees an actual pointy arrow.
\begin{mfpfig}{fig32}
  \begin{mfpic}{0}{4}{0}{4}
    \setlength{\headlen}{20pt}
    \penwd{3pt}
    \arrow\polyline{(0,3),(4,3)}
      \headshape{.5}{2}{true}
    \arrow\polyline{(0,2),(4,2)}
    \arrow*\polyline{(0,1),(4,1)}
  \end{mfpic}
\end{mfpfig}


\chapter{Color}

We saw the use of color in earlier sections, and now it's time to be
systematic about it. The several rendering commands have a color option;
examples are \cs{draw}, \cs{gfill}, \cs{arrow}, and the hatching
commands. However, even those commands that don't provide such an option
can have the color of their rendering changed. \Mfp{} provides the
following commands to change certain colors. Those commands with a
color option can be used without that option and then they will the use
the appropriate color described here. Each of these color-changing
commands takes a mandatory argument containing the color to change to,
and an optional argument to be described later.

\begin{description}
\item[\cs{backgroundcolor}] This sets the color to be used by \cs{gclear}.
        It is the same color used by \cs{point} for the inside of the
        points when \cs{pointfillfalse} has been used.  In \MP{}, the
        only way to clear the inside of a region is to cover it up. The
        default color for this purpose is \mfc{white}. Use this command
        to change that default.
\item[\cs{drawcolor}] This sets the default color used by those rendering
        commands that draw a path. This includes \cs{draw}, but also
        includes \cs{dashed}, \cs{dotted}, \cs{plot} and \cs{plotnodes}.
        It is also used by other commands that produce lines or curves:
        figure macros used without any rendering prefix, as well as
        \cs{axes} and related commands.
\item[\cs{fillcolor}] This sets the default color used by \cs{gfill}. It
        is also used by \cs{polkadot} (which has no color option).
\item[\cs{hatchcolor}] This sets the default color used by any hatching
        command.
\item[\cs{headcolor}] This sets the default color for arrowheads added
        by the \cs{arrow} command. It is also the color of
        arrowheads on any coordinate axis.
\item[\cs{pointcolor}] This sets the color used by \cs{point}, \cs{grid},
        and \cs{plotsymbol} (the last one will be described later).
\item[\cs{tlabelcolor}] This sets the color used for all text labels if
        the \opt{mplabels} option is turned on.
\end{description}

The color can be a common name for a color, provided that name is one of
the following: \texttt{white}, \texttt{black}, \texttt{red},
\texttt{green}, \texttt{blue}, \texttt{cyan}, \texttt{magenta}, or
\texttt{yellow}. We have already seen this usage. It can also be a color
name defined in the file \file{dvipsnam.mp} that accompanies \mfp. It can
also be an explicit color formula, where color formulas are described in
the \mfp{} manual.

The optional argument is one of the \emph{color models}. See the manual
for details, but the syntax is just like that of the \prog{color}
package's \cs{color} command. For example,
\begin{verbatim}
\pointcolor[rgb]{0,1,0}
\end{verbatim}
would use the color model \opt{rgb} with parameters 0, 1, and 0 (this is
green). After each of these commands a certain color name is assigned a
value. For example, a use of the \cs{pointcolor} command assigns a value
to the color named \gbc{pointcolor}. Also \cs{drawcolor} sets
\gbc{drawcolor} and this pattern is followed for all the color setting
commands above except \cs{backgroundcolor}, which assigns its value to
the color named \gbc{background}.

Color names for \mfp{} use can be defined using the \cs{mfpdefinecolor}
command. Here's an example (figure~\ref{fig33}). Note the use of the
color name \gbc{pointcolor} to make arrowheads and points have the same
color.
\begin{verbatim}
\begin{mfpic}{0}{3.5}{0}{3.5}
  \tlabelcolor{red}
  \pointcolor{rgb(0,1,0)}% green
  \drawcolor[rgb]{0,0,1} % blue
  \fillcolor{Goldenrod}  % from dvipsnam.mp
  \headcolor{pointcolor} % will be green after above
  \mfpdefinecolor{DarkerRed}{rgb}{.67,0,0}
  \hatchcolor{DarkerRed}
  \penwd{1pt}
  \gfill\circle{(1,1),.5}
  \point[3pt]{(1,.5),(1,1.5),(.5,1),(1.5,1)}
  \hatch\rect{(2.5,2.5),(3.5,3.5)}
  \arrow[l 5pt]\polyline{(1,1),(3,3)}
  \tlabel[cc](1,3){Examples\\of\\colors}
\end{mfpic}
\end{verbatim}
\begin{mfpfig}{fig33}
  \begin{mfpic}{0}{3.5}{0}{3.5}
    \tlabelcolor{red}
    \pointcolor{rgb(0,1,0)}% green
    \drawcolor[rgb]{0,0,1} % blue
    \fillcolor{Goldenrod}  % from dvipsnam.mp
    \headcolor{pointcolor} % will be green after above
    \mfpdefinecolor{DarkerRed}{rgb}{.67,0,0}
    \hatchcolor{DarkerRed}
    \penwd{1pt}
    \gfill\circle{(1,1),.5}
    \point[3pt]{(1,.5),(1,1.5),(.5,1),(1.5,1)}
    \hatch\rect{(2.5,2.5),(3.5,3.5)}
    \arrow[l 5pt]\polyline{(1,1),(3,3)}
    \tlabel[cc](1,3){Examples\\of\\colors}
  \end{mfpic}
\end{mfpfig}

\chapter{Closing paths}

There are many different ways to modify a figure. We have already seen
\cs{arrow}, which appends an arrowhead, \cs{reverse} which reverses the
sense, and several that apply an affine transformation
(\cs{rotatepath}, \cs{shiftpath}, etc.).  Now we will see the simple
operation of closing a path.

All methods of closing a path have to connect the end to the start, but
simply drawing a connection is not enough. \MP{} has to be told to close
the path, and what kind of connection is desired. We have several macros
that can do the job, the simplest being \cs{lclosed}, which closes with
a straight line. Putting \cs{lclosed} in front of \cs{polyline}, for
example, produces the same result as \cs{polygon}. Another macro is
\cs{sclosed} which produces a smooth closure. Putting it in front of
\cs{curve} gives (almost) the same result as \cs{cyclic}. There is one
other useful macro, \cs{bclosed}, which also informs \MP{} to make a
smooth closure. The difference between \cs{sclosed} and \cs{bclosed} is
that the first modifies slightly the original path (in order to achieve
the effect that \cs{sclosed}${}+{}$\cs{curve}${}\approx{}$\cs{cyclic}),
the second just asks \MP{} to do its best to connect the ends smoothly.
Here's an example comparing the two smooth methods (figure~\ref{fig34}).
\begin{verbatim}
\begin{mfpic}{0}{4}{0}{4}
 % an open curve:
  \curve{(0.49,3),(.5,3.7),(1,4),(1.5,3.7),(1.51,3)}
 % \sclosed a shifted copy:
  \draw\gfill[green]\sclosed\shiftpath{(2,0)}
      \curve{(0.49,3),(.5,3.7),(1,4),(1.5,3.7),(1.51,3)}
 % \bclosed another copy:
  \draw\gfill[yellow]\bclosed\shiftpath{(2,-2)}
      \curve{(0.49,3),(.5,3.7),(1,4),(1.5,3.7),(1.51,3)}
 % \cyclic with same points, shifted:
  \draw\gfill[red]\shiftpath{(0,-2)}
      \cyclic{(0.49,3),(.5,3.7),(1,4),(1.5,3.7),(1.51,3)}
  \tlabeljustify{bc}
  \nomplabels
  \tlabels{
    (1,2.4){\cs{curve}}
    (3,2.4){\cs{sclosed}}
    (1,0.4){\cs{cyclic}}
    (3,0.4){\cs{bclosed}}
  }
 % Some points to help illustrate
  \point{(0.49,3),(.5,3.7),(1,4),(1.5,3.7),(1.51,3)}
  \point{(2.49,3),(2.5,3.7),(3,4),(3.5,3.7),(3.51,3)}
  \point{(0.49,1),(.5,1.7),(1,2),(1.5,1.7),(1.51,1)}
  \point{(2.49,1),(2.5,1.7),(3,2),(3.5,1.7),(3.51,1)}
\end{mfpic}
\end{verbatim}
\begin{mfpfig}{fig34}
  \begin{mfpic}{0}{4}{0}{4}
    \curve{(0.49,3),(.5,3.7),(1,4),(1.5,3.7),(1.51,3)}
    \draw\gfill[green]\sclosed\shiftpath{(2,0)}
        \curve{(0.49,3),(.5,3.7),(1,4),(1.5,3.7),(1.51,3)}
    \draw\gfill[yellow]\bclosed\shiftpath{(2,-2)}
        \curve{(0.49,3),(.5,3.7),(1,4),(1.5,3.7),(1.51,3)}
    \draw\gfill[red]\shiftpath{(0,-2)}
        \cyclic{(0.49,3),(.5,3.7),(1,4),(1.5,3.7),(1.51,3)}
    \tlabeljustify{bc}
    \nomplabels
    \tlabels{
      (1,2.4){\cs{curve}}
      (3,2.4){\cs{sclosed}}
      (1,0.4){\cs{cyclic}}
      (3,0.4){\cs{bclosed}}
    }
    \point{(0.49,3),(.5,3.7),(1,4),(1.5,3.7),(1.51,3)}
    \point{(2.49,3),(2.5,3.7),(3,4),(3.5,3.7),(3.51,3)}
    \point{(0.49,1),(.5,1.7),(1,2),(1.5,1.7),(1.51,1)}
    \point{(2.49,1),(2.5,1.7),(3,2),(3.5,1.7),(3.51,1)}
  \end{mfpic}
\end{mfpfig}

A word about the labels: we turned off \opt{mplabels} with the command
\verb$\nomplabels$, because we used a command (\verb$\cs$) defined for
this document and not known to basic \TeX{} or \LaTeX. The labels
therefore are positioned by \LaTeX{} while it assembles this document,
instead of by \MP{} which would call a separate instance of \TeX{} or
\LaTeX{} where \cs{cs} was unknown. We could have kept \opt{mplabels},
provided we had used \verb$\mfpverbtex$ to write the appropriate
\LaTeX{} preamble to the \file{.mp} output. It would need to be some
subset of the preamble of this document.


\chapter*{Appendices}
\addcontentsline{toc}{section}{Appendices}

In addition to \pdfLaTeX, \Mfp{} works with plain \pdfTeX, \LaTeX, and
plain \TeX{}. Instead of \MP{} as the figure processor, \MF{} can also
be used. Let's start with the difference between using \mfp{} in a
plain~\TeX{} document and using it in a \LaTeX{} document.

\renewcommand\thesubsection{\Alph{subsection}}

\section{\Mfp{} in plain \TeX}

Here is a sample plain \pdfTeX{} document with results the same as our
first ``Hello, world'' example. Let's call this file \file{plfirst}
\begin{verbatim}
\input mfpic
  \usemetapost
\opengraphsfile{myfigs}
  My first figure:
  \mfpic[72]{-1}{1}{-1}{1}
    \ellipse{(0,0),1,.5}
  \endmfpic
\closegraphsfile
\end
\end{verbatim}
The main difference is the lack of \LaTeX{} commands. The crucial
difference is in the first two lines. There we simply
\verb$\input mfpic$ and we turn on \MP{} support with the
\verb$\usemetapost$ command instead of an option to \verb$\usepackage$.

Since \verb$\usepackage$ and its options don't exist in plain \TeX{},
all those features that we select with options in \LaTeX{} must be
selected by some command in plain. For example, the \opt{mplabels}
option is replaced with the command \verb$\usemplabels$ (which can also
be used in \LaTeX).

Also, plain \TeX{} doesn't have environments, so instead of
\verb$\begin{mfpic}$ we just use \verb$\mfpic$ and instead of
\verb$\end{mfpic}$ we use \verb$\endmfpic$.

The external processing is essentially the same:
\begin{verbatim}
pdftex plfirst
mpost myfigs
pdftex plfirst
\end{verbatim}
should produce \texttt{plfirst.pdf} with the same picture of an ellipse.


\section{\Mfp{} without PDF}

If we wish to use nonPDF versions of \LaTeX{} or plain \TeX, the only
difference is in the processing steps. To process \file{first.tex} with
\LaTeX, run the command
\begin{verbatim}
latex first
\end{verbatim}
followed by
\begin{verbatim}
mpost myfigs
\end{verbatim}
followed by latex again.
\begin{verbatim}
latex first
\end{verbatim}
Then run the dvi processor of your choice. It should be one that can
successfully handle eps figures (or at least the simple eps produced by
\MP). Certainly \prog{dvips} can do it:
\begin{verbatim}
dvips first
\end{verbatim}
will produce \file{first.ps}. The file{.ps} file can be viewed with
\prog{gsview} or printed, or converted to PDF with some distillation
program like \prog{ps2pdf}. Also \prog{dvipdfm} (if properly
configured) can be used convert the \file{.dvi} file to PDF.

\section{\Mfp{} without \MP}

\Mfp{} can produce figures using \MF{} instead of \MP. What it does is
work with \MF{} to produce a made-to-order font, where each picture is a
large character in that font.

Since \pdfTeX{} and \pdfLaTeX{} do not work well with the fonts produced
by \MF, and many PDF viewers don't display them well anyway, I do not
recommend using \mfp{} to produce PDF without turning on \MP{} support.
However, all dvi viewers and \prog{dvips} \emph{do} work well with such
fonts, so it can make sense to use \mfp{} with \MF{} \emph{if} you don't
need the features that \MP{} enables: color and rotation of labels. One
advantage of doing this is the smaller number of files produced. If
there are 100 \mfp{} figures in a document, \MP{} produces 100 files
(apart from a couple of temporary files and the \file{.log} file), but
the \MF{} procedure produces only four files no matter how many figures
are present.

To use \mfp{} without \MP{}, omit the \opt{metapost} option or the
\verb$\usemetapost$ command. If you want a visible reminder of the fact
that \MF{} is being used, you can use the \opt{metafont} option or the
\verb$\usemetafont$ command. Of course, you may not use \opt{mplabels}
without \MP. You may use the color commands and options, but the only
colors actually produced will be black and white (and occasionally a
pattern of pixels that simulate gray). The processing steps are
different. After
\begin{ex}
\texttt{latex first} \quad(or \texttt{tex plfirst})
\end{ex}
run \MF:
\begin{ex}
    \texttt{mf myfigs}
\end{ex}
This should produce three files: \texttt{myfigs.log},
\texttt{myfigs.tfm}, and \texttt{myfigs.600gf}. The last one (which
might have a different number on your system) is called a \emph{generic
font} (GF) file and contains the bitmap descriptions. If the file
produced is \texttt{myfigs.2602gf}, and the \file{.tfm} is not produced,
that indicates a configuration problem with your system that we'll get
to later. If this did work, one needs to convert the GF file to a
PK font file, the standard format for bitmap fonts in the \TeX{} world.
This may be done with
\begin{ex}
    \texttt{gftopk myfigs.600gf}
\end{ex}
Some systems may require you to name the output file on the command
line:
\begin{ex}
    \texttt{gftopk myfigs.600gf myfigs.600pk}
\end{ex}
And some systems may require the extension to be simply \file{.pk}:
\begin{ex}
    \texttt{gftopk myfigs.600gf myfigs.pk}
\end{ex}
Finally, some systems may have a \prog{makepk} or \prog{mktexpk} command
that can be used in place of the combination of \MF{} and \prog{gftopk}.
You'll have to check what your system has and what its usage might be,
and what it might do with the PK file produced.

After the above, one again runs `\texttt{latex first}' (or `\texttt{tex
plfirst}'), and then the \file{.dvi} can be viewed or processed with dvips.
The two files \file{myfigs.log} and \file{myfigs.600gf} can be deleted;
only \file{myfigs.tfm} and \file{myfigs.600pk} are needed. If the viewed
image shows the pictures at a far different size than you expect, this can
also indicate a configuration problem.

Some systems permit on-the-fly creation of PK files by various
\file{.dvi} processing programs. It is not wise to allow this to happen
when working with \mfp. The problem is that this automatic creation
process is \emph{not} repeated when a figure is edited unless the old PK
files are deleted, and it may take some hunting to even locate them. One
should \emph{always} follow the \MF{} step with the \prog{gftopk} step.
You might even want to write a batch script or makefile to ensure that
this happens.

Another problem (more an annoyance) that can occur comes from the
behavior of most dvi viewers: most will reload a \file{.dvi} file if
they detect that it has changed (or if asked to), but none that I know
of will reload any fonts even if they have changed. So if one is going
through a edit-compile-view cycle involving \mfp{} figures, one usually
has to close the viewer and open it again before one can see changes
that were made in the figures after starting the viewer. It is also
possible that PK fonts are cached and shared by other programs, so you
may need to close other programs to ensure the cache is cleared and the
new figures loaded.

\section{\MF{} configuration problems}

To diagnose these problems it important to know something about
\emph{printer modes}. \CMF{} produces bitmap images of characters. This
means a description of a block of pixels, telling which ones are black
and which are white. If the description says that 60 pixels in a row are
black, that produces a thin black line. How long that line is depends on
the size of a printer's pixels. For the LaserJet IV, there are 600
pixels to the inch, so 60 pixels makes 1/10 of an inch. The LaserJet~II,
however, has 300 pixels to the inch, so 60 pixels is 1/5 of an inch
long. What \MF{} needs in order to produce an image that is the correct
size is (at a minimum) the \emph{resolution} of the intended printer.
This is typically reported in DPI (dots per inch) and \MF{} keeps the
value in the variable \mfc{pixels_per_inch}.

As part of the configuration of your DVI viewer or of \prog{dvips} you
may have needed to select a printer from a list, or edit a line in some
configuration file (e.g., \file{config.ps}). What was going on then was
the assigning of a default \MF{} printer mode. There is a file on most
\TeX{} systems named \file{modes.mf} which assigns symbolic names to a
set of parameters that enable \MF{} to tune its output to a particular
printer. For example, the LaserJet~IV is given the name
`\texttt{ljfour}' and that name is associated with the value 600 for
\mfc{pixels_per_inch}. In order to tell \MF{} to make output for the
LaserJet~IV, one can put that information on the command line:
\begin{ex}
    \texttt{mf \cs{mode}:=ljfour; input myfigs}
\end{ex}
Your operating system or \TeX{} distribution may require you to quote
the backslash in the above command.

There is a system for making the selection of the correct mode
semi-automatic, not requiring a command line specification. Near the end
of \file{modes.mf} is a line similar to
\begin{ex}
    \mfc{localfont:=ljfour;}
\end{ex}
This is intended to equate the symbolic name \mfc{localfont} with the
user's default printer. If the LaserJet~IV is your default printer, the
line above would be the correct one. If it is not, then that line should
be changed. This can be done with an ordinary text editor, or your
\TeX{} system may have a configuration utility to take care of it.

If you say ``\texttt{mf myfigs}'' on an \mfp{} file \file{myfigs.mf}, \mfp's
internal code will detect that no mode was defined on the command line.
It will then check if \mfc{localfont} is defined and if so, use that for
the printer mode. If that fails, it will try to select \mfc{ljfour}.
If even that is unknown, \mfp{} will define its own generic 600 DPI mode.

\Mfp{} doesn't need to know all the parameters associated to a printer
mode, only the value of \mfc{pixels_per_inch}. If you get a GF file that
indicates an incorrect DPI value for your printer, you should arrange
for the line in \file{modes.mf} that sets \mfc{localfont} to be
corrected. At the very least it should equate \mfc{localfont} to a name
defined in \file{modes.mf} and associated to a printer with the same DPI
as yours. After changing \file{modes.mf}, you need to run whatever
programs your \TeX{} system requires to remake the \MF{} format.

\end{document}
