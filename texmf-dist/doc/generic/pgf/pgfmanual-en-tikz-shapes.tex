% Copyright 2019 by Till Tantau
%
% This file may be distributed and/or modified
%
% 1. under the LaTeX Project Public License and/or
% 2. under the GNU Free Documentation License.
%
% See the file doc/generic/pgf/licenses/LICENSE for more details.


\section{Nodes and Edges}
\label{section-nodes}

\subsection{Overview}

In the present section, the usage of \emph{nodes} in \tikzname\ is explained. A
node is typically a rectangle or circle or another simple shape with some text
on it.

Nodes are added to paths using the special path operation |node|. Nodes
\emph{are not part of the path itself}. Rather, they are added to the picture
just before or after the path has been drawn.

In Section~\ref{section-nodes-basic} the basic syntax of the node operation is
explained, followed in Section~\ref{section-nodes-multi} by the syntax for
multi-part nodes, which are nodes that contain several different text parts.
After this, the different options for the text in nodes are explained. In
Section~\ref{section-nodes-anchors} the concept of \emph{anchors} is introduced
along with their usage. In Section~\ref{section-nodes-transformations} the
different ways transformations affect nodes are studied.
Sections~\ref{section-nodes-placing-1} and~\ref{section-nodes-placing-2} are
about placing nodes on or next to straight lines and curves.
Section~\ref{section-nodes-connecting} explains how a node can be used as a
``pseudo-coordinate''. Section~\ref{section-nodes-edges} introduces the |edge|
operation, which works similar to the |to| operation and also similar to the
|node| operation.


\subsection{Nodes and Their Shapes}
\label{section-nodes-basic}

In the simplest case, a node is just some text that is placed at some
coordinate. However, a node can also have a border drawn around it or have a
more complex background and foreground. Indeed, some nodes do not have a text
at all, but consist solely of the background. You can name nodes so that you
can reference their coordinates later in the same picture or, if certain
precautions are taken as explained in Section~\ref{section-cross-picture-tikz},
also in different pictures.

There are no special \TeX\ commands for adding a node to a picture; rather,
there is path operation called |node| for this. Nodes are created whenever
\tikzname\ encounters |node| or |coordinate| at a point on a path where it
would expect a normal path operation (like |-- (1,1)| or |rectangle (1,1)|). It
is also possible to give node specifications \emph{inside} certain path
operations as explained later.

The node operation is typically followed by some options, which apply only to
the node. Then, you can optionally \emph{name} the node by providing a name in
parentheses. Lastly, for the |node| operation you must provide some label text
for the node in curly braces, while for the |coordinate| operation you may not.
The node is placed at the current position of the path either \emph{after the
path has been drawn} or (more seldomly and only if you add the |behind path|
option) \emph{just before the path is drawn.} Thus, all nodes are drawn ``on
top'' or ``behind'' the path and are retained until the path is complete. If
there are several nodes on a path, perhaps some behind and some on top of the
path, first come the nodes behind the path in the order they were encountered,
then comes that path, and then come the remaining node, again in the order they
are encountered.
%
\begin{codeexample}[]
\tikz \fill [fill=yellow!80!black]
     (0,0) node              {first node}
  -- (1,1) node[behind path] {second node}
  -- (2,0) node              {third node};
\end{codeexample}


\subsubsection{Syntax of the Node Command}

The syntax for specifying nodes is the following:
%
\begin{pathoperation}{node}{
    \opt{\meta{foreach statements}}
    \opt{|[|\meta{options}|]|}
    \opt{|(|\meta{name}|)|}
    \opt{|at(|\meta{coordinate}|)|}
    \opt{|:|\meta{animation attribute}|=|\marg{options}}
    \opt{\marg{node contents}}%
}
    Since this path operation is one of the most involved around, let us go
    over it step by step.

    \medskip
    \textbf{Order of the parts of the specification.}
    Everything between ``|node|'' and the opening brace of a node is optional.
    If there are \meta{foreach statements}, they must come first, directly
    following ``|node|''. Other than that, the ordering of all the other
    elements of a node specification (the \meta{options}, the  \meta{name},
    \meta{coordinate}, and \meta{animation attribute}) is arbitrary, indeed,
    there can be multiple occurrences of any of these elements (although for
    the name and the coordinate this makes no sense).

    \medskip
    \textbf{The text of a node.}
    At the end of a node, you must (normally) provide some \meta{node contents}
    in curly braces; indeed, the ``end'' of the node specification is detected
    by the opening curly brace. For normal nodes it is possible to use
    ``fragile'' stuff inside the \meta{node contents} like the |\verb| command
    (for the technically savvy: code inside the \meta{node contents} is allowed
    to change catcodes; however, this rule does not apply to ``nodes on a
    path'' to be discussed later).

    Instead of giving \meta{node contents} at the end of the node in curly
    braces, you can also use the following key:
    %
    \begin{key}{/tikz/node contents=\meta{node contents}}
    \label{option-node-contents}%
        This key sets the contents of the node to the given text as if you had
        given it at the end in curly braces. When the option is used inside the
        options of a node, the parsing of the node stops immediately after the
        end of the option block. In particular, the option block cannot be
        followed by further option blocks or curly braces (or, rather, these do
        not count as part of the node specification.) Also note that the
        \meta{node contents} may not contain fragile stuff since the catcodes
        get fixed upon reading the options. Here is an example:
        %
\begin{codeexample}[]
\tikz {
  \path (0,0) node [red]                    {A}
        (1,0) node [blue]                   {B}
        (2,0) node [green, node contents=C]
        (3,0) node [node contents=D]           ;
}
\end{codeexample}
        %
\end{key}

    \medskip
    \textbf{Specifying the location of the node.}
    Nodes are placed at the last position mentioned on the path. The effect of
    adding ``|at|'' to a node specification is that the coordinate given after
    |at| is used instead. The |at| syntax is not available when a node is given
    inside a path operation (it would not make any sense there).

    \begin{key}{/tikz/at=\meta{coordinate}}
        This is another way of specifying the |at| coordinate. Note that,
        typically, you will have to enclose the \meta{coordinate} in curly
        braces so that a comma inside the \meta{coordinate} does not confuse
        \TeX.
    \end{key}

    Another aspect of the ``location'' of a node is whether it appears \emph{in
    front of} or \emph{behind} the current path. You can change which of these
    two possibilities happens on a node-by-node basis using the following keys:
    %
    \begin{key}{/tikz/behind path}
        When this key is set, either as a local option for the node or some
        surrounding scope, the node will be drawn behind the current path. For
        this, \tikzname\ collects all nodes defined on the current path with
        this option set and then inserts all of them, in the order they appear,
        just before it draws the path. Thus, several nodes with this option set
        may obscure one another, but never the path itself. ``Just before it
        draws the path'' actually means that the nodes are inserted into the
        page output just before any pre-actions are applied to the path (see
        below for what pre-actions are).
        %
\begin{codeexample}[]
\tikz \fill [fill=blue!50, draw=blue, very thick]
      (0,0)   node [behind path, fill=red!50]   {first node}
   -- (1.5,0) node [behind path, fill=green!50] {second node}
   -- (1.5,1) node [behind path, fill=brown!50] {third node}
   -- (0,1)   node [             fill=blue!30]  {fourth node};
\end{codeexample}

        Note that |behind path| only applies to the current path; not to the
        current scope or picture. To put a node ``behind everything'' you need
        to use layers and options like |on background layer|, see the
        |backgrounds| library in Section~\ref{section-tikz-backgrounds}.
    \end{key}

    \begin{key}{/tikz/in front of path}
        This is the opposite of |behind path|: It causes nodes to be drawn on
        top of the path. Since this is the default behavior, you usually do
        not need this option; it is only needed when an enclosing scope has
        used |behind path| and you now wish to ``switch back'' to the normal
        behavior.
    \end{key}

    \medskip
    \textbf{The name of a node.}
    The |(|\meta{name}|)| is a name for later reference and it is optional. You
    may also add the option |name=|\meta{name} to the \meta{option} list; it
    has the same effect.

    \begin{key}{/tikz/name=\meta{node name}}
        Assigns a name to the node for later reference. Since this is a
        ``high-level'' name (drivers never know of it), you can use spaces,
        number, letters, or whatever you like when naming a node. Thus, you can
        name a node just |1| or perhaps |start of chart| or even |y_1|. Your
        node name should \emph{not} contain any punctuation like a dot, a
        comma, or a colon since these are used to detect what kind of
        coordinate you mean when you reference a node.
    \end{key}

    \begin{key}{/tikz/alias=\meta{another node name}}
        This option allows you to provide another name for the node. Giving
        this option multiple times will allow you to access the node via
        several aliases. Using the |node also| syntax, you can also assign an
        alias name to a node at a later point, see
        Section~\ref{section-node-also}.
    \end{key}

    \medskip
    \textbf{The options of a node.}
    The \meta{options} is an optional list of options that \emph{apply only to
    the node} and have no effect outside. The other way round, most ``outside''
    options also apply to the node, but not all. For example, the ``outside''
    rotation does not apply to nodes (unless some special options are used,
    sigh). Also, the outside path action, like |draw| or |fill|, never applies
    to the node and must be given in the node (unless some special other
    options are used, deep sigh).

    \medskip
    \textbf{The shape of a node.}
    As mentioned before, we can add a border and even a background to a node:
    %
\begin{codeexample}[]
\tikz \fill[fill=yellow!80!black]
      (0,0) node {first node}
   -- (1,1) node[draw, behind path] {second node}
   -- (0,2) node[fill=red!20,draw,double,rounded corners] {third node};
\end{codeexample}

    The ``border'' is actually just a special case of a much more general
    mechanism. Each node has a certain \emph{shape} which, by default, is a
    rectangle. However, we can also ask \tikzname\ to use a circle shape
    instead or an ellipse shape (you have to include one of the
    |shapes.geometric| library for the latter shape):
    %
\begin{codeexample}[preamble={\usetikzlibrary{shapes.geometric}}]
\tikz \fill[fill=yellow!80!black]
      (0,0) node                            {first node}
   -- (1,1) node[ellipse,draw, behind path] {second node}
   -- (0,2) node[circle,fill=red!20]        {third node};
\end{codeexample}

    There are many more shapes available such as, say, a shape for a resistor
    or a large arrow, see the |shapes| library in
    Section~\ref{section-libs-shapes} for details.

    To select the shape of a node, the following option is used:
    %
    \begin{key}{/tikz/shape=\meta{shape name} (initially rectangle)}
        Select the shape either of the current node or, when this option is not
        given inside a node but somewhere outside, the shape of all nodes in
        the current scope.%
        \indexoption{\meta{shape name}}

        Since this option is used often, you can leave out the |shape=|. When
        \tikzname\ encounters an option like |circle| that it does not know, it
        will, after everything else has failed, check whether this option is
        the name of some shape. If so, that shape is selected as if you had
        said |shape=|\meta{shape name}.

        By default, the following shapes are available: |rectangle|, |circle|,
        |coordinate|. Details of these shapes, like their anchors and size
        options, are discussed in Section~\ref{section-the-shapes}.
    \end{key}

    \medskip
    \textbf{Animating a node.}
    When you say |:|\meta{animation attribute}|={|\meta{options}|}|, an
    \emph{animation} of the specified attribute is added to the node.
    Animations are discussed in detail in
    Section~\ref{section-tikz-animations}. Here is a typical example of how
    this syntax can be used:
    %
\begin{codeexample}[preamble={\usetikzlibrary{animations}},animation list={0.5,1,1.5,2}]
\tikz
  \node  :fill opacity = { 0s="1", 2s="0", begin on=click }
         :rotate = { 0s="0", 2s="90", begin on=click }
         [fill = blue!20, draw = blue, ultra thick, circle]
    {Click me!};
\end{codeexample}

    \medskip
    \textbf{The foreach statement for nodes.}
    At the beginning of a node specification (and only there) you can provide
    multiple \meta{foreach statements}, each of which has the form |foreach|
    \meta{var} |in| \meta{list} (note that there is no backslash before
    |foreach|). When they are given, instead of a single node, multiple nodes
    will be created: The \meta{var} will iterate over all values of \meta{list}
    and for each of them, a new node is created. These nodes are all created
    using all the text following the \meta{foreach statements}, but in each
    copy the \meta{var} will have the current value of the current element in
    the \meta{list}.

    As an example, the following two codes have the same effect:
    %
\begin{codeexample}[]
\tikz \draw (0,0) node foreach \x in {1,2,3} at (\x,0) {\x};
\end{codeexample}
\begin{codeexample}[]
\tikz \draw (0,0) node at (1,0) {1} node at (2,0) {2} node at (3,0) {3};
\end{codeexample}
%
    When you provide several |foreach| statements, they work like ``nested
    loops'':
    %
\begin{codeexample}[]
\tikz \node foreach \x in {1,...,4} foreach \y in {1,2,3}
            [draw] at (\x,\y) {\x,\y};
\end{codeexample}
    %
    As the example shows, a \meta{list} can contain ellipses (three dots) to
    indicate that a larger number of numbers is meant. Indeed, you can use the
    full power of the |\foreach| command here, including multiple parameters
    and options, see Section~\ref{section-foreach}.

    \medskip
    \textbf{Styles for nodes.}
    The following styles influence how nodes are rendered:
    %
    \begin{stylekey}{/tikz/every node (initially \normalfont empty)}
        This style is installed at the beginning of every node.
        %
\begin{codeexample}[]
\begin{tikzpicture}[every node/.style={draw}]
  \draw (0,0) node {A} -- (1,1) node {B};
\end{tikzpicture}
\end{codeexample}
    \end{stylekey}
    %
    \begin{stylekey}{/tikz/every \meta{shape} node (initially \normalfont empty)}
        These styles are installed at the beginning of a node of a given
        \meta{shape}. For example, |every rectangle node| is used for rectangle
        nodes, and so on.
        %
\begin{codeexample}[]
\begin{tikzpicture}
  [every rectangle node/.style={draw},
   every circle node/.style={draw,double}]
  \draw (0,0) node[rectangle] {A} -- (1,1) node[circle] {B};
\end{tikzpicture}
\end{codeexample}
    \end{stylekey}

    \begin{key}{/tikz/execute at begin node=\meta{code}}
        This option causes \meta{code} to be executed at the beginning of a
        node. Using this option multiple times will cause the code to
        accumulate.
    \end{key}

    \begin{key}{/tikz/execute at end node=\meta{code}}
        This option installs \meta{code} that will be executed at the end of
        the node. Using this option multiple times will cause the code to
        accumulate.
        %
\begin{codeexample}[]
\begin{tikzpicture}
  [execute at begin node={A},
   execute at end node={D}]
  \node[execute at begin node={B}] {C};
\end{tikzpicture}
\end{codeexample}
    %
    \end{key}

    \medskip
    \textbf{Name scopes.}
    It turns out that the name of a node can further be influenced using two
    keys:
    %
    \begin{key}{/tikz/name prefix=\meta{text} (initially \normalfont empty)}
        The value of this key is prefixed to every node inside the current
        scope. This includes both the naming of the node (via the |name| key or
        via the implicit |(|\meta{name}|)| syntax) as well as any referencing
        of the node. Outside the scope, the nodes can (and need to) be
        referenced using ``full name'' consisting of the prefix and the node
        name.

        The net effect of this is that you can set the name prefix at the
        beginning of a scope to some value and then use short and simple names
        for the nodes inside the scope. Later, outside the scope, you can
        reference the nodes via their full name:
        %
\begin{codeexample}[]
\tikz {
  \begin{scope}[name prefix = top-]
    \node (A) at (0,1) {A};
    \node (B) at (1,1) {B};
    \draw (A) -- (B);
  \end{scope}
  \begin{scope}[name prefix = bottom-]
    \node (A) at (0,0) {A};
    \node (B) at (1,0) {B};
    \draw (A) -- (B);
  \end{scope}

  \draw [red] (top-A) -- (bottom-B);
}
\end{codeexample}
        %
        As can be seen, name prefixing makes it easy to write reusable code.
    \end{key}
    %
    \begin{key}{/tikz/name suffix=\meta{text} (initially \normalfont empty)}
        Works as |name prefix|, only the \meta{text} is appended to every node
        name in the current scope.
    \end{key}
\end{pathoperation}

There is a special syntax for specifying ``light-weight'' nodes:

\begin{pathoperation}{coordinate}{\opt{|[|\meta{options}|]|}|(|\meta{name}|)|\opt{|at(|\meta{coordinate}|)|}}
    This has the same effect as

    |\node[shape=coordinate]|\verb|[|\meta{options}|](|\meta{name}|)at(|\meta{coordinate}|){}|,

    where the |at| part may be omitted.
\end{pathoperation}

Since nodes are often the only path operation on paths, there are two special
commands for creating paths containing only a node:

\begin{command}{\node}
    Inside |{tikzpicture}| this is an abbreviation for |\path node|.
\end{command}

\begin{command}{\coordinate}
    Inside |{tikzpicture}| this is an abbreviation for |\path coordinate|.
\end{command}


\subsubsection{Predefined Shapes}
\label{section-nodes-predefined}
\label{section-the-shapes}

\pgfname\ \todosp{why two labels for the same point? The first doesn't seem to
be used anywhere} and \tikzname\ define three shapes, by default:
%
\begin{itemize}
    \item |rectangle|,
    \item |circle|, and
    \item |coordinate|.
\end{itemize}
%
By loading library packages, you can define more shapes like ellipses or
diamonds; see Section~\ref{section-libs-shapes} for the complete list of
shapes.

\label{section-tikz-coordinate-shape}%
The |coordinate| shape is handled in a special way by \tikzname. When a node
|x| whose shape is |coordinate| is used as a coordinate |(x)|, this has the
same effect as if you had said |(x.center)|. None of the special ``line
shortening rules'' apply in this case. This can be useful since, normally, the
line shortening causes paths to be segmented and they cannot be used for
filling. Here is an example that demonstrates the difference:
%
\begin{codeexample}[]
\begin{tikzpicture}[every node/.style={draw}]
  \path[yshift=1.5cm,shape=rectangle]
    (0,0) node(a1){} (1,0) node(a2){}
    (1,1) node(a3){} (0,1) node(a4){};
  \filldraw[fill=yellow!80!black] (a1) -- (a2) -- (a3) -- (a4);

  \path[shape=coordinate]
    (0,0) coordinate(b1) (1,0) coordinate(b2)
    (1,1) coordinate(b3) (0,1) coordinate(b4);
  \filldraw[fill=yellow!80!black] (b1) -- (b2) -- (b3) -- (b4);
\end{tikzpicture}
\end{codeexample}


\subsubsection{Common Options: Separations, Margins, Padding and
               Border Rotation}
\label{section-shape-seps}
\label{section-shape-common-options}

The \todosp{why two labels for the same point?} exact behavior of shapes
differs, shapes defined for more special purposes (like a, say, transistor
shape) will have even more custom behaviors. However, there are some options
that apply to most shapes:

\begin{key}{/pgf/inner sep=\meta{dimension} (initially .3333em)}
        \keyalias{tikz}
    An additional (invisible) separation space of \meta{dimension} will be
    added inside the shape, between the text and the shape's background path.
    The effect is as if you had added appropriate horizontal and vertical skips
    at the beginning and end of the text to make it a bit ``larger''.

    For those familiar with \textsc{css}, this is the same as \emph{padding}.
    %
\begin{codeexample}[]
\begin{tikzpicture}
  \draw (0,0)     node[inner sep=0pt,draw] {tight}
        (0cm,2em) node[inner sep=5pt,draw] {loose}
        (0cm,4em) node[fill=yellow!80!black]   {default};
\end{tikzpicture}
\end{codeexample}
    %
\end{key}

\begin{key}{/pgf/inner xsep=\meta{dimension} (initially .3333em)}
        \keyalias{tikz}
    Specifies the inner separation in the $x$-direction, only.
\end{key}

\begin{key}{/pgf/inner ysep=\meta{dimension} (initially .3333em)}
        \keyalias{tikz}
    Specifies the inner separation in the $y$-direction, only.
\end{key}

\begin{key}{/pgf/outer sep=\meta{dimension or ``auto''}}
        \keyalias{tikz}
    This option adds an additional (invisible) separation space of
    \meta{dimension} outside the background path. The main effect of this
    option is that all anchors will move a little ``to the outside''.

    For those familiar with \textsc{css}, this is same as \emph{margin}.

    The default for this option is half the line width. When the default is
    used and when the background path is draw, the anchors will lie exactly on
    the ``outside border'' of the path (not on the path itself).
    %
\begin{codeexample}[]
\begin{tikzpicture}
  \draw[line width=5pt]
    (0,0)  node[fill=yellow!80!black] (f) {filled}
    (2,0)  node[draw]                 (d) {drawn}
    (1,-2) node[draw,scale=2]         (s) {scaled};

  \draw[->] (1,-1) -- (f);
  \draw[->] (1,-1) -- (d);
  \draw[->] (1,-1) -- (s);
\end{tikzpicture}
\end{codeexample}

    As the above example demonstrates, the standard settings for the outer sep
    are not always ``correct''. First, when a shape is filled, but not drawn,
    the outer sep should actually be |0|. Second, when a node is scaled, for
    instance by a factor of 5, the outer separation also gets scaled by a
    factor of 5, while the line width stays at its original width; again
    causing problems.

    In such cases, you can say |outer sep=auto| to make \tikzname\ \emph{try}
    to compensate for the effects described above. This is done by, firstly,
    setting the outer sep to |0| when no drawing is done and, secondly, setting
    the outer separations to half the line width (as before) times two
    adjustment factors, one for the horizontal separations and one for the
    vertical separations (see Section~\ref{section-adjustment-transformations}
    for details on these factors). Note, however, that these factors can
    compensate only for transformations that are either scalings plus rotations
    or scalings with different magnitudes in the horizontal and the vertical
    direction. If you apply slanting, the factors will only approximate the
    correct values.

    In general, it is a good idea to say |outer sep=auto| at some early stage.
    It is not the default mainly for compatibility with earlier versions.
    %
\begin{codeexample}[]
\begin{tikzpicture}[outer sep=auto]
  \draw[line width=5pt]
    (0,0)  node[fill=yellow!80!black] (f) {filled}
    (2,0)  node[draw]                 (d) {drawn}
    (1,-2) node[draw,scale=2]         (s) {scaled};

  \draw[->] (1,-1) -- (f);
  \draw[->] (1,-1) -- (d);
  \draw[->] (1,-1) -- (s);
\end{tikzpicture}
\end{codeexample}
    %
\end{key}

\begin{key}{/pgf/outer xsep=\meta{dimension} (initially .5\string\pgflinewidth)}
        \keyalias{tikz}
    Specifies the outer separation in the $x$-direction, only. This value will
    be overwritten when |outer sep| is set, either to the value given there or
    a computed value in case of |auto|.
\end{key}

\begin{key}{/pgf/outer ysep=\meta{dimension} (initially .5\string\pgflinewidth)}
        \keyalias{tikz}
    Specifies the outer separation in the $y$-direction, only.
\end{key}

\begin{key}{/pgf/minimum height=\meta{dimension} (initially 1pt)}
        \keyalias{tikz}
    This option ensures that the height of the shape (including the inner, but
    ignoring the outer separation) will be at least \meta{dimension}. Thus, if
    the text plus the inner separation is not at least as large as
    \meta{dimension}, the shape will be enlarged appropriately. However, if the
    text is already larger than \meta{dimension}, the shape will not be shrunk.
    %
\begin{codeexample}[]
\begin{tikzpicture}
  \draw (0,0) node[minimum height=1cm,draw] {1cm}
        (2,0) node[minimum height=0cm,draw] {0cm};
\end{tikzpicture}
\end{codeexample}
    %
\end{key}

\begin{key}{/pgf/minimum width=\meta{dimension} (initially 1pt)}
        \keyalias{tikz}
    Same as |minimum height|, only for the width.
    %
\begin{codeexample}[]
\begin{tikzpicture}
  \draw (0,0) node[minimum height=2cm,minimum width=3cm,draw] {$3 \times 2$};
\end{tikzpicture}
\end{codeexample}
    %
\end{key}

\begin{key}{/pgf/minimum size=\meta{dimension}}
        \keyalias{tikz}
    Sets both the minimum height and width at the same time.
    %
\begin{codeexample}[]
\begin{tikzpicture}
  \draw (0,0)  node[minimum size=2cm,draw] {square};
  \draw (0,-2) node[minimum size=2cm,draw,circle] {circle};
\end{tikzpicture}
\end{codeexample}
    %
\end{key}

\begin{key}{/pgf/shape aspect=\meta{aspect ratio}}
        \keyalias{tikz}
    Sets a desired aspect ratio for the shape. For the |diamond| shape, this
    option sets the ratio between width and height of the shape.
    %
\begin{codeexample}[preamble={\usetikzlibrary{shapes.geometric}}]
\begin{tikzpicture}
  \draw (0,0)  node[shape aspect=1,diamond,draw] {aspect 1};
  \draw (0,-2) node[shape aspect=2,diamond,draw] {aspect 2};
\end{tikzpicture}
\end{codeexample}
    %
\end{key}

    \label{section-rotating-shape-borders}

Some shapes (but not all), support a special kind of rotation. This rotation
affects only the border of a shape and is independent of the node contents, but
\emph{in addition} to any other transformations.
%
\begin{codeexample}[preamble={\usetikzlibrary{shapes.geometric}}]
\tikzset{every node/.style={dart, shape border uses incircle,
  inner sep=1pt, draw}}
\tikz \node foreach \a/\b/\c in {A/0/0, B/45/0, C/0/45, D/45/45}
            [shape border rotate=\b, rotate=\c] at (\b/36,-\c/36) {\a};
\end{codeexample}

There are two types of rotation: restricted and unrestricted. Which type of
rotation is applied is determined by on how the shape border is constructed. If
the shape border is constructed using an incircle, that is, a circle that
tightly fits the node contents (including the |inner sep|), then the rotation
can be unrestricted. If, however, the border is constructed using the natural
dimensions of the node contents, the rotation is restricted to integer
multiples of 90 degrees.

Why should there be two kinds of rotation and border construction? Borders
constructed using the natural dimensions of the node contents provide a much
tighter fit to the node contents, but to maintain this tight fit, the border
rotation must be restricted to integer multiples of 90 degrees. By using an
incircle, unrestricted rotation is possible, but the border will not make a
very tight fit to the node contents.
%
\begin{codeexample}[preamble={\usetikzlibrary{shapes.geometric}}]
\tikzset{every node/.style={isosceles triangle, draw}}
\begin{tikzpicture}
  \node {abc};
  \node [shape border uses incircle] at (2,0) {abc};
\end{tikzpicture}
\end{codeexample}

There are \pgfname{} keys that determine how a shape border is constructed, and
to specify its rotation. It should be noted that not all shapes support these
keys, so reference should be made to the documentation for individual shapes.

\begin{key}{/pgf/shape border uses incircle=\opt{\meta{boolean}} (default true)}
        \keyalias{tikz}
    Determines if the border of a shape is constructed using the incircle. If
    no value is given \meta{boolean} will take the default value |true|.
\end{key}

\begin{key}{/pgf/shape border rotate=\meta{angle} (initially 0)}
        \keyalias{tikz}
    Rotates the border of a shape independently of the node contents, but in
    addition to any other transformations. If the shape border is not
    constructed using the incircle, the rotation will be rounded to the nearest
    integer multiple of 90 degrees when the shape is drawn.
\end{key}

Note that if the border of the shape is rotated, the compass point anchors, and
`text box' anchors (including |mid east|, |base west|, and so on), \emph{do not
rotate}, but the other anchors do:
%
\begin{codeexample}[preamble={\usetikzlibrary{shapes.geometric}}]
\tikzset{every node/.style={shape=trapezium, draw, shape border uses incircle}}
\begin{tikzpicture}
  \node at (0,0)  (A) {A};
  \node [shape border rotate=30] at (1.5,0) (B) {B};
  \foreach \s/\t in
    {left side/base east, bottom side/north, bottom left corner/base}{
       \fill[red]  (A.\s) circle(1.5pt) (B.\s) circle(1.5pt);
       \fill[blue] (A.\t) circle(1.5pt) (B.\t) circle(1.5pt);
  }
\end{tikzpicture}
\end{codeexample}

Finally, a somewhat unfortunate side-effect of rotating shape borders is that
the supporting shapes do not distinguish between |outer xsep| and |outer ysep|,
and typically, the larger of the two values will be used.


\subsection{Multi-Part Nodes}
\label{section-nodes-multi}

Most nodes just have a single simple text label. However, nodes of a more
complicated shape might be made up from several \emph{node parts}. For example,
in automata theory a so-called Moore state has a state name, drawn in the upper
part of the state circle, and an output text, drawn in the lower part of the
state circle. These two parts are quite independent. Similarly, a \textsc{uml}
class shape would have a name part, a method part, and an attributes part.
Different molecule shapes might use parts for the different atoms to be drawn
at the different positions, and so on.

Both \pgfname\ and \tikzname\ support such multipart nodes. On the lower level,
\pgfname\ provides a system for specifying that a shape consists of several
parts. On the \tikzname\ level, you specify the different node parts by using
the following command:

\begin{command}{\nodepart\opt{|[|\meta{options}|]|}\marg{part name}}
    This command can only be used inside the \meta{text} argument of a |node|
    path operation. It works a little bit like a |\part| command in \LaTeX. It
    will stop the typesetting of whatever node part was typeset until now and
    then start putting all following text into the node part named \meta{part
    name} -- until another |\nodepart| is encountered or until the node
    \meta{text} ends. The \meta{options} will be local to this part.
    %
\begin{codeexample}[preamble={\usetikzlibrary{shapes.multipart}}]
\begin{tikzpicture}
  \node [circle split,draw,double,fill=red!20]
  {
    % No \nodepart has been used, yet. So, the following is put in the
    % ``text'' node part by default.
    $q_1$
    \nodepart{lower} % Ok, end ``text'' part, start ``output'' part
    $00$
  }; % output part ended.
\end{tikzpicture}
\end{codeexample}

    You will have to lookup which parts are defined by a shape.

    The following styles influences node parts:
    %
    \begin{stylekey}{/tikz/every \meta{part name} node part (initially \normalfont empty)}
        This style is installed at the beginning of every node part named
        \meta{part name}.
        %
\begin{codeexample}[preamble={\usetikzlibrary{shapes.multipart}}]
\tikz [every lower node part/.style={red}]
  \node [circle split,draw] {$q_1$ \nodepart{lower} $00$};
\end{codeexample}
    \end{stylekey}
\end{command}


\subsection{The Node Text}
\label{section-nodes-options}

\subsubsection{Text Parameters: Color and Opacity}

The simplest option for the text in nodes is its color. Normally, this color is
just the last color installed using |color=|, possibly inherited from another
scope. However, it is possible to specifically set the color used for text
using the following option:

\begin{key}{/tikz/text=\meta{color}}
    Sets the color to be used for text labels. A |color=| option will
    immediately override this option.
    %
\begin{codeexample}[]
\begin{tikzpicture}
  \draw[red]       (0,0) -- +(1,1) node[above]     {red};
  \draw[text=red]  (1,0) -- +(1,1) node[above]     {red};
  \draw            (2,0) -- +(1,1) node[above,red] {red};
\end{tikzpicture}
\end{codeexample}
    %
\end{key}

Just like the color itself, you may also wish to set the opacity of the text
only. For this, use the |text opacity| option, which is detailed in
Section~\ref{section-tikz-transparency}.


\subsubsection{Text Parameters: Font}

Next, you may wish to adjust the font used for the text. Naturally, you can
just use a font command like |\small| or |\rm| at the beginning of a node.
However, the following two options make it easier to set the font used in nodes
on a general basis. Let us start with:

\begin{key}{/tikz/node font=\meta{font commands}}
    This option sets the font used for all text used in a node.
    %
\begin{codeexample}[]
\begin{tikzpicture}
  \draw[node font=\itshape] (1,0) -- +(1,1) node[above] {italic};
\end{tikzpicture}
\end{codeexample}
    %
    Since the \meta{font commands} are executed at a very early stage in the
    construction of the node, the font selected using this command will also
    dictate the values of dimensions defined in terms of |em| or |ex|. For
    instance, when the |minimum height| of a node is |3em|, the actual height
    will be (at least) three times the line distance selected by the \meta{font
    commands}:
\begin{codeexample}[]
\tikz \node [node font=\tiny,  minimum height=3em, draw] {tiny};
\tikz \node [node font=\small, minimum height=3em, draw] {small};
\end{codeexample}
    %
\end{key}

The other font command is:
%
\begin{key}{/tikz/font=\meta{font commands}}
    Sets the font used for the text inside nodes. However, this font will
    \emph{not} (yet) be installed when any of the dimensions of the node are
    being computed, so dimensions like |1em| will be with respect to the font
    used outside the node (usually the font that was in force when the picture
    started).
    %
\begin{codeexample}[]
\begin{tikzpicture}
  \node [font=\itshape] {italic};
\end{tikzpicture}
\end{codeexample}

\begin{codeexample}[]
\tikz \node [font=\tiny,  minimum height=3em, draw] {tiny};
\tikz \node [font=\small, minimum height=3em, draw] {small};
\end{codeexample}

    A useful example of how the |font| option can be used is the following:
    %
\begin{codeexample}[preamble={\usetikzlibrary{shapes.multipart}}]
\tikz [every text node part/.style={font=\itshape},
       every lower node part/.style={font=\footnotesize}]
  \node [circle split,draw] {state \nodepart{lower} output};
\end{codeexample}

    As can be seen, the font can be changed for each node part. This does
    \emph{not} work with the |node font| command since, as the name suggests,
    this command can only be used to select the ``overall'' font for the node
    and this is done very early.
\end{key}


\subsubsection{Text Parameters: Alignment and Width for Multi-Line Text}

Normally, when a node is typeset, all the text you give in the braces is put in
one long line (in an |\hbox|, to be precise) and the node will become as wide
as necessary.

From time to time you may wish to create nodes that contain multiple lines of
text. There are three different ways of achieving this:
%
\begin{enumerate}
    \item Inside the node, you can put some standard environment that produces
        multi-line, aligned text. For instance, you can use a |{tabular}|
        inside a node:
        %
\begin{codeexample}[width=5cm]
\tikz \node [draw] {
  \begin{tabular}{cc}
    upper left & upper right\\
    lower left & lower right
  \end{tabular}
};
\end{codeexample}
        %
        This approach offers the most flexibility in the sense that it allows
        you to use all of the alignment commands offered by your format of
        choice.
    \item You use |\\| inside your node to mark the end of lines and then
        request \tikzname\ to arrange these lines in some manner. This will
        only be done, however, if the |align| option has been given.
        %
\begin{codeexample}[]
\tikz[align=left] \node[draw] {This is a\\demonstration.};
\end{codeexample}
        %
\begin{codeexample}[]
\tikz[align=center] \node[draw] {This is a\\demonstration.};
\end{codeexample}
        %
        The |\\| command takes an optional extra space as an argument in square
        brackets.
        %
\begin{codeexample}[]
\tikz \node[fill=yellow!80!black,align=right]
  {This is a\\[-2pt] demonstration text for\\[1ex] alignments.};
\end{codeexample}
        %
    \item You can request that \tikzname\ does an automatic line-breaking for
        you inside the node by specifying a fixed |text width| for the node. In
        this case, you can still use |\\| to enforce a line-break. Note that
        when you specify a text width, the node will have this width,
        independently of whether the text actually ``reaches the end'' of the
        node.
\end{enumerate}

Let us now first have a look at the |text width| command.
    %
\begin{key}{/tikz/text width=\meta{dimension}}
    This option will put the text of a node in a box of the given width
    (something akin to a |{minipage}| of this width, only portable across
    formats). If the node text is not as wide as \meta{dimension}, it will
    nevertheless be put in a box of this width. If it is larger, line breaking
    will be done.

    By default, when this option is given, a ragged right border will be used
    (|align=left|). This is sensible since, typically, these boxes are narrow
    and justifying the text looks ugly. You can, however, change the alignment
    using |align| or directly using commands line |\centering|.
    %
\begin{codeexample}[]
\tikz \draw (0,0) node[fill=yellow!80!black,text width=3cm]
  {This is a demonstration text for showing how line breaking works.};
\end{codeexample}
    %
    Setting \meta{dimension} to an empty string causes the automatic line
    breaking to be disabled.
\end{key}

\begin{key}{/tikz/align=\meta{alignment option}}
    This key is used to set up an alignment for multi-line text inside a node.
    If |text width| is set to some width (let us call this \emph{alignment with
    line breaking}), the |align| key will setup the |\leftskip| and the
    |\rightskip| in such a way that the text is broken and aligned according to
    \meta{alignment option}. If |text width| is not set (that is, set to the
    empty string; let us call this \emph{alignment without line breaking}),
    then a different mechanism is used internally, namely the key
    |node halign header|, is set to an appropriate value. While this key, which
    is documented below, is not to be used by beginners, the net effect is
    simple: When |text width| is not set, you can use |\\| to break lines and
    align them according to \meta{alignment option} and the resulting node's
    width will be minimal to encompass the resulting lines.

    In detail, you can set \meta{alignment option} to one of the following values:
    %
    \begin{description}
        \item[|align=|\declare{|left|}] For alignment without line breaking,
            the different lines are simply aligned such that their left borders
            are below one another.
            %
\begin{codeexample}[]
\tikz \node[fill=yellow!80!black,align=left]
  {This is a\\ demonstration text for\\ alignments.};
\end{codeexample}
            %
            For alignment with line breaking, the same will happen; only the
            lines will now, additionally, be broken automatically:
            %
\begin{codeexample}[]
\tikz \node[fill=yellow!80!black,text width=3cm,align=left]
  {This is a demonstration text for showing how line breaking works.};
\end{codeexample}
            %
        \item[|align=|\declare{\texttt{flush left}}] For alignment without line
            breaking this option has exactly the same effect as |left|.
            However, for alignment with line breaking, there is a difference:
            While |left| uses the original plain \TeX\ definition of a ragged
            right border, in which \TeX\ will try to balance the right border
            as well as possible, |flush left| causes the right border to be
            ragged in the \LaTeX-style, in which no balancing occurs. This
            looks ugly, but it may be useful for very narrow boxes and when you
            wish to avoid hyphenations.
            %
\begin{codeexample}[]
\tikz \node[fill=yellow!80!black,text width=3cm,align=flush left]
  {This is a demonstration text for showing how line breaking works.};
\end{codeexample}
            %
        \item[|align=|\declare{|right|}] Works like |left|, only for right
            alignment.
            %
\begin{codeexample}[]
\tikz \node[fill=yellow!80!black,align=right]
  {This is a\\ demonstration text for\\ alignments.};
\end{codeexample}
            %
\begin{codeexample}[]
\tikz \node[fill=yellow!80!black,text width=3cm,align=right]
  {This is a demonstration text for showing how line breaking works.};
\end{codeexample}
            %
        \item[|align=|\declare{\texttt{flush right}}] Works like |flush left|,
            only for right alignment.
            %
\begin{codeexample}[]
\tikz \node[fill=yellow!80!black,text width=3cm,align=flush right]
  {This is a demonstration text for showing how line breaking works.};
\end{codeexample}
            %
        \item[|align=|\declare{|center|}] Works like |left| or |right|, only
            for centered alignment.
            %
\begin{codeexample}[]
\tikz \node[fill=yellow!80!black,align=center]
  {This is a\\ demonstration text for\\ alignments.};
\end{codeexample}
\begin{codeexample}[]
\tikz \node[fill=yellow!80!black,text width=3cm,align=center]
  {This is a demonstration text for showing how line breaking works.};
\end{codeexample}

            There is one annoying problem with the |center| alignment (but not
            with |flush center| and the other options): If you specify a large
            line width and the node text fits on a single line and is, in fact,
            much shorter than the specified |text width|, an underfull
            horizontal box will result. Unfortunately, this cannot be avoided,
            due to the way \TeX\ works (more precisely, I have thought long and
            hard about this and have not been able to figure out a sensible way
            to avoid this). For this reason, \tikzname\ switches off horizontal
            badness warnings inside boxes with |align=center|. Since this will
            also suppress some ``wanted'' warnings, there is also an option for
            switching the warnings on once more:
            %
            \begin{key}{/tikz/badness warnings for centered text=\meta{true or false} (initially false)}
                If set to true, normal badness warnings will be issued for
                centered boxes. Note that you may get annoying warnings for
                perfectly normal boxes, namely whenever the box is very large
                and the contents is not long enough to fill the box
                sufficiently.
            \end{key}
        \item[|align=|\declare{\texttt{flush center}}] Works like |flush left|
            or |flush right|, only for center alignment. Because of all the
            trouble that results from the |center| option in conjunction with
            narrow lines, I suggest picking this option rather than  |center|
            \emph{unless} you have longer text, in which case |center| will
            give the typographically better results.
            %
\begin{codeexample}[]
\tikz \node[fill=yellow!80!black,text width=3cm,align=flush center]
  {This is a demonstration text for showing how line breaking works.};
\end{codeexample}
            %
        \item[|align=|\declare{|justify|}] For alignment without line breaking,
            this has the same effect as |left|. For alignment with line
            breaking, this causes the text to be ``justified''. Use this only
            with rather broad nodes.
{%
\hbadness=10000
\begin{codeexample}[]
\tikz \node[fill=yellow!80!black,text width=3cm,align=justify]
  {This is a demonstration text for showing how line breaking works.};
\end{codeexample}
}
            In the above example, \TeX\ complains (rightfully) about three very
            badly typeset lines. (For this manual I asked \TeX\ to stop
            complaining by using |\hbadness=10000|, but this is a foul deed,
            indeed.)
        \item[|align=|\declare{|none|}] Disables all alignments and |\\| will
            not be redefined.
    \end{description}
\end{key}

\begin{key}{/tikz/node halign header=\meta{macro storing a header} (initially \normalfont empty)}
    This is the key that is used by |align| internally for alignment without
    line breaking. Read the following only if you are familiar with the
    |\halign| command.

    This key only has an effect if |text width| is empty, otherwise it is
    ignored. Furthermore, if \meta{macro storing a header} is empty, then this
    key also has no effect. So, suppose |text width| is empty, but
    \meta{header} is not. In this case the following happens:

    When the node text is parsed, the command |\\| is redefined internally.
    This redefinition is done in such a way that the text from the start of the
    node to the first occurrence of |\\| is put in an |\hbox|. Then the text
    following |\\| up to the next |\\| is put in another |\hbox|. This goes on
    until the text between the last |\\| and the closing |}| is also put in an
    |\hbox|.

    The \meta{macro storing a header} should be a macro that contains some text
    suitable for use as a header for the |\halign| command. For instance, you
    might define
    %
\begin{codeexample}[code only]
\def\myheader{\hfil\hfil##\hfil\cr}
\tikz [node halign header=\myheader] ...
\end{codeexample}
    %
    You cannot just say |node halign header=\hfil\hfil#\hfil\cr| because this
    confuses \TeX\ inside matrices, so this detour via a macro is needed.

    Next, conceptually, all these boxes are recursively put inside an |\halign|
    command. Assuming that \meta{first} is the first of the above boxes, the
    command |\halign{|\meta{header} |\box|\meta{first} |\cr}| is used to create
    a new box, which we will call the \meta{previous box}. Then, the following
    box is created, where \meta{second} is the second input box:
    |\halign{|\meta{header} |\box|\meta{previous box} |\cr|
    |\box|\meta{second}|\cr}|. Let us call the resulting box the \meta{previous
    box} once more. Then the next box that is created is
    |\halign{|\meta{header} |\box|\meta{previous box} |\cr|
    |\box|\meta{third}|\cr}|.

    All of this means that if \meta{header} is an |\halign| header like
    |\hfil#\hfil\cr|, then all boxes will be centered relative to one another.
    Similarly, a \meta{header} of |\hfil#\cr| causes the text to be flushed
    right.

    Note that this mechanism is not flexible enough to all multiple columns
    inside \meta{header}. You will have to use a |tabular| or a |matrix| in
    such cases.

    One further note: Since the text of each line is placed in a box, settings
    will be local to each ``line''. This is very similar to the way a cell in a
    |tabular| or a |matrix| behaves.
\end{key}


\subsubsection{Text Parameters: Height and Depth of Text}

In addition to changing the width of nodes, you can also change the height of
nodes. This can be done in two ways: First, you can use the option
|minimum height|, which ensures that the height of the whole node is at least
the given height (this option is described in more detail later). Second, you
can use the option |text height|, which sets the height of the text itself,
more precisely, of the \TeX\ text box of the text. Note that the |text height|
typically is not the height of the shape's box: In addition to the
|text height|, an internal |inner sep| is added as extra space and the text
depth is also taken into account.

I recommend using |minimum size| instead of |text height| except for special
situations.

\begin{key}{/tikz/text height=\meta{dimension}}
    Sets the height of the text boxes in shapes. Thus, when you write something
    like |node {text}|, the |text| is first typeset, resulting in some box of a
    certain height. This height is then replaced by the height |text height|.
    The resulting box is then used to determine the size of the shape, which
    will typically be larger. When you write |text height=| without specifying
    anything, the ``natural'' size of the text box remains unchanged.
    %
\begin{codeexample}[]
\tikz \node[draw]                  {y};
\tikz \node[draw,text height=10pt] {y};
\end{codeexample}
    %
\end{key}

\begin{key}{/tikz/text depth=\meta{dimension}}
    This option works like |text height|, only for the depth of the text box.
    This option is mostly useful when you need to ensure a uniform depth of
    text boxes that need to be aligned.
\end{key}


\subsection{Positioning Nodes}
\label{section-nodes-anchors}

When you place a node at some coordinate, the node is centered on this
coordinate by default. This is often undesirable and it would be better to have
the node to the right or above the actual coordinate.


\subsubsection{Positioning Nodes Using Anchors}

\pgfname\ uses a so-called anchoring mechanism to give you a very fine control
over the placement. The idea is simple: Imagine a node of rectangular shape of
a certain size. \pgfname\ defines numerous anchor positions in the shape. For
example to upper right corner is called, well, not ``upper right anchor'', but
the |north east| anchor of the shape. The center of the shape has an anchor
called |center| on top of it, and so on. Here are some examples (a complete
list is given in Section~\ref{section-the-shapes}).

\medskip\noindent
\begin{tikzpicture}
  \path node[minimum height=2cm,minimum width=5cm,fill=blue!25](x) {Big node};
  \fill (x.north)      circle (2pt) node[above] {|north|}
        (x.north east) circle (2pt) node[above] {|north east|}
        (x.north west) circle (2pt) node[above] {|north west|}
        (x.west) circle (2pt)       node[left]  {|west|}
        (x.east) circle (2pt)       node[right] {|east|}
        (x.base) circle (2pt)       node[below] {|base|};
\end{tikzpicture}

Now, when you place a node at a certain coordinate, you can ask \tikzname\ to
place the node shifted around in such a way that a certain anchor is at the
coordinate. In the following example, we ask \tikzname\ to shift the first node
such that its  |north east| anchor is at coordinate |(0,0)| and that the |west|
anchor of the second node is at coordinate |(1,1)|.
%
\begin{codeexample}[]
\tikz \draw           (0,0) node[anchor=north east] {first node}
            rectangle (1,1) node[anchor=west] {second node};
\end{codeexample}

Since the default anchor is |center|, the default behavior is to shift the node
in such a way that it is centered on the current position.

\begin{key}{/tikz/anchor=\meta{anchor name}}
    Causes the node to be shifted such that its anchor \meta{anchor name} lies
    on the current coordinate.

    The only anchor that is present in all shapes is |center|. However, most
    shapes will at least define anchors in all ``compass directions''.
    Furthermore, the standard shapes also define a |base| anchor, as well as
    |base west| and |base east|, for placing things on the baseline of the
    text.

    The standard shapes also define a |mid| anchor (and |mid west| and
    |mid east|). This anchor is half the height of the character ``x'' above
    the base line. This anchor is useful for vertically centering multiple
    nodes that have different heights and depth. Here is an example:
    %
\begin{codeexample}[]
\begin{tikzpicture}[scale=3,transform shape]
  % First, center alignment -> wobbles
  \draw[anchor=center] (0,1)  node{x} -- (0.5,1)  node{y} -- (1,1)  node{t};
  % Second, base alignment -> no wobble, but too high
  \draw[anchor=base]   (0,.5) node{x} -- (0.5,.5) node{y} -- (1,.5) node{t};
  % Third, mid alignment
  \draw[anchor=mid]    (0,0)  node{x} -- (0.5,0)  node{y} -- (1,0)  node{t};
\end{tikzpicture}
\end{codeexample}
    %
\end{key}


\subsubsection{Basic Placement Options}

Unfortunately, while perfectly logical, it is often rather counter-intuitive
that in order to place a node \emph{above} a given point, you need to specify
the |south| anchor. For this reason, there are some useful options that allow
you to select the standard anchors more intuitively:

\begin{key}{/tikz/above=\meta{offset} (default 0pt)}
    Does the same as |anchor=south|. If the \meta{offset} is specified, the
    node is additionally shifted upwards by the given \meta{offset}.
    %
\begin{codeexample}[]
\tikz \fill (0,0) circle (2pt) node[above] {above};
\end{codeexample}
    %
\begin{codeexample}[]
\tikz \fill (0,0) circle (2pt) node[above=2pt] {above};
\end{codeexample}
    %
\end{key}

\begin{key}{/tikz/below=\meta{offset} (default 0pt)}
    Similar to |above|.
\end{key}

\begin{key}{/tikz/left=\meta{offset} (default 0pt)}
    Similar to |above|.
\end{key}

\begin{key}{/tikz/right=\meta{offset} (default 0pt)}
    Similar to |above|.
\end{key}

\begin{key}{/tikz/above left}
    Does the same as |anchor=south east|. Note that giving both |above| and
    |left| options does not have the same effect as |above left|, rather only
    the last |left| ``wins''. Actually, this option also takes an \meta{offset}
    parameter, but using this parameter without using the |positioning| library
    is deprecated. (The |positioning| library changes the meaning of this
    parameter to something more sensible.)
    %
\begin{codeexample}[]
\tikz \fill (0,0) circle (2pt) node[above left] {above left};
\end{codeexample}
    %
\end{key}

\begin{key}{/tikz/above right}
    Similar to  |above left|.
    %
\begin{codeexample}[]
\tikz \fill (0,0) circle (2pt) node[above right] {above right};
\end{codeexample}
    %
\end{key}

\begin{key}{/tikz/below left}
    Similar to |above left|.
\end{key}

\begin{key}{/tikz/below right}
    Similar to |above left|.
\end{key}

\begin{key}{/tikz/centered}
    A shorthand for |anchor=center|.
\end{key}

% A second set of options behaves similarly, namely the |above of|,
% |below of|, and so on options. They cause the same anchors to be set
% as the options without |of|, however, their parameter is different:
% You must provide the name of another node. The current node will then
% be placed, say, above this specified node at a distance given by the
% option |node distance|.
% \begin{key}{/tikz/above of=\meta{node}}
%   This option causes the node to be placed at the distance
%   |node distance| above of \meta{node}. The anchor is |center|.
% \begin{codeexample}[]
% \begin{tikzpicture}[node distance=1cm]
%   \draw[help lines] (0,0) grid (3,2);
%   \node (a)                    {a};
%   \node (b) [above of=a]       {b};
%   \node (c) [above of=b]       {c};
%   \node (d) [right of=c]       {d};
%   \node (e) [below right of=d] {e};
% \end{tikzpicture}
% \end{codeexample}
% \end{key}
%
% \begin{key}{/tikz/above left of=\meta{node}}
%   Works like |above of|, only the node is now put above and left. The
%   |node distance| is the Euclidean distance between the two nodes, not
%   the $L_1$-distance.
% \end{key}
%
% \begin{key}{/tikz/above right of=\meta{node}}
%   Works similarly.
% \end{key}
% \begin{key}{/tikz/left of=\meta{node}}
%   Works similarly.
% \end{key}
% \begin{key}{/tikz/right of=\meta{node}}
%   Works similarly.
% \end{key}
% \begin{key}{/tikz/below of=\meta{node}}
%   Works similarly.
% \end{key}
% \begin{key}{/tikz/below left of=\meta{node}}
%   Works similarly.
% \end{key}
% \begin{key}{/tikz/below right of=\meta{node}}
%   Works similarly.
% \end{key}
% \begin{key}{/tikz/node distance=\meta{dimension}}
%   Sets the distance between nodes that are placed using the
%   |... of| options. Note that this distance is the distance between
%   the centers of the nodes, not the distance between their borders.
% \end{key}


\subsubsection{Advanced Placement Options}

While the standard placement options suffice for simple cases, the
|positioning| library offers more convenient placement options.

\begin{tikzlibrary}{positioning}
    The library defines additional options for placing nodes conveniently. It
    also redefines the standard options like |above| so that they give you
    better control of node placement.
\end{tikzlibrary}

When this library is loaded, the options like |above| or |above left| behave
differently.

\begin{key}{/tikz/above=\opt{\meta{specification}} (default 0pt)}
    With the |positioning| library loaded, the |above| option does not take a
    simple \meta{dimension} as its parameter. Rather, it can (also) take a more
    elaborate \meta{specification} as parameter. This \meta{specification} has
    the following general form: It starts with an optional \meta{shifting part}
    and is followed by an optional \meta{of-part}. Let us start with the
    \meta{shifting part}, which can have three forms:
    %
    \begin{enumerate}
        \item It can simply be a \declare{\meta{dimension}} (or a mathematical
            expression that evaluates to a dimension) like |2cm| or
            |3cm/2+4cm|. In this case, the following happens: the node's anchor
            is set to |south| and the node is vertically shifted upwards by
            \meta{dimension}.
            %
\begin{codeexample}[]
\begin{tikzpicture}
  \draw[help lines] (0,0) grid (2,2);
  \node at (1,1) [above=2pt+3pt,draw] {above};
\end{tikzpicture}
\end{codeexample}
            %
            This use of the |above| option is the same as if the |positioning|
            library were not loaded.
        \item It can be a \declare{\meta{number}} (that is, any mathematical
            expression that does not include a unit like |pt| or |cm|).
            Examples are |2| or |3+sin(60)|. In this case, the anchor is also
            set to |south| and the node is vertically shifted by the vertical
            component of the coordinate |(0,|\meta{number}|)|.
            %
\begin{codeexample}[]
\begin{tikzpicture}
  \draw[help lines] (0,0) grid (2,2);
  \node at (1,1) [above=.2,draw] {above};
  % south border of the node is now 2mm above (1,1)
\end{tikzpicture}
\end{codeexample}
            %
        \item It can be of the form
            \declare{\meta{number or dimension 1}| and |\meta{number or dimension 2}}.
            This specification does not make particular sense for the |above|
            option, it is much more useful for options like |above left|. The
            reason it is allowed for the |above| option is that it is sometimes
            automatically used, as explained later.

            The effect of this option is the following. First, the point
            |(|\meta{number or dimension 2}|,|\meta{number or dimension 1}|)|
            is computed (note the inverted order), using the normal rules for
            evaluating such a coordinate, yielding some position. Then, the
            node is shifted by the vertical component of this point. The anchor
            is set to |south|.
            %
\begin{codeexample}[preamble={\usetikzlibrary{positioning}}]
\begin{tikzpicture}
  \draw[help lines] (0,0) grid (2,2);
  \node at (1,1) [above=.2 and 3mm,draw] {above};
  % south border of the node is also 2mm above (1,1)
\end{tikzpicture}
\end{codeexample}
    \end{enumerate}
    %
    The \meta{shifting part} can optionally be followed by a \meta{of-part},
    which has one of the following forms:
    %
    \begin{enumerate}
        \item The \meta{of-part} can be
            \declareandlabel{of}| |\meta{coordinate}, where \meta{coordinate} is
            \emph{not} in parentheses and it is \emph{not} just a node name. An
            example would be |of somenode.north| or |of {2,3}|. In this case, the
            following happens: First, the node's |at| parameter is set to the
            \meta{coordinate}. Second, the node is shifted according to the
            \meta{shift-part}. Third, the anchor is set to |south|.

            Here is a basic example:
            %
\begin{codeexample}[preamble={\usetikzlibrary{positioning}}]
\begin{tikzpicture}[every node/.style=draw]
  \draw[help lines] (0,0) grid (2,2);
  \node (somenode) at (1,1) {some node};

  \node [above=5mm of somenode.north east] {\tiny 5mm of somenode.north east};
  \node [above=1cm of somenode.north]      {\tiny 1cm of somenode.north};
\end{tikzpicture}
\end{codeexample}
            %
            As can be seen the |above=5mm of somenode.north east| option does,
            indeed, place the node 5mm above the north east anchor of
            |somenode|. The same effect could have been achieved writing
            |above=5mm| followed by |at=(somenode.north east)|.

            If the \meta{shifting-part} is missing, the shift is not zero, but
            rather the value of the |node distance| key is used, see below.
        \item The \meta{of-part} can be |of |\meta{node name}. An example would
            be |of somenode|. In this case, the following usually happens:
            %
            \begin{itemize}
                \item The anchor is set to |south|.
                \item The node is shifted according to the \meta{shifting part}
                    or, if it is missing, according to the value of
                    |node distance|.
                \item The node's |at| parameter is set to \meta{node
                    name}|.north|.
            \end{itemize}
            %
            The net effect of all this is that the new node will be placed in
            such a way that the distance between its south border and \meta{node
            name}'s north border is exactly the given distance.
            %
\begin{codeexample}[preamble={\usetikzlibrary{positioning}}]
\begin{tikzpicture}[every node/.style=draw]
  \draw[help lines] (0,0) grid (2,2);
  \node (some node) at (1,1) {some node};

  \node (other node) [above=1cm of some node] {\tiny above=1cm of some node};

  \draw [<->] (some node.north) -- (other node.south)
                                node [midway,right,draw=none] {1cm};
\end{tikzpicture}
\end{codeexample}
            %
            It is possible to change the behavior of this \meta{specification}
            rather drastically, using the following key:
            %
            \begin{key}{/tikz/on grid=\meta{boolean} (initially false)}
                When this key is set to |true|, an \meta{of-part} of the
                current form behaves differently: The anchors set for the
                current node as well as the anchor used for the other
                \meta{node name} are set to |center|.

                This has the following effect: When you say
                |above=1cm of somenode| with |on grid| set to true, the new
                node will be placed in such a way that its center is 1cm above
                the center of |somenode|. Repeatedly placing nodes in this way
                will result in nodes that are centered on ``grid coordinate'',
                hence the name of the option.
                %
\begin{codeexample}[preamble={\usetikzlibrary{positioning}}]
\begin{tikzpicture}[every node/.style=draw]
  \draw[help lines] (0,0) grid (2,3);

  % Not gridded
  \node (a1) at (0,0) {not gridded};
  \node (b1) [above=1cm of a1] {fooy};
  \node (c1) [above=1cm of b1] {a};

  % gridded
  \node (a2) at (2,0) {gridded};
  \node (b2) [on grid,above=1cm of a2] {fooy};
  \node (c2) [on grid,above=1cm of b2] {a};
\end{tikzpicture}
\end{codeexample}
            \end{key}
    \end{enumerate}

    \begin{key}{/tikz/node distance=\meta{shifting part} (initially 1cm and 1cm)}
        The value of this key is used as \meta{shifting part} is used if and
        only if a \meta{of-part} is present, but no \meta{shifting part}.
        %
\begin{codeexample}[preamble={\usetikzlibrary{positioning}}]
\begin{tikzpicture}[every node/.style=draw,node distance=5mm]
  \draw[help lines] (0,0) grid (2,3);

  % Not gridded
  \node (a1) at (0,0) {not gridded};
  \node (b1) [above=of a1] {fooy};
  \node (c1) [above=of b1] {a};

  % gridded
  \begin{scope}[on grid]
    \node (a2) at (2,0) {gridded};
    \node (b2) [above=of a2] {fooy};
    \node (c2) [above=of b2] {a};
  \end{scope}
\end{tikzpicture}
\end{codeexample}
    \end{key}
\end{key}

\begin{key}{/tikz/below=\opt{\meta{specification}}}
    This key is redefined in the same manner as |above|.
\end{key}

\begin{key}{/tikz/left=\opt{\meta{specification}}}
    This key is redefined in the same manner as |above|, only all vertical
    shifts are replaced by horizontal shifts.
\end{key}

\begin{key}{/tikz/right=\opt{\meta{specification}}}
    This key is redefined in the same manner as |left|.
\end{key}

\begin{key}{/tikz/above left=\opt{\meta{specification}}}
    This key is also redefined in a manner similar to the above, but behavior
    of the \meta{shifting part} is more complicated:
    %
    \begin{enumerate}
        \item When the \meta{shifting part} is of the form
            \meta{number or dimension}| and |\meta{number or dimension}, it has
            (essentially) the effect of shifting the node vertically upwards by
            the first \meta{number or dimension} and to the left by the second.
            To be more precise, the coordinate |(|\meta{second number or
            dimension}|,|\meta{first number or dimension}|)| is computed and
            then the node is shifted vertically by the $y$-part of the
            resulting coordinate and horizontally be the negated $x$-part of
            the result. (This is exactly what you expect, except possibly when
            you have used the |x| and |y| options to modify the |xy|-coordinate
            system so that the unit  vectors no longer point in the expected
            directions.)
        \item When the \meta{shifting part} is of the form \meta{number or
            dimension}, the node is shifted by this \meta{number or dimension}
            in the direction of $135^\circ$. This means that there is a
            difference between a \meta{shifting part} of |1cm| and of
            |1cm and 1cm|: In the second case, the node is shifted by 1cm
            upward and 1cm to the left; in the first case it is shifted by
            $\frac{1}{2}\sqrt{2}$cm upward and by the same amount to the left.
            A more mathematical way of phrasing this is the following: A plain
            \meta{dimension} is measured in the $l_2$-norm, while a
            \meta{dimension}| and |\meta{dimension} is measured in the
            $l_1$-norm.
    \end{enumerate}
    %
    The following example should help to illustrate the difference:
    %
\begin{codeexample}[preamble={\usetikzlibrary{positioning}}]
\begin{tikzpicture}[every node/.style={draw,circle}]
  \draw[help lines] (0,0) grid (2,5);
  \begin{scope}[node distance=5mm and 5mm]
    \node (b) at (1,4) {b};
    \node [left=of b] {1};       \node [right=of b] {2};
    \node [above=of b] {3};      \node [below=of b] {4};
    \node [above left=of b] {5}; \node [above right=of b] {6};
    \node [below left=of b] {7}; \node [below right=of b] {8};
  \end{scope}
  \begin{scope}[node distance=5mm]
    \node (a) at (1,1) {a};
    \node [left=of a] {1};       \node [right=of a] {2};
    \node [above=of a] {3};      \node [below=of a] {4};
    \node [above left=of a] {5}; \node [above right=of a] {6};
    \node [below left=of a] {7}; \node [below right=of a] {8};
  \end{scope}
\end{tikzpicture}
\end{codeexample}
    %
\begin{codeexample}[preamble={\usetikzlibrary{positioning}}]
\begin{tikzpicture}[every node/.style={draw,rectangle}]
  \draw[help lines] (0,0) grid (2,5);
  \begin{scope}[node distance=5mm and 5mm]
    \node (b) at (1,4) {b};
    \node [left=of b] {1};       \node [right=of b] {2};
    \node [above=of b] {3};      \node [below=of b] {4};
    \node [above left=of b] {5}; \node [above right=of b] {6};
    \node [below left=of b] {7}; \node [below right=of b] {8};
  \end{scope}
  \begin{scope}[node distance=5mm]
    \node (a) at (1,1) {a};
    \node [left=of a] {1};       \node [right=of a] {2};
    \node [above=of a] {3};      \node [below=of a] {4};
    \node [above left=of a] {5}; \node [above right=of a] {6};
    \node [below left=of a] {7}; \node [below right=of a] {8};
  \end{scope}
\end{tikzpicture}
\end{codeexample}
    %
\begin{codeexample}[preamble={\usetikzlibrary{positioning}}]
\begin{tikzpicture}[every node/.style={draw,rectangle},on grid]
  \draw[help lines] (0,0) grid (4,4);
  \begin{scope}[node distance=1]
    \node (a) at (2,3) {a};
    \node [left=of a] {1};       \node [right=of a] {2};
    \node [above=of a] {3};      \node [below=of a] {4};
    \node [above left=of a] {5}; \node [above right=of a] {6};
    \node [below left=of a] {7}; \node [below right=of a] {8};
  \end{scope}
  \begin{scope}[node distance=1 and 1]
    \node (b) at (2,0) {b};
    \node [left=of b] {1};       \node [right=of b] {2};
    \node [above=of b] {3};      \node [below=of b] {4};
    \node [above left=of b] {5}; \node [above right=of b] {6};
    \node [below left=of b] {7}; \node [below right=of b] {8};
  \end{scope}
\end{tikzpicture}
\end{codeexample}
    %
\end{key}

\begin{key}{/tikz/below left=\opt{\meta{specification}}}
    Works similar to |above left|.
\end{key}

\begin{key}{/tikz/above right=\opt{\meta{specification}}}
    Works similar to |above left|.
\end{key}

\begin{key}{/tikz/below right=\opt{\meta{specification}}}
    Works similar to |above left|.
\end{key}

The |positioning| package also introduces the following new placement keys:
%
\begin{key}{/tikz/base left=\opt{\meta{specification}}}
    This key works like the |left| key, only instead of the |east| anchor, the
    |base east| anchor is used and, when the second form of an \meta{of-part}
    is used, the corresponding |base west| anchor.

    This key is useful for chaining together nodes so that their base lines are
    aligned.
    %
\begin{codeexample}[preamble={\usetikzlibrary{positioning}}]
\begin{tikzpicture}[node distance=1ex]
  \draw[help lines] (0,0) grid (3,1);
  \huge
  \node (X) at (0,1)     {X};
  \node (a) [right=of X] {a};
  \node (y) [right=of a] {y};

  \node (X) at (0,0)          {X};
  \node (a) [base right=of X] {a};
  \node (y) [base right=of a] {y};
\end{tikzpicture}
\end{codeexample}
    %
\end{key}

\begin{key}{/tikz/base right=\opt{\meta{specification}}}
    Works like |base left|.
\end{key}
\begin{key}{/tikz/mid left=\opt{\meta{specification}}}
    Works like |base left|, but with |mid east| and |mid west| anchors instead
    of |base east| and |base west|.
\end{key}

\begin{key}{/tikz/mid right=\opt{\meta{specification}}}
    Works like |mid left|.
\end{key}


\subsubsection{Advanced Arrangements of Nodes}

The simple |above| and |right| options may not always suffice for arranging a
large number of nodes. For such situations \tikzname\ offers libraries that
make positioning easier: The |matrix| library and the |graphdrawing| library.
These libraries for positioning nodes are described in two separate
Sections~\ref{section-matrices} and~\ref{section-intro-gd}.


\subsection{Fitting Nodes to a Set of Coordinates}
\label{section-nodes-fitting}

It is sometimes desirable that the size and position of a node is not given
using anchors and size parameters, rather one would sometimes have a box be
placed and be sized such that it ``is just large enough to contain this, that,
and that point''. This situation typically arises when a picture has been drawn
and, afterwards, parts of the picture are supposed to be encircled or
highlighted.

In this situation the |fit| option from the |fit| library is useful, see
Section~\ref{section-library-fit} for the details. The idea is that you may
give the |fit| option to a node. The |fit| option expects a list of coordinates
(one after the other without commas) as its parameter. The effect will be that
the node's text area has exactly the necessary size so that it contains all the
given coordinates. Here is an example:
%
\begin{codeexample}[preamble={\usetikzlibrary{fit,shapes.geometric}}]
\begin{tikzpicture}[level distance=8mm]
  \node (root) {root}
    child { node (a) {a} }
    child { node (b) {b}
      child { node (d) {d} }
      child { node (e) {e} } }
    child { node (c) {c} };

  \node[draw=red,inner sep=0pt,thick,ellipse,fit=(root) (b) (d) (e)] {};
  \node[draw=blue,inner sep=0pt,thick,ellipse,fit=(b) (c) (e)] {};
\end{tikzpicture}
\end{codeexample}

If you want to fill the fitted node you will usually have to place it on a
background layer.
%
\begin{codeexample}[preamble={\usetikzlibrary{backgrounds,fit,shapes.geometric}}]
\begin{tikzpicture}[level distance=8mm]
  \node (root) {root}
    child { node (a) {a} }
    child { node (b) {b}
      child { node (d) {d} }
      child { node (e) {e} } }
    child { node (c) {c} };

  \begin{scope}[on background layer]
    \node[fill=red!20,inner sep=0pt,ellipse,fit=(root) (b) (d) (e)] {};
    \node[fill=blue!20,inner sep=0pt,ellipse,fit=(b) (c) (e)] {};
  \end{scope}
\end{tikzpicture}
\end{codeexample}


\subsection{Transformations}
\label{section-nodes-transformations}

It is possible to transform nodes, but, by default, transformations do not
apply to nodes. The reason is that you usually do \emph{not} want your text to
be scaled or rotated even if the main graphic is transformed. Scaling text is
evil, rotating slightly less so.

However, sometimes you \emph{do} wish to transform a node, for example, it
certainly sometimes makes sense to rotate a node by 90 degrees. There are two
ways to achieve this:
%
\begin{enumerate}
    \item You can use the following option:
        %
        \begin{key}{/tikz/transform shape}
            Causes the current ``external'' transformation matrix to be applied
            to the shape. For example, if you said |\tikz[scale=3]| and then
            say |node[transform shape] {X}|, you will get a ``huge'' X in your
            graphic.
        \end{key}
    \item You can give transformation options \emph{inside} the option list of
        the node. \emph{These} transformations always apply to the node.
        %
\begin{codeexample}[
    preamble={\usepgfmodule{nonlineartransformations}\usetikzlibrary{curvilinear}},
    pre={\makeatletter},
]
\begin{tikzpicture}[every node/.style={draw}]
  \draw[help lines](0,0) grid (3,2);
  \draw            (1,0) node{A}
                   (2,0) node[rotate=90,scale=1.5] {B};
  \draw[rotate=30] (1,0) node{A}
                   (2,0) node[rotate=90,scale=1.5] {B};
  \draw[rotate=60] (1,0) node[transform shape] {A}
                   (2,0) node[transform shape,rotate=90,scale=1.5] {B};
\end{tikzpicture}
\end{codeexample}
        %
\end{enumerate}

Even though \tikzname\ currently does not allow you to configure so-called
\emph{nonlinear transformations,} see
Section~\ref{section-nonlinear-transformations}, there is an option that
influences how nodes are transformed when nonlinear transformations are in
force:
%
\begin{key}{/tikz/transform shape nonlinear=\opt{\meta{true or false}}  (initially false)}
    When set to true, \tikzname\ will try to apply any current nonlinear
    transformation also to nodes. Typically, for the text in nodes this is not
    possible in general, in such cases a linear approximation of the nonlinear
    transformation is used. For more details, see
    Section~\ref{section-nonlinear-transformations}.
    %
\makeatletter
\begin{codeexample}[
    preamble={\usepgfmodule{nonlineartransformations}\usetikzlibrary{curvilinear}},
    pre={\makeatletter},
]
\begin{tikzpicture}
   % Install a nonlinear transformation:
   \pgfsetcurvilinearbeziercurve
      {\pgfpoint{0mm}{20mm}}
      {\pgfpoint{10mm}{20mm}}
      {\pgfpoint{10mm}{10mm}}
      {\pgfpoint{20mm}{10mm}}
   \pgftransformnonlinear{\pgfpointcurvilinearbezierorthogonal\pgf@x\pgf@y}%

   % Draw something:
   \draw [help lines] (0,-30pt) grid [step=10pt] (80pt,30pt);

   \foreach \x in {0,20,...,80}
     \node [fill=red!20]  at (\x pt, -20pt) {\x};

   \foreach \x in {0,20,...,80}
     \node [fill=blue!20, transform shape nonlinear] at (\x pt, 20pt) {\x};
\end{tikzpicture}
\end{codeexample}
    %
\end{key}


\subsection{Placing Nodes on a Line or Curve Explicitly}
\label{section-nodes-placing-1}

Until now, we always placed nodes on a coordinate that is mentioned in the path.
Often, however, we wish to place nodes on ``the middle'' of a line and we do
not wish to compute these coordinates ``by hand''. To facilitate such
placements, \tikzname\ allows you to specify that a certain node should be
somewhere ``on'' a line. There are two ways of specifying this: Either
explicitly by using the |pos| option or implicitly by placing the node
``inside'' a path operation. These two ways are described in the following.

    \label{section-pos-option}

\begin{key}{/tikz/pos=\meta{fraction}}
    When this option is given, the node is not anchored on the last coordinate.
    Rather, it is anchored on some point on the line from the previous
    coordinate to the current point. The \meta{fraction} dictates how ``far''
    on the line the point should be. A \meta{fraction} of 0 is the previous
    coordinate, 1 is the current one, everything else is in between. In
    particular, 0.5 is the middle.

    Now, what is ``the previous line''? This depends on the previous path
    construction operation.

    In the simplest case, the previous path operation was a ``line-to''
    operation, that is, a |--|\meta{coordinate} operation:
    %
\begin{codeexample}[]
\tikz \draw (0,0) -- (3,1)
    node[pos=0]{0} node[pos=0.5]{1/2} node[pos=0.9]{9/10};
\end{codeexample}

    For the |arc| operation, the position is simply the corresponding position
    on the arc:
    %
\begin{codeexample}[]
\tikz {
  \draw [help lines] (0,0) grid (3,2);
  \draw (2,0) arc [x radius=1, y radius=2, start angle=0, end angle=180]
              node foreach \t in {0,0.125,...,1} [pos=\t,auto] {\t};
}
\end{codeexample}

    The next case is the curve-to operation (the |..| operation). In this case,
    the ``middle'' of the curve, that is, the position |0.5| is not necessarily
    the point at the exact half distance on the line. Rather, it is some point
    at ``time'' 0.5 of a point traveling from the start of the curve, where it
    is at time 0, to the end of the curve, which it reaches at time 0.5. The
    ``speed'' of the point depends on the length of the support vectors (the
    vectors that connect the start and end points to the control points). The
    exact math is a bit complicated (depending on your point of view, of
    course); you may wish to consult a good book on computer graphics and
    Bézier curves if you are intrigued.
    %
\begin{codeexample}[]
\tikz \draw (0,0) .. controls +(right:3.5cm) and +(right:3.5cm) .. (0,3)
  node foreach \p in {0,0.125,...,1} [pos=\p]{\p};
\end{codeexample}

    Another interesting case are the horizontal/vertical line-to operations
    \verb!|-! and \verb!-|!. For them, the position (or time) |0.5| is exactly
    the corner point.
    %
\begin{codeexample}[]
\tikz \draw (0,0) |- (3,1)
  node[pos=0]{0} node[pos=0.5]{1/2} node[pos=0.9]{9/10};
\end{codeexample}

\begin{codeexample}[]
\tikz \draw (0,0) -| (3,1)
  node[pos=0]{0} node[pos=0.5]{1/2} node[pos=0.9]{9/10};
\end{codeexample}

    For all other path construction operations, \emph{the position placement
    does not work}, currently.
\end{key}

\begin{key}{/tikz/auto=\opt{\meta{direction}} (default \normalfont is scope's setting)}
    This option causes an anchor position to be calculated automatically
    according to the following rule. Consider a line between two points. If the
    \meta{direction} is |left|, then the anchor is chosen such that the node is
    to the left of this line. If the \meta{direction} is |right|, then the node
    is to the right of this line. Leaving out \meta{direction} causes automatic
    placement to be enabled with the last value of |left| or |right| used. A
    \meta{direction} of |false| disables automatic placement. This happens also
    whenever an anchor is given explicitly by the |anchor| option or by one of
    the |above|, |below|, etc.\ options.

    This option only has an effect for nodes that are placed on lines or
    curves.
    %
\begin{codeexample}[]
\begin{tikzpicture}
  [scale=.8,auto=left,every node/.style={circle,fill=blue!20}]
  \node (a) at (-1,-2) {a};
  \node (b) at ( 1,-2) {b};
  \node (c) at ( 2,-1) {c};
  \node (d) at ( 2, 1) {d};
  \node (e) at ( 1, 2) {e};
  \node (f) at (-1, 2) {f};
  \node (g) at (-2, 1) {g};
  \node (h) at (-2,-1) {h};

  \foreach \from/\to in {a/b,b/c,c/d,d/e,e/f,f/g,g/h,h/a}
    \draw [->] (\from) -- (\to)
               node[midway,fill=red!20] {\from--\to};
\end{tikzpicture}
\end{codeexample}
    %
\end{key}

\begin{key}{/tikz/swap}
    This option exchanges the roles of |left| and |right| in automatic
    placement. That is, if |left| is the current |auto| placement, |right| is
    set instead and the other way round.
    %
\begin{codeexample}[preamble={\usetikzlibrary{automata}}]
\begin{tikzpicture}[auto]
  \draw[help lines,use as bounding box] (0,-.5) grid (4,5);

  \draw (0.5,0) .. controls (9,6) and (-5,6) .. (3.5,0)
    node foreach \pos in {0,0.1,0.2,0.3,0.4,0.5,0.6,0.7,0.8,0.9,1}
         [pos=\pos,swap,fill=red!20] {\pos}
    node foreach \pos in {0.025,0.2,0.4,0.6,0.8,0.975}
         [pos=\pos,fill=blue!20] {\pos};
\end{tikzpicture}
\end{codeexample}
    %
\begin{codeexample}[preamble={\usetikzlibrary{automata}}]
\begin{tikzpicture}[shorten >=1pt,node distance=2cm,auto]
  \draw[help lines] (0,0) grid (3,2);

  \node[state] (q_0)                      {$q_0$};
  \node[state] (q_1) [above right of=q_0] {$q_1$};
  \node[state] (q_2) [below right of=q_0] {$q_2$};
  \node[state] (q_3) [below right of=q_1] {$q_3$};

  \path[->] (q_0) edge              node        {0} (q_1)
                  edge              node [swap] {1} (q_2)
            (q_1) edge              node        {1} (q_3)
                  edge [loop above] node        {0} ()
            (q_2) edge              node [swap] {0} (q_3)
                  edge [loop below] node        {1} ();
\end{tikzpicture}
\end{codeexample}
    %
\end{key}

\begin{key}{/tikz/'}
    This is a very short alias for |swap|.
\end{key}

\begin{key}{/tikz/sloped}
    This option causes the node to be rotated such that a horizontal line
    becomes a tangent to the curve. The rotation is normally done in such a way
    that text is never ``upside down''. To get upside-down text, use can use
    |[rotate=180]| or |[allow upside down]|, see below.
    %
\begin{codeexample}[]
\tikz \draw (0,0) .. controls +(up:2cm) and +(left:2cm) .. (1,3)
    node foreach \p in {0,0.25,...,1} [sloped,above,pos=\p]{\p};
\end{codeexample}
    %
\begin{codeexample}[]
\begin{tikzpicture}[->]
  \draw (0,0)   -- (2,0.5) node[midway,sloped,above] {$x$};
  \draw (2,-.5) -- (0,0)   node[midway,sloped,below] {$y$};
\end{tikzpicture}
\end{codeexample}
    %
\end{key}


\begin{key}{/tikz/allow upside down=\meta{boolean} (default true, initially false)}
    If set to |true|, \tikzname\ will not ``righten'' upside down text.
    %
\begin{codeexample}[]
\tikz [allow upside down]
  \draw (0,0) .. controls +(up:2cm) and +(left:2cm) .. (1,3)
    node foreach \p in {0,0.25,...,1} [sloped,above,pos=\p]{\p};
\end{codeexample}
    %
\begin{codeexample}[]
\begin{tikzpicture}[->,allow upside down]
  \draw (0,0)   -- (2,0.5) node[midway,sloped,above] {$x$};
  \draw (2,-.5) -- (0,0)   node[midway,sloped,below] {$y$};
\end{tikzpicture}
\end{codeexample}
    %
\end{key}

There exist styles for specifying positions a bit less ``technically'':

\begin{stylekey}{/tikz/midway}
    This has the same effect as |pos=0.5|.
    %
\begin{codeexample}[]
\tikz \draw (0,0) .. controls +(up:2cm) and +(left:3cm) .. (1,5)
       node[at end]          {\texttt{at end}}
       node[very near end]   {\texttt{very near end}}
       node[near end]        {\texttt{near end}}
       node[midway]          {\texttt{midway}}
       node[near start]      {\texttt{near start}}
       node[very near start] {\texttt{very near start}}
       node[at start]        {\texttt{at start}};
\end{codeexample}
    %
\end{stylekey}

\begin{stylekey}{/tikz/near start}
    Set to |pos=0.25|.
\end{stylekey}

\begin{stylekey}{/tikz/near end}
    Set to |pos=0.75|.
\end{stylekey}

\begin{stylekey}{/tikz/very near start}
    Set to |pos=0.125|.
\end{stylekey}

\begin{stylekey}{/tikz/very near end}
    Set to |pos=0.875|.
\end{stylekey}

\begin{stylekey}{/tikz/at start}
    Set to |pos=0|.
\end{stylekey}

\begin{stylekey}{/tikz/at end}
    Set to |pos=1|.
\end{stylekey}


\subsection{Placing Nodes on a Line or Curve Implicitly}
\label{section-nodes-placing-2}

When you wish to place a node on the line |(0,0) -- (1,1)|, it is natural to
specify the node not following the |(1,1)|, but ``somewhere in the middle''.
This is, indeed, possible and you can write |(0,0) -- node{a} (1,1)| to place a
node midway between |(0,0)| and |(1,1)|.

What happens is the following: The syntax of the line-to path operation is
actually |--| \opt{|node|\meta{node specification}}\meta{coordinate}. (It is
even possible to give multiple nodes in this way.) When the optional |node| is
encountered, that is, when the |--| is directly followed by |node|, then the
specification(s) are read and ``stored away''. Then, after the
\meta{coordinate} has finally been reached, they are inserted again, but with
the |pos| option set.

There are two things to note about this: When a node specification is
``stored'', its catcodes become fixed. This means that you cannot use overly
complicated verbatim text in them. If you really need, say, a verbatim text,
you will have to put it in a normal node following the coordinate and add the
|pos| option.

Second, which |pos| is chosen for the node? The position is inherited from the
surrounding scope. However, this holds only for nodes specified in this
implicit way. Thus, if you add the option |[near end]| to a scope, this does
not mean that \emph{all} nodes given in this scope will be put on near the end
of lines. Only the nodes for which an implicit |pos| is added will be placed
near the end. Typically, this is what you want. Here are some examples that
should make this clearer:
%
\begin{codeexample}[]
\begin{tikzpicture}[near end]
  \draw (0cm,4em) -- (3cm,4em) node{A};
  \draw (0cm,3em) --           node{B}          (3cm,3em);
  \draw (0cm,2em) --           node[midway] {C} (3cm,2em);
  \draw (0cm,1em) -- (3cm,1em) node[midway] {D} ;
\end{tikzpicture}
\end{codeexample}

Like the line-to operation, the curve-to operation |..| also allows you to
specify nodes ``inside'' the operation. After both the first |..| and also
after the second |..| you can place node specifications. Like for the |--|
operation, these will be collected and then reinserted after the operation with
the |pos| option set.


\subsection{The Label and Pin Options}

\subsubsection{Overview}

In addition to the |node| path operation, the two options |label| and |pin| can
be used to ``add a node next to another node''. As an example, suppose we want
to draw a graph in which the nodes are small circles:
%
\begin{codeexample}[preamble={\usetikzlibrary{positioning}}]
\tikz [circle] {
  \node [draw] (s) {};
  \node [draw] (a) [right=of s] {} edge (s);
  \node [draw] (b) [right=of a] {} edge (a);
  \node [draw] (t) [right=of b] {} edge (b);
}
\end{codeexample}

Now, in the above example, suppose we wish to indicate that the first node is
the start node and the last node is the target node. We could write
|\node (s) {$s$};|, but this would enlarge the first node. Rather, we want the
``$s$'' to be placed next to the node. For this, we need to create
\emph{another} node, but next to the existing node. The |label| and |pin|
option allow us to do exactly this without having to use the cumbersome |node|
syntax:
%
\begin{codeexample}[preamble={\usetikzlibrary{positioning}}]
\tikz [circle] {
  \node [draw] (s) [label=$s$]  {};
  \node [draw] (a) [right=of s] {} edge (s);
  \node [draw] (b) [right=of a] {} edge (a);
  \node [draw] (t) [right=of b, label=$t$] {} edge (b);
}
\end{codeexample}


\subsubsection{The Label Option}

\begin{key}{/tikz/label=\opt{|[|\meta{options}|]|\meta{angle}|:|}\meta{text}}
        \label{label-option}%
    When this option is given to a |node| operation, it causes \emph{another}
    node to be added to the path after the current node has been finished. This
    extra node will have the text \meta{text}. It is placed, in principle, in
    the direction \meta{angle} relative to the main node, but the exact rules
    are a bit complex. Suppose the |node| currently under construction is
    called |main node| and let us call the label node |label node|. Then the
    following happens:
    %
    \begin{enumerate}
        \item The \meta{angle} is used to determine a position on the border of
            the |main node|. If the \meta{angle} is missing, the value of the
            following key is used instead:
            %
            \begin{key}{/tikz/label position=\meta{angle} (initially above)}
                Sets the default position for labels.
            \end{key}
            %
            The \meta{angle} determines the position on the border of the shape
            in two different ways. Normally, the border position is given by
            |main node.|\meta{angle}. This means that the \meta{angle} can
            either be a number like |0| or |-340|, but it can also be an anchor
            like |north|. Additionally, the special angles |above|, |below|,
            |left|, |right|, |above left|, and so on are automatically replaced
            by the corresponding angles |90|, |270|, |180|, |0|, |135|, and so
            on.

            A special case arises when the following key is set:
            %
            \begin{key}{/tikz/absolute=\meta{true or false} (default true)}
                When this key is set, the \meta{angle} is interpreted
                differently: We still use a point on the border of the
                |main node|, but the angle is measured ``absolutely'', that is,
                an angle of |0| refers to the point on the border that lies on
                a straight line from the |main node|'s center to the right
                (relative to the paper, not relative to the local coordinate
                system of either the node or the scope).

                The difference can be seen in the following example:
                %
\begin{codeexample}[]
\tikz [rotate=-80,every label/.style={draw,red}]
  \node [transform shape,rectangle,draw,label=right:label] {main node};
\end{codeexample}
                %
\begin{codeexample}[]
\tikz [rotate=-80,every label/.style={draw,red},absolute]
  \node [transform shape,rectangle,draw,label=right:label] {main node};
\end{codeexample}
            \end{key}
        \item Then, an anchor point for the |label node| is computed. It is
            determined in such a way that the |label node| will ``face away''
            from the border of the |main node|. The anchor that is chosen
            depends on the position of the border point that is chosen and its
            position relative to the center of the |main node| and on whether
            the |transform shape| option is set. In detail, when the computed
            border point is at $0^\circ$, the anchor |west| will be used.
            Similarly, when the border point is at $90^\circ$, the anchor
            |south| will be used, and so on for $180^\circ$ and $270^\circ$.

            For angles between these ``major'' angles, like $30^\circ$ or
            $110^\circ$, combined anchors, like |south west| for $30^\circ$ or
            |south east| for $110^\circ$, are used. However, for angles close
            to the major angles, (differing by up to $2^\circ$ from the major
            angle), the anchor for the major angle is used. Thus, a label at a
            border point for $2^\circ$ will have the anchor |west|, while a
            label for $3^\circ$ will have the anchor |south west|, resulting in
            a ``jump'' of the anchor. You can set the anchor ``by hand'' using
            the |anchor| key or indirect keys like |left|.
            %
\begin{codeexample}[]
\tikz
  \node [circle, draw,
         label=default,
         label=60:$60^\circ$,
         label=below:$-90^\circ$,
         label=3:$3^\circ$,
         label=2:$2^\circ$,
         label={[below]180:$180^\circ$},
         label={[centered]135:$135^\circ$}] {my circle};
\end{codeexample}
        \item One \meta{angle} is special: If you set the \meta{angle} to
            |center|, then the label will be placed on the center of the main
            node. This is mainly useful for adding a label text to an existing
            node, especially if it has been rotated.
            %
\begin{codeexample}[]
\tikz \node [transform shape,rotate=90,
             rectangle,draw,label={[red]center:R}] {main node};
\end{codeexample}
    \end{enumerate}

    You can pass \meta{options} to the node |label node|. For this, you provide
    the options in square brackets before the \meta{angle}. If you do so, you
    need to add braces around the whole argument of the |label| option and this
    is also the case if you have brackets or commas or semicolons or anything
    special in the \meta{text}.
    %
\begin{codeexample}[]
\tikz \node [circle,draw,label={[red]above:X}] {my circle};
\end{codeexample}

\begin{codeexample}[]
\begin{tikzpicture}
  \node [circle,draw,label={[name=label node]above left:$a,b$}] {};
  \draw (label node) -- +(1,1);
\end{tikzpicture}
\end{codeexample}

    If you provide multiple |label| options, then multiple extra label nodes
    are added in the order they are given.

    The following styles influence how labels are drawn:
    %
    \begin{key}{/tikz/label distance=\meta{distance} (initially 0pt)}
        The \meta{distance} is additionally inserted between the main node and
        the label node.
        %
\begin{codeexample}[]
\tikz[label distance=5mm]
  \node [circle,draw,label=right:X,
                     label=above right:Y,
                     label=above:Z]       {my circle};
\end{codeexample}
    \end{key}
    %
    \begin{stylekey}{/tikz/every label (initially \normalfont empty)}
        This style is used in every node created by the |label| option. The
        default is |draw=none,fill=none|.
    \end{stylekey}
\end{key}

See Section~\ref{section-label-quotes} for an easier syntax for specifying
nodes.


\subsubsection{The Pin Option}

\begin{key}{/tikz/pin=\opt{|[|\meta{options}|]|}\meta{angle}|:|\meta{text}}
    This option is quite similar to the |label| option, but there is one
    difference: In addition to adding an extra node to the picture, it also
    adds an edge from this node to the main node. This causes the node to look
    like a pin that has been added to the main node:
    %
\begin{codeexample}[]
\tikz \node [circle,fill=blue!50,minimum size=1cm,pin=60:$q_0$] {};
\end{codeexample}

    The meaning of the \meta{options} and the \meta{angle} and the \meta{text}
    is exactly the same as for the |node| option. Only, the options and styles
    the influence the way pins look are different:
    %
    \begin{key}{/tikz/pin distance=\meta{distance} (initially 3ex)}
        This \meta{distance} is used instead of the |label distance| for the
        distance between the main node and the label node.
        %
\begin{codeexample}[]
\tikz[pin distance=1cm]
  \node [circle,draw,pin=right:X,
                     pin=above right:Y,
                     pin=above:Z]       {my circle};
\end{codeexample}
        %
    \end{key}

    \begin{stylekey}{/tikz/every pin (initially {draw=none,fill=none})}
        This style is used in every node created by the |pin| option.
    \end{stylekey}

    \begin{key}{/tikz/pin position=\meta{angle} (initially above)}
        The default pin position. Works like |label position|.
    \end{key}

    \begin{stylekey}{/tikz/every pin edge (initially help lines)}
        This style is used in every edge created by the |pin| options.
        %
\begin{codeexample}[preamble={\usetikzlibrary{decorations.pathmorphing}}]
\tikz [pin distance=15mm,
       every pin edge/.style={<-,shorten <=1pt,decorate,
                              decoration={snake,pre length=4pt}}]
  \node [circle,draw,pin=right:X,
                     pin=above right:Y,
                     pin=above:Z]       {my circle};
\end{codeexample}
    \end{stylekey}

    \begin{key}{/tikz/pin edge=\meta{options} (initially \normalfont empty)}
        This option can be used to set the options that are to be used in the
        edge created by the |pin| option.
        %
\begin{codeexample}[]
\tikz[pin distance=10mm]
  \node [circle,draw,pin={[pin edge={blue,thick}]right:X},
                     pin=above:Z]       {my circle};
\end{codeexample}
        %
\begin{codeexample}[]
\tikz [every pin edge/.style={},
       initial/.style={pin={[pin distance=5mm,
                             pin edge={<-,shorten <=1pt}]left:start}}]
  \node [circle,draw,initial] {my circle};
\end{codeexample}
    \end{key}
\end{key}


\subsubsection{The Quotes Syntax}
\label{section-label-quotes}

The |label| and |pin| options provide a syntax for creating nodes next to
existing nodes, but this syntax is often a bit too verbose. By including the
following library, you get access to an even more concise syntax:

\begin{tikzlibrary}{quotes}
    Enables the quotes syntax for labels, pins, edge nodes, and pic texts.
\end{tikzlibrary}

Let us start with the basics of what this library does: Once loaded, inside the
options of a |node| command, instead of the usual \meta{key}|=|\meta{value}
pairs, you may also provide strings of the following form (the actual syntax is
slightly more general, see the detailed descriptions later on):
%
\begin{quote}
    |"|\meta{text}|"|\opt{\meta{options}}
\end{quote}
%
The \meta{options} must be surrounded in curly braces when they contain a
comma, otherwise the curly braces are optional. The \meta{options} may be
preceded by an optional space.

When a \meta{string} of the above form is encountered inside the options of a
|node|, then it is internally transformed to
%
% (the double vertical bar after = is needed to avoid the two opening brackets
%  being typeset in italics)
\begin{quote}
    |label=||{[|\meta{options}|]|\meta{text}|}|
\end{quote}

Let us have a look at an example:
%
\begin{codeexample}[preamble={\usetikzlibrary{quotes}}]
\tikz \node ["my label" red, draw] {my node};
\end{codeexample}
%
The above has the same effect as the following:
%
\begin{codeexample}[]
\tikz \node [label={[red]my label}, draw] {my node};
\end{codeexample}

Here are further examples, one where no \meta{options} are added to the
|label|, one where a position is specified, and examples with more complicated
options in curly braces:
%
\begin{codeexample}[preamble={\usetikzlibrary{quotes}}]
\begin{tikzpicture}
  \matrix [row sep=5mm] {
    \node [draw, "label"]                  {A}; \\
    \node [draw, "label" left]             {B}; \\
    \node [draw, "label" centered]         {C}; \\
    \node [draw, "label" color=red]        {D}; \\
    \node [draw, "label" {red,draw,thick}] {E}; \\
  };
\end{tikzpicture}
\end{codeexample}

Let us now have a more detailed look at what commands this library
provides:

\begin{key}{/tikz/quotes mean label}
    When this option is used (which is the default when this library is
    loaded), then, as described above, inside the options of a node a special
    syntax check is done.

    \medskip
    \noindent\textbf{The syntax.}
    For each string in the list of options it is tested whether it starts with
    a quotation mark (note that this will never happen for normal keys since
    the normal keys of \tikzname\ do not start with quotation marks). When this
    happens, the \meta{string} should not be a key--value pair, but, rather,
    must have the form:
    %
    \begin{quote}
        |"|\meta{text}|"|\opt{|'|}\opt{\meta{options}}
    \end{quote}

    (We will discuss the optional apostrophe in a moment. It is not really
    important for the current option, but only for edge labels, which are
    discussed later).

    \medskip
    \noindent\textbf{Transformation to a label option.}
    When a \meta{string} has the above form, it is treated (almost) as if you
    had written
    %
    \begin{quote}
        |label={[|\meta{options}|]|\meta{text}|}|
    \end{quote}
    %
    instead. The ``almost'' refers to the following additional feature: In
    reality, before the \meta{options} are executed inside the |label| command,
    the direction keys |above|, |left|, |below right| and so on are redefined
    so that |above| is a shorthand for |label position=90| and similarly for
    the other keys. The net effect is that in order to specify the position of
    the \meta{text} relative to the main node you can just put something like
    |left| or |above right| inside the \meta{options}:
    %
\begin{codeexample}[preamble={\usetikzlibrary{quotes}}]
\tikz
  \node ["$90^\circ$" above, "$180^\circ$" left, circle, draw] {circle};
\end{codeexample}

    Alternatively, you can also use \meta{direction}|:|\meta{actual text} as
    your \meta{text}. This works since the |label| command allows you to
    specify a direction at the beginning when it is separated by a colon:
    %
\begin{codeexample}[preamble={\usetikzlibrary{quotes}}]
\tikz
  \node ["90:$90^\circ$", "left:$180^\circ$", circle, draw] {circle};
\end{codeexample}
    %
    Arguably, placing |above| or |left| behind the \meta{text} seems more
    natural than having it inside the \meta{text}.

    In addition to the above, before the \meta{options} are executed, the
    following style is also executed:
    %
    \begin{stylekey}{/tikz/every label quotes}
\begin{codeexample}[preamble={\usetikzlibrary{quotes}}]
\tikz [every label quotes/.style=red]
  \node ["90:$90^\circ$", "left:$180^\circ$", circle, draw] {circle};
\end{codeexample}
    \end{stylekey}

    \medskip
    \noindent\textbf{Handling commas and colons inside the text.}
    The \meta{text} may not contain a comma, unless it is inside curly braces.
    The reason is that the key handler separates the total options of a |node|
    along the commas it finds. So, in order to have text containing a comma,
    just add curly braces around either the comma or just around the whole
    \meta{text}:
    %
\begin{codeexample}[preamble={\usetikzlibrary{quotes}}]
\tikz \node ["{yes, we can}", draw] {foo};
\end{codeexample}
    %
    The same is true for a colon, only in this case you may need to surround
    specifically the colon by curly braces to stop the |label| option from
    interpreting everything before the colon as a direction:
    %
\begin{codeexample}[preamble={\usetikzlibrary{quotes}}]
\tikz \node ["yes{:} we can", draw] {foo};
\end{codeexample}

    \medskip
    \noindent\textbf{The optional apostrophe.}
    Following the closing quotation marks in a \meta{string} there may (but
    need not) be a single quotation mark (an apostrophe), possibly surrounded
    by whitespaces. If it is present, it is simply added to the \meta{options}
    as another option (and, indeed, a single apostrophe is a legal option in
    \tikzname, it is a shorthand for |swap|):

    \begin{tabular}{ll}
        String         & has the same effect as \\\hline
        |"foo"'|       & |"foo" {'}| \\
        |"foo"' red|   & |"foo" {',red}| \\
        |"foo"'{red}|  & |"foo" {',red}| \\
        |"foo"{',red}| & |"foo" {',red}| \\
        |"foo"{red,'}| & |"foo" {red,'}| \\
        |"foo"{'red}|  & |"foo" {'red}| (illegal; there is no key |'red|)\\
        |"foo" red'|   & |"foo" {red'}| (illegal; there is no key |red'|)\\
    \end{tabular}
\end{key}

\begin{key}{/tikz/quotes mean pin}
    This option has exactly the same effect as |quotes mean label|, only
    instead of transforming quoted text to the |label| option, they get
    transformed to the |pin| option:
    %
\begin{codeexample}[preamble={\usetikzlibrary{quotes}}]
\tikz [quotes mean pin]
  \node ["$90^\circ$" above, "$180^\circ$" left, circle, draw] {circle};
\end{codeexample}
    %
    Instead of |every label quotes|, the following style is executed
    with each such pin:
    %
    \begin{stylekey}{/tikz/every pin quotes}
    \end{stylekey}
\end{key}

If instead of |label|s or |pin|s you would like quoted strings to be
interpreted in a different manner, you can also define your own handlers:

\begin{key}{/tikz/node quotes mean=\meta{replacement}}
    This key allows you to define your own handler for quotes options. Inside
    the options of a |node|, whenever a key--value pair with the syntax
    %
    \begin{quote}
        |"|\meta{text}|"|\opt{|'|}\opt{\meta{options}}
    \end{quote}
    %
    is encountered, the following happens: The above string gets replaced by
    \meta{replacement} where inside the \meta{replacement} the parameter |#1|
    is \meta{text} and |#2| is \meta{options}. If the apostrophe is present
    (see also the discussion of |quotes mean label|), the \meta{options} start
    with |',|.

    The \meta{replacement} is then parsed normally as options (using
    |\pgfkeys|).

    Here is an example, where the quotes are used to define labels that are
    automatically named according to the |text|:
    %
\begin{codeexample}[preamble={\usetikzlibrary{quotes}}]
\tikzset{node quotes mean={label={[#2,name={#1}]#1}}}

\tikz {
  \node ["1", "2" label position=left, circle, draw] {circle};
  \draw (1) -- (2);
}
\end{codeexample}
    %
\end{key}

Some further options provided by the |quotes| library concern labels next to
edges rather than nodes and they are described in
Section~\ref{section-edge-quotes}.


\subsection{Connecting Nodes: Using Nodes as Coordinates}
\label{section-nodes-connecting}

Once you have defined a node and given it a name, you can use this name to
reference it. This can be done in two ways, see also
Section~\ref{section-node-coordinates}. Suppose you have said
|\path(0,0) node(x) {Hello World!};| in order to define a node named |x|.
%
\begin{enumerate}
    \item Once the node |x| has been defined, you can use |(x.|\meta{anchor}|)|
        wherever you would normally use a normal coordinate. This will yield
        the position at which the given \meta{anchor} is in the picture. Note
        that transformations do not apply to this coordinate, that is,
        |(x.north)| will be the northern anchor of |x| even if you have said
        |scale=3| or |xshift=4cm|. This is usually what you would expect.
    \item You can also just use |(x)| as a coordinate. In most cases, this
        gives the same coordinate as |(x.center)|. Indeed, if the |shape| of
        |x| is |coordinate|, then |(x)| and |(x.center)| have exactly the same
        effect.

        However, for most other shapes, some path construction operations like
        |--| try to be ``clever'' when they are asked to draw a line from such
        a coordinate or to such a coordinate. When you say |(x)--(1,1)|, the
        |--| path operation will not draw a line from the center of |x|, but
        \emph{from the border} of |x| in the direction going towards |(1,1)|.
        Likewise, |(1,1)--(x)| will also have the line end on the border in the
        direction coming from |(1,1)|.

        If the specified coordinate is almost identical to the node center, for
        example |(x)--(0,0)|, no line will be drawn and a warning message will
        be printed.

        In addition to |--|, the curve-to path operation |..| and the path
        operations \verb!-|! and \verb!|-! will also handle nodes without
        anchors correctly. Here is an example, see also
        Section~\ref{section-node-coordinates}:
        %
\begin{codeexample}[]
\begin{tikzpicture}
  \path (0,0) node             (x) {Hello World!}
        (3,1) node[circle,draw](y) {$\int_1^2 x \mathrm d x$};

  \draw[->,blue]   (x) -- (y);
  \draw[->,red]    (x) -| node[near start,below] {label} (y);
  \draw[->,orange] (x) .. controls +(up:1cm) and +(left:1cm) .. node[above,sloped] {label} (y);
\end{tikzpicture}
\end{codeexample}
        %
\end{enumerate}


\subsection{Connecting Nodes: Using the Edge Operation}
\label{section-nodes-edges}

\subsubsection{Basic Syntax of the Edge Operation}

The |edge| operation works like a |to| operation that is added after the main
path has been drawn, much like a node is added after the main path has been
drawn. This allows each |edge| to have a different appearance. As the |node|
operation, an |edge| temporarily suspends the construction of the current path
and a new path $p$ is constructed. This new path $p$ will be drawn after the
main path has been drawn. Note that $p$ can be totally different from the main
path with respect to its options. Also note that if there are several |edge|
and/or |node| operations in the main path, each creates its own path(s) and
they are drawn in the order that they are encountered on the main path.

\begin{pathoperation}{edge}{\opt{|[|\meta{options}|]|} \opt{\meta{nodes}} |(|\meta{coordinate}|)|}
    The effect of the |edge| operation is that after the main path the
    following path is added to the picture:
    %
    \begin{quote}
        |\path[every edge,|\meta{options}|] (\tikztostart) |\meta{path}|;|
    \end{quote}
    %
    Here, \meta{path} is the |to path|. Note that, unlike the path added by the
    |to| operation, the |(\tikztostart)| is added before the \meta{path} (which
    is unnecessary for the |to| operation, since this coordinate is already
    part of the main path).

    The |\tikztostart| is the last coordinate on the path just before the
    |edge| operation, just as for the |node| or |to| operations. However, there
    is one exception to this rule: If the |edge| operation is directly preceded
    by a |node| operation, then this just-declared node is the start coordinate
    (and not, as would normally be the case, the coordinate where this
    just-declared node is placed -- a small, but subtle difference). In this
    regard, |edge| differs from both |node| and |to|.

    If there are several |edge| operations in a row, the start coordinate is
    the same for all of them as their target coordinates are not, after all,
    part of the main path. The start coordinate is, thus, the coordinate
    preceding the first |edge| operation. This is similar to nodes insofar as
    the |edge| operation does not modify the current path at all. In
    particular, it does not change the last coordinate visited, see the
    following example:
    %
\begin{codeexample}[]
\begin{tikzpicture}
  \node (a) at   (0:1) {$a$};
  \node (b) at  (90:1) {$b$} edge [->]     (a);
  \node (c) at (180:1) {$c$} edge [->]     (a)
                             edge [<-]     (b);
  \node (d) at (270:1) {$d$} edge [->]     (a)
                             edge [dotted] (b)
                             edge [<-]     (c);
\end{tikzpicture}
\end{codeexample}

    A different way of specifying the above graph using the |edge| operation is
    the following:
    %
\begin{codeexample}[]
\begin{tikzpicture}
  \node foreach \name/\angle in {a/0,b/90,c/180,d/270}
        (\name) at (\angle:1) {$\name$};

  \path[->] (b) edge (a)
                edge (c)
                edge [-,dotted] (d)
            (c) edge (a)
                edge (d)
            (d) edge (a);
\end{tikzpicture}
\end{codeexample}

    As can be seen, the path of the |edge| operation inherits the options from
    the main path, but you can locally overrule them.
    %
\begin{codeexample}[]
\begin{tikzpicture}
  \node foreach \name/\angle in {a/0,b/90,c/180,d/270}
        (\name) at (\angle:1.5) {$\name$};

  \path[->] (b) edge            node[above right]  {$5$}     (a)
                edge                                         (c)
                edge [-,dotted] node[below,sloped] {missing} (d)
            (c) edge                                         (a)
                edge                                         (d)
            (d) edge [red]      node[above,sloped] {very}
                                node[below,sloped] {bad}     (a);
\end{tikzpicture}
\end{codeexample}

    Instead of |every to|, the style |every edge| is installed at the beginning
    of the main path.
    %
    \begin{stylekey}{/tikz/every edge (initially draw)}
        Executed for each |edge|.
        %
\begin{codeexample}[]
\begin{tikzpicture}[every edge/.style={draw,dashed}]
  \path (0,0) edge (3,2);
\end{tikzpicture}
\end{codeexample}
    \end{stylekey}
\end{pathoperation}


\subsubsection{Nodes on Edges: Quotes Syntax}
\label{section-edge-quotes}

The standard way of specifying nodes that are placed ``on'' an edge (or on a
to-path; all of the following is also true for to--paths) is to put node
specifications after the |edge| keyword, but before the target coordinate.
Another way is to use the |edge node| option and its friends. Yet another way
is to use the quotes syntax.

The syntax is essentially the same as for labels added to nodes as described in
Section~\ref{section-label-quotes} and you also need to load the |quotes|
library.

In detail, when the |quotes| library is loaded, each time a key--value pair in
a list of options passed to an |edge| or a |to| path command starts with |"|,
the key--value pair must actually be a string of the following form:
%
\begin{quote}
    |"|\meta{text}|"|\opt{|'|}\opt{\meta{options}}
\end{quote}
%
This string is transformed into the following:
%
\begin{quote}
    |edge node=node [every edge quotes,|\meta{options}|]{|\meta{text}|}|
\end{quote}
%
As described in Section~\ref{section-label-quotes}, the apostrophe becomes part
of the \meta{options}, when present.

The following style is important for the placement of the labels:

\begin{stylekey}{/tikz/every edge quotes (initially auto)}
    This style is |auto| by default, which causes labels specified using the
    quotes-syntax to be placed next to the edges. Unless the setting of |auto|
    has been changed, they will be placed to the left.
    %
\begin{codeexample}[preamble={\usetikzlibrary{quotes}}]
\tikz \draw (0,0) edge ["left", ->] (2,0);
\end{codeexample}

    In order to place all labels to the right by default, change this style to
    |auto=right|:
    %
\begin{codeexample}[preamble={\usetikzlibrary{quotes}}]
\tikz [every edge quotes/.style={auto=right}]
  \draw (0,0) edge ["right", ->] (2,0);
\end{codeexample}

    To place all nodes ``on'' the edge, just make this style empty (and,
    possibly, make your labels opaque):
    %
\begin{codeexample}[preamble={\usetikzlibrary{quotes}}]
\tikz [every edge quotes/.style={fill=white,font=\footnotesize}]
  \draw (0,0) edge ["mid", ->] (2,1);
\end{codeexample}
    %
\end{stylekey}

You may often wish to place some edge nodes to the right of edges and some to
the left. For this, the special treatment of the apostrophe is particularly
convenient: Recall that in \tikzname\ there is an option just called |'|, which
is a shorthand for |swap|. Now, following the closing quotation mark come the
options of an edge node. Thus, if the closing quotation mark is followed by an
apostrophe, the |swap| option will be added to the edge label, causing it is be
placed on the other side. Because of the special treatment, you can even add
another option like |near end| after the apostrophe without having to add curly
braces and commas:
%
\begin{codeexample}[preamble={\usetikzlibrary{quotes}}]
\tikz
  \draw (0,0) edge ["left", "right"',
                    "start" near start,
                    "end"' near end] (4,0);
\end{codeexample}

In order to modify the distance between the edge labels and the edge, you
should consider introducing some styles:
%
\begin{codeexample}[preamble={\usetikzlibrary{quotes}}]
\tikz [tight/.style={inner sep=1pt}, loose/.style={inner sep=.7em}]
  \draw (0,0) edge ["left"   tight,
                    "right"' loose,
                    "start"  near start] (4,0);
\end{codeexample}


\subsection{Referencing Nodes Outside the Current Picture}
\label{section-cross-picture-tikz}

\subsubsection{Referencing a Node in a Different Picture}

It is possible (but not quite trivial) to reference nodes in pictures other
than the current one. This means that you can create a picture and a node
therein and, later, you can draw a line from some other position to this node.

To reference nodes in different pictures, proceed as follows:
%
\begin{enumerate}
    \item You need to add the |remember picture| option to all pictures that
        contain nodes that you wish to reference and also to all pictures from
        which you wish to reference a node in another picture.
    \item You need to add the |overlay| option to paths or to whole pictures
        that contain references to nodes in different pictures. (This option
        switches the computation of the bounding box off.)
    \item You need to use a driver that supports picture remembering and you
        need to run \TeX\ twice.
\end{enumerate}
%
(For more details on what is going on behind the scenes, see
Section~\ref{section-cross-pictures-pgf}.)

Let us have a look at the effect of these options.
%
\begin{key}{/tikz/remember picture=\meta{boolean} (initially false)}
    This option tells \tikzname\ that it should attempt to remember the
    position of the current picture on the page. This attempt may fail
    depending on which backend driver is used. Also, even if remembering works,
    the position may only be available on a second run of \TeX.

    Provided that remembering works, you may consider saying
    %
\begin{codeexample}[code only]
\tikzset{every picture/.append style={remember picture}}
\end{codeexample}
    %
    to make \tikzname\ remember all pictures. This will add one line in the
    |.aux| file for each picture in your document -- which typically is not
    very much. Then, you do not have to worry about remembered pictures at all.
\end{key}

\begin{key}{/tikz/overlay=\meta{boolean} (default true)}
    This option is mainly intended for use when nodes in other pictures are
    referenced, but you can also use it in other situations. The effect of this
    option is that everything within the current scope is not taken into
    consideration when the bounding box of the current picture is computed.

    You need to specify this option on all paths (or at least on all parts of
    paths) that contain a reference to a node in another picture. The reason is
    that, otherwise, \tikzname\ will attempt to make the current picture large
    enough to encompass \emph{the node in the other picture}. However, on a
    second run of \TeX\ this will create an even bigger picture, leading to
    larger and larger pictures. Unless you know what you are doing, I suggest
    specifying the |overlay| option with all pictures that contain references
    to other pictures.
\end{key}

Let us now have a look at a few examples. These examples work only if this
document is processed with a driver that supports picture remembering.
\medskip

\noindent%
\begin{minipage}{\textwidth}
Inside the current text we place two pictures, containing nodes named |n1| and
|n2|, using
%
\begin{codeexample}[code only]
\tikz[remember picture] \node[circle,fill=red!50] (n1) {};
\end{codeexample}
%
which yields \tikz[remember picture] \node[circle,fill=red!50] (n1) {};, and
%
\begin{codeexample}[code only]
\tikz[remember picture] \node[fill=blue!50] (n2) {};
\end{codeexample}
%
yielding the node \tikz[remember picture] \node[fill=blue!50] (n2) {};. To
connect these nodes, we create another picture using the |overlay| option and
also the |remember picture| option.
%
\begin{codeexample}[]
\begin{tikzpicture}[remember picture,overlay]
  \draw[->,very thick] (n1) -- (n2);
\end{tikzpicture}
\end{codeexample}
%
Note that the last picture is seemingly empty. What happens is that it has zero
size and contains an arrow that lies well outside its bounds. As a last
example, we connect a node in another picture to the first two nodes. Here, we
provide the |overlay| option only with the line that we do not wish to count as
part of the picture.
%
\begin{codeexample}[]
\begin{tikzpicture}[remember picture]
  \node (c) [circle,draw] {Big circle};

  \draw [overlay,->,very thick,red,opacity=.5]
    (c) to[bend left] (n1) (n1) -| (n2);
\end{tikzpicture}
\end{codeexample}
\end{minipage}


\subsubsection{Referencing the Current Page Node -- Absolute Positioning}

There is a special node called |current page| that can be used to access the
current page. It is a node of shape rectangle whose |south west| anchor is the
lower left corner of the page and whose |north east| anchor is the upper right
corner of the page. While this node is handled in a special way internally, you
can reference it as if it were defined in some remembered picture other than
the current one. Thus, by giving the |remember picture| and the |overlay|
options to a picture, you can position nodes \emph{absolutely} on a page.

The first example places some text in the lower left corner of the current
page:
%
\begin{codeexample}[]
\begin{tikzpicture}[remember picture,overlay]
  \node [xshift=1cm,yshift=1cm] at (current page.south west)
        [text width=7cm,fill=red!20,rounded corners,above right]
  {
    This is an absolutely positioned text in the
    lower left corner. No shipout-hackery is used.
  };
\end{tikzpicture}
\end{codeexample}

The next example adds a circle in the middle of the page.
%
\begin{codeexample}[]
\begin{tikzpicture}[remember picture,overlay]
  \draw [line width=1mm,opacity=.25]
    (current page.center) circle (3cm);
\end{tikzpicture}
\end{codeexample}

The final example overlays some text over the page (depending on where this
example is found on the page, the text may also be behind the page).
%
\begin{codeexample}[]
\begin{tikzpicture}[remember picture,overlay]
  \node [rotate=60,scale=10,text opacity=0.2]
    at (current page.center) {Example};
\end{tikzpicture}
\end{codeexample}


\subsection{Late Code and Late Options}
\label{section-node-also}

All options given to a node only locally affect this one node. While this is a
blessing in most cases, you may sometimes want to cause options to have effects
``later'' on. The other way round, you may sometimes note ``only later'' that
some options should be added to the options of a node. For this, the following
version of the |node| path command can be used:

\begin{pathoperation}{node also}{\opt{|[|\meta{late options}|]|}|(|\meta{name}|)|}
    Note that the \meta{name} is compulsory and that \emph{no} text may be
    given. Also, the ordering of options and node label must be as above.

    The effect of the above is the following effect: The node \meta{name} must
    already  be existing. Now, the \meta{late options} are executed in a local
    scope. Most of these options will have no effect since you \emph{cannot
    change the appearance of the node,} that is, you cannot change a red node
    into a green node using these ``late'' options. However, giving the
    |append after command| and |prefix after command| options inside the
    \meta{late options} (directly or indirectly) does have the desired effect:
    The given path gets executed with the |\tikzlastnode| set to the determined
    node.

    The net effect of all this is that you can provide, say, the |label| option
    inside the \meta{options} to a add a label to a node that has already been
    constructed.
    %
\begin{codeexample}[]
\begin{tikzpicture}
  \node      [draw,circle]       (a) {Hello};
  \node also [label=above:world] (a);
\end{tikzpicture}
\end{codeexample}
    %
\end{pathoperation}

As explained in Section~\ref{section-paths}, you can use the options
|append after command| and |prefix after command| to add a path after a node.
The following macro may be useful there:
%
\begin{command}{\tikzlastnode}
    Expands to the last node on the path.
\end{command}

Instead of the |node also| syntax, you can also use the following option:

\begin{key}{/tikz/late options=\meta{options}}
    This option can be given on a path (but not as an argument to a |node| path
    command) and has the same effect as the |node also| path command. Inside
    the \meta{options}, you should use the |name| option to specify the node
    for which you wish to add late options:
    %
\begin{codeexample}[]
\begin{tikzpicture}
  \node      [draw,circle]       (a) {Hello};
  \path [late options={name=a, label=above:world}];
\end{tikzpicture}
\end{codeexample}
    %
\end{key}


%%% Local Variables:
%%% mode: latex
%%% TeX-master: "pgfmanual"
%%% End:
