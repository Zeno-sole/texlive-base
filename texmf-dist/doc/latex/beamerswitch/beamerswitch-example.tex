%%
%% This is file `beamerswitch-example.tex',
%% generated with the docstrip utility.
%%
%% The original source files were:
%%
%% beamerswitch.dtx  (with options: `example')
%% ----------------------------------------------------------------
%% beamerswitch --- Convenient mode selection in Beamer documents
%% Author:  Alex Ball
%% E-mail:  ab318@bath.ac.uk
%% License: Released under the LaTeX Project Public License v1.3c or later
%% See:     http://www.latex-project.org/lppl.txt
%% ----------------------------------------------------------------
%% 
\documentclass[%
  beameroptions={ignorenonframetext,14pt},
  articleoptions={a4paper,12pt},
  also={trans,handout,article}]{beamerswitch}
\handoutlayout{nup=3plus,border=1pt}
\articlelayout{maketitle,frametitles=none}
\usepackage[british]{babel}
\mode<article>{
  \usepackage[hmargin=3cm,vmargin=2.5cm]{geometry}
}
\mode<presentation>{
  \usefonttheme{professionalfonts}
}
\mode<handout>{
  \usecolortheme{dove}
}
\usepackage{libertine}

\title{A demonstration of the \textsf{beamerswitch} class}
\subtitle{Testing features}
\author{Alex Ball\inst{1}}
\institute{\inst{1}University of Life}
\date{1 September 2016}
\subject{A LaTeX class}
\keywords{CTAN, literate programming}

\begin{document}
  \begin{frame}
    \maketitle
  \end{frame}

  This very brief demonstration shows how to use the \textsf{beamerswitch} class.
  It allows easy switching between four \textsf{beamer} modes:

  \begin{frame}{Beamer modes}
    \begin{itemize}[<+->]
      \item \textbf{beamer:} regular slides
      \item \textbf{trans:} slides suitable for printing on transparencies
      \item \textbf{handout:} slides suitable for printing on paper
      \item \textbf{article:} transcript, paper, notes or other article-style
        document based on the slides
    \end{itemize}
  \end{frame}

  Notice how the text outside frames is only shown in article mode. Also,

  \begin{frame}{Features shown in this example}
    \begin{itemize}[<+->]
      \item Different class options are passed to the \textsf{beamer} and
        \textsf{article} classes.
      \item The `trans' and `handout' versions do not have the intermediate
        slides used by the `beamer' version for uncovering content.
      \item The handout has three slides to a page with room for handwritten
        notes at the side, and is in black and white.
    \end{itemize}

    \uncover<+->{See the source code of this example to see how it was done.}
  \end{frame}

  This PDF also has title and author information saved in the metadata (look
  at the properties in your PDF viewer).

  Happy {\LaTeX}ing!
\end{document}
%% 
%% Copyright (C) 2016-2022 by Alex Ball <ab318@bath.ac.uk>
%%
%% End of file `beamerswitch-example.tex'.
