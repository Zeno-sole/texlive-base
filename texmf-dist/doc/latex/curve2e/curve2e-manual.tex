% !TEX TS-program = pdflatex
% !TEX encoding = UTF-8 Unicode
\documentclass[a4paper]{article}\errorcontextlines=100
\hfuzz 10pt
\usepackage[utf8]{inputenc}
\usepackage{lmodern,textcomp}
\usepackage{mflogo}
\usepackage{multicol,amsmath,fancyvrb,graphics,shortvrb}
\usepackage{xcolor,verbatim,enumitem,trace}
\usepackage{curve2e}
\MakeShortVerb{\|}
\providecommand*\diff{\mathop{}\!\mathrm{d}}
\providecommand\meta{}\providecommand\marg{}
\renewcommand\meta[1]{{\normalfont\textlangle\textit{#1}\textrangle}}
\renewcommand\marg[1]{\texttt{\{\meta{#1}\}}}
\providecommand\Marg{}
\renewcommand\Marg[1]{\texttt{\{#1\}}}
\providecommand\oarg{}
\renewcommand\oarg[1]{\texttt{[\meta{#1}]}}
\providecommand\Oarg{}
\renewcommand\Oarg[1]{\texttt{[#1]}}
\providecommand\aarg{}
\renewcommand*\aarg[1]{\texttt{<\meta{#1}>}}
\providecommand\Aarg{}
\renewcommand\Aarg[1]{\texttt{<#1>}}
\providecommand\parg{}
\renewcommand\parg[1]{\texttt{(\meta{#1})}}
\providecommand\Parg{}
\renewcommand\Parg[1]{\texttt{(#1)}}
\providecommand\pack{}
\renewcommand\pack[1]{{\normalfont\textsf{#1}}}
\providecommand\file{}
\renewcommand\file[1]{{\normalfont\texttt{#1}}}
\providecommand\env{}
\renewcommand\env[1]{{\normalfont\textsf{\slshape#1}}}\let\amb\env
\providecommand\cs{}
\renewcommand\cs[1]{{\normalfont\texttt{\char92#1}}}

\def\LissajousCoefs#1,#2,#3,#4,#5,#6!{%
\edef\LAu{#1}\edef\LNu{#2}\edef\LFu{#3}%
\edef\LAd{#4}\edef\LNd{#5}\edef\LFd{#6}}

\def\LissajousCode#1#2{%
\edef\X{\fpeval{\LAu*cosd(\LNu*#1+\LFu)}}%
\edef\Y{\fpeval{\LAd*cosd(\LNd*#1+\LFd)}}%
\CopyVect\X,\Y to#2\ignorespaces}

\NewDocumentCommand\Lissajous{m o m}{%
\IfValueTF{#2}{\LissajousCoefs#2!\relax
\LissajousCode{#1}{#3}}%
{\ifcsname LAu\endcsname
\LissajousCode{#1}{#3}%
\else\PackageError{Lissajous}%
  {I parametri di questa curva di Lissajous\MessageBreak
   Non sono mai stati definiti}{Non faccio nulla}\fi}%
\ignorespaces}

\makeatletter

\NewDocumentCommand\Pall{O{1.5} R(){0,0}}{\put(#2){\circle*{#1}}}

\providecommand\setfontsize{}
\DeclareRobustCommand\setfontsize[2][1.2]{%
\linespread{#1}\fontsize{#2}{#2}\selectfont}

\newwrite\example@out
\ProvideDocumentEnvironment{Esempio}{ O{\footnotesize} D(){0.50}}
% All definitions and assignments are local
% #1 := Latex font size command or  \setfontsize{<size>}
% %2 := \textwidth percentage for the code box; the complement, less the
%       column gap for the compiled result 
%
% Syntax: \begin{Esempio}[<size command>](<code box percentage width>)
%
% Warning: No extra space is put before and/or after the display; this
%       environment is assumed to be used within a figure environment
%       that already provides its own vertical spaces. Should this
%       environment be used  between paragraphs, the user should manually
%       provide a convenient amount of space above and below it.
% Remember: This environment sets itself in vertical mode
{\par\dimendef\Wboxu=2570 \dimendef\Wboxd=2572
\Wboxu=#2\linewidth\relax
\Wboxd=\dimexpr\linewidth-\columnsep-\Wboxu\relax
\begingroup
\@bsphack
\immediate\openout\example@out\jobname-temp.tex
\let\do\@makeother\dospecials\catcode`\^^M\active
\def\verbatim@processline{%
  \immediate\write\example@out{\the\verbatim@line}}%
\verbatim@start\relax}%
{\immediate\closeout\example@out\@esphack\endgroup
\begin{lrbox}{0}%
\begin{minipage}{\linewidth}%
\begin{minipage}{\Wboxu}#1\relax
\verbatiminput{\jobname-temp.tex}
\end{minipage}%
\hfill
\begin{minipage}{\Wboxd}\raggedleft
\input{\jobname-temp}%
\end{minipage}
\end{minipage}%
\end{lrbox}\par\medskip
\noindent\makebox[\linewidth]{\box0}\par}

\providecommand\GetFileInfo[1]{%
  \def\filename{#1}%
  \def\@tempb##1 v.##2 ##3\relax##4\relax{%
    \def\filedate{##1}%
    \def\fileversion{##2}%
    \def\fileinfo{##3}}%
  \edef\@tempa{\csname ver@#1\endcsname}%
  \expandafter\@tempb\@tempa\relax? ? \relax\relax}

\begin{document}

\author{Claudio Beccari\thanks{E-mail: \texttt{claudio dot beccari at gmail dot com}}}
\title{Package \pack{curve2e} user manual}
\GetFileInfo{curve2e.sty}
\date{Version \fileversion~--~Last revised \filedate}
\maketitle
\columnseprule=0.4pt
\begin{multicols}{2}
 \tableofcontents
 \end{multicols}

\begin{abstract}\setfontsize{9.5}
This file contains the user manual of the \pack{curve2e} extension package to  the \pack{pict2e} bundle; the latter was described by Lamport himself in the 1994 second edition of his \LaTeX\ handbook.

Please take notice that on April 2011 a package \pack{pict2e} upgraded version  has been released that incorporates some of the commands defined in early versions of this package \pack{curve2e}; apparently there are no conflicts, because this package contains only the advanced features that extend the above package.

Since this extension redefines some commands and introduces some more drawing facilities (that allow to draw circular arcs and arbitrary curves with the minimum of user intervention) the user needs a user manual that contains several actual examples; this auxiliary manual is contained in file |curve2e-manual.pdf|. The software available to show the drawing code and its result after compilation is incompatible with the usual 
\LaTeX\ |ltxdoc| class for code documentation, therefore a separate user manual has been made available. If the user wants to explore the \pack{curve2e} code and its documentation s/he has available the |curve2e.pdf| file. Either file is readable by entering in a terminal window the command |texdoc curve2e-manual| or |texdoc curve2e.pdf|; please do not forget the extension when you want to read the code documentation.
\end{abstract}

%%%%%%%%%%%%%%%%%%%%%%%%%%%%%%%%%%
\section{Installation} 
%%%%%%%%%%%%%%%%%%%%%%%%%%%%%%%%%%
If \TeX\ system was installed with a \TeX~Live or a MiK\TeX\ complete and updated distribution this package is already installed; in order to verify open a terminal or command prompt window and use it to enter the \texttt{texdoc curve2e-manual}; as soon as the command is executed a screen window should open displaying this manual. If this does not happen, either you misspelt the command (it happens more often than not), or your installation is not complete and updated. 

I suggest you to use your computer installation facilities to install this bundle. Otherwise, download this curve2e.zip package from one of  the CTAN (Comprehensive TeX Archive Network) archives, to your downloads folder. 
Before doing anything else verify if you have a \emph{personal} \texttt{texmf} tree; if not, create one reading you distribution instruction; let us assume that your personal |texmf| archive is in |HOME/texmf| (on Windows change the slash with a backslash) and |HOME| is a path starting from the root of you hard disk and going though several other folders. On Windows~10 it might be |C:\Users\YourName|; on Linux it might simply be  |~|; on Mac it would be |~/\Library|, but sometimes the |HOME| might be different, especially on Windows platforms. Create the following subfolders:
\begin{enumerate}[noitemsep]
\item |HOME/texmf/source/latex/curve2e/|
\item |HOME/texmf/doc/latex/curve2e/|
\item |HOME/texmf/tex/latex/curve2e/|
\end{enumerate} 

Now move file \file{curve2e.zip} to the  |.../source/latex/curve2e/| folder; then decompress the |.zip| file with the software you have available on your platform. Run |pdflatex| on the |.dtx| file; then do the same with the |curve2e-manual.tex|. You might need to repeat these compilations two or three times in order to have the table of contents and all the references correctly connected.

This done move the |.pdf| files to the |.../doc/latex/curve2e/| folder. and move the |.sty| files to the |.../tex/latex/curve2e/| folder.

Clear the |.../source/latex/curve2e/| folder from the auxiliary files, all those remaining in the folder except those that have the extensions |.zip|, |.dtx|, |.tex|, and |.txt|. Read the \file{README.txt} file.

If your \TeX\ system is correctly set up, your files in your personal tree should be immediately usable; probably you have to create the file-name database with MiK\TeX; in this case read the documentation of your MiK\TeX\ installation to discover how to~do~it.

Remember to delete all these subfolders if you decide to install a complete updated version of your favourite distribution, and you'd better keep it updated approximately once every 7 or 10 days. This is much simpler than to struggle with these manual operations. 

%%%%%%%%%%%%%%%%%%%%%%%%%%%%%%%%%%
 \section{Package \pack{pict2e} and this extension \pack{curve2e}}
%%%%%%%%%%%%%%%%%%%%%%%%%%%%%%%%%%

 Package \pack{pict2e} was announced in issue 15 of \file{latexnews}
 around December 2003; it was declared that the new package would
 replace the dummy one that had been accompanying every release of
 \LaTeXe\ since its beginnings in 1994. The dummy package was just
 issuing an info message that simply announced the temporary
 unavailability of the real package.

 Eventually Gäßlein and Niepraschk implemented what Lamport himself had
 already documented in the second edition of his \LaTeX\ handbook, it was
 a \LaTeX\ package that contained the macros capable of removing all
 the limitations contained in the standard commands of the original
 \texttt{picture} environment; specifically what follows.
 
 \begin{enumerate}[noitemsep]
 \item The line and vector slopes were limited to the ratios of relative
 prime one-digit integers of magnitude not exceeding 6 for lines and 4
 for vectors.
%
 \item Filled and unfilled full circles were limited by the necessarily
 limited number of specific glyphs contained in the special \LaTeX\
 \texttt{picture} fonts.
%
 \item Quarter circles radii  were also limited for the same reason.
%
 \item Ovals (rectangles with rounded corners) could not be too small
 because of the unavailability of small radius quarter circles, nor
 could be too large, in the sense that after a certain radius the
 rounded corners remained the same and would not increase proportionally
 to the oval size.
%
 \item Vector tips had only one possible shape and matched the limited
 number of vector slopes.
%
 \item For circles and inclined lines and vectors just two possible
 thicknesses  were available.
 \end{enumerate}

 Package \pack{pict2e} removes most if not all the above
 limitations.
 \begin{enumerate}[noitemsep]
 \item Line and vector slopes are virtually unlimited; the only
 remaining limitation is that the direction coefficients must be
 three-digit integer numbers; they need not be relatively prime; with
 the 2009 upgrade even this limitation was removed and now slope
 coefficients can be any fractional number whose magnitude does not
 exceed 16\,384, the maximum dimension in points that \TeX\ can handle.
%
 \item Filled and unfilled circles can be of any size.
%
 \item Ovals can be designed with any specified corner curvature and
 there is virtually no limitation to such curvatures; of course corner
 radii should not exceed half the lower value between the base and the
 height of the oval.
%
 \item There are two shapes for the arrow tips; the triangular one
 traditional with \LaTeX\ vectors, or the arrow tip with PostScript
 style.
%
 \item The |\linethickness| command changes the thickness of all lines,
 straight, curved, vertical, horizontal, arrow tipped, et cetera.
 \end{enumerate}

 This specific extension package \pack{curve2e} adds the following 
 features.
 \begin{enumerate}[noitemsep]
%
 \item  Point coordinates can be specified with macros; this is similar
 to “naming” points; it eases editing the user's graphic work, because 
 points that are used several times are specified with a single macro; 
 it also eases the transmission of coordinates between different macros 
 and environments. It is also important for the following feature, i.e. 
 the user possibility to use coordinates in both cartesian or polar form.
%
 \item  Point coordinates my be specified in both cartesian and polar
 form: internally they are handled as cartesian coordinates, but the
 user can specify  his/her points also in polar form. In order to avoid
 confusion with other graphic packages, \pack{curve2e} uses the usual
 comma separated couple \meta{$x,y$} of integer or fractional numbers for
 cartesian coordinates, and the couple \meta{$\theta$}:\meta{$\rho$} for
 polar coordinates (the angle preceding the radius). 
 All graphic object commands accept polar or cartesian coordinates at
 the choice of the user who may use for each object the formalism s/he
 prefers. Also the |\put| and |\multiput| commands have been redefined so
 as to accept cartesian or polar coordinates.

 Of course the user should pay attention to the meaning of cartesian
 vs.~polar coordinates. Both imply a displacement with respect to the
 actual origin of the axes. So when a circle center  is placed at 
 coordinates $a,b$ with a normal |\put| command, the circle is placed 
 exactly in that point; with a normal |\put| command the same happens if
 coordinates $\alpha{:}\rho$ are specified.
 But if the |\put| command is nested into another |\put| command, the
 current origin of the axes is displaced — this is obvious and the
 purpose of nesting |\put| commands is exactly that. But if a segment
 is specified so that its ending point is at a specific distance and in
 specific direction form its starting point, polar coordinates appear to 
 be the most convenient to use; in this case, though, the origin of the
 axes becomes the starting point of the segment, therefore the segment
 might be drawn in a strange way. Attention has been
 paid to avoid such misinterpretation, but maybe some unusual
 situation may not have come to my mind; feedback is very welcome.
 Meanwhile pay attention when you use polar coordinates.
%
\item At user level most if not all coordinate pairs and slope pairs are
 treated as \emph{ordered pairs}, that is \emph{complex numbers}; in
 practice the user does not notice any difference from what s/he was
 used to, but all the mathematical treatment to be applied to these
 entities is coded as complex number operations, since complex numbers
 may be viewed non only as ordered pairs, but also as vectors or as
 roto-amplification operators.
%
 \item Commands for setting the line terminations were introduced; the
 user can chose between square or rounded caps; the default is set to
 rounded caps; now this feature is directly available with \pack{pict2e}.
%
 \item Commands for specifying the way two straight or curves lines 
 join to one another.
 %
 \item Originally the |\line| macro was redefined so as to allow large
 (up to three digits) integer direction coefficients, but maintaining
 the same syntax as in the original \texttt{picture} environment; now 
 \pack{pict2e} removes the integer number limitations and allows 
 fractional values, initially implemented by \pack{curve2e}; direction
 coefficients may be specified in polar form.
 %
 \item A new macro |\Line| was originally defined by \pack{curve2e} so 
 as to avoid the need to specify the horizontal projection of inclined 
 lines; now this functionality is available directly with \pack{pict2e};
 but this \pack{curve2e} macro name now conflicts with the \pack{pict2e} 
 2009 version; therefore its name is changed to |\LIne| and supposedly it
 will not be used very often, if ever, by the end user (but it is used 
 within this package macros).
 %
 \item A new macro |\LINE| was defined in order to join two points
 specified with their coordinates; this is now the normal behaviour of
 the |\Line| macro of \pack{pict2e}, so that in this package |\LINE| is
 now renamed |\segment|; there is no need to use the |\put| command with
 this segment specification.
 %
 \item A new macro |\DashLine| (alias: |\Dline|) is defined in order to
 draw dashed lines joining any two given points; the dash length and
 gap (equal to one another) get specified through one of the macro
 arguments.The starting point may be specified in cartesiano or polar
 form; the end point in cartesian format specifies the desired end
 point; but, if the second point is in polar form, it is meant
 \emph{relative to the starting point}, not as an absolute end point.
 See the examples further on.
 %
 \item A similar new macro |\Dotline| is defined in order to draw dotted 
 straight lines as a sequence of equally spaced dots, where the gap can
 be specified by the user; such straight line may have any inclination,
 as well as the above dashed lines. Polar coordinates for the second
 point have the same relative meaning as specified for the |\Dashline|
 macro. The dot diameter may be specified with an optional argument; by 
 default this diameter equals the |1pt| width.
 %
 \item Similar macros are redefined for vectors; |\vector| redefines the
 original macro but with the vector slope limitations removed and the
 vector direction may be given in polar form; |\Vector|
 gets specified with its two horizontal and vertical components in
 analogy with |\LIne|; |\VECTOR| joins two specified points (without
 using the |\put| command) with the arrow pointing to the second point. 
 |\VVECTOR| may be available if used with a sufficiently recent \LaTeX\ 
 kernel version; is draws a vector between two given points, with arrow
 tips at both ends. 

%
 \item A new macro |\polyline| for drawing polygonal lines is defined
 that accepts from two vertices up to an arbitrary (reasonably limited)
 number of them (available now also in \pack{pict2e}); here it is 
 redefined so as to allow an optional specification of the way segments 
 for the polyline are joined to one another. Vertices may be specified 
 with polar coordinates and are always relative to the preceding point.
%
 \item The \pack{pict2e} |polygon| macro draws closed polylines (in
 practice general polygons) has been redefined in such a way that it
 can accept the various vertices specified with (relative) polar
 coordinates. The |polygon*| macro produces a color filled polygon; the
 default color is black, but a different color may be specified with the
 usual |\color| command given within the same group where |\polygon*| is
 enclosed.
%
 \item A new macro |\Arc| is defined in order to draw an arc with
 arbitrary radius and arbitrary aperture (angle amplitude); this
 amplitude is specified in sexagesimal degrees, not in radians; a
 similar functionality is now achieved with the |\arc| macro of
 \pack{pict2e}, which provides also the starred version |\arc*| that 
 fills up with the current color the sector generated by a circular arc.
 It must be noticed that the syntax is slightly different, so that it's
 reasonable that these commands, in spite of producing identical arcs,
 might be more comfortable with this or that syntax.
%
 \item Two new macros |\VectorArc| and |\VectorARC| (alias |\VVectorArc|) 
 are defined in order to draw circular arcs with an arrow at one or both 
 ends.
%
 \item A new macro |\Curve| is defined so as to draw arbitrary curved
 lines by means of cubic Bézier splines; the |\Curve| macro requires
 only the curve nodes and the directions of the tangents at each
 node. The starred version fills up the interior of the curve with the
 current color.
%
 \item The above |\Curve| macro is a recursive macro that can draw an
 unlimited (reasonably limited) number of connected Bézier spline arcs
 with% continuous tangents except for cusps; these arcs require only the
 specification of the tangent direction at the interpolation nodes.
 It is possible to use a lower level macro |\CbezierTo| that does the
 same but lets the user specify the control points of each arc; it is
 more difficult to use but it is more performant.
%
 \item The basic macros used within the cumulative |\Curve| macro can be
 used individually in order to draw any curve, one cubic arc at the
 time; but they are intended for internal use, even if it is not
 prohibited to use them; by themselves such arcs are not different form
 those used by |Curve|, but the final command, |\FillCurve|, should be
 used in place of |\CurveFinish|, so as to fill up the closed path with
 the locally specified color; see figure~\ref{fig:colored-curve}. 
 It is much more convenient to use the starred version of the |\Curve|
 macro.
 \end{enumerate}

 The \pack{pict2e} package already defines macros such as |\moveto|,
 |\lineto|, |\curveto|, |\closepath|, |\fillpath|, and |\strokepath|;
 |curve2e| just redefines them so as to accept also polar coordinates;
 of course these macros can be used by the end user, and sometimes they
 perform better than the macros defined in this package, because the
 user has a better control on the position of the Bézier splines
 control points; in this case the control points are sort of rigid. It
 would be very useful to resort to the \pack{hobby} package, but its 
 macros are compatible with those of the \pack{tikz} and \pack{pgf} 
 packages, not with\pack{curve2e}; an interface should be created in 
 order to deal with  the \pack{hobby} package, but this has not been 
 done yet. 

 In order to make the necessary calculations many macros have been
 defined so as to use complex number arithmetics to manipulate point
 coordinates, directions (unit vectors, also known as `versors'),
 rotations and the like. In the first versions of this package the
 trigonometric functions were also defined in a way that the author
 believed to be more efficient than those defined by the \texttt{trig}
 package; in any case the macro names were sufficiently different to
 accommodate both definition sets in the same \LaTeX\ run. With the
 progress of the \LaTeX3 language, package \pack{xfp} has become
 available, by which any sort of calculations can be done with floating
 point decimal numbers; therefore the most common algebraic, irrational
 and transcendental functions can be computed in the background with the
 stable internal floating point facilities. We maintain some computation
 with complex number algebra, but use the |xfp| functionalities for
 other computations.

 Many aspects of this extension could be fine tuned for better
 performance; many new commands could be defined in order to further
 extend this extension. If the new service macros are accepted by other
 \TeX\ and \LaTeX\ programmers, this version could become the start for
 a real extension of the \pack{pict2e} package or even become a part of
 it. Actually some macros have already been included in the \pack{pict2e}
 package. The |\Curve| algorithm, as said before, might be redefined
 so as to use the macros  introduced by the \pack{hobby} package, that
 implements for the \pack{tikz} and \pack{pgf} packages the same 
 functionalities that John Hobby implemented for the \MF\ and \MP\ 
 programs. 

 For these reasons I suppose that every enhancement should be submitted
 to Gäßlein, Niepraschk, and Tkadlec who are the prime maintainers of
 \pack{pict2e}; they are the only ones who can decide whether or not
 to incorporate new macros in their package.

%%%%%%%%%%%%%%%%%%%%%%%%%%%%%%%%
 \section{Summary and examples of new commands}
%%%%%%%%%%%%%%%%%%%%%%%%%%%%%%%%

 This package \pack{curve2e} extends the power of \pack{pict2e} with
 the following modifications and the following new commands.
 \begin{enumerate}[noitemsep]
 \item This package \pack{curve2e} calls directly the \LaTeX\ packages
 |color| and \pack{pict2e}; it passes to the latter one any possible 
 option that it can receive; actually the only options that make sense 
 for \pack{pict2e} are those concerning the arrow tip shapes, either 
 \LaTeX\ (default) or PostScript styled, because it is assumed that if
 this package is used, the original \LaTeX\ commands are of no interest: 
 see the \pack{pict2e} documentation in order to find the correct options 
 \pack{pict2e} can receive. If the user wants to use the \pack{xcolor} 
 package, it has to load it \emph{before} \pack{curve2e}.
%%
\item Since they are used very much in the following examples, we recall some commands to label certain drawing elements, such as points, lines, arcs, and similar lines, and to insert legends in some figures. They are \cs{legenda}, \cs{Zbox} and \cs{Pbox}; their syntax is the following:
\begin{flushleft}\ttfamily\obeylines
|\legenda|\parg{coordinates}\marg{formula}
|\Zbox|\parg{coordinates}\oarg{position}\marg{formula}\oarg{dot diameter}
|\Pbox|\parg{coordinates}\oarg{position}\marg{text}\oarg{dot diameter}\meta{\texttt{*}}\aarg{angle}
\end{flushleft}
They are described in any up to date documentation of package \pack{curve2e}.
%%
 \item The user is offered new commands in order to control the line
 terminators and the line joins; specifically:
 \begin{itemize}
 \item |\roundcap|: the line is terminated with a semicircle;
 \item |\squarecap|: the line is terminated with a half square;
 \item |\roundjoin|: two lines are joined with a rounded join;
 \item |\beveljoin|: two lines are joined with a bevel join;
 \item |\miterjoin|: two lines are joined with a miter join.
 \end{itemize}
 All the above commands should respect the intended range; but since
 they act at the PostScript or PDF level, not at \TeX\ level, it might
 be necessary to issue the necessary commands in order to restore the
 previous terminator or join; in other words, groups and environments do
 not have any influence on these commands.
%%
 \item The commands |\linethickness|, |\thicklines|, |\thinlines|
 together with |\defaultlinethickness| always redefine the internal
 |\@wholewidth| and |\@halfwidth| so that the latter ones always
 refer to a full width and to a half of it in this way: if you issue
 the command {\setfontsize{8.75}|\defaultlinethickness{2pt}|} all thin 
 lines will be drawn with a thickness of 1\,pt while, if a drawing 
 command directly refers to the internal value |\@wholewidth|, its 
 line will be drawn with a thickness of 2\,pt. If one issues the
 declaration |\thinlines| all lines will be drawn with
 a 1\,pt width, but if a command refers to the internal value
 |\@halfwidth| the line will be drawn with a thickness of 0.5\,pt.
 The command |\linethickness| redefines the above internals but
 does not change the default width value; all these width
 specifications apply to all lines, straight ones, curved ones,
 circles, ovals, vectors, dashed lines, and so on. It's better
 to recall that |\thinlines| and |\thicklines| are declarations
 that do not take arguments; on the opposite the other commands
 follow the standard syntax:
 \begin{flushleft}
 |\linethickness|\marg{dimensioned value}\\
 |\defaultlinethickness|\marg{dimensioned value}
 \end{flushleft}
 where \meta{dimensioned value} means either a length specification
 complete of its units, or a dimensional expression.
%%
 \item Straight lines and vectors are redefined in such a way that
 fractional slope coefficients may be specified; the zero length
 line does not produce errors and is ignored; the zero length vectors 
 draw only the arrow tips. 
%%
 \item New line and vector macros are defined that avoid the necessity
 of specifying the horizontal component; |\put(3,4){\LIne(25,15)}|
 specifies a segment that starts at point $(3,4)$ and goes to point
 $(3+25,4+15)$; the command |\segment(3,4)(28,19)| achieves the same
 result without the need of using the |\put| command. Therefore |\LIne|
 is just for internal usage, rather than a user command. Now 
 \pack{curve2e} has available also the “arc vectors” with the arrow tips
 at one or at both  ends.

 The same applies to the vector commands |\Vector| and |\VECTOR| and
 |\VVECTOR|; the latter command behaves as |\VECTOR| but draws a vector
 with arrow tips at both ends; furthermore this command is available
 only with main versions~2 or higher of \pack{curve2e}.
 
 Experience has shown that the commands intended to join two specified
 points are particularly useful.

\begin{figure}
\begin{Esempio}[\small](0.60)
\unitlength=0.01\linewidth
\begin{picture}(80,20)
\AutoGrid
\put(0,0){\vector(1.5,2.3){10}}
\put(20,0){\Vector(10,15.33333)}
\VECTOR(40,0)(50,15.33333)
\ifdefined\VVECTOR \VVECTOR(60,0)(80,10)\fi
\end{picture}
\end{Esempio}
  \caption{Three (displaced) identical vectors obtained with the three
           vector macros\ifdefined\VVECTOR; a double tipped vector is
           also shown\fi.}\label{fig:vectors}
\end{figure}
%%
 \item The |\polyline| command has already been introduced in \pack{pict2e}:  in \pack{curve2e} it is redefined so as to accept also polar coordinates; this new version of |\polyline| accepts also an optional argument to specify how two consecutive segments join together; it accepts an unlimited list of point coordinates, possibly stored in macros, enclosed within round parentheses; the command draws a sequence of connected segments that join in order the specified points; the syntax is:
	\begin{flushleft}
\cs{polyline}\oarg{optional join style}\parg{$P_1$}\parg{$P_2$}\texttt{...}\parg{$P_n$}
	\end{flushleft}
 See figure~\ref{fig:polyline} where a regular pentagon is drawn; usage
 of polar coordinates is also shown; please notice how  polar
 coordinates act in this figure.

	\begin{figure}[!ht]
	\begin{Esempio}[\small](0.55)
	\unitlength=.5mm
	\begin{picture}(40,32)(-20,-17)
	\polyline(90:20)(162:20)(234:20)(306:20)(378:20)(90:20)
	\end{picture}
\end{Esempio}
	\caption{Polygonal line obtained by means of the \texttt{\string\polyline}
   command; vertex coordinates are in polar form.}
	\label{fig:polyline}
	\end{figure}

 Examples of using polar and cartesian coordinates are shown in 
 figure~\ref{fig:polar}.

\begin{figure}[htb]
\begin{Esempio}[\normalsize]%
\unitlength=0.02\textwidth
\begin{picture}(40,30)
\AutoGrid(40,30)
\Zbox(40,0)[l]{40,0}[1]
\Zbox(90:30)[bc]{90{:}30}[1]
\Zbox(45:30)[bc]{45{:}30}[1]
\Zbox(30,30)[bc]{30,30}[1]
\multiput(0,0)(20:10){5}%
   {\makebox(0,0){\rule{1.5mm}{1.5mm}}}
\end{picture}
\end{Esempio}
\caption[Use of cartesian and absolute polar coordinates]{Use of
 cartesian and absolute polar coordinates. The \texttt{\string\Zbox}
 macro is just a shortcut to set a small dot with a (math) legend close
 to it.}
\label{fig:polar}
\end{figure}

 A similar example may be obtained with the |\polygon| macro that does
 not require to terminate the polyline at the starting point.
 Figure~\ref{fig:polygon} shows how to get a coloured filled pentagon.

\begin{figure}[!ht]
\begin{Esempio}[\normalsize]%
\unitlength=.5mm
\begin{picture}(40,32)(-20,-20)
\color{magenta}
\polygon*(90:20)(162:20)(234:20)(306:20)(378:20)
\end{picture}
\end{Esempio}
	\caption{A pentagon obtained by means of the \texttt{\string\polygon*}
   command; vertex coordinates are in relative polar form.}
	\label{fig:polygon}
	\end{figure}
%
 \item The new command |\Dashline| (alias: |\Dline| for backwards
 compatibility):
 \begin{flushleft}
 |\Dashline|\parg{first point}\parg{second point}\marg{dash length}
 \end{flushleft}
 draws a dashed line containing as many dashes as possible, just as long
 as specified, and separated by a gap exactly the same size; actually,
 in order to make an even gap-dash sequence, the desired dash length is
 used to do some computations in order to find a suitable length, close
 to the one specified, such that the distance of the end points is
 evenly divided in equally sized dashes and gaps.
 The end points may be anywhere in the drawing area, without any
 constraint on the slope of the joining segment. The desired dash length
 is specified as a fractional multiple of |\unitlength|; see
 figure~\ref{fig:dashline}.

\begin{figure}[!ht]
\begin{Esempio}[\normalsize]%
\unitlength=1mm
\begin{picture}(40,40)
\AutoGrid(40,40)
\Dashline(0,0)(40,10){4}
\put(0,0){\circle*{2}}
\Dashline(40,10)(0,25){4}
\put(40,10){\circle*{2}}
\Dashline(0,25)(20,40){4}
\put(0,25){\circle*{2}}
\put(20,40){\circle*{2}}
\Dotline(0,0)(40,40){2}[0.75mm]
\put(40,40){\circle*{2}}
\end{picture}
\end{Esempio}
\caption{Dashed lines and graph grid}\label{fig:dashline}
\end{figure}

 Another example of usage of cartesian and polar coordinates usage is 
 shown in figure~\ref{fig:polar} together with its code.
 
\begin{figure}
\begin{Esempio}[\normalsize]%
\unitlength=0.025\textwidth
\begin{picture}(40,30)
\AutoGrid(40,30)
\Dashline(0,0)(40,10){2}
\Dashline(0,0)(40,20){2}
\Dashline(0,0)(40,30){2}
\Dashline(0,0)(30,30){2}
\Dashline(0,0)(20,30){2}
\Dashline(0,0)(10,30){2}
{\color{blue}%
\Dashline(40,0)(108:30){2}
\Dashline(40,0)(126:30){2}
\Dashline(40,0)(144:30){2}
\Dashline(40,0)(162:30){2}}
\end{picture}
\end{Esempio}
\caption{Different length dashed lines with the same nominal dash
 length; notice the relative polar coordinates used for the dashed
 lines starting at the grid lower right vertex.}
\label{fig:dashedlines}
\end{figure}
%%
\item Analogous to |\Dashline|, a new command |\Dotline| draws a dotted
 line with the syntax:
 \begin{flushleft}
 |\Dotline|\parg{first point}\parg{end point}\marg{dot gap}
 \end{flushleft}
 See figures~\ref{fig:dashline} and~\ref{fig:dottedlines} for examples.

\begin{figure}[htb]
\begin{Esempio}[\normalsize]%
\unitlength=0.025\textwidth
\begin{picture}(40,30)
\AutoGrid(40,30)
\Dotline(0,0)(40,10){1.5}[2pt]
\Dotline(0,0)(40,20){1.5}[2pt]
\Dotline(0,0)(40,30){1.5}[2pt]
\Dotline(0,0)(30,30){1.5}[2pt]
\Dotline(0,0)(20,30){1.5}[2pt]
\Dotline(0,0)(10,30){1.5}[2pt]
{\color{red}\relax
\Dotline(40,0)(108:30){1.5}
\Dotline(40,0)(126:30){1.5}[2pt]
\Dotline(40,0)(144:30){1.5}[2pt]
\Dotline(40,0)(162:30){1.5}[2pt]}%
\end{picture}
\end{Esempio}
\caption{Different length dotted lines with the same nominal dot gap;
 again notice the relative polar coordinates for the dotted lines
 starting at the grid lower right vertex.}
\label{fig:dottedlines}
\end{figure}
%%
 \item |\GraphGrid| and |\AutoGrid| are commands that draw a red grid
 under the drawing with lines separated |10\unitlength|s apart; it is 
 described only with a comma separated couple of numbers, representing
 the base and the height of the grid, see figure~\ref{fig:dashline}; it's 
 better to specify multiples of ten and the grid can be placed anywhere
 in the drawing canvas by means of |\put|, whose cartesian coordinates 
 are multiples of 10; nevertheless the grid line distance is rounded to 
 the nearest multiple of 10, while the point coordinates specified to 
 |\put| are not rounded at all; therefore some care should be used to 
 place the working grid on the drawing canvas. This grid is intended as 
 an aid while drawing; even if you sketch your drawing on millimetre 
 paper, the drawing grid turns out to be very useful; one must only 
 delete or comment out the command when the drawing is finished. Several 
 examples of usage of such grid are shown in several figures. |\Autogrid| 
 does not require arguments, but requires the canvas dimensions and 
 offsets to be specified as multiples of~10; if the latter are specified 
 they are simply ignored.
%%
 \item New trigonometric function macros have been  computed by means of
 the functionalities of the |xfp| package. The difference with the other
 existing macros is that angles are specified in sexagesimal degrees, so
 that the users need not transform to radians. The computations are done
 taking into account that “abnormal” values can occasionally be avoided,
 for example $\tan90^\circ$ must be avoided and replaced with a suitably
 large number, because the TeX\ system does not handle “infinity”.

 These trigonometric functions are used within the complex number
 macros; but if the user wants to use them the syntax is the following:
\begin{flushleft}
      \cs{SinOf}\meta{angle}\texttt{to}\meta{control sequence}\\
      \cs{CosOf}\meta{angle}\texttt{to}\meta{control sequence}\\
      \cs{TanOf}\meta{angle}\texttt{to}\meta{control sequence}
\end{flushleft}
 The \meta{control sequence} may then be used, for example, as a
 multiplying factor of a length.
 
%%
 \item Arcs can be drawn as simple circular arcs, or with one or two
 arrows at their ends (curved vectors);  the syntax is:
\begin{flushleft}
       \cs{Arc}\parg{center}\parg{starting point}\marg{angle}\\
 \cs{VectorArc}\parg{center}\parg{starting point}\marg{angle}\\
 \cs{VectorARC}\parg{center}\parg{starting point}\marg{angle}\\
\cs{VVectorArc}\parg{center}\parg{starting point}\marg{angle}\\
\end{flushleft}
 If the angle is specified numerically it must be enclosed in braces,
 while if it is specified with a control sequence the braces (curly
 brackets) are not necessary. The above macro |\Arc| draws a simple
 circular arc without arrows; |\VectorArc| draws an arc with an arrow
 tip at the ending point; |\VectorARC| (alias |\VVectorArc|) draws an 
 arc with arrow tips at both ends; see figure~\ref{fig:arcs}.
	\begin{figure}
	\begin{Esempio}[\small]%
	\unitlength=0.5mm
	\begin{picture}(60,40)
	\GraphGrid(60,40)
	\Arc(0,20)(30,0){60}
	\VECTOR(0,20)(30,0)\VECTOR(0,20)(32.5,36)
	\VectorArc(0,20)(15,10){60}
	\put(20,20){\makebox(0,0)[l]{$60^\circ$}}
	\VectorARC(60,20)(60,0){-180}
	\VVectorArc(50,20)(60,20){90}
	\end{picture}
	\end{Esempio}
	\caption{Arcs and curved vectors}\label{fig:arcs}
	\end{figure}
%%
 \item The available commands allow to create the drawings necessary to prove some geometrical theorems; for example let us prove the Pitagora's theorem. Figure~\ref{fig:pitagora} displays a right triangle with its hypothenuse laying horizontal; its vertices are labelled \textsf{A}, \textsf{B}, and \textsf{C}, being \textsf{A} the right angle vertex. The height relative to the hypothenuse intersects this side in point \textsf{H}, and divides the whole triangle \textsf{ABC},  in two similar smaller ones \textsf{AHC}, and\textsf{ABH}. Segments \textsf{CH}, and\textsf{HB}, add to the whole hypothenuse length. Figure~\ref{fig:pitagora} displays also the square \textsf{CBDE}, built on the hypothenuse and divided in two rectangles \textsf{CHFE}, and \textsf{HBDF}, where \textsf{HF} is the continuation of the height line.

The lengths of the original right triangle sides are $a, b, c$ as marked on the figure; Point \textsf{H} divides the hypothenuse of length $c$ in two shorter segments of lengths $d, e$, respectively,  as marked in the figure
  
Since the three triangles are similar, we can set up the relationships between their sides:
\begin{align*}
d : a = a : c &\Longrightarrow d = a^2/c\\
e : b = b : c &\Longrightarrow e = b^2/c
\end{align*}

Therefore rectangle \textsf{CHFE} area equals $ c\cdot d = c\cdot a^2/c = a^2$. Similarly rectangle \textsf{HBDF} area equals $ c\cdot e = c\cdot b^2/c = b^2$. The square built on the hypothenuse has an area equal to $c^2$; therefore it is 
\begin{equation*}
c^2 = a^2 + b^2
\end{equation*}
which proves the Pitagora's theorem.

\begin{figure}
\centering
\begin{Esempio}
\unitlength=1mm
\begin{picture}(70,80)(-10,-50)
\AutoGrid
\thicklines
\polygon(0,0)(50,0)(18,24)
\thinlines
\polyline(0,0)(0,-50)(50,-50)(50,0)
\Pbox(0,0)[r]{C}[0.75ex]
\Pbox(18,0)[tr]{H}[0.75ex]
\Pbox(50,0)[l]{B}[0.75ex]
\Pbox(18,24)[b]{A}[0.75ex]
\Pbox(0,-50)[br]{E}[0.75ex]
\Pbox(50,-50)[bl]{D}[0.75ex]
\Pbox(18,-50)[br]{F}[0.75ex]
\Zbox(10,12)[br]{a}[0]
\Zbox(33,12)[bl]{b}[0]
\Zbox(25,0)[t]{c}[0]
\Zbox(9,0)[b]{d}[0]
\Zbox(30,0)[b]{e}[0]
\Zbox(50,-25)[r]{c}[0]
\Zbox(25,-50)[b]{c}[0]
\Zbox(0,-25)[l]{c}[0]
\Dashline(18,24)(18,-50){1.5}
\end{picture}
\end{Esempio}
\caption{Geometrical construction to prove Pitagora's theorem}
\label{fig:pitagora}
\end{figure}
 
 \item A multitude of commands have been defined in order to manage
 complex numbers; actually complex numbers are represented as a comma
 separated pair of fractional numbers; internally these macros use 
 only the cartesian form, and output, unless 
 differently specified is also in cartesian form; but input can be in 
 polar form. They are used to address specific points in the drawing
 plane, but also as operators so as to scale and rotate other objects.
 In the following \meta{vector} means a comma separated pair of
 fractional numbers, \meta{vector macro} means a macro that contains a
 comma separated pair of fractional numbers; \meta{angle macro} means a
 macro that contains the angle of a vector in sexagesimal degrees;
 \meta{argument} means a brace delimited numeric value, even a macro;
 \meta{numeric macro} means a macro that contains a fractional number;
 \textit{macro} is a valid macro name, i.e.~a backslash followed by
 letters, or anything else that can receive a definition. A
 \emph{direction} of a vector is its versor; the angle of a vector is
 the angle between the vector and the positive $x$ axis in
 counterclockwise direction, as it is used in the
 Euler formula $ \vec{v} = Me^{\mathrm{j}\varphi}$.

 \begin{itemize}
 \item |\MakeVectorFrom|\meta{numeric macro}\meta{numeric macro}|to|\meta{vector macro}
 \item |\CopyVect|\meta{first vector}|to|\meta{second vector macro}
 \item |\ModOfVect|\meta{vector}|to|\meta{modulus macro}
 \item |\DirOfvect|\meta{vector}|to|\meta{versor macro}
 \item |\ModAndDirOfVect|\meta{vector}|to|\meta{modulus macro}|and|\meta{versor macro}
 \item |\ModAndAngleOfVect|\meta{vector}|to|\meta{modulus macro}|and|\meta{angle macro}
 \item {\setfontsize{10}|\DistanceAndDirOfVect|\meta{1st vector} |minus|\meta{2nd vector} |to|\meta{distance macro} |and|\meta{versor macro}}
 \item |\XpartOfVect|\meta{vector}|to|\meta{macro}
 \item |\YpartOfVect|\meta{vector}|to|\meta{macro}
 \item |\DirFromAngle|\meta{angle}|to|\meta{versor macro}
 \item |\ArgOfVect|\meta{vector}|to|\meta{angle macro}
 \item |\ScaleVect|\meta{vector}|by|\meta{scaling factor}|to|\meta{vector macro}
 \item |\ConjVect|\meta{vector}|to|\meta{conjugate vector macro}
 \item |\SubVect|\meta{subtrahend vector}|from|\meta{minuend vector}|to|\meta{vector macro}
 \item |\AddVect|\meta{first vector}|and|\meta{second vector}|to|\meta{vector macro}
 \item |\Multvect|\marg{first vector}*\marg{second vector}*\marg{vector macro} (the asterisks are optional; either one changes the second vector into its complex conjugate)
 \item |\MultVect|\meta{first vector}|by|\meta{second vector}|to|\meta{vector macro} (discouraged; maintained for backwards compatibility)
 \item |\MultVect|\meta{first vector}|by*|\meta{second vector}|to|\meta{vector macro} (discouraged; maintained for backwards compatibility)
 \item |\Divvect|\marg{dividend vector}\marg{divisor vector}\marg{ vector macro}
 \item |\DivVect|\meta{dividend vector}|by|\meta{divisor vector}|to|\meta{vector macro} (maintained for backwards compatibility)
 \end{itemize}
%%
 \item General curves can be drawn with the \pack{pict2e} macro 
 |\curve| but it requires the specification of the third-order 
 Bézier-spline control points; sometimes it's better to be very 
 specific with the control points and there is no other means to 
 do a decent graph; sometimes the curves to be drawn are not so 
 tricky and a general set of macros can be defined so as to 
 compute the control points, while letting the user specify only 
 the nodes through which the curve must pass, and the tangent 
 direction of the curve in such nodes. Such commands are the following:
\begin{itemize}[noitemsep]

\item \cs{Curve}  draws a sequence of arcs as explained above, using
 third order (cubic) Bézier splines. The starred version of this command
 fills the internal part of the curve with the current color; if the
 last arc finishes where the fist arc starts, it is clear what is the
 interior; if it does not, the driver (not the code of this package,
 but the driver between this code and the physical representation on
 paper or screen) assumes a straight line closure of the whole path.

\item \cs{Qurve} is similar to |\Curve|, but with second order 
(quadratic) Bézier splines. The starred version fills the interior 
with the current color.

\item \cs{CurveBetween} draws a single cubic Bézier spline between two
 given nodes and with two given direction vectors.

\item \cs{CBezierBetween}  draws a single cubic Bézier spline between
 two given nodes, with two given direction versors along which the
 control node distances are specified. This is the most general macro
 (rather difficult to use) with which not only the arc end points are
 specified but also the control nodes coordinates are given.

\end{itemize}

% le equazioni parametriche del cuore, da stackexchange, sono
%\def\x(#1){sin(#1)^3}
%\def\y(#1){(13*cos(#1)-5*cos(2*#1)-2*cos(3*#1)-cos(4*#1))/16}
% Varrebbe a penda di provarle con curve2e
%
	\begin{figure}[!htbp]
\begin{Esempio}[\small]%
	\unitlength=8mm\relax
	\begin{picture}(5,5)
	\put(0,0){\framebox(5,5){}}\thicklines\roundcap
	\Curve(2.5,0)<0.1,1>(5,3.5)<0,1>%
	  (4,5)<-1,0>(2.5,3.5)<-0.1,-1.2>[-0.1,1.2]%
	  (1,5)<-1,0>(0,3.5)<0,-1>(2.5,0)<0.1,-1>
	\end{picture}
\end{Esempio}
	\caption{A heart shaped curve with cusps drawn with \texttt{\string\Curve}}
	\label{fig:curve}

\vspace*{2\baselineskip}

\begin{Esempio}[\small]%
	\unitlength=8mm\relax
	\begin{picture}(5,5)
	\put(0,0){\framebox(5,5){}}\thicklines\roundcap
   \color{green}\relax
	\Curve*(2.5,0)<0.1,1>(5,3.5)<0,1>%
	  (4,5)<-1,0>(2.5,3.5)<-0.1,-1.2>[-0.1,1.2]%
	  (1,5)<-1,0>(0,3.5)<0,-1>(2.5,0)<0.1,-1>
	\end{picture}
\end{Esempio}
\caption{Coloring the inside of a closed path drawn with \texttt{\string\Curve*}}
\label{fig:colored-curve}
	\end{figure}

 The main macro is |\Curve| and must be followed by an “unlimited”
 sequence of node-direction coordinates as a quadruple
 defined as
\[
 \parg{node coordinates}\aarg{direction vector}
\]
 Possibly if a sudden change of direction has to be performed (cusp)
 another item can be inserted after one of those quadruples in the form
\[
 \mbox{\dots\parg{...}\aarg{...}\oarg{new direction vector}\parg{...}\aarg{...}\dots}
\]

 Sometimes it is necessary to specify the “tension” or the “looseness”
 of a specific Bézier arc; such tension parameters range from 0 (zero)
 to~4; the zero value implies a very stiff arc, as if it was a string
 subject to a high tension (i.e. with zero looseness); a value of~4
 implies a very low tension (very high looseness), almost as if the
 string was not subject to any tension. In \MF\ or \MP\ language such a
 concept is used very often; in this package, where the Hobby
 algorithms are not used, the parameter value appears to mean the
 opposite of tension.
 A couple of comma separated tension values may be optionally used, they
 are separated with a semicolon from the direction vector,
 and they apply to the arc terminating with the last node; their
 specification  must precede any possible change of tangent according to
 this syntax\footnote{The tension may be specified only for cubic
 splines, because the quadratic ones do not use enough parameters to
 control the tension; not all commands for drawing cubic splines accept
 this optional tension specification.}:
\[
\mbox{\dots\parg{...}\Aarg{\meta{direction vector};\meta{start tension},\meta{end tension}}\parg{...}\aarg{...}\dots}
\]

 The |\Curve| macro does not (still) have facilities for cycling the
 path, that is to close the path from the last specified node-direction
 to the first specified node-direction; but, as already mentioned, if
 the ending node of the last arc does not coincide with the starting
 node of the first arc, a straight line is assumed to join such nodes;
 this line does not get drawn, but with starred commands no lines are
 drawn because only the interior is coloured.
 The tangent direction need not be specified with a unit vector,
 although only its direction is relevant; the scaling of the specified
 direction vector to a unit vector is performed by the macro itself.
 Therefore one cannot specify the fine tuning of the curve convexity as
 it can be done with other programs or commands, as, for example, with
 \MF\  or the |pgf/tikz| package and environment.
 See figure~\ref{fig:curve} for an example.

 With the starred version of |\Curve|, instead of stroking the contour,
 the macro fills up the contour with the selected current color,
 figure~\ref{fig:colored-curve}.

 Figure~\ref{fig:arcspline} shows a geometric construction that
 contains the geometric elements and symbols used to determine the
 parameters of a cubic spline required to draw a quarter circle. This
 construction contains many of the commands described so far.

\begin{figure}[p]
\begin{minipage}{\linewidth}\small
\begin{verbatim}
\unitlength=0.007\textwidth
\begin{picture}(100,90)(-50,-50)
\put(-50,0){\vector(1,0){100}}\put(50,1){\makebox(0,0)[br]{$x$}}%
\put(20,-1){\makebox(0,0)[t]{$s$}}%
\put(0,0){\circle*{2}}\put(-1,-1){\makebox(0,0)[tr]{$M$}}%
\legenda(12,-45){s=\overline{MP_2}=R\sin\theta}%
\put(0,-50){\vector(0,1){90}}%
\put(1,40){\makebox(0,0)[tl]{$y$}}%
\put(0,-40){\circle*{2}}\put(1,-41){\makebox(0,0)[lt]{$C$}}%
\segment(0,-40)(-40,0)\segment(0,-40)(40,0)%
\put(-41,1){\makebox(0,0)[br]{$P_1$}}\put(-40,0){\circle*{2}}%
\put(41,1){\makebox(0,0)[bl]{$P_2$}}\put(40,0){\circle*{2}}%
\put(0,0){\linethickness{1pt}\Arc(0,-40)(40,0){90}}%
\segment(-40,0)(-20,20)\put(-20,20){\circle*{2}}%
\put(-20,21.5){\makebox(0,0)[b]{$C_1$}}%
\segment(40,0)(20,20)\put(20,20){\circle*{2}}%
\put(20,21.5){\makebox(0,0)[b]{$C_2$}}%
\put(0,-40){\put(0,56.5685){\circle*{2}}%
\put(1,58){\makebox(0,0)[bl]{$P$}}}%
\VectorARC(0,-40)(15,-25){45}\put(10,-18){\makebox(0,0)[c]{$\theta$}}%
\VectorARC(40,0)(20,0){-45}\put(19,5){\makebox(0,0)[r]{$\theta$}}%
\VectorARC(-40,0)(-20,0){45}\put(-19,5){\makebox(0,0)[l]{$\theta$}}%
\put(-20,-18){\makebox(0,0)[bl]{$R$}}%
\put(-32,13){\makebox(0,0)[bl]{$K$}}%
\put(32,13){\makebox(0,0)[br]{$K$}}%
\end{picture}
\end{verbatim}
\end{minipage}\vspace{\stretch{1}}

\begin{minipage}{\linewidth}\centering
\unitlength=0.007\textwidth
\begin{picture}(100,90)(-50,-50)
\put(-50,0){\vector(1,0){100}}\put(50,1){\makebox(0,0)[br]{$x$}}%
\put(20,-1){\makebox(0,0)[t]{$s$}}%
\put(0,0){\circle*{2}}\put(-1,-1){\makebox(0,0)[tr]{$M$}}%
\legenda(12,-45){s=\overline{MP_2}=R\sin\theta}%
\put(0,-50){\vector(0,1){90}}%
\put(1,40){\makebox(0,0)[tl]{$y$}}%
\put(0,-40){\circle*{2}}\put(1,-41){\makebox(0,0)[lt]{$C$}}%
\segment(0,-40)(-40,0)\segment(0,-40)(40,0)%
\put(-41,1){\makebox(0,0)[br]{$P_1$}}\put(-40,0){\circle*{2}}%
\put(41,1){\makebox(0,0)[bl]{$P_2$}}\put(40,0){\circle*{2}}%
\put(0,0){\linethickness{1pt}\Arc(0,-40)(40,0){90}}%
\segment(-40,0)(-20,20)\put(-20,20){\circle*{2}}%
\put(-20,21.5){\makebox(0,0)[b]{$C_1$}}%
\segment(40,0)(20,20)\put(20,20){\circle*{2}}%
\put(20,21.5){\makebox(0,0)[b]{$C_2$}}%
\put(0,-40){\put(0,56.5685){\circle*{2}}%
\put(1,58){\makebox(0,0)[bl]{$P$}}}%
\VectorARC(0,-40)(15,-25){45}\put(10,-18){\makebox(0,0)[c]{$\theta$}}%
\VectorARC(40,0)(20,0){-45}\put(19,5){\makebox(0,0)[r]{$\theta$}}%
\VectorARC(-40,0)(-20,0){45}\put(-19,5){\makebox(0,0)[l]{$\theta$}}%
\put(-20,-18){\makebox(0,0)[bl]{$R$}}%
\put(-32,13){\makebox(0,0)[bl]{$K$}}%
\put(32,13){\makebox(0,0)[br]{$K$}}%
\end{picture}
\end{minipage}
\caption{The code to display the nodes and control points for an arc to
 be approximated with a cubic Bézier spline}
\label{fig:arcspline}
\end{figure}


 To show what you can do with |\CurveBetween| see the code and result
 shown in figure~\ref{fig:curva-due-punti}. Notice the effect of
 changing the directions at both or a the end nodes of a single cubic
 spline. The directions are conveniently expressed with unit vectors
 described by polar coordinates. The |\CurveBetween| macro is built on |\CbezierTp| ; this ltter file is very complicated to describe and its use is reserved to experienced users; its syntax is described in the code documentation file |cureve2e.pdf|; the reader is encouraged to examine it in case s/he is willing to use it.

\begin{figure}\centering\unitlength=0.004\textwidth
\begin{picture}(220,120)(-50,-20)
\put(0,60){\Line(-50,0)(50,0)
\CurveBetween-50,0and50,0WithDirs15:1and{-15:1}
\CurveBetween-50,0and50,0WithDirs30:1and{-30:1}
\CurveBetween-50,0and50,0WithDirs45:1and{-45:1}
\CurveBetween-50,0and50,0WithDirs60:1and{-60:1}
\CurveBetween-50,0and50,0WithDirs75:1and{-75:1}
\CurveBetween-50,0and50,0WithDirs90:1and{-90:1}}
\put(120,60){%
\Line(-50,0)(50,0)
\CurveBetween-50,0and50,0WithDirs15:1and{15:1}
\CurveBetween-50,0and50,0WithDirs30:1and{30:1}
\CurveBetween-50,0and50,0WithDirs45:1and{45:1}
\CurveBetween-50,0and50,0WithDirs60:1and{60:1}
\CurveBetween-50,0and50,0WithDirs75:1and{75:1}
\CurveBetween-50,0and50,0WithDirs90:1and{90:1}}
\put(0,0){%
\Line(-50,0)(50,0)
\CurveBetween-50,0and50,0WithDirs45:1and{-15:1}
\CurveBetween-50,0and50,0WithDirs45:1and{-30:1}
\CurveBetween-50,0and50,0WithDirs45:1and{-45:1}
\CurveBetween-50,0and50,0WithDirs45:1and{-60:1}
\CurveBetween-50,0and50,0WithDirs45:1and{-75:1}
\CurveBetween-50,0and50,0WithDirs45:1and{-90:1}}
\put(120,0){%
\Line(-50,0)(50,0)
\CurveBetween-50,0and50,0WithDirs45:1and{15:1}
\CurveBetween-50,0and50,0WithDirs45:1and{30:1}
\CurveBetween-50,0and50,0WithDirs45:1and{45:1}
\CurveBetween-50,0and50,0WithDirs45:1and{60:1}
\CurveBetween-50,0and50,0WithDirs45:1and{75:1}
\CurveBetween-50,0and50,0WithDirs45:1and{90:1}}
\end{picture}
\caption{Curves between two points with different start and end slopes}\label{fig:curva-due-punti}
\end{figure}

 A little more complicated is the use of the |\CBezierBetween| macro,
 figure~\ref{fig:Cbezier}. The directions are specified with unit
 vectors in polar form; the control points are specified by adding their
 distances from their neighbouring nodes; actually the right distance
 is maintained to the value~1, while the left one increases from~4
 to~10.
 The black line corresponds to the standard |\CurveBetween| where the
 default distance is computed to trace an arc of a circle and
 is approximately~3.5. 

\begin{figure}[!tb]
\begin{minipage}[t]{0.52\textwidth}\small
\begin{verbatim}
\unitlength=0.1\textwidth
\begin{picture}(10,3)
\CurveBetween0,0and10,0WithDirs1,1and{1,-1}
\color{red}%
\CbezierBetween0,0And10,0 WithDirs45:1And-45:1UsingDists4And{1}
\CbezierBetween0,0And10,0 WithDirs45:1And-45:1UsingDists6And{1}
\CbezierBetween0,0And10,0 WithDirs45:1And-45:1UsingDists8And{1}
\CbezierBetween0,0And10,0 WithDirs45:1And-45:1UsingDists10And{1}
\CbezierBetween0,0And10,0 WithDirs45:1And-45:1UsingDists12And{1}
\end{picture}
\end{verbatim}
\end{minipage}
\hfill
\begin{minipage}{0.40\textwidth}\raggedleft
\unitlength=0.1\textwidth
\begin{picture}(10,3)(0,1.25)
\CurveBetween0,0and10,0WithDirs1,1and{1,-1}
\color{red}%
\CbezierBetween0,0And10,0 WithDirs45:1And-45:1UsingDists4And{1}
\CbezierBetween0,0And10,0 WithDirs45:1And-45:1UsingDists6And{1}
\CbezierBetween0,0And10,0 WithDirs45:1And-45:1UsingDists8And{1}
\CbezierBetween0,0And10,0 WithDirs45:1And-45:1UsingDists10And{1}
\CbezierBetween0,0And10,0 WithDirs45:1And-45:1UsingDists12And{1}
\end{picture}
\end{minipage}
\caption{Comparison between similar arcs drawn with \cs{CurveBetween} with the default tension values (black)
 and with several tension values specified (red)}
\label{fig:Cbezier}
\end{figure}

 In figure~\ref{fig:tensions} the effect of tension specification is
 shown. The red line corresponds to the default tension, since the
 tension values are not specified. The black lines correspond to the
 various values used in the various commands to the |\Curve| macro.
 With a tension of zero, the spline is almost coincident with the
 horizontal base line of the frame. Increasing the parameter value
 to~4.5, the curved becomes taller and taller, until it wraps itself
 displaying an evident loop. We would say that the value of ~2 is a
 reasonable  maximum one and that increasing that value is just to
 obtain special effects.

\begin{figure}[!htb]\centering
\begin{Esempio}[\setfontsize{8.5}]%
\unitlength=0.01\textwidth
\begin{picture}(70,70)
\put(0,0){\color{blue}\framebox(70,70){}}
\put(0,0){\color{red}%
  \Curve(0,0)<1,1>(70,0)<1,-1>}
\Curve(0,0)<1,1>(70,0)<1,-1;0,0>
\Curve(0,0)<1,1>(70,0)<1,-1;0.2,0.2>
\Curve(0,0)<1,1>(70,0)<1,-1;2,2>
\Curve(0,0)<1,1>(70,0)<1,-1;4.5,4.5>
\Curve(0,0)<1,1>(70,0)<1,-1;0,3>
\Curve(0,0)<1,1>(70,0)<1,-1;3,0>
\end{picture}
\end{Esempio}
\caption{The effects of tension factors}\label{fig:tensions}
\end{figure}

 Figure~\ref{fig:sinewave} displays two approximations of a sine wave;
 Bézier splines can approximate transcendental curves, but the
 approximation may be a poor one, depending on the approximated curve,
 when few arcs are used to draw it. With arcs specified with more
 complicated macros the approximation is better even with a lower number
 of arcs.  With many arcs it is possible to approximate almost anything.
 On the left side a modest approximation is obtained with just three
 standard arcs obtained with |\Curve| and four node specifications;
 on the right we have just two arcs created with |CBezierBetween| 
 with tension specification and control point distances; this drawing 
 is almost undistinguishable from a real sinusoid.

\begin{figure}[!htb]
\begin{minipage}{\linewidth}\small
\begin{verbatim}
\unitlength=0.01\textwidth
\begin{picture}(100,50)(0,-25)
\put(0,0){%
\VECTOR(0,0)(45,0)\VECTOR(0,-25)(0,25)
\Pbox(45,0)[b]{x}[0]\Pbox(0,26)[tl]{y[0]}
\Curve(0,0)<77:1>(10,20)<1,0;2,0.4>(30,-20)<1,0;0.4,0.4>(40,0)<77:1;0.4,2>
}
\put(55,0){%
\VECTOR(0,0)(45,0)\VECTOR(0,-25)(0,25)
\Pbox(45,0)[b]{x}[0]\Pbox(0,26)[tl]{y}[0]
\CbezierBetween0,0And20,0WithDirs77:1And-77:1UsingDists28And{28}
\CbezierBetween20,0And40,0WithDirs-77:1And77:1UsingDists28And{28}}
\end{picture}
\end{verbatim}
\end{minipage}\vspace{\baselineskip}

\begin{minipage}{\linewidth}
\unitlength=0.01\textwidth
\begin{picture}(100,50)(0,-25)
\put(0,0){\VECTOR(0,0)(45,0)\VECTOR(0,-25)(0,25)
\Pbox(45,0)[b]{x}[0]\Pbox(0,26)[tl]{y}[0]
\Curve(0,0)<77:1>(10,20)<1,0;2,0.4>(30,-20)<1,0;0.4,0.4>(40,0)<77:1;0.4,2>
}
\put(55,0){\VECTOR(0,0)(45,0)\VECTOR(0,-25)(0,25)
\Pbox(45,0)[b]{x}[0]\Pbox(0,26)[tl]{y}[0]
\CbezierBetween0,0And20,0WithDirs77:1And-77:1UsingDists28And{28}
\CbezierBetween20,0And40,0WithDirs-77:1And77:1UsingDists28And{28}}
\end{picture}
\end{minipage}
\caption{A sequence of arcs; the left figure has been drawn with the
 \cs{Curve} command with a sequence of four couples of point-direction
 arguments; the right figure has been drawn with two commands
 \cs{CbezierBetween} that include also the specification of the control
 points}
\label{fig:sinewave}
\end{figure}

 In figure~\ref{fig:quadratic-arcs} some lines are shown; they are
 drawn  with quadratic splines by means of the |\Qurve| macro. In the
 left there are some open and closed curves inscribed within a square.
 On the right a “real" circle is compared to a quadratic spline circle;
 the word “real” is emphasised because it actually is an approximation
 with four quarter-circle cubic splines that, in spite of being drawn
 with third degree parametric polynomials, approximate very well a real
 circle; on the opposite the quadratic spline circle is clearly a poor
 approximation even if the maximum radial error amounts just to about
 6\% of the radius.

\begin{figure}[!htb]
\begin{minipage}{\linewidth}
\begin{Verbatim}[fontsize=\setfontsize{7.75}]
\unitlength=0.0045\textwidth
\begin{picture}(100,100)
\put(0,0){\framebox(100,100){}}
\put(50,50){%
  \Qurve(0,-50)<1,0>(50,0)<0,1>(0,50)<-1,0>(-50,0)<0,-1>(0,-50)<1,0>
\color{green}
  \Qurve*(0,-50)<0,1>(50,0)<1,0>[-1,0](0,50)<0,1>[0,-1](-50,0)<-1,0>[1,0](0,-50)<0,-1> 
}
\Qurve(0,0)<1,4>(50,50)<1,0>(100,100)<1,4>
\put(5,50){\Qurve(0,0)<1,1.5>(22.5,20)<1,0>(45,0)<1,-1.5>%
(67.5,-20)<1,0>(90,0)<1,1.5>}
\Zbox(0,0)[tc]{0,0}\Zbox(100,0)[tc]{100,0}
\Zbox(100,100)[bc]{100,100}\Zbox(0,100)[bc]{0,100}
\Pall[2](0,0)\Pall[2](100,0)\Pall[2](100,100)\Pall[2](0,100)
\end{picture}
\hfill
\begin{picture}(100,100)
\put(0,0){\framebox(100,100){}}
\put(50,50){%
\Qurve(0,-50)<1,0>(50,0)<0,1>(0,50)<-1,0>(-50,0)<0,-1>(0,-50)<1,0>
\Curve(0,-50)<1,0>(50,0)<0,1>(0,50)<-1,0>(-50,0)<0,-1>(0,-50)<1,0>}
\Zbox(50,50)[t]{O}\Pall[2](50,50)\put(50,50){\Vector(45:50)}\Zbox(67,70)[tl]{R}
\end{picture}
\end{Verbatim}
\end{minipage}\vspace{2\baselineskip}

\begin{minipage}{\linewidth}
\unitlength=0.0045\textwidth
\begin{picture}(100,100)
\put(0,0){\framebox(100,100){}}
\put(50,50){\Qurve(0,-50)<1,0>(50,0)<0,1>(0,50)<-1,0>(-50,0)<0,-1>(0,-50)<1,0>}
\put(50,50){\color{green}%
\Qurve*(0,-50)<0,1>(50,0)<1,0>[-1,0](0,50)<0,1>[0,-1](-50,0)<-1,0>[1,0](0,-50)<0,-1>}
\Qurve(0,0)<1,4>(50,50)<1,0>(100,100)<1,4>
\put(5,50){\Qurve(0,0)<1,1.5>(22.5,20)<1,0>(45,0)<1,-1.5>(67.5,-20)<1,0>(90,0)<1,1.5>}
\Zbox(0,0)[tc]{0,0}\Zbox(100,0)[tc]{100,0}
\Zbox(100,100)[bc]{100,100}\Zbox(0,100)[bc]{0,100}
\Pall[2](0,0)\Pall[2](100,0)\Pall[2](100,100)\Pall[2](0,100)
\end{picture}
\hfill
\begin{picture}(100,100)
\put(0,0){\framebox(100,100){}}
\put(50,50){\Qurve(0,-50)<1,0>(50,0)<0,1>(0,50)<-1,0>(-50,0)<0,-1>(0,-50)<1,0>
\Curve(0,-50)<1,0>(50,0)<0,1>(0,50)<-1,0>(-50,0)<0,-1>(0,-50)<1,0>}
\Zbox(50,50)[t]{O}\Pall[2](50,50)\put(50,50){\Vector(45:50)}\Zbox(67,70)[tl]{R}
\end{picture}
\end{minipage}

\caption{\rule{0pt}{4ex}Several graphs drawn with quadratic Bézier
 splines. On the right a quadratic spline circle is compared with a
 cubic line circle.}
\label{fig:quadratic-arcs}
\end{figure}

 Notice that the previous version of \pack{curve2e} contained an error 
 and would color the outside of the green four-pointed star. 
 The |curve2e-v161| package, attached to this bundle, has been corrected;
 therefore it is not actually identical to the previous version,
 although the latter one performed correctly for everything else except
 for color-filled quadratic paths.
 %%
 \item The new version of |\multiput| is backwards compatibile with
 the original version contained in the \LaTeX\ kernel. The new macro
 adds the handling of the coordinate increments from one position to
 the next for the \meta{object} to include in the drawing.

 On page~\pageref{pag:multiput} we show the code for the figure shown
 there. The red grid is nothing new, except that it displays the
 traditional |\multiput| used in this code, shown in a previous example,
 produces exactly the same result. But for what concerns the four
 “graphs” on the grid, it displays an alignment of black dots along the
 diagonal of the grid (again traditional |\multiput| rendered with the 
 new version); a number of blue dots along a parabola; another number of 
 magenta coloured dots alined along a half sine wave; a number of little 
 green squares aligned along a $-15~\circ$ sloping line starting from 
 the center of the grid; notice the polar values that are used as polar 
 relative coordinate increments.

\noindent\begin{figure}[!htb]
\begin{minipage}{0.45\linewidth}
\begin{Verbatim}[fontsize=\setfontsize{8}]
\unitlength=0.01\linewidth
\begin{picture}(100,100)
\GraphGrid(100,100)
\multiput(0,0)(10,10){11}{\circle*{2}}
\color{blue!70!white}
\multiput(0,0)(10,0){11}{\circle*{2}}%
  [\GetCoord(\R)\X\Y
  \edef\X{\fpeval{\X+10}}
  \edef\Y{\fpeval{(\X/10)**2}}
  \CopyVect\X,\Y to\R]
\color{magenta}
\multiput(0,0)(10,1){11}{\circle*{2}}%
  [\GetCoord(\R)\X\Y
   \edef\X{\fpeval{\X+10}}
   \edef\Y{\fpeval{sind(\X*1.8)*100}}
   \CopyVect\X\Y to\R]
\color{green!80!black}
\multiput(50,50)(-15:5){11}}{%
\polygon*(-1,-1)(1,-1)(1,1)(-1,1)}
\end{picture}
\end{Verbatim}
\end{minipage}
\hfill
\begin{minipage}{0.45\linewidth}
\unitlength=0.01\linewidth
\begin{picture}(100,100)
\GraphGrid(100,100)
\multiput(0,0)(10,10){11}{\circle*{2}}
\color{blue!70!white}
\multiput(0,0)(10,0){11}{\circle*{2}}%
  [\GetCoord(\R)\X\Y
  \edef\X{\fpeval{\X+10}}
  \edef\Y{\fpeval{(\X/10)**2}}
  \CopyVect\X,\Y to\R]
\color{magenta}
\multiput(0,0)(10,1){11}{\circle*{2}}%
  [\GetCoord(\R)\X\Y
   \edef\X{\fpeval{\X+10}}
   \edef\Y{\fpeval{sind(\X*1.8)*100}}
   \CopyVect\X,\Y to\R]
\color{green!60!black}
\multiput(50,50)(-15:5){11}{\polygon*(-1,-1)(1,-1)(1,1)(-1,1)}
\end{picture}
\end{minipage}
\caption{Some examples of the \meta{handler} optional argument}\label{pag:multiput}
\end{figure}

 A new command |\xmultiput| (not available with the previous versions
 of \pack{curve2e}) extended with respect to the original |\multiput| is
 defined by using some L3 functions; in particular the cycling
 counter is accessible to the \LaTeX\ commands and it is stepped
 up from~1 to the value specified in the proper command argument
 (in the original command it starts from that value and is stepped down
 to zero). See the figure on page~\ref{pag:orologio} to inspect its
 usage. It is important to notice that if the command|\rotatebox|
 has to be used, as in the example of figure~\ref{pag:orologio}, the
 package |graphicx| should be also loaded, because \pack{curve2e} does 
 not load it.

\begin{figure}[!htb]
\begin{minipage}{0.45\textwidth}\setfontsize{9.5}%
\begin{verbatim}
\unitlength=0.0095\linewidth
\begin{picture}(100,100)
\GraphGrid(100,100)
\put(50,50){\thicklines\circle{100}}
\xmultiput[50,50](60:40)(-30:1){12}%
  {\makebox(0,0){\circle*{2}}}%
  [\MultVect\R by\D to\R]% 
\xmultiput[50,50](60:46)(-30:1){12}%
  {\ArgOfVect\R to\Ang
  \rotatebox{\fpeval{\Ang-90}}%
  {\makebox(0,0)[b]{\Roman{multicnt}}}}%
  [\Multvect{\R}{\D}\R]
\end{picture}
\end{verbatim}
\end{minipage}
\hfill
\begin{minipage}{0.40\textwidth}\raggedleft
\unitlength=0.0095\linewidth
\begin{picture}(100,100)
\GraphGrid(100,100)
\put(50,50){\thicklines\circle{100}}
\xmultiput[50,50](60:40)(-30:1){12}%
  {\makebox(0,0){\circle*{2}}}[\MultVect\R by\D to\R]% 
\xmultiput[50,50](60:43)(-30:1){12}%
  {\ArgOfVect\R to\Ang\rotatebox{\fpeval{\Ang-90}}%
    {\makebox(0,0)[b]{\Roman{multicnt}}}}%
      [\Multvect{\R}{\D}\R]
\end{picture}
\end{minipage}
\caption{Usage example of the \texttt{\string\xmultiput} command}
\label{pag:orologio}\hfill
\end{figure}
 
 \item This implementation of \pack{curve2e} includes an extension
 to package |xfp|, in the sense that adds three more L3 commands:
 |\fptest|, |\fpdowhile|, |\fpwhiledo| to the two already contained and
 documented  in the latter package. The syntax of such new commands
 is the following
  \begin{flushleft}\obeylines
  \cs{fptest}\marg{test}\marg{true}\marg{false}
  \cs{fpdowhile}\marg{test}\marg{operations to be repeated}
  \cs{fpwhiledo}\marg{test}\marg{operations to be repeated}
  \end{flushleft}
 The macro |\fptest| requires two further arguments that contain
 what to do if the \meta{test} is true, and what to do if the
 \meta{test} is false.
 The \meta{test} is a logical expressions that connects math relation
 expressions, even floating point ones, by means of \emph{logical
 operators}; such operators are \verb+||+, \verb|&&|, and \verb|!|,
 respectively for OR, AND, NOT; math relation expressions contain
 relation operators, even negated ones: for example \verb|!<| means 
 “not lower than”, which is equivalent to “equal or grater than”, i.e. 
 \verb|=>|. The logical expression is parsed left to right and normal 
 parentheses may be used to alter this sequence. The logical operators 
 work also between logical variables, therefore the \meta{test} may 
 contain an interesting mixture of relation and logical operators.

 Before using |\fpdowhile| and |\fpwhiledo| the arguments of \meta{test} 
 depends-on must be set so that the test is true; during the execution
 of the \meta{operations to be repeated} there must be some setting
 that eventually renders the \meta{test} false. The user should
 pay attention to set the elements that \meta{test} depends on,
 because the risk is to enter an infinite loop and end up with some
 error message stating that the working memory of the program is full.
 Notice that |\fpdowhile| first puts the \meta{operations to be repeated} 
 into the work flow then checks the \meta{test} and possibly repeats the 
 cycle; on the opposite, |\fpwhiledo| first checks the \meta{test} then 
 possibly inserts the \meta{operations to be repeated} and cycles. 
 Evidently with the same \meta{test} the two while cycles produce 
 different results with a little, but important difference: if the input 
 data are macros defined by previous computations, there is no guarantee 
 that the \meta{test} is initially true; if it sis false, |fpwhiledo| 
 does not do anything, while |\fpdowhile| executes one cycle and produces 
 in the output stream something that might be nonsense. Cycles done 
 with |\fpwhiledo| should be safer and should be preferred.
 
 Nevertheless such commands are very useful also for drawing graphics;
 the |xmultiput| already makes use of such L3~functions.

 As an example of use, we show how to plot a mathematical function
 expressed in parametric form:
\[\begin{cases}
x = f_1(t)\\
y = f_2(t)
\end{cases}\]
 The plot is executed with a piecewise linear approximation of the
 curve; if the $t$ steps are sufficiently small, the plot turns out to
 be very nice; here we show an example where we plot a Lissajous curve
 with two sinusoids of different period.

 We start by defining the Lissajous function with arguments
 to specify the parameter $t$, the sinusoid amplitudes $A_1, A_2$,
 the respective “frequencies”, by means of integer multiples of
 a unit pulsation, $N_1, N_2$, and the initial phases
 $\phi_1, \phi_2$ of such sinusoids. It is better to keep
 apart the input of the curve coefficients from the actual curve
 argument/parameter and output point coordinates:
\begin{verbatim}
\def\LissajousCoefs#1,#2,#3,#4,#5,#6!{%
\edef\LAu{#1}\edef\LNu{#2}\edef\LFu{#3}%
\edef\LAd{#4}\edef\LNd{#5}\edef\LFd{#6}}

\def\LissajousCode#1#2{%
\edef\X{\fpeval{\LAu*cosd(\LNu*#1+\LFu)}}%
\edef\Y{\fpeval{\LAd*cosd(\LNd*#1+\LFd)}}%
\CopyVect\X,\Y to#2\ignorespaces}
\end{verbatim}
 As it is shown, the coefficients are specified as a comma separated
 list; the  \verb|!| list terminator is taken care by the actual
 drawing command.

 Then the curve drawing command requires the coefficient specification
 only the first time it is used; some messages\footnote{Here it would
 be better to have available a “Macro Error” message; such a macro
 is not available, but it would be possible to define it by means
 of the \cs{GenericError} macro provided by the \LaTeX2e\ kernel.
 Here we skip this definition in order to avoid overloading this
 documentation with such details.} are output if the coefficients
 have been “forgotten”.
\begin{verbatim}
\NewDocumentCommand\Lissajous{m o m}{%
\IfValueTF{#2}{\LissajousCoefs#2!\relax
\LissajousCode{#1}{#3}}%
{\ifcsname LAu\endcsname
\LissajousCode{#1}{#3}%
\else\PackageError{curve2e}%
  {This Lissajous' curve coefficients\MessageBreak
   are missing}{Nothing done}\fi}%
\ignorespaces}
\end{verbatim}

 The syntax is the following:
\begin{flushleft}\obeylines
\cs{Lissajous}\marg{in}\Oarg{\meta{$A_1$},\meta{$N_1$},\meta{$\phi_1$},\meta{$A_2$},\meta{$N_2$},\meta{$\phi_2$}}\meta{$P_{\mathrm{out}}$}
\end{flushleft}
 where \meta{$P_{\mathrm{out}}$} is a macro that gets
 defined with the cartesian  coordinates of the computed output 
 point. Arguments \meta{in} (the $t$ parameter) and \meta{$P_\mathrm{out}$} (the computed coordinates) need not be enclosed within braces if they are given as macros; actually the code shown in figure~\ref{fig:lissajous} shows such procedure that renders the input code simpler to read. 

 After this definition the diagram is plotted in
 figure~\ref{fig:lissajous}.
%
\begin{figure}[!htb]\centering
\begin{Esempio}[\setfontsize{8.5}](0.55)
\unitlength=0.01\linewidth
\begin{picture}(100,100)(-50,-50)
\AutoGrid
\VECTOR(-50,0)(50,0)\Pbox(50,0)[tr]{x}[0]
\VECTOR(0,-50)(0,50)\Pbox(0,50)[tr]{y}[0]
\Pbox(0,0)[tr]{O}[2]
%
{\countdef\I=2560 \I=0
\fpdowhile{\I !> 360}{%
  \fptest{\I=0}%
     {\Lissajous\I[40,2,90,40,3,0]\Pout
       \moveto(\Pout)}%
     {\Lissajous\I\Pout
       \lineto(\Pout)}%
  \advance\I by1}\strokepath}%
\end{picture}
\end{Esempio}
\caption{A Lissajous diagram}\label{fig:lissajous}
\end{figure}

 For the independent variable $t$, the parameter of the Lissajous
 parametric equations, it is better to work with degrees instead
 of radians, and with integer numbers, so that the whole range
 from $0^\circ$ to $360^\circ$ is certainly spanned. Notice the
 braces that include the code for the Lissajous diagram; they may
 be useful to render that group suitable to be |\put| somewhere
 else than with its center at the origin of the canvas axes,
 and/or to be used as the second argument of a 
 \cs{rotatebox}\marg{angle} command so as to rotate the whole
 diagram.

 \end{enumerate}

%%%%%%%%%%%%%%%%%%%%%%%%%%%%%%%%
 \section{Remark}
%%%%%%%%%%%%%%%%%%%%%%%%%%%%%%%%

 In spite of the relative simplicity of the macros contained in this
 package, the described macros, as well as the original ones included in
 the \pack{pict2e} package, they allow to produce fine drawings that were
 unconceivable with the original \LaTeX\ picture environment. Leslie
 Lamport himself announced an extension to his environment when \LaTeXe\
 was first released in 1994; in the |latexnews| news-letter of December
 2003, the first implementation of Lamport's extension was announced;
 the first version of this package \pack{curve2e} was issued in 2006.
 It was time to have a  better drawing environment; this package
 is a simple attempt to follow the initial path while further extending 
 the drawing facilities.

 There are other packages in the \textsc{ctan} archives that deal with
 tracing curves of various kinds. \pack{PSTricks} and \pack{pgf/tikz} are 
 the most powerful ones. And they are becoming the standard for 
 programmed  drawing.
 Their documentation is huge and the multitude of extra modules to 
 perform special tasks is countless. Therefore they are difficult to use; 
 when the user gets used to their particular syntax and got sufficient 
 familiarity  with several modules, s/he can use these bundles very 
 comfortably. 
 
 This difficulty in becoming a TikZ or PS expert is why I think a simpler 
 drawing machinery should be appreciated. I admit it: I like the 
 \env{picture} environment; and I like to deal with simple codes so as to 
 create my own macros.
 
 But there is also \pack{curves} that is intended to draw almost
 anything by using little dots or other symbols partially superimposed
 to one another. It uses only quadratic Bézier curves and the curve
 tracing is eased by specifying only the curve nodes, without specifying
 the control nodes; with a suitable package option, it is
 possible to reduce the memory usage by using short straight segments
 drawn with the PostScript facilities offered by the |dvips| driver.

 Another package, \pack{ebezier} performs about the same as 
 \pack{curve2e}  but draws its Bézier curves by using little dots 
 partially superimposed to one another. The documentation is quite 
 interesting since it explains very clearly what exactly are the Bézier 
 splines. Apparently \pack{ebezier} should be used only for DVI output 
 without recourse to PostScript or PDF machinery.

 The \pack{picture} package extends the performance of the \env{picture}
 environment (extended with \pack{pict2e}) by accepting coordinates
 and lengths in real absolute dimensions, not only as multiples of
 |\unitlength|; it provides commands to extend that functionality to
 other packages. In certain circumstances it may be very useful.

 Package \pack{xpicture} builds over the \env{picture} \LaTeX\ 
 environment so as to allow to draw the usual curves that are part 
 of an introductory analytic geometry course; lines, circles, 
 parabolas, ellipses, hyperbolas, and polynomials; the syntax is 
 rather comfortable, although it is not a simple extension of the 
 \env{picture} own syntax; for all these curves it uses the quadratic 
 Bézier splines.
 
 Package \pack{hobby} extends the cubic Bézier spline handling with the
 algorithms John Hobby created for \MF\ and \MP. But by now this package
 interfaces very well with |tikz|; it has not (yet) been adapted to the
 common \env{picture} environment extended with \pack{pict2e}, and,
 why not, with \pack{curve2e}.

%%%%%%%%%%%%%%%%%%%%%%%%%%%%%%%%
 \section{Acknowledgements}
%%%%%%%%%%%%%%%%%%%%%%%%%%%%%%%%

 I wish to express my deepest thanks to Michel Goosens who spotted some
 errors and very kindly submitted them to me so that I was able to
 correct them.

 Josef Tkadlec and the author collaborated extensively in order to 
 implement a better real long division so as to get correctly the 
 quotient fractional part and to avoid as much as possible any numeric 
 overflow; many Josef's ideas are incorporated in the macro that was 
 implemented in the previous version of this package, although the 
 macro used by Josef was slightly  different. Both versions aim/aimed 
 at a better accuracy and at widening the operand ranges. In this 
 version of \pack{curve2e} we abandoned our long division macro, and 
 substituted it with the floating point division provided by the |xfp| 
 package.

 Daniele Degiorgi spotted a fault in the kernel definition of
 |\linethickness| that heavily influenced also \pack{curve2e}; see in
 the code documentation \file{curve2e.pdf} file.
 
 Many thanks to Domenicus van der Wijst who spotted a sneaky bug in the updated |\Arc| macro; for arcs non wider that $180^\circ$ it worked properly, while for larger angle apertures it produced syntax errors. It was just a typo, but very sneaky.

 Thanks also to Jin-Hwan Cho and Juho Lee who suggested a small but
 crucial modification in order to have \pack{curve2e} work smoothly
 also with XeTeX (XeLaTeX). Actually if \pack{pict2e}, version 0.2x or 
 later dated 2009/08/05  or later, is being used, such modification is 
 not necessary any more, but it is true that it becomes imperative when 
 legacy versions were used. 
 
 Many thanks also to Ashish Kumar Das who spotted
 an inconsistency in the design of vectors with PostScript style arrow
 tips with large line width settings, that did not show up with \LaTeX\ 
 styled ones.


\end{document}



