\documentclass[twoside,DIV=13]{scrartcl}
%% $Id: capPos.tex 103 2021-05-31 12:37:11Z herbert $

\documentclass[a4paper,11pt]{article}
\usepackage[color]{lapdf}
\textheight25.12cm
\textwidth18.92cm
\oddsidemargin-1.5cm
\evensidemargin-1.5cm
\topmargin-0.5cm
\topskip0cm
\headheight0cm
\headsep0cm
\parskip0.5cm
\parindent0cm
\unitlength1cm


\usepackage{multicol}
\usepackage[all=!htb]{hvfloat-fps}

\setlength\columnseprule{0.4pt}
\def\capShortText{A short caption}
\def\capLongText{Here comes a caption to show the  justification of the text relative to the object. It refers to 
the optional argument \texttt{capPos}.}

\let\OrigBlindtext\blindtext
\def\myblindtext{\TeX\ is a typesetting language. Instead of visually formatting your text, you enter y
our manuscript text intertwined with \TeX\ commands in a plain text file. You then run \TeX\ to 
produce formatted output, such as a PDF file. %Thus, in contrast to standard word processors, 
%your document is a separate file that does not pretend to be a representation of the final 
%typeset output, and so can be easily edited and manipulated.
}



\begin{document}

\tableofcontents

\newpage

\section{Examples for \texttt{capPos} and onecolumn mode}
\let\NewColumn\clearpage
\let\blindtext\myblindtext
\documentclass[twoside,DIV=13]{scrartcl}
%% $Id: capPos.tex 103 2021-05-31 12:37:11Z herbert $

\documentclass[a4paper,11pt]{article}
\usepackage[color]{lapdf}
\textheight25.12cm
\textwidth18.92cm
\oddsidemargin-1.5cm
\evensidemargin-1.5cm
\topmargin-0.5cm
\topskip0cm
\headheight0cm
\headsep0cm
\parskip0.5cm
\parindent0cm
\unitlength1cm


\usepackage{multicol}
\usepackage[all=!htb]{hvfloat-fps}

\setlength\columnseprule{0.4pt}
\def\capShortText{A short caption}
\def\capLongText{Here comes a caption to show the  justification of the text relative to the object. It refers to 
the optional argument \texttt{capPos}.}

\let\OrigBlindtext\blindtext
\def\myblindtext{\TeX\ is a typesetting language. Instead of visually formatting your text, you enter y
our manuscript text intertwined with \TeX\ commands in a plain text file. You then run \TeX\ to 
produce formatted output, such as a PDF file. %Thus, in contrast to standard word processors, 
%your document is a separate file that does not pretend to be a representation of the final 
%typeset output, and so can be easily edited and manipulated.
}



\begin{document}

\tableofcontents

\newpage

\section{Examples for \texttt{capPos} and onecolumn mode}
\let\NewColumn\clearpage
\let\blindtext\myblindtext
\documentclass[twoside,DIV=13]{scrartcl}
%% $Id: capPos.tex 103 2021-05-31 12:37:11Z herbert $

\documentclass[a4paper,11pt]{article}
\usepackage[color]{lapdf}
\textheight25.12cm
\textwidth18.92cm
\oddsidemargin-1.5cm
\evensidemargin-1.5cm
\topmargin-0.5cm
\topskip0cm
\headheight0cm
\headsep0cm
\parskip0.5cm
\parindent0cm
\unitlength1cm


\usepackage{multicol}
\usepackage[all=!htb]{hvfloat-fps}

\setlength\columnseprule{0.4pt}
\def\capShortText{A short caption}
\def\capLongText{Here comes a caption to show the  justification of the text relative to the object. It refers to 
the optional argument \texttt{capPos}.}

\let\OrigBlindtext\blindtext
\def\myblindtext{\TeX\ is a typesetting language. Instead of visually formatting your text, you enter y
our manuscript text intertwined with \TeX\ commands in a plain text file. You then run \TeX\ to 
produce formatted output, such as a PDF file. %Thus, in contrast to standard word processors, 
%your document is a separate file that does not pretend to be a representation of the final 
%typeset output, and so can be easily edited and manipulated.
}



\begin{document}

\tableofcontents

\newpage

\section{Examples for \texttt{capPos} and onecolumn mode}
\let\NewColumn\clearpage
\let\blindtext\myblindtext
\documentclass[twoside,DIV=13]{scrartcl}
%% $Id: capPos.tex 103 2021-05-31 12:37:11Z herbert $

\input{preamble.ltx}
\usepackage{multicol}
\usepackage[all=!htb]{hvfloat-fps}

\setlength\columnseprule{0.4pt}
\def\capShortText{A short caption}
\def\capLongText{Here comes a caption to show the  justification of the text relative to the object. It refers to 
the optional argument \texttt{capPos}.}

\let\OrigBlindtext\blindtext
\def\myblindtext{\TeX\ is a typesetting language. Instead of visually formatting your text, you enter y
our manuscript text intertwined with \TeX\ commands in a plain text file. You then run \TeX\ to 
produce formatted output, such as a PDF file. %Thus, in contrast to standard word processors, 
%your document is a separate file that does not pretend to be a representation of the final 
%typeset output, and so can be easily edited and manipulated.
}



\begin{document}

\tableofcontents

\newpage

\section{Examples for \texttt{capPos} and onecolumn mode}
\let\NewColumn\clearpage
\let\blindtext\myblindtext
\input{capPos.inc}

\section{Examples for \texttt{capVPos} and onecolumn mode }
Horizontal alignment with the optional arguments from package \texttt{caption}, e.g. \texttt{singlelinecheck=off} for short captions.

\input{capVPos.inc}


\twocolumn[\section{Examples for \texttt{capPos} and twocolumn mode}]
\let\NewColumn\columnbreak
\input{capPos.inc}

\let\blindtext\OrigBlindtext
\section{Examples for \texttt{capVPos} and twocolumn mode (\texttt{capPos=right})}
\let\NewColumn\columnbreak
\input{capVPos.inc}


\end{document}

\section{Examples for \texttt{capVPos} and onecolumn mode }
Horizontal alignment with the optional arguments from package \texttt{caption}, e.g. \texttt{singlelinecheck=off} for short captions.

\input{capVPos.inc}


\twocolumn[\section{Examples for \texttt{capPos} and twocolumn mode}]
\let\NewColumn\columnbreak
\documentclass[twoside,DIV=13]{scrartcl}
%% $Id: capPos.tex 103 2021-05-31 12:37:11Z herbert $

\input{preamble.ltx}
\usepackage{multicol}
\usepackage[all=!htb]{hvfloat-fps}

\setlength\columnseprule{0.4pt}
\def\capShortText{A short caption}
\def\capLongText{Here comes a caption to show the  justification of the text relative to the object. It refers to 
the optional argument \texttt{capPos}.}

\let\OrigBlindtext\blindtext
\def\myblindtext{\TeX\ is a typesetting language. Instead of visually formatting your text, you enter y
our manuscript text intertwined with \TeX\ commands in a plain text file. You then run \TeX\ to 
produce formatted output, such as a PDF file. %Thus, in contrast to standard word processors, 
%your document is a separate file that does not pretend to be a representation of the final 
%typeset output, and so can be easily edited and manipulated.
}



\begin{document}

\tableofcontents

\newpage

\section{Examples for \texttt{capPos} and onecolumn mode}
\let\NewColumn\clearpage
\let\blindtext\myblindtext
\input{capPos.inc}

\section{Examples for \texttt{capVPos} and onecolumn mode }
Horizontal alignment with the optional arguments from package \texttt{caption}, e.g. \texttt{singlelinecheck=off} for short captions.

\input{capVPos.inc}


\twocolumn[\section{Examples for \texttt{capPos} and twocolumn mode}]
\let\NewColumn\columnbreak
\input{capPos.inc}

\let\blindtext\OrigBlindtext
\section{Examples for \texttt{capVPos} and twocolumn mode (\texttt{capPos=right})}
\let\NewColumn\columnbreak
\input{capVPos.inc}


\end{document}

\let\blindtext\OrigBlindtext
\section{Examples for \texttt{capVPos} and twocolumn mode (\texttt{capPos=right})}
\let\NewColumn\columnbreak
\input{capVPos.inc}


\end{document}

\section{Examples for \texttt{capVPos} and onecolumn mode }
Horizontal alignment with the optional arguments from package \texttt{caption}, e.g. \texttt{singlelinecheck=off} for short captions.

\input{capVPos.inc}


\twocolumn[\section{Examples for \texttt{capPos} and twocolumn mode}]
\let\NewColumn\columnbreak
\documentclass[twoside,DIV=13]{scrartcl}
%% $Id: capPos.tex 103 2021-05-31 12:37:11Z herbert $

\documentclass[a4paper,11pt]{article}
\usepackage[color]{lapdf}
\textheight25.12cm
\textwidth18.92cm
\oddsidemargin-1.5cm
\evensidemargin-1.5cm
\topmargin-0.5cm
\topskip0cm
\headheight0cm
\headsep0cm
\parskip0.5cm
\parindent0cm
\unitlength1cm


\usepackage{multicol}
\usepackage[all=!htb]{hvfloat-fps}

\setlength\columnseprule{0.4pt}
\def\capShortText{A short caption}
\def\capLongText{Here comes a caption to show the  justification of the text relative to the object. It refers to 
the optional argument \texttt{capPos}.}

\let\OrigBlindtext\blindtext
\def\myblindtext{\TeX\ is a typesetting language. Instead of visually formatting your text, you enter y
our manuscript text intertwined with \TeX\ commands in a plain text file. You then run \TeX\ to 
produce formatted output, such as a PDF file. %Thus, in contrast to standard word processors, 
%your document is a separate file that does not pretend to be a representation of the final 
%typeset output, and so can be easily edited and manipulated.
}



\begin{document}

\tableofcontents

\newpage

\section{Examples for \texttt{capPos} and onecolumn mode}
\let\NewColumn\clearpage
\let\blindtext\myblindtext
\documentclass[twoside,DIV=13]{scrartcl}
%% $Id: capPos.tex 103 2021-05-31 12:37:11Z herbert $

\input{preamble.ltx}
\usepackage{multicol}
\usepackage[all=!htb]{hvfloat-fps}

\setlength\columnseprule{0.4pt}
\def\capShortText{A short caption}
\def\capLongText{Here comes a caption to show the  justification of the text relative to the object. It refers to 
the optional argument \texttt{capPos}.}

\let\OrigBlindtext\blindtext
\def\myblindtext{\TeX\ is a typesetting language. Instead of visually formatting your text, you enter y
our manuscript text intertwined with \TeX\ commands in a plain text file. You then run \TeX\ to 
produce formatted output, such as a PDF file. %Thus, in contrast to standard word processors, 
%your document is a separate file that does not pretend to be a representation of the final 
%typeset output, and so can be easily edited and manipulated.
}



\begin{document}

\tableofcontents

\newpage

\section{Examples for \texttt{capPos} and onecolumn mode}
\let\NewColumn\clearpage
\let\blindtext\myblindtext
\input{capPos.inc}

\section{Examples for \texttt{capVPos} and onecolumn mode }
Horizontal alignment with the optional arguments from package \texttt{caption}, e.g. \texttt{singlelinecheck=off} for short captions.

\input{capVPos.inc}


\twocolumn[\section{Examples for \texttt{capPos} and twocolumn mode}]
\let\NewColumn\columnbreak
\input{capPos.inc}

\let\blindtext\OrigBlindtext
\section{Examples for \texttt{capVPos} and twocolumn mode (\texttt{capPos=right})}
\let\NewColumn\columnbreak
\input{capVPos.inc}


\end{document}

\section{Examples for \texttt{capVPos} and onecolumn mode }
Horizontal alignment with the optional arguments from package \texttt{caption}, e.g. \texttt{singlelinecheck=off} for short captions.

\input{capVPos.inc}


\twocolumn[\section{Examples for \texttt{capPos} and twocolumn mode}]
\let\NewColumn\columnbreak
\documentclass[twoside,DIV=13]{scrartcl}
%% $Id: capPos.tex 103 2021-05-31 12:37:11Z herbert $

\input{preamble.ltx}
\usepackage{multicol}
\usepackage[all=!htb]{hvfloat-fps}

\setlength\columnseprule{0.4pt}
\def\capShortText{A short caption}
\def\capLongText{Here comes a caption to show the  justification of the text relative to the object. It refers to 
the optional argument \texttt{capPos}.}

\let\OrigBlindtext\blindtext
\def\myblindtext{\TeX\ is a typesetting language. Instead of visually formatting your text, you enter y
our manuscript text intertwined with \TeX\ commands in a plain text file. You then run \TeX\ to 
produce formatted output, such as a PDF file. %Thus, in contrast to standard word processors, 
%your document is a separate file that does not pretend to be a representation of the final 
%typeset output, and so can be easily edited and manipulated.
}



\begin{document}

\tableofcontents

\newpage

\section{Examples for \texttt{capPos} and onecolumn mode}
\let\NewColumn\clearpage
\let\blindtext\myblindtext
\input{capPos.inc}

\section{Examples for \texttt{capVPos} and onecolumn mode }
Horizontal alignment with the optional arguments from package \texttt{caption}, e.g. \texttt{singlelinecheck=off} for short captions.

\input{capVPos.inc}


\twocolumn[\section{Examples for \texttt{capPos} and twocolumn mode}]
\let\NewColumn\columnbreak
\input{capPos.inc}

\let\blindtext\OrigBlindtext
\section{Examples for \texttt{capVPos} and twocolumn mode (\texttt{capPos=right})}
\let\NewColumn\columnbreak
\input{capVPos.inc}


\end{document}

\let\blindtext\OrigBlindtext
\section{Examples for \texttt{capVPos} and twocolumn mode (\texttt{capPos=right})}
\let\NewColumn\columnbreak
\input{capVPos.inc}


\end{document}

\let\blindtext\OrigBlindtext
\section{Examples for \texttt{capVPos} and twocolumn mode (\texttt{capPos=right})}
\let\NewColumn\columnbreak
\input{capVPos.inc}


\end{document}

\section{Examples for \texttt{capVPos} and onecolumn mode }
Horizontal alignment with the optional arguments from package \texttt{caption}, e.g. \texttt{singlelinecheck=off} for short captions.

\input{capVPos.inc}


\twocolumn[\section{Examples for \texttt{capPos} and twocolumn mode}]
\let\NewColumn\columnbreak
\documentclass[twoside,DIV=13]{scrartcl}
%% $Id: capPos.tex 103 2021-05-31 12:37:11Z herbert $

\documentclass[a4paper,11pt]{article}
\usepackage[color]{lapdf}
\textheight25.12cm
\textwidth18.92cm
\oddsidemargin-1.5cm
\evensidemargin-1.5cm
\topmargin-0.5cm
\topskip0cm
\headheight0cm
\headsep0cm
\parskip0.5cm
\parindent0cm
\unitlength1cm


\usepackage{multicol}
\usepackage[all=!htb]{hvfloat-fps}

\setlength\columnseprule{0.4pt}
\def\capShortText{A short caption}
\def\capLongText{Here comes a caption to show the  justification of the text relative to the object. It refers to 
the optional argument \texttt{capPos}.}

\let\OrigBlindtext\blindtext
\def\myblindtext{\TeX\ is a typesetting language. Instead of visually formatting your text, you enter y
our manuscript text intertwined with \TeX\ commands in a plain text file. You then run \TeX\ to 
produce formatted output, such as a PDF file. %Thus, in contrast to standard word processors, 
%your document is a separate file that does not pretend to be a representation of the final 
%typeset output, and so can be easily edited and manipulated.
}



\begin{document}

\tableofcontents

\newpage

\section{Examples for \texttt{capPos} and onecolumn mode}
\let\NewColumn\clearpage
\let\blindtext\myblindtext
\documentclass[twoside,DIV=13]{scrartcl}
%% $Id: capPos.tex 103 2021-05-31 12:37:11Z herbert $

\documentclass[a4paper,11pt]{article}
\usepackage[color]{lapdf}
\textheight25.12cm
\textwidth18.92cm
\oddsidemargin-1.5cm
\evensidemargin-1.5cm
\topmargin-0.5cm
\topskip0cm
\headheight0cm
\headsep0cm
\parskip0.5cm
\parindent0cm
\unitlength1cm


\usepackage{multicol}
\usepackage[all=!htb]{hvfloat-fps}

\setlength\columnseprule{0.4pt}
\def\capShortText{A short caption}
\def\capLongText{Here comes a caption to show the  justification of the text relative to the object. It refers to 
the optional argument \texttt{capPos}.}

\let\OrigBlindtext\blindtext
\def\myblindtext{\TeX\ is a typesetting language. Instead of visually formatting your text, you enter y
our manuscript text intertwined with \TeX\ commands in a plain text file. You then run \TeX\ to 
produce formatted output, such as a PDF file. %Thus, in contrast to standard word processors, 
%your document is a separate file that does not pretend to be a representation of the final 
%typeset output, and so can be easily edited and manipulated.
}



\begin{document}

\tableofcontents

\newpage

\section{Examples for \texttt{capPos} and onecolumn mode}
\let\NewColumn\clearpage
\let\blindtext\myblindtext
\documentclass[twoside,DIV=13]{scrartcl}
%% $Id: capPos.tex 103 2021-05-31 12:37:11Z herbert $

\input{preamble.ltx}
\usepackage{multicol}
\usepackage[all=!htb]{hvfloat-fps}

\setlength\columnseprule{0.4pt}
\def\capShortText{A short caption}
\def\capLongText{Here comes a caption to show the  justification of the text relative to the object. It refers to 
the optional argument \texttt{capPos}.}

\let\OrigBlindtext\blindtext
\def\myblindtext{\TeX\ is a typesetting language. Instead of visually formatting your text, you enter y
our manuscript text intertwined with \TeX\ commands in a plain text file. You then run \TeX\ to 
produce formatted output, such as a PDF file. %Thus, in contrast to standard word processors, 
%your document is a separate file that does not pretend to be a representation of the final 
%typeset output, and so can be easily edited and manipulated.
}



\begin{document}

\tableofcontents

\newpage

\section{Examples for \texttt{capPos} and onecolumn mode}
\let\NewColumn\clearpage
\let\blindtext\myblindtext
\input{capPos.inc}

\section{Examples for \texttt{capVPos} and onecolumn mode }
Horizontal alignment with the optional arguments from package \texttt{caption}, e.g. \texttt{singlelinecheck=off} for short captions.

\input{capVPos.inc}


\twocolumn[\section{Examples for \texttt{capPos} and twocolumn mode}]
\let\NewColumn\columnbreak
\input{capPos.inc}

\let\blindtext\OrigBlindtext
\section{Examples for \texttt{capVPos} and twocolumn mode (\texttt{capPos=right})}
\let\NewColumn\columnbreak
\input{capVPos.inc}


\end{document}

\section{Examples for \texttt{capVPos} and onecolumn mode }
Horizontal alignment with the optional arguments from package \texttt{caption}, e.g. \texttt{singlelinecheck=off} for short captions.

\input{capVPos.inc}


\twocolumn[\section{Examples for \texttt{capPos} and twocolumn mode}]
\let\NewColumn\columnbreak
\documentclass[twoside,DIV=13]{scrartcl}
%% $Id: capPos.tex 103 2021-05-31 12:37:11Z herbert $

\input{preamble.ltx}
\usepackage{multicol}
\usepackage[all=!htb]{hvfloat-fps}

\setlength\columnseprule{0.4pt}
\def\capShortText{A short caption}
\def\capLongText{Here comes a caption to show the  justification of the text relative to the object. It refers to 
the optional argument \texttt{capPos}.}

\let\OrigBlindtext\blindtext
\def\myblindtext{\TeX\ is a typesetting language. Instead of visually formatting your text, you enter y
our manuscript text intertwined with \TeX\ commands in a plain text file. You then run \TeX\ to 
produce formatted output, such as a PDF file. %Thus, in contrast to standard word processors, 
%your document is a separate file that does not pretend to be a representation of the final 
%typeset output, and so can be easily edited and manipulated.
}



\begin{document}

\tableofcontents

\newpage

\section{Examples for \texttt{capPos} and onecolumn mode}
\let\NewColumn\clearpage
\let\blindtext\myblindtext
\input{capPos.inc}

\section{Examples for \texttt{capVPos} and onecolumn mode }
Horizontal alignment with the optional arguments from package \texttt{caption}, e.g. \texttt{singlelinecheck=off} for short captions.

\input{capVPos.inc}


\twocolumn[\section{Examples for \texttt{capPos} and twocolumn mode}]
\let\NewColumn\columnbreak
\input{capPos.inc}

\let\blindtext\OrigBlindtext
\section{Examples for \texttt{capVPos} and twocolumn mode (\texttt{capPos=right})}
\let\NewColumn\columnbreak
\input{capVPos.inc}


\end{document}

\let\blindtext\OrigBlindtext
\section{Examples for \texttt{capVPos} and twocolumn mode (\texttt{capPos=right})}
\let\NewColumn\columnbreak
\input{capVPos.inc}


\end{document}

\section{Examples for \texttt{capVPos} and onecolumn mode }
Horizontal alignment with the optional arguments from package \texttt{caption}, e.g. \texttt{singlelinecheck=off} for short captions.

\input{capVPos.inc}


\twocolumn[\section{Examples for \texttt{capPos} and twocolumn mode}]
\let\NewColumn\columnbreak
\documentclass[twoside,DIV=13]{scrartcl}
%% $Id: capPos.tex 103 2021-05-31 12:37:11Z herbert $

\documentclass[a4paper,11pt]{article}
\usepackage[color]{lapdf}
\textheight25.12cm
\textwidth18.92cm
\oddsidemargin-1.5cm
\evensidemargin-1.5cm
\topmargin-0.5cm
\topskip0cm
\headheight0cm
\headsep0cm
\parskip0.5cm
\parindent0cm
\unitlength1cm


\usepackage{multicol}
\usepackage[all=!htb]{hvfloat-fps}

\setlength\columnseprule{0.4pt}
\def\capShortText{A short caption}
\def\capLongText{Here comes a caption to show the  justification of the text relative to the object. It refers to 
the optional argument \texttt{capPos}.}

\let\OrigBlindtext\blindtext
\def\myblindtext{\TeX\ is a typesetting language. Instead of visually formatting your text, you enter y
our manuscript text intertwined with \TeX\ commands in a plain text file. You then run \TeX\ to 
produce formatted output, such as a PDF file. %Thus, in contrast to standard word processors, 
%your document is a separate file that does not pretend to be a representation of the final 
%typeset output, and so can be easily edited and manipulated.
}



\begin{document}

\tableofcontents

\newpage

\section{Examples for \texttt{capPos} and onecolumn mode}
\let\NewColumn\clearpage
\let\blindtext\myblindtext
\documentclass[twoside,DIV=13]{scrartcl}
%% $Id: capPos.tex 103 2021-05-31 12:37:11Z herbert $

\input{preamble.ltx}
\usepackage{multicol}
\usepackage[all=!htb]{hvfloat-fps}

\setlength\columnseprule{0.4pt}
\def\capShortText{A short caption}
\def\capLongText{Here comes a caption to show the  justification of the text relative to the object. It refers to 
the optional argument \texttt{capPos}.}

\let\OrigBlindtext\blindtext
\def\myblindtext{\TeX\ is a typesetting language. Instead of visually formatting your text, you enter y
our manuscript text intertwined with \TeX\ commands in a plain text file. You then run \TeX\ to 
produce formatted output, such as a PDF file. %Thus, in contrast to standard word processors, 
%your document is a separate file that does not pretend to be a representation of the final 
%typeset output, and so can be easily edited and manipulated.
}



\begin{document}

\tableofcontents

\newpage

\section{Examples for \texttt{capPos} and onecolumn mode}
\let\NewColumn\clearpage
\let\blindtext\myblindtext
\input{capPos.inc}

\section{Examples for \texttt{capVPos} and onecolumn mode }
Horizontal alignment with the optional arguments from package \texttt{caption}, e.g. \texttt{singlelinecheck=off} for short captions.

\input{capVPos.inc}


\twocolumn[\section{Examples for \texttt{capPos} and twocolumn mode}]
\let\NewColumn\columnbreak
\input{capPos.inc}

\let\blindtext\OrigBlindtext
\section{Examples for \texttt{capVPos} and twocolumn mode (\texttt{capPos=right})}
\let\NewColumn\columnbreak
\input{capVPos.inc}


\end{document}

\section{Examples for \texttt{capVPos} and onecolumn mode }
Horizontal alignment with the optional arguments from package \texttt{caption}, e.g. \texttt{singlelinecheck=off} for short captions.

\input{capVPos.inc}


\twocolumn[\section{Examples for \texttt{capPos} and twocolumn mode}]
\let\NewColumn\columnbreak
\documentclass[twoside,DIV=13]{scrartcl}
%% $Id: capPos.tex 103 2021-05-31 12:37:11Z herbert $

\input{preamble.ltx}
\usepackage{multicol}
\usepackage[all=!htb]{hvfloat-fps}

\setlength\columnseprule{0.4pt}
\def\capShortText{A short caption}
\def\capLongText{Here comes a caption to show the  justification of the text relative to the object. It refers to 
the optional argument \texttt{capPos}.}

\let\OrigBlindtext\blindtext
\def\myblindtext{\TeX\ is a typesetting language. Instead of visually formatting your text, you enter y
our manuscript text intertwined with \TeX\ commands in a plain text file. You then run \TeX\ to 
produce formatted output, such as a PDF file. %Thus, in contrast to standard word processors, 
%your document is a separate file that does not pretend to be a representation of the final 
%typeset output, and so can be easily edited and manipulated.
}



\begin{document}

\tableofcontents

\newpage

\section{Examples for \texttt{capPos} and onecolumn mode}
\let\NewColumn\clearpage
\let\blindtext\myblindtext
\input{capPos.inc}

\section{Examples for \texttt{capVPos} and onecolumn mode }
Horizontal alignment with the optional arguments from package \texttt{caption}, e.g. \texttt{singlelinecheck=off} for short captions.

\input{capVPos.inc}


\twocolumn[\section{Examples for \texttt{capPos} and twocolumn mode}]
\let\NewColumn\columnbreak
\input{capPos.inc}

\let\blindtext\OrigBlindtext
\section{Examples for \texttt{capVPos} and twocolumn mode (\texttt{capPos=right})}
\let\NewColumn\columnbreak
\input{capVPos.inc}


\end{document}

\let\blindtext\OrigBlindtext
\section{Examples for \texttt{capVPos} and twocolumn mode (\texttt{capPos=right})}
\let\NewColumn\columnbreak
\input{capVPos.inc}


\end{document}

\let\blindtext\OrigBlindtext
\section{Examples for \texttt{capVPos} and twocolumn mode (\texttt{capPos=right})}
\let\NewColumn\columnbreak
\input{capVPos.inc}


\end{document}

\let\blindtext\OrigBlindtext
\section{Examples for \texttt{capVPos} and twocolumn mode (\texttt{capPos=right})}
\let\NewColumn\columnbreak
\input{capVPos.inc}


\end{document}