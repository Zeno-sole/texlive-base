\input preamble.tex

\def\bf{\textbf}
\def\it{\textit}
\def\tt{\texttt}

% ---------------------------------------------------------------------------
\begin{document}
\unitlength0.85cm

\begin{center}
{\Huge \bf{Drawing with \pdfTeX} \bigskip}
\end{center}

Now you can draw arbitrary lines, polygons, integral and rational bezier
curves up to a degree of seven within your \pdfTeX{} document. The \Lapdf{}
style allows to draw directly with PDF, without any other files. You can
scale your drawing and use drawing primitives that look and function like
corresponding postscript commands. You can use PDF drawing primitives and
also \LaTeX{} typesetting of text in the same environment.

\begin{center}
\begin{lapdf}(18,18)(-9,-6)
 \Red
 \Rcurve(64)(-8,0,1)(2,12,3)(8,0,1) \Stroke
 \Green
 \Rcurve(64)(-8,0,1)(2,12,1)(8,0,1) \Stroke
 \Blue
 \Rcurve(64)(-8,0,3)(2,12,1)(8,0,3) \Gfill(0.9)
 \Dgray
 \Rcurve(64)(-8,0,1)(2,12,0)(8,0,1) \Stroke
 \Magenta
 \Rcurve(64)(-8,0,3)(2,12,-1)(8,0,3) \Gfill(0.9)
 \Dash(1)
 \Setwidth(0.01)
 \Black
 \Polygon(-8,0)(2,12)(8,0) \Stroke
 \Line(-1,-6)(2,12) \Stroke
 \Dash(0)
 \Point(0)(-8,0)
 \Point(0)(2,12)
 \Point(0)(8,0)
 \Point(1)(1.5,9)
 \Point(1)(1,6)
 \Point(1)(0.5,3)
 \Point(1)(0,0)
 \Point(1)(-1,-6)
 \Text(-8.0,0.2,br){$P_0$}
 \Text(2,12.2,bc){$P_1$}
 \Text(8.0,0.2,bl){$P_2$}
 \Text(-0.8,-5.8,bl){$-1/3$}
 \Text(0.2,0.2,bl){$w=0$}
 \Text(0.7,3.2,bl){$1/3$}
 \Text(1.2,6.2,bl){$1$}
 \Text(1.7,9.2,bl){$3$}
\end{lapdf}

\it{Various conic arcs defined by $w=(3,1,1/3,0,-1/3)$}.
\end{center}

This file shows some capabilities of the \Lapdf{} style. Rational quadratic
bezier curves can form all conic curves. These enable you to exactly draw any
parabolas, hyperbolas, ellipses and circles.

The above curves share the same control points, the only difference are the
curve's weights, which control the curve shape. \Lapdf{} has commands like
\it{Stroke, Fill, Setcol} and many others and it supports color.
\end{document}
