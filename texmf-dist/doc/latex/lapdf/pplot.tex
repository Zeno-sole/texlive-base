\input preamble.tex

\newdimen\a
\newdimen\b
\newdimen\c
\newdimen\r

% ---------------------------------------------------------------------------
\begin{document}
\begin{center}
{\Huge \bf{Polar Functions I}}
\bigskip

\begin{lapdf}(10,11)(-5,-6)
 \Polgrid(1,4)(4)
 \Red
 \def\Px(#1,#2){\Sin(#1,#2) \Dadd(#2,1) #2=2#2}
 \Pplot(60)(0,2) \Stroke
 \Cyan
 \def\Px(#1,#2){\Sin(#1,#2) \Dsub(#2,1) #2=-2#2}
 \Pplot(60)(0,2) \Stroke
 \Green
 \def\Px(#1,#2){\Cos(#1,#2) \Dadd(#2,1) #2=2#2}
 \Pplot(60)(0,2) \Stroke
 \Blue
 \def\Px(#1,#2){\Cos(#1,#2) \Dsub(#2,1) #2=-2#2}
 \Pplot(60)(0,2) \Stroke
\end{lapdf}
\bigskip

\begin{lapdf}(10,10)(-5,-5)
 \Polgrid(0,2)(5)
 \Magenta
 \def\Px(#1,#2){\Dset(#2,#1) \Mul(#2,4) \Div(#2,3) \Cos(\Np#2,#2) #2=5#2}
 \Pplot(600)(0,6) \Stroke
\end{lapdf}

\newpage

{\Huge \bf{Polar Functions II}}
\bigskip

\begin{lapdf}(10,11)(-5,-6)
 \Polgrid(1,2)(5)
 \Magenta
 \def\Px(#1,#2){\Dset(#2,#1) #2=0.2#2}
 \Pplot(300)(0,8) \Stroke
 \Green
 \def\Px(#1,#2){\Dset(#2,#1) #2=0.1835#2 \Exp(\Np#2,#2) \Div(#2,20)}
 \Pplot(200)(0,8) \Stroke
\end{lapdf}
\bigskip

\begin{lapdf}(10,10)(-5,-5)
 \Polgrid(1,3)(5)
 \Red
 \def\Px(#1,#2){\Dset(\a,#1) \b=2\a \c=6\a \a=4\a
  \Cos(\Np\a,#2) #2=4#2 \Sin(\Np\b,\b) \Sin(\Np\c,\c) \Sub(#2,\b) \Add(#2,\c)}
 \Pplot(200)(0,2) \Stroke
\end{lapdf}

\newpage

{\Huge \bf{Polar Functions III}}
\bigskip

\begin{lapdf}(10,12)(-5,-3)
 \Lingrid(10)(0,2)(-5,5)(-2,8)
 \Red
 \def\Px(#1,#2){\Sin(#1,#2) \Dset(\a,#1) \a=2.5\a
  \Sin(\Np\a,\a) \r=\a \Dmul(\r,\r) \Dmul(\a,\r) \Add(#2,\a) #2=4#2}
 \Pplot(400)(0,4) \Stroke
\end{lapdf}
\bigskip

\begin{lapdf}(10,10)(-5,-5)
 \Lingrid(10)(0,2)(-5,5)(-5,5)
 \Red
 \def\Px(#1,#2){\Dset(#2,#1) \Cos(\Np#2,#2) #2=3#2 \Dadd(#2,1)}
 \Pplot(100)(0,2) \Stroke
 \Green
 \def\Px(#1,#2){\Dset(#2,#1) \Sin(\Np#2,#2) #2=4#2 \Dadd(#2,0.5)}
 \Pplot(100)(0,2) \Stroke
 \Blue
 \def\Px(#1,#2){\Dset(#2,#1) #2=0.1835#2 \Exp(\Np#2,#2) \Div(#2,20)}
 \Pplot(200)(0,8) \Stroke
\end{lapdf}
\end{center}
\end{document}
