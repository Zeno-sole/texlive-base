%%%%%%%%%%%%%%%%%%%%%%%%%%%%%%%%%%%%%%%%%%%%%%%%%%%%%%%%%%%%
The following examples could be used for students or for any other purposes.
\\ [10pt]\pgfPTMbuildcell(8,3)[(1;1.4-2.8;Z),(1;3;radio),(2-3;1.5-3.5;CS),(4.2;1-3;name), %
(5.4;1-3;Ar),(6.5;1-3;eDist),(7.55-8.95;1-2.25;DiscC),(7.55-8.95;2.25-3.8;DiscY)%
]%
\pgfPTbuildcell(8,3)[%
(1;1.4-2.8;Z),(1;3;radio),%
(2-3;1.5-3.5;CS),(4.2;1-3;name),%
(5.4;1-3;Ar),(6.5;1-3;eDist),%
(7.55-8.95;1-2.25;DiscC),%
(7.55-8.95;2.25-3.8;DiscY)%
]%
\\ [-4pt]\pgfPTMmacrobox{pgfPT}[]%
\\ [10pt]\makebox[\linewidth][c]{\scalebox{.6}{\pgfPT}}%
\vfill%\\ [10pt]
\pgfPTMmacrobox{pgfPT}[eDist color=blue!70!black,Ar precision=2,DiscC font=\string\fontsize{4}{4}\string\selectfont,DiscY font=\string\fontsize{4}{4}\string\selectfont\string\bfseries]
\\ [10pt]\makebox[\linewidth][c]{\scalebox{.6}{\pgfPT[eDist color=blue!70!black,Ar precision=2,DiscC font=\fontsize{4}{4}\selectfont,DiscY font=\fontsize{4}{4}\selectfont\bfseries]}}%
\newpage%
\pgfPTMbuildcell(8,3)[(1;1-2;Z),(1;3;radio),(2-3;1-3;CS),(4;1-3;name),(5;1-2.5;Ar),(5;2.5-3;spectra), %
(7;1-2.5;DiscY),(7;2.5-3;DiscC),(8;1-3;eDist)%
]%
\pgfPTbuildcell(8,3)[%
(1;1-2;Z),(1;3;radio),%
(2-3;1-3;CS),(4;1-3;name),%
(5;1-2.5;Ar),(5;2.5-3;spectra),%
(7;1-2.5;DiscY),(7;2.5-3;DiscC),%
(8;1-3;eDist)%
]%
\\ [-4pt]\pgfPTMmacrobox{pgfPT}[csPS,Ar label=w,background={left color=black!20}]%
\\ [10pt]\makebox[\linewidth][c]{\scalebox{.6}{\pgfPT[csPS,Ar label=w,background={left color=black!20}]}}%
\vfill%
\pgfPTMbuildcell(8,3)[(1;1-3;Z),(1;3;radio),(2-3;1.5-3.5;CS),(4.2;1-3;name),(5.4;1-3;Ar), %
(6.5;1-3;eConfignl),(7.55-8.95;1-2.45;DiscC),(7.55-8.95;2.45-3;DiscY)%
]%
\pgfPTbuildcell(8,3)[%
(1;1-3;Z),(1;3;radio),%
(2-3;1.5-3.5;CS),(4.2;1-3;name),%
(5.4;1-3;Ar),%
(6.5;1-3;eConfignl),%
(7.55-8.95;1-2.45;DiscC),%
(7.55-8.95;2.45-3;DiscY)%
]%
\\ [-4pt]\pgfPTMmacrobox{pgfPT}[eConfignl color=blue!70!black,Ar precision=2,DiscC font=\string\fontsize{4}{4}\string\selectfont,DiscY font=\string\fontsize{4}{4}\string\selectfont\string\bfseries]%
\\ [10pt]\makebox[\linewidth][c]{\scalebox{.6}{\pgfPT[eConfignl color=blue!70!black,Ar precision=2,DiscC font=\fontsize{4}{4}\selectfont,DiscY font=\fontsize{4}{4}\selectfont\bfseries]}}%
\newpage\ %
\vfill%
\pgfPTbuildcell(8,3)[%
(1;1-3;Z),(1;3;radio),%
(2-3;1.5-3.5;CS),(4.2;1-3;name),%
(5.4;1-3;Ar),%
(6.5;1-3;eDist),%
(7.55-8.95;1-2.45;DiscC),%
(7.55-8.95;2.45-3;DiscY)%
]%
\pgfdeclarelayer{back}\pgfsetlayers{back,main}
\def\grupo[#1][#2] #3{%
\begin{tikzpicture}[inner xsep=0pt]
\node[below left,text width=1.75cm,text centered] (figura) at (0,0) %
{\scalebox{.5}{\pgfPT[show title=false,show label LaAc=true,show legend=false,back color scheme=MNM,%
        font=Roboto-TLF,CS font=\fontfamily{RobotoSlab-TLF}\bfseries\large,eDist color=blue!70!black,%
        DiscC font=\fontsize{4}{4}\selectfont,DiscY font=\fontsize{4}{4}\selectfont\bfseries,%
        name font=\fontseries{l}\fontsize{6pt}{6pt}\selectfont,name color=red!50!black,%
        Ar precision=2,Z list=G#2]}};%
\node[right,text width={\linewidth-2.25cm}] (descricao) at (figura.east) {#1\\ [4pt]#3};
\draw[draw=none,left color=black!20,right color=black!60] (figura.north west) rectangle ++(\linewidth,2pt);
\draw[draw=none,left color=black!20,right color=black!60] (figura.south west) rectangle ++(\linewidth,-2pt);
\begin{pgfonlayer}{back}
\draw[draw=none,left color=black!20,right color=black!60,opacity=.25] (figura.north west) rectangle ([xshift=\linewidth]figura.south west);
\end{pgfonlayer}
\end{tikzpicture}
}%
\tcexemplo[Representative elements: element families]{%
For the \textbf{\textit{representative elements}} (groups \textbf{1}, \textbf{2} and \textbf{13} to \textbf{18}) it is common to speak of families that reflect their common characteristics. So we have \textbf{the families}:
\\ [10pt]\grupo[GROUP \textcolor{blue!50!black}{\textbf{1}}: \textbf{Alkali metals}][1*]
{\red{\raisebox{1.25pt}{$\boldsymbol{\blacktriangleright}$} \textit{lithium, sodium, potassium, rubidium, cesium and francium}.}%
\\ [4.5pt]The atoms of these elements \textbf{have} only \textbf{\textcolor{blue!50!black}{one} valence electron}.%
\vspace{4.5pt}\small\begin{itemlist}
\item They react violently with water to form hydroxides.%
\item They have a silver-gray color, with the exception of cesium, which has a golden hue.%
\end{itemlist}
}%
\\ \grupo[GROUP \textcolor{blue!50!black}{\textbf{2}}: \textbf{Alkaline earth metals}][2]
{\red{\raisebox{1.25pt}{$\boldsymbol{\blacktriangleright}$} \textit{beryllium, magnesium, calcium, strontium, barium and radium}.}%
\\ [4.5pt]The atoms of these elements \textbf{have \textcolor{blue!50!black}{two} valence electrons}.%
\vspace{4.5pt}\small\begin{itemlist}
\item Their oxides remain solid at high temperatures and form alkaline solutions.%
\item They react violently with water to form hydroxides.%
\item When they burn, they have reddish flames, excluding barium, which presents a greenish flame.%
\end{itemlist}
}%
\\ \grupo[GROUP 1\textcolor{blue!50!black}{\textbf{3}}: \textbf{\textit{Boron} group}][13]
{\red{\raisebox{1.25pt}{$\boldsymbol{\blacktriangleright}$} \textit{boron, aluminium, gallium, indium, thallium and nihonium}.}%
\\ [4.5pt]The atoms of these elements \textbf{have \textcolor{blue!50!black}{three} valence electrons}.%
\vspace{4.5pt}\small\begin{itemlist}
\item Boron is a metalloid and the other are metals.%
\item Boron, aluminium, gallium, indium and thallium are often used as p-type silicon dopants.%
\item Aluminium is the third most abundant element in the Earth's crust (7.4\%)%
\end{itemlist}
}%
\\ \grupo[GROUP 1\textcolor{blue!50!black}{\textbf{4}}: \textbf{\textit{Carbon} group}][14]
{\red{\raisebox{1.25pt}{$\boldsymbol{\blacktriangleright}$} \textit{carbon, silicon, germanium, tin, lead and flerovium}.}%
\\ [4.5pt]The atoms of these elements \textbf{have \textcolor{blue!50!black}{four} valence electrons}.%
\vspace{4.5pt}\small\begin{itemlist}
\item Carbon is a non-metal, silicon and germanium are metalloids, and tin and lead are metals.%
\item Silicon and germanium are used in semiconductors.%
\end{itemlist}
}%
\\ \grupo[GROUP 1\textcolor{blue!50!black}{\textbf{5}}: \textbf{Pnictogens}][15]
{\red{\raisebox{1.25pt}{$\boldsymbol{\blacktriangleright}$} \textit{nitrogen, phosphorus, arsenic, antimony, bismuth and moscovium}.}%
\\ [4.5pt]The atoms of these elements \textbf{have \textcolor{blue!50!black}{five} valence electrons}.%
\vspace{4.5pt}\small\begin{itemlist}
\item Nitrogen and phosphorus are non-metals, arsenic and antimony are metalloids and bismuth is a metal.%
\item Phosphorus, arsenic, antimony and bismuth are often used as n-type silicon dopants.%
\item Diatomic nitrogen is the main constituent of the Earth's atmosphere (78\%).%
\end{itemlist}
}%
\\ \grupo[GROUP 1\textcolor{blue!50!black}{\textbf{6}}: \textbf{Chalcogens}][16]
{\red{\raisebox{1.25pt}{$\boldsymbol{\blacktriangleright}$} \textit{oxygen, sulfur, selenium, tellurium, polonium and livermorium}.}%
\\ [4.5pt]The atoms of these elements \textbf{have \textcolor{blue!50!black}{six} valence electrons}.%
\vspace{4.5pt}\small\begin{itemlist}
\item Oxygen, sulfur and selenium are non-metals, tellurium is a metalloid and polonium is a metal.%
\item Diatomic oxygen is the second constituent of the Earth's atmosphere (21\%).%
\end{itemlist}
}%
\\ \grupo[GROUP 1\textcolor{blue!50!black}{\textbf{7}}: \textbf{Halogens}][17]
{\red{\raisebox{1.25pt}{$\boldsymbol{\blacktriangleright}$} \textit{fluorine, chlorine, bromine, iodine, astatine and tennessine}.}%
\\ [4.5pt]The atoms of these elements \textbf{have \textcolor{blue!50!black}{seven} valence electrons}.%
\vspace{4.5pt}\small\begin{itemlist}
\item They are extremely reactive elements, as they are very electronegative.%
\item Fluorine is able to \textit{attack} inert substances, including the heavier noble gas atoms.%
\end{itemlist}
}%
\\ \grupo[GROUP 1\textcolor{blue!50!black}{\textbf{8}}: \textbf{Noble gases}][18]
{\red{\raisebox{1.25pt}{$\boldsymbol{\blacktriangleright}$} \textit{helium, neon, argon, krypton, xenon, radon and oganesson}.}%
\\ [4.5pt]The atoms of these elements have the valence shell fully filled, which corresponds to \textbf{\textcolor{blue!50!black}{eight} valence electrons}, with the exception Helium, which has only one shell and, consequently, has \textbf{two valence electrons}.
\vspace{4.5pt}\small\begin{itemlist}
\item They are extremely inert elements, that is, they do not react with other elements, as they are the most stable elements in Nature.%
\end{itemlist}
}%
}%
\vfill%
\blue{\textit{For the source of this example please see the file} pgf-PeriodicTableManual\_Examples.tex}
\vfill%
\newpage
\mymfbox{%
\textbf{\underline{EXERCISE}:}
\\ [3pt]In the following scheme of the Periodic Table, the positions of some chemical elements are represented by letters:
\\ [3pt]\makebox[\linewidth][c]{\textit{\scriptsize\blue{THE LETTERS DO NOT CORRESPOND TO THE CHEMICAL SYMBOLS OF THE ELEMENTS.}}}
\\ [6pt]\makebox[\linewidth][c]{\pgfPT[Z exercise list={1,2,3,4,9,12,17,18,19,20,25,27,32,34,35,49,54,74,86,87},Z list=spd,%s
                                                           cell size=1.5em,ex={c=blue,f=\bfseries}]}
\\ [6pt]\textbf{Using the letters shown}:
\begin{enumerate}
\item identify group 2 elements of the Periodic Table.%: \hrulefill
\item identify the elements of the 2\raisebox{3pt}{\scriptsize nd} period of the Periodic Table.%: \hrulefill
\item identify group 17 elements of the Periodic Table.%: \hrulefill
\item identify the elements of block s.%: \hrulefill
\item identify the elements of block p.%: \hrulefill
\item identify the elements of block d.%: \hrulefill
\item identify the metallic elements.%: \hrulefill
\item identify the non-metallic elements.%: \hrulefill
\item identify the transition metals.%: \hrulefill
\item identify the alkaline earth metals.%: \hrulefill
\item identify the noble gases.%: \hrulefill
\item tell which element belongs, simultaneously, to the 4\raisebox{3pt}{\scriptsize th} period and to group 14.%\\ [6pt]\makebox[\linewidth][s]{\hrulefill}
\item identify the representative elements that tend to generate positive ions.%:\\ [6pt]\makebox[\linewidth][s]{\hrulefill}
\item indicate an element that forms binegative ions.%: \hrulefill
\item indicate the halogen whose mononegative ion has the largest radius.%: \hrulefill
\item write the chemical formula of the compound formed by the elements \textbf{\blue{F}} and \textbf{\blue{O}}.%\\ [6pt]\makebox[\linewidth][s]{\hrulefill}
\item identify, justifying, the element with the largest atomic radius.%:\\ [6pt]\makebox[\linewidth][s]{\hrulefill}\\ [6pt]\makebox[\linewidth][s]{\hrulefill}
\item identify, justifying, the element with the lowest 1\raisebox{3pt}{\scriptsize st} ionization \mbox{energy}.%:\\ [6pt]\makebox[\linewidth][s]{\hrulefill}\\ [6pt]\makebox[\linewidth][s]{ \hrulefill}
\end{enumerate}
}%
\vfill%
\blue{\textit{For the source of this example please see the file} pgf-PeriodicTableManual\_Examples.tex}
\vfill%
\newpage
\def\xbox{\tikz[baseline=(x.base)]{\node[text width=15pt,text centered,font=\Large,draw,thick,rounded corners=.5pt,inner sep=0pt] (x) {\vbox to 15pt{\vfil\color{gray}x\vfil}};}}%
\def\obox{\tikz[baseline=(x.base)]{\node[text width=15pt,text centered,draw,thick,rounded corners=.5pt,inner sep=0pt] (x) {\vbox to 15pt{\vfil\color{gray}$\bigcirc$\vfil}};}}%
\def\dbox{\tikz[baseline=(x.base)]{\node[text width=15pt,text centered,font=\Large,draw,thick,rounded corners=.5pt,inner sep=0pt] (x) {\vbox to 15pt{\vfil\color{gray}$\Delta$\vfil}};}}%
\mymfbox{%
\textbf{\underline{EXERCISE}:}
\\ [3pt]Using the following notation,
\begin{itemize}
\item[\xbox] for the elements in the gaseous state (NTP),
\item[\obox] for the elements in the liquid state (NTP) and
\item[\dbox] for the synthetic elements,
\end{itemize}
fill in the following Periodic Table:
\\ [10pt]\makebox[\linewidth][c]{\scalebox{.6}{\pgfPT[only cells]}}
}
\vspace{15pt}%
\blue{\textit{For the source of this example please see the file} pgf-PeriodicTableManual\_Examples.tex}%
\endinput
