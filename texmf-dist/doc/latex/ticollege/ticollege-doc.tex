\documentclass[10pt,french,a4paper]{article}

\usepackage[utf8]{inputenc}
\usepackage[T1]{fontenc}
\usepackage{mathpazo}
\usepackage[euler-digits]{eulervm}
\usepackage[dvipsnames,table]{xcolor}
\usepackage{ticollege}
\usepackage[margin=2cm]{geometry}
\usepackage{tabularx,titlesec,wrapfig}
\usepackage{babel}
\usepackage[pdfborder={0 0 0},bookmarksnumbered,pdfpagelabels]{hyperref}
\usepackage{tcolorbox}
\tcbuselibrary{listingsutf8,documentation}
\tcbset{color key=blue,color value=ForestGreen,index colorize=true,doclang/keys=options,doclang/key=option,doclang/values=valeurs,doclang/value=valeur}
\usepackage{makeidx}
\makeindex

\newcommand*\Speciale{Cette option n'a aucune incidence sur le style \docValue{arrows}.\xspace}
\newcommand*\Couleur[1]{L'option \docAuxKey{color #1} est également disponible.\xspace}

%%% FORMAT SECTIONS DE mathbook.cls de Stéphane PASQUET

% couleurs section
\definecolor{section@title@color}{cmyk}{1,0.2,0.3,0.1}
\definecolor{subsection@title@color}{cmyk}{0,0.6,0.9,0}
\definecolor{shadow@color}{cmyk}{.07,0,0,0.49}
% fontes section
\def\sectiontitle@font{\fontfamily{ugq}\selectfont}
\def\subsectiontitle@font{\fontfamily{ugq}\selectfont}
% Décalages numéro de sections / titres des sections
\newlength\decalnumsec
\newlength\decalnumsubsec
\setlength{\decalnumsec}{-0.5em}
\setlength{\decalnumsubsec}{-0.5em}
\newlength\decalxtitlesubsec
\setlength{\decalxtitlesubsec}{\parindent}
% Espace entre le numéro de section et le titre
\newlength\spacetitlesec
\newlength\spacetitlesubsec
\setlength{\spacetitlesec}{0.5em}
\setlength{\spacetitlesubsec}{0.2em}

%%%%%%%%%%%%% Titre de section

\renewcommand{\thesection}{\Roman{section}}
\titleformat{\section}[block]
{%
	\bfseries\Large
	\color{section@title@color}
	\sectiontitle@font
}
{
\raisebox{\decalnumsec}
{%
\begin{tikzpicture}
\node (numsec) {\sectiontitle@font\thesection};
\fill[rounded corners=2pt,fill=shadow@color] ($(numsec.north west)+(2pt,-2pt)$) -- ($(numsec.north east)+(1mm,0mm)+(2pt,-2pt)$) -- ($(numsec.south east)+(2pt,-2pt)$) -- ($(numsec.south west)+(-1mm,0)+(2pt,-2pt)$) -- cycle;
\fill[rounded corners=2pt,fill=section@title@color] (numsec.north west) -- ($(numsec.north east)+(1mm,0mm)$) -- (numsec.south east) -- ($(numsec.south west)+(-1mm,0)$) -- cycle;
\node[white] at (numsec) {\sectiontitle@font\thesection};
\end{tikzpicture}
}
}
{\spacetitlesec}
{}

%%%%%%%%%%%%% Titre de subsection

\renewcommand{\thesubsection}{\arabic{subsection}}
\titleformat{\subsection}[block]
{%
	\hspace*{\decalxtitlesubsec}
	\bfseries\large
	\color{subsection@title@color}
	\subsectiontitle@font
}
{
	\raisebox{\decalnumsubsec}
	{%
	\begin{tikzpicture}
	\node (numsec) {\subsectiontitle@font\thesubsection};
	\fill[rounded corners=2pt,fill=shadow@color] ($(numsec.north west)+(2pt,-2pt)$) -- ($(numsec.north east)+(1mm,0mm)+(2pt,-2pt)$) -- ($(numsec.south east)+(2pt,-2pt)$) -- ($(numsec.south west)+(-1mm,0)+(2pt,-2pt)$) -- cycle;
	\fill[rounded corners=2pt,fill=subsection@title@color] (numsec.north west) -- ($(numsec.north east)+(1mm,0mm)$) -- (numsec.south east) -- ($(numsec.south west)+(-1mm,0)$) -- cycle;
	\node[white] at (numsec) {\subsectiontitle@font\thesubsection};
	\end{tikzpicture}
	}
}
{%
	\spacetitlesubsec
}
{%
	%\itemclass{subsection@title@color}{\subsection@font}
}

\begin{document}
\thispagestyle{empty}

\begin{center}
    \begin{tcolorbox}[enhanced,lifted shadow={1mm}{-2mm}{3mm}{0.1mm}{black!50!white},width=0.65\linewidth]
    \Huge\bfseries\centering ticollege.sty
    \end{tcolorbox}\vspace{\stretch{1}}

    \TiCCalc[title=documentation]\bigskip

    \today, version 1.0\medskip

    Philippe \bsc{De Sousa} (\href{mailto:philou.desousa@gmail.com}{philou.desousa@gmail.com})
\end{center}\vspace{\stretch{1}}

\begin{center}
    \small\bfseries
    Résumé
\end{center}

\begin{center}
    \begin{minipage}{0.9\linewidth}
    \small\parindent=15pt
    \begin{wrapfigure}{l}{0.9cm}
        \TiCCalc*
    \end{wrapfigure}
        En collège, les enseignants sont souvent amenés à travailler avec les élèves sur une calculatrice scientifique. La technologie d'aujourd'hui nous permet de vidéo-projeter la calculatrice et certains logiciels permettent de manipuler en même temps que les élèves pour leur montrer les différentes fonctionnalités de toutes les touches.\par\smallskip
        Mais une fois chez eux, comment se souvenir de ce qui a été fait en classe ?\par\smallskip
        J'ai créé ce package en m'appuyant sur un modèle de calculatrice répandu au collège afin de constituer des fiches méthodes que les élèves pourront utiliser à la maison et conserver d'année en année.\par
        \textsf{ticollege.sty} s'appuie beaucoup sur mon précédent package \textsf{tipfr.sty} pour les calculatrices graphiques. Il y a donc de nombreuses similitudes dans l'utilisation des deux packages et, pour un souci d'utilisation commune, j'ai été amené à nommer différemment des commandes effectuant le même travail.
    \end{minipage}
\end{center}\clearpage

\tableofcontents
\vspace*{2cm}

\section{Les touches}

\subsection{Dessiner une touche}

\begin{docCommand}{TiC}{\oarg{options}}
    Voici la commande qui fournit tous les dessins de touche. L'appel à la commande \cs{TiC} sans aucune option réalise la touche $\sin$ par défaut.
\end{docCommand}

\begin{dispExample*}{sidebyside, center lower}
\TiC
\end{dispExample*}
\medbreak

Pour obtenir d'autres types de touches, on utilise alors différentes options auxquelles on spécifie une valeur :\medbreak

\begin{docKey}{style}{=\meta{text}}{valeur par défaut : \docValue{general}}
Crée un dessin de touche selon le style précisé. Les différentes valeurs sont :
\end{docKey}
\begin{description}
    \item[\docValue{general}] dessine une touche aux bords arrondis pour les touches de fonctions ;
    \item[\docValue{number}] dessine une touche de forme différentes pour indiquer les chiffres ;
    \item[\docValue{arrows}] dessine la touche qui représente les quatres flèches de la calculatrice.
\end{description}

\begin{dispExample*}{sidebyside, center lower}
\TiC[style=general]
\TiC[style=number]
\TiC[style=arrows]
\end{dispExample*}
\medbreak

Le style \docValue{number} admet des variantes avec un bord arrondi.

\begin{docKey}{rounded}{=\meta{text}}{valeur par défaut : \docValue{none}}
Arrondis une touche sur le côté gauche ou droit. Cette option n'a d'incidence que sur le style \docValue{number}.
\end{docKey}
\begin{description}
    \item[\docValue{none}] la touche est rectangulaire ;
    \item[\docValue{left}] touche arrondie à gauche ;
    \item[\docValue{right}] touche arrondie à droite.
\end{description}

\begin{dispExample*}{sidebyside, center lower}
\TiC[style=number, rounded=left, principal=1]
\TiC[style=number, rounded=none, principal=2]
\TiC[style=number, rounded=right, principal=3]
\end{dispExample*}
\medbreak

Hormis pour les touches flèches, on constate que la valeur principale par défaut est $\sin$. L'exemple précédent nous montre qu'il est possible de modifier ce texte. De plus, sa position et sa taille sont également modifiable.\medbreak

\begin{docKey}{principal}{=\meta{text}}{valeur par défaut : \docValue{sin}}
Précise le \meta{text} à mettre à l'intérieur d'une touche. \Speciale
\end{docKey}

La commande \cs{Aff} a été créée pour dessiner le symbole suivant : \raisebox{-0.5em}{\Aff}

\begin{dispExample*}{sidebyside, center lower}
\TiC[principal=stats]
\TiC[style=number, principal=2]
\TiC[principal=\Aff]
\end{dispExample*}
\medbreak

\begin{docKey}{position}{=\meta{nombre}}{valeur par défaut : \docValue{0.9}}
Permet d'ajuster la position du texte principal à l'intérieur de la touche. \Speciale
\end{docKey}

\begin{dispExample*}{sidebyside, center lower}
\TiC[principal=op]
\TiC[principal=op, position = 0.5]
\TiC[principal=op, position = 1.1]
\end{dispExample*}
\medbreak

\begin{docKey}{fontsize}{=\meta{dim}}{valeur par défaut : \docValue{6pt}}
    L'option \docAuxKey{fontsize} permet de modifier ponctuellement la taille de la fonte utilisée dans la touche. \Speciale
\end{docKey}
\begin{dispExample*}{sidebyside, center lower}
\TiC[principal={$\triangleright$ simp}]
\TiC[principal={$\triangleright$ simp}, fontsize=5pt, position=1]
\qquad
\TiC[style=number, principal=2]
\TiC[style=number, principal=2,fontsize=9pt]
\end{dispExample*}
\medbreak

\begin{docKey}{raise}{=\meta{dim}}{valeur par défaut : \docValue{0ex}}
Permet d'ajuster la hauteur de la touche par rapport à la ligne de base. Les valeurs négatives sont autorisées.
\end{docKey}

\begin{dispExample*}{sidebyside, center lower}
\TiC[style=arrows]
\TiC[style=general]
\TiC[style=arrows, raise=-0.25cm]
\end{dispExample*}
\medbreak

Les touches ne sont pas uniquement composées de leur texte principal. Parfois, elles possèdent une fonction exécutée appelée à l'aide de la touche
\TiC[principal=2nde, colour text=black, colour key=TIJaune, raise=-1ex].

\begin{docKey}{second}{=\meta{text}}{\sffamily fonction inactive par défaut}
    \'Ecrit en couleur un texte secondaire centré au dessus de la touche. \Speciale
\end{docKey}

\begin{dispExample*}{sidebyside, center lower}
\TiC[principal=suppr, second=insérer, position=1]
\end{dispExample*}
\medbreak

Les commandes \cs{TiRacine}, \cs{ContrastDown}, \cs{ContrastUp} et \cs{Div} ont été créées pour dessiner différents symboles existant sur la calculatrice :

\begin{dispExample*}{sidebyside, center lower}
\TiC[principal={\large :}, second=\Div]
\TiC[principal=$x^n$, second={\TiRacine[n]}]
\TiC[principal=$x^2$, second={\TiRacine}]
\TiC[principal=$-$, second=\ContrastDown]
\TiC[principal=$+$, second=\ContrastUp]
\end{dispExample*}
\medbreak

Pour finir sur le dessin d'une touche, on peut spécifier la couleur du texte principal et celle du texte secondaire ainsi que la couleur de la touche. Les couleurs \docValue{TIJaune}, \docValue{TIRouge} et \docValue{TIOrange} ont été créées à cette occasion.\medbreak

\begin{docKey}{colour text}{=\meta{colour}}{valeur par défaut : \docValue{white}}
    Modifie la couleur du texte principal de la touche. \Couleur{text} \Speciale
\end{docKey}

\begin{dispExample*}{sidebyside, center lower}
\TiC[principal=\textit{cos}, position=1, second=\textit{arccos}]
\TiC[principal=\textit{cos}, position=1, second=\textit{arccos}, colour text=black]
\end{dispExample*}
\medbreak

\begin{docKey}{colour second}{=\meta{colour}}{valeur par défaut : \docValue{TIOrange}}
    Modifie la couleur du texte secondaire de la touche. \Couleur{second} \Speciale
\end{docKey}

\begin{dispExample*}{sidebyside, center lower}
\TiC[principal=\textit{cos}, position=1, second=\textit{arccos}]
\TiC[principal=\textit{cos}, position=1, second=\textit{arccos}, colour second=black]
\end{dispExample*}
\medbreak

\begin{docKey}{colour key}{=\meta{colour}}{valeur par défaut : \docValue{TIRouge}}
    Modifie la couleur de la touche. \Couleur{key}
\end{docKey}

\begin{dispExample*}{sidebyside, center lower}
\TiC[style=arrows, colour key=TIOrange]
\TiC[principal=2nde, colour key=TIJaune, colour text=black]
\TiC[style=number, principal=2, fontsize=9pt, colour key=white, colour text=black]
\end{dispExample*}
\medbreak

\subsection{Entourer une touche}

\begin{docKey}{circle}{=\docValue*{true|false}}{valeur par défaut : \docValue{false}}
    Permet d'entourer la touche à l'aide d'un cercle dont on peut alors préciser le rayon, l'épaisseur et la couleur.
\end{docKey}

\begin{docKey}{radius}{=\meta{dim}}{valeur par défaut : \docValue{20pt}}
    On spécifie ici le rayon du cercle qui ne sera pris en compte que si \docAuxKey{circle}=\docValue{true}.
\end{docKey}

\begin{docKey}{colour circle}{=\meta{colour}}{valeur par défaut : \docValue{red}}
    On spécifie ici la couleur du cercle qui ne sera prise en compte que si \docAuxKey{circle}=\docValue{true}. \Couleur{circle}
\end{docKey}

\begin{docKey}{thickness}{=\meta{dim}}{valeur par défaut : \docValue{1pt}}
    On spécifie ici l'épaisseur du cercle qui ne sera prise en compte que si \docAuxKey{circle}=\docValue{true}.
\end{docKey}

\begin{dispExample*}{sidebyside, center lower}
\TiC[principal=$x^2$, second={\TiRacine}, circle=true, thickness=0.5pt]
\TiC[style=arrows, colour key=TIOrange, circle=true, radius=25pt, colour circle=blue]
\end{dispExample*}
\medbreak

Chaque petite flèche du style \docValue{arrows} peut-être entouré individuellement. La couleur et l'épaisseur du cercle sont modifiables mais cette fois, le rayon du cercle est fixé.

\begin{docKey}{circleup}{=\docValue*{true|false}}{valeur par défaut : \docValue{false}}
Entoure la petite flèche du haut.
\end{docKey}

\begin{docKey}{circledown}{=\docValue*{true|false}}{valeur par défaut : \docValue{false}}
Entoure la petite flèche du bas.
\end{docKey}

\begin{docKey}{circleleft}{=\docValue*{true|false}}{valeur par défaut : \docValue{false}}
Entoure la petite flèche de gauche.
\end{docKey}

\begin{docKey}{circleright}{=\docValue*{true|false}}{valeur par défaut : \docValue{false}}
Entoure la petite flèche de droite.
\end{docKey}

\begin{dispExample*}{sidebyside, center lower}
\TiC[style=arrows, colour key=TIOrange, circleup=true]
\TiC[style=arrows, colour key=TIOrange, circledown=true, colour circle=blue]
\TiC[style=arrows, colour key=TIOrange, circleleft=true, circleright=true, colour circle=purple]
\end{dispExample*}
\medbreak

\subsection{Nommer une touche}\label{subsec:NomTouche}

\begin{docKey}{name}{=\meta{text}}{valeur par défaut : \docValue{NOM}}
    La touche sera référencée à l'aide d'un n{\oe}ud nommé \meta{text}.
\end{docKey}

\begin{dispExample}
Pour éteindre la calculatrice, on utilise la séquence suivante :
\begin{center}
    \TiC[principal=2nde, colour text=black, colour key=TIJaune]
    \TiC[principal=on, second=off, name=ON]
\end{center}
\begin{tikzpicture}[overlay, remember picture, >=latex']
    \draw[red, line width=1pt] ($(ON)+(0,0.4)$) circle (7pt);
    \draw[blue, line width=0.5pt, <-, rounded corners=10pt]
        ($(ON)+(0.4,0.4)$) -- ++(1,0)
        node[right] {\textsf{off} permet d'éteindre la calculatrice};
\end{tikzpicture}
\end{dispExample}

Les touches sont définies au sein d'un environnement \texttt{tikzpicture}. Afin de pouvoir s'y référer à l'intérieur d'un autre environnement de ce type, il faudra penser à utiliser les options \texttt{overlay} et \texttt{remember picture}. De plus, au minimum deux compilations seront nécessaires.



\section{Créer des menus}

En plus des différentes touches de la calculatrice, on pourra parler aux élèves des menus affichés par la calculatrice

\begin{docCommand}{TiCMenu}{\oarg{options}\marg{nom}}
    Cette commande écrit \meta{nom} en majuscule dans une fonte à chasse fixe de type {\ttfamily machine à écrire} pour nommer un menu de calculatrice. Ce nom est enfermé dans une boîte à fond blanc exactement à sa taille.
\end{docCommand}

\begin{dispExample*}{sidebyside, center lower}
\TiCMenu{Math} \TiCMenu{num} \TiCMenu{rnd} \TiCMenu{pol}
\end{dispExample*}
\medbreak

La taille peut être modifiée à l'aide de l'option suivante

\begin{docKey}{size}{=\meta{dim}}{valeur par défaut : \docValue{15pt}}
    Si l'unité de mesure n'est pas spécifié dans \meta{dim}, alors le \texttt{pt} sera utilisé par défaut.
\end{docKey}

\begin{dispExample*}{sidebyside, center lower}
\TiCMenu[size=1cm]{Math}
\TiCMenu{num}
\TiCMenu[size=8]{rnd}
\TiCMenu[size=8pt]{pol}
\end{dispExample*}
\medbreak

\begin{docKey}{select}{\docValue*{=true|false}}{valeur par défaut : \docValue{false}}
    Permet d'écrire le nom du menu en blanc sur fond noir pour signifier qu'il est sélectionné.
\end{docKey}

\begin{dispExample*}{sidebyside, center lower}
\TiCMenu{Math} \TiCMenu{num}
\TiCMenu[select=true]{rnd}
\TiCMenu{pol}
\end{dispExample*}
\medbreak

\begin{docKey}{colour box}{=\meta{colour}}{valeur par défaut : \docValue{white}}
    Détermine la couleur la boîte contenant le texte du menu lorsque celui-ci \textit{n'est pas} sélectionné. \Couleur{box}
\end{docKey}

\begin{dispExample*}{sidebyside, center lower}
\TiCMenu{Math} \TiCMenu{num}
\TiCMenu[select=true]{rnd}
\TiCMenu[colour box=red]{pol}
\end{dispExample*}
\medbreak

\begin{docKey}{text}{=\meta{text}}{valeur par défaut : \docValue{\symbol{92}unskip}}
    Cette dernière option permet de spécifier si un texte doit être écrit à côté du nom du menu. Pratique pour les menus sous forme de listes verticales. Le \meta{text} est sensible à l'option \docAuxKey{size}.
\end{docKey}

\begin{dispExample*}{sidebyside, center lower}
\TiCMenu[size=10pt, text={PGCD(}]{1 :}\par
\TiCMenu[size=10pt, select=true, text=PPCM(]{2 :}\par
\TiCMenu[size=10pt, text={abs(}]{3 :}
\end{dispExample*}

Voilà par exemple les quatres menus disponibles avec la touche \TiC[principal={maths},second={système}, raise=-1ex] :
\begin{center}\renewcommand\tabcolsep{-7pt}
    \begin{tabular}{*{4}{llll|}}
        \TiCMenu[size=10,select=true]{Maths} & \TiCMenu[size=10]{num} & \TiCMenu[size=10]{rnd} & \TiCMenu[size=10]{pol} &
                \TiCMenu[size=10]{Maths} & \TiCMenu[size=10,select=true]{num} & \TiCMenu[size=10]{rnd} & \TiCMenu[size=10]{pol} &
                \TiCMenu[size=10]{Maths} & \TiCMenu[size=10]{num} & \TiCMenu[size=10,select=true]{rnd} & \TiCMenu[size=10]{pol} &
                \TiCMenu[size=10]{Maths} & \TiCMenu[size=10]{num} & \TiCMenu[size=10]{rnd} & \TiCMenu[size=10,select=true]{pol} \\[-8pt]
%
        \multicolumn{4}{l|}{\TiCMenu[select=true, size=9, text=PGCD(]{1 :}} &
        \multicolumn{4}{l|}{\TiCMenu[select=true, size=9, text=arrondi(]{1 :}} &
        \multicolumn{4}{l|}{\TiCMenu[select=true, size=9, text=rand]{1 :}} &
        \multicolumn{4}{l|}{\TiCMenu[select=true, size=9, text={R$\blacktriangleright$P$r$}]{1 :}} \\[-8pt]
%
        \multicolumn{4}{l|}{\TiCMenu[size=9, text=PPCM(]{2 :}} &
        \multicolumn{4}{l|}{\TiCMenu[size=9, text=reste(]{2 :}} &
        \multicolumn{4}{l|}{\TiCMenu[size=9, text=randn(]{2 :}} &
        \multicolumn{4}{l|}{\TiCMenu[size=9, text={R$\blacktriangleright$P$\theta$}]{2 :}} \\[-8pt]
%
        \multicolumn{4}{l|}{\TiCMenu[size=9, text=abs(]{3 :}} &
        \multicolumn{4}{l|}{\TiCMenu[size=9, text=partEnt(]{3 :}} &
        \multicolumn{4}{l|}{} &
        \multicolumn{4}{l|}{\TiCMenu[size=9, text={P$\blacktriangleright$R$x$}]{3 :}}
%
    \end{tabular}
\end{center}


\section{Afficher un écran}
\subsection{Généralités}

\begin{docCommand}{TiCScreen}{\oarg{options}\marg{expression/résultat}}
    Cette commande permet d'afficher un écran de calculatrice.
\end{docCommand}

\begin{dispExample*}{center lower}
\TiCScreen{}
\end{dispExample*}

On peut modifier l'aspect général à l'aide des options suivantes :

\begin{docKey}{colour screen}{=\meta{colour}}{valeur par défaut : \docValue{ForestGreen}\docValue*{!15}}
    Détermine la couleur de fond de l'écran. \Couleur{screen}
\end{docKey}

\begin{docKey}{screenname}{=\meta{text}}{valeur par défaut : \docValue{ecran}}
    Donne un nom à l'écran afin de pouvoir s'y référencer plus tard avec des environnements \texttt{tikzpicture}. Les mêmes précautions que pour les touches doivent être prises (voir sous-section \ref{subsec:NomTouche} page \pageref{subsec:NomTouche}).
\end{docKey}

\begin{docKey}{width}{=\meta{number}}{valeur par défaut : \docValue{4}}
Permet de fixer la longueur de l'écran. L'unité de mesure est le \texttt{cm}.
\end{docKey}

\begin{docKey}{height}{=\meta{number}}{valeur par défaut : \docValue{1}}
Permet de fixer la largeur de l'écran. L'unité de mesure est le \texttt{cm}.
\end{docKey}

\begin{dispExample*}{sidebyside, center lower}
\TiCScreen[width=3, screenname=first]{}
\TiCScreen[width=3, colour screen=blue!50, screenname=second]{}
\tikz[remember picture, overlay]{\draw (first.center) circle (5pt);}
\tikz[remember picture, overlay]{\draw (second.north west) -- (second.south east);}
\end{dispExample*}

\subsection{\'Ecran de calculs}

Revenons sur l'argument obligatoire de la commande \cs{TiCScreen}. Cet argument est une liste de couples \meta{expression/résultat} séparés par une virgule.\par
On peut ne rien écrire à la place de \meta{expression} ou \meta{résultat} mais, dans ce cas, il ne faut pas mettre d'espace non plus. Les deux peuvent être laissés vides mais alors rien ne se passe (pas de création de ligne vide).\par
On pensera à utiliser des accolades si l'expression ou le résultat utilise les symboles \texttt{,} ou \texttt{/}.\medbreak

\begin{dispExample*}{sidebyside, center lower}
\TiCScreen[colour screen=blue!10, height=2, width=6]%
{%
3+2/5,
/,
1+2+3+4+5+6+7+8+9+10+11+12+13+14+15+16%
$\blacktriangleright$/5050}
\end{dispExample*}
\medbreak

\begin{dispExample*}{sidebyside, center lower}
\TiCScreen[colour screen=blue!10, height=2, width=6]%
{%
$\frac 35$/{0,6},
{0,6 $\blacktriangleright$\unskip f\unskip$\triangleleft \triangleright$ \unskip d}/{$\downarrow\frac{6}{10}$}
}
\end{dispExample*}
\medbreak

\begin{dispExample*}{sidebyside, center lower}
\TiCScreen[colour screen=blue!10]%
{%
{$\frac{6}{10}$ $\blacktriangleright$\unskip simp}/
{\raisebox{0.5ex}{\scriptsize Fac=2} $\frac{3}{5}$}
}
\end{dispExample*}
\medbreak

\begin{dispExample*}{center lower}
\TiCScreen[width=6,height=3]{
{\renewcommand\tabcolsep{-7pt}
    \begin{tabular}{llll}
        \TiCMenu[size=10,select=true]{Maths} & \TiCMenu[colour box={ForestGreen!15}, size=10]{num} & \TiCMenu[colour box={ForestGreen!15}, size=10]{rnd} & \TiCMenu[colour box={ForestGreen!15}, size=10]{pol} \\[-8pt]
        \multicolumn{4}{l}{\TiCMenu[select=true, size=9, text=PGCD(]{1 :}} \\[-8pt]
        \multicolumn{4}{l}{\TiCMenu[colour box={ForestGreen!15}, size=9, text=PPCM(]{2 :}} \\[-8pt]
        \multicolumn{4}{l}{\TiCMenu[colour box={ForestGreen!15}, size=9, text=abs(]{3 :}}
    \end{tabular}
}/{}
}
\end{dispExample*}

\section{La calculatrice}
\subsection{Version grand format}

\begin{docCommand}{TiCCalc}{\oarg{options}}
    Cette commande permet d'afficher la calculatrice en entier. Chaque touche a été nommée individuellement pour pouvoir s'y référer. Le tableau suivant donne le nom de chaque touche.
\end{docCommand}

\noindent%
\begin{tabularx}{\linewidth}{>{\bfseries}c *{8}{>{\centering\arraybackslash} X}}
    \hline
            Touche & \TiC[principal=2nde, colour key=TIJaune, colour text=black, name=scd] &
            \TiC[principal=mode, second=quitter, name=mode] &
            \TiC[principal=suppr, second=insérer, name=supp, position=1.1] &
            \TiC[principal={$\boldsymbol{\frac{n}{d}}$}, second={$\boldsymbol{f \triangleleft\triangleright d}$}, fontsize=7pt, name=nd] &
            \TiC[principal=stats, second=stats calc, name=stats] &
            \TiC[principal={$\boldsymbol{f(x)}$}, second=expr, name=fx] &
            \TiC[principal={$\triangleright$ simp}, second={$\triangleright$ décomp}, name=simp, fontsize=5pt]&
            \TiC[principal={$\times 10^n$}, second={$\frac1x$}, name=pdix]\\
\hline
    Nom & scd & mode & supp & nd & stats & fx & simp & pdix\\
\hline\hline
    Touche & \TiC[principal=op, second=déf op, name=op, position=1.1] &
    \TiC[principal=maths, second=système, name=math] &
    \TiC[principal=annul, name=annul] &
    \TiC[principal={\large$\pi$}, second=angle, name=pi] &
    \TiC[principal={\itshape sin}, second={\itshape arcsin}, name=sin] &
    \TiC[principal={\itshape cos}, second={\itshape arccos}, name=cos] &
    \TiC[principal={\itshape tan}, second={\itshape arctan}, name=tan] &
    \TiC[principal=$\div$, second=\Div, colour key=TIOrange, name=div] \\
\hline
    Nom & op & math & annul & pi & sin & cos & tan & div \\
\hline\hline
    Touche & \TiC[principal={$x^n$}, second={\TiRacine[n]}, name=pow] &
    \TiC[principal=\%, second={$\triangleright$ \%}, name=pcent] &
    \TiC[principal={(}, second={$\triangleright a\, \cdot 10^n$}, name=PO] &
    \TiC[principal={)}, second={$\triangleright$ norm}, name=PF] &
    \TiC[principal=$\times$, colour key=TIOrange, name=times] &
    \TiC[principal={$x^2$}, second={\TiRacine}, name=sqr] &
    \TiC[style=number, principal=7, rounded=left, colour key=white, colour text=black, name=T7] &
    \TiC[style=number, principal=8, colour key=white, colour text=black, name=T8] \\
\hline
    Nom & pow & pcent & PO & PF & times & sqr & T7 & T8 \\
\hline\hline
    Touche & \TiC[style=number, principal=9, rounded=right, colour key=white, colour text=black, name=T9] &
    \TiC[principal=$-$, second=\ContrastDown, colour key=TIOrange, name=sub] &
    \TiC[principal={$x^{yzt}_{abc}$}, second={eff var}, name=var] &
    \TiC[style=number, principal=4, rounded=left, colour key=white, colour text=black, name=T4] &
    \TiC[style=number, principal=5, colour key=white, colour text=black, name=T5] &
    \TiC[style=number, principal=6, rounded=right, colour key=white, colour text=black, name=T6] &
    \TiC[principal=$+$, second=\ContrastUp, colour key=TIOrange, name=plus] &
    \TiC[principal={sto $\triangleright$}, second={rap var}, name=sto] \\
\hline
    Nom & T9 & sub & var & T4 & T5 & T6 & plus & sto \\
\hline\hline
    Touche & \TiC[style=number, principal=1, rounded=left, colour key=white, colour text=black, name=T1] &
    \TiC[style=number, principal=2, colour key=white, colour text=black, name=T2] &
    \TiC[style=number, principal=3, rounded=right, colour key=white, colour text=black, name=T3] &
    \TiC[principal=\Aff, colour key=TIOrange, name=aff] &
    \TiC[principal={on}, second={off}, name=on] &
    \TiC[style=number, principal=0, second=réinit, rounded=left, colour key=white, colour text=black, name=T0] &
    \TiC[style=number, principal={,}, fontsize=9pt, position=1.2, second={;}, colour key=white, colour text=black, name=virgule] &
    \TiC[style=number, principal=(--), second=rép, rounded=right, colour key=white, colour text=black, name=minus] \\
\hline
    Nom & T1 & T2 & T3 & aff & on & T0 & virgule & minus \\
\hline\hline
    Touche & \TiC[principal=entrer, colour key=TIOrange, fontsize=5pt, name=entrer] & \textbf{Nom} & entrer &
    \textbf{Touche} & \multicolumn{2}{c}{\TiC[style=arrows, colour key=TIOrange, name=arrows, raise=-0.3cm]} & \textbf{Nom} & arrows
\end{tabularx}\label{tableau}

\begin{docKey}{title}{=\meta{text}}{valeur par défaut : \docValue{TI-Collège}}
    Permet de spécifier un titre au-dessus de l'écran de la calculatrice.
\end{docKey}

\begin{docKey}{colour calc}{=\meta{colour}}{valeur par défaut : \docValue{TIOrange}\docValue*{!50}}
    Détermine la couleur de la calculatrice. \Couleur{calc}
\end{docKey}

\begin{dispListing}
\TiCCalc[title=Structure, colour calc=brown!20]
\begin{tikzpicture}[overlay, remember picture]
        \draw[blue, line width=1pt,rounded corners = 5pt]
            ($(T0) + (-0.6,-0.3)$) -|
            ($(T9) + (0.5,0.4)$) -| cycle;
        \node[blue] at ($(virgule)+(0,-0.65)$) {\textbf Touches numériques};
\end{tikzpicture}
\end{dispListing}

\begin{center}
\TiCCalc[title=Structure, colour calc=brown!20]
\begin{tikzpicture}[overlay, remember picture]
        \draw[blue, line width=1pt,rounded corners = 5pt]
            ($(T0) + (-0.6,-0.3)$) -|
            ($(T9) + (0.5,0.4)$) -| cycle;
        \node[blue] at ($(virgule)+(0,-0.65)$) {\textbf Touches numériques};
\end{tikzpicture}
\end{center}

\subsection{Version petit format}

\begin{docCommand}{TiCCalc*}{\oarg{options}}
    Cette commande permet d'afficher une calculatrice en petit format à utiliser dans des fiches méthodes par exemple.
\end{docCommand}

\begin{dispExample*}{sidebyside, center lower}
\TiCCalc*
\end{dispExample*}

L'aspect de la calculatrice est modifiable.

\begin{docKey}{calcscale}{=\meta{number}}{valeur par défaut : \docValue{0.5}}
Permet de modifier la taille de la calculatrice. Plus la calculatrice est petite, moins les dessins de touches seront précis.
\end{docKey}

\begin{dispExample*}{sidebyside, center lower}
\TiCCalc*[calcscale=1]
\TiCCalc*[calcscale=0.25]
\end{dispExample*}

\begin{docKey}{calcrotate}{=\meta{number}}{valeur par défaut : \docValue{-30}}
Permet de changer l'angle d'affichage de la calculatrice.
\end{docKey}

\begin{dispExample*}{sidebyside, center lower}
\TiCCalc*[calcrotate=0] \textbf{Méthode}
\qquad
\TiCCalc*[calcrotate=90] \textbf{Méthode}
\par\bigskip
\TiCCalc*[calcrotate=-30]
\hspace{-1em}\textbf{Méthode}
\qquad
\rotatebox{90}{\textbf{Méthode}}
\TiCCalc*[calcrotate=0]
\end{dispExample*}

\begin{docKey}{calcraise}{=\meta{dim}}{valeur par défaut : \docValue{-2ex}}
Permet de modifier la hauteur de la calculatrice en fonction de la ligne de base.
\end{docKey}

\begin{dispExample*}{sidebyside, center lower}
\TiCCalc*[calcrotate=0, calcraise=0ex] \textbf{Méthode}
\qquad
\rotatebox{90}{\textbf{Méthode}}
\TiCCalc*[calcrotate=0, calcraise=1ex]
\end{dispExample*}


\clearpage

\printindex
\end{document} 