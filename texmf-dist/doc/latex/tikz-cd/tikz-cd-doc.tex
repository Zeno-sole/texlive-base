%% tikz-cd-doc.tex
%% Copyright 2011, 2012, 2014, 2018, 2021 Augusto Stoffel
%
% This work may be distributed and/or modified under the
% conditions of the LaTeX Project Public License, either version 1.3
% of this license or (at your option) any later version.
% The latest version of this license is in
%   http://www.latex-project.org/lppl.txt
% and version 1.3 or later is part of all distributions of LaTeX
% version 2005/12/01 or later.
%
% This work has the LPPL maintenance status `maintained'.
%
% The Current Maintainer of this work is Augusto Stoffel.
%
% This work consists of the files tikzlibrarycd.code.tex, tikz-cd.sty
% and tikz-cd-doc.tex.
%
% ------------------------------------------------------------------------
%
% This document requires the file pgfmanual-en-macros.tex (distributed
% with pgf) to compile.  To get cross-references to the pgf manual,
% compile it and copy all resulting .aux files to the directory of
% tikz-cd-doc.tex.

\def\pgfautoxrefs{1}

\documentclass[a4paper]{ltxdoc}
\usepackage[hyphens]{url}

\def\xcolorversion{2.00}

\usepackage[version=latest]{pgf}

\usepackage{xkeyval,calc,listings,tikz,fp}

\usepackage{hyperref}
\hypersetup{%
	colorlinks=false,	% use true to enable colors below:
	linkcolor=blue,%red,
	filecolor=blue,%magenta,
	pagecolor=blue,%red,
	urlcolor=blue,%cyan,
	citecolor=blue,
	pdfborder=0 0 0,
}

% We need lots of libraries...
\usetikzlibrary{
  arrows,
  arrows.meta,
  calc,
  fit,
  patterns,
  plotmarks,
  shapes.geometric,
  shapes.misc,
  shapes.symbols,
  shapes.arrows,
  shapes.callouts,
  shapes.multipart,
  shapes.gates.logic.US,
  shapes.gates.logic.IEC,
  circuits.logic.US,
  circuits.logic.IEC,
  circuits.logic.CDH,
  circuits.ee.IEC,
  datavisualization,
  datavisualization.formats.functions,
  er,
  automata,
  backgrounds,
  chains,
  topaths,
  trees,
  petri,
  mindmap,
  matrix,
  calendar,
  folding,
  fadings,
  shadings,
  spy,
  through,
  turtle,
  positioning,
  scopes,
  decorations.fractals,
  decorations.shapes,
  decorations.text,
  decorations.pathmorphing,
  decorations.pathreplacing,
  decorations.footprints,
  decorations.markings,
  shadows,
  lindenmayersystems,
  intersections,
  fixedpointarithmetic,
  fpu,
  svg.path,
  external,
}

\usepackage[a4paper,left=2.25cm,right=2.25cm,top=2.5cm,bottom=2.5cm,nohead]{geometry}
\usepackage{amsmath,amssymb}
\usepackage{xxcolor}
\usepackage{pifont}
\usepackage{makeidx}
\usepackage{tikz-cd,xr,multicol,microtype}


\IfFileExists{pgfmanual-en-macros}
  {% Copyright 2019 by Till Tantau
%
% This file may be distributed and/or modified
%
% 1. under the LaTeX Project Public License and/or
% 2. under the GNU Free Documentation License.
%
% See the file doc/generic/pgf/licenses/LICENSE for more details.

% $Header$


\newcount\pgfmanualtargetcount

\colorlet{examplefill}{yellow!80!black}
\definecolor{graphicbackground}{rgb}{0.96,0.96,0.8}
\definecolor{codebackground}{rgb}{0.9,0.9,1}
\definecolor{animationgraphicbackground}{rgb}{0.96,0.96,0.8}

\newenvironment{pgfmanualentry}{\list{}{\leftmargin=2em\itemindent-\leftmargin\def\makelabel##1{\hss##1}}}{\endlist}
\newcounter{pgfmanualentry}
\newcommand\pgfmanualentryheadline[1]{%
  \itemsep=0pt\parskip=0pt{\raggedright\item\refstepcounter{pgfmanualentry}\strut{#1}\par}\topsep=0pt}
\newcommand\pgfmanualbody{\parskip3pt}

\let\origtexttt=\texttt
\def\texttt#1{{\def\textunderscore{\char`\_}\def\textbraceleft{\char`\{}\def\textbraceright{\char`\}}\origtexttt{#1}}}
\def\exclamationmarktext{!}
\def\atmarktext{@}

{
  \catcode`\|=12
  \gdef\pgfmanualnormalbar{|}
  \catcode`\|=13
  \AtBeginDocument{\gdef|{\ifmmode\pgfmanualnormalbar\else\expandafter\verb\expandafter|\fi}}
}



\newenvironment{pgflayout}[1]{
  \begin{pgfmanualentry}
    \pgfmanualentryheadline{%
      \pgfmanualpdflabel{#1}{}%
      \texttt{\string\pgfpagesuselayout\char`\{\declare{#1}\char`\}}\oarg{options}%
    }
    \index{#1@\protect\texttt{#1} layout}%
    \index{Page layouts!#1@\protect\texttt{#1}}%
    \pgfmanualbody
}
{
  \end{pgfmanualentry}
}


\newenvironment{sysanimateattribute}[1]{
  \begin{pgfmanualentry}
    \pgfmanualentryheadline{%
      \pgfmanualpdflabel{#1}{}%
      \texttt{\string\pgfsysanimate\char`\{\declare{#1}\char`\}}%
    }
    \index{#1@\protect\texttt{#1} system layer animation attribute}%
    \index{Animation attributes (system layer)!#1@\protect\texttt{#1}}%
    \pgfmanualbody
}
{
  \end{pgfmanualentry}
}


\newenvironment{animateattribute}[1]{
  \begin{pgfmanualentry}
    \pgfmanualentryheadline{%
      \pgfmanualpdflabel{#1}{}%
      \texttt{\string\pgfanimateattribute\char`\{\declare{#1}\char`\}\marg{options}}%
    }
    \index{#1@\protect\texttt{#1} basic layer animation attribute}%
    \index{Animation attributes (basic layer)!#1@\protect\texttt{#1}}%
    \pgfmanualbody
}
{
  \end{pgfmanualentry}
}


\newenvironment{tikzanimateattribute}[1]{
  \begin{pgfmanualentry}
    \pgfmanualentryheadline{%
      \foreach \attr in{#1} {\expandafter\pgfmanualpdflabel\expandafter{\attr}{}}%
      \textbf{Animation attribute} \foreach \attr[count=\i]
      in{#1}{{\ifnum\i>1 \textbf,\fi} \texttt{:\declare{\attr}}}%
    }
    \foreach\attr in{#1}{%
      \edef\indexcall{%
        \noexpand\index{\attr@\noexpand\protect\noexpand\texttt{\attr} animation attribute}%
        \noexpand\index{Animation attributes!\attr@\noexpand\protect\noexpand\texttt{\attr}}%
      }%
      \indexcall%
    }%
    \pgfmanualbody
}
{
  \end{pgfmanualentry}
}


\newenvironment{command}[1]{
  \begin{pgfmanualentry}
    \extractcommand#1\@@
    \pgfmanualbody
}
{
  \end{pgfmanualentry}
}

\makeatletter

\def\includeluadocumentationof#1{
  \directlua{require 'pgf.manual.DocumentParser'}
  \directlua{pgf.manual.DocumentParser.include '#1'}
}

\newenvironment{luageneric}[4]{
  \pgfmanualentry
    \pgfmanualentryheadline{#4 \texttt{#1\declare{#2}}#3}
    \index{#2@\protect\texttt{#2} (Lua)}%
    \def\temp{#1}
    \ifx\temp\pgfutil@empty\else
      \index{#1@\protect\texttt{#1}!#2@\protect\texttt{#2} (Lua)}%
    \fi
  \pgfmanualbody
}{\endpgfmanualentry}

\newenvironment{luatable}[3]{
  \medskip
  \luageneric{#1}{#2}{ (declared in \texttt{#3})}{\textbf{Lua table}}
}{\endluageneric}

\newenvironment{luafield}[1]{
  \pgfmanualentry
    \pgfmanualentryheadline{Field \texttt{\declare{#1}}}
  \pgfmanualbody
}{\endpgfmanualentry}


\newenvironment{lualibrary}[1]{
  \pgfmanualentry
  \pgfmanualentryheadline{%
    \pgfmanualpdflabel{#1}{}%
    \textbf{Graph Drawing Library} \texttt{\declare{#1}}%
  }
    \index{#1@\protect\texttt{#1} graph drawing library}%
    \index{Libraries!#1@\protect\texttt{#1}}%
    \index{Graph drawing libraries!#1@\protect\texttt{#1}}%
    \vskip.25em
    {\ttfamily\char`\\usegdlibrary\char`\{\declare{#1}\char`\}\space\space \char`\%\space\space  \LaTeX\space and plain \TeX}\\
    {\ttfamily\char`\\usegdlibrary[\declare{#1}]\space \char`\%\space\space Con\TeX t}\smallskip\par
    \pgfmanualbody
}{\endpgfmanualentry}

\newenvironment{luadeclare}[4]{
  \pgfmanualentry
  \def\manual@temp@default{#3}%
  \def\manual@temp@initial{#4}%
  \def\manual@temp@{#3#4}%
  \pgfmanualentryheadline{%
    \pgfmanualpdflabel{#1}{}%
    {\ttfamily/graph
      drawing/\declare{#1}\opt{=}}\opt{#2}\hfill%
    \ifx\manual@temp@\pgfutil@empty\else%
    (\ifx\manual@temp@default\pgfutil@empty\else%
    default {\ttfamily #3}\ifx\manual@temp@initial\pgfutil@empty\else, \fi%
    \fi%
    \ifx\manual@temp@initial\pgfutil@empty\else%
    initially {\ttfamily #4}%
    \fi%
    )\fi%
  }%
  \index{#1@\protect\texttt{#1} key}%
  \pgfmanualbody
  \gdef\myname{#1}%
%  \keyalias{tikz}
%  \keyalias{tikz/graphs}
}{\endpgfmanualentry}

\newenvironment{luadeclarestyle}[4]{
  \pgfmanualentry
  \def\manual@temp@para{#2}%
  \def\manual@temp@default{#3}%
  \def\manual@temp@initial{#4}%
  \def\manual@temp@{#3#4}%
  \pgfmanualentryheadline{%
    \pgfmanualpdflabel{#1}{}%
    {\ttfamily/graph drawing/\declare{#1}}\ifx\manual@temp@para\pgfutil@empty\else\opt{\texttt=}\opt{#2}\fi\hfill%
    (style\ifx\manual@temp@\pgfutil@empty\else, %
    \ifx\manual@temp@default\pgfutil@empty\else%
    default {\ttfamily #3}\ifx\manual@temp@initial\pgfutil@empty\else, \fi%
    \fi%
    \ifx\manual@temp@initial\pgfutil@empty\else%
    initially {\ttfamily #4}%
    \fi%
    \fi)%
  }%
  \index{#1@\protect\texttt{#1} key}%
  \pgfmanualbody%
  \gdef\myname{#1}%
%  \keyalias{tikz}
%  \keyalias{tikz/graphs}
}{\endpgfmanualentry}

\newenvironment{luanamespace}[2]{
  \luageneric{#1}{#2}{}{\textbf{Lua namespace}}
}{\endluageneric}

\newenvironment{luafiledescription}[1]{}{}

\newenvironment{luacommand}[4]{
  \hypertarget{pgf/lua/#1}{\luageneric{#2}{#3}{\texttt{(#4)}}{\texttt{function}}}
}{\endluageneric}

\newenvironment{luaparameters}{\par\emph{Parameters:}%
  \parametercount=0\relax%
  \let\item=\parameteritem%
  \let\list=\restorelist%
}
{\par
}

\newenvironment{luareturns}{\par\emph{Returns:}%
  \parametercount=0\relax%
  \let\item=\parameteritem%
  \let\list=\restorelist%
}
{\par
}

\newcount\parametercount

\newenvironment{parameterdescription}{\unskip%
  \parametercount=0\relax%
  \let\item=\parameteritem%
  \let\list=\restorelist%
}
{\par
}
\let\saveditemcommand=\item
\let\savedlistcommand=\list
\def\denselist#1#2{\savedlistcommand{#1}{#2}\parskip0pt\itemsep0pt}
\def\restorelist{\let\item=\saveditemcommand\denselist}
\def\parameteritem{\pgfutil@ifnextchar[\parameteritem@{}}%}
\def\parameteritem@[#1]{\advance\parametercount by1\relax\hskip0.15em plus 1em\emph{\the\parametercount.}\kern1ex\def\test{#1}\ifx\test\pgfutil@empty\else#1\kern.5em\fi}

\newenvironment{commandlist}[1]{%
  \begin{pgfmanualentry}
  \foreach \xx in {#1} {%
    \expandafter\extractcommand\xx\@@
  }%
  \pgfmanualbody
}{%
  \end{pgfmanualentry}
}%

% \begin{internallist}[register]{\pgf@xa}
% \end{internallist}
%
% \begin{internallist}[register]{\pgf@xa,\pgf@xb}
% \end{internallist}
\newenvironment{internallist}[2][register]{%
  \begin{pgfmanualentry}
  \foreach \xx in {#2} {%
    \expandafter\extractinternalcommand\expandafter{\xx}{#1}%
  }%
  \pgfmanualbody
}{%
  \end{pgfmanualentry}
}%
\def\extractinternalcommand#1#2{%
  \removeats{#1}%
  \pgfmanualentryheadline{%
    \pgfmanualpdflabel{\textbackslash\strippedat}{}%
    Internal #2 \declare{\texttt{\string#1}}}%
  \index{Internals!\strippedat @\protect\myprintocmmand{\strippedat}}%
  \index{\strippedat @\protect\myprintocmmand{\strippedat}}%
}

%% MW: START MATH MACROS
\def\mvar#1{{\ifmmode\textrm{\textit{#1}}\else\rmfamily\textit{#1}\fi}}

\makeatletter

\def\extractmathfunctionname#1{\extractmathfunctionname@#1(,)\tmpa\tmpb}
\def\extractmathfunctionname@#1(#2)#3\tmpb{\def\mathname{#1}}

\makeatother
  
\newenvironment{math-function}[1]{
  \def\mathdefaultname{#1}
  \extractmathfunctionname{#1}
  \edef\mathurl{{math:\mathname}}\expandafter\hypertarget\expandafter{\mathurl}{}%
  \begin{pgfmanualentry}
    \pgfmanualentryheadline{\texttt{#1}}%
    \index{\mathname @\protect\texttt{\mathname} math function}%
    \index{Math functions!\mathname @\protect\texttt{\mathname}}%
    \pgfmanualbody
}
{
  \end{pgfmanualentry}
}

\def\pgfmanualemptytext{}
\def\pgfmanualvbarvbar{\char`\|\char`\|}

\newenvironment{math-operator}[4][]{%
  \begin{pgfmanualentry}
  \csname math#3operator\endcsname{#2}{#4}
  \def\mathtest{#4}%
  \ifx\mathtest\pgfmanualemptytext%
    \def\mathtype{(#3 operator)}
  \else%
    \def\mathtype{(#3 operator; uses the \texttt{#4} function)}
  \fi%
  \pgfmanualentryheadline{\mathexample\hfill\mathtype}%
  \def\mathtest{#1}%
  \ifx\mathtest\pgfmanualemptytext%
    \index{#2@\protect\texttt{#2} #3 math operator}%  
    \index{Math operators!#2@\protect\texttt{#2}}%
  \fi%
  \pgfmanualbody
}
{\end{pgfmanualentry}}

\newenvironment{math-operators}[5][]{%
  \begin{pgfmanualentry}
  \csname math#4operator\endcsname{#2}{#3}
  \def\mathtest{#5}%
  \ifx\mathtest\pgfmanualemptytext%
    \def\mathtype{(#4 operators)}
  \else%
    \def\mathtype{(#4 operators; use the \texttt{#5} function)}
  \fi%
  \pgfmanualentryheadline{\mathexample\hfill\mathtype}%
  \def\mathtest{#1}%
  \ifx\mathtest\pgfmanualemptytext%
    \index{#2#3@\protect\texttt{#2\protect\ #3} #4 math operators}% 
    \index{Math operators!#2#3@\protect\texttt{#2\protect\ #3}}%
  \fi%
  \pgfmanualbody
}
{\end{pgfmanualentry}}

\def\mathinfixoperator#1#2{%
  \def\mathoperator{\texttt{#1}}%
  \def\mathexample{\mvar{x}\space\texttt{#1}\space\mvar{y}}%
}

\def\mathprefixoperator#1#2{%
  \def\mathoperator{\texttt{#1}}%
  \def\mathexample{\texttt{#1}\mvar{x}}%
}

\def\mathpostfixoperator#1#2{%
  \def\mathoperator{\texttt{#1}}
  \def\mathexample{\mvar{x}\texttt{#1}}%
}

\def\mathgroupoperator#1#2{%
  \def\mathoperator{\texttt{#1\ #2}}%
  \def\mathexample{\texttt{#1}\mvar{x}\texttt{#2}}%
}

\expandafter\let\csname matharray accessoperator\endcsname=\mathgroupoperator
\expandafter\let\csname matharrayoperator\endcsname=\mathgroupoperator

\def\mathconditionaloperator#1#2{%
  \def\mathoperator{#1\space#2}
  \def\mathexample{\mvar{x}\ \texttt{#1}\ \mvar{y}\ {\texttt{#2}}\ \mvar{z}}
}

\newcommand\mathcommand[1][\mathdefaultname]{%
  \expandafter\makemathcommand#1(\empty)\stop%
  \expandafter\extractcommand\mathcommandname\@@%
  \medskip
}
\makeatletter

\def\makemathcommand#1(#2)#3\stop{%
  \expandafter\def\expandafter\mathcommandname\expandafter{\csname pgfmath#1\endcsname}%
  \ifx#2\empty%
  \else%
    \@makemathcommand#2,\stop,
  \fi}
\def\@makemathcommand#1,{%
  \ifx#1\stop%
  \else%
    \expandafter\def\expandafter\mathcommandname\expandafter{\mathcommandname{\ttfamily\char`\{#1\char`\}}}%
    \expandafter\@makemathcommand%
  \fi}
\makeatother

\def\calcname{\textsc{calc}}

\newenvironment{math-keyword}[1]{
  \extracttikzmathkeyword#1@
  \begin{pgfmanualentry}
    \pgfmanualentryheadline{\texttt{\color{red}\mathname}\mathrest}%
    \index{\mathname @\protect\texttt{\mathname} tikz math function}%
    \index{TikZ math functions!\mathname @\protect\texttt{\mathname}}%
    \pgfmanualbody
}
{
  \end{pgfmanualentry}
}

\def\extracttikzmathkeyword#1#2@{%
  \def\mathname{#1}%
  \def\mathrest{#2}%
}

%% MW: END MATH MACROS


\def\extractcommand#1#2\@@{%
  \removeats{#1}%
  \pgfmanualentryheadline{%
    \pgfmanualpdflabel{\textbackslash\strippedat}{}%
    \declare{\expandafter\texttt\expandafter{\string#1}}#2%
  }%
  \index{\strippedat @\protect\myprintocmmand{\strippedat}}
}

\def\luaextractcommand#1#2\relax{%
  \declare{\texttt{\string#1}}#2\par%
%  \removeats{#1}%
 % \index{\strippedat @\protect\myprintocmmand{\strippedat}}
 % \pgfmanualpdflabel{\textbackslash\strippedat}{}%
}


% \begin{environment}{{name}\marg{arguments}}
\renewenvironment{environment}[1]{
  \begin{pgfmanualentry}
    \extractenvironement#1\@@
    \pgfmanualbody
}
{
  \end{pgfmanualentry}
}

\def\extractenvironement#1#2\@@{%
  \pgfmanualentryheadline{%
    \pgfmanualpdflabel{#1}{}%
    {\ttfamily\char`\\begin\char`\{\declare{#1}\char`\}}#2%
  }%
  \pgfmanualentryheadline{{\ttfamily\ \ }\meta{environment contents}}%
  \pgfmanualentryheadline{{\ttfamily\char`\\end\char`\{\declare{#1}\char`\}}}%
  \index{#1@\protect\texttt{#1} environment}%
  \index{Environments!#1@\protect\texttt{#1}}
}


\newenvironment{plainenvironment}[1]{
  \begin{pgfmanualentry}
    \extractplainenvironement#1\@@
    \pgfmanualbody
}
{
  \end{pgfmanualentry}
}

\def\extractplainenvironement#1#2\@@{%
  \pgfmanualentryheadline{{\ttfamily\declare{\char`\\#1}}#2}%
  \pgfmanualentryheadline{{\ttfamily\ \ }\meta{environment contents}}%
  \pgfmanualentryheadline{{\ttfamily\declare{\char`\\end#1}}}%
  \index{#1@\protect\texttt{#1} environment}%
  \index{Environments!#1@\protect\texttt{#1}}%
}


\newenvironment{contextenvironment}[1]{
  \begin{pgfmanualentry}
    \extractcontextenvironement#1\@@
    \pgfmanualbody
}
{
  \end{pgfmanualentry}
}

\def\extractcontextenvironement#1#2\@@{%
  \pgfmanualentryheadline{{\ttfamily\declare{\char`\\start#1}}#2}%
  \pgfmanualentryheadline{{\ttfamily\ \ }\meta{environment contents}}%
  \pgfmanualentryheadline{{\ttfamily\declare{\char`\\stop#1}}}%
  \index{#1@\protect\texttt{#1} environment}%
  \index{Environments!#1@\protect\texttt{#1}}}


\newenvironment{shape}[1]{
  \begin{pgfmanualentry}
    \pgfmanualentryheadline{%
      \pgfmanualpdflabel{#1}{}%
      \textbf{Shape} {\ttfamily\declare{#1}}%
    }%
    \index{#1@\protect\texttt{#1} shape}%
    \index{Shapes!#1@\protect\texttt{#1}}
    \pgfmanualbody
}
{
  \end{pgfmanualentry}
}

\newenvironment{pictype}[2]{
  \begin{pgfmanualentry}
    \pgfmanualentryheadline{%
      \pgfmanualpdflabel{#1}{}%
      \textbf{Pic type} {\ttfamily\declare{#1}#2}%
    }%
    \index{#1@\protect\texttt{#1} pic type}%
    \index{Pic Types!#1@\protect\texttt{#1}}
    \pgfmanualbody
}
{
  \end{pgfmanualentry}
}

\newenvironment{shading}[1]{
  \begin{pgfmanualentry}
    \pgfmanualentryheadline{%
      \pgfmanualpdflabel{#1}{}%
      \textbf{Shading} {\ttfamily\declare{#1}}}%
    \index{#1@\protect\texttt{#1} shading}%
    \index{Shadings!#1@\protect\texttt{#1}}
    \pgfmanualbody
}
{
  \end{pgfmanualentry}
}


\newenvironment{graph}[1]{
  \begin{pgfmanualentry}
    \pgfmanualentryheadline{%
      \pgfmanualpdflabel{#1}{}%
      \textbf{Graph} {\ttfamily\declare{#1}}}%
    \index{#1@\protect\texttt{#1} graph}%
    \index{Graphs!#1@\protect\texttt{#1}}
    \pgfmanualbody
}
{
  \end{pgfmanualentry}
}

\newenvironment{gdalgorithm}[2]{
  \begin{pgfmanualentry}
    \pgfmanualentryheadline{%
      \pgfmanualpdflabel{#1}{}%
      \textbf{Layout} {\ttfamily/graph drawing/\declare{#1}\opt{=}}\opt{\meta{options}}}%
    \index{#1@\protect\texttt{#1} layout}%
    \index{Layouts!#1@\protect\texttt{#1}}%
    \foreach \algo in {#2}
    {\edef\marshal{\noexpand\index{#2@\noexpand\protect\noexpand\texttt{#2} algorithm}}\marshal}%
    \index{Graph drawing layouts!#1@\protect\texttt{#1}}
    \item{\small alias {\ttfamily/tikz/#1}}\par
    \item{\small alias {\ttfamily/tikz/graphs/#1}}\par
    \item{\small Employs {\ttfamily algorithm=#2}}\par
    \pgfmanualbody
}
{
  \end{pgfmanualentry}
}

\newenvironment{dataformat}[1]{
  \begin{pgfmanualentry}
    \pgfmanualentryheadline{%
      \pgfmanualpdflabel{#1}{}%
      \textbf{Format} {\ttfamily\declare{#1}}}%
    \index{#1@\protect\texttt{#1} format}%
    \index{Formats!#1@\protect\texttt{#1}}
    \pgfmanualbody
}
{
  \end{pgfmanualentry}
}

\newenvironment{stylesheet}[1]{
  \begin{pgfmanualentry}
    \pgfmanualentryheadline{%
      \pgfmanualpdflabel{#1}{}%
      \textbf{Style sheet} {\ttfamily\declare{#1}}}%
    \index{#1@\protect\texttt{#1} style sheet}%
    \index{Style sheets!#1@\protect\texttt{#1}}
    \pgfmanualbody
}
{
  \end{pgfmanualentry}
}

\newenvironment{handler}[1]{
  \begin{pgfmanualentry}
    \extracthandler#1\@nil%
    \pgfmanualbody
}
{
  \end{pgfmanualentry}
}

\def\gobble#1{}
\def\extracthandler#1#2\@nil{%
  \pgfmanualentryheadline{%
    \pgfmanualpdflabel{/handlers/#1}{}%
    \textbf{Key handler} \meta{key}{\ttfamily/\declare{#1}}#2}%
  \index{\gobble#1@\protect\texttt{#1} handler}%
  \index{Key handlers!#1@\protect\texttt{#1}}
}


\makeatletter


\newenvironment{stylekey}[1]{
  \begin{pgfmanualentry}
    \def\extrakeytext{style, }
    \extractkey#1\@nil%
    \pgfmanualbody
}
{
  \end{pgfmanualentry}
}

\def\choicesep{$\vert$}%
\def\choicearg#1{\texttt{#1}}

\newif\iffirstchoice

% \mchoice{choice1,choice2,choice3}
\newcommand\mchoice[1]{%
  \begingroup
  \firstchoicetrue
  \foreach \mchoice@ in {#1} {%
    \iffirstchoice
      \global\firstchoicefalse
    \else
      \choicesep
    \fi
    \choicearg{\mchoice@}%
  }%
  \endgroup
}%

% \begin{key}{/path/x=value}
% \begin{key}{/path/x=value (initially XXX)}
% \begin{key}{/path/x=value (default XXX)}
\newenvironment{key}[1]{
  \begin{pgfmanualentry}
    \def\extrakeytext{}
    %\def\altpath{\emph{\color{gray}or}}%
    \extractkey#1\@nil%
    \pgfmanualbody
}
{
  \end{pgfmanualentry}
}

% \insertpathifneeded{a key}{/pgf} -> assign mykey={/pgf/a key}
% \insertpathifneeded{/tikz/a key}{/pgf} -> assign mykey={/tikz/a key}
%
% #1: the key
% #2: a default path (or empty)
\def\insertpathifneeded#1#2{%
  \def\insertpathifneeded@@{#2}%
  \ifx\insertpathifneeded@@\empty
    \def\mykey{#1}%
  \else
    \insertpathifneeded@#2\@nil
    \ifpgfutil@in@
      \def\mykey{#2/#1}%
    \else
      \def\mykey{#1}%
    \fi
  \fi
}%
\def\insertpathifneeded@#1#2\@nil{%
  \def\insertpathifneeded@@{#1}%
  \def\insertpathifneeded@@@{/}%
  \ifx\insertpathifneeded@@\insertpathifneeded@@@
    \pgfutil@in@true
  \else
    \pgfutil@in@false
  \fi
}%

% \begin{keylist}[default path]
%   {/path/option 1=value,/path/option 2=value2}
% \end{keylist}
\newenvironment{keylist}[2][]{%
  \begin{pgfmanualentry}
    \def\extrakeytext{}%
  \foreach \xx in {#2} {%
    \expandafter\insertpathifneeded\expandafter{\xx}{#1}%
    \expandafter\extractkey\mykey\@nil%
  }%
  \pgfmanualbody
}{%
  \end{pgfmanualentry}
}%

\def\extractkey#1\@nil{%
  \pgfutil@in@={#1}%
  \ifpgfutil@in@%
    \extractkeyequal#1\@nil
  \else%
    \pgfutil@in@{(initial}{#1}%
    \ifpgfutil@in@%
      \extractequalinitial#1\@nil%
    \else
      \pgfmanualentryheadline{%
      \def\mykey{#1}%
      \def\mypath{}%
      \gdef\myname{}%
      \firsttimetrue%
      \pgfmanualdecomposecount=0\relax%
      \decompose#1/\nil%
        {\ttfamily\declare{#1}}\hfill(\extrakeytext no value)}%
    \fi
  \fi%
}

\def\extractkeyequal#1=#2\@nil{%
  \pgfutil@in@{(default}{#2}%
  \ifpgfutil@in@%
    \extractdefault{#1}#2\@nil%
  \else%
    \pgfutil@in@{(initial}{#2}%
    \ifpgfutil@in@%
      \extractinitial{#1}#2\@nil%
    \else
      \pgfmanualentryheadline{%
        \def\mykey{#1}%
        \def\mypath{}%
        \gdef\myname{}%
        \firsttimetrue%
        \pgfmanualdecomposecount=0\relax%
        \decompose#1/\nil%
        {\ttfamily\declare{#1}=}#2\hfill(\extrakeytext no default)}%
    \fi%
  \fi%
}

\def\extractdefault#1#2(default #3)\@nil{%
  \pgfmanualentryheadline{%
    \def\mykey{#1}%
    \def\mypath{}%
    \gdef\myname{}%
    \firsttimetrue%
    \pgfmanualdecomposecount=0\relax%
    \decompose#1/\nil%
    {\ttfamily\declare{#1}\opt{=}}\opt{#2}\hfill (\extrakeytext default {\ttfamily#3})}%
}

\def\extractinitial#1#2(initially #3)\@nil{%
  \pgfmanualentryheadline{%
    \def\mykey{#1}%
    \def\mypath{}%
    \gdef\myname{}%
    \firsttimetrue%
    \pgfmanualdecomposecount=0\relax%
    \decompose#1/\nil%
    {\ttfamily\declare{#1}=}#2\hfill (\extrakeytext no default, initially {\ttfamily#3})}%
}

\def\extractequalinitial#1 (initially #2)\@nil{%
  \pgfmanualentryheadline{%
    \def\mykey{#1}%
    \def\mypath{}%
    \gdef\myname{}%
    \firsttimetrue%
    \pgfmanualdecomposecount=0\relax%
    \decompose#1/\nil%
    {\ttfamily\declare{#1}}\hfill (\extrakeytext initially {\ttfamily#2})}%
}

% Introduces a key alias '/#1/<name of current key>'
% to be used inside of \begin{key} ... \end{key}
\def\keyalias#1{\vspace{-3pt}\item{\small alias {\ttfamily/#1/\myname}}\vspace{-2pt}\par
  \pgfmanualpdflabel{/#1/\myname}{}%
}

\newif\iffirsttime
\newcount\pgfmanualdecomposecount

\makeatother

\def\decompose/#1/#2\nil{%
  \def\test{#2}%
  \ifx\test\empty%
    % aha.
    \index{#1@\protect\texttt{#1} key}%
    \index{\mypath#1@\protect\texttt{#1}}%
    \gdef\myname{#1}%
    \pgfmanualpdflabel{#1}{}
  \else%
    \advance\pgfmanualdecomposecount by1\relax%
    \ifnum\pgfmanualdecomposecount>2\relax%
      \decomposetoodeep#1/#2\nil%
    \else%
      \iffirsttime%
        \begingroup%  
          % also make a pdf link anchor with full key path.
          \def\hyperlabelwithoutslash##1/\nil{%
            \pgfmanualpdflabel{##1}{}%
          }%
          \hyperlabelwithoutslash/#1/#2\nil%
        \endgroup%
        \def\mypath{#1@\protect\texttt{/#1/}!}%
        \firsttimefalse%
      \else%
        \expandafter\def\expandafter\mypath\expandafter{\mypath#1@\protect\texttt{#1/}!}%
      \fi%
      \def\firsttime{}%
      \decompose/#2\nil%
    \fi%
  \fi%
}

\def\decomposetoodeep#1/#2/\nil{%
  % avoid too-deep nesting in index
  \index{#1/#2@\protect\texttt{#1/#2} key}%
  \index{\mypath#1/#2@\protect\texttt{#1/#2}}%
  \decomposefindlast/#1/#2/\nil%
}
\makeatletter
\def\decomposefindlast/#1/#2\nil{%
  \def\test{#2}%
  \ifx\test\pgfutil@empty%
    \gdef\myname{#1}%
  \else%
    \decomposefindlast/#2\nil%
  \fi%
}
\makeatother
\def\indexkey#1{%
  \def\mypath{}%
  \decompose#1/\nil%
}

\newenvironment{predefinedmethod}[1]{
  \begin{pgfmanualentry}
    \extractpredefinedmethod#1\@nil
    \pgfmanualbody
}
{
  \end{pgfmanualentry}
}
\def\extractpredefinedmethod#1(#2)\@nil{%
  \pgfmanualentryheadline{%
    \pgfmanualpdflabel{#1}{}%
    Method \declare{\ttfamily #1}\texttt(#2\texttt) \hfill(predefined for all classes)}
  \index{#1@\protect\texttt{#1} method}%
  \index{Methods!#1@\protect\texttt{#1}}
}


\newenvironment{ooclass}[1]{
  \begin{pgfmanualentry}
    \def\currentclass{#1}
    \pgfmanualentryheadline{%
      \pgfmanualpdflabel{#1}{}%
      \textbf{Class} \declare{\texttt{#1}}}
    \index{#1@\protect\texttt{#1} class}%
    \index{Class #1@Class \protect\texttt{#1}}%
    \index{Classes!#1@\protect\texttt{#1}}
    \pgfmanualbody
}
{
  \end{pgfmanualentry}
}

\newenvironment{method}[1]{
  \begin{pgfmanualentry}
    \extractmethod#1\@nil
    \pgfmanualbody
}
{
  \end{pgfmanualentry}
}
\def\extractmethod#1(#2)\@nil{%
  \def\test{#1}
  \ifx\test\currentclass
    \pgfmanualentryheadline{%
      \pgfmanualpdflabel{#1}{}%
      Constructor \declare{\ttfamily #1}\texttt(#2\texttt)}
  \else
    \pgfmanualentryheadline{%
      \pgfmanualpdflabel{#1}{}%
      Method \declare{\ttfamily #1}\texttt(#2\texttt)}
  \fi
  \index{#1@\protect\texttt{#1} method}%
  \index{Methods!#1@\protect\texttt{#1}}
  \index{Class \currentclass!#1@\protect\texttt{#1}}%
}

\newenvironment{classattribute}[1]{
  \begin{pgfmanualentry}
    \extractattribute#1\@nil
    \pgfmanualbody
}
{
  \end{pgfmanualentry}
}
\def\extractattribute#1=#2;\@nil{%
  \def\test{#2}%
  \ifx\test\@empty
    \pgfmanualentryheadline{%
      \pgfmanualpdflabel{#1}{}%
      Private attribute \declare{\ttfamily #1} \hfill (initially empty)}
  \else
    \pgfmanualentryheadline{%
      \pgfmanualpdflabel{#1}{}%
      Private attribute \declare{\ttfamily #1} \hfill (initially {\ttfamily #2})}
  \fi
  \index{#1@\protect\texttt{#1} attribute}%
  \index{Attributes!#1@\protect\texttt{#1}}
  \index{Class \currentclass!#1@\protect\texttt{#1}}%
}



\newenvironment{predefinednode}[1]{
  \begin{pgfmanualentry}
    \pgfmanualentryheadline{%
      \pgfmanualpdflabel{#1}{}%
      \textbf{Predefined node} {\ttfamily\declare{#1}}}%
    \index{#1@\protect\texttt{#1} node}%
    \index{Predefined node!#1@\protect\texttt{#1}}
    \pgfmanualbody
}
{
  \end{pgfmanualentry}
}

\newenvironment{coordinatesystem}[1]{
  \begin{pgfmanualentry}
    \pgfmanualentryheadline{%
      \pgfmanualpdflabel{#1}{}%
      \textbf{Coordinate system} {\ttfamily\declare{#1}}}%
    \index{#1@\protect\texttt{#1} coordinate system}%
    \index{Coordinate systems!#1@\protect\texttt{#1}}
    \pgfmanualbody
}
{
  \end{pgfmanualentry}
}

\newenvironment{snake}[1]{
  \begin{pgfmanualentry}
    \pgfmanualentryheadline{\textbf{Snake} {\ttfamily\declare{#1}}}%
    \index{#1@\protect\texttt{#1} snake}%
    \index{Snakes!#1@\protect\texttt{#1}}
    \pgfmanualbody
}
{
  \end{pgfmanualentry}
}

\newenvironment{decoration}[1]{
  \begin{pgfmanualentry}
    \pgfmanualentryheadline{\textbf{Decoration} {\ttfamily\declare{#1}}}%
    \index{#1@\protect\texttt{#1} decoration}%
    \index{Decorations!#1@\protect\texttt{#1}}
    \pgfmanualbody
}
{
  \end{pgfmanualentry}
}


\def\pgfmanualbar{\char`\|}
\makeatletter
\newenvironment{pathoperation}[3][]{
  \begin{pgfmanualentry}
    \def\pgfmanualtest{#1}%
    \pgfmanualentryheadline{%
      \ifx\pgfmanualtest\@empty%
        \pgfmanualpdflabel{#2}{}%
      \fi%
      \textcolor{gray}{{\ttfamily\char`\\path}\
        \ \dots}
      \declare{\texttt{\noligs{#2}}}#3\ \textcolor{gray}{\dots\texttt{;}}}%
    \ifx\pgfmanualtest\@empty%
      \index{#2@\protect\texttt{#2} path operation}%
      \index{Path operations!#2@\protect\texttt{#2}}%
    \fi%
    \pgfmanualbody
}
{
  \end{pgfmanualentry}
}
\newenvironment{datavisualizationoperation}[3][]{
  \begin{pgfmanualentry}
    \def\pgfmanualtest{#1}%
    \pgfmanualentryheadline{%
      \ifx\pgfmanualtest\@empty%
        \pgfmanualpdflabel{#2}{}%
      \fi%
      \textcolor{gray}{{\ttfamily\char`\\datavisualization}\
        \ \dots}
      \declare{\texttt{\noligs{#2}}}#3\ \textcolor{gray}{\dots\texttt{;}}}%
    \ifx\pgfmanualtest\@empty%
      \index{#2@\protect\texttt{#2} (data visualization)}%
      \index{Data visualization!#2@\protect\texttt{#2}}%
    \fi%
    \pgfmanualbody
}
{
  \end{pgfmanualentry}
}
\makeatother

\def\doublebs{\texttt{\char`\\\char`\\}}


\newenvironment{package}[1]{
  \begin{pgfmanualentry}
    \pgfmanualentryheadline{%
      \pgfmanualpdflabel{#1}{}%
      {\ttfamily\char`\\usepackage\char`\{\declare{#1}\char`\}\space\space \char`\%\space\space  \LaTeX}}
    \index{#1@\protect\texttt{#1} package}%
    \index{Packages and files!#1@\protect\texttt{#1}}%
    \pgfmanualentryheadline{{\ttfamily\char`\\input \declare{#1}.tex\space\space\space \char`\%\space\space  plain \TeX}}
    \pgfmanualentryheadline{{\ttfamily\char`\\usemodule[\declare{#1}]\space\space \char`\%\space\space  Con\TeX t}}
    \pgfmanualbody
}
{
  \end{pgfmanualentry}
}


\newenvironment{pgfmodule}[1]{
  \begin{pgfmanualentry}
    \pgfmanualentryheadline{%
      \pgfmanualpdflabel{#1}{}%
      {\ttfamily\char`\\usepgfmodule\char`\{\declare{#1}\char`\}\space\space\space
        \char`\%\space\space  \LaTeX\space and plain \TeX\space and pure pgf}}
    \index{#1@\protect\texttt{#1} module}%
    \index{Modules!#1@\protect\texttt{#1}}%
    \pgfmanualentryheadline{{\ttfamily\char`\\usepgfmodule[\declare{#1}]\space\space \char`\%\space\space  Con\TeX t\space and pure pgf}}
    \pgfmanualbody
}
{
  \end{pgfmanualentry}
}

\newenvironment{pgflibrary}[1]{
  \begin{pgfmanualentry}
    \pgfmanualentryheadline{%
      \pgfmanualpdflabel{#1}{}%
      \textbf{\tikzname\ Library} \texttt{\declare{#1}}}
    \index{#1@\protect\texttt{#1} library}%
    \index{Libraries!#1@\protect\texttt{#1}}%
    \vskip.25em%
    {{\ttfamily\char`\\usepgflibrary\char`\{\declare{#1}\char`\}\space\space\space
        \char`\%\space\space  \LaTeX\space and plain \TeX\space and pure pgf}}\\
    {{\ttfamily\char`\\usepgflibrary[\declare{#1}]\space\space \char`\%\space\space  Con\TeX t\space and pure pgf}}\\
    {{\ttfamily\char`\\usetikzlibrary\char`\{\declare{#1}\char`\}\space\space
        \char`\%\space\space  \LaTeX\space and plain \TeX\space when using \tikzname}}\\
    {{\ttfamily\char`\\usetikzlibrary[\declare{#1}]\space
        \char`\%\space\space  Con\TeX t\space when using \tikzname}}\\[.5em]
    \pgfmanualbody
}
{
  \end{pgfmanualentry}
}

\newenvironment{purepgflibrary}[1]{
  \begin{pgfmanualentry}
    \pgfmanualentryheadline{%
      \pgfmanualpdflabel{#1}{}%
      \textbf{{\small PGF} Library} \texttt{\declare{#1}}}
    \index{#1@\protect\texttt{#1} library}%
    \index{Libraries!#1@\protect\texttt{#1}}%
    \vskip.25em%
    {{\ttfamily\char`\\usepgflibrary\char`\{\declare{#1}\char`\}\space\space\space
        \char`\%\space\space  \LaTeX\space and plain \TeX}}\\
    {{\ttfamily\char`\\usepgflibrary[\declare{#1}]\space\space \char`\%\space\space  Con\TeX t}}\\[.5em]
    \pgfmanualbody
}
{
  \end{pgfmanualentry}
}

\newenvironment{tikzlibrary}[1]{
  \begin{pgfmanualentry}
    \pgfmanualentryheadline{%
      \pgfmanualpdflabel{#1}{}%
      \textbf{\tikzname\ Library} \texttt{\declare{#1}}}
    \index{#1@\protect\texttt{#1} library}%
    \index{Libraries!#1@\protect\texttt{#1}}%
    \vskip.25em%
    {{\ttfamily\char`\\usetikzlibrary\char`\{\declare{#1}\char`\}\space\space \char`\%\space\space  \LaTeX\space and plain \TeX}}\\
    {{\ttfamily\char`\\usetikzlibrary[\declare{#1}]\space \char`\%\space\space Con\TeX t}}\\[.5em]
    \pgfmanualbody
}
{
  \end{pgfmanualentry}
}



\newenvironment{filedescription}[1]{
  \begin{pgfmanualentry}
    \pgfmanualentryheadline{File {\ttfamily\declare{#1}}}%
    \index{#1@\protect\texttt{#1} file}%
    \index{Packages and files!#1@\protect\texttt{#1}}%
    \pgfmanualbody
}
{
  \end{pgfmanualentry}
}


\newenvironment{packageoption}[1]{
  \begin{pgfmanualentry}
    \pgfmanualentryheadline{{\ttfamily\char`\\usepackage[\declare{#1}]\char`\{pgf\char`\}}}
    \index{#1@\protect\texttt{#1} package option}%
    \index{Package options for \textsc{pgf}!#1@\protect\texttt{#1}}%
    \pgfmanualbody
}
{
  \end{pgfmanualentry}
}



\newcommand\opt[1]{{\color{black!50!green}#1}}
\newcommand\ooarg[1]{{\ttfamily[}\meta{#1}{\ttfamily]}}

\def\opt{\afterassignment\pgfmanualopt\let\next=}
\def\pgfmanualopt{\ifx\next\bgroup\bgroup\color{black!50!green}\else{\color{black!50!green}\next}\fi}



\def\beamer{\textsc{beamer}}
\def\pdf{\textsc{pdf}}
\def\eps{\texttt{eps}}
\def\pgfname{\textsc{pgf}}
\def\tikzname{Ti\emph{k}Z}
\def\pstricks{\textsc{pstricks}}
\def\prosper{\textsc{prosper}}
\def\seminar{\textsc{seminar}}
\def\texpower{\textsc{texpower}}
\def\foils{\textsc{foils}}

{
  \makeatletter
  \global\let\myempty=\@empty
  \global\let\mygobble=\@gobble
  \catcode`\@=12
  \gdef\getridofats#1@#2\relax{%
    \def\getridtest{#2}%
    \ifx\getridtest\myempty%
      \expandafter\def\expandafter\strippedat\expandafter{\strippedat#1}
    \else%
      \expandafter\def\expandafter\strippedat\expandafter{\strippedat#1\protect\printanat}
      \getridofats#2\relax%
    \fi%
  }

  \gdef\removeats#1{%
    \let\strippedat\myempty%
    \edef\strippedtext{\stripcommand#1}%
    \expandafter\getridofats\strippedtext @\relax%
  }
  
  \gdef\stripcommand#1{\expandafter\mygobble\string#1}
}

\def\printanat{\char`\@}

\def\declare{\afterassignment\pgfmanualdeclare\let\next=}
\def\pgfmanualdeclare{\ifx\next\bgroup\bgroup\color{red!75!black}\else{\color{red!75!black}\next}\fi}


\let\textoken=\command
\let\endtextoken=\endcommand

\def\myprintocmmand#1{\texttt{\char`\\#1}}

\def\example{\par\smallskip\noindent\textit{Example: }}
\def\themeauthor{\par\smallskip\noindent\textit{Theme author: }}


\def\indexoption#1{%
  \index{#1@\protect\texttt{#1} option}%
  \index{Graphic options and styles!#1@\protect\texttt{#1}}%
}

\def\itemcalendaroption#1{\item \declare{\texttt{#1}}%
  \index{#1@\protect\texttt{#1} date test}%
  \index{Date tests!#1@\protect\texttt{#1}}%
}



\def\class#1{\list{}{\leftmargin=2em\itemindent-\leftmargin\def\makelabel##1{\hss##1}}%
\extractclass#1@\par\topsep=0pt}
\def\endclass{\endlist}
\def\extractclass#1#2@{%
\item{{{\ttfamily\char`\\documentclass}#2{\ttfamily\char`\{\declare{#1}\char`\}}}}%
  \index{#1@\protect\texttt{#1} class}%
  \index{Classes!#1@\protect\texttt{#1}}}

\def\partname{Part}

\makeatletter
\def\index@prologue{\section*{Index}\addcontentsline{toc}{section}{Index}
  This index only contains automatically generated entries. A good
  index should also contain carefully selected keywords. This index is
  not a good index.
  \bigskip
}
\c@IndexColumns=2
  \def\theindex{\@restonecoltrue
    \columnseprule \z@  \columnsep 29\p@
    \twocolumn[\index@prologue]%
       \parindent -30pt
       \columnsep 15pt
       \parskip 0pt plus 1pt
       \leftskip 30pt
       \rightskip 0pt plus 2cm
       \small
       \def\@idxitem{\par}%
    \let\item\@idxitem \ignorespaces}
  \def\endtheindex{\onecolumn}
\def\noindexing{\let\index=\@gobble}


\newenvironment{arrowtipsimple}[1]{
  \begin{pgfmanualentry}
    \pgfmanualentryheadline{\textbf{Arrow Tip Kind} {\ttfamily#1}}
    \index{#1@\protect\texttt{#1} arrow tip}%
    \index{Arrow tips!#1@\protect\texttt{#1}}%
    \def\currentarrowtype{#1}
    \pgfmanualbody}
{
  \end{pgfmanualentry}
}

\newenvironment{arrowtip}[4]{
  \begin{pgfmanualentry}
    \pgfmanualentryheadline{\textbf{Arrow Tip Kind} {\ttfamily#1}}
    \index{#1@\protect\texttt{#1} arrow tip}%
    \index{Arrow tips!#1@\protect\texttt{#1}}%
    \pgfmanualbody
    \def\currentarrowtype{#1}
    \begin{minipage}[t]{10.25cm}
      #2
    \end{minipage}\hskip5mm\begin{minipage}[t]{4.75cm}
      \leavevmode\vskip-2em
    \tikz{
      \draw [black!50,line width=5mm,-{#1[#3,color=black]}] (-4,0) -- (0,0);
      \foreach \action in {#4}
      { \expandafter\processaction\action\relax }
    }
    \end{minipage}\par\smallskip
  }
{
  \end{pgfmanualentry}
}

\newenvironment{arrowcap}[5]{
  \begin{pgfmanualentry}
    \pgfmanualentryheadline{\textbf{Arrow Tip Kind} {\ttfamily#1}}
    \index{#1@\protect\texttt{#1} arrow tip}%
    \index{Arrow tips!#1@\protect\texttt{#1}}%
    \pgfmanualbody
    \def\currentarrowtype{#1}
    \begin{minipage}[t]{10.25cm}
      #2
    \end{minipage}\hskip5mm\begin{minipage}[t]{4.75cm}
      \leavevmode\vskip-2em
    \tikz{
      \path [tips, line width=10mm,-{#1[#3,color=black]}] (-4,0) -- (0,0);
      \draw [line width=10mm,black!50] (-3,0) -- (#5,0);
      \foreach \action in {#4}
      { \expandafter\processaction\action\relax }
    }
    \end{minipage}\par\smallskip
  }
{
  \end{pgfmanualentry}
}

\newenvironment{pattern}[1]{
  \begin{pgfmanualentry}
    \pgfmanualentryheadline{\textbf{Pattern} {\ttfamily#1}}
    \index{#1@\protect\texttt{#1} pattern}%
    \index{Patterns!#1@\protect\texttt{#1}}%
    \pgfmanualbody
}
{
  \end{pgfmanualentry}
}

\def\processaction#1=#2\relax{
  \expandafter\let\expandafter\pgf@temp\csname manual@action@#1\endcsname
  \ifx\pgf@temp\relax\else
    \pgf@temp#2/0/\relax
  \fi
}
\def\manual@action@length#1/#2/#3\relax{%
  \draw [red,|<->|,semithick,xshift=#2] ([yshift=4pt]current bounding
  box.north -| -#1,0) coordinate (last length) -- node
  [above=-2pt] {|length|} ++(#1,0);
}
\def\manual@action@width#1/#2/#3\relax{%
  \draw [overlay, red,|<->|,semithick] (.5,-#1/2) -- node [below,sloped] {|width|} (.5,#1/2);
}
\def\manual@action@inset#1/#2/#3\relax{%
  \draw [red,|<->|,semithick,xshift=#2] ([yshift=-4pt]current bounding
  box.south -| last length) -- node [below] {|inset|} ++(#1,0);
}

\newenvironment{arrowexamples}
{\begin{tabbing}
    \hbox to \dimexpr\linewidth-5.5cm\relax{\emph{Appearance of the below at line width} \hfil} \= 
     \hbox to 1.9cm{\emph{0.4pt}\hfil} \= \hbox to 2cm{\emph{0.8pt}\hfil} \= \emph{1.6pt} \\
  }
{\end{tabbing}\vskip-1em}

\newenvironment{arrowcapexamples}
{\begin{tabbing}
    \hbox to \dimexpr\linewidth-5.5cm\relax{\emph{Appearance of the below at line width} \hfil} \= 
     \hbox to 1.9cm{\emph{1ex}\hfil} \= \hbox to 2cm{\emph{1em}\hfil} \\
  }
{\end{tabbing}\vskip-1em}

\def\arrowcapexample#1[#2]{\def\temp{#1}\ifx\temp\pgfutil@empty\arrowcapexample@\currentarrowtype[{#2}]\else\arrowcapexample@#1[{#2}]\fi}
\def\arrowcapexample@#1[#2]{%
  {\sfcode`\.1000\small\texttt{#1[#2]}} \>
  \kern-.5ex\tikz [baseline,>={#1[#2]}] \draw [line
  width=1ex,->] (0,.5ex) -- (2em,.5ex);  \>
  \kern-.5em\tikz [baseline,>={#1[#2]}] \draw [line
  width=1em,->] (0,.5ex) -- (2em,.5ex);  \\
}

\def\arrowexample#1[#2]{\def\temp{#1}\ifx\temp\pgfutil@empty\arrowexample@\currentarrowtype[{#2}]\else\arrowexample@#1[{#2}]\fi}
\def\arrowexample@#1[#2]{%
  {\sfcode`\.1000\small\texttt{#1[#2]}} \>
  \tikz [baseline,>={#1[#2]}] \draw [line
  width=0.4pt,->] (0,.5ex) -- (2em,.5ex); thin \>
  \tikz [baseline,>={#1[#2]}] \draw [line
  width=0.8pt,->] (0,.5ex) -- (2em,.5ex); \textbf{thick} \>
  \tikz [baseline,>={#1[#2]}] \draw [line
  width=1.6pt,->] (0,.5ex) -- (3em,.5ex); \\
}
\def\arrowexampledup[#1]{\arrowexample[{#1] \currentarrowtype[}]}
\def\arrowexampledupdot[#1]{\arrowexample[{#1] . \currentarrowtype[}]}

\def\arrowexampledouble#1[#2]{\def\temp{#1}\ifx\temp\pgfutil@empty\arrowexampledouble@\currentarrowtype[{#2}]\else\arrowexampledouble@#1[{#2}]\fi}
\def\arrowexampledouble@#1[#2]{%
  {\sfcode`\.1000\small\texttt{#1[#2]} on double line} \>
  \tikz [baseline,>={#1[#2]}]
    \draw [double equal sign distance,line width=0.4pt,->] (0,.5ex) -- (2em,.5ex); thin \>
  \tikz [baseline,>={#1[#2]}]
    \draw [double equal sign distance,line width=0.8pt,->] (0,.5ex) -- (2em,.5ex); \textbf{thick} \>
  \tikz [baseline,>={#1[#2]}]
    \draw [double equal sign distance, line width=1.6pt,->] (0,.5ex) -- (3em,.5ex); \\
}



\newcommand\symarrow[1]{%
  \index{#1@\protect\texttt{#1} arrow tip}%
  \index{Arrow tips!#1@\protect\texttt{#1}}%
  \texttt{#1}& yields thick  
  \begin{tikzpicture}[arrows={#1-#1},thick,baseline]
    \useasboundingbox (-1mm,-0.5ex) rectangle (1.1cm,2ex);
    \fill [black!15] (1cm,-.5ex) rectangle (1.1cm,1.5ex) (-1mm,-.5ex) rectangle (0mm,1.5ex) ;
    \draw (0pt,.5ex) -- (1cm,.5ex);
  \end{tikzpicture} and thin
  \begin{tikzpicture}[arrows={#1-#1},thin,baseline]
    \useasboundingbox (-1mm,-0.5ex) rectangle (1.1cm,2ex);
    \fill [black!15] (1cm,-.5ex) rectangle (1.1cm,1.5ex) (-1mm,-.5ex) rectangle (0mm,1.5ex) ;
    \draw (0pt,.5ex) -- (1cm,.5ex);
  \end{tikzpicture}
}
\newcommand\symarrowdouble[1]{%
  \index{#1@\protect\texttt{#1} arrow tip}%
  \index{Arrow tips!#1@\protect\texttt{#1}}%
  \texttt{#1}& yields thick  
  \begin{tikzpicture}[arrows={#1-#1},thick,baseline]
    \useasboundingbox (-1mm,-0.5ex) rectangle (1.1cm,2ex);
    \fill [black!15] (1cm,-.5ex) rectangle (1.1cm,1.5ex) (-1mm,-.5ex) rectangle (0mm,1.5ex) ;
    \draw (0pt,.5ex) -- (1cm,.5ex);
  \end{tikzpicture}
  and thin
  \begin{tikzpicture}[arrows={#1-#1},thin,baseline]
    \useasboundingbox (-1mm,-0.5ex) rectangle (1.1cm,2ex);
    \fill [black!15] (1cm,-.5ex) rectangle (1.1cm,1.5ex) (-1mm,-.5ex) rectangle (0mm,1.5ex) ;
    \draw (0pt,.5ex) -- (1cm,.5ex);
  \end{tikzpicture}, double 
  \begin{tikzpicture}[arrows={#1-#1},thick,baseline]
    \useasboundingbox (-1mm,-0.5ex) rectangle (1.1cm,2ex);
    \fill [black!15] (1cm,-.5ex) rectangle (1.1cm,1.5ex) (-1mm,-.5ex) rectangle (0mm,1.5ex) ;
    \draw[double,double equal sign distance] (0pt,.5ex) -- (1cm,.5ex);
  \end{tikzpicture} and 
  \begin{tikzpicture}[arrows={#1-#1},thin,baseline]
    \useasboundingbox (-1mm,-0.5ex) rectangle (1.1cm,2ex);
    \fill [black!15] (1cm,-.5ex) rectangle (1.1cm,1.5ex) (-1mm,-.5ex) rectangle (0mm,1.5ex) ;
    \draw[double,double equal sign distance] (0pt,.5ex) -- (1cm,.5ex);
  \end{tikzpicture}
}

\newcommand\sarrow[2]{%
  \index{#1@\protect\texttt{#1} arrow tip}%
  \index{Arrow tips!#1@\protect\texttt{#1}}%
  \index{#2@\protect\texttt{#2} arrow tip}%
  \index{Arrow tips!#2@\protect\texttt{#2}}%
  \texttt{#1-#2}& yields thick  
  \begin{tikzpicture}[arrows={#1-#2},thick,baseline]
    \useasboundingbox (-1mm,-0.5ex) rectangle (1.1cm,2ex);
    \fill [black!15] (1cm,-.5ex) rectangle (1.1cm,1.5ex) (-1mm,-.5ex) rectangle (0mm,1.5ex) ;
    \draw (0pt,.5ex) -- (1cm,.5ex);
  \end{tikzpicture} and thin
  \begin{tikzpicture}[arrows={#1-#2},thin,baseline]
    \useasboundingbox (-1mm,-0.5ex) rectangle (1.1cm,2ex);
    \fill [black!15] (1cm,-.5ex) rectangle (1.1cm,1.5ex) (-1mm,-.5ex) rectangle (0mm,1.5ex) ;
    \draw (0pt,.5ex) -- (1cm,.5ex);
  \end{tikzpicture}
}

\newcommand\sarrowdouble[2]{%
  \index{#1@\protect\texttt{#1} arrow tip}%
  \index{Arrow tips!#1@\protect\texttt{#1}}%
  \index{#2@\protect\texttt{#2} arrow tip}%
  \index{Arrow tips!#2@\protect\texttt{#2}}%
  \texttt{#1-#2}& yields thick  
  \begin{tikzpicture}[arrows={#1-#2},thick,baseline]
    \useasboundingbox (-1mm,-0.5ex) rectangle (1.1cm,2ex);
    \fill [black!15] (1cm,-.5ex) rectangle (1.1cm,1.5ex) (-1mm,-.5ex) rectangle (0mm,1.5ex) ;
    \draw (0pt,.5ex) -- (1cm,.5ex);
  \end{tikzpicture} and thin
  \begin{tikzpicture}[arrows={#1-#2},thin,baseline]
    \useasboundingbox (-1mm,-0.5ex) rectangle (1.1cm,2ex);
    \fill [black!15] (1cm,-.5ex) rectangle (1.1cm,1.5ex) (-1mm,-.5ex) rectangle (0mm,1.5ex) ;
    \draw (0pt,.5ex) -- (1cm,.5ex);
  \end{tikzpicture}, double 
  \begin{tikzpicture}[arrows={#1-#2},thick,baseline]
    \useasboundingbox (-1mm,-0.5ex) rectangle (1.1cm,2ex);
    \fill [black!15] (1cm,-.5ex) rectangle (1.1cm,1.5ex) (-1mm,-.5ex) rectangle (0mm,1.5ex) ;
    \draw[double,double equal sign distance] (0pt,.5ex) -- (1cm,.5ex);
  \end{tikzpicture} and 
  \begin{tikzpicture}[arrows={#1-#2},thin,baseline]
    \useasboundingbox (-1mm,-0.5ex) rectangle (1.1cm,2ex);
    \fill [black!15] (1cm,-.5ex) rectangle (1.1cm,1.5ex) (-1mm,-.5ex) rectangle (0mm,1.5ex) ;
    \draw[double,double equal sign distance] (0pt,.5ex) -- (1cm,.5ex);
  \end{tikzpicture}
}

\newcommand\carrow[1]{%
  \index{#1@\protect\texttt{#1} arrow tip}%
  \index{Arrow tips!#1@\protect\texttt{#1}}%
  \texttt{#1}& yields for line width 1ex
  \begin{tikzpicture}[arrows={#1-#1},line width=1ex,baseline]
    \useasboundingbox (-1mm,-0.5ex) rectangle (1.6cm,2ex);
    \fill [black!15] (1.5cm,-.5ex) rectangle (1.6cm,1.5ex) (-1mm,-.5ex) rectangle (0mm,1.5ex) ;
    \draw (0pt,.5ex) -- (1.5cm,.5ex);
  \end{tikzpicture}
}
\def\myvbar{\char`\|}
\newcommand\plotmarkentry[1]{%
  \index{#1@\protect\texttt{#1} plot mark}%
  \index{Plot marks!#1@\protect\texttt{#1}}
  \texttt{\char`\\pgfuseplotmark\char`\{\declare{\noligs{#1}}\char`\}} &
  \tikz\draw[color=black!25] plot[mark=#1,mark options={fill=examplefill,draw=black}] coordinates{(0,0) (.5,0.2) (1,0) (1.5,0.2)};\\
}
\newcommand\plotmarkentrytikz[1]{%
  \index{#1@\protect\texttt{#1} plot mark}%
  \index{Plot marks!#1@\protect\texttt{#1}}
  \texttt{mark=\declare{\noligs{#1}}} & \tikz\draw[color=black!25]
  plot[mark=#1,mark options={fill=examplefill,draw=black}] 
    coordinates {(0,0) (.5,0.2) (1,0) (1.5,0.2)};\\
}



\ifx\scantokens\@undefined
  \PackageError{pgfmanual-macros}{You need to use extended latex
    (elatex) or (pdfelatex) to process this document}{}
\fi

\begingroup
\catcode`|=0
\catcode`[= 1
\catcode`]=2
\catcode`\{=12
\catcode `\}=12
\catcode`\\=12 |gdef|find@example#1\end{codeexample}[|endofcodeexample[#1]]
|endgroup

% define \returntospace.
%
% It should define NEWLINE as {}, spaces and tabs as \space.
\begingroup
\catcode`\^=7
\catcode`\^^M=13
\catcode`\^^I=13
\catcode`\ =13%
\gdef\returntospace{\catcode`\ =13\def {\space}\catcode`\^^I=13\def^^I{\space}}
\gdef\showreturn{\show^^M}
\endgroup

\begingroup
\catcode`\%=13
\catcode`\^^M=13
\gdef\commenthandler{\catcode`\%=13\def%{\@gobble@till@return}}
\gdef\@gobble@till@return#1^^M{}
\gdef\@gobble@till@return@ignore#1^^M{\ignorespaces}
\gdef\typesetcomment{\catcode`\%=13\def%{\@typeset@till@return}}
\gdef\@typeset@till@return#1^^M{{\def%{\char`\%}\textsl{\char`\%#1}}\par}
\endgroup

% Define tab-implementation functions
%   \codeexample@tabinit@replacementchars@
% and
%   \codeexample@tabinit@catcode@
%
% They should ONLY be used in case that tab replacement is active.
%
% This here is merely a preparation step.
%
% Idea:
% \codeexample@tabinit@catcode@ will make TAB active
% and
% \codeexample@tabinit@replacementchars@ will insert as many spaces as
% /codeexample/tabsize contains.
{
\catcode`\^^I=13
% ATTENTION: do NOT use tabs in these definitions!!
\gdef\codeexample@tabinit@replacementchars@{%
 \begingroup
 \count0=\pgfkeysvalueof{/codeexample/tabsize}\relax
 \toks0={}%
 \loop
 \ifnum\count0>0
  \advance\count0 by-1
  \toks0=\expandafter{\the\toks0\ }%
 \repeat
 \xdef\codeexample@tabinit@replacementchars@@{\the\toks0}%
 \endgroup
 \let^^I=\codeexample@tabinit@replacementchars@@
}%
\gdef\codeexample@tabinit@catcode@{\catcode`\^^I=13}%
}%

% Called after any options have been set. It assigns
%   \codeexample@tabinit@catcode
% and
%   \codeexample@tabinit@replacementchars
% which are used inside of 
%\begin{codeexample}
% ...
%\end{codeexample}
%
% \codeexample@tabinit@catcode  is either \relax or it makes tab
% active.
%
% \codeexample@tabinit@replacementchars is either \relax or it inserts
% a proper replacement sequence for tabs (as many spaces as
% configured)
\def\codeexample@tabinit{%
  \ifnum\pgfkeysvalueof{/codeexample/tabsize}=0\relax
    \let\codeexample@tabinit@replacementchars=\relax
    \let\codeexample@tabinit@catcode=\relax
  \else
    \let\codeexample@tabinit@catcode=\codeexample@tabinit@catcode@
    \let\codeexample@tabinit@replacementchars=\codeexample@tabinit@replacementchars@
  \fi
}

\newif\ifpgfmanualtikzsyntaxhilighting

\pgfqkeys{/codeexample}{%
  width/.code=  {\setlength\codeexamplewidth{#1}},
  graphic/.code=  {\colorlet{graphicbackground}{#1}},
  code/.code=  {\colorlet{codebackground}{#1}},
  execute code/.is if=code@execute,
  hidden/.is if=code@hidden,
  code only/.code=  {\code@executefalse},
  setup code/.code=  {\pgfmanual@setup@codetrue\code@executefalse},
  multipage/.code=  {\code@executefalse\pgfmanual@multipage@codetrue},
  pre/.store in=\code@pre,
  post/.store in=\code@post,
  % #1 is the *complete* environment contents as it shall be
  % typeset. In particular, the catcodes are NOT the normal ones.
  typeset listing/.code=  {#1},
  render instead/.store in=\code@render,
  vbox/.code=  {\def\code@pre{\vbox\bgroup\setlength{\hsize}{\linewidth-6pt}}\def\code@post{\egroup}},
  ignorespaces/.code=  {\let\@gobble@till@return=\@gobble@till@return@ignore},
  leave comments/.code=  {\def\code@catcode@hook{\catcode`\%=12}\let\commenthandler=\relax\let\typesetcomment=\relax},
  tabsize/.initial=0,% FIXME : this here is merely used for indentation. It is just a TAB REPLACEMENT.
  every codeexample/.style={width=4cm+7pt, tikz syntax=true},
  from file/.code={\codeexamplefromfiletrue\def\codeexamplesource{#1}},
  tikz syntax/.is if=pgfmanualtikzsyntaxhilighting,
  animation list/.store in=\code@animation@list,
  animation pre/.store in=\code@animation@pre,
  animation post/.store in=\code@animation@post,
  animation scale/.store in=\pgfmanualanimscale,
  animation bb/.style={
    animation pre={
      \tikzset{
        every picture/.style={
          execute at begin picture={
            \useasboundingbox[clip] #1;}
        }
      }
    }
  },
  preamble/.store in=\code@preamble,
}

\def\pgfmanualanimscale{.5}

\newread\examplesource


% Opening, reading and closing the results file

\def\opensource#1{
  \immediate\openin\examplesource=#1
}
\def\do@codeexamplefromfile{%
  \immediate\openin\examplesource\expandafter{\codeexamplesource}%
  \def\examplelines{}%
  \readexamplelines
  \closein\examplesource
  \expandafter\endofcodeexample\expandafter{\examplelines}%
}

\def\readexamplelines{
  \ifeof\examplesource%
  \else
    \immediate\read\examplesource to \exampleline
    \expandafter\expandafter\expandafter\def\expandafter\expandafter\expandafter\examplelines\expandafter\expandafter\expandafter{\expandafter\examplelines\exampleline}
    \expandafter\readexamplelines%
  \fi
}

\let\code@animation@pre\pgfutil@empty
\let\code@animation@post\pgfutil@empty
\let\code@animation@list\pgfutil@empty

\let\code@pre\pgfutil@empty
\let\code@post\pgfutil@empty
\let\code@render\pgfutil@empty
\let\code@preamble\pgfutil@empty
\def\code@catcode@hook{}

\newif\ifpgfmanual@multipage@code
\newif\ifpgfmanual@setup@code
\newif\ifcodeexamplefromfile
\newdimen\codeexamplewidth
\newif\ifcode@execute
\newif\ifcode@hidden
\newbox\codeexamplebox
\def\codeexample[#1]{%
  \global\let\pgfmanual@do@this\relax%
  \aftergroup\pgfmanual@do@this%
  \begingroup%
  \code@executetrue
  \pgfqkeys{/codeexample}{every codeexample,#1}%
  \pgfmanualswitchoncolors%
  \ifcodeexamplefromfile\begingroup\fi
  \codeexample@tabinit% assigns \codeexample@tabinit@[catcode,replacementchars]
  \parindent0pt
  \begingroup%
  \par% this \par is not inside \ifcode@hidden because we want to switch to vmode
  \ifcode@hidden\else
    \medskip%
  \fi
  \let\do\@makeother%
  \dospecials%
  \obeylines%
  \@vobeyspaces%
  \catcode`\%=13%
  \catcode`\^^M=13%
  \code@catcode@hook%
  \codeexample@tabinit@catcode
  \relax%
  \ifcodeexamplefromfile%
    \expandafter\do@codeexamplefromfile%
  \else%
    \expandafter\find@example%
  \fi}
\def\endofcodeexample#1{%
  \endgroup%
  \ifpgfmanual@setup@code%
    \gdef\pgfmanual@do@this{%
      {%
        \returntospace%
        \commenthandler%
        \xdef\code@temp{#1}% removes returns and comments
      }%
      \edef\pgfmanualmcatcode{\the\catcode`\^^M}%
      \catcode`\^^M=9\relax%
      \expandafter\scantokens\expandafter{\code@temp}%
      \catcode`\^^M=\pgfmanualmcatcode%
    }%
  \fi%
  \ifcode@hidden\else
    \ifcode@execute%
      \setbox\codeexamplebox=\hbox{%
        \ifx\code@render\pgfutil@empty%
        {%
          {%
            \returntospace%
            \commenthandler%
            \xdef\code@temp{#1}% removes returns and comments
          }%
          \catcode`\^^M=9%
          \colorbox{graphicbackground}{\color{black}\ignorespaces%
            \code@pre\expandafter\scantokens\expandafter{\code@temp\ignorespaces}\code@post\ignorespaces}%
        }%
        \else%
          \global\let\code@temp\code@render%
          \colorbox{graphicbackground}{\color{black}\ignorespaces%
            \code@render}%
        \fi%
      }%
      \ifx\code@animation@list\pgfutil@empty%
      \else%
      \setbox\codeexampleboxanim=\vbox{%
        \rightskip0pt\leftskip0pt plus1filll%
        \ifdim\wd\codeexamplebox>\codeexamplewidth%
        \else%
          \hsize\codeexamplewidth%
          \advance\hsize by2cm%
        \fi%
        \leavevmode\catcode`\^^M=9%
        \foreach \pgfmanualtime/\pgfmanualtimehow in\code@animation@list{%
          \setbox\codeexampleboxanim=\hbox{\colorbox{animationgraphicbackground}{%
              \tikzset{make snapshot of=\pgfmanualtime}%
              \scalebox{\pgfmanualanimscale}{\color{black}\ignorespaces%
                \code@animation@pre\expandafter\scantokens\expandafter{\code@temp\ignorespaces}\code@animation@post\ignorespaces}%
            }}%
          \space\raise4pt\hbox to0pt{\vrule width0pt height1em\hbox
            to\wd\codeexampleboxanim{\hfil\scriptsize$t{=}\pgfmanualtimehow \mathrm s$\hfil}\hss}%
          \lower\ht\codeexampleboxanim\box\codeexampleboxanim\hfil\penalty0\hskip0ptplus-1fil%
        }%
      }%
      \setbox\codeexampleboxanim=\hbox{\hbox{}\hskip-2cm\box\codeexampleboxanim}%
      \fi%
      \ifdim\wd\codeexamplebox>\codeexamplewidth%
        \def\code@start{\par}%
        \def\code@flushstart{}\def\code@flushend{}%
        \def\code@mid{\parskip2pt\par\noindent}%
        \def\code@width{\linewidth-6pt}%
        \def\code@end{}%
      \else%
        \def\code@start{%
          \linewidth=\textwidth%
          \parshape \@ne 0pt \linewidth
          \leavevmode%
          \hbox\bgroup}%
        \def\code@flushstart{\hfill}%
        \def\code@flushend{\hbox{}}%
        \def\code@mid{\hskip6pt}%
        \def\code@width{\linewidth-12pt-\codeexamplewidth}%
        \def\code@end{\egroup}%
      \fi%
      \code@start%
      \noindent%
      \begin{minipage}[t]{\codeexamplewidth}\raggedright
        \hrule width0pt%
        \footnotesize\vskip-1em%
        \code@flushstart\box\codeexamplebox\code@flushend%
        \vskip0pt%
        \leavevmode%
        \box\codeexampleboxanim%
        \vskip-1ex
        \leavevmode%
      \end{minipage}%
    \else%
      \def\code@mid{\par}
      \def\code@width{\linewidth-6pt}
      \def\code@end{}
    \fi%
    \code@mid%
      \ifpgfmanual@multipage@code%
        {%
          \pgfkeysvalueof{/codeexample/prettyprint/base color}%
          \pgfmanualdolisting{#1}%
        }%
      \else%
        \colorbox{codebackground}{%
          \pgfkeysvalueof{/codeexample/prettyprint/base color}%
          \begin{minipage}[t]{\code@width}%
            \pgfmanualdolisting{#1}%
          \end{minipage}}%
      \fi%
    \code@end%
    \par%
    \medskip
  \fi
  \endcodeexample\endgroup%
}

\def\endcodeexample{\endgroup}
\newbox\codeexampleboxanim

\def\pgfmanualdolisting#1{%
      {%
        \let\do\@makeother
        \dospecials
        \frenchspacing\@vobeyspaces
        \normalfont\ttfamily\footnotesize
        \typesetcomment%
        \codeexample@tabinit@replacementchars
        \@tempswafalse
        \def\par{%
          \if@tempswa
          \leavevmode \null \@@par\penalty\interlinepenalty
          \else
          \@tempswatrue
          \ifhmode\@@par\penalty\interlinepenalty\fi
          \fi}%
        \obeylines
        \everypar \expandafter{\the\everypar \unpenalty}%
        \ifx\code@preamble\pgfutil@empty\else
          \pgfutil@tempdima=\hsize
          \vbox{\hsize=\pgfutil@tempdima
              \raggedright\scriptsize\detokenize\expandafter{\code@preamble}}%
        \fi
        \pgfkeysvalueof{/codeexample/typeset listing/.@cmd}{#1}\pgfeov
      }%
}

\makeatother

\usepackage{pgfmanual}


% autoxref is now always on

% \makeatletter
% % \pgfautoxrefs will be defined by 'make dist'
% \pgfutil@ifundefined{pgfautoxrefs}{%
%   \renewcommand\pgfmanualpdflabel[3][]{#3}% NO-OP
%   \def\pgfmanualpdfref#1#2{#2}%
%   \pgfkeys{
%     /pdflinks/codeexample links=false,% DISABLED.
%   }%
% }{}
% \makeatother

\newdimen\pgfmanualcslinkpreskip

% Styling of the pretty printer
\pgfkeys{
  /codeexample/syntax hilighting/.style={
    /codeexample/prettyprint/key name/.code={\textcolor{keycolor}{\pgfmanualpdfref{##1}{\noligs{##1}}}},
    /codeexample/prettyprint/key name with handler/.code 2 args={\textcolor{keycolor}{\pgfmanualpdfref{##1}{\noligs{##1}}}/\textcolor{blue!70!black}{\pgfmanualpdfref{/handlers/##2}{\noligs{##2}}}},
    /codeexample/prettyprint/key value display only/.code={\textcolor{keycolor}{{\itshape{\let\pgfmanualwordstartup\relax\pgfmanualprettyprintcode{##1}}}}},
    /codeexample/prettyprint/cs/.code={\textcolor{cscolor}{\pgfmanualcslinkpreskip4.25pt\pgfmanualpdfref{##1}{\noligs{##1}}}},
    /codeexample/prettyprint/cs with args/.code 2 args={\textcolor{black}{\pgfmanualcslinkpreskip4.25pt\pgfmanualpdfref{##1}{\noligs{##1}}}\{\textcolor{black}{\pgfmanualprettyprintcode{##2}}\pgfmanualclosebrace},
    /codeexample/prettyprint/cs arguments/pgfkeys/.initial=1,
    /codeexample/prettyprint/cs/pgfkeys/.code 2 args={\textcolor{black}{\pgfmanualcslinkpreskip4.25pt\pgfmanualpdfref{##1}{\noligs{##1}}}\{\textcolor{black}{\pgfmanualprettyprintpgfkeys{##2}}\pgfmanualclosebrace},
    /codeexample/prettyprint/cs arguments/begin/.initial=1,
    /codeexample/prettyprint/cs/begin/.code 2 args={\textcolor{black}{##1}\{\textcolor{cscolor}{\pgfmanualpdfref{##2}{\noligs{##2}}}\pgfmanualclosebrace},
    /codeexample/prettyprint/cs arguments/end/.initial=1,
    /codeexample/prettyprint/cs/end/.code 2 args={\textcolor{black}{##1}\{\textcolor{cscolor}{\pgfmanualpdfref{##2}{\noligs{##2}}}\pgfmanualclosebrace},
    /codeexample/prettyprint/word/.code={\pgfmanualwordstartup{\begingroup\pgfkeyssetvalue{/pdflinks/search key prefixes in}{}\pgfmanualpdfref{##1}{\noligs{##1}}\endgroup}},
    /codeexample/prettyprint/point/.code={\textcolor{pointcolor}{\noligs{##1}}},%
    /codeexample/prettyprint/point with cs/.code 2 args={\textcolor{pointcolor}{(\pgfmanualpdfref{##1}{\noligs{##1}}:\noligs{##2}}},%
    /codeexample/prettyprint/comment font=\itshape,
    /codeexample/prettyprint/base color/.initial=\color{basecolor},
    /pdflinks/render hyperlink/.code={%
      {\setbox0=\hbox{##1}%
        \rlap{{\color{linkcolor}\dimen0\wd0\advance\dimen0by-\pgfmanualcslinkpreskip\hskip\pgfmanualcslinkpreskip\vrule width\dimen0 height-1pt depth1.6pt}}%
        \box0%
      }%
    }
  },/codeexample/syntax hilighting
}

\colorlet{keycolor}{black}
\colorlet{cscolor}{black}
\colorlet{pointcolor}{black}
\colorlet{basecolor}{black}
\colorlet{linkcolor}{black!8}

\def\pgfmanualswitchoncolors{%
  \colorlet{keycolor}{green!50!black}%
  \colorlet{cscolor}{blue!70!black}
  \colorlet{pointcolor}{violet}
  \colorlet{basecolor}{black!55}
  \colorlet{linkcolor}{white}
}

\makeatletter

\def\pgfmanualwordstartup{\textcolor{black}}

\def\noligs#1{\pgfmanualnoligs#1\kern0pt--\pgf@stop}%
\def\pgfmanualnoligs#1--{%
  \pgfutil@ifnextchar\pgf@stop{#1\pgfutil@gobble}{#1-\kern0pt-\kern0pt\pgfmanualnoligs}%
}
\makeatother


%%% Local Variables: 
%%% mode: latex
%%% TeX-master: "beameruserguide"
%%% End: 
}
  {\PackageError{tikz-cd-doc}{
This document requires the file pgfmanual-en-macros.tex (distributed
with pgf) to compile.  Please place a copy of that file in the current
directory}{}}

\IfFileExists{pgfmanual.aux}
  {\externaldocument[pgfman-]{pgfmanual}}
  {\PackageWarning{tikz-cd-doc}{
Couldn't find pgfmanual.aux.  To get cross-references to the pgf
manual, compile it and copy all resulting .aux files to the current
directory}{}}

\makeindex

% Global styles:
\tikzset{
  every plot/.style={prefix=plots/pgf-},
  shape example/.style={
    color=black!30,
    draw,
    fill=yellow!30,
    line width=.5cm,
    inner xsep=2.5cm,
    inner ysep=0.5cm}
}

\hypersetup{unicode,
            pdftitle={tikzcd: Commutative diagrams with TikZ},
            pdfkeywords={Commutative diagrams; TeX; LaTeX; ConTeXt; TikZ; pgf; tikz-cd; tikzcd},
           }

\pgfkeys{
  /pdflinks/search key prefixes in=
    {/tikz/commutative diagrams/},%,/tikz/,/pgf/}  % This really needs to be commented out!
  /pdflinks/internal link prefix=tikzcd,
	%
  /pdflinks/warnings=false,
	% for debugging:
  /pdflinks/show labels=false,
}

\tikzset{/codeexample/every codeexample/.append style={/tikz/commutative diagrams/background color=graphicbackground}}

\tikzset{gray x/.tip={Rays[color=gray]}}

\newcommand{\displayarrow}[2][]{%
  \index{#2@\protect\texttt{#2} arrow tip}%
  \index{Arrow tips!#2@\protect\texttt{#2}}%
  \texttt{#2} & yields \tikz[baseline=-axis_height] \draw[{#2}-{#2}, line width=rule_thickness, #1] (0,0) -- (1,0);}

\newcommand{\displayarrowstyle}[1]{
  \def\mykey{/tikz/commutative diagrams/#1}%
  \def\mypath{}%
  \def\myname{}%
  \firsttimetrue%
  %\pgfmanualdecomposecount=0\relax%
  \decompose/tikz/commutative diagrams/#1/\nil%
  \texttt{#1} & yields
  \tikz[baseline=-axis_height, line width=rule_thickness] \draw[arrows=gray
  x-gray x,commutative diagrams/.cd, #1] (0,0) -- (0.6,0);}

\begin{document}
\begin{center}
\vspace*{1em}
\tikz\node[scale=1.2]{%
  \color{gray}\Huge\ttfamily \char`\{\textcolor{red!75!black}{tikzcd}\char`\}};

\vspace{0.5em}
{\Large\bfseries Commutative diagrams with \tikzname}

\vspace{1em}
{Version 1.0 \qquad May 4, 2021}
\end{center}

\vspace{1.5em}
The general-purpose drawing package Ti\emph{k}Z can be used to typeset
commutative diagrams and other kinds of mathematical pictures,
generating high-quality results.  The present package facilitates the
creation of such diagrams by providing a convenient set of macros and
reasonable default settings.  Familiarity with Ti\emph{k}Z is helpful
but not necessary, since the examples contained here cover the most
common situations.

This work may be distributed and/or modified under the conditions of
the \LaTeX\ Project Public License, either version 1.3 of this license
or (at your option) any later version.

\tableofcontents

\section{Getting started}
\label{sec:basic-usage}

To load this package in \LaTeX, type
\begin{verse}
    \index{tikz-cd@\protect\texttt{tikz-cd} package}%
    \index{Packages and files!tikz-cd@\protect\texttt{tikz-cd}}%
    |\usepackage{tikz-cd}|%
\end{verse}
or load \tikzname{} and then type
\begin{verse}%
   \index{cd@\protect\texttt{cd} library}%
   \index{Libraries!cd@\protect\texttt{cd}}%
   |\usetikzlibrary{cd}|%
\end{verse}
This package is compatible with other \TeX\ formats as well.
See~\S\ref{sec:usage-plain-tex} and \S\ref{sec:usage-context}.

\subsection{Creating a diagram}
\label{sec:creating-diagrams}

The basic tool to create a commutative diagram is the following
environment.

\begin{environment}{{tikzcd}\opt{\oarg{options}}}
\end{environment}

The environment contents should describe a matrix, as in \LaTeX's
familiar |{tabular}| environment.  The optional argument
\meta{options} may be used to modify the appearance of the diagram.
Any of the customization keys described in this manual, as well as
those originally provided by \tikzname{}, can be used here.  Arrows
between matrix entries can be created with the |\arrow| command
described below.

Everything inside |{tikzcd}| is typeset in math mode, but you will
probably want to use it inside an |{equation}| environment or
|\[| \dots |\]|,
so that the diagram is placed on a new line and centered.

It is important to note that \textsc{dvi} viewers will not display
diagrams correctly.  It is necessary to convert the \textsc{dvi} file
to \textsc{pdf} or \textsc{ps} format---or, even better, use a tool
that generates \textsc{pdf} files directly, such as \texttt{pdflatex}.

\subsection{Inserting arrows}
\label{sec:inserting-arrows}

Inside the |{tikzcd}| environment, the following synonymous commands
are provided to produce arrows.

\begin{pgfmanualentry}
  \extractcommand\arrow|[|\meta{options}|]|\@@
  \extractcommand\ar|[|\meta{options}|]|\@@
  \pgfmanualbody
\end{pgfmanualentry}

Here, \meta{options} is a comma-separated list of options which can be
used to specify the arrow target, add labels, change arrow tips, and
perform additional modifications.

The arrow target can be specified by a direction parameter, which
consists of a string of characters |r|, |l|, |d|, |u| (standing for
right, left, down and up).  Labels can be placed on an arrow by means
of the quotes syntax, described in detail in the \pgfname{} manual
\cite[\S\ref*{pgfman-section-label-quotes}]{pgfman}.  Notice the use
of |"\phi"| in the example below.

\begin{codeexample}[]
  \begin{tikzcd}
    A \arrow[rd] \arrow[r, "\phi"] & B \\
                                   & C
  \end{tikzcd}
\end{codeexample}

To further modify the appearance of an arrow, note that \meta{options}
may contain any key that can be passed to \tikzname's |\path| command.
Similarly, a label can receive additional options via the syntax
\begin{verse}
  |"|\meta{label text}|"|\opt{\meta{label options}}.
\end{verse}
Both \meta{label text} and \meta{label options} need to be enclosed in
curly braces if they contain commas.

{\catcode`\|=12
\begin{codeexample}[]
\begin{tikzcd}
  A \arrow[r, "\phi"] \arrow[d, red]
    & B \arrow[d, "\psi" red] \\
  C \arrow[r, red, "\eta" blue]
    &  |[blue, rotate=-15]| D
\end{tikzcd}
\end{codeexample}
}

The example above also showcases one of the possible ways to pass
special options to a \tikzname\ matrix node.  See the \pgfname{}
manual \cite[\S\ref*{pgfman-section-matrices}]{pgfman} for more
information.

Arrows can have an arbitrary number of labels, by repeated use of
arguments in quotes.  The example below shows how to control the
positioning of labels.  Notice in particular that an apostrophe as
\meta{label option} causes the label to be placed on the opposite side
of the arrow.

\begin{codeexample}[]
\begin{tikzcd}
  A \arrow[r, "\phi" near start, "\psi"', "\eta" near end] & B    
\end{tikzcd}
\end{codeexample}

We provide two real-life examples.

\begin{codeexample}[]
\begin{tikzcd}
  T
  \arrow[drr, bend left, "x"]
  \arrow[ddr, bend right, "y"]
  \arrow[dr, dotted, "{(x,y)}" description] & & \\
    & X \times_Z Y \arrow[r, "p"] \arrow[d, "q"]
      & X \arrow[d, "f"] \\
    & Y \arrow[r, "g"]
      & Z
\end{tikzcd}
\end{codeexample}

\begin{codeexample}[]
\begin{tikzcd}[column sep=tiny]
    & \pi_1(U_1) \ar[dr] \ar[drr, "j_1", bend left=20]
      &
        &[1.5em] \\
  \pi_1(U_1\cap U_2) \ar[ur, "i_1"] \ar[dr, "i_2"']
    &
      & \pi_1(U_1) \ast_{ \pi_1(U_1\cap U_2)} \pi_1(U_2) \ar[r, dashed, "\simeq"]
        & \pi_1(X) \\
    & \pi_1(U_2) \ar[ur]\ar[urr, "j_2"', bend right=20]
      &
        &
\end{tikzcd}
\end{codeexample}

\subsection{Changing arrow tips}
\label{sec:changing-arrow-tips}

A set of |\arrow| options is provided to create different kinds of
arrows.  Some of these options have a short descriptive name, such as
|hook|, and others are named after \TeX{} arrow-producing commands
(without a ``|\|''), like |dashrightarrow|.%|

\begin{codeexample}[]
  \begin{tikzcd}
    X \arrow[r, hook] \arrow[dr, dashrightarrow]
      & \bar{X} \arrow[d]\\
      & Y
  \end{tikzcd}
\end{codeexample}

The following list shows all available arrow types (each of them is a
style key located in |/tikz/commutative diagrams|).

\begin{multicols}{2}\raggedcolumns
  \subsubsection*{Basic arrows}
  \begin{tabular}{ll}
    \displayarrowstyle{to head}\\
    \displayarrowstyle{rightarrow}\\
    \displayarrowstyle{leftarrow}\\
    \displayarrowstyle{leftrightarrow}\\
    \displayarrowstyle{Rightarrow}\\
    \displayarrowstyle{Leftarrow}\\
    \displayarrowstyle{Leftrightarrow}
  \end{tabular}

  \subsubsection*{Arrows from bar}
  \begin{tabular}{ll}
    \displayarrowstyle{maps to}\\
    \displayarrowstyle{mapsto}\\
    \displayarrowstyle{mapsfrom}\\
    \displayarrowstyle{Mapsto}\\
    \displayarrowstyle{Mapsfrom}\\
  \end{tabular}

  \subsubsection*{Arrows with hook}
  \begin{tabular}{ll}
    \displayarrowstyle{hook}\\
    \displayarrowstyle{hook'}\\
    \displayarrowstyle{hookrightarrow}\\
    \displayarrowstyle{hookleftarrow}\\
  \end{tabular}

  \subsubsection*{Arrows with tail}
  \begin{tabular}{ll}
    \displayarrowstyle{tail}\\
    \displayarrowstyle{rightarrowtail}\\
    \displayarrowstyle{leftarrowtail}\\
  \end{tabular}

  \subsubsection*{Two-headed arrows}
  \begin{tabular}{ll}
    \displayarrowstyle{two heads}\\
    \displayarrowstyle{twoheadrightarrow}\\
    \displayarrowstyle{twoheadleftarrow}\\
  \end{tabular}

  \subsubsection*{Harpoons}
  \begin{tabular}{ll}
    \displayarrowstyle{harpoon}\\
    \displayarrowstyle{harpoon'}\\
    \displayarrowstyle{rightharpoonup}\\
    \displayarrowstyle{rightharpoondown}\\
    \displayarrowstyle{leftharpoonup}\\
    \displayarrowstyle{leftharpoondown}\\
  \end{tabular}

  \subsubsection*{Dashed arrows}
  \begin{tabular}{ll}
    \displayarrowstyle{dashed}\\
    \displayarrowstyle{dashrightarrow}\\
    \displayarrowstyle{dashleftarrow}\\
  \end{tabular}

  \subsubsection*{Squiggly arrows}
  \begin{tabular}{ll}
    \displayarrowstyle{squiggly}\\
    \displayarrowstyle{rightsquigarrow}\\
    \displayarrowstyle{leftsquigarrow}\\
    \displayarrowstyle{leftrightsquigarrow}
  \end{tabular}

  \subsubsection*{Non-arrows}
  \begin{tabular}{ll}
    \displayarrowstyle{no head}\\
    \displayarrowstyle{no tail}\\
    \displayarrowstyle{dash}\\
    \displayarrowstyle{equal}\\
  \end{tabular}
\end{multicols}

A gray cross (\tikz \path[/pgf/tips=true,gray x-] (0,0) -- (1mm,0);)
in the samples above indicates that the corresponding tip is kept
unchanged.  This allows several arrow styles to be superimposed.

\begin{codeexample}[]
\begin{tikzcd}
  A \arrow[r, tail, two heads, dashed] & B
\end{tikzcd}
\end{codeexample}

% It is also possible to select arrow tips using \tikzname{}'s |arrow|
% option directly (e.g., by saying |right hook->|).  However, the method
% described here above is more appropriate, as it abstracts the choice
% of an actual arrow tip design (see also the |arrow style| key below),
% and is arguably better from the mnemonic standpoint.


\subsection{Alternative syntax for arrows}
\label{sec:altern-synt-arrows}

The following forms of the arrow command were used before the
appearance of the quotes syntax for labels, and now may seem somewhat
convoluted.  They are nonetheless still available for backwards
compatibility.

\begin{command}{\arrow\opt{\oarg{options}}\marg{direction}\meta{labels}}
\end{command}

Here, \meta{direction} is a string containing the characters |r|, |l|,
|d|, |u| and is used to determine the arrow target.  Alternatively,
you can specify an explicit matrix cell by replacing \meta{direction}
with something of the form \meta{row number}\texttt{-}\meta{column
  number}, or the name of a node.  The trailing \meta{labels} can be the
empty string or of the form
\begin{verse}
  \opt{\oarg{label options}}\marg{label text}\meta{more labels}.
\end{verse}
The equivalent command |\ar| can also be used in this form.  Here is
an example.

\begin{codeexample}[]
\begin{tikzcd}
  A \arrow{d} \arrow{r}[near start]{\phi}[near end]{\psi}
    & B \arrow[red]{d}{\xi} \\
  C \arrow[red]{r}[blue]{\eta}
    & D
\end{tikzcd}
\end{codeexample}

There are further shortened forms:
\begin{pgfmanualentry}
  \extractcommand\rar\opt{\oarg{options}}\meta{labels}\@@
  \extractcommand\lar\opt{\oarg{options}}\meta{labels}\@@
  \extractcommand\dar\opt{\oarg{options}}\meta{labels}\@@
  \extractcommand\uar\opt{\oarg{options}}\meta{labels}\@@
  \extractcommand\drar\opt{\oarg{options}}\meta{labels}\@@
  \extractcommand\urar\opt{\oarg{options}}\meta{labels}\@@
  \extractcommand\dlar\opt{\oarg{options}}\meta{labels}\@@
  \extractcommand\ular\opt{\oarg{options}}\meta{labels}\@@
  \pgfmanualbody
\end{pgfmanualentry}
The first one is equivalent to
\begin{verse}
  |\arrow|{\oarg{options}}|{r}|\meta{labels}
\end{verse}
and the other ones work analogously.

\subsection{Usage in plain \TeX{}}
\label{sec:usage-plain-tex}

To use this package in plain \TeX{}, load \tikzname{} and the
\texttt{cd} library by saying
\begin{verse}
  |\input tikz.tex|\\
  \index{cd@\protect\texttt{cd} library}%
  \index{Libraries!cd@\protect\texttt{cd}}%
  |\usetikzlibrary{cd}|
\end{verse}
The |{tikzcd}| environment should then be replaced by the following:
\begin{plainenvironment}{{tikzcd}\opt{\oarg{options}}}
\end{plainenvironment}
All other functions of this library work as described in this manual
without change.

\subsection{Usage in Con\TeX t}
\label{sec:usage-context}

To use this package in Con\TeX t, load \tikzname{} and then the
\texttt{cd} library by saying
\begin{verse}
  |\usemodule[tikz]|\\
  \index{cd@\protect\texttt{cd} library}%
  \index{Libraries!cd@\protect\texttt{cd}}%
  |\usetikzlibrary[cd]|
\end{verse}
The |{tikzcd}| environment should then be replaced by the following:
\begin{contextenvironment}{{tikzcd}\opt{\oarg{options}}}
\end{contextenvironment}
Moreover, you may replace the column and row separators |&|, |\\| by
their Con\TeX t analogues |\NC|, |\NR|.  However, you should use |\NC|
only \emph{between} cells, and not before the first column or after
the last column, as in usual Con\TeX t tables.  Similarly, |\NR|
should be used only between rows.

All other functions of this library work as described in this manual
without change.

\section{Controlling the appearance of diagrams}
\label{sec:chang-appe-diagr}

This section describes a number of customization keys defined by this
package.  All keys are located in the path |/tikz/commutative diagrams|.
Options passed to |{tikzcd}| or |\arrow| are searched for
in that path, and, if not found there, in |/tikz|.  To set options
globally, it is convenient to use the following command.

\begin{command}{\tikzcdset\marg{options}}
  Executes \meta{options} in the path |/tikz/commutative diagrams|.
\end{command}

Besides the keys described in this manual, numerous \tikzname\
parameters can affect the appearance of a diagram.  However, only a
few of them (namely those appearing in |every diagram|, |every cell|,
|every arrow|, and |every label| below) are reinitialized when
|{tikzcd}| is called.  This means that modifying a certain \tikzname\
parameter globally may or may not affect the output of |{tikzcd}|.

We also point out that besides the options and styles provided by this
package, several keys defined by \tikzname{} are useful for arrows.
Some examples are \texttt{dashed}, |dotted|, and its relatives,
|line width=|\meta{dimension}, |color=|\meta{color}, |bend right|, |bend left|,
|in=|\meta{angle}, |out=|\meta{angle}, |loop|, etc.  See the
\pgfname{} manual~\cite[\S\ref*{pgfman-section-cap-joins} and
\S\ref*{pgfman-library-to-paths}]{pgfman}.  Likewise, \tikzname{}
provides several keys that are useful for labels, such as |above|,
|below|, |left|, |right|, |swap| (which makes the label be placed on
the right side of the arrow, relative to its direction), |sloped|,
|pos=|\meta{fraction}, |near start|, |near end|, |inner sep=|\meta{dimension},
|font=|\meta{font command}, |text width=|\meta{dimension}, etc.  See the \pgfname{}
manual~\cite[\S\ref*{pgfman-section-nodes}, esp.\
\S\ref*{pgfman-section-pos-option}]{pgfman}.

\subsection{General options}
\label{sec:general-options}

\begin{stylekey}{/tikz/commutative diagrams/every diagram}
  This style is applied to every |{tikzcd}| environment.  Initially, it
  contains the following:
\begin{verse}
|row sep=normal||,|\\
|column sep=normal||,|\\
|/tikz/baseline=0pt|
\end{verse}
\end{stylekey}

The |baseline=0pt| setting is used to make equation numbers be placed
correctly (as an exception, one-row diagrams are anchored at their
matrix base, which is exactly what you want).

\begin{key}{/tikz/commutative diagrams/diagrams=\meta{options}}
  This key appends \meta{options} to the style |every diagram|.
\end{key}

\begin{stylekey}{/tikz/commutative diagrams/every matrix}
  This style is applied to the \tikzname{} matrix created internally
  by |{tikzcd}|.  Initially, it contains the following:
\begin{verse}
|/tikz/inner sep=0pt|
\end{verse}
\end{stylekey}

\begin{stylekey}{/tikz/commutative diagrams/every cell}
  This style is applied to every \tikzname{} matrix cell created by
  |{tikzcd}|.  Initially, it contains the following:
\begin{verse}
|/tikz/shape=asymmetrical rectangle||,|\\
|/tikz/inner xsep=1ex||,|\\
|/tikz/inner ysep=0.85ex|
\end{verse}
\end{stylekey}

The |asymmetrical rectangle| shape is described in
\S\ref{sec:asymm-rect-shape}.  The |inner xsep|, |inner ysep| options
determine the spacing between a diagram entry and any arrows reaching
it.

\begin{key}{/tikz/commutative diagrams/cells=\meta{options}}
  This key appends \meta{options} to the style |every cell|.
\end{key}

\def\printsepaux+#1em{#1\,em}
\def\printsep#1#2{\edef\temp{\pgfkeysvalueof{/tikz/commutative diagrams/#1 sep/#2}}\expandafter\printsepaux\temp}

\begin{key}{/tikz/commutative diagrams/row sep=\meta{size}}
  This key acts as a frontend to \tikzname's |/tikz/row sep| key.  If
  the key
  \begin{verse}
    |/tikz/commutative diagrams/row sep/|\meta{size}
  \end{verse}
  stores a \meta{value}, then it is read and
  |/tikz/row sep|=\meta{value} is set.  If the key above is not
  initialized, then \meta{size} is presumed to be a dimension, and
  |/tikz/row sep|=\meta{size} is set.

  The initially available sizes, and their values, are
  the following:
  \begin{center}
    \begin{tabular}{cccccc}
      |tiny| & |small| & |scriptsize| & |normal| & |large| & |huge| \\
      \printsep{row}{tiny} & \printsep{row}{small} & \printsep{row}{scriptsize} & \printsep{row}{normal} & \printsep{row}{large} & \printsep{row}{huge}
    \end{tabular}
  \end{center}
\end{key}

Notice that setting, say, |row sep=1cm| globally with |\tikzcdset| will
have no effect, since the |row sep| option is re-set at the beginning
of each diagram.  To make all diagrams have |row sep| equal to 1\,cm,
you can modify the meaning of |normal| by saying
\begin{verse}
|\tikzcdset{row sep/normal=1cm}|.
\end{verse}
You can also create new sizes, but note that \pgfname\ requires new keys to be
initialized explicitly.  For example, to create a size |my size|,
meaning 1\,ex, you should use
\begin{verse}
|\tikzcdset{row sep/my size/.initial=1ex}|.
\end{verse}

\begin{key}{/tikz/commutative diagrams/column sep=\meta{size}}
  This works analogously to the |row sep| key above.  The sizes available
  initially are the following:
  \begin{center}
    \begin{tabular}{cccccc}
      |tiny| & |small| & |scriptsize| & |normal| & |large| & |huge| \\
      \printsep{column}{tiny} & \printsep{column}{small} & \printsep{column}{scriptsize} & \printsep{column}{normal} & \printsep{column}{large} & \printsep{column}{huge}
    \end{tabular}
  \end{center}
\end{key}

\begin{key}{/tikz/commutative diagrams/sep=\meta{size}}
  This key sets |row sep=|\meta{size}, |column sep=|\meta{size}.
\end{key}

In the examples below, the triangular diagrams would look too wide or
too tall if the column or row separation were not set appropriately.

\begin{codeexample}[]
\begin{tikzcd}[column sep=small]
               & A \arrow[dl] \arrow[dr] & \\
  B \arrow{rr} &                         & C 
\end{tikzcd}
\end{codeexample}

\begin{codeexample}[]
\begin{tikzcd}[row sep=tiny]
                          & B \arrow[dd] \\
  A \arrow[ur] \arrow[dr] &              \\
                          & C
\end{tikzcd}
\end{codeexample}

Section \ref*{pgfman-section-matrices}.3.2 of the \pgfname{} manual
\cite{pgfman} contains further details on the spacing of matrix cells.

\begin{stylekey}{/tikz/commutative diagrams/cramped}
  By default, a generous amount of white space is added around diagram
  cells, which is appropriate for large, displayed diagrams.  The
  present style removes some of this extra white space, and is
  intended for smaller diagrams that should blend with the surrounding
  text, or very wide material that wouldn't fit the page otherwise.
\end{stylekey}

The picture below shows the (somewhat subtle) difference
between the cramped and the non-cramped styles.

\begin{codeexample}[pre=\minipage{6cm},post=\endminipage]
This \begin{tikzcd} A \arrow[r] & B \end{tikzcd} is a regular diagram.\\
This \begin{tikzcd}[cramped, sep=small] A \arrow[r] & B \end{tikzcd} is a cramped diagram.\\
This $A \to B$ is just a formula.
\end{codeexample}

Keep in mind that while there are some legitimate uses for |{tikzcd}|
diagrams on inline formulas, standard \LaTeX\ constructs such as
|\overset| and |\xrigthtarrow| are often sufficient and should be
preferred.

\begin{key}{/tikz/commutative diagrams/math mode=\meta{boolean}
     (default true)}
   This key determines whether or not the contents of a diagram are
   typeset in math mode.  If set globally or diagram-wise, it affects
   both the diagram entries and arrow labels.  If used with |\arrow|,
   it affects only its labels.
\end{key}

\begin{key}{/tikz/commutative diagrams/background color=\meta{color} (initially white)}
  This key stores the name of a color, and is read by 
  styles that fill the background, such as |description| and |crossing over|.
  It does not cause the background of diagrams to be filled.
\end{key}

\subsection{Global options for arrows}
\label{sec:options-arrows}

\begin{stylekey}{/tikz/commutative diagrams/every arrow}
  This style is applied to every |\arrow|.  Initially, it contains the
  following:
\begin{verse}
|/tikz/draw,|\\
|/tikz/line width=rule_thickness||,|\\
|rightarrow|
\end{verse}
\end{stylekey}

\begin{key}{/tikz/commutative diagrams/arrows=\meta{options}}
  This key appends \meta{options} to the style |every arrow|.
\end{key}

\begin{key}{/tikz/commutative diagrams/arrow style=\meta{style}}
  This key determines which collection of arrow tips is used by the
  arrow tip selection styles listed in
  \S\ref{sec:changing-arrow-tips}.  The initial setting is suitable
  for documents using the Computer Modern font at any size.  The
  available choices for \meta{style} are:
  \begin{description}
  \item[\texttt{Latin Modern}] A small variant of the initial
    settings, intended for documents using the Latin Modern font at
    any size.
  \item[\texttt{math font}] This setting uses the |Glyph| meta arrow
    tip described in \S\ref{sec:font-arrow-tips}.
  \item[\texttt{tikz}] This setting uses the arrow tips defined in
    \tikzname's |arrows.meta| library.  It honors the option
    |/tikz/>|.
  \end{description}

  This key is usually invoked in the document preamble, and should be
  set only once.
\end{key}

If you are using a font different from Computer Modern or Latin
Modern, you may find the best results by selecting the |math font|
style.  As detailed in \S\ref{sec:font-arrow-tips}, this is not
guaranteed to work perfectly with all fonts, but gives good results in
many cases.  If the \texttt{math font} style gives unsatisfactory
results, you can try selecting the \texttt{tikz} style, and setting
|/tikz/>| to the value that best matches your font (among those shown
in \cite[\S\ref*{pgfman-section-arrows-meta}]{pgfman}).

\begin{codeexample}[]
% in preamble
\tikzcdset{
  arrow style=tikz,
  diagrams={>={Straight Barb[scale=0.8]}}
}

% in document body
\begin{tikzcd}
  A \arrow[r, tail] \arrow[rd] & B \arrow[d, two heads]\\
                               & D
\end{tikzcd}
\end{codeexample}

\subsection{Absolute placement of arrows}
\label{sec:absol-positioning}

The usual behavior of |\arrow| is to produce an arrow starting at the
diagram entry where the command appears, and ending at an entry whose
location is specified relative to that.  The following keys override
this behavior, allowing source and target to be selected explicitly.

\begin{key}{/tikz/commutative diagrams/from=\meta{argument}}
  If \meta{argument} is of the form \meta{row
    number}\texttt{-}\meta{column number}, or if it is a string of
  characters |r|, |l|, |d|, |u|, this key sets the arrow source to be
  the corresponding cell in the diagram matrix.  Otherwise, it assumes
  the argument is the name of a node and sets the arrow source to
  \meta{argument}.
\end{key}

\begin{key}{/tikz/commutative diagrams/to=\meta{argument}}
  Similar to |from|, but refers to the arrow target.
\end{key}

Recall that it is possible to give a specific entry of a \tikzname{}
matrix a name by using the \verb!|[!\meta{options}\verb!]|! syntax, as
done for entry $C$ in the example below.  You must be careful not to
create nodes whose name contains only the characters |l|, |r|, |u|,
|d| if you want to refer to them using |from| or |to|.  The following
illustrates several different uses of these keys.

{\catcode`\|=12
\begin{codeexample}[]
\begin{tikzcd}
  A \arrow[to=Z, red] \arrow[to=2-2, blue]
    & B \\
  |[alias=Z]| C
    & D
  \arrow[from=ul, to=1-2, purple]
\end{tikzcd}
\end{codeexample}
}

In the next examples, empty labels are used to create nodes for later
reference.  The |draw=red| option is used to show where these empty
nodes are located, but of course you want to remove that when using
this technique.

\begin{codeexample}[]
\begin{tikzcd}[column sep=scriptsize]
  A \arrow[dr] \arrow[rr, ""{name=U, below, draw=red}]{}
    & & B \arrow[dl] \\
    & C \arrow[Rightarrow, from=U, "\psi"]
\end{tikzcd}
\end{codeexample}

\begin{codeexample}[]
\begin{tikzcd}
  A \arrow[r, bend left=50, ""{name=U, below, draw=red}]
    \arrow[r, bend right=50, ""{name=D, draw=red}]
    & B
  \arrow[Rightarrow, from=U, to=D]
\end{tikzcd}
\end{codeexample}

\subsection{Phantom arrows}
\label{sec:phantom-arrows}

Sometimes it is necessary to insert a symbol outside the grid
subjacent to the diagram.  The easiest way to achieve this is as a
label to an invisible arrow.

\begin{stylekey}{/tikz/commutative diagrams/phantom}
  Creates an invisible arrow.  Labels to this arrow are not invisible.
  They will be anchored at their center and typeset in full size
  (i.e., with |\textstyle|).  To get smaller labels, as in ordinary
  arrows, use the |\scriptstyle| command.
\end{stylekey}

In the picture below, the arrow containing the |phantom| option goes from
$A$ to $D$, and the |\ulcorner| symbol ($\ulcorner$) is inserted
closer to the starting point $A$.
\begin{codeexample}[]
\begin{tikzcd}
  A \arrow[r] \arrow[d] \arrow[dr, phantom, "\ulcorner", very near start]
    & B \arrow[d] \\
  C \arrow[r]
    & D
\end{tikzcd}
\end{codeexample}

\subsection{Fine-tuning the placement of arrows}
\label{sec:fine-tuning-arrows}

\begin{key}{/tikz/commutative diagrams/shift left=\meta{dimension}
    (default 0.56ex)}
  Shifts arrows by \meta{dimension} to the left, relative to the arrow
  direction.  A dimensionless argument causes that multiple of the
  default value to be used.
\end{key}

\begin{key}{/tikz/commutative diagrams/shift right=\meta{dimension}
    (default 1)}
  A shortcut to |shift left=-|\meta{dimension}.
\end{key}

\begin{codeexample}[]
\begin{tikzcd}
  A \arrow[r, red, shift left=1.5ex] \arrow[r]
    \arrow[dr, blue, shift right=1.5ex] \arrow[dr]
    & B \arrow[d, purple, shift left=1.5ex] \arrow[d]\\
    & C
\end{tikzcd}
\end{codeexample}

The default values of |shift left| and |shift right| are appropriate
for a pair of parallel arrows, and dimensionless arguments are useful
to create sets of multiple parallel arrows.
\begin{codeexample}[]
\begin{tikzcd}
  A \arrow[r]
    & B \arrow[r, shift left]
        \arrow[r, shift right]
      & C \arrow[r]
          \arrow[r, shift left=2]
          \arrow[r, shift right=2]
        & \cdots
\end{tikzcd}
\end{codeexample}

\begin{key}{/tikz/commutative diagrams/shift=\marg{coordinate}}
  Shifts arrows by \meta{coordinate}.
\end{key}

\begin{key}{/tikz/commutative diagrams/xshift=\meta{dimension}}
  Shifts arrows right by \meta{dimension}.
\end{key}

\begin{key}{/tikz/commutative diagrams/yshift=\meta{dimension}}
  Shifts arrows up by \meta{dimension}.
\end{key}

\begin{codeexample}[]
\begin{tikzcd}
  A \arrow[r, yshift=0.7ex] \arrow[r, yshift=-0.7ex]
    & B \arrow[d, xshift=0.7ex] \arrow[d, xshift=-0.7ex] \\
    & C
\end{tikzcd}
\end{codeexample}

\begin{key}{/tikz/commutative diagrams/start anchor=\opt{{\ttfamily\char`\{}\ooarg{coordinate transformations}}\meta{anchor}\opt{\ttfamily\char`\}}}
  This key specifies at which anchor of the source node the arrow
  should start.  Optionally, additional coordinate transformations can
  be supplied.  An empty \meta{anchor} argument causes no anchor to be
  specified, which is the usual behavior.
\end{key}

\begin{key}{/tikz/commutative diagrams/end anchor=\opt{{\ttfamily\char`\{}\ooarg{coordinate transformations}}\meta{anchor}\opt{\ttfamily\char`\}}}
  This key works analogously, but refers to the target node of the arrow.
\end{key}

See the picture on \S\ref{sec:asymm-rect-shape} for some of the
possible values for \meta{anchor}.
\begin{codeexample}[]
\begin{tikzcd}[cells={nodes={draw=gray}}]
  A \arrow[r, black]
    \arrow[r, blue, end anchor=north east]
    \arrow[r,
           red,
           start anchor={[xshift=-1ex]},
           end anchor={[yshift=2ex]north east}]
    & B
\end{tikzcd}
\end{codeexample}

\begin{key}{/tikz/commutative diagrams/shorten=\meta{dimension}}
  This key shortens each end of the arrow by \meta{dimension}.

\begin{codeexample}[]
\begin{tikzcd}
  A \arrow[r, shift left]
    \ar[r, shorten=2mm, shift right]
    & B
\end{tikzcd}
\end{codeexample}
\end{key}

Note that the ends of an arrow can be shortened individually using
\tikzname's built-in keys |shorten <| and |shorten >|.

\subsection{Three-dimensional diagrams}
\label{sec:crossing-over}

\begin{stylekey}{/tikz/commutative diagrams/crossing over}
  This style makes a thicker line, with color |background color|, to
  be drawn under the current arrow, simulating the effect of its
  passing over other arrows.

\begin{codeexample}[]
\begin{tikzcd}
  A \arrow[dr] & B \arrow[dl, crossing over] \\
  C            & D
\end{tikzcd}
\end{codeexample}
\end{stylekey}

Note that, since arrows are drawn in the order they are read, it may
be necessary to defer the drawing of certain arrows to achieve the
desired result.  This can be done using the |from| key, as shown in the
following picture.

\begin{codeexample}[]
\begin{tikzcd}[row sep=scriptsize, column sep=scriptsize]
  & f^* E_V \arrow[dl] \arrow[rr] \arrow[dd] & & E_V \arrow[dl] \arrow[dd] \\
  f^* E \arrow[rr, crossing over] \arrow[dd] & & E \\
  & U \arrow[dl] \arrow[rr] &  & V \arrow[dl] \\
  M \arrow[rr] & & N \arrow[from=uu, crossing over]\\
\end{tikzcd}
\end{codeexample}
  
\begin{key}{/tikz/commutative diagrams/crossing over clearance=\meta{dimension}
    (initially 1.5ex)}
  This key specifies the width of the background-colored line drawn
  under a |crossing over| arrow.
\end{key}

\subsection{Options for labels}
\label{sec:options-labels}

\begin{stylekey}{/tikz/commutative diagrams/every label}
  This style is applied to every label produced with |\arrow|.  It is initially set
  to

\begin{verse}
|/tikz/auto,|\\
|/tikz/font=|\meta{something}|,|\\
|/tikz/inner sep=0.5ex|
\end{verse}
where \meta{something} is something that makes |\scriptstyle| be
applied to labels in math mode.
\end{stylekey}

The key |/tikz/auto| makes the label be placed on the left side of the
arrow, relative to its direction.  The key |/tikz/inner sep| controls
the distance between a label and the corresponding arrow.

\begin{key}{/tikz/commutative diagrams/labels=\meta{options}}
  This key appends \meta{options} to |every label|.
\end{key}

\begin{stylekey}{/tikz/commutative diagrams/marking}
  This style causes the label to be placed over the arrow.  It is
  useful to decorate arrows using ordinary math symbols.

\begin{codeexample}[]
\begin{tikzcd}
  A \arrow[r, "/" marking]
    \arrow[rd, "\circ" marking]
    & B \\
    & C
\end{tikzcd}
\end{codeexample}
\end{stylekey}

\begin{stylekey}{/tikz/commutative diagrams/description}
  This style causes the label to be placed over the arrow, with the
  background filled.  The clearance around the label is determined by
  \texttt{/tikz/inner sep}.

\begin{codeexample}[]
\begin{tikzcd}
  A \arrow[r, "\phi" description] & B
\end{tikzcd}
\end{codeexample}
\end{stylekey}

\section{Advanced usage}
\label{sec:advanced-usage}

This section provides further details on the functioning of this
package, with the aim of allowing the advanced user to make a more or
less arbitrary use of other \tikzname{} features within |{tikzcd}|.

\subsection{Internals of \texttt{tikzcd} and the arrow commands}
\label{sec:intern-arrow-comm}

The |{tikzcd}| environment works by substituting code of the form
\begin{verse}
  |\begin{tikzcd}[|\meta{options}|]|\\
  \hspace*{1.5ex}  \meta{contents}\\
  |\end{tikzcd}|
\end{verse}
with roughly the following:
\begin{verse}
  |\begin{tikzpicture}[|\meta{options}|]|\\
  \hspace*{1.5ex}|  \matrix[matrix of nodes] {|\\
  \hspace*{3ex}|    |\meta{contents} |\\|\\
  \hspace*{1.5ex}|  };|\\
  \hspace*{1.5ex}|  |\meta{paths}\\
  |\end{tikzpicture}|
\end{verse}

Not shown above are a number of initialization procedures, such as
defining |\arrow| and its relatives, as well as applying the default
settings specified by |every diagram| and its relatives.  Note that
the next-row command |\\| for the last row is inserted by |{tikzcd}|,
and therefore should not be present in \meta{contents}.  Notice also
that you can use the key |execute at end picture| in \meta{options}
to have arbitrary \tikzname{} code executed after a diagram is drawn.

Initially, \meta{paths} is the empty string.  A command
|\arrow[|\meta{options}|]| does nothing at the point it is inserted,
and causes the following code to be appended to \meta{paths}:
\begin{verse}
  |\path[|\meta{options}|] (|\meta{source~node}|) to (|\meta{target~node}|);|
\end{verse}
By default, \meta{source node} and \meta{target node} refer to the
node corresponding to the matrix cell where the command |\arrow| is
present.  This can be changed using the |from| and |to| keys, or a
direction argument (a string consisting of characters |r|, |l|, |d|,
|u|).

\subsection{Tweaking \texttt{to} paths}
\label{sec:tweaking-to-paths}

Recall that the \texttt{to} path operation used in the paths created by
|\arrow| can take a number of options, as described in 
\S\ref*{pgfman-library-to-paths} of the \pgfname{} manual~\cite{pgfman}.  In particular, the
key |/tikz/to path| determines the path that is actually drawn, and
can be used to do all sorts of fiddling.

\begin{codeexample}[]
\begin{tikzcd}
  A \arrow[dr, controls={+(1.5,0.5) and +(-1,0.8)}]
    \arrow[dr, dashed, to path=|- (\tikztotarget)]
    & \\
    & B \arrow[loop right]
\end{tikzcd}
\end{codeexample}%|

The following example shows how to produce a ``snake'' map.  The arrow
with the |phantom| option (going from $B$ to $E$) has the sole purpose
of creating a coordinate, named |Z|, lying halfway between these two
cells.  The arrow starting at $C$ has target $D$, so the macros
|\tikztostart| and |\tikztotarget| will expand to the nodes
corresponding to these two cells in the argument of |to path|.  Notice
also the use of |\tikztonodes| at the point where we want the label to
be inserted.
\begin{codeexample}[]
\begin{tikzcd}
  A \arrow[r]
    & B \arrow[r]
        \arrow[d, phantom, ""{coordinate, name=Z}]
      & C \arrow[dll,
                 "\delta",
                 rounded corners,
                 to path={ -- ([xshift=2ex]\tikztostart.east)
                           |- (Z) [near end]\tikztonodes
                           -| ([xshift=-2ex]\tikztotarget.west)
                           -- (\tikztotarget)}] \\
   D \arrow[r]
    & E \arrow[r]
      & F
\end{tikzcd}
\end{codeexample}

\subsection{Drawing diagrams directly with Ti\emph{k}Z}
\label{sec:draw-diagr-directly}

If you find that this package is not flexible enough for some
particular application, you can use the methods described in
\cite{milne} and draw diagrams directly with \tikzname.  In this case,
you can still use the styles provided here to obtain pictures with a
uniform appearance throughout your document.  The pictures below show
how this can be done (the second one is adapted from \cite{milne}).

\begin{codeexample}[]
\begin{tikzpicture}[commutative diagrams/every diagram]
  \matrix[matrix of math nodes, name=m, commutative diagrams/every cell] {
    X & \bar X \\
      & Y      \\};
  \path[commutative diagrams/.cd, every arrow, every label]
    (m-1-1) edge[commutative diagrams/hook] (m-1-2)
            edge[commutative diagrams/dashed] (m-2-2)
    (m-1-2) edge (m-2-2);
\end{tikzpicture}
\end{codeexample}


\begin{codeexample}[]
\begin{tikzpicture}[commutative diagrams/every diagram]
  \node (P0) at (90:2.3cm) {$X\otimes (Y\otimes (Z\otimes T))$};
  \node (P1) at (90+72:2cm) {$X\otimes ((Y\otimes Z)\otimes T))$} ;
  \node (P2) at (90+2*72:2cm) {\makebox[5ex][r]{$(X\otimes (Y\otimes Z))\otimes T$}};
  \node (P3) at (90+3*72:2cm) {\makebox[5ex][l]{$((X\otimes Y)\otimes Z)\otimes T$}};
  \node (P4) at (90+4*72:2cm) {$(X\otimes Y)\otimes (Z\otimes T)$};

  \path[commutative diagrams/.cd, every arrow, every label]
    (P0) edge node[swap] {$1\otimes\phi$} (P1)
    (P1) edge node[swap] {$\phi$} (P2)
    (P2) edge node {$\phi\otimes 1$} (P3)
    (P4) edge node {$\phi$} (P3)
    (P0) edge node {$\phi$} (P4);
\end{tikzpicture}
\end{codeexample}

\subsection{Issues with active characters}
\label{sec:issues-with-active-ampersand}

By default, \tikzname{} makes the character |&| active inside
matrices, and this causes the error message
\begin{verse}
  |! Package pgfbasematrix Error: Single ampersand used with wrong catcode.|
\end{verse}
when |{tikzcd}| is used inside the argument to a macro such as a Beamer
frame or a footnote.  One solution to this problem is to call |{tikzcd}|
with the option |ampersand replacement=\&|, and replace all
occurrences of |&| with |\&| in the diagram.  This procedure is also
needed if you want to use matrices in a diagram cell or label.

\begin{codeexample}[/tikz/commutative diagrams/diagrams={column sep=large}]
\begin{tikzcd}[ampersand replacement=\&]
  A \oplus B \ar[r, "{\begin{pmatrix}  e & f \\ g & h \end{pmatrix}}"]
    \& C \oplus D
\end{tikzcd}
\end{codeexample}

An alternative fix to this issue that does not require replacing |&|
with a different column separator consists in adding the following
line to your document after all packages have been loaded:
\begin{verse}
  |\def\temp{&} \catcode`&=||\active \let&=||\temp|
\end{verse}
However, this may interfere in unexpected ways with other
packages.  Use this trick at your own risk.

A different but related issue is that some packages, notably
\texttt{babel}, modify the catcodes of certain characters in a way
that may upset \tikzname's parser.  To fix this, add
\begin{verse}
  |\usetikzlibrary{babel}|
\end{verse}
to your document preamble.

\section{Additional goodies}
\label{sec:general-infra}

This package provides some general \pgfname\ infrastructure to achieve
its goals.  These additional goodies are documented in this section.

\subsection{The \texttt{asymmetrical rectangle} shape}
\label{sec:asymm-rect-shape}

The following shape is used inside |{tikzcd}| to ensure that arrows
between nodes in the same row are perfectly horizontal, even if the
nodes contain text with different heights and depths.

\begin{shape}{asymmetrical rectangle}
  This shape is similar to the |rectangle| shape, but its |center| is
  located at a fixed distance of the |base|, as determined by the
  |center yshift| key, rather than lying at the shape's geometric
  center.  The numerical anchors, as well as |east| and |west|, are
  modified accordingly, and there are anchors called |real center|,
  |real east|, and |real west| matching |rectangle|'s original
  definitions.  All other anchors provided by |rectangle| are
  available and remain unmodified.
\end{shape}

\begin{key}{/tikz/commutative diagrams/center yshift=\meta{dimension}
    (initially axis\_height)}
  Determines the distance between |asymmetrical rectangle|'s |base|
  and |center| anchors.
\end{key}

The picture below shows some of the available anchors.

\begin{center}\Huge
  \begin{tikzpicture}
    \node[name=s,shape=asymmetrical rectangle,shape example]
    {Asymmetrical rectangle\vrule width 1pt height 2cm}; \foreach
    \anchor/\placement in {north west/above left, north/above, north
      east/above right, west/left, center/right, east/right, real
      west/left, real center/right, real east/right, base west/left,
      base/right, base east/right, south west/below left, south/below,
      south east/below right, text/left, 10/right, 130/above,
      230/below, -10/below} \draw[shift=(s.\anchor)] plot[mark=x]
    coordinates{(0,0)} node[\placement] {\scriptsize\texttt{\anchor}};
  \end{tikzpicture}
\end{center}

% This shape is used for |{tikzcd}|'s matrix cells to ensure arrows
% between nodes in the same row are drawn horizontally.  When
% \tikzname{} draws a path between two nodes, it chooses endpoints lying
% on their borders in such a way that the path points towards the
% \texttt{center} of the nodes.  With the \texttt{rectangle} shape, the
% anchor \texttt{center} lies halfway between two nonadjacent vertices
% and therefore its position depends on the height and depth of the
% text, potentially causing slanted lines to be draw.  The picture
% above, for instance, has a humongous ascender (namely, the rule after
% the word ``rectangle''), so its \texttt{real center} lies well above
% its \texttt{center}.

\subsection{Reading font parameters}
\label{sec:read-font-param}

The following are |pgfmath| functions used to access relevant math
font parameters.  They take no arguments, but the result depends of
the currently selected font size.

\begin{math-function}{axis\_height}
Returns the axis height parameter (a.k.a.\ $\sigma_{22}$) of the document's math font.
\end{math-function}

\begin{math-function}{rule\_thickness}
Returns the fraction rule thickness (a.k.a.\ $\xi_8$) of the document's math font.
\end{math-function}

\subsection{Computer Modern arrow tips}
\label{sec:comp-modern-arrow}

The following arrow tips mimic the Computer Modern designs.  It is
useful to know that at size 10\,pt, the Computer Modern arrow stems
are 0.4\,pt thick; for other font sizes, scale this parameter
accordingly, or set \texttt{line width=rule\_thickness}.

Notice that by using the mechanism explained in
\S\ref{sec:changing-arrow-tips}, it is not necessary, and in fact not
advisable, to directly refer to the arrow tips listed in this section
inside a |{tikzcd}|.

\begin{multicols}{2}\raggedcolumns
  \begin{tabular}{ll}
    \displayarrow{cm to}\\
    \displayarrow[/tikz/commutative diagrams/double line]{cm implies}\\
    \displayarrow[line width=1.5*rule_thickness]{cm bold to}\\
    \displayarrow{cm double to}\\
    \displayarrow{cm to reversed}\\
  \end{tabular}

  \begin{tabular}{ll}
    \displayarrow{cm bar}\\
    \displayarrow{cm left to}\\
    \displayarrow{cm right to}\\
    \displayarrow{cm left hook}\\
    \displayarrow{cm right hook}\\
  \end{tabular}
\end{multicols}

% Incidentally, \tikzname's original \texttt{to} arrow tip seems to be
% based on the pre-1992 version of Computer Modern, which, in spite of
% its author's wish \cite{knuth}, can still be found in many systems.
% \TeX{}Live, for instance, distributed the old version up until 2007 or
% 2008.  Therefore, an up-to-date \TeX{} distribution may be necessary
% to get matching arrows in formulas and diagrams.

\subsection{Glyph arrow tips}
\label{sec:font-arrow-tips}

As an attempt to provide a general solution to the problem of having
matching arrow tips in text and pictures, this feature produces arrow
tips that consist of (pieces of) font glyphs carefully placed at the
endpoints of the path.  To activate it in |{tikzcd}| diagrams, refer
to the |arrow style| key.

\begin{arrowtipsimple}{Glyph}
  An arrow tip made from a piece of text.  It accepts the following
parameters.

\begin{key}{/pgf/arrow keys/glyph math command=\meta{name}}
  The name of a command (to be used inside |$\csname| \dots
  |\endcsname$|) producing the desired glyph.
\end{key}

\begin{key}{/pgf/arrow keys/glyph length=\meta{dimension} (initially 1ex)}
  The length of the portion of the glyph not clipped away.  Also used
  to set the `tip end' parameter.
\end{key}

\begin{key}{/pgf/arrow keys/glyph axis=\meta{dimension} (initially axis\_height)}
  A vertical displacement applied to the glyph in order to make the
  glyph's central axis (typically an arrow stem) aligned with the
  path.
\end{key}

\begin{key}{/pgf/arrow keys/glyph shorten=\meta{dimension} (initially -0.1ex)}
  An additional amount by which the end of the path is shortened.
  This is used to compensate for the gap that usually exists between
  the tip end and the glyph's bounding box.
\end{key}
\end{arrowtipsimple}

Below are some usage examples.  Notice that glyph arrow tips do not
scale with \pgfname{} line width but their size depends on the current
font size, so you will probably want to set \texttt{line
  width=rule\_thickness} when using them.  Also, contrarily to the
arrow parameters defined by the \texttt{arrows.meta} library, the
parameters described above are evaluated only at the time the arrow
tip is drawn, so they can (and should) be given in the units em or ex.

\begin{codeexample}[]
\tikzset{
  math to/.tip={Glyph[glyph math command=rightarrow]},
  loop/.tip={Glyph[glyph math command=looparrowleft, swap]},
  weird/.tip={Glyph[glyph math command=Rrightarrow, glyph length=1.5ex]},
  pi/.tip={Glyph[glyph math command=pi, glyph length=1.5ex, glyph axis=0pt]},
}

\begin{tikzpicture}[line width=rule_thickness]
  \draw[loop-math to, bend left] (0,2) to (1,2);
  \draw[math to-weird] (0,1) to (1,1);
  \draw[pi-pi] (0,0) to (1,0);
\end{tikzpicture}
\end{codeexample}

It is important to be aware of some drawbacks of this feature.  First,
the transition between a line and the arrow tip may become visible
with some printers (especially in low resolutions or draft mode) and
document viewers, as you may be able to see in the samples above.
Second, these rather long tips may
(\tikz[baseline=-axis_height]\draw[dash pattern=on 0.8ex off
0.4ex,-{Glyph[glyph math command=rightarrow]}] (0,0) -- (3.4ex,0);) or may
not (\tikz[baseline=-axis_height]\draw[dash pattern=on 0.8ex off
0.4ex,-{Glyph[glyph math command=rightarrow]}] (0,0) -- (4ex,0);) fit
nicely with dashed or curved lines.  Finally, the method used to place
the arrow tip at the end of a stroked path and clip away the arrow
stem makes certain assumptions about the font design and could fail in
cases where unusual design choices are made.

\begin{thebibliography}{9}
\bibitem{milne}
  James Milne,
  \emph{Guide to commutative diagrams}.
  Available at \url{http://www.jmilne.org/not/CDGuide.html}.
\bibitem{pgfman}
  Till Tantau,
  \emph{The \tikzname{} and \pgfname{} packages:  Manual for version 3.1.9}.
  Available at \url{http://mirrors.ctan.org/graphics/pgf/base/doc/pgfmanual.pdf}.
\end{thebibliography}

\printindex

\end{document}

%%% Local Variables:
%%% mode: tex
%%% mode: auto-fill
%%% TeX-engine: luatex
%%% End:
