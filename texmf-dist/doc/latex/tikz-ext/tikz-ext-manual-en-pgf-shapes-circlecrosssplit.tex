% !TeX spellcheck = en_US
% !TeX root = tikz-ext-manual.tex
% Copyright 2022 by Qrrbrbirlbel
%
% This file may be distributed and/or modified
%
% 1. under the LaTeX Project Public License and/or
% 2. under the GNU Free Documentation License.
%
\section{Shape: Circle Cross Split}
\begin{pgflibrary}{ext.shapes.circlecrosssplit}
  A circular shape with four parts that can be individually filled.
  \inspiration{ShapeSplitCircle-Q}{ShapeSplitCircle-A}
\end{pgflibrary}
\begin{shape}{circle cross split}
This shape has four node parts that are placed near the center of a circle.

\begin{key}{/pgf/circle cross split part fill=\marg{list} (initially none)}
Sets the custom fill color for each node part shape.
The items in \meta{list} should be separated by commas
(so if there is more than one item in \meta{list}, it must be surrounded by braces).
If \meta{list} has less entries than node parts,
then the remaining node parts use the color from the last entry in the list.
This key will automatically set |/pgf/circle cross split uses custom fill|.
\end{key}
\begin{key}{/pgf/circle cross split uses custom fill=\opt{\meta{boolean}} (default true)}
This enables the use of a custom fill for each of the node parts
(including the area covered by the |inner sep|).
The background path for the shape should not be filled (e.\,g., in \tikzname,
the |fill| option for the node must be implicitly or explicitly set to |none|).
Internally, this key sets the \TeX-if |\ifpgfcirclecrosssplitcustomfill| appropriately. 
\end{key}
\begin{codeexample}[preamble=\usepgflibrary{ext.shapes.circlecrosssplit}]
\begin{tikzpicture}\Huge
\node[name=s, shape=circle cross split, shape example, inner xsep=1.5cm, fill=none,
  circle cross split part fill={green,blue,red,yellow!90!black}]
 {\nodepart{text}text\nodepart{two}two
         \nodepart{three}three\nodepart{four}four};
\foreach \anchor/\placement in
    {north west/above left, north/above,      north east/above right,
           west/left,      center/left,             east/right,
       mid west/right,        mid/left,         mid east/left,
      base west/left,        base/left,        base east/right,
lower base west/left,  lower base/below, lower base east/right,
 lower mid west/left,  lower mid/above,   lower mid east/right,
     south west/below left, south/below,      south east/below right,
   text/below, 10/right, 130/above, two/left, three/left, four/left}
   \draw[shift=(s.\anchor)] plot[mark=x] coordinates{(0,0)}
     node[\placement] {\scriptsize\texttt{(s.\anchor)}};
\end{tikzpicture}
\end{codeexample}
\end{shape}
\endinput