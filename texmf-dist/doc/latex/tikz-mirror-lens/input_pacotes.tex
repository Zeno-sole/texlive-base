% !TeX spellcheck = en_US
% !TeX encoding = UTF-8
% =============================

\usepackage[utf8]{inputenc}
\usepackage[T1]{fontenc}
\usepackage[margin=2.5cm]{geometry}
\usepackage{amsmath,adjustbox}
\usepackage{animate}
\usepackage[section]{placeins}
\usepackage{tikz-mirror-lens}
\usepackage{multicol}
\usepackage{multirow}
\usepackage{abstract}
\usepackage{enumitem}
\usepackage{csquotes}
\usepackage{indentfirst}

\usepackage[toc]{multitoc}

% ========== FHZ packages
%\usepackage{FHZ-listings-showexpl-style}
%\usepackage{FHZ-tcolorbox}
\usepackage[skins,listings,breakable,raster]{tcolorbox}
% ----------------------------------------------------------
%\usepackage{FHZ-textos}
\newcommand{\TikZ}{Ti\textit{k}Z}
% ----------------------------------------------------------

% ----------------------------------------------------------
% \usepackage{FHZ-formatacao-hf-headings_h_R_thepage}
\usepackage{fancyhdr}
\pagestyle{headings} % fancy, myheadings
%\fancypagestyle{plain}{ % alteração de estilo pré-definido.
%	\fancyhf{} % clear all header and footer fields
%	\fancyhead[R]{\thepage} % except the right top corner
%	\renewcommand{\headrulewidth}{0pt} % remove line between header and main text
%}
% ----------------------------------------------------------

% ----------------------------------------------------------
% \usepackage{FHZ-formatacao-subcaption}
\usepackage{graphicx}
\usepackage{caption}
\usepackage{subcaption}
\usepackage{adjustbox}
%\captionsetup{font=small,labelsep=period,textfont=bf,labelfont=bf,textformat=period}
\captionsetup{font=small,textformat=period} % ,labelfont=bf,labelsep=period,textfont=bf
\captionsetup[table]{position=top}
\captionsetup[figure]{position=below}
\captionsetup[subtable]{textfont={}, font=footnotesize}
\captionsetup[subfigure]{textfont={}, font=footnotesize}
% ==========

% ----------------------------------------------------------
%\usepackage{FHZ-capa-article}
\newcommand{\FHZCapaArticleCabecalho}[3]{
	\begin{center}
		\Large{#1}

		{#2}
	\end{center}
	\begin{center}
		\Large
    {#3}
	\end{center}
}
% ----------------------------------------------------------

\usepackage[colorlinks]{hyperref}

\newtcolorbox{FHZmirroLensTcolorbox}{
  enhanced, breakable,
  colback=cyan!10!white,
  colframe=blue!90!black,
}

% ========== Dados capa folha rosto ========== Igual entre versões PT e EN.
\newcommand{\edicao}{1}
\newcommand{\versao}{1.0.2}

\newcommand{\Cidade}{\textbf{tikz-mirror-lens package}\\} %{Cidade --}
\newcommand{\Estado}{\url{https://www.ctan.org/pkg/tikZ-mirror-lens}\\} %{Estado --}
% ======================
\newcommand{\AutorA}{\textbf{FHZ}}
% ====================== Input_Folha_Rosto_Livro_Versao