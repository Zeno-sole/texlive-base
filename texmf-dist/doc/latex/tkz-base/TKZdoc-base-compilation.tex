\section{Compilation of examples}
%–––––––––––––––––––––––––––––––––––––––––––––––––––––––––––––––––––––––––––>
\subsection{Installation test}  
The code below allows you to test your installation of \tkzname{tkz-base}. Please note that \NamePack{xfp} as well as \NamePack{numprint} must be present as well as version 3.01 (or higher) of \tkzNamePack{pgf}. All examples and this documentation have been compiled using Lua\LATEX. 

\medskip
\begin{minipage}{0.45\textwidth}
{%\setlength\linewidth{12cm}
\begin{tkzltxexample}[right margin=6pt]  
\documentclass{standalone}
\usepackage{tkz-base}
\begin{document}
\begin{tikzpicture}
 \tkzInit[xmax=4,ymax=4]
 \tkzGrid
 \tkzAxeXY
\end{tikzpicture}
\end{document}
\end{tkzltxexample}}
\end{minipage}
\begin{minipage}{0.45\textwidth}
\begin{tikzpicture}
 \tkzInit[xmax=4,ymax=4]
 \tkzGrid
 \tkzAxeXY
\end{tikzpicture}
\end{minipage}

\emph{Notes on this test} 

\begin{enumerate}
\item The compilation of this document and examples is obtained with \tkzimp{lua\LATEX}.
\item  \tkzNamePack{tkz-base} loads \tkzNamePack{numprint} with the option \tkzNamePack{autolanguage}, \tkzNamePack{xfp} and of course {\TIKZ}.
\item \TIKZ\  seems that version 3 of pgf has fixed those problems. In case of difficulty, it is recommended to load the \NameLib{babel} library with \tkzcname{usetikzlabry\{babel\}}. Another possibility is to compile with Lua\LATEX.
\end{enumerate} 

\subsection {\tkzNamePack{xfp} and \tkzNamePack{numprint}} 

\tkzNamePack{xfp} now replaces \tkzNamePack{fp} in this package. One of the advantages for the user is a simplified syntax. It allows to manage calculations on large or very small numbers with precision. This slows down the compilation a bit, so it is better not to overuse it. \tkzNamePack{xfp} is used above all, to obtain correct graduations.                           
\tkzNamePack{numprint} was present when I started to write this series of packages, since \tkzNamePack{siunitx} has grown and I can understand that some people prefer it. In a future version, I plan to leave the choice of the package for displaying numbers.

\endinput        
