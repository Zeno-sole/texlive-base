\section{Lines parallel to the axes} 

\subsection{ Draw a horizontal line with \tkzcname{tkzHLine}}  \hypertarget{thl}{}
\tkzHandBomb The syntax is that of \tkzname{xfp}!   
\begin{NewMacroBox}{tkzHLine}{\oarg{local options}\marg{decimal number}}%
\begin{tabular}{lll}%
arguments &  example & definition  \\ 
\midrule
\TAline{decimal number}{\tkzcname{tkzHLine\{1\}}}{Draw the straight line $y=1$}
\bottomrule
\end{tabular} 

\medskip
\begin{tabular}{lll}%  
options  & default & definition             \\   
\midrule
\TOline{color     }{|black| }{  line colour}
\TOline{line width}{|0.6pt| }{  point thickness}
\TOline{style     }{|solid|}{  line style }
\bottomrule
\end{tabular}

{see the lines options  in \TIKZ} 
\end{NewMacroBox}   

\subsubsection{Horizontal line }

\begin{tkzexample}[latex=7cm,small] 
\begin{tikzpicture}[scale=2]
   \tkzInit[xmax=3,ymax=1.5]
   \tkzAxeXY
   \tkzHLine[color      = blue,
             style      = dashed,
             line width = 2pt]{1}
\end{tikzpicture}
\end{tkzexample} 

   

\subsubsection{Horizontal line and value calculated by \tkzname{xfp} }
\begin{tkzexample}[latex=7cm,small]
\begin{tikzpicture}
  \tkzInit[xmin=-3,xmax=3,ymin=-2,ymax=1.5]
  \foreach\v in {-1,1}
  {\tkzHLine[color=red]{\v*pi/2}}
  \tkzDrawX
  \tkzAxeY[trig=2]
  \tkzLabelY
\end{tikzpicture}
\end{tkzexample}

\subsection{Horizontal lines with \tkzcname{tkzHLines} }  
\hypertarget{thls}{} 
\tkzHandBomb The syntax is that of \tkzname{xfp}! 
\begin{NewMacroBox}{tkzHLines}{\oarg{local options}\marg{list of values}}%
\begin{tabular}{lll}%
arguments &  example & definition  \\ 
\midrule
\TAline{list of values}{\tkzcname{tkzHLines\{1,4\}}}{draws the lines $y=1$ and $y=4$}
\end{tabular} 
\end{NewMacroBox}  

\subsubsection{Horizontal lines}  


\begin{tkzexample}[latex=7cm,small] 
\begin{tikzpicture}
 	\tkzInit[xmax=5,ymax=4]
 	\tkzAxeXY
	 \tkzHLines[color = magenta]{1,...,3}
\end{tikzpicture} 
\end{tkzexample}


\subsection{ Draw a vertical line with \tkzcname{tkzVLine}} \hypertarget{tvl}{} 
\tkzHandBomb The syntax is that of \tkzname{xfp}!
\begin{NewMacroBox}{tkzVLine}{\oarg{local options}\marg{decimal number}}%
\begin{tabular}{lll}%
arguments &  example & definition  \\
 
\midrule
\TAline{decimal number}{\tkzcname{tkzVLine\{1\}}}{Draw the line $x=1$}
\bottomrule
\end{tabular} 

\medskip
\begin{tabular}{lll}%  
\toprule
options  & default & definition             \\   
\midrule
\TOline{color     }{|black| }{line colour}
\TOline{line width}{|0.6pt| }{point thickness}
\TOline{style     }{|solid|}{line style }
\bottomrule
\end{tabular}

See options the lines in \TIKZ. 
\end{NewMacroBox}


\subsubsection{Vertical line }

\begin{tkzexample}[latex=8cm,small] 
\begin{tikzpicture}[scale=2]
   \tkzInit[xmax=3,ymax=1]
   \tkzAxeXY
   \tkzVLine[color      = blue,
             style      = dashed,
             line width = 2pt]{1/3}
\end{tikzpicture}
\end{tkzexample}      

\subsubsection{Vertical line and value calculated by \tkzname{xfp} }
\begin{tkzexample}[latex=8cm,small]
\begin{tikzpicture}
  \tkzInit[xmax=7,ymin=-1,ymax=1]
  \foreach\v in {1,2}
  {\tkzVLine[color=red]{\v*pi}}
  \tkzDrawY
  \tkzAxeX[trig=2]
  \tkzLabelY
\end{tikzpicture}
\end{tkzexample}


\subsection{Vertical lines with \tkzcname{tkzVLines} }  
\hypertarget{tvls}{}  
\tkzHandBomb The syntax is that of \tkzname{xfp}!
\begin{NewMacroBox}{tkzVLines}{\oarg{local options}\marg{list of values}}%
\begin{tabular}{lll}%
arguments &  example & definition  \\ 
\midrule
\TAline{list of values}{\tkzcname{tkzVLines\{1,4\}}}{Trace the lines $x=$1 and $x=4$}
\end{tabular} 
\end{NewMacroBox}  

\subsubsection{Vertical lines}  

\begin{tkzexample}[latex=7cm,small]
\begin{tikzpicture}
 \tkzInit[xmax=5,ymax=2]
 \tkzAxeXY
 \tkzVLines[color = green]{1,2,...,4}
\end{tikzpicture}
\end{tkzexample}

\section{Ticks on the axes} 
%<–––––––––––––––––––––––––––––––––––––––––––––––––––––––––––––––––––––––––––>
\subsection{ Drawing one tick on the abscissa axis \tkzcname{tkzHTick}} \hypertarget{tht}{} 
\begin{NewMacroBox}{tkzHTick}{\oarg{local options}\marg{decimal number}}%
\begin{tabular}{lll}%
arguments &  example & definition  \\ 
\midrule
\TAline{decimal number}{\tkzcname{tkzHTick\{1\}}}{the abscissa of the tick is 1}
\bottomrule
\end{tabular} 

\medskip
\begin{tabular}{lll}%  
options  & default & definition             \\   
\midrule
\TOline{mark     }{* }{full disk}
\TOline{mark size}{3 pt }{symbol size}
\TOline{mark options}{empty}{allows you to use color for example}
\bottomrule
\end{tabular}  

See options for \TIKZ. 
\end{NewMacroBox} 

\subsubsection{Example} 

\begin{tkzexample}[latex=7cm,small] 
\begin{tikzpicture}
  \tkzInit[xmax=6]
  \tkzDrawX
  \tkzHTick[mark=ball,mark size=3pt]{pi/2} 
  \tkzHTick[mark=*,
     mark options={color=purple}]{2*exp(1)}
\end{tikzpicture}    
\end{tkzexample}

\subsection{ Drawing ticks on the abscissa axis \tkzcname{tkzHTicks}} \hypertarget{thts}{} 
\begin{NewMacroBox}{tkzHTicks}{\oarg{local options}\marg{list of numbers}}%
\begin{tabular}{lll}
arguments &  example & definition  \\ 
\midrule
\TAline{decimal number}{\tkzcname{tkzHTicks\{1\}}}{the abscissa of the tick is 1}
\bottomrule
\end{tabular} 

See options for \TIKZ.  
\end{NewMacroBox} 

\subsection{ Drawing one tick on the ordinate axis \tkzcname{tkzVTick}} \hypertarget{tvt}{} 
\begin{NewMacroBox}{tkzVTick}{\oarg{local options}\marg{decimal number}}%
\begin{tabular}{lll}%
arguments &  example & definition  \\ 
\midrule
\TAline{decimal number}{\tkzcname{tkzVTick\{1\}}}{the ordinate of the tick is 1}
\bottomrule
\end{tabular} 

See options for \TIKZ.  
\end{NewMacroBox} 

\subsection{ Drawing ticks on the ordinate axis \tkzcname{tkzVTicks}} \hypertarget{tvts}{} 
\begin{NewMacroBox}{tkzVTicks}{\oarg{local options}\marg{decimal number}}%
\begin{tabular}{lll}
arguments &  example & definition  \\ 
\midrule
\TAline{decimal number}{\tkzcname{tkzVTicks\{1,3\}}}{the ordinates of the ticks are 1 and 3}
\bottomrule
\end{tabular}

See options for \TIKZ.  
\end{NewMacroBox} 

 \endinput