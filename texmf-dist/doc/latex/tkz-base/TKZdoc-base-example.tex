\section{Here are a few examples}

Here is a very simple example that shows you that once the reference is defined, it is easy to work with the tools of my module. It is however possible to use the Tikz tools, but in this case you have to go back to the original coordinate system.

\subsection{Recipe by month}

\begin{tkzexample}[vbox,small]
\begin{tikzpicture}
  \tkzInit[xmax=12,ymin=1000,ymax=11000,ystep=1000]
  \tkzClip[space=2]
  \tkzAxeX[label=month,below=10pt]
  \tkzAxeY[label=Recipe]
  \tkzDefPoint(1,2000){A} 
  \tkzDefPoint(2,3000){B}
  \tkzDefPoint(4,2500){C} 
  \tkzDefPoint(5,4200){D} 
  \tkzDrawSegments[color=brown!50](A,B B,C C,D)  
  \tkzDrawMarks[mark=ball](A,B,C,D)           
  \tkzText[draw,color = red,fill = red!10,text width=3cm](5,6000)%
  {\begin{center}\color{blue}%
  Recipe by month\end{center}%
  }  
\end{tikzpicture} 
\end{tkzexample}

\endinput
