\subsection{Coloring a disc}
This was possible with the macro \tkzcname{tkzDrawCircle}, but disk tracing was mandatory, this is no longer the case.
  
\begin{NewMacroBox}{tkzFillCircle}{\oarg{local options}\parg{A,B}}%
\begin{tabular}{lll}%
options             & default & definition                         \\ 
\midrule
\TOline{radius}  {radius}{two points define a radius}
\TOline{R} {radius}{a point and the measurement of a radius }
\bottomrule
\end{tabular}

\medskip
You don't need to put \tkzname{radius} because that's the default option. Of course, you have to add all the styles of \TIKZ\ for the plots.
\end{NewMacroBox}  


\subsubsection{Yin and Yang} 
\begin{tkzexample}[latex=8cm,small]
  \begin{tikzpicture}[scale=.75]
    \tkzDefPoint(0,0){O}
    \tkzDefPoint(-4,0){A}
    \tkzDefPoint(4,0){B}
    \tkzDefPoint(-2,0){I}
    \tkzDefPoint(2,0){J}
    \tkzDrawSector[fill=teal](O,A)(B)
    \tkzFillCircle[fill=white](J,B) 
    \tkzFillCircle[fill=teal](I,A) 
    \tkzDrawCircle(O,A) 
  \end{tikzpicture}
\end{tkzexample}


\subsubsection{From a sangaku} 

\begin{tkzexample}[latex=7cm,small]
\begin{tikzpicture}
   \tkzDefPoint(0,0){B}  \tkzDefPoint(6,0){C}%
   \tkzDefSquare(B,C)    \tkzGetPoints{D}{A} 
   \tkzClipPolygon(B,C,D,A) 
   \tkzDefMidPoint(A,D)  \tkzGetPoint{F}
   \tkzDefMidPoint(B,C)  \tkzGetPoint{E}
   \tkzDefMidPoint(B,D)  \tkzGetPoint{Q}           
   \tkzDefLine[tangent from = B](F,A) \tkzGetPoints{H}{G} 
   \tkzInterLL(F,G)(C,D) \tkzGetPoint{J}
   \tkzInterLL(A,J)(F,E) \tkzGetPoint{K}
   \tkzDefPointBy[projection=onto B--A](K)   
   \tkzGetPoint{M}  
   \tkzDrawPolygon(A,B,C,D)
   \tkzFillCircle[red!20](E,B)
   \tkzFillCircle[blue!20](M,A)
   \tkzFillCircle[green!20](K,Q)
  \tkzDrawCircles(B,A M,A E,B K,Q) 
\end{tikzpicture}
\end{tkzexample} 

\subsubsection{Clipping and filling part I} 
\begin{tkzexample}[latex=7cm,small]
\begin{tikzpicture}
\tkzDefPoints{0/0/A,4/0/B,2/2/O,3/4/X,4/1/Y,1/0/Z,
              0/3/W,3/0/R,4/3/S,1/4/T,0/1/U}
\tkzDefSquare(A,B)\tkzGetPoints{C}{D}
\tkzDefPointWith[colinear normed=at X,K=1](O,X)
 \tkzGetPoint{F}
\begin{scope}
  \tkzFillCircle[fill=teal!20](O,F)
  \tkzFillPolygon[white](A,...,D)
  \tkzClipPolygon(A,...,D)
  \foreach \c/\t in {S/C,R/B,U/A,T/D}
  {\tkzFillCircle[teal!20](\c,\t)}
\end{scope}
\foreach \c/\t in {X/C,Y/B,Z/A,W/D}
{\tkzFillCircle[white](\c,\t)}
  \foreach \c/\t in {S/C,R/B,U/A,T/D}
  {\tkzFillCircle[teal!20](\c,\t)}
\end{tikzpicture}
\end{tkzexample}

\subsubsection{Clipping and filling part II} 
\begin{tkzexample}[latex=7cm, small]  
\begin{tikzpicture}[scale=.75]
\tkzDefPoints{0/0/A,8/0/B,8/8/C,0/8/D}
\tkzDefMidPoint(A,B) \tkzGetPoint{F}
\tkzDefMidPoint(B,C) \tkzGetPoint{E}
\tkzDefMidPoint(D,B) \tkzGetPoint{I}
\tkzDefMidPoint(I,B) \tkzGetPoint{a}
\tkzInterLC(B,I)(B,C) \tkzGetSecondPoint{K}
\tkzDefMidPoint(I,K) \tkzGetPoint{b}
\begin{scope}
  \tkzFillSector[fill=blue!10](B,C)(A)
  \tkzDefMidPoint(A,B) \tkzGetPoint{x}
  \tkzDrawSemiCircle[fill=white](x,B)
  \tkzDefMidPoint(B,C) \tkzGetPoint{y}
  \tkzDrawSemiCircle[fill=white](y,C)
  \tkzClipCircle(E,B)
  \tkzClipCircle(F,B)
  \tkzFillCircle[fill=blue!10](B,A)
\end{scope}
\tkzDrawSemiCircle[thick](F,B)
\tkzDrawSemiCircle[thick](E,C)
\tkzDrawArc[thick](B,C)(A)
\tkzDrawSegments[thick](A,B B,C)
\tkzDrawPoints(A,B,C,E,F)
\tkzLabelPoints[centered](a,b)
\tkzLabelPoints(A,B,C,E,F)
\end{tikzpicture}
\end{tkzexample}

\subsubsection{Clipping and filling part III} 

\begin{tkzexample}[latex=7cm, small] 
\begin{tikzpicture}
  \tkzDefPoint(0,0){A} \tkzDefPoint(1,0){B}
  \tkzDefPoint(2,0){C} \tkzDefPoint(-3,0){a} 
  \tkzDefPoint(3,0){b}  \tkzDefPoint(0,3){c} 
  \tkzDefPoint(0,-3){d}
\begin{scope}
 \tkzClipPolygon(a,b,c,d)
 \tkzFillCircle[teal!20](A,C)
\end{scope}
 \tkzFillCircle[white](A,B)
 \tkzDrawCircle[color=red](A,C)
 \tkzDrawCircle[color=red](A,B)
\end{tikzpicture}
\end{tkzexample}

\subsection{Coloring a polygon} 
 \begin{NewMacroBox}{tkzFillPolygon}{\oarg{local options}\parg{points list}}%
You can color by drawing the polygon, but in this case you color the inside of the polygon without drawing it.

\medskip
\begin{tabular}{lll}%
\toprule
arguments                & example & explanation                         \\ 
\midrule
\TAline{\parg{pt1,pt2,\dots}}{\parg{A,B,\dots}}{}
%\bottomrule
 \end{tabular}
\end{NewMacroBox} 

\subsubsection{\tkzcname{tkzFillPolygon}} 
\begin{tkzexample}[latex=7cm, small]  
\begin{tikzpicture}[scale=.5]
   \tkzDefPoint(0,0){C} \tkzDefPoint(4,0){A}
   \tkzDefPoint(0,3){B}
   \tkzDefSquare(B,A)      \tkzGetPoints{E}{F}
   \tkzDefSquare(A,C)      \tkzGetPoints{G}{H}
   \tkzDefSquare(C,B)       \tkzGetPoints{I}{J}
   \tkzFillPolygon[color  =  orange!30   ](A,C,G,H)
   \tkzFillPolygon[color  =  teal!40  ](C,B,I,J)
   \tkzFillPolygon[color  =  purple!20](B,A,E,F)
   \tkzDrawPolygon[line width  =  1pt](A,B,C)
   \tkzDrawPolygon[line width  =  1pt](A,C,G,H)
   \tkzDrawPolygon[line width  =  1pt](C,B,I,J)
   \tkzDrawPolygon[line width  =  1pt](B,A,E,F)
   \tkzLabelSegment[above](C,A){$a$}
   \tkzLabelSegment[right](B,C){$b$}
   \tkzLabelSegment[below left](B,A){$c$}
\end{tikzpicture}
\end{tkzexample}

\subsection{\tkzcname{tkzFillSector}}
\tkzHandBomb\ Attention the arguments vary according to the options. 
\begin{NewMacroBox}{tkzFillSector}{\oarg{local options}\parg{O,\dots}\parg{\dots}}%
\begin{tabular}{lll}%
options          & default & definition      \\ 
\midrule
\TOline{towards}{towards}{$O$ is the center and the arc from $A$ to $(OB)$}
\TOline{rotate} {towards}{the arc starts from A and the angle determines its length } 
\TOline{R}{towards}{We give the radius and two angles}
\TOline{R with nodes}{towards}{We give the radius and two points}
\bottomrule
\end{tabular} 

\medskip
Of course, you have to add all the styles of \TIKZ\ for the tracings...

\medskip
\begin{tabular}{lll}%
\toprule
options             & arguments & example                         \\ 
\midrule
\TOline{towards}{\parg{pt,pt}\parg{pt}}{\tkzcname{tkzFillSector(O,A)(B)}}
\TOline{rotate} {\parg{pt,pt}\parg{an}}{\tkzcname{tkzFillSector[rotate,color=red](O,A)(90)}}
\TOline{R}{\parg{pt,$r$}\parg{an,an}}{\tkzcname{tkzFillSector[R,color=blue](O,2)(30,90)}} 
\TOline{R with nodes}{\parg{pt,$r$}\parg{pt,pt}}{\tkzcname{tkzFillSector[R with nodes](O,2)(A,B)}}
\end{tabular}   
\end{NewMacroBox} 

\subsubsection{\tkzcname{tkzFillSector} and \tkzname{towards}} 
It is useless to put \tkzname{towards} and you will notice that the contours are not drawn, only the surface is colored.
\begin{tkzexample}[latex=5.75cm,small]
  \begin{tikzpicture}[scale=.6] 
  \tkzDefPoint(0,0){O}
  \tkzDefPoint(-30:3){A}
  \tkzDefPointBy[rotation = center O angle -60](A)   
  \tkzFillSector[fill=purple!20](O,A)(tkzPointResult) 
    \begin{scope}[shift={(-60:1)}]
    \tkzDefPoint(0,0){O}
    \tkzDefPoint(-30:3){A}
    \tkzDefPointBy[rotation = center O angle -60](A)
    \tkzGetPoint{A'}
    \tkzFillSector[color=teal!40](O,A')(A)
      \end{scope}
  \end{tikzpicture}
\end{tkzexample}


\subsubsection{\tkzcname{tkzFillSector} and \tkzname{rotate}} 
\begin{tkzexample}[latex=5.75cm,small]
\begin{tikzpicture}[scale=1.5]
 \tkzDefPoint(0,0){O} \tkzDefPoint(2,2){A}
 \tkzFillSector[rotate,color=purple!20](O,A)(30)
 \tkzFillSector[rotate,color=teal!40](O,A)(-30)
\end{tikzpicture}    
\end{tkzexample} 

\subsection{Colour an angle: \tkzcname{tkzFillAngle}}

The simplest operation
\begin{NewMacroBox}{tkzFillAngle}{\oarg{local options}\parg{A,O,B}}%
$O$ is the vertex of the angle. $OA$ and $OB$ are the sides. Attention the angle is determined by the order of the points.

\medskip

\begin{tabular}{lll}%
\toprule
options             & default & definition                        \\ 
\midrule
\TOline{size}{1}{this option determines the radius of the coloured angular sector.}

\bottomrule
\end{tabular} 

\medskip
Of course, you have to add all the styles of \TIKZ, like the use of fill and shade... 
\end{NewMacroBox}  

\subsubsection{Example with \tkzname{size}}  
\begin{tkzexample}[latex=7cm,small]
\begin{tikzpicture} 
   \tkzInit 
   \tkzDefPoints{0/0/O,2.5/0/A,1.5/2/B}
   \tkzFillAngle[size=2, fill=gray!10](A,O,B)
   \tkzDrawLines(O,A O,B)
   \tkzDrawPoints(O,A,B)
\end{tikzpicture}
\end{tkzexample}


\subsubsection{Changing the order of items} 
\begin{tkzexample}[latex=7cm,small]
\begin{tikzpicture} 
   \tkzInit 
   \tkzDefPoints{0/0/O,2.5/0/A,1.5/2/B}
   \tkzFillAngle[size=2,fill=gray!10](B,O,A)
   \tkzDrawLines(O,A O,B)
   \tkzDrawPoints(O,A,B)
\end{tikzpicture}
\end{tkzexample}

\begin{tkzexample}[latex=7cm,small]
\begin{tikzpicture} 
   \tkzInit 
   \tkzDefPoints{0/0/O,5/0/A,3/4/B}
   % Don't forget {} to get, () to use
   \tkzFillAngle[size=4,left color=white, 
                 right color=red!50](A,O,B)
   \tkzDrawLines(O,A O,B)
   \tkzDrawPoints(O,A,B)
\end{tikzpicture}
\end{tkzexample}

\begin{NewMacroBox}{tkzFillAngles}{\oarg{local options}\parg{A,O,B}\parg{A',O',B'}etc.}%
With common options, there is a macro for multiple angles.
  \end{NewMacroBox}  
  
\subsubsection{Multiples angles}  
\begin{tkzexample}[latex=5cm,small]
\begin{tikzpicture}[scale=0.5]
  \tkzDefPoints{0/0/B,8/0/C,0/8/A,8/8/D}
  \tkzDrawPolygon(B,C,D,A)
  \tkzDefTriangle[equilateral](B,C) \tkzGetPoint{M}
  \tkzInterLL(D,M)(A,B) \tkzGetPoint{N}
  \tkzDefPointBy[rotation=center N angle -60](D) 
  \tkzGetPoint{L}
  \tkzInterLL(N,L)(M,B)     \tkzGetPoint{P}
  \tkzInterLL(M,C)(D,L)     \tkzGetPoint{Q}
  \tkzDrawSegments(D,N N,L L,D B,M M,C)
  \tkzDrawPoints(L,N,P,Q,M,A,D) 
  \tkzLabelPoints[left](N,P,Q)
  \tkzLabelPoints[above](M,A,D)
  \tkzLabelPoints(L,B,C)
  \tkzMarkAngles(C,B,M B,M,C M,C,B D,L,N L,N,D N,D,L)
  \tkzFillAngles[fill=red!20,opacity=.2](C,B,M%
      B,M,C M,C,B D,L,N L,N,D N,D,L)
\end{tikzpicture}
\end{tkzexample} 
\endinput
