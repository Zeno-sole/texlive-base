% !TEX TS-program = lualatex
% encoding : utf8 
% Documentation of tkz-euclide v4
% Copyright 2022  Alain Matthes
% This work may be distributed and/or modified under the
% conditions of the LaTeX Project Public License, either version 1.3
% of this license or (at your option) any later version.
% The latest version of this license is in
% http://www.latex-project.org/lppl.txt
% and version 1.3 or later is part of all distributions of LaTeX
% version 2005/12/01 or later.
% This work has the LPPL maintenance status “maintained”.
% The Current Maintainer of this work is Alain Matthes.
\PassOptionsToPackage{unicode}{hyperref}

\documentclass[DIV         = 14,
               fontsize    = 10,
               index       = totoc,
               twoside,
               cadre,
               headings    = small
               ]{tkz-doc}
%\usepackage{etoc}
\gdef\tkznameofpack{tkz-euclide}
\gdef\tkzversionofpack{4.25c}
\gdef\tkzdateofpack{\today}
\gdef\tkznameofdoc{doc-tkz-euclide}
\gdef\tkzversionofdoc{4.25c} 
\gdef\tkzdateofdoc{\today}
\gdef\tkzauthorofpack{Alain Matthes}
\gdef\tkzadressofauthor{}
\gdef\tkznamecollection{AlterMundus}
\gdef\tkzurlauthor{http://altermundus.fr}
\gdef\tkzengine{lualatex}
\gdef\tkzurlauthorcom{http://altermundus.fr}
\nameoffile{\tkznameofpack}
% -- Packages ---------------------------------------------------          
\usepackage[dvipsnames,svgnames]{xcolor}
\usepackage{calc}
\usepackage{tkz-base,tkz-euclide,pgfornament} 
\usetikzlibrary{backgrounds}
\usepackage[colorlinks,pdfencoding=auto, psdextra]{hyperref}
\hypersetup{
      linkcolor=Gray,
      citecolor=Green,
      filecolor=Mulberry,
      urlcolor=NavyBlue,
      menucolor=Gray,
      runcolor=Mulberry,
      linkbordercolor=Gray,
      citebordercolor=Green,
      filebordercolor=Mulberry,
      urlbordercolor=NavyBlue,
      menubordercolor=Gray,
      runbordercolor=Mulberry,
      pdfsubject={Euclidean Geometry},
      pdfauthor={\tkzauthorofpack},
      pdftitle={\tkznameofpack},
      pdfcreator={\tkzengine}
}
\usepackage{tkzexample}
\usepackage{fontspec}
\setmainfont{texgyrepagella}[
  Extension = .otf,
  UprightFont = *-regular ,
  ItalicFont  = *-italic  ,
  BoldFont    = *-bold    ,
  BoldItalicFont = *-bolditalic
]
\setsansfont{texgyreheros}[
  Extension = .otf,
  UprightFont = *-regular ,
  ItalicFont  = *-italic  ,
  BoldFont    = *-bold    ,
  BoldItalicFont = *-bolditalic ,
]

\setmonofont{lmmono10-regular.otf}[
  Numbers={Lining,SlashedZero},
  ItalicFont=lmmonoslant10-regular.otf,
  BoldFont=lmmonolt10-bold.otf,
  BoldItalicFont=lmmonolt10-boldoblique.otf,
]
\newfontfamily\ttcondensed{lmmonoltcond10-regular.otf}
%% (La)TeX font-related declarations:
\linespread{1.05}      % Pagella needs more space between lines
%\usepackage{unicode-math}
\usepackage[math-style=literal,bold-style=literal]{unicode-math}
\usepackage{fourier-otf}
\let\rmfamily\ttfamily
\usepackage{multicol,lscape}
\usepackage[english]{babel}
\usepackage[normalem]{ulem}
\usepackage{multirow,multido,booktabs,cellspace}
\usepackage{shortvrb,fancyvrb,bookmark} 
\usepackage{makeidx}
\makeindex 

%<---------------------------------------------------------------------------> 
% settings styles
\tkzSetUpColors[background=white,text=black]  
\tkzSetUpCompass[color=orange, line width=.2pt,delta=10]
\tkzSetUpArc[color=gray,line width=.2pt]
\tkzSetUpPoint[size=2,color=teal]
\tkzSetUpLine[line width=.2pt,color=teal]
\tkzSetUpStyle[color=orange,line width=.2pt]{new}
\tikzset{every picture/.style={line width=.2pt}}
\tikzset{label angle style/.append style={color=teal,font=\footnotesize}} 
\tikzset{label style/.append style={below,color=teal,font=\scriptsize}}
\tikzset{new/.style={color=orange,line width=.2pt}} 

\AtBeginDocument{\MakeShortVerb{\|}} % link to shortvrb
\begin{document} 
  
\parindent=0pt
\tkzTitleFrame{tkz-euclide\\Euclidean Geometry}
\clearpage

\defoffile{\lefthand\
From version 4.00, \tkzname{\tkznameofpack} became independent from  \tkzname{tkz-base} . This has implied some changes : the next major step will be the version 5 which will see the introduction of Lua. To prepare for this change, I removed the last macros that allowed to plot and define at the same time. Indeed Lua will be there to make all the calculations and define all the necessary nodes. As for \TIKZ\ , it will remain to carry out the tracings, the markings and the labels.\\
\tkzname{\tkznameofpack} is a set of convenient macros for drawing in a plane (fundamental two-dimensional object) with a Cartesian coordinate system. It  handles the most classic situations in Euclidean Geometry. \tkzname{\tkznameofpack} is built on top of PGF and its associated front-end \TIKZ\ and is a (La)TeX-friendly drawing package. The aim is to provide a high-level user interface  to build graphics  relatively simply.  The idea is to allow you to follow step by step a construction that would be done by hand as naturally as possible.\\
English is  not my native language so there  might be some errors.
}

\presentation

\vspace*{1cm}
\lefthand\ Firstly, I would like to thank \textbf{Till Tantau} for the  beautiful \LaTeX{}  package, namely  \href{http://sourceforge.net/projects/pgf/}{\TIKZ}.

\vspace*{12pt}
\lefthand\ Acknowledgements : I received much valuable advice, remarks, corrections and examples from \tkzimp{Jean-Côme Charpentier}, \tkzimp{Josselin Noirel}, \tkzimp{Manuel Pégourié-Gonnard}, \tkzimp{Franck Pastor}, \tkzimp{David Arnold}, \tkzimp{Ulrike Fischer}, \tkzimp{Stefan Kottwitz}, \tkzimp{Christian Tellechea}, \tkzimp{Nicolas Kisselhoff}, \tkzimp{David Arnold}, \tkzimp{Wolfgang Büchel}, \tkzimp{John Kitzmiller}, \tkzimp{Dimitri Kapetas}, \tkzimp{Gaétan Marris}, \tkzimp{Mark Wibrow}, \tkzimp{Yves Combe} for his work on a protractor, \tkzimp{Paul Gaborit}, \tkzimp{Laurent Van Deik} for all his corrections, remarks and questions and \tkzimp{Muzimuzhi Z} for the code about the option "dim".

\vspace*{12pt}
\lefthand\ I would also like to thank Eric Weisstein, creator of MathWorld:
\href{http://mathworld.wolfram.com/about/author.html}{MathWorld}.

\vspace*{12pt}
\lefthand\ You can find some examples on my site:
\href{http://altermundus.fr}{altermundus.fr}. \hspace{2cm} under construction!

\vfill
Please report typos or any other comments to this documentation to: \href{mailto:al.ma@mac.com}{\textcolor{blue}{Alain Matthes}}.

This file can be redistributed and/or modified under the terms of the \LaTeX{} 
Project Public License Distributed from \href{http://www.ctan.org/}{CTAN}\  archives.

\clearpage
\tableofcontents

\clearpage
\newpage

\part{General survey : a brief but comprehensive review}
\section{News and compatibility}


Some changes have been made to make the syntax more homogeneous and especially to distinguish the definition and search for coordinates from the rest, i.e. drawing, marking and labelling.
In the future, the definition macros being isolated, it will be easier to introduce a phase of coordinate calculations using \tkzimp{Lua}.

An important novelty is the recent replacement of the \tkzNamePack{fp} package by \tkzNamePack{xfp}.  This is to improve the calculations a little bit more and to make it easier to use.


Here are some of the changes. 
\vspace{1cm}
 \begin{itemize}\setlength{\itemsep}{10pt} 

\item Improved code and bug fixes;

\item With \tkzimp{tkz-euclide} loads all objects, so there's no need to place \tkzcname{usetkzobj\{all\}};\item The bounding box is now controlled in each macro (hopefully) to avoid the use of \tkzcname{tkzInit} followed by \tkzcname{tkzClip};\item Added macros for the bounding box: \tkzcname{tkzSaveBB} \tkzcname{tkzClipBB} and so on;\item  Logically most macros accept \TIKZ\ options. So I removed the "duplicate" options when possible thus the "label options" option is removed;

\item Random points are now in \tkzname{\tkznameofpack} and the macro \tkzcname{tkzGetRandPointOn} is replaced by \tkzcname{tkzDefRandPointOn}. For homogeneity reasons, the points must be retrieved with \tkzcname{tkzGetPoint};

\item The options \tkzname{end} and \tkzname{start} which allowed to give a label to a straight  line are removed. You now have to use the macro \tkzcname{tkzLabelLine};

\item Introduction of the libraries \NameLib{quotes} and \NameLib{angles}; it allows to give a label to a point, even if I am not in favour of this practice;

\item  The notion of vector disappears, to draw a vector just pass "->" as an option to \tkzcname{tkzDrawSegment};

\item Many macros still exist, but are obsolete and will disappear:
\begin{itemize}

\item |\tkzDrawMedians| trace and create midpoints on the sides of a triangle. The creation and drawing separation is not respected so it is preferable to first create the coordinates of these points with |\tkzSpcTriangle[median]| and then to choose the ones you are going to draw with |\tkzDrawSegments| or |\tkzDrawLines|;

\item |\tkzDrawMedians(A,B)(C)| is now spelled |\tkzDrawMedians(A,C,B)|. This defines the median from $C$;
  
\item Another example |\tkzDrawTriangle[equilateral]| was handy but it is better to get the third point with |\tkzDefTriangle[equilateral]| and then draw with |\tkzDrawPolygon|;
  
\item |\tkzDefRandPointOn| is replaced by |\tkzGetRandPointOn|;\item now |\tkzTangent| is replaced by |\tkzDefTangent|;

\item You can use |global path name| if you want find intersection  but it's very slow like in \TIKZ.

\end{itemize}


\item Appearance of the macro \tkzcname{usetkztool} which allows to load new "tools".
\end{itemize}

\endinput
 \section{Installation}

\tkzNamePack{tkz-euclide} and \tkzNamePack{tkz-base} are now on the server of the \tkzname{CTAN}\footnote{\tkzNamePack{tkz-base} and \tkzNamePack{tkz-euclide} are part of \NameDist{TeXLive} and \tkzname{tlmgr} allows you to install them. These packages are also part of \NameDist{MiKTeX} under \NameSys{Windows}.}. If you want to test a beta version, just put the following files in a texmf folder that your system can find.
You will have to check several points:

\begin{itemize}\setlength{\itemsep}{5pt}
\item  The \tkzNamePack{tkz-base} and \tkzNamePack{tkz-euclide} folders must be located on a path recognized by \tkzname{latex}.
\item  The \tkzNamePack{xfp}\footnote{\tkzNamePack{xfp} replaces \tkzNamePack{fp}.}, \tkzNamePack{numprint} and \tkzNamePack{tikz 3.00} must be installed as they are mandatory, for the proper functioning of \tkzNamePack{tkz-euclide}.
\item This documentation and all examples were obtained with \tkzname{lualatex-dev} but \tkzname{pdflatex} should be suitable.
\end{itemize}

\subsection{List of folder files \tkzname{tkzbase}  and \tkzname{tkzeuclide}}

In the folder \tkzname{base}:

\begin{itemize}
\item  \tkzname{tkz-base.cfg}
\item  \tkzname{tkz-base.sty}
\item  \tkzname{tkz-lib-marks.tex}
\item  \tkzname{tkz-obj-axes.tex}
\item  \tkzname{tkz-obj-grids.tex}
\item  \tkzname{tkz-obj-marks.tex}
\item  \tkzname{tkz-obj-points.tex}
\item  \tkzname{tkz-obj-rep.tex}
\item  \tkzname{tkz-tools-arith.tex}
\item  \tkzname{tkz-tools-base.tex}
\item  \tkzname{tkz-tools-BB.tex}
\item  \tkzname{tkz-tools-misc.tex}
\item  \tkzname{tkz-tools-modules.tex}
\item  \tkzname{tkz-tools-print.tex}
\item  \tkzname{tkz-tools-text.tex}
\item  \tkzname{tkz-tools-utilities.tex}
\end{itemize}

In the folder \tkzname{euclide}:

\begin{itemize}
\item   \tkzname{tkz-euclide.sty}
\item   \tkzname{tkz-obj-eu-angles.tex}
\item   \tkzname{tkz-obj-eu-arcs.tex}
\item   \tkzname{tkz-obj-eu-circles.tex}
\item   \tkzname{tkz-obj-eu-compass.tex}
\item   \tkzname{tkz-obj-eu-draw-circles.tex}
\item   \tkzname{tkz-obj-eu-draw-lines.tex}
\item   \tkzname{tkz-obj-eu-draw-polygons.tex}
\item   \tkzname{tkz-obj-eu-draw-triangles.tex}
\item   \tkzname{tkz-obj-eu-lines.tex}
\item   \tkzname{tkz-obj-eu-points-by.tex}
\item   \tkzname{tkz-obj-eu-points-rnd.tex}
\item   \tkzname{tkz-obj-eu-points-with.tex}
\item   \tkzname{tkz-obj-eu-points.tex}
\item   \tkzname{tkz-obj-eu-polygons.tex}
\item   \tkzname{tkz-obj-eu-protractor.tex}
\item   \tkzname{tkz-obj-eu-sectors.tex}
\item   \tkzname{tkz-obj-eu-show.tex}
\item   \tkzname{tkz-obj-eu-triangles.tex}
\item   \tkzname{tkz-tools-angles.tex}
\item   \tkzname{tkz-tools-intersections.tex}
\item   \tkzname{tkz-tools-math.tex}
\end{itemize}
\tkzHandBomb\ Now \tkzname{tkz-euclide} loads all the files. 
\endinput

\section{Presentation and Overview}

\begin{tkzexample}[latex=5cm,small]
  \begin{tikzpicture}[scale=.25]
  \tkzDefPoints{00/0/A,12/0/B,6/12*sind(60)/C}
  \foreach \density in {20,30,...,240}{%
    \tkzDrawPolygon[fill=teal!\density](A,B,C)
    \pgfnodealias{X}{A}
    \tkzDefPointWith[linear,K=.15](A,B) \tkzGetPoint{A}
    \tkzDefPointWith[linear,K=.15](B,C) \tkzGetPoint{B}
    \tkzDefPointWith[linear,K=.15](C,X) \tkzGetPoint{C}}
  \end{tikzpicture}
\end{tkzexample}

\vspace*{12pt}

\subsection{Why \tkzname{\tkznameofpack}? }
My initial goal was to provide other mathematics teachers and myself with a tool to quickly create Euclidean geometry figures without investing too much effort in learning a new programming language.
Of course, \tkzname{\tkznameofpack}  is for math teachers who use \LATEX\ and  makes it possible to easily create correct  drawings by means of \LATEX.

It appeared that the simplest method was to reproduce the one used to obtain construction by hand. 
To describe a construction, you must, of course, define the objects but also the actions that you perform. It seemed to me that syntax close to the language of mathematicians and their students would be more easily understandable; moreover, it also seemed to me that this syntax should be close to that of \LaTeX. 
The objects, of course, are points, segments, lines, triangles, polygons and circles. As for actions, I considered five to be sufficient, namely: define, create, draw, mark and label.

The syntax is perhaps too verbose but it is, I believe, easily accessible.
As a result, the students like teachers were able to easily access this tool.

\subsection{\tkzname{\tkznameofpack}  vs \tkzname{\TIKZ } }

I love programming with  \TIKZ,  and without  \TIKZ\  I would never have had the idea to create \tkzname{\tkznameofpack}  but never forget that behind it there is  \TIKZ\  and that it is always possible to insert code from  \TIKZ. \tkzname{\tkznameofpack}  doesn't prevent you from using  \TIKZ.
That said, I don't think mixing syntax is a good thing. 

There is no need to compare \TIKZ\  and \tkzname{\tkznameofpack}.  The latter is not addressed to the same audience as  \TIKZ. The first one allows you to do a lot of things, the second one only does geometry drawings. The first one can do everything the second one does, but the second one will more easily do what you want.

\subsection{How it works}

\subsubsection{Example Part I: gold triangle}
\begin{center}
\begin{tikzpicture}
  
\tkzDefPoint(0,0){C} % possible \tkzDefPoint[label=below:$C$](0,0){C} but don't do this
\tkzDefPoint(2,6){B}
% We get D and E with a rotation
\tkzDefPointBy[rotation= center B angle 36](C) \tkzGetPoint{D} 
\tkzDefPointBy[rotation= center B angle 72](C) \tkzGetPoint{E} 
% Toget A we use an intersection of lines
\tkzInterLL(B,E)(C,D) \tkzGetPoint{A}
\tkzInterLL(C,E)(B,D) \tkzGetPoint{H}
% drawing
\tkzDrawArc[delta=10](B,C)(E)
\tkzDrawPolygon(C,B,D)
\tkzDrawSegments(D,A B,A C,E)
% angles 
\tkzMarkAngles(C,B,D E,A,D) %this is to draw the arcs
\tkzLabelAngles[pos=1.5](C,B,D E,A,D){$\alpha$}
\tkzMarkRightAngle(B,H,C)
\tkzDrawPoints(A,...,E)
% Label only now
\tkzLabelPoints[below left](C,A)
\tkzLabelPoints[below right](D)
\tkzLabelPoints[above](B,E)
\end{tikzpicture}
\end{center}

Let's analyze the figure
\begin{enumerate}
  \item $CBD$ and $DBE$ are isosceles triangles; 
  
  \item $BC=BE$ and $(BD)$ is a bisector of the angle $CBE$;
  
  \item From this we deduce that the $CBD$ and $DBE$ angles are equal and have the same measure $\alpha$
   \[\widehat{BAC} +\widehat{ABC} + \widehat{BCA}=180^\circ \ \text{in the triangle}\ BAC \]
   \[3\alpha + \widehat{BCA}=180^\circ\  \text{in the triangle}\ CBD\]
   then 
     \[\alpha + 2\widehat{BCA}=180^\circ \] 
   or 
     \[\widehat{BCA}=90^\circ -\alpha/2 \] 
    
    \item  Finally   \[\widehat{CBD}=\alpha=36^\circ \] 
     the triangle $CBD$ is a "gold" triangle.
\end{enumerate}

\vspace*{24pt}
How construct a gold triangle or an angle of $36^\circ$?

\begin{enumerate}
  \item We place the fixed points $C$ and $D$. |\tkzDefPoint(0,0){C}| and |\tkzDefPoint(4,0){D}|;
  \item  We construct a square $CDef$ and we construct the midpoint $m$ of $[Cf]$;
  
  We can do all of this with a compass and a rule;
  \item Then we trace an arc with center $m$ through $e$. This arc cross the line $(Cf)$ at $n$;
  \item Now the two arcs with center $C$ and $D$ and radius $Cn$ define the point $B$.
\end{enumerate}


\begin{minipage}{.4\textwidth}
  \begin{tikzpicture}
  \tkzDefPoint(0,0){C}
  \tkzDefPoint(4,0){D}
  \tkzDefSquare(C,D)                     
  \tkzGetPoints{e}{f}
  \tkzDefMidPoint(C,f)                   
  \tkzGetPoint{m}
  \tkzInterLC(C,f)(m,e)                  
  \tkzGetSecondPoint{n}
  \tkzInterCC[with nodes](C,C,n)(D,C,n) 
  \tkzGetFirstPoint{B}
  \tkzDrawSegment[brown,dashed](f,n)
  \pgfinterruptboundingbox 
  \tkzDrawPolygon[brown,dashed](C,D,e,f)
  \tkzDrawArc[brown,dashed](m,e)(n)
  \tkzCompass[brown,dashed,delta=20](C,B)
  \tkzCompass[brown,dashed,delta=20](D,B)
  \endpgfinterruptboundingbox 
  \tkzDrawPoints(C,D,B)
  \tkzDrawPolygon(B,...,D)
  \end{tikzpicture}
\end{minipage}
\begin{minipage}{.6\textwidth}
  \begin{tkzexample}[code only,small]
  \begin{tikzpicture}
  \tkzDefPoint(0,0){C}
  \tkzDefPoint(4,0){D}
  \tkzDefSquare(C,D)                     
  \tkzGetPoints{e}{f}
  \tkzDefMidPoint(C,f)                   
  \tkzGetPoint{m}
  \tkzInterLC(C,f)(m,e)                  
  \tkzGetSecondPoint{n}
  \tkzInterCC[with nodes](C,C,n)(D,C,n) 
  \tkzGetFirstPoint{B}
  \tkzDrawSegment[brown,dashed](f,n)
  \pgfinterruptboundingbox 
  \tkzDrawPolygon[brown,dashed](C,D,e,f)
  \tkzDrawArc[brown,dashed](m,e)(n)
  \tkzCompass[brown,dashed,delta=20](C,B)
  \tkzCompass[brown,dashed,delta=20](D,B)
  \endpgfinterruptboundingbox 
  \tkzDrawPoints(C,D,B)
  \tkzDrawPolygon(B,...,D)
  \end{tikzpicture}
  \end{tkzexample}
\end{minipage}


After building the golden triangle $BCD$, we build the point $A$ by noticing that $BD=DA$. Then we get the point $E$ and finally the point $F$. This is done with already intersections of defined objects  (line and circle).
 

\begin{center}
  \begin{tikzpicture}
    \tkzDefPoint(0,0){C}
    \tkzDefPoint(4,0){D}
    \tkzDefSquare(C,D)                     
    \tkzGetPoints{e}{f}
    \tkzDefMidPoint(C,f)                   
    \tkzGetPoint{m}
    \tkzInterLC(C,f)(m,e)                  
    \tkzGetSecondPoint{n}
    \tkzInterCC[with nodes](C,C,n)(D,C,n) 
    \tkzGetFirstPoint{B}
    \tkzInterLC(C,D)(D,B) \tkzGetSecondPoint{A}
    \tkzInterLC(B,A)(B,D) \tkzGetSecondPoint{E}
    \tkzInterLL(B,D)(C,E) \tkzGetPoint{F}
    \tkzDrawPoints(C,D,B)
    \tkzDrawPolygon(B,...,D)  
    \tkzDrawPolygon(B,C,D)
    \tkzDrawSegments(D,A A,B C,E)
    \tkzDrawArc[delta=10](B,C)(E)
    \tkzMarkRightAngle[fill=blue!20](B,F,C)  
    \tkzFillAngles[fill=blue!10](C,B,D E,A,D)
    \tkzMarkAngles(C,B,D E,A,D)
    \tkzLabelAngles[pos=1.5](C,B,D E,A,D){$\alpha$} 
    \tkzLabelPoints[below](A,C,D,E)
    \tkzLabelPoints[above right](B,F)
    \tkzDrawPoints(A,...,F) 
  \end{tikzpicture} 
\end{center}



\begin{tkzexample}[code only,small]
  \begin{tikzpicture}
    \tkzDefPoint(0,0){C}
    \tkzDefPoint(4,0){D}
    \tkzDefSquare(C,D)                     
    \tkzGetPoints{e}{f}
    \tkzDefMidPoint(C,f)                   
    \tkzGetPoint{m}
    \tkzInterLC(C,f)(m,e)                  
    \tkzGetSecondPoint{n}
    \tkzInterCC[with nodes](C,C,n)(D,C,n) 
    \tkzGetFirstPoint{B}
    \tkzInterLC(C,D)(D,B) \tkzGetSecondPoint{A}
    \tkzInterLC(B,A)(B,D) \tkzGetSecondPoint{E}
    \tkzInterLL(B,D)(C,E) \tkzGetPoint{F}
    \tkzDrawPoints(C,D,B)
    \tkzDrawPolygon(B,...,D)  
    \tkzDrawPolygon(B,C,D)
    \tkzDrawSegments(D,A A,B C,E)
    \tkzDrawArc[delta=10](B,C)(E)
    \tkzDrawPoints(A,...,F) 
    \tkzMarkRightAngle[fill=blue!20](B,F,C)  
    \tkzFillAngles[fill=blue!10](C,B,D E,A,D)
    \tkzMarkAngles(C,B,D E,A,D)
    \tkzLabelAngles[pos=1.5](C,B,D E,A,D){$\alpha$} 
    \tkzLabelPoints[below](A,C,D,E)
    \tkzLabelPoints[above right](B,F)
  \end{tikzpicture} 
\end{tkzexample}

\subsubsection{Example Part II: two others methods gold and euclide triangle}

\tkzname{\tkznameofpack} knows how to define a "gold" or "euclide" triangle. We can define $BCD$ and $BCA$ like gold triangles.


  \begin{center}
    \begin{tkzexample}[code only,small]
      \begin{tikzpicture}
        \tkzDefPoint(0,0){C}
        \tkzDefPoint(4,0){D}
        \tkzDefTriangle[euclide](C,D)
        \tkzGetPoint{B}
        \tkzDefTriangle[euclide](B,C)
        \tkzGetPoint{A}
        \tkzInterLC(B,A)(B,D) \tkzGetSecondPoint{E}
        \tkzInterLL(B,D)(C,E) \tkzGetPoint{F}
        \tkzDrawPoints(C,D,B)
        \tkzDrawPolygon(B,...,D)  
        \tkzDrawPolygon(B,C,D)
        \tkzDrawSegments(D,A A,B C,E)
        \tkzDrawArc[delta=10](B,C)(E)
        \tkzDrawPoints(A,...,F) 
        \tkzMarkRightAngle[fill=blue!20](B,F,C)  
        \tkzFillAngles[fill=blue!10](C,B,D E,A,D)
        \tkzMarkAngles(C,B,D E,A,D)
        \tkzLabelAngles[pos=1.5](C,B,D E,A,D){$\alpha$} 
        \tkzLabelPoints[below](A,C,D,E)
        \tkzLabelPoints[above right](B,F)
      \end{tikzpicture} 
    \end{tkzexample}
  \end{center}

Here is a final method that uses rotations:  

\begin{center}
  \begin{tkzexample}[code only,small]
  \begin{tikzpicture} 
  \tkzDefPoint(0,0){C} % possible 
  % \tkzDefPoint[label=below:$C$](0,0){C} 
  % but don't do this
  \tkzDefPoint(2,6){B}
  % We get D and E with a rotation
  \tkzDefPointBy[rotation= center B angle 36](C) \tkzGetPoint{D} 
  \tkzDefPointBy[rotation= center B angle 72](C) \tkzGetPoint{E} 
  % To get A we use an intersection of lines
  \tkzInterLL(B,E)(C,D) \tkzGetPoint{A}
  \tkzInterLL(C,E)(B,D) \tkzGetPoint{H}
  % drawing
  \tkzDrawArc[delta=10](B,C)(E)
  \tkzDrawPolygon(C,B,D)
  \tkzDrawSegments(D,A B,A C,E)
  % angles 
  \tkzMarkAngles(C,B,D E,A,D) %this is to draw the arcs
  \tkzLabelAngles[pos=1.5](C,B,D E,A,D){$\alpha$}
  \tkzMarkRightAngle(B,H,C)
  \tkzDrawPoints(A,...,E)
  % Label only now
  \tkzLabelPoints[below left](C,A)
  \tkzLabelPoints[below right](D)
  \tkzLabelPoints[above](B,E)
  \end{tikzpicture}
  \end{tkzexample}
\end{center}


\subsubsection{Complete but minimal example}


A unit of length being chosen, the example shows how to obtain a segment of length $\sqrt{a}$ from a segment of length $a$, using a ruler and a compass.

$IB=a$, $AI=1$

\vspace{12pt}
\hypertarget{firstex}{}

\begin{tikzpicture}[scale=1,ra/.style={fill=gray!20}]
   % fixed points
   \tkzDefPoint(0,0){A}
   \tkzDefPoint(1,0){I}
   % calculation
   \tkzDefPointBy[homothety=center A ratio  10 ](I) \tkzGetPoint{B}  
   \tkzDefMidPoint(A,B)              \tkzGetPoint{M}
   \tkzDefPointWith[orthogonal](I,M) \tkzGetPoint{H}
   \tkzInterLC(I,H)(M,B)             \tkzGetSecondPoint{C}
   \tkzDrawSegment[style=orange](I,C)
   \tkzDrawArc(M,B)(A)
   \tkzDrawSegment[dim={$1$,-16pt,}](A,I)
   \tkzDrawSegment[dim={$a/2$,-10pt,}](I,M)
   \tkzDrawSegment[dim={$a/2$,-16pt,}](M,B)   
   \tkzMarkRightAngle[ra](A,I,C)
   \tkzDrawPoints(I,A,B,C,M)  
   \tkzLabelPoint[left](A){$A(0,0)$} 
   \tkzLabelPoints[above right](I,M)
   \tkzLabelPoints[above left](C)
   \tkzLabelPoint[right](B){$B(10,0)$}
   \tkzLabelSegment[right=4pt](I,C){$\sqrt{a^2}=a \ (a>0)$}
\end{tikzpicture}

\emph{Comments}
 
\begin{itemize}
\item The Preamble


 Let us first look at the preamble. If you need it, you have to load \tkzname{xcolor} before \tkzname{tkz-euclide}, that is, before \TIKZ. \TIKZ\ may cause problems with the active characters, but...
 provides a library in its latest version that's supposed to solve these problems \NameLib{babel}.
 
\begin{tkzltxexample}[]
\documentclass{standalone} % or another class
   % \usepackage{xcolor} % before tikz or tkz-euclide if necessary
\usepackage{tkz-euclide} % no need to load TikZ
   % \usetkzobj{all}  is no longer necessary 
   % \usetikzlibrary{babel}  if there are problems with the active characters
\end{tkzltxexample}

The following code consists of several parts:

   \item  Definition of fixed points: the first part includes the definitions of the points necessary for the construction, these are the fixed points. The macros \tkzcname{tkzInit} and \tkzcname{tkzClip} in most cases are not necessary.

\begin{tkzltxexample}[]
  \tkzDefPoint(0,0){A}
  \tkzDefPoint(1,0){I}
\end{tkzltxexample}
 
  \item The second part is dedicated to the creation of new points from the fixed points;
  a $B$ point is placed at $10$~cm    from $A$. The middle of $[AB]$ is defined by $M$ and then the orthogonal line to the $(AB)$ line is searched for at the $I$ point. Then we look for the intersection of this line with the semi-circle of center $M$ passing through $A$.  
  
\begin{tkzltxexample}[]
   \tkzDefPointBy[homothety=center A ratio  10 ](I)
      \tkzGetPoint{B}
   \tkzDefMidPoint(A,B)
      \tkzGetPoint{M}
   \tkzDefPointWith[orthogonal](I,M)
      \tkzGetPoint{H}
   \tkzInterLC(I,H)(M,A)
      \tkzGetSecondPoint{B}
 \end{tkzltxexample}  
     

 \item The third one includes the different drawings;
 \begin{tkzltxexample}[]
   \tkzDrawSegment[style=orange](I,H)
   \tkzDrawPoints(O,I,A,B,M)
   \tkzDrawArc(M,A)(O)
   \tkzDrawSegment[dim={$1$,-16pt,}](O,I)  
   \tkzDrawSegment[dim={$a/2$,-10pt,}](I,M)
   \tkzDrawSegment[dim={$a/2$,-16pt,}](M,A)
 \end{tkzltxexample}
 
\item  Marking: the fourth is devoted to marking;


\begin{tkzltxexample}[]
   \tkzMarkRightAngle(A,I,B)
 \end{tkzltxexample}
 
 \item Labelling: the latter only deals with the placement of labels.
\begin{tkzltxexample}[]
   \tkzLabelPoint[left](O){$A(0,0)$}
   \tkzLabelPoint[right](A){$B(10,0)$}
   \tkzLabelSegment[right=4pt](I,B){$\sqrt{a^2}=a \ (a>0)$}
\end{tkzltxexample}


\item The full code:


\begin{tkzexample}[code only]
  \begin{tikzpicture}[scale=1,ra/.style={fill=gray!20}]
     % fixed points
     \tkzDefPoint(0,0){A}
     \tkzDefPoint(1,0){I}
     % calculation
     \tkzDefPointBy[homothety=center A ratio  10 ](I) \tkzGetPoint{B}  
     \tkzDefMidPoint(A,B)              \tkzGetPoint{M}
     \tkzDefPointWith[orthogonal](I,M) \tkzGetPoint{H}
     \tkzInterLC(I,H)(M,B)             \tkzGetSecondPoint{C}
     \tkzDrawSegment[style=orange](I,C)
     \tkzDrawArc(M,B)(A)
     \tkzDrawSegment[dim={$1$,-16pt,}](A,I)
     \tkzDrawSegment[dim={$a/2$,-10pt,}](I,M)
     \tkzDrawSegment[dim={$a/2$,-16pt,}](M,B)   
     \tkzMarkRightAngle[ra](A,I,C)
     \tkzDrawPoints(I,A,B,C,M)  
     \tkzLabelPoint[left](A){$A(0,0)$} 
     \tkzLabelPoints[above right](I,M)
     \tkzLabelPoints[above left](C)
     \tkzLabelPoint[right](B){$B(10,0)$}
     \tkzLabelSegment[right=4pt](I,C){$\sqrt{a^2}=a \ (a>0)$}
  \end{tikzpicture}
\end{tkzexample}
\end{itemize}

\subsection{The Elements of tkz code}
In this paragraph, we start looking at the "rules" and "symbols" used to create a figure with \tkzname{\tkznameofpack}.

 The primitive objects are points. You can refer to a point at any time using the name given when defining it. (it is possible to assign a different name later on).

\medskip
In general, \tkzname{\tkznameofpack} macros have a name beginning with tkz. There are four main categories starting with:
|\tkzDef...| |\tkzDraw...| |\tkzMark...| and |\tkzLabel...|

Among the first category, |\tkzDefPoint| allows you to define fixed points. It will be studied in detail later. Here we will see in detail the macro  |\tkzDefTriangle|.

This macro makes it possible to associate to a pair of points a third point in order to define a certain triangle |\tkzDefTriangle(A,B)|. The obtained point is referenced |tkzPointResult| and it is possible to choose another reference with |\tkzGetPoint{C}| for example.
Parentheses are used to pass arguments. In |(A,B)| $A$ and $B$ are the points with which a third will be defined.

However, in |{C}| we use braces to retrieve the new point.
In order to choose a certain type of triangle among the following choices:
  |equilateral|, |half|, |pythagoras|, |school|, |golden or sublime|, |euclide|, |gold|, |cheops|...
 and |two angles| you just have to choose between hooks, for example:
 
|\tkzDefTriangle[euclide](A,B) \tkzGetPoint{C}|

\begin{minipage}{0.5\textwidth}
  \begin{tikzpicture}[scale=.75]
  \tkzDefPoints{0/0/A,8/0/B}
  \foreach \tr in {equilateral,half,pythagore,%
          school,golden,euclide, gold,cheops}
  {\tkzDefTriangle[\tr](A,B) \tkzGetPoint{C}
  \tkzDrawPoint(C)
  \tkzLabelPoint[right](C){\tr}
  \tkzDrawSegments(A,C C,B)}
  \tkzDrawPoints(A,B)
  \tkzDrawSegments(A,B)
  \end{tikzpicture}
\end{minipage}
\begin{minipage}{0.5\textwidth}
  \begin{tkzexample}[code only,small]
    \begin{tikzpicture}[scale=.75]
    \tkzDefPoints{0/0/A,8/0/B}
    \foreach \tr in {equilateral,half,pythagore,%
            school,golden,euclide, gold,cheops}
    {\tkzDefTriangle[\tr](A,B) \tkzGetPoint{C}
    \tkzDrawPoint(C)
    \tkzLabelPoint[right](C){\tr}
    \tkzDrawSegments(A,C C,B)}
    \tkzDrawPoints(A,B)
    \tkzDrawSegments(A,B)
    \end{tikzpicture}
  \end{tkzexample}

\end{minipage}


\subsection{Notations and conventions}

I deliberately chose to use the geometric French and personal  conventions  to describe the geometric objects represented. The objects defined and represented by \tkzname{\tkznameofpack} are points, lines and circles located in a plane. They are the primary objects of Euclidean geometry from which we will construct figures.

According to \tkzimp{Euclidian} these figures will only illustrate pure ideas produced by our brain.
Thus a point has no dimension and therefore no real existence. In the same way the line has no width and therefore no existence in the real world. The objects that we are going to consider are only representations of ideal mathematical objects. \tkzname{\tkznameofpack} will follow the steps of the ancient Greeks to obtain geometrical constructions using the ruler and the compass. 

Here are the notations that will be used:


\begin{itemize}
\item The points are represented geometrically either by a small disc or by the intersection of two lines (two straight lines, a straight line and a circle or two circles). In this case, the point is represented by a cross. 

\begin{tkzexample}[latex=6cm, small]     
  \begin{tikzpicture}       
    \tkzDefPoints{0/0/A,4/2/B}       
    \tkzDrawPoints(A,B)       
    \tkzLabelPoints(A,B)     
  \end{tikzpicture}    
\end{tkzexample}

or else

\begin{tkzexample}[latex=6cm, small]     
  \begin{tikzpicture}       
    \tkzSetUpPoint[shape=cross, color=red]       
    \tkzDefPoints{0/0/A,4/2/B}       
    \tkzDrawPoints(A,B)       
    \tkzLabelPoints(A,B)     
    \end{tikzpicture}    
    \end{tkzexample}  

The existence of a point being established, we can give it a label which will be a capital letter (with some exceptions) of the Latin alphabet such as $A$, $B$ or $C$. For example:
\begin{itemize}
\item $O$ is a center for a circle, a rotation, etc.;
\item $M$ defined a midpoint;
\item $H$ defined the foot of an altitude;
\item $P'$ is the image of $P$ by a transformation ;
\end{itemize}

It is important to note that the reference name of a point in the code may be different from the label to designate it in the text. So we can define a point A and give it as label $P$. In particular the style will be different, point A will be labeled $A$. 

\begin{tkzexample}[latex=6cm, small]     
  \begin{tikzpicture}       
    \tkzDefPoints{0/0/A}       
    \tkzDrawPoints(A)       
    \tkzLabelPoint(A){$P$}     
  \end{tikzpicture}    
\end{tkzexample}

Exceptions: some points such as the middle of the sides of a triangle share a characteristic, so it is normal that their names also share a common character. We will designate these points by $M_a$, $M_b$ and $M_c$ or $M_A$, $M_B$ and $M_C$.

In the code, these points will be referred to as: M\_A, M\_B and M\_C.

Another exception relates to intermediate construction points which will not be labelled. They will often be designated by a lowercase letter in the code.

\item The line segments are designated by two points representing their ends in square brackets: $[AB]$. 

\item The straight lines are in Euclidean geometry defined by two points so $A$ and $B$ define the straight line $(AB)$. We can also designate this stright line using the Greek alphabet and name it $(\delta)$ or $(\Delta)$. It is also possible to designate the straight line with lowercase letters such as $d$ and $d'$.

\item The semi-straight line is designated as follows $[AB)$.


\item Relation between the straight lines. Two perpendicular $(AB)$ and $(CD)$ lines will be written $(AB) \perp (CD)$ and if they are parallel we will write $(AB) \parallelslant (CD)$.

\item The lengths of the sides of triangle ABC are $AB$, $AC$ and $BC$. The numbers are also designated by a lowercase letter so we will write: $AB=c$, $AC=b$ and $BC=a$. The letter $a$ is also used to represent an angle, and $r$ is frequently used to represent a radius, $d$ a diameter, $l$ a length, $d$ a distance.

\item Polygons are designated afterwards by their vertices so $ABC$ is a triangle, $EFGH$ a quadrilateral.

\item Angles are generally measured in degrees (ex $60^\circ$) and in an equilateral $ABC$ triangle we will write $\widehat{ABC}=\widehat{B}=60^\circ$.

\item The arcs are designated by their extremities. For example if $A$ and $B$ are two points of the same circle then $\widearc{AB}$.


\item Circles are noted either $\mathcal{C}$ if there is no possible confusion or $\mathcal{C}$ $(O~;~A)$ for a circle with center $O$ and passing through the point $A$ or $\mathcal{C}$ $(O~;~1)$ for a circle with center O and radius 1 cm.

\item  Name of the particular lines of a triangle: I used the terms bisector, bisector out, mediator (sometimes called perpendicular bisectors), altitude, median and symmedian.

\item ($x_1$,$y_1$) coordinates of the point $A_1$, ($x_A$,$y_A$) coordinates of the point $A$.

\end{itemize}




\subsection{How to use the \tkzname{\tkznameofpack} package ?}
\subsubsection{Let's look at a classic example}
In order to show the right way, we will see how to build an equilateral triangle. Several possibilities are open to us, we are going to follow the steps of Euclid.

\begin{itemize}
\item   First of all you have to use a document class. The best choice to test your code is to create a single figure with the class \tkzname{standalone}\index{standalone}.
\begin{verbatim}  
\documentclass{standalone}
\end{verbatim}
\item Then load the \tkzname{\tkznameofpack} package:
\begin{verbatim}  
\usepackage{tkz-euclide}
\end{verbatim}

 You don't need to load \TIKZ\ because the \tkzname{\tkznameofpack} package works on top of TikZ and loads it.
 \item  {\color{red} \bomb \sout{|\BS usetkzobj{all}| }}
 With the new version 3.03 you don't need this line anymore. All objects are now loaded.
 \item Start the document and open a TikZ picture environment:
\begin{verbatim}
\begin{document}
\begin{tikzpicture}
\end{verbatim}

\item Now we define two fixed points:
\begin{verbatim}
\tkzDefPoint(O,O){A}
\tkzDefPoint(5,2){B}
\end{verbatim}

\item Two points define two circles, let's use these circles:

 circle with center $A$ through $B$ and circle with center $B$ through $A$. These two circles have two points in common.
\begin{verbatim}
\tkzInterCC(A,B)(B,A)
\end{verbatim}
We can get the points of intersection with
\begin{verbatim}
\tkzGetPoints{C}{D}
\end{verbatim}

\item All the necessary points are obtained, we can move on to the final steps including the plots.
\begin{verbatim}
\tkzDrawCircles[gray,dashed](A,B B,A)
\tkzDrawPolygon(A,B,C)% The triangle
\end{verbatim}
\item Draw all points $A$, $B$, $C$ and $D$:
\begin{verbatim}
\tkzDrawPoints(A,...,D)
\end{verbatim}

\item The final step, we print labels to the points and use options for positioning:\\
\begin{verbatim}
\tkzLabelSegments[swap](A,B){$c$}
\tkzLabelPoints(A,B,D)
\tkzLabelPoints[above](C)
\end{verbatim}
\item We finally close both environments
\begin{verbatim}
\end{tikzpicture}
\end{document}
\end{verbatim}

\item The complete code

\begin{tkzexample}[latex=8cm,small]
 \begin{tikzpicture}[scale=.5]
   % fixed points
  \tkzDefPoint(0,0){A}
  \tkzDefPoint(5,2){B}
  % calculus
  \tkzInterCC(A,B)(B,A)
  \tkzGetPoints{C}{D}
  % drawings
  \tkzDrawCircles[gray,dashed](A,B B,A)
  \tkzDrawPolygon(A,B,C)
  \tkzDrawPoints(A,...,D)
  % marking
  \tkzMarkSegments[mark=s||](A,B B,C C,A)
  % labelling
  \tkzLabelSegments[swap](A,B){$c$}
  \tkzLabelPoints(A,B,D)
  \tkzLabelPoints[above](C)
\end{tikzpicture}
\end{tkzexample}

 \end{itemize}

\subsubsection{\tkzname{Set, Calculate, Draw, Mark, Label}}
The title could have been: \texttt{Separation of Calculus and Drawings}

When a document is prepared using the \LATEX\ system, the source code of the document can be divided into two parts: the document body and the preamble.
Under this methodology,  publications can be structured, styled and typeset with minimal effort.
I propose a similar methodology for creating figures with \tkzname{\tkznameofpack}.

The first part defines the fixed points, the second part allows the creation of new points. These are the two main parts. All that is left to do is to draw, mark and label.




\endinput














\section{The Elements of tkz code}

To work with my package, you need to have notions of \LATEX\ as well as \TIKZ.

In this paragraph, we start looking at the "rules" and "symbols" used to create a figure with \tkzname{\tkznameofpack}.

\subsection{Objects and language}

 The primitive objects are points. You can refer to a point at any time using the name given when defining it. (it is possible to assign a different name later on).

To get new points you will use macros. \tkzname{\tkznameofpack} macros have a name beginning with tkz. There are four main categories starting with:
|\tkzDef...| |\tkzDraw...| |\tkzMark...| and |\tkzLabel...|. 
The used points are passed as parameters between parentheses while the created points are between braces.

The code of the figures is placed in an environment \tkzimp{tikzpicture}


Contrary to \TIKZ, you should not end a macro with  ";". We thus lose the important notion which is the \tkzimp{path}. However, it is possible to place some code between the macros \tkzname{\tkznameofpack}.
 

Among the first category, |\tkzDefPoint| allows you to define fixed points. It will be studied in detail later. Here we will see in detail the macro  |\tkzDefTriangle|.

This macro makes it possible to associate to a pair of points a third point in order to define a certain triangle |\tkzDefTriangle(A,B)|. The obtained point is referenced |tkzPointResult| and it is possible to choose another reference with |\tkzGetPoint{C}| for example.

|\tkzDefTriangle[euclid](A,B) \tkzGetPoint{C}|

Parentheses are used to pass arguments. In |(A,B)| $A$ and $B$ are the points with which a third will be defined. However, in |{C}| we use braces to retrieve the new point.

In order to choose a certain type of triangle among the following choices:
  |equilateral|,  |isosceles right|, |half|, |pythagoras|, |school|, |golden or sublime|, |euclid|, |gold|, |cheops|...
 and |two angles| you just have to choose between hooks, for example:
 
\begin{minipage}{0.5\textwidth}
  \begin{tikzpicture}[scale=.5]
  \tkzDefPoints{0/0/A,8/0/B}
  \foreach \tr in {golden, equilateral}
  {\tkzDefTriangle[\tr](A,B) \tkzGetPoint{C}
  \tkzDrawPoint(C)
  \tkzLabelPoint[right](C){\tr}
  \tkzDrawSegments(A,C C,B)}
  \tkzDrawPoints(A,B)
  \tkzDrawSegments(A,B)
   \tkzLabelPoints(A,B)
  \end{tikzpicture}
\end{minipage}
\begin{minipage}{0.5\textwidth}
  \begin{tkzexample}[code only,small]
    \begin{tikzpicture}[scale=.5]
    \tkzDefPoints{0/0/A,8/0/B}
   \foreach \tr in {golden, equilateral}
    {\tkzDefTriangle[\tr](A,B) \tkzGetPoint{C}
    \tkzDrawPoint(C)
    \tkzLabelPoint[right](C){\tr}
    \tkzDrawSegments(A,C C,B)}
    \tkzDrawPoints(A,B)
    \tkzDrawSegments(A,B)
    \tkzLabelPoints(A,B)
    \end{tikzpicture}
  \end{tkzexample}
\end{minipage}

\subsection{Notations and conventions}

I deliberately chose to use the geometric French and personal  conventions  to describe the geometric objects represented. The objects defined and represented by \tkzname{\tkznameofpack} are points, lines and circles located in a plane. They are the primary objects of Euclidean geometry from which we will construct figures.

According to \tkzimp{Euclid}, these figures will only illustrate pure ideas produced by our brain.
Thus a point has no dimension and therefore no real existence. In the same way the line has no width and therefore no existence in the real world. The objects that we are going to consider are only representations of ideal mathematical objects. \tkzname{\tkznameofpack} will follow the steps of the ancient Greeks to obtain geometrical constructions using the ruler and the compass. 

Here are the notations that will be used:

\begin{itemize}
\item The points are represented geometrically either by a small disc or by the intersection of two lines (two straight lines, a straight line and a circle or two circles). In this case, the point is represented by a cross. 

\begin{tkzexample}[latex=6cm, small]     
  \begin{tikzpicture}       
    \tkzDefPoints{0/0/A,4/2/B}       
    \tkzDrawPoints(A,B)       
    \tkzLabelPoints(A,B)     
  \end{tikzpicture}    
\end{tkzexample}

or else

\begin{tkzexample}[latex=6cm, small]     
  \begin{tikzpicture}       
    \tkzSetUpPoint[shape=cross, color=red]       
    \tkzDefPoints{0/0/A,4/2/B}       
    \tkzDrawPoints(A,B)       
    \tkzLabelPoints(A,B)     
    \end{tikzpicture}    
    \end{tkzexample}  

The existence of a point being established, we can give it a label which will be a capital letter (with some exceptions) of the Latin alphabet such as $A$, $B$ or $C$. For example:
\begin{itemize}
\item $O$ is a center for a circle, a rotation, etc.;
\item $M$ defined a midpoint;
\item $H$ defined the foot of an altitude;
\item $P'$ is the image of $P$ by a transformation ;
\end{itemize}

It is important to note that the reference name of a point in the code may be different from the label to designate it in the text. So we can define a point A and give it as label $P$. In particular the style will be different, point A will be labeled $A$. 

\begin{tkzexample}[latex=6cm, small]     
  \begin{tikzpicture}       
    \tkzDefPoint(0,0){A}       
    \tkzDrawPoints(A)       
    \tkzLabelPoint(A){$P$}     
  \end{tikzpicture}    
\end{tkzexample}

Exceptions: some points such as the middle of the sides of a triangle share a characteristic, so it is normal that their names also share a common character. We will designate these points by $M_a$, $M_b$ and $M_c$ or $M_A$, $M_B$ and $M_C$.

In the code, these points will be referred to as: M\_A, M\_B and M\_C.

Another exception relates to intermediate construction points which will not be labelled. They will often be designated by a lowercase letter in the code.

\item The line segments are designated by two points representing their ends in square brackets: $[AB]$. 

\item The straight lines are in Euclidean geometry defined by two points so $A$ and $B$ define the straight line $(AB)$. We can also designate this stright line using the Greek alphabet and name it $(\delta)$ or $(\Delta)$. It is also possible to designate the straight line with lowercase letters such as $d$ and $d'$.

\item The semi-straight line is designated as follows $[AB)$.

\item Relation between the straight lines. Two perpendicular $(AB)$ and $(CD)$ lines will be written $(AB) \perp (CD)$ and if they are parallel we will write $(AB) \parallelslant (CD)$.

\item The lengths of the sides of triangle ABC are $AB$, $AC$ and $BC$. The numbers are also designated by a lowercase letter so we will write: $AB=c$, $AC=b$ and $BC=a$. The letter $a$ is also used to represent an angle, and $r$ is frequently used to represent a radius, $d$ a diameter, $l$ a length, $d$ a distance.

\item Polygons are designated afterwards by their vertices so $ABC$ is a triangle, $EFGH$ a quadrilateral.

\item Angles are generally measured in degrees (ex $60^\circ$) and in an equilateral $ABC$ triangle we will write $\widehat{ABC}=\widehat{B}=60^\circ$.

\item The arcs are designated by their extremities. For example if $A$ and $B$ are two points of the same circle then $\widearc{AB}$.

\item Circles are noted either $\mathcal{C}$ if there is no possible confusion or $\mathcal{C}$ $(O~;~A)$ for a circle with center $O$ and passing through the point $A$ or $\mathcal{C}$ $(O~;~1)$ for a circle with center O and radius 1 cm.

\item  Name of the particular lines of a triangle: I used the terms bisector, bisector out, mediator (sometimes called perpendicular bisectors), altitude, median and symmedian.

\item ($x_1$,$y_1$) coordinates of the point $A_1$, ($x_A$,$y_A$) coordinates of the point $A$.
\end{itemize}

\subsection{\tkzname{Set, Calculate, Draw, Mark, Label}}
The title could have been: \texttt{Separation of Calculus and Drawings}

When a document is prepared using the \LATEX\ system, the source code of the document can be divided into two parts: the document body and the preamble.
Under this methodology,  publications can be structured, styled and typeset with minimal effort.
I propose a similar methodology for creating figures with \tkzname{\tkznameofpack}.

The first part defines the fixed points, the second part allows the creation of new points. \tkzname{Set and Calculate} are the two main parts. All that is left to do is to draw (or fill), mark and label. It is possible that \tkzname{\tkznameofpack} is insufficient for some of these latter actions but you can use \TIKZ

One last remark that I think is important, it is best to avoid introducing coordinates within a code as much as possible. I think that the coordinates should appear at the beginning of the code with the fixed points. Then the use of references is recommended. Most macros have the option \tkzname{nodes} or \tkzname{with nodes}.

I also think it's best to define the styles of the different objects from the beginning.
\endinput
\section{About this documentation and the examples}

It is obtained by compiling with "lualatex". I use a class \tkzname{doc.cls} based on \tkzname{scrartcl}.

Below the list of styles used in the docuimentation. To understand how to use the styles see the section \ref{custom}

|\tkzSetUpColors[background=white,text=black]  |

|\tkzSetUpCompass[color=orange, line width=.2pt,delta=10]|

|\tkzSetUpArc[color=gray,line width=.2pt]|

|\tkzSetUpPoint[size=2,color=teal]|

|\tkzSetUpLine[line width=.2pt,color=teal]|

|\tkzSetUpStyle[color=orange,line width=.2pt]{new}|

|\tikzset{every picture/.style={line width=.2pt}}|

|\tikzset{label angle style/.append style={color=teal,font=\footnotesize}}|


|\tikzset{label style/.append style={below,color=teal,font=\scriptsize}}|

Some examples use predefined styles like 


|\tikzset{new/.style={color=orange,line width=.2pt}}  |
\part{Setting}
\section{First step: fixed points}

The first step in a geometric construction is to define the fixed points from which the figure will be constructed.

The general idea is to avoid manipulating coordinates and to prefer to use the references of the points fixed in the first step or obtained using the tools provided by the package. Even if it's possible, I think it's a bad idea to work directly with coordinates. Preferable is to use named points.

\tkzname{\tkznameofpack} uses macros and vocabulary specific to geometric construction. It is of course possible to use the tools of \TIKZ\ but it seems more logical to me not to mix the different syntaxes.

A point in \tkzname{\tkznameofpack} is a particular "node" for \TIKZ. In the next section we will see how to define points using coordinates. The style of the points (color and shape) will not be discussed. You will find some indications in some examples; for more information you can read the following section \ref{custom}.


\section{Definition of a point : \tkzcname{tkzDefPoint} or \tkzcname{tkzDefPoints}}

 Points can be specified in any of the following ways:
\begin{itemize}
\item Cartesian coordinates;
\item Polar coordinates;
\item Named points;
\item Relative points.
\end{itemize}


A point is defined if it has a name linked to a unique pair of decimal numbers. 
 Let $(x,y)$ or $(a:d)$  i.e. ($x$ abscissa, $y$ ordinate) or  ($a$ angle: $d$ distance).
 This is possible because the plan has been provided with an orthonormed Cartesian coordinate system.   The working axes are  (ortho)normed with unity equal to $1$~cm.
 
 The Cartesian coordinate $(a,b)$ refers to the
 point $a$ centimeters in the $x$-direction and $b$ centimeters in the
 $y$-direction.

 A point in polar coordinates requires an angle $\alpha$, in degrees,
 and a distance  $d$ from the origin with a dimensional
 unit by default it's the \texttt{cm}.
 
 The \tkzNameMacro{tkzDefPoint} macro is used to define a point by assigning coordinates to it. This macro is based on 
 
 \tkzNameMacro{coordinate}, a macro of \TIKZ. It can use \TIKZ-specific options such as \tkzname{shift}. If calculations are required then the \tkzNamePack{xfp} package is chosen. We can use Cartesian or polar coordinates.
 
\begin{minipage}[t]{0.45\textwidth}
 Cartesian coordinates 
\begin{tkzexample}[code only,small]
\begin{tikzpicture}[scale=1] 
  \tkzInit[xmax=5,ymax=5]
  % necessary to limit
  % the size of the axes
  \tkzDrawX[>=latex]
  \tkzDrawY[>=latex]
  \tkzDefPoints{0/0/O,1/0/I,0/1/J}
  \tkzDefPoint(3,4){A} 
  \tkzDrawPoints(O,A) 
  \tkzLabelPoint[above](A){$A_1 (x_1,y_1)$} 
  \tkzShowPointCoord[xlabel=$x_1$,
                     ylabel=$y_1$](A) 
  \tkzLabelPoints(O,I)
  \tkzLabelPoints[left](J)
  \tkzDrawPoints[shape=cross](I,J) 
\end{tikzpicture}
\end{tkzexample}%
\end{minipage}
\begin{minipage}[t]{0.45\textwidth}
 Polar coordinates
\begin{tkzexample}[code only,small]
\begin{tikzpicture}[,scale=1]
  \tkzInit[xmax=5,ymax=5]
  \tkzDrawX[>=latex]
  \tkzDrawY[>=latex]  
  \tkzDefPoints{0/0/O,1/0/I,0/1/J}
  \tkzDefPoint(40:4){P} 
  \tkzDrawSegment[dim={$d$,
                 16pt,above=6pt}](O,P)
  \tkzDrawPoints(O,P) 
  \tkzMarkAngle[mark=none,->](I,O,P) 
  \tkzFillAngle[opacity=.5](I,O,P) 
  \tkzLabelAngle[pos=1.25](I,O,P){%
                              $\alpha$}  
  \tkzLabelPoint[right](P){$P  (\alpha : d )$} 
  \tkzDrawPoints[shape=cross](I,J) 
  \tkzLabelPoints(O,I)
  \tkzLabelPoints[left](J) 
\end{tikzpicture}
\end{tkzexample}
\end{minipage}%

\begin{minipage}[b]{0.45\textwidth}
\begin{tikzpicture}[scale=1] 
  \tkzInit[xmax=5,ymax=5]
  \tkzDrawX[>=latex]
  \tkzDrawY[>=latex]
  \tkzDefPoints{0/0/O,1/0/I,0/1/J}
  \tkzDefPoint(3,4){A} 
  \tkzDrawPoints(O,A) 
  \tkzLabelPoint[above](A){$A_1 (x_1,y_1)$} 
  \tkzShowPointCoord[xlabel=$x_1$,ylabel=$y_1$](A) 
  \tkzLabelPoints(O,I)
  \tkzLabelPoints[left](J)
  \tkzDrawPoints[shape=cross](I,J) 
\end{tikzpicture}
\end{minipage}
\begin{minipage}[b]{0.45\textwidth}
\begin{tikzpicture}[,scale=1]
  \tkzInit[xmax=5,ymax=5]
  \tkzDrawX[>=latex]
  \tkzDrawY[>=latex]  
  \tkzDefPoints{0/0/O,1/0/I,0/1/J}
  \tkzDefPoint(40:4){P} 
  \tkzDrawSegment[dim={$d$,
                 16pt,above=6pt}](O,P)
  \tkzDrawPoints(O,P) 
  \tkzMarkAngle[mark=none,->](I,O,P) 
  \tkzFillAngle[opacity=.5](I,O,P) 
  \tkzLabelAngle[pos=1.25](I,O,P){$\alpha$}  
  \tkzLabelPoint[right](P){$P  (\alpha : d )$} 
  \tkzDrawPoints[shape=cross](I,J) 
  \tkzLabelPoints(O,I)
  \tkzLabelPoints[left](J) 
\end{tikzpicture}
\end{minipage}%

\subsection{Defining a named point  \tkzcname{tkzDefPoint}}

\begin{NewMacroBox}{tkzDefPoint}{\oarg{local options}\parg{$x,y$}\marg{ref} or \parg{$\alpha$:$d$}\marg{ref}}%
\begin{tabular}{lll}%
arguments &  default & definition  \\ 
\midrule
\TAline{($x,y$)}{no default}{$x$ and $y$ are two dimensions, by default in cm.}
\TAline{($\alpha$:$d$)}{no default}{$\alpha$ is an angle in degrees, $d$ is a dimension}
\TAline{\{ref\}}{no default}{Reference assigned to the point: $A$, $T\_a$ ,$P1$ or $P_1$}
\bottomrule
\end{tabular}

\medskip
\emph{The obligatory arguments of this macro are two dimensions expressed with decimals, in the first case they are two measures of length, in the second case they are a measure of length and the measure of an angle in degrees. Do not confuse the reference with the name of a point. The reference is used by calculations, but frequently, the name is identical to the reference.}

\medskip
\begin{tabular}{lll}%
\toprule
options             & default & definition  \\ 
\midrule
\TOline{label} {no default} {allows you to place a label at a predefined distance}
\TOline{shift} {no default} {adds $(x,y)$ or $(\alpha:d)$ to all coordinates}
\end{tabular}
\end{NewMacroBox}

\subsubsection{Cartesian coordinates }
 
\begin{tkzexample}[latex=5cm,small]
  \begin{tikzpicture}
  \tkzInit[xmax=5,ymax=5] % limits the size of the axes
  \tkzDrawX[>=latex]
  \tkzDrawY[>=latex]  
  \tkzDefPoint(0,0){A}
  \tkzDefPoint(4,0){B}
  \tkzDefPoint(0,3){C} 
  \tkzDrawPolygon(A,B,C)
  \tkzDrawPoints(A,B,C)
  \end{tikzpicture}
\end{tkzexample}

\subsubsection{Calculations with \tkzNamePack{xfp}}

 \begin{tkzexample}[latex=7cm,small]
\begin{tikzpicture}[scale=1]
  \tkzInit[xmax=4,ymax=4]
  \tkzDrawX\tkzDrawY
  \tkzDefPoint(-1+2,sqrt(4)){O}
  \tkzDefPoint({3*ln(exp(1))},{exp(1)}){A}
  \tkzDefPoint({4*sin(pi/6)},{4*cos(pi/6)}){B}
  \tkzDrawPoints(O,B,A)
\end{tikzpicture}
\end{tkzexample}

\subsubsection{Polar coordinates }

\begin{tkzexample}[latex=7cm,small]
  \begin{tikzpicture}
  \foreach \an [count=\i] in {0,60,...,300}
   { \tkzDefPoint(\an:3){A_\i}}
  \tkzDrawPolygon(A_1,A_...,A_6)
  \tkzDrawPoints(A_1,A_...,A_6)
  \end{tikzpicture}
\end{tkzexample}

\subsubsection{Relative points}
First, we can use the \tkzNameEnv{scope} environment from \TIKZ.
In the following example, we have a way to define an equilateral triangle.

\begin{tkzexample}[latex=7cm,small]
\begin{tikzpicture}[scale=1]
 \begin{scope}[rotate=30]
  \tkzDefPoint(2,3){A}
  \begin{scope}[shift=(A)]
     \tkzDefPoint(90:5){B}
     \tkzDefPoint(30:5){C}
  \end{scope}
 \end{scope}
 \tkzDrawPolygon(A,B,C)
\tkzLabelPoints[above](B,C)
\tkzLabelPoints[below](A)
\tkzDrawPoints(A,B,C)
\end{tikzpicture}
\end{tkzexample}

\subsection{Point relative to another: \tkzcname{tkzDefShiftPoint}}
\begin{NewMacroBox}{tkzDefShiftPoint}{\oarg{Point}\parg{$x,y$}\marg{ref} or \parg{$\alpha$:$d$}\marg{ref}}%
\begin{tabular}{lll}%
arguments &  default & definition \\
\midrule
\TAline{($x,y$)}{no default}{$x$ and $y$ are two dimensions, by default in cm.}
\TAline{($\alpha$:$d$)}{no default}{$\alpha$ is an angle in degrees, $d$ is a dimension}
\TAline{\{ref\}}{no default}{Reference assigned to the point: $A$, $T\_a$ ,$P1$ or $P_1$}

\midrule
options &  default & definition \\

\midrule
\TOline{[pt]} {no default} {\tkzcname{tkzDefShiftPoint}[A](0:4)\{B\}}
\end{tabular}
\end{NewMacroBox}

\subsubsection{Isosceles triangle}
This macro allows you to place one point relative to another. This is equivalent to a translation. Here is how to construct an isosceles triangle with main vertex $A$ and angle at vertex of $30^{\circ} $.

\begin{tkzexample}[latex=7cm,small]
\begin{tikzpicture}[rotate=-30]
 \tkzDefPoint(2,3){A}
 \tkzDefShiftPoint[A](0:4){B}
 \tkzDefShiftPoint[A](30:4){C}
 \tkzDrawSegments(A,B B,C C,A)
 \tkzMarkSegments[mark=|](A,B A,C)
 \tkzDrawPoints(A,B,C)
 \tkzLabelPoints[right](B,C)
 \tkzLabelPoints[above left](A)
\end{tikzpicture}
\end{tkzexample}

\subsubsection{Equilateral triangle}
Let's see how to get an equilateral triangle (there is much simpler)

\begin{tkzexample}[latex=7cm,small]
\begin{tikzpicture}[scale=1]
 \tkzDefPoint(2,3){A}
 \tkzDefShiftPoint[A](30:3){B}
 \tkzDefShiftPoint[A](-30:3){C}
 \tkzDrawPolygon(A,B,C)
 \tkzDrawPoints(A,B,C)
 \tkzLabelPoints[right](B,C)
 \tkzLabelPoints[above left](A)
 \tkzMarkSegments[mark=|](A,B A,C B,C)
\end{tikzpicture}
\end{tkzexample}

\subsubsection{Parallelogram}
There's a simpler way
\begin{tkzexample}[latex=7cm,small]
\begin{tikzpicture}
 \tkzDefPoint(0,0){A}
 \tkzDefPoint(30:3){B}
 \tkzDefShiftPointCoord[B](10:2){C}
 \tkzDefShiftPointCoord[A](10:2){D}
 \tkzDrawPolygon(A,...,D)
 \tkzDrawPoints(A,...,D)
\end{tikzpicture}
\end{tkzexample}

\subsection{Definition of multiple points: \tkzcname{tkzDefPoints}}

\begin{NewMacroBox}{tkzDefPoints}{\oarg{local options}\marg{$x_1/y_1/n_1,x_2/y_2/r_2$, ...}}%
$x_i$ and $y_i$ are the coordinates of a referenced point $r_i$

\begin{tabular}{lll}%
\toprule
arguments &  default  & example  \\
\midrule
\TAline{$x_i/y_i/r_i$}{}{\tkzcname{tkzDefPoints\{0/0/O,2/2/A\}}}
\end{tabular}

\medskip
\begin{tabular}{lll}%
options             & default & definition   \\ 
\midrule
\TOline{shift} {no default} {Adds $(x,y)$ or $(\alpha:d)$ to all coordinates}
\end{tabular}
\end{NewMacroBox}

\subsection{Create a triangle}
\begin{tkzexample}[latex=6cm,small]
\begin{tikzpicture}[scale=.75]
 \tkzDefPoints{0/0/A,4/0/B,4/3/C}
 \tkzDrawPolygon(A,B,C)
 \tkzDrawPoints(A,B,C)
\end{tikzpicture}
\end{tkzexample}

\subsection{Create a square}
Note here the syntax for drawing the polygon.
\begin{tkzexample}[latex=6cm,small]
\begin{tikzpicture}[scale=1]
 \tkzDefPoints{0/0/A,2/0/B,2/2/C,0/2/D}
 \tkzDrawPolygon(A,...,D)
 \tkzDrawPoints(A,...,D)
\end{tikzpicture}
\end{tkzexample}

\endinput

\part{Calculating}
\section{Special points relating to a triangle}

\subsection{Triangle center: \tkzcname{tkzDefTriangleCenter}}

This macro allows you to define the center of a triangle.


\begin{NewMacroBox}{tkzDefTriangleCenter}{\oarg{local options}\parg{A,B,C}}%
\tkzHandBomb\ Be careful, the arguments are lists of three points. This macro is used in conjunction with \tkzcname{tkzGetPoint} to get the center you are looking for. You can use \tkzname{tkzPointResult} if it is not necessary to keep the results.

\medskip
\begin{tabular}{lll}%
\toprule
arguments & default & definition \\

\midrule
\TAline{(pt1,pt2,pt3)}{no default}{three points}
\midrule
options             & default & definition                         \\
\midrule
\TOline{ortho}  {circum}{intersection of the altitudes of a triangle}
\TOline{centroid} {circum}{centre of gravity. Intersection of the medians }
\TOline{circum}{circum}{circle center circumscribed}
\TOline{in}    {circum}{center of the circle inscribed in a triangle }
\TOline{ex}    {circum}{center of a circle exinscribed to a triangle }
\TOline{euler}{circum}{center of Euler's circle }
\TOline{symmedian} {circum}{Lemoine's point or symmedian centre or Grebe's point }
\TOline{spieker} {circum}{Spieker Circle Center}
\TOline{nagel}{circum}{Nagel Center}
\TOline{mittenpunkt} {circum}{also called the middlespoint}
\TOline{feuerbach}{circum}{Feuerbach Point}

\end{tabular}
\end{NewMacroBox}

\subsubsection{Option \tkzname{ortho} or \tkzname{orthic}}
 The intersection $H$ of the three altitudes  of a triangle is called the orthocenter.

\begin{tkzexample}[latex=5cm,small]
\begin{tikzpicture}
  \tkzDefPoint(0,0){A}
  \tkzDefPoint(5,1){B}
  \tkzDefPoint(1,4){C}
  \tkzClipPolygon(A,B,C)
  \tkzDefTriangleCenter[ortho](B,C,A)
    \tkzGetPoint{H}
  \tkzDefSpcTriangle[orthic,name=H](A,B,C){a,b,c}
  \tkzDrawPolygon[color=blue](A,B,C)
  \tkzDrawPoints(A,B,C,H)
  \tkzDrawLines[add=0 and 1](A,Ha B,Hb C,Hc)
  \tkzLabelPoint(H){$H$}
  \tkzAutoLabelPoints[center=H](A,B,C)
  \tkzMarkRightAngles(A,Ha,B B,Hb,C C,Hc,A)
\end{tikzpicture}
\end{tkzexample}

\subsubsection{Option \tkzname{centroid}}
\begin{tkzexample}[latex=5cm,small]
\begin{tikzpicture}[scale=.75]
  \tkzDefPoints{-1/1/A,5/1/B}
  \tkzDefEquilateral(A,B)
  \tkzGetPoint{C}
  \tkzDefTriangleCenter[centroid](A,B,C)
      \tkzGetPoint{G}
  \tkzDrawPolygon[color=brown](A,B,C)
  \tkzDrawPoints(A,B,C,G)
  \tkzDrawLines[add = 0 and 2/3](A,G B,G C,G)
\end{tikzpicture}
\end{tkzexample}

\subsubsection{Option \tkzname{circum}}
\begin{tkzexample}[latex=6cm,small]
 \begin{tikzpicture}
  \tkzDefPoints{0/1/A,3/2/B,1/4/C}
  \tkzDefTriangleCenter[circum](A,B,C)
  \tkzGetPoint{G}
  \tkzDrawPolygon[color=brown](A,B,C)
  \tkzDrawCircle(G,A)
  \tkzDrawPoints(A,B,C,G)
 \end{tikzpicture}
\end{tkzexample}

\subsubsection{Option \tkzname{in}}
In geometry, the incircle or inscribed circle of a triangle is the largest circle contained in the triangle; it touches (is tangent to) the three sides. The center of the incircle is a triangle center called the triangle's incenter.
The center of the incircle, called the incenter, can be found as the intersection of the three internal angle bisectors. The center of an excircle is the intersection of the internal bisector of one angle (at vertex $A$, for example) and the external bisectors of the other two. The center of this excircle is called the excenter relative to the vertex $A$, or the excenter of $A$. Because the internal bisector of an angle is perpendicular to its external bisector, it follows that the center of the incircle together with the three excircle centers form an orthocentric system.(\url{https://en.wikipedia.org/wiki/Incircle_and_excircles_of_a_triangle})
 
 \medskip
 We get the centre of the inscribed circle of the triangle. The result is of course in \tkzname{tkzPointResult}. We can retrieve it with \tkzcname{tkzGetPoint}.

\begin{tkzexample}[latex=6cm,small]
\begin{tikzpicture}
  \tkzDefPoints{0/1/A,3/2/B,1/4/C}
  \tkzDefTriangleCenter[in](A,B,C)\tkzGetPoint{I}
  \tkzDefPointBy[projection=onto A--C](I)
  \tkzGetPoint{Ib}
  \tkzDrawPolygon[color=blue](A,B,C)
  \tkzDrawPoints(A,B,C,I)
  \tkzDrawLines[add = 0 and 2/3](A,I B,I C,I)
  \tkzDrawCircle(I,Ib)
\end{tikzpicture}
\end{tkzexample}

\subsubsection{Option \tkzname{ex}}
An excircle or escribed circle of the triangle is a circle lying outside the triangle, tangent to one of its sides and tangent to the extensions of the other two. Every triangle has three distinct excircles, each tangent to one of the triangle's sides.
(\url{https://en.wikipedia.org/wiki/Incircle_and_excircles_of_a_triangle})


 We get the centre of an inscribed circle of the triangle. The result is of course in \tkzname{tkzPointResult}. We can retrieve it with \tkzcname{tkzGetPoint}.

\begin{tkzexample}[latex=8cm,small]
  \begin{tikzpicture}[scale=.5]
    \tkzDefPoints{0/1/A,3/2/B,1/4/C}
    \tkzDefTriangleCenter[ex](B,C,A)
    \tkzGetPoint{J_c}
    \tkzDefPointBy[projection=onto A--B](J_c)
    \tkzGetPoint{Tc}
    %or
    % \tkzDefCircle[ex](B,C,A)
    % \tkzGetFirstPoint{J_c}
    % \tkzGetSecondPoint{Tc}
    \tkzDrawPolygon[color=blue](A,B,C)
    \tkzDrawPoints(A,B,C,J_c)
    \tkzDrawCircle[red](J_c,Tc)
    \tkzDrawLines[add=1.5 and 0](A,C B,C)
    \tkzLabelPoints(J_c)
  \end{tikzpicture}
\end{tkzexample}

\subsubsection{Option \tkzname{euler}}
This macro allows to obtain the center of the circle of the nine points or euler's circle or Feuerbach's circle.
The nine-point circle, also called Euler's circle or the Feuerbach circle, is the circle that passes through the perpendicular feet $H_A$, $H_B$, and $H_C$ dropped from the vertices of any reference triangle $ABC$ on the sides opposite them. Euler showed in 1765 that it also passes through the midpoints $M_A$, $M_B$, $M_C$ of the sides of $ABC$. By Feuerbach's theorem, the nine-point circle also passes through the midpoints $E_A$, $E_B$, and $E_C$ of the segments that join the vertices and the orthocenter $H$. These points are commonly referred to as the Euler points. (\url{http://mathworld.wolfram.com/Nine-PointCircle.html})

\begin{tkzexample}[latex=7cm,small]
\begin{tikzpicture}[scale=1]
 \tkzDefPoints{0/0/A,6/0/B,0.8/4/C}
 \tkzDefSpcTriangle[medial,
     name=M](A,B,C){_A,_B,_C}
 \tkzDefTriangleCenter[euler](A,B,C)
    \tkzGetPoint{N} % I= N nine points
 \tkzDefTriangleCenter[ortho](A,B,C)
    \tkzGetPoint{H}
 \tkzDefMidPoint(A,H) \tkzGetPoint{E_A}
 \tkzDefMidPoint(C,H) \tkzGetPoint{E_C}
 \tkzDefMidPoint(B,H) \tkzGetPoint{E_B}
 \tkzDefSpcTriangle[ortho,name=H](A,B,C){_A,_B,_C}
 \tkzDrawPolygon[color=blue](A,B,C)
 \tkzDrawCircle(N,E_A)
 \tkzDrawSegments[blue](A,H_A B,H_B C,H_C)
 \tkzDrawPoints(A,B,C,N,H)
 \tkzDrawPoints[red](M_A,M_B,M_C)
 \tkzDrawPoints[blue]( H_A,H_B,H_C)
 \tkzDrawPoints[green](E_A,E_B,E_C)
 \tkzAutoLabelPoints[center=N,
  font=\scriptsize](A,B,C,%
   M_A,M_B,M_C,%
   H_A,H_B,H_C,%
   E_A,E_B,E_C)
 \tkzLabelPoints[font=\scriptsize](H,N)
 \tkzMarkSegments[mark=s|,size=3pt,
     color=blue,line width=1pt](B,E_B E_B,H)
\end{tikzpicture}
\end{tkzexample}


\subsubsection{Option \tkzname{symmedian}}

\begin{tkzexample}[latex=6cm,small]
\begin{tikzpicture}
  \tkzDefPoint(0,0){A}
  \tkzDefPoint(5,0){B}
  \tkzDefPoint(1,4){C}
  \tkzDefTriangleCenter[symmedian](A,B,C)\tkzGetPoint{K}
  \tkzDefTriangleCenter[median](A,B,C)\tkzGetPoint{G}
  \tkzDefTriangleCenter[in](A,B,C)\tkzGetPoint{I}
  \tkzDefSpcTriangle[centroid,name=M](A,B,C){a,b,c}
  \tkzDefSpcTriangle[incentral,name=I](A,B,C){a,b,c}
  \tkzDrawPolygon[color=blue](A,B,C)
  \tkzDrawLines[add = 0 and 2/3,blue](A,K B,K C,K)
  \tkzDrawSegments[red,dashed](A,Ma B,Mb C,Mc)
  \tkzDrawSegments[orange,dashed](A,Ia B,Ib C,Ic)
  \tkzDrawLine[add=2 and 2](G,I)
  \tkzDrawPoints(A,B,C,K,G,I)
\end{tikzpicture}
\end{tkzexample}


\subsubsection{Option \tkzname{nagel}}
Let $Ta$ be the point at which the excircle with center $Ja$ meets the side $BC$ of a triangle $ABC$, and define $Tb$ and $Tc$ similarly. Then the lines $ATa$, $BTb$, and $CTc$ concur in the Nagel point $Na$.
\href{http://mathworld.wolfram.com/NagelPoint.html}{Weisstein, Eric W. "Nagel point." From MathWorld--A Wolfram Web Resource. }


\begin{tkzexample}[latex=8cm,small]
  \begin{tikzpicture}[scale=.5]
  \tkzDefPoints{0/0/A,6/0/B,4/6/C}
  \tkzDefSpcTriangle[ex](A,B,C){Ja,Jb,Jc}
  \tkzDefSpcTriangle[extouch](A,B,C){Ta,Tb,Tc}
  \tkzDrawPoints(Ja,Jb,Jc,Ta,Tb,Tc)
  \tkzLabelPoints(Ja,Jb,Jc,Ta,Tb,Tc)
  \tkzDrawPolygon[blue](A,B,C)
  \tkzDefTriangleCenter[nagel](A,B,C) \tkzGetPoint{Na}
  \tkzDrawPoints[blue](B,C,A)
  \tkzDrawPoints[red](Na)
  \tkzLabelPoints[blue](B,C,A)
  \tkzLabelPoints[red](Na)
  \tkzDrawLines[add=0 and 1](A,Ta B,Tb C,Tc)
  \tkzShowBB\tkzClipBB
  \tkzDrawLines[add=1 and 1,dashed](A,B B,C C,A)
  \tkzDrawCircles[ex,gray](A,B,C C,A,B B,C,A)
  \tkzDrawSegments[dashed](Ja,Ta Jb,Tb Jc,Tc)
  \tkzMarkRightAngles[fill=gray!20](Ja,Ta,C
              Jb,Tb,A Jc,Tc,B)
  \end{tikzpicture}
\end{tkzexample}


\subsubsection{Option   \tkzname{mittenpunkt}} 
\begin{tkzexample}[latex=8cm,small]
\begin{tikzpicture}[scale=.5]
 \tkzDefPoints{0/0/A,6/0/B,4/6/C}
 \tkzDefSpcTriangle[centroid](A,B,C){Ma,Mb,Mc}
 \tkzDefSpcTriangle[ex](A,B,C){Ja,Jb,Jc}
 \tkzDefSpcTriangle[extouch](A,B,C){Ta,Tb,Tc}
 \tkzDefTriangleCenter[mittenpunkt](A,B,C) 
 \tkzGetPoint{Mi}
 \tkzDrawPoints(Ma,Mb,Mc,Ja,Jb,Jc)
 \tkzClipBB
 \tkzDrawPolygon[blue](A,B,C)
 \tkzDrawLines[add=0 and 1](Ja,Ma 
               Jb,Mb Jc,Mc)
 \tkzDrawLines[add=1 and 1](A,B A,C B,C)
 \tkzDrawCircles[gray](Ja,Ta Jb,Tb Jc,Tc)
 \tkzDrawPoints[blue](B,C,A)
 \tkzDrawPoints[red](Mi)
 \tkzLabelPoints[red](Mi)
 \tkzLabelPoints[left](Mb)
 \tkzLabelPoints(Ma,Mc,Jb,Jc)
 \tkzLabelPoints[above left](Ja,Jc)
 \tkzShowBB
\end{tikzpicture}
\end{tkzexample}
%<---------------------------------------------------------------------->
%<---------------------------------------------------------------------->
\section{Draw a point}
\subsubsection{Drawing points \tkzcname{tkzDrawPoint}} \hypertarget{tdrp}{}

\begin{NewMacroBox}{tkzDrawPoint}{\oarg{local options}\parg{name}}%
\begin{tabular}{lll}%
arguments &  default & definition                 \\
\midrule
\TAline{name of point} {no default}  {Only one point name is accepted}
\bottomrule
\end{tabular}

\medskip
The argument is required. The disc takes the color of the circle, but  lighter. It is possible to change everything. The point is a node and therefore it is invariant if the drawing is modified by scaling.

\medskip
\begin{tabular}{lll}%
\toprule
options             & default & definition \\
\midrule
\TOline{shape}  {circle}{Possible \tkzname{cross} or \tkzname{cross out}}
\TOline{size}  {6}{$6 \times$ \tkzcname{pgflinewidth}}
\TOline{color}  {black}{the default color can be changed }
\bottomrule
\end{tabular}

\medskip
{We can create other forms such as \tkzname{cross}}
\end{NewMacroBox}

\subsubsection{Example of point drawings}
Note that \tkzname{scale} does not affect the shape of the dots. Which is normal.  Most of the time, we are satisfied with a single point shape that we can define from the beginning, either with a macro or by modifying a configuration file.


\begin{tkzexample}[latex=5cm,small]
  \begin{tikzpicture}[scale=.5]
   \tkzDefPoint(1,3){A}
   \tkzDefPoint(4,1){B}
   \tkzDefPoint(0,0){O}
   \tkzDrawPoint[color=red](A)
   \tkzDrawPoint[fill=blue!20,draw=blue](B)
   \tkzDrawPoint[color=green](O)
  \end{tikzpicture}
\end{tkzexample}

It is possible to draw several points at once but this macro is a little slower than the previous one. Moreover, we have to make do with the same options for all the points.

\hypertarget{tdrps}{}
\begin{NewMacroBox}{tkzDrawPoints}{\oarg{local options}\parg{liste}}%
\begin{tabular}{lll}%
arguments &  default  & definition \\
\midrule
\TAline{points list}{no default}{example \tkzcname{tkzDrawPoints(A,B,C)}}
\bottomrule
\end{tabular}

\medskip
\begin{tabular}{lll}%
options             & default & definition \\
\midrule
\TOline{shape}  {circle}{Possible \tkzname{cross} or \tkzname{cross out}}
\TOline{size}  {6}{$6 \times$ \tkzcname{pgflinewidth}}
\TOline{color}  {black}{the default color can be changed }
\bottomrule
\end{tabular}

\medskip
\tkzHandBomb\ Beware of the final "s", an oversight leads to cascading errors if you try to draw multiple points. The options are the same as for the previous macro.
\end{NewMacroBox}

\subsubsection{First example}

\begin{tkzexample}[latex=7cm,small]
\begin{tikzpicture}
  \tkzDefPoint(1,3){A} 
  \tkzDefPoint(4,1){B} 
  \tkzDefPoint(0,0){C} 
  \tkzDrawPoints[size=6,color=red,
               fill=red!50](A,B,C)
\end{tikzpicture}
\end{tkzexample}

\subsubsection{Second example}

\begin{tkzexample}[latex=7cm,small]
\begin{tikzpicture}[scale=.5]
 \tkzDefPoint(2,3){A}  \tkzDefPoint(5,-1){B}
 \tkzDefPoint[label=below:$\mathcal{C}$,
               shift={(2,3)}](-30:5.5){E}
 \begin{scope}[shift=(A)]
    \tkzDefPoint(30:5){C}
 \end{scope}
 \tkzCalcLength[cm](A,B)\tkzGetLength{rAB}
 \tkzDrawCircle[R](A,\rAB cm)
 \tkzDrawSegment(A,B)
 \tkzDrawPoints(A,B,C)
 \tkzLabelPoints(B,C)
 \tkzLabelPoints[above](A)
\end{tikzpicture}
\end{tkzexample}

\section{Point on line or circle}
\subsection{Point on a line}

\begin{NewMacroBox}{tkzDefPointOnLine}{\oarg{local options}\parg{A,B}}%
\begin{tabular}{lll}%
arguments &  default & definition                 \\
\midrule
\TAline{pt1,pt2} {no default}  {Two points to define a line}
\bottomrule
\end{tabular}

\medskip
\begin{tabular}{lll}%
options       & default & definition \\
\midrule
\TOline{pos=nb}  {}{nb is a decimal  }
\end{tabular}
\end{NewMacroBox}

\subsubsection{Use of option \tkzname{pos}}
\begin{tkzexample}[latex=9cm,small]
  \begin{tikzpicture}
  \tkzDefPoints{0/0/A,4/0/B}
  \tkzDrawLine[red](A,B)
  \tkzDefPointOnLine[pos=1.2](A,B) 
  \tkzGetPoint{P}
  \tkzDefPointOnLine[pos=-0.2](A,B) 
  \tkzGetPoint{R}
  \tkzDefPointOnLine[pos=0.5](A,B) 
  \tkzGetPoint{S}
  \tkzDrawPoints(A,B,P)
  \tkzLabelPoints(A,B)
  \tkzLabelPoint[above](P){pos=$1.2$}
  \tkzLabelPoint[above](R){pos=$-.2$}
  \tkzLabelPoint[above](S){pos=$.5$}
  \tkzDrawPoints(A,B,P,R,S)
  \tkzLabelPoints(A,B)
  \end{tikzpicture}
\end{tkzexample}

\subsection{Point on a circle}

\begin{NewMacroBox}{tkzDefPointOnCircle}{\oarg{local options}}%
\begin{tabular}{lll}%
options   & default & definition \\
\midrule
\TOline{angle}  {0}{angle formed with the abscissa axis}
\TOline{center}  {|tkzPointResult|}{circle center required}
\TOline{radius}  {|\BS tkzLengthResult|}{radius circle}
\end{tabular}
\end{NewMacroBox}

\begin{tkzexample}[latex=7cm,small]
\begin{tikzpicture}
 \tkzDefPoints{0/0/A,4/0/B,0.8/3/C}  
 \tkzDefPointOnCircle[angle=90,center=B,radius=1 cm]
 \tkzGetPoint{I}    
 \tkzDefCircle[circum](A,B,C)
 \tkzGetPoint{G} \tkzGetLength{rG}
 \tkzDefPointOnCircle[angle=30,center=G,radius=\rG pt]
 \tkzGetPoint{J}
 \tkzDrawCircle[R,teal](B,1cm) 
 \tkzDrawPoint[teal](I)
 \tkzDrawPoints(A,B,C)
 \tkzDrawCircle(G,J)
 \tkzDrawPoints(G,J)
 \tkzDrawPoint[red](J)
 \tkzLabelPoints(G,J)
\end{tikzpicture}
\end{tkzexample}
\endinput
\section{Definition of points by transformation; \tkzcname{tkzDefPointBy} }
These transformations are:

\begin{itemize}
   \item translation;
   \item homothety;
   \item orthogonal reflection or symmetry;
   \item central symmetry;
   \item orthogonal projection;
   \item rotation (degrees or radians);
   \item inversion with respect to a circle.
\end{itemize}

The choice of transformations is made through the options. There are two macros, one for the transformation of a single point \tkzcname{tkzDefPointBy} and the other for the transformation of a list of points \tkzcname{tkzDefPointsBy}. By default the image of $A$ is $A'$. For example, we'll write:
\begin{tkzltxexample}[]
\tkzDefPointBy[translation= from A to A'](B) 
\end{tkzltxexample}
The result is in \tkzname{tkzPointResult}
\medskip
\begin{NewMacroBox}{tkzDefPointBy}{\oarg{local options}\parg{pt}}%
The argument is a simple existing point and its image is stored in \tkzname{tkzPointResult}. If you want to keep this point then the macro \tkzcname{tkzGetPoint\{M\}} allows you to assign the name \tkzname{M} to the point.

\begin{tabular}{lll}%
\toprule
arguments &  definition & examples               \\ 
\midrule
\TAline{pt}   {existing point name}   {$(A)$}
\bottomrule
\end{tabular}

\begin{tabular}{lll}%
options     &     & examples                         \\ 
\midrule
\TOline{translation}{= from \#1 to \#2}{[translation=from A to B](E)}
\TOline{homothety}  {= center \#1 ratio \#2}{[homothety=center A ratio .5](E)}
\TOline{reflection} {= over \#1--\#2}{[reflection=over A--B](E)}
\TOline{symmetry }  {= center \#1}{[symmetry=center A](E)}
\TOline{projection }{= onto \#1--\#2}{[projection=onto A--B](E)}
\TOline{rotation }  {= center \#1 angle \#2}{[rotation=center O angle 30](E)}
\TOline{rotation in rad}{= center \#1 angle \#2}{[rotation in rad=center O angle pi/3](E)} 
\TOline{inversion}{= center \#1 through \#2}{[inversion =center O through A](E)} 
\bottomrule
\end{tabular}

The image is only defined and not drawn.
\end{NewMacroBox} 

\subsection{Examples of transformations}
\subsubsection{Example of translation} 

\subsection{Example of translation}
\begin{tkzexample}[latex=7cm,small]
\begin{tikzpicture}[>=latex] 
 \tkzDefPoint(0,0){A}  \tkzDefPoint(3,1){B}
 \tkzDefPoint(3,0){C}  
 \tkzDefPointBy[translation= from B to A](C) 
 \tkzGetPoint{D} 
 \tkzDrawPoints[teal](A,B,C,D)
 \tkzLabelPoints[color=teal](A,B,C,D) 
 \tkzDrawSegments[orange,->](A,B D,C) 
\end{tikzpicture} 
\end{tkzexample}

\subsubsection{Example of reflection (orthogonal symmetry)} 

\begin{tkzexample}[vbox,small]
\begin{tikzpicture}[scale=1]
 \tkzDefPoints{1.5/-1.5/C,-4.5/2/D}    
 \tkzDefPoint(-4,-2){O} 
 \tkzDefPoint(-2,-2){A}
 \foreach \i in {0,1,...,4}{%
 \pgfmathparse{0+\i * 72}
 \tkzDefPointBy[rotation=%
 center O angle \pgfmathresult](A)
  \tkzGetPoint{A\i} 
 \tkzDefPointBy[reflection = over C--D](A\i)
  \tkzGetPoint{A\i'}}
 \tkzDrawPolygon(A0, A2, A4, A1, A3)    
 \tkzDrawPolygon(A0', A2', A4', A1', A3')
 \tkzDrawLine[add= .5 and .5](C,D)
\end{tikzpicture}
\end{tkzexample}


\subsubsection{Example of \tkzname{homothety} and \tkzname{projection}}

\begin{tkzexample}[vbox,small] 
\begin{tikzpicture}[scale=1.2] 
  \tkzDefPoint(0,1){A}   \tkzDefPoint(5,3){B}   \tkzDefPoint(3,4){C} 
  \tkzDefLine[bisector](B,A,C)                     \tkzGetPoint{a} 
  \tkzDrawLine[add=0 and 0,color=magenta!50 ](A,a) 
  \tkzDefPointBy[homothety=center A ratio .5](a)   \tkzGetPoint{a'} 
  \tkzDefPointBy[projection = onto A--B](a')       \tkzGetPoint{k'}  
  \tkzDefPointBy[projection = onto A--B](a)       \tkzGetPoint{k} 
  \tkzDrawLines[add= 0 and .3](A,k A,C)   
  \tkzDrawSegments[blue](a',k' a,k) 
  \tkzDrawPoints(a,a',k,k',A)
  \tkzDrawCircles(a',k' a,k)   
  \tkzLabelPoints(a,a',k,A)
\end{tikzpicture}
\end{tkzexample}  


\subsubsection{Example of projection}
\begin{tkzexample}[vbox,small] 
\begin{tikzpicture}[scale=1.5]  
 \tkzDefPoint(0,0){A}
 \tkzDefPoint(0,4){B}
 \tkzDefTriangle[pythagore](B,A) \tkzGetPoint{C}
 \tkzDefLine[bisector](B,C,A) \tkzGetPoint{c}
 \tkzInterLL(C,c)(A,B)        \tkzGetPoint{D}
 \tkzDefPointBy[projection=onto B--C](D) \tkzGetPoint{G}
 \tkzInterLC(C,D)(D,A) \tkzGetPoints{E}{F}
 \tkzDrawPolygon[teal](A,B,C)
 \tkzDrawSegment(C,D)
 \tkzDrawCircle(D,A)
 \tkzDrawSegment[orange](D,G)
 \tkzMarkRightAngle[fill=orange!20](D,G,B)
 \tkzDrawPoints(A,C,F) \tkzLabelPoints(A,C,F)
 \tkzDrawPoints(B,D,E,G)   
 \tkzLabelPoints[above right](B,D,E,G)
 \end{tikzpicture}
 \end{tkzexample} 

\subsubsection{Example of symmetry}
\begin{tkzexample}[vbox,small] 
\begin{tikzpicture}[scale=1]
  \tkzDefPoint(0,0){O}
  \tkzDefPoint(2,-1){A}
  \tkzDefPoint(2,2){B}
  \tkzDefPointsBy[symmetry=center O](B,A){}
  \tkzDrawLine(A,A')
  \tkzDrawLine(B,B')
  \tkzMarkAngle[mark=s,arc=lll,
      size=2 cm,mkcolor=red](A,O,B) 
  \tkzLabelAngle[pos=1,circle,draw,
     fill=blue!10](A,O,B){$60^{\circ}$} 
  \tkzDrawPoints(A,B,O,A',B') 
  \tkzLabelPoints(A,B,O,A',B') 
\end{tikzpicture}  
\end{tkzexample}

\subsubsection{Example of rotation} 
\begin{tkzexample}[latex=7cm,small] 
\begin{tikzpicture}[scale=0.5] 
 \tkzDefPoint(0,0){A} 
 \tkzDefPoint(5,0){B}
 \tkzDrawSegment(A,B)
 \tkzDefPointBy[rotation=center A angle 60](B) 
 \tkzGetPoint{C} 
 \tkzDefPointBy[symmetry=center C](A) 
 \tkzGetPoint{D} 
 \tkzDrawSegment(A,tkzPointResult) 
 \tkzDrawLine(B,D)
 \tkzDrawArc[orange,delta=10](A,B)(C) 
 \tkzDrawArc[orange,delta=10](B,C)(A)
 \tkzDrawArc[orange,delta=10](C,D)(D)  
 \tkzMarkRightAngle(D,B,A)  
\end{tikzpicture}  
\end{tkzexample}  

\subsubsection{Example of rotation in radian} 
\begin{tkzexample}[latex=6cm,small]
\begin{tikzpicture}
  \tkzDefPoint["$A$" left](1,5){A}
  \tkzDefPoint["$B$" right](5,2){B}
  \tkzDefPointBy[rotation in rad= center A angle pi/3](B)
  \tkzGetPoint{C}  
  \tkzDrawSegment(A,B)
  \tkzDrawPoints(A,B,C) 
  \tkzCompass[color=red](A,C)
  \tkzCompass[color=red](B,C) 
  \tkzLabelPoints(C)
\end{tikzpicture}
\end{tkzexample} 

\subsubsection{Inversion of points}
\begin{tkzexample}[latex=8cm,small]  
\begin{tikzpicture}[scale=1.5]
  \tkzDefPoint(0,0){O}
  \tkzDefPoint(1,0){A}
  \tkzDefPoint(-1.5,-1.5){z1}
  \tkzDefPoint(0.35,0){z2} 
  \tkzDefPointBy[inversion =%
      center O through A](z1)
  \tkzGetPoint{Z1} 
  \tkzDefPointBy[inversion =%
      center O through A](z2)
  \tkzGetPoint{Z2}
  \tkzDrawCircle(O,A)  
  \tkzDrawPoints[color=black,
      fill=red,size=4](Z1,Z2)    
  \tkzDrawSegments(z1,Z1 z2,Z2)
  \tkzDrawPoints[color=black,
      fill=red,size=4](O,z1,z2)
  \tkzLabelPoints(O,A,z1,z2,Z1,Z2)  
\end{tikzpicture}
\end{tkzexample} 


\subsubsection{Point Inversion: Orthogonal Circles} 
\begin{tkzexample}[latex=8cm,small]
\begin{tikzpicture}[scale=1.5]
  \tkzDefPoint(0,0){O}
  \tkzDefPoint(1,0){A}
  \tkzDrawCircle(O,A) 
  \tkzDefPoint(0.5,-0.25){z1}
  \tkzDefPoint(-0.5,-0.5){z2}
  \tkzDefPointBy[inversion = %
     center O through A](z1)
  \tkzGetPoint{Z1} 
  \tkzCircumCenter(z1,z2,Z1)
  \tkzGetPoint{c}
  \tkzDrawCircle(c,Z1)
  \tkzDrawPoints[color=black,
     fill=red,size=4](O,z1,z2,Z1,O,A) 
\end{tikzpicture}
\end{tkzexample}

\subsection{Transformation of multiple points; \tkzcname{tkzDefPointsBy} }
Variant of the previous macro for defining multiple images.
You must give the names of the images as arguments, or indicate that the names of the images are formed from the names of the antecedents, leaving the argument empty. 

\begin{tkzltxexample}[]
\tkzDefPointsBy[translation= from A to A'](B,C){}
\end{tkzltxexample}
The images are $B'$ and $C'$.

\begin{tkzltxexample}[]
\tkzDefPointsBy[translation= from A to A'](B,C){D,E} 
\end{tkzltxexample}
The images are $D$ and $E$.

\begin{tkzltxexample}[]
\tkzDefPointsBy[translation= from A to A'](B)
\end{tkzltxexample}
The image is $B'$.
\begin{NewMacroBox}{tkzDefPointsBy}{\oarg{local options}\parg{list of points}\marg{list of points}}%
\begin{tabular}{lll}%
arguments &  examples  &                  \\ 
\midrule
\TAline{\parg{list of points}\marg{list of pts}}{(A,B)\{E,F\}}{$E$ is the image of $A$ and $F$ is the image of $B$.}   \\
\bottomrule
\end{tabular}

\medskip
If the list of images is empty then the name of the image is the name of the antecedent to which " ' " is added.

\medskip
\begin{tabular}{lll}%
\toprule
options     &     & examples                         \\ 
\midrule
\TOline{translation = from \#1 to \#2}{}{[translation=from A to B](E)\{\}}
\TOline{homothety = center \#1 ratio \#2}{}{[homothety=center A ratio .5](E)\{F\}}
\TOline{reflection = over \#1--\#2}{}{[reflection=over A--B](E)\{F\}}
\TOline{symmetry = center \#1}{}{[symmetry=center A](E)\{F\}}
\TOline{projection = onto \#1--\#2}{}{[projection=onto A--B](E)\{F\}}
\TOline{rotation = center \#1 angle \#2}{}{[rotation=center  angle 30](E)\{F\}}
\TOline{rotation in rad = center \#1 angle \#2}{}{for instance angle pi/3}
\bottomrule
\end{tabular}

\medskip
The points are only defined and not drawn.
\end{NewMacroBox}

\subsubsection{Example of translation}
\begin{tkzexample}[latex=7cm,small]
\begin{tikzpicture}[>=latex] 
 \tkzDefPoint(0,0){A}  \tkzDefPoint(3,1){A'}
 \tkzDefPoint(3,0){B}  \tkzDefPoint(1,2){C} 
 \tkzDefPointsBy[translation= from A to A'](B,C){} 
 \tkzDrawPolygon[color=blue](A,B,C)
 \tkzDrawPolygon[color=red](A',B',C')
 \tkzDrawPoints[color=blue](A,B,C)
 \tkzDrawPoints[color=red](A',B',C') 
 \tkzLabelPoints(A,B,A',B')  
 \tkzLabelPoints[above](C,C')
 \tkzDrawSegments[color = gray,->,
              style=dashed](A,A' B,B' C,C') 
\end{tikzpicture}
\end{tkzexample}

\endinput

\section{Defining points using a vector}

\subsection{\tkzcname{tkzDefPointWith}}
There are several possibilities to create points that meet certain vector conditions.
This can be done with 

\tkzcname{tkzDefPointWith}. The general principle is as follows, two points are passed as arguments, i.e. a vector. The different options allow to obtain a new point forming with the first point (with some exceptions) a collinear vector or a vector orthogonal to the first vector. Then the length is either proportional to that of the first one, or proportional to the unit. Since this point is only used temporarily, it does not have to be named immediately. The result is in \tkzname{tkzPointResult}. The macro \tkzNameMacro{tkzGetPoint} allows you to retrieve the point and name it differently.

 There are options to define the distance between the given point and the obtained point.
In the general case this distance is the distance between the 2 points given as arguments if the option is of the "normed" type then the distance between the given point and the obtained point is 1 cm. Then the $K$ option allows to obtain multiples.

\begin{NewMacroBox}{tkzDefPointWith}{\parg{pt1,pt2}}%
 It is in fact the definition of a point meeting vectorial conditions.

\medskip
  
\begin{tabular}{lll}%
\toprule
arguments             & definition & explanation                         \\ 
\midrule
\TAline{(pt1,pt2)} {point couple}{the result is a point in \tkzname{tkzPointResult} } \\

\bottomrule
\end{tabular}

\medskip
In what follows, it is assumed that the point is recovered by \tkzNameMacro{tkzGetPoint\{C\}}

\begin{tabular}{lll}%
\toprule
options             & example & explanation                         \\ 
\midrule
\TOline{orthogonal}{[orthogonal](A,B)}{$AC=AB$ and $\overrightarrow{AC} \perp \overrightarrow{AB}$}
\TOline{orthogonal normed}{[orthogonal normed](A,B)}{$AC=1$ and $\overrightarrow{AC} \perp \overrightarrow{AB}$} 
\TOline{linear}{[linear](A,B)}{$\overrightarrow{AC}=K \times \overrightarrow{AB}$}
\TOline{linear normed}{[linear normed](A,B)}{$AC=K$ and $\overrightarrow{AC}=k\times \overrightarrow{AB}$ }  
\TOline{colinear= at \#1}{[colinear= at C](A,B)}{$\overrightarrow{CD}= \overrightarrow{AB}$ }
\TOline{colinear normed= at \#1}{[colinear normed= at C](A,B)}{$\overrightarrow{CD}= \overrightarrow{AB}$ }
\TOline{K}{[linear](A,B),K=2}{$\overrightarrow{AC}=2\times \overrightarrow{AB}$}
\end{tabular}
\end{NewMacroBox}

\subsubsection{Option \tkzname{colinear at}, simple example}
 $(\overrightarrow{AB}=\overrightarrow{CD})$
\begin{tkzexample}[latex=6cm,small]
\begin{tikzpicture}[scale=1.2,
   vect/.style={->,shorten >=1pt,>=latex'}]
  \tkzDefPoint(2,3){A}   \tkzDefPoint(4,2){B}
  \tkzDefPoint(0,1){C}
  \tkzDefPointWith[colinear=at C](A,B)
  \tkzGetPoint{D}
  \tkzDrawPoints[new](A,B,C,D)
  \tkzLabelPoints[above right=3pt](A,B,C,D)
  \tkzDrawSegments[vect](A,B C,D)
\end{tikzpicture}
\end{tkzexample}

\subsubsection{Option \tkzname{colinear at}, complex example}
\begin{tkzexample}[vbox,small]
\begin{tikzpicture}[scale=.75]
\tkzDefPoints{0/0/B,3.6/0/C,1.5/4/A}
\tkzDefSpcTriangle[ortho](A,B,C){Ha,Hb,Hc}
\tkzDefTriangleCenter[ortho](A,B,C) \tkzGetPoint{H}
\tkzDefSquare(A,C) \tkzGetPoints{R}{S}
\tkzDefSquare(B,A) \tkzGetPoints{M}{N}
\tkzDefSquare(C,B) \tkzGetPoints{P}{Q}
\tkzDefPointWith[colinear= at M](A,S) \tkzGetPoint{A'}
\tkzDefPointWith[colinear= at P](B,N) \tkzGetPoint{B'}
\tkzDefPointWith[colinear= at Q](C,R) \tkzGetPoint{C'}
\tkzDefPointBy[projection=onto P--Q](Ha) \tkzGetPoint{Pa}
\tkzDrawPolygon[teal,thick](A,C,R,S)\tkzDrawPolygon[teal,thick](A,B,N,M)
\tkzDrawPolygon[teal,thick](C,B,P,Q)
\tkzDrawPoints[teal,size=2](A,B,C,Ha,Hb,Hc,A',B',C')
\tkzDrawSegments[ultra thin,red](M,A' A',S P,B' B',N Q,C' C',R B,S C,M C,N B,R A,P A,Q)
\tkzDrawSegments[ultra thin,teal, dashed](A,Ha B,Hb C,Hc)
\tkzDefPointBy[rotation=center A angle 90](S) \tkzGetPoint{S'}
\tkzDrawSegments[ultra thin,teal,dashed](B,S' A,S' A,A' M,S' B',Q P,C' M,S Ha,Pa)
\tkzDrawArc(A,S)(S')
\end{tikzpicture}
\end{tkzexample}

\subsubsection{Option \tkzname{colinear at}}
How to use $K$
\begin{tkzexample}[latex=7cm,small]
\begin{tikzpicture}[vect/.style={->,
               shorten >=1pt,>=latex'}]
  \tkzDefPoints{0/0/A,5/0/B,1/2/C}
  \tkzDefPointWith[colinear=at C](A,B)
  \tkzGetPoint{G}
  \tkzDefPointWith[colinear=at C, K=0.5](A,B)
  \tkzGetPoint{H}
  \tkzLabelPoints(A,B,C,G,H)
  \tkzDrawPoints(A,B,C,G,H)
  \tkzDrawSegments[vect](A,B C,H)
\end{tikzpicture}
\end{tkzexample}

\subsubsection{Option \tkzname{colinear at} } 
With $K=\frac{\sqrt{2}}{2}$

\begin{tkzexample}[latex=6cm,small]
\begin{tikzpicture}[vect/.style={->,
            shorten >=1pt,>=latex'}]
 \tkzDefPoints{1/1/A,4/2/B,2/2/C}
 \tkzDefPointWith[colinear=at C,K=sqrt(2)/2](A,B)
 \tkzGetPoint{D}
 \tkzDrawPoints[color=red](A,B,C,D)
 \tkzDrawSegments[vect](A,B C,D)
\end{tikzpicture}
\end{tkzexample}

\subsubsection{Option \tkzname{orthogonal}}
AB=AC since $K=1$.
\begin{tkzexample}[latex=6cm,small]
\begin{tikzpicture}[scale=1.2,
  vect/.style={->,shorten >=1pt,>=latex'}]
  \tkzDefPoints{2/3/A,4/2/B}
   \tkzDefPointWith[orthogonal,K=1](A,B)
     \tkzGetPoint{C}
   \tkzDrawPoints[color=red](A,B,C)
   \tkzLabelPoints[right=3pt](B,C)
   \tkzLabelPoints[below=3pt](A)
   \tkzDrawSegments[vect](A,B A,C)
   \tkzMarkRightAngle(B,A,C)
\end{tikzpicture}
\end{tkzexample}



\subsubsection{Option \tkzname{orthogonal}}
 With $K=-1$
OK=OI since $\lvert K \rvert=1$ then OI=OJ=OK.

\begin{tkzexample}[latex=7cm,small]
\begin{tikzpicture}[scale=.75]
  \tkzDefPoints{1/2/O,2/5/I}
  \tkzDefPointWith[orthogonal](O,I)
  \tkzGetPoint{J}
  \tkzDefPointWith[orthogonal,K=-1](O,I)
  \tkzGetPoint{K}
  \tkzDrawSegment(O,I)
  \tkzDrawSegments[->](O,J O,K)
  \tkzMarkRightAngles(I,O,J I,O,K)
  \tkzDrawPoints(O,I,J,K)
  \tkzLabelPoints(O,I,J,K)
\end{tikzpicture}
\end{tkzexample}

\subsubsection{Option \tkzname{orthogonal} more complicated example}
\begin{tkzexample}[latex=7cm,small]
\begin{tikzpicture}[scale=.75]
  \tkzDefPoints{0/0/A,6/0/B}
  \tkzDefMidPoint(A,B)
    \tkzGetPoint{I}
  \tkzDefPointWith[orthogonal,K=-.75](B,A)
  \tkzGetPoint{C}
  \tkzInterLC(B,C)(B,I)
     \tkzGetPoints{D}{F}
  \tkzDuplicateSegment(B,F)(A,F)
  \tkzGetPoint{E}
  \tkzDrawArc[delta=10](F,E)(B)
  \tkzInterLC(A,B)(A,E)
    \tkzGetPoints{N}{M}
  \tkzDrawArc[delta=10](A,M)(E)
  \tkzDrawLines(A,B B,C A,F)
  \tkzCompass(B,F)
  \tkzDrawPoints(A,B,C,F,M,E)
  \tkzLabelPoints(A,B,C,F,M)
  \tkzLabelPoints[above](E)
\end{tikzpicture}
\end{tkzexample}

\subsubsection{Options \tkzname{colinear} and \tkzname{orthogonal}}
\begin{tkzexample}[latex=7cm,small]
\begin{tikzpicture}[scale=1.2,
  vect/.style={->,shorten >=1pt,>=latex'}]
  \tkzDefPoints{2/1/A,6/2/B}
  \tkzDefPointWith[orthogonal,K=.5](A,B)
  \tkzGetPoint{C}
  \tkzDefPointWith[colinear=at C,K=.5](A,B)
  \tkzGetPoint{D}
  \tkzMarkRightAngle[fill=gray!20](B,A,C)
  \tkzDrawSegments[vect](A,B A,C C,D)
  \tkzDrawPoints(A,...,D)
\end{tikzpicture}
\end{tkzexample}

\subsubsection{Option  \tkzname{orthogonal normed}}
 $K=1$ $AC=1$.

\begin{tkzexample}[latex=7cm,small]
\begin{tikzpicture}[scale=1.2,
  vect/.style={->,shorten >=1pt,>=latex'}]
  \tkzDefPoints{2/3/A,4/2/B}
  \tkzDefPointWith[orthogonal normed](A,B)
  \tkzGetPoint{C}
  \tkzDrawPoints[color=red](A,B,C)
  \tkzDrawSegments[vect](A,B A,C)
  \tkzMarkRightAngle[fill=gray!20](B,A,C)
\end{tikzpicture}
\end{tkzexample}

\subsubsection{Option \tkzname{orthogonal normed} and K=2}
$K=2$ therefore $AC=2$.

\begin{tkzexample}[latex=7cm,small]
\begin{tikzpicture}[scale=1.2,
   vect/.style={->,shorten >=1pt,>=latex'}]
  \tkzDefPoints{2/3/A,5/1/B}
  \tkzDefPointWith[orthogonal normed,K=2](A,B)
  \tkzGetPoint{C}
  \tkzDrawPoints[color=red](A,B,C)
  \tkzDefCircle[R](A,2) \tkzGetPoint{a}
  \tkzDrawCircle(A,a)
  \tkzDrawSegments[vect](A,B A,C)
  \tkzMarkRightAngle[fill=gray!20](B,A,C)
  \tkzLabelPoints[above=3pt](A,B,C)
\end{tikzpicture}
\end{tkzexample}

\subsubsection{Option \tkzname{linear}}
Here $K=0.5$.

This amounts to applying a homothety or a multiplication of a vector by a real. Here is the middle of $[AB]$.

\begin{tkzexample}[latex=7cm,small]
\begin{tikzpicture}[scale=1.2]
  \tkzDefPoints{1/3/A,4/2/B}
  \tkzDefPointWith[linear,K=0.5](A,B)
  \tkzGetPoint{C}
  \tkzDrawPoints[color=red](A,B,C)
  \tkzDrawSegment(A,B)
  \tkzLabelPoints[above right=3pt](A,B,C)
\end{tikzpicture}
\end{tkzexample}

\subsubsection{Option \tkzname{linear normed}}
In the following example $AC=1$ and $C$ belongs to $(AB)$.

\begin{tkzexample}[latex=7cm,small]
\begin{tikzpicture}[scale=1.2]
 \tkzDefPoints{1/3/A,4/2/B}
 \tkzDefPointWith[linear normed](A,B)
 \tkzGetPoint{C}
 \tkzDrawPoints[color=red](A,B,C)
 \tkzDrawSegment(A,B)
 \tkzLabelSegment(A,C){$1$}
 \tkzLabelPoints[above right=3pt](A,B,C)
\end{tikzpicture}
\end{tkzexample}
%<--------------------------------------------------------------------------–>
%         tkzGetVectxy
%<--------------------------------------------------------------------------–>
\subsection{\tkzcname{tkzGetVectxy} }
Retrieving the coordinates of a vector.

\begin{NewMacroBox}{tkzGetVectxy}{\parg{$A,B$}\var{text}}%
Allows to obtain the coordinates of a vector.

\medskip
\begin{tabular}{lll}%
\toprule
arguments    & example & explanation      \\

\midrule

\TAline{(point)\{name of macro\}} {\tkzcname{tkzGetVectxy}(A,B)\{V\}}{\tkzcname{Vx},\tkzcname{Vy}: coordinates of $\overrightarrow{AB}$}
\end{tabular}
\end{NewMacroBox}

\subsubsection{Coordinate transfer with \tkzcname{tkzGetVectxy}}

\begin{tkzexample}[latex=7cm,small]
\begin{tikzpicture}
 \tkzDefPoints{0/0/O,1/1/A,4/2/B}
 \tkzGetVectxy(A,B){v}
 \tkzDefPoint(\vx,\vy){V}
 \tkzDrawSegment[->,color=red](O,V)
 \tkzDrawSegment[->,color=blue](A,B)
 \tkzDrawPoints(A,B,O)
 \tkzLabelPoints(A,B,O,V)
\end{tikzpicture}
\end{tkzexample}
\endinput
\section{The straight lines}

It is of course essential to draw straight lines, but before this can be done, it is necessary to be able to define certain particular lines such as mediators, bisectors, parallels or even perpendiculars. The principle is to determine two points on the straight line. 


\subsection{Definition of straight lines}

\begin{NewMacroBox}{tkzDefLine}{\oarg{local options}\parg{pt1,pt2} or \parg{pt1,pt2,pt3}}%
The argument is a list of two or three points. Depending on the case, the macro defines one or two points necessary to obtain the line sought. Either the macro \tkzcname{tkzGetPoint} or the macro \tkzcname{tkzGetPoints} must be used.

\medskip
\begin{tabular}{lll}%
\toprule
arguments           & example & explication                         \\
\midrule
\TAline{\parg{pt1,pt2}}{\parg{A,B}} {[mediator](A,B)}
\TAline{\parg{pt1,pt2,pt3}}{\parg{A,B,C}} {[bisector](B,A,C)}
\end{tabular}

\medskip
\begin{tabular}{lll}%
\toprule
options             & default & definition                         \\ 
\TOline{mediator}{}{two points are defined} 
\TOline{perpendicular=through\dots}{mediator}{perpendicular to a straight line passing through a point} 
\TOline{orthogonal=through\dots}{mediator}{see above }
\TOline{parallel=through\dots}{mediator}{parallel to a straight line passing through a point}
\TOline{bisector}{mediator}{bisector of an angle defined by three points}
\TOline{bisector out}{mediator}{Exterior Angle Bisector}
\TOline{K}{1}{coefficient for the perpendicular line}
\TOline{normed}{false}{normalizes the created segment}
\end{tabular}
\end{NewMacroBox}  

\subsubsection{Example with \tkzname{mediator}}  
\begin{tkzexample}[latex=5 cm,small]
\begin{tikzpicture}[rotate=25]
 \tkzDefPoints{-2/0/A,1/2/B}
 \tkzDefLine[mediator](A,B)          \tkzGetPoints{C}{D}
 \tkzDefPointWith[linear,K=.75](C,D) \tkzGetPoint{D}
 \tkzDefMidPoint(A,B)                \tkzGetPoint{I}
 \tkzFillPolygon[color=orange!30](A,C,B,D)
 \tkzDrawSegments(A,B C,D)
 \tkzMarkRightAngle(B,I,C) 
 \tkzDrawSegments(D,B D,A)
 \tkzDrawSegments(C,B C,A)
\end{tikzpicture}
\end{tkzexample}  

\subsubsection{Example with \tkzname{bisector} and \tkzname{normed}} 
\begin{tkzexample}[latex=7 cm,small] 
\begin{tikzpicture}[rotate=25,scale=.75]
 \tkzDefPoints{0/0/C, 2/-3/A, 4/0/B}
 \tkzDefLine[bisector,normed](B,A,C) \tkzGetPoint{a}
 \tkzDrawLines[add= 0 and .5](A,B A,C)
 \tkzShowLine[bisector,gap=4,size=2,color=red](B,A,C)
 \tkzDrawLines[blue!50,dashed,add= 0 and 3](A,a)
\end{tikzpicture}
\end{tkzexample} 

\subsubsection{Example with \tkzname{orthogonal} and \tkzname{parallel}}    
\begin{tkzexample}[latex=5 cm,small]
\begin{tikzpicture}
   \tkzDefPoints{-1.5/-0.25/A,1/-0.75/B,-0.7/1/C}
   \tkzDrawLine(A,B)
   \tkzLabelLine[pos=1.25,below left](A,B){$(d_1)$}
   \tkzDrawPoints(A,B,C)
   \tkzDefLine[orthogonal=through C](B,A) \tkzGetPoint{c}
   \tkzDrawLine(C,c) 
   \tkzLabelLine[pos=1.25,left](C,c){$(\delta)$}
   \tkzInterLL(A,B)(C,c) \tkzGetPoint{I}
   \tkzMarkRightAngle(C,I,B) 
   \tkzDefLine[parallel=through C](A,B) \tkzGetPoint{c'}
   \tkzDrawLine(C,c') 
   \tkzLabelLine[pos=1.25,below left](C,c'){$(d_2)$}
   \tkzMarkRightAngle(I,C,c')   
\end{tikzpicture}
\end{tkzexample}

\subsubsection{An envelope}
Based on a figure from O. Reboux with pst-eucl by D Rodriguez.

\begin{tkzexample}[vbox,small]
\begin{tikzpicture}[scale=.75]
 \tkzInit[xmin=-6,ymin=-4,xmax=6,ymax=6] % necessary
 \tkzClip
 \tkzDefPoint(0,0){O} 
 \tkzDefPoint(132:4){A}
 \tkzDefPoint(5,0){B}
 \foreach \ang in {5,10,...,360}{%
  \tkzDefPoint(\ang:5){M}
  \tkzDefLine[mediator](A,M)
  \tkzDrawLine[color=magenta,add= 3 and 3](tkzFirstPointResult,tkzSecondPointResult)}
\end{tikzpicture}
\end{tkzexample}

\subsubsection{A parabola}
Based on a figure from O. Reboux with pst-eucl by D Rodriguez.
It is not necessary to name the two points that define the mediator.

\begin{tkzexample}[vbox,small]
\begin{tikzpicture}[scale=.75]
 \tkzInit[xmin=-6,ymin=-4,xmax=6,ymax=6] 
 \tkzClip
 \tkzDefPoint(0,0){O} 
 \tkzDefPoint(132:5){A}
 \tkzDefPoint(4,0){B}
 \foreach \ang in {5,10,...,360}{%
  \tkzDefPoint(\ang:4){M}
  \tkzDefLine[mediator](A,M) 
  \tkzDrawLine[color=magenta,add= 3 and 3](tkzFirstPointResult,tkzSecondPointResult)}
\end{tikzpicture}
\end{tkzexample}

%<---------------------------------------------------------------------------->
\subsection{Specific lines:  Tangent to a circle}
Two constructions are proposed. The first one is the construction of a tangent to a circle at a given point of this circle and the second one is the construction of a tangent to a circle passing through a given point outside a disc. 

\begin{NewMacroBox}{tkzDefTangent}{\oarg{local options}\parg{pt1,pt2} or \parg{pt1,dim}}%
The parameter in brackets is the center of the circle or the center of the circle and a point on the circle or the center and the radius. This macro replaces the old one: \tkzcname{tkzTangent}.

\medskip
\begin{tabular}{lll}%
\toprule
arguments           & example & explication                         \\
\midrule
\TAline{\parg{pt1,pt2 or \parg{pt1,dim}} }{\parg{A,B} or \parg{A,2cm}} {$[AB]$ is radius $A$ is the center}
\bottomrule
\end{tabular} 

\medskip
\begin{tabular}{lll}%
options             & default & definition                         \\ 
\midrule
\TOline{at=pt}{at}{tangent to a point on the circle} 
\TOline{from=pt} {at}{tangent to a circle passing through a point}
\TOline{from with R=pt} {at}{idem, but the circle is defined by center = radius}  
\bottomrule
\end{tabular}

The tangent is not drawn. A second point of the tangent is given by \tkzname{tkzPointResult}.
\end{NewMacroBox}

\subsubsection{Example of a tangent passing through a point on the circle } 
\begin{tkzexample}[latex=7cm,small]
\begin{tikzpicture}[scale=.75]
  \tkzDefPoint(0,0){O}
  \tkzDefPoint(6,6){E}
  \tkzDefRandPointOn[circle=center O radius 3cm]
  \tkzGetPoint{A}
  \tkzDrawSegment(O,A)
  \tkzDrawCircle(O,A)
  \tkzDefTangent[at=A](O)
  \tkzGetPoint{h}
  \tkzDrawLine[add = 4 and 3](A,h)
  \tkzMarkRightAngle[fill=red!30](O,A,h)
\end{tikzpicture}
\end{tkzexample}

\subsubsection{Example of tangents passing through an external point } 
\begin{tkzexample}[latex=7cm,small]
\begin{tikzpicture}[scale=.8] 
  \tkzDefPoint(3,3){c}
  \tkzDefPoint(6,3){a0}  
  \tkzRadius=1 cm 
  \tkzDrawCircle[R](c,\tkzRadius) 
  \foreach \an in {0,10,...,350}{
     \tkzDefPointBy[rotation=center c angle \an](a0)  
     \tkzGetPoint{a}  
     \tkzDefTangent[from with R = a](c,\tkzRadius)  
     \tkzGetPoints{e}{f} 
     \tkzDrawLines[color=magenta](a,f a,e) 
      \tkzDrawSegments(c,e c,f)
      }%
\end{tikzpicture} 
\end{tkzexample}

\subsubsection{Example of Andrew Mertz}
\begin{tkzexample}[latex=6cm,small]
 \begin{tikzpicture}[scale=.5] 
 \tkzDefPoint(100:8){A}\tkzDefPoint(50:8){B}  
 \tkzDefPoint(0,0){C} \tkzDefPoint(0,4){R} 
 \tkzDrawCircle(C,R)
 \tkzDefTangent[from = A](C,R)  \tkzGetPoints{D}{E}
 \tkzDefTangent[from = B](C,R)  \tkzGetPoints{F}{G}
 \tkzDrawSector[fill=blue!80!black,opacity=0.5](A,D)(E)
 \tkzFillSector[color=red!80!black,opacity=0.5](B,F)(G)
 \tkzInterCC(A,D)(B,F) \tkzGetSecondPoint{I}
 \tkzDrawPoint[color=black](I)
 \end{tikzpicture}
\end{tkzexample}
\url{http://www.texample.net/tikz/examples/}  

\subsubsection{Drawing a tangent option \tkzimp{from with R} and \tkzimp{at}}
\begin{tkzexample}[latex=7cm,small]
  \begin{tikzpicture}[scale=.5] 
  \tkzDefPoint(0,0){O}
  \tkzDefRandPointOn[circle=center O radius 4cm]
  \tkzGetPoint{A}
  \tkzDefTangent[at=A](O)
  \tkzGetPoint{h}
  \tkzDrawSegments(O,A) 
  \tkzDrawCircle(O,A) 
  \tkzDrawLine[add = 1 and 1](A,h)
  \tkzMarkRightAngle[fill=red!30](O,A,h)
  \end{tikzpicture}
\end{tkzexample}

\subsubsection{Drawing a tangent option \tkzimp{from}}
\begin{tkzexample}[latex=5cm,small]
\begin{tikzpicture}[scale=.5] 
 \tkzDefPoint(0,0){B} 
 \tkzDefPoint(0,8){A} 
 \tkzDefSquare(A,B)
 \tkzGetPoints{C}{D}
 \tkzDrawSquare(A,B)
 \tkzClipPolygon(A,B,C,D)
 \tkzDefPoint(4,8){F}
 \tkzDefPoint(4,0){E}
 \tkzDefPoint(4,4){Q}
 \tkzFillPolygon[color = green](A,B,C,D)
 \tkzDrawCircle[fill = orange](B,A)
 \tkzDrawCircle[fill = purple](E,B)  
 \tkzDefTangent[from=B](F,A)
 \tkzInterLL(F,tkzFirstPointResult)(C,D)
 \tkzInterLL(A,tkzPointResult)(F,E) 
 \tkzDrawCircle[fill = yellow](tkzPointResult,Q)  
 \tkzDefPointBy[projection= onto B--A](tkzPointResult)
 \tkzDrawCircle[fill = blue!50!black](tkzPointResult,A)
\end{tikzpicture}
\end{tkzexample}


\section{Drawing, naming the lines}
The following macros are simply used to draw, name lines.
\subsection{Draw a straight line}
To draw a normal straight line, just give a couple of points. You can  use the \tkzname{add} option to extend the line (This option is due to \tkzimp{Mark Wibrow}, see the code below). 

\begin{tkzltxexample}[]
  \tikzset{%
    add/.style args={#1 and #2}{
        to path={%
 ($(\tikztostart)!-#1!(\tikztotarget)$)--($(\tikztotarget)!-#2!(\tikztostart)$)%
  \tikztonodes}}}
\end{tkzltxexample}

In the special case of lines defined in a triangle, the number of arguments is a list of three points (the vertices of the triangle). The second point is where the line will come from. The first and last points determine the target segment. The old method has therefore been slightly modified. So for \tkzcname{tkzDrawMedian}, instead of $(A,B)(C)$ you have to write $(B,C,A)$ where $C$ is the point that will be linked to the middle of the segment $[A,B]$.

\begin{NewMacroBox}{tkzDrawLine}{\oarg{local options}\parg{pt1,pt2} or \parg{pt1,pt2,pt3}}%
The arguments are a list of two points or three points.

\begin{tabular}{lll}%
\toprule
options             & default & definition                         \\ 
\midrule
\TOline{median}{none}{[median](A,B,C) median from $B$}
\TOline{altitude}{none}{[altitude](C,A,B) altitude from $A$} 
\TOline{bisector}{none}{[bisector](B,C,A) bisector from $C$}
\TOline{none}{none}{draw the straight line $(AB)$} 
\TOline{add= nb1 and nb2}{.2 and .2}{extends the segment} 
 \bottomrule
\end{tabular}

\tkzname{add} defines the length of the line passing through the points pt1 and pt2. Both numbers are percentages. The styles of \TIKZ\ are accessible for plots.
\end{NewMacroBox}

\subsubsection{Examples  with \tkzname{add}}
\begin{tkzexample}[latex=5cm,small]
\begin{tikzpicture}
 \tkzInit[xmin=-2,xmax=3,ymin=-2.25,ymax=2.25]
 \tkzClip[space=.25]
 \tkzDefPoint(0,0){A} \tkzDefPoint(2,0.5){B}
 \tkzDefPoint(0,-1){C}\tkzDefPoint(2,-0.5){D} 
 \tkzDefPoint(0,1){E} \tkzDefPoint(2,1.5){F} 
 \tkzDefPoint(0,-2){G} \tkzDefPoint(2,-1.5){H}
 \tkzDrawLine(A,B)    \tkzDrawLine[add = 0 and .5](C,D) 
 \tkzDrawLine[add = 1 and 0](E,F)
 \tkzDrawLine[add = 0 and 0](G,H) 
 \tkzDrawPoints(A,B,C,D,E,F,G,H)    
 \tkzLabelPoints(A,B,C,D,E,F,G,H)  
\end{tikzpicture}
\end{tkzexample} 

It is possible to draw several lines, but with the same options. 
\begin{NewMacroBox}{tkzDrawLines}{\oarg{local options}\parg{pt1,pt2 pt3,pt4 ...}}% 
Arguments are a list of pairs of  points separated by spaces.  The styles of \TIKZ\ are available for the draws. 
\end{NewMacroBox}      

\subsubsection{Example with \tkzcname{tkzDrawLines}}    

\begin{tkzexample}[latex=8cm,small]
\begin{tikzpicture}
  \tkzDefPoint(0,0){A}
  \tkzDefPoint(2,0){B}
  \tkzDefPoint(1,2){C}
  \tkzDefPoint(3,2){D}   
  \tkzDrawLines(A,B C,D A,C B,D)
  \tkzLabelPoints(A,B,C,D)
\end{tikzpicture}
\end{tkzexample}

\subsubsection{Example with  the option \tkzname{add}}   
\begin{tkzexample}[latex=8cm,small]
\begin{tikzpicture}[scale=.5]
 \tkzDefPoint(0,0){O}
 \tkzDefPoint(3,1){I}
 \tkzDefPoint(1,4){J}
 \tkzDefLine[bisector](I,O,J)     
   \tkzGetPoint{i}   
 \tkzDefLine[bisector out](I,O,J) 
   \tkzGetPoint{j}
 \tkzDrawLines[add = 1 and .5,color=red](O,I O,J) 
 \tkzDrawLines[add = 1 and .5,color=blue](O,i O,j) 
\end{tikzpicture} 
\end{tkzexample}

\subsubsection{Medians in a triangle}
\begin{tkzexample}[latex=7 cm,small]
\begin{tikzpicture}[scale=1.25]
  \tkzDefPoint(0,0){A} \tkzDefPoint(4,0){B}
  \tkzDefPoint(1,3){C} \tkzDrawPolygon(A,B,C)
  \tkzSetUpLine[color=blue]
  \tkzDrawLine[median](B,C,A)
  \tkzDrawLine[median](C,A,B)
  \tkzDrawLine[median](A,B,C)
\end{tikzpicture}
\end{tkzexample}

\subsubsection{Altitudes in a triangle}
\begin{tkzexample}[latex=7 cm,small]
\begin{tikzpicture}[scale=1.25]
 \tkzDefPoint(0,0){A} \tkzDefPoint(4,0){B}
 \tkzDefPoint(1,3){C} \tkzDrawPolygon(A,B,C)
 \tkzSetUpLine[color=magenta]
 \tkzDrawLine[altitude](B,C,A)
 \tkzDrawLine[altitude](C,A,B)
 \tkzDrawLine[altitude](A,B,C)
\end{tikzpicture}
\end{tkzexample}

\subsubsection{Bisectors in a triangle}
You have to give the angles in a straight line.

\begin{tkzexample}[latex=7 cm,small]
\begin{tikzpicture}[scale=1.25]
 \tkzDefPoint(0,0){A} \tkzDefPoint(4,0){B}
 \tkzDefPoint(1,3){C} \tkzDrawPolygon(A,B,C)
 \tkzSetUpLine[color=purple]
 \tkzDrawLine[bisector](B,C,A)
 \tkzDrawLine[bisector](C,A,B)
 \tkzDrawLine[bisector](A,B,C)
\end{tikzpicture}
\end{tkzexample}

\subsection{Add labels on a straight line \tkzcname{tkzLabelLine}}% 
\begin{NewMacroBox}{tkzLabelLine}{\oarg{local options}\parg{pt1,pt2}\marg{label}}
\begin{tabular}{lll}%
arguments &  default & definition   \\ 
\midrule
\TAline{label}{}{\tkzcname{tkzLabelLine(A,B)}\{\$\tkzcname{Delta}\$\}}
\bottomrule
\end{tabular}

\begin{tabular}{lll}%
options             & default & definition   \\ 
\midrule
\TOline{pos}{.5}{\tkzname{pos} is an option for \TIKZ, but essential in this case\dots} 
\end{tabular}

As an option, and in addition to the \tkzname{pos}, you can use all styles of \TIKZ, especially the placement with \tkzname{above}, \tkzname{right}, \dots
\end{NewMacroBox}

\subsubsection{Example with \tkzcname{tkzLabelLine}}
An important option is \tkzname{pos}, it's the one that allows you to place the label along the right. The value of \tkzname{pos} can be greater than 1 or negative.

\begin{tkzexample}[latex=6cm,small]
\begin{tikzpicture}
   \tkzDefPoints{0/0/A,3/0/B,1/1/C}
   \tkzDefLine[perpendicular=through C,K=-1](A,B)
   \tkzGetPoint{c}
   \tkzDrawLines(A,B C,c)
   \tkzLabelLine[pos=1.25,blue,right](C,c){$(\delta)$} 
   \tkzLabelLine[pos=-0.25,red,left](C,c){again $(\delta)$} 
\end{tikzpicture}
\end{tkzexample}

\section{Draw, Mark segments}
There is, of course, a macro to simply draw a segment (it would be possible, as for a half line, to create a style with \tkzcname{add}).
\subsection{Draw a segment \tkzcname{tkzDrawSegment}} 
\begin{NewMacroBox}{tkzDrawSegment}{\oarg{local options}\parg{pt1,pt2}}%
The arguments are a list of two points. The styles of \TIKZ\ are available for the drawings.
 
\medskip
\begin{tabular}{lll}%
argument    & example & definition    \\
\midrule
\TAline{(pt1,pt2)}{(A,B)}{draw the segment $[A,B]$}
\bottomrule 
\end{tabular}
 
\medskip
\begin{tabular}{lll}%
options    & example & definition    \\
\midrule
\TOline{\TIKZ\ options}{}{all \TIKZ\ options are valid.}
\TOline{add}{0 and 0}{add = $kl$ and $kr$, \dots}
\TOline{\dots}{\dots}{allows the segment to be extended to the left and right. }
\TOline{dim}{no default}{dim = \{label,dim,option\}, \dots}
\TOline{\dots}{\dots}{allows you to add dimensions to a figure.}
\bottomrule 
\end{tabular}

This is of course equivalent to \tkzcname{draw (A)--(B);} 
\end{NewMacroBox}

\subsubsection{Example with point references}     

\begin{tkzexample}[latex=6cm,small]
\begin{tikzpicture}[scale=1.5]
  \tkzDefPoint(0,0){A}
  \tkzDefPoint(2,1){B}
  \tkzDrawSegment[color=red,thin](A,B)
  \tkzDrawPoints(A,B)    
  \tkzLabelPoints(A,B)  
\end{tikzpicture}
\end{tkzexample}

\subsubsection{Example of extending an segment with option \tkzname{add}} 

\begin{tkzexample}[latex=7cm,small]
  \begin{tikzpicture}
  \tkzDefPoints{0/0/A,6/0/B,0.8/4/C}
  \tkzDefTriangleCenter[euler](A,B,C) 
  \tkzGetPoint{E}
  \tkzDrawCircle[euler,red](A,B,C)
  \tkzDrawLines[add=.5 and .5](A,B A,C B,C)
  \tkzDrawPoints(A,B,C,E)
  \tkzLabelPoints(A,B,C,E)
  \end{tikzpicture}
\end{tkzexample}

\subsubsection{Example of adding dimensions with option \tkzname{dim}} 
\begin{tkzexample}[vbox,small]
\begin{tikzpicture}[scale=4]
 \pgfkeys{/pgf/number format/.cd,fixed,precision=2}
 % Define the first two points
 \tkzDefPoint(0,0){A}
 \tkzDefPoint(3,0){B}
 \tkzDefPoint(1,1){C}
 % Draw the triangle and the points
 \tkzDrawPolygon(A,B,C)
 \tkzDrawPoints(A,B,C)
 % Label the sides
 \tkzCalcLength[cm](A,B)\tkzGetLength{ABl}
 \tkzCalcLength[cm](B,C)\tkzGetLength{BCl}
 \tkzCalcLength[cm](A,C)\tkzGetLength{ACl}
 % add dim
 \tkzDrawSegment[dim={\pgfmathprintnumber\BCl,6pt,transform shape}](C,B)
 \tkzDrawSegment[dim={\pgfmathprintnumber\ACl,6pt,transform shape}](A,C)
 \tkzDrawSegment[dim={\pgfmathprintnumber\ABl,-6pt,transform shape}](A,B)
\end{tikzpicture}
\end{tkzexample}

 
\subsection{Drawing segments \tkzcname{tkzDrawSegments}} 
If the options are the same we can plot several segments with the same macro. 

\begin{NewMacroBox}{tkzDrawSegments}{\oarg{local options}\parg{pt1,pt2 pt3,pt4 ...}}%
The arguments are a two-point couple list. The styles of \TIKZ\ are available for the plots.
\end{NewMacroBox}

\begin{tkzexample}[latex=6cm,small]
\begin{tikzpicture}
  \tkzInit[xmin=-1,xmax=3,ymin=-1,ymax=2]
  \tkzClip[space=1]
  \tkzDefPoint(0,0){A}
  \tkzDefPoint(2,1){B} 
  \tkzDefPoint(3,0){C} 
  \tkzDrawSegments(A,B B,C)
  \tkzDrawPoints(A,B,C)    
  \tkzLabelPoints(A,C) 
  \tkzLabelPoints[above](B)  
\end{tikzpicture}
\end{tkzexample}

\subsubsection{Place an arrow on segment}
\begin{tkzexample}[latex=6cm,small]
\begin{tikzpicture}
  \tikzset{
    arr/.style={postaction=decorate,
    decoration={markings, 
    mark=at position .5 with {\arrow[thick]{#1}}
      }}}
  \tkzDefPoint(0,0){A}
  \tkzDefPoint(4,-4){B}
  \tkzDrawSegments[arr=stealth](A,B)
  \tkzDrawPoints(A,B) 
\end{tikzpicture}
\end{tkzexample}

\subsection{Mark a segment \tkzcname{tkzMarkSegment}}
\hypertarget{tms}{}  
  
 \begin{NewMacroBox}{tkzMarkSegment}{\oarg{local options}\parg{pt1,pt2}}% 
The macro allows you to place a mark on a segment.

\medskip
\begin{tabular}{lll}%
\toprule
options             & default & definition   \\
\midrule
\TOline{pos}{.5}{position of the mark} 
\TOline{color}{black}{color of the mark} 
\TOline{mark}{none}{choice of the mark} 
\TOline{size}{4pt}{size of the mark}
\bottomrule
\end{tabular}

Possible marks are those provided by \TIKZ, but other marks have been created based on an idea by Yves Combe.
\end{NewMacroBox} 

\subsubsection{Several marks }
\begin{tkzexample}[latex=5cm,small] 
\begin{tikzpicture}
  \tkzDefPoint(2,1){A}
  \tkzDefPoint(6,4){B}
  \tkzDrawSegment(A,B)
  \tkzMarkSegment[color=brown,size=2pt,pos=0.4, mark=z](A,B) 
  \tkzMarkSegment[color=blue,pos=0.2, mark=oo](A,B)
  \tkzMarkSegment[pos=0.8,mark=s,color=red](A,B) 
\end{tikzpicture}
\end{tkzexample}

\subsubsection{Use of \tkzname{mark}}      
\begin{tkzexample}[latex=5cm,small] 
\begin{tikzpicture}
  \tkzDefPoint(2,1){A} 
  \tkzDefPoint(6,4){B}
  \tkzDrawSegment(A,B)
  \tkzMarkSegment[color=gray,pos=0.2,mark=s|](A,B)
  \tkzMarkSegment[color=gray,pos=0.4,mark=s||](A,B)
  \tkzMarkSegment[color=brown,pos=0.6,mark=||](A,B)
  \tkzMarkSegment[color=red,pos=0.8,mark=|||](A,B)
\end{tikzpicture}
\end{tkzexample}


\subsection{Marking segments \tkzcname{tkzMarkSegments}}
\hypertarget{tmss}{} 
 
\begin{NewMacroBox}{tkzMarkSegments}{\oarg{local options}\parg{pt1,pt2 pt3,pt4 ...}}%
Arguments are a list of pairs of points separated by spaces. The styles of \TIKZ\ are available for plots.
\end{NewMacroBox} 

\subsubsection{Marks for an isosceles triangle}      
\begin{tkzexample}[latex=6cm,small]
\begin{tikzpicture}[scale=1]
 \tkzDefPoints{0/0/O,2/2/A,4/0/B,6/2/C}
 \tkzDrawSegments(O,A A,B)
 \tkzDrawPoints(O,A,B)
 \tkzDrawLine(O,B)   
 \tkzMarkSegments[mark=||,size=6pt](O,A A,B)
\end{tikzpicture}
\end{tkzexample} 

\subsection{Another marking}   
\begin{tkzexample}[latex=5cm,small] 
 \begin{tikzpicture}[scale=1]
  \tkzDefPoint(0,0){A}\tkzDefPoint(3,2){B} 
  \tkzDefPoint(4,0){C}\tkzDefPoint(2.5,1){P}
  \tkzDrawPolygon(A,B,C)
  \tkzDefEquilateral(A,P) \tkzGetPoint{P'}
  \tkzDefPointsBy[rotation=center A angle 60](P,B){P',C'}
  \tkzDrawPolygon(A,P,P')
  \tkzDrawPolySeg(P',C',A,P,B)
  \tkzDrawSegment(C,P)
  \tkzDrawPoints(A,B,C,C',P,P')
  \tkzMarkSegments[mark=s|,size=6pt,
  color=blue](A,P P,P' P',A) 
  \tkzMarkSegments[mark=||,color=orange](B,P P',C')
  \tkzLabelPoints(A,C) \tkzLabelPoints[below](P) 
  \tkzLabelPoints[above right](P',C',B) 
\end{tikzpicture} 
\end{tkzexample}  

\hypertarget{tls}{}  
\begin{NewMacroBox}{tkzLabelSegment}{\oarg{local options}\parg{pt1,pt2}\marg{label}}
This macro allows you to place a label along a segment or a line. The options are those of \TIKZ\ for example \tkzname{pos}.

\medskip
\begin{tabular}{lll}%%
argument    & example & definition    \\
\midrule
\TAline{label}{\tkzcname{tkzLabelSegment(A,B)\{$5$\}}}{label text} 
\TAline{(pt1,pt2)}{(A,B)}{label along $[AB]$} 
\bottomrule
\end{tabular}


\medskip
\begin{tabular}{lll}%
options  & default & definition    \\
\midrule
\TOline{pos}{.5}{label's position} 
\end{tabular}
\end{NewMacroBox}  

\subsubsection{Multiple labels}      
\begin{tkzexample}[latex=7 cm,small]
\begin{tikzpicture}
\tkzInit
\tkzDefPoint(0,0){A}
\tkzDefPoint(6,0){B}
\tkzDrawSegment(A,B)
\tkzLabelSegment[above,pos=.8](A,B){$a$}
\tkzLabelSegment[below,pos=.2](A,B){$4$}
\end{tikzpicture} 
\end{tkzexample}  

\subsubsection{Labels and right-angled triangle}
\begin{tkzexample}[latex=7cm,small]
\begin{tikzpicture}[rotate=-60]
\tikzset{label seg style/.append style = {%
	        color      = red,
	        }}
\tkzDefPoint(0,1){A}
\tkzDefPoint(2,4){C}
\tkzDefPointWith[orthogonal normed,K=7](C,A)
\tkzGetPoint{B}
\tkzDrawPolygon[green!60!black](A,B,C)
\tkzDrawLine[altitude,dashed,color=magenta](B,C,A)
\tkzGetPoint{P}
\tkzLabelPoint[left](A){$A$}
\tkzLabelPoint[right](B){$B$}
\tkzLabelPoint[above](C){$C$}
\tkzLabelPoint[below](P){$P$}
\tkzLabelSegment[](B,A){$c$}
\tkzLabelSegment[swap](B,C){$a$}
\tkzLabelSegment[swap](C,A){$b$}
\tkzMarkAngles[size=1cm,
     color=cyan,mark=|](C,B,A A,C,P)
\tkzMarkAngle[size=0.75cm,
     color=orange,mark=||](P,C,B)
\tkzMarkAngle[size=0.75cm,
      color=orange,mark=||](B,A,C)
\tkzMarkRightAngles[german](A,C,B B,P,C)
\end{tikzpicture} 
\end{tkzexample}

\hypertarget{tlss}{} 
 \begin{NewMacroBox}{tkzLabelSegments}{\oarg{local options}\parg{pt1,pt2 pt3,pt4 ...}}%
The arguments are a two-point couple list. The styles of \TIKZ\ are available for plotting.
\end{NewMacroBox} 
 
\subsubsection{Labels for an isosceles triangle}      
\begin{tkzexample}[latex=6cm,small]
\begin{tikzpicture}[scale=1]
 \tkzDefPoints{0/0/O,2/2/A,4/0/B,6/2/C}
 \tkzDrawSegments(O,A A,B)
 \tkzDrawPoints(O,A,B)
 \tkzDrawLine(O,B)   
 \tkzLabelSegments[color=red,above=4pt](O,A A,B){$a$}
\end{tikzpicture}
\end{tkzexample}  
\endinput
\section{Triangles}

\subsection{Definition of triangles \tkzcname{tkzDefTriangle}}
The following macros will allow you to define or construct a triangle from \tkzname{at least} two points.

 At the moment, it is possible to define the following triangles:
 \begin{itemize}
\item  \tkzname{two angles}  determines a triangle with two angles;
\item  \tkzname{equilateral}  determines an equilateral triangle;
\item  \tkzname{isosceles right}  determines an isoxsceles right triangle;
\item \tkzname{half} determines a right-angled triangle such that the ratio of the measurements of the two adjacent sides to the right angle is equal to $2$;
\item \tkzname{pythagore} determines a right-angled triangle whose side measurements are proportional to 3, 4 and 5;
\item \tkzname{school} determines a right-angled triangle whose angles are 30, 60 and 90 degrees;
\item \tkzname{golden} determines a right-angled triangle such that the ratio of the measurements on the two adjacent sides to the right angle is equal to $\Phi=1.618034$, I chose "golden triangle" as the denomination because it comes from the golden rectangle and I kept the denomination "gold triangle" or "Euclid's triangle" for the isosceles triangle whose angles at the base are 72 degrees;

\item  \tkzname{euclid} or \tkzname{gold} for the gold triangle; in the previous version the option was "euclide" with an "e".

\item \tkzname{cheops} determines a third point such that the triangle is isosceles with side measurements proportional to $2$, $\Phi$ and $\Phi$.
\end{itemize}

\newpage
\begin{NewMacroBox}{tkzDefTriangle}{\oarg{local options}\parg{A,B}}%
The points are ordered because the triangle is constructed following the direct direction of the trigonometric circle. This macro is either used in partnership with \tkzcname{tkzGetPoint} or by using \tkzname{tkzPointResult} if it is not necessary to keep the name.

\medskip
\begin{tabular}{lll}%
\toprule
options             & default & definition                        \\
\midrule
\TOline{two angles= \#1 and \#2}{no defaut}{triangle knowing two angles}
\TOline{equilateral} {equilateral}{equilateral triangle }
\TOline{half} {equilateral}{B rectangle  $AB=2BC$ $AC$ hypothenuse }
\TOline{isosceles right} {equilateral}{isosceles right triangle }
\TOline{pythagore}{equilateral}{proportional to the pythagorean triangle 3-4-5}
\TOline{pythagoras}{equilateral}{same as above}
\TOline{egyptian}{equilateral}{same as above}
\TOline{school} {equilateral}{angles of 30, 60 and 90 degrees }
\TOline{gold}{equilateral}{B rectangle and $AB/AC = \Phi$}
\TOline{euclid} {equilateral}{angles of 72, 72 and 36 degrees, $A$ is the apex}
\TOline{golden} {equilateral}{angles of 72, 72 and 36 degrees, $C$ is the apex}
\TOline{sublime} {equilateral}{angles of 72, 72 and 36 degrees, $C$ is the apex}
\TOline{cheops} {equilateral}{AC=BC, AC and BC are proportional to $2$ and $\Phi$.}
\TOline{swap} {false}{gives the symmetric point with respect to $AB$}
\bottomrule
\end{tabular}

\medskip
\emph{\tkzcname{tkzGetPoint} allows you to store the point otherwise \tkzname{tkzPointResult} allows for immediate use.}
\end{NewMacroBox}

\subsubsection{Option \tkzname{equilateral}}
\begin{tkzexample}[latex=7 cm,small]
\begin{tikzpicture}
  \tkzDefPoint(0,0){A}
  \tkzDefPoint(4,0){B}
  \tkzDefTriangle[equilateral](A,B)
  \tkzGetPoint{C}
  \tkzDrawPolygons(A,B,C)
  \tkzDefTriangle[equilateral](B,A)
  \tkzGetPoint{D}
  \tkzDrawPolygon(B,A,D)
  \tkzMarkSegments[mark=s|](A,B B,C A,C A,D B,D)
\end{tikzpicture}
\end{tkzexample}


\subsubsection{Option \tkzname{two angles}}
\begin{tkzexample}[latex=6 cm,small]
\begin{tikzpicture}
\tkzDefPoint(0,0){A} 
\tkzDefPoint(5,0){B} 
\tkzDefTriangle[two angles = 50 and 70](A,B)
\tkzGetPoint{C} 
\tkzDrawSegment(A,B) 
\tkzDrawPoints(A,B) 
\tkzLabelPoints(A,B) 
\tkzDrawSegments[new](A,C B,C) 
\tkzDrawPoints[new](C)
\tkzLabelPoints[above,new](C)
\tkzLabelAngle[pos=1.4](B,A,C){$50^\circ$}
\tkzLabelAngle[pos=0.8](C,B,A){$70^\circ$}
\end{tikzpicture}
\end{tkzexample}

\subsubsection{Option \tkzname{school}}
The angles are 30, 60 and 90 degrees.

\begin{tkzexample}[latex=6 cm,small]
\begin{tikzpicture}
  \tkzDefPoints{0/0/A,4/0/B}
  \tkzDefTriangle[school](A,B)  
  \tkzGetPoint{C}
  \tkzMarkRightAngles(C,B,A)
  \tkzLabelAngle[pos=0.8](B,A,C){$30^\circ$}
  \tkzLabelAngle[pos=0.8](C,B,A){$90^\circ$}
  \tkzLabelAngle[pos=0.8](A,C,B){$60^\circ$} 
  \tkzDrawSegments(A,B)
  \tkzDrawSegments[new](A,C B,C)
  \tkzLabelPoints(A,B)
  \tkzLabelPoints[above](C)
\end{tikzpicture}
\end{tkzexample}

\subsubsection{Option \tkzname{pythagore}}
This triangle has sides whose lengths are proportional to 3, 4 and 5.

\begin{tkzexample}[latex=6 cm,small]
\begin{tikzpicture} 
  \tkzDefPoints{0/0/A,4/0/B} 
  \tkzDefTriangle[pythagore](A,B) 
  \tkzGetPoint{C} 
  \tkzDrawSegments(A,B)
  \tkzDrawSegments[new](A,C B,C)
  \tkzMarkRightAngles(A,B,C)
  \tkzDrawPoints[new](C) 
  \tkzDrawPoints(A,B) 
  \tkzLabelPoints[above](A,B)
  \tkzLabelPoints[new](C)  
\end{tikzpicture}
\end{tkzexample}

\subsubsection{Option \tkzname{pythagore} and \tkzname{swap}}
This triangle has sides whose lengths are proportional to 3, 4 and 5.

\begin{tkzexample}[latex=6 cm,small]
\begin{tikzpicture} 
  \tkzDefPoints{0/0/A,4/0/B} 
  \tkzDefTriangle[pythagore,swap](A,B) 
  \tkzGetPoint{C} 
  \tkzDrawSegments(A,B)
  \tkzDrawSegments[new](A,C B,C)
  \tkzMarkRightAngles(A,B,C)
  \tkzLabelPoint[above,new](C){$C$} 
  \tkzDrawPoints[new](C) 
  \tkzDrawPoints(A,B) 
  \tkzLabelPoints(A,B) 
\end{tikzpicture}
\end{tkzexample}

\subsubsection{Option \tkzname{golden}}
\begin{tkzexample}[latex=6 cm,small]
\begin{tikzpicture}[scale=.8]
\tkzDefPoint(0,0){A} \tkzDefPoint(4,0){B} 
\tkzDefTriangle[golden](A,B)\tkzGetPoint{C} 
\tkzDefSpcTriangle[in,name=M](A,B,C){a,b,c}
\tkzDrawPolygon(A,B,C) 
\tkzDrawPoints(A,B) 
\tkzDrawSegment(C,Mc) 
\tkzDrawPoints[new](C)
\tkzLabelPoints(A,B) 
\tkzLabelPoints[above,new](C)
\end{tikzpicture}
\end{tkzexample}

\subsubsection{Option \tkzname{euclid}}
\tkzimp{Euclid} and \tkzimp{golden} are identical but the segment AB is a base in one and a side in the other. 

\begin{tkzexample}[latex=7 cm,small]
\begin{tikzpicture}[scale=.75]
 \tkzDefPoint(0,0){A} \tkzDefPoint(4,0){B}
 \tkzDefTriangle[euclid](A,B)\tkzGetPoint{C}
 \tkzDrawPolygon(A,B,C)
 \tkzDrawPoints(A,B,C)
 \tkzLabelPoints(C)
 \tkzLabelPoints[above](A,B)
 \tkzLabelAngle[pos=0.8](A,B,C){$72^\circ$}
 \tkzLabelAngle[pos=0.8](B,C,A){$72^\circ$}
 \tkzLabelAngle[pos=0.8](C,A,B){$36^\circ$}
\end{tikzpicture}
\end{tkzexample}

\subsubsection{Option \tkzname{isosceles right}}
\begin{tkzexample}[latex=7 cm,small]
\begin{tikzpicture}
  \tkzDefPoint(0,0){A}
  \tkzDefPoint(4,0){B}
  \tkzDefTriangle[isosceles right](A,B)
  \tkzGetPoint{C}
  \tkzDrawPolygons(A,B,C)
  \tkzDrawPoints(A,B,C)
  \tkzMarkRightAngles(A,C,B)
  \tkzLabelPoints(A,B)
  \tkzLabelPoints[above](C)
\end{tikzpicture}
\end{tkzexample}

\subsubsection{Option \tkzname{gold} }
\begin{tkzexample}[latex=6 cm,small]
\begin{tikzpicture}
 \tkzDefPoints{0/0/A,4/0/B} 
 \tkzDefTriangle[gold](A,B)
 \tkzGetPoint{C}
 \tkzDrawPolygon(A,B,C)
 \tkzDrawPoints(A,B,C)
 \tkzLabelPoints[above](A,B) 
 \tkzLabelPoints[below](C)
 \tkzMarkRightAngle(A,B,C)
 \tkzText(0,-2){$\dfrac{AC}{AB}=\varphi$}
\end{tikzpicture}
\end{tkzexample}

\clearpage
\subsection{Specific triangles with \tkzcname{tkzDefSpcTriangle}}

The centers of some triangles have been defined in the "points" section, here it is a question of determining the three vertices of specific triangles.

\begin{NewMacroBox}{tkzDefSpcTriangle}{\oarg{local options}\parg{p1,p2,p3}\marg{r1,r2,r3}}
The order of the points is important! p1p2p3 defines a triangle then the result is a triangle whose vertices have as reference a combination with \tkzname{name} and r1,r2, r3. If \tkzname{name} is empty then the references are  r1,r2 and r3.

\medskip
\begin{tabular}{lll}%
\toprule
options             & default & definition                        \\
\midrule
\TOline{orthic} {centroid}{determined by endpoints of the altitudes ...}
\TOline{centroid or medial}{centroid}{intersection of the triangle's three triangle medians}
\TOline{in or incentral}{centroid}{determined with the angle bisectors}
\TOline{ex or excentral} {centroid}{determined with the excenters}
\TOline{extouch}{centroid}{formed by the points of tangency  with the excircles}
\TOline{intouch or contact} {centroid}{formed by the points of tangency of the incircle}
\TOline{} {}{each of the vertices}
\TOline{euler} {centroid}{formed by Euler points on the  nine-point circle}
\TOline{symmedial} {centroid}{intersection points of the symmedians}
\TOline{tangential}{centroid}{formed by the lines tangent to the circumcircle}
\TOline{feuerbach} {centroid}{formed by the points of tangency of the nine-point ...}
\TOline{} {} {circle with the excircles}
\TOline{name} {empty}{used to name the vertices}
\midrule
\end{tabular}
\end{NewMacroBox}

\subsubsection{How to name the vertices}

With \tkzcname{tkzDefSpcTriangle[medial,name=M](A,B,C)\{\_A,\_B,\_C\}} you get three vertices named $M_A$, $M_B$ and $M_C$.

With \tkzcname{tkzDefSpcTriangle[medial](A,B,C)\{a,b,c\}} you get three vertices named and labeled $a$, $b$ and $c$.

Possible \tkzcname{tkzDefSpcTriangle[medial,name=M\_](A,B,C)\{A,B,C\}} you get three vertices named $M_A$, $M_B$ and $M_C$.

\subsection{Option \tkzname{medial} or \tkzname{centroid} }
The geometric centroid  of the polygon vertices of a triangle is the point $G$ (sometimes also denoted $M$) which is also the intersection of the triangle's three triangle medians. The point is therefore sometimes called the median point. The centroid is always in the interior of the triangle.
\\

\href{http://mathworld.wolfram.com/TriangleCentroid.html}{Weisstein, Eric W. "Centroid triangle" From MathWorld--A Wolfram Web Resource.}

In the following example, we obtain the Euler circle which passes through the previously defined points.

\begin{tkzexample}[latex=7cm,small]
  \begin{tikzpicture}[rotate=90,scale=.75]
   \tkzDefPoints{0/0/A,6/0/B,0.8/4/C}
   \tkzDefTriangleCenter[centroid](A,B,C)
   \tkzGetPoint{M}
   \tkzDefSpcTriangle[medial,name=M](A,B,C){_A,_B,_C}
   \tkzDrawPolygon(A,B,C)
   \tkzDrawSegments[dashed,new](A,M_A B,M_B C,M_C)
   \tkzDrawPolygon[new](M_A,M_B,M_C)
   \tkzDrawPoints(A,B,C)
   \tkzDrawPoints[new](M,M_A,M_B,M_C)
   \tkzLabelPoints[above](B)
   \tkzLabelPoints[below](A,C,M_B)
   \tkzLabelPoints[right](M_C)
   \tkzLabelPoints[left](M_A)
   \tkzLabelPoints[font=\scriptsize](M)
  \end{tikzpicture}
\end{tkzexample}

\subsubsection{Option \tkzname{in} or \tkzname{incentral} }

The incentral triangle is the triangle whose vertices are determined by
the intersections of the reference triangle’s angle bisectors with the
respective opposite sides.
\\
\href{http://mathworld.wolfram.com/ContactTriangle.html}{Weisstein, Eric W. "Incentral triangle" From MathWorld--A Wolfram Web Resource.}


\begin{tkzexample}[latex=7cm,small]
\begin{tikzpicture}[scale=1]
  \tkzDefPoints{ 0/0/A,5/0/B,2/3/C}
  \tkzDefSpcTriangle[in,name=I](A,B,C){_a,_b,_c}
  \tkzDefCircle[in](A,B,C) \tkzGetPoints{I}{a}
  \tkzDrawCircle(I,a)
  \tkzDrawPolygon(A,B,C)
  \tkzDrawPolygon[new](I_a,I_b,I_c)
  \tkzDrawSegments[dashed,new](A,I_a B,I_b C,I_c)
  \tkzDrawPoints(A,B,C,I,I_a,I_b,I_c) 
  \tkzLabelPoints[below](A,B,I_c)
  \tkzLabelPoints[above left](I_b)
  \tkzLabelPoints[above right](C,I_a)
\end{tikzpicture}
\end{tkzexample}

\subsubsection{Option \tkzname{ex} or \tkzname{excentral} }

The excentral triangle of a triangle $ABC$ is the triangle $J_aJ_bJ_c$ with vertices corresponding to the excenters of $ABC$.

\begin{tkzexample}[latex=7cm,small]
\begin{tikzpicture}[scale=.6]
 \tkzDefPoints{0/0/A,6/0/B,0.8/4/C}
 \tkzDefSpcTriangle[excentral,name=J](A,B,C){_a,_b,_c}
 \tkzDefSpcTriangle[extouch,name=T](A,B,C){_a,_b,_c}
 \tkzDrawPolygon(A,B,C)
 \tkzDrawPolygon[new](J_a,J_b,J_c)
 \tkzClipBB
 \tkzDrawPoints(A,B,C)
 \tkzDrawPoints[new](J_a,J_b,J_c)
 \tkzLabelPoints(A,B,C)
 \tkzLabelPoints[new](J_b,J_c)
 \tkzLabelPoints[new,above](J_a)
 \tkzDrawCircles[gray](J_a,T_a J_b,T_b J_c,T_c) 
\end{tikzpicture}
\end{tkzexample}


\subsubsection{Option \tkzname{intouch} or \tkzname{contact}}
The contact triangle of a triangle $ABC$, also called the intouch triangle, is the triangle  formed by the points of tangency of the incircle of $ABC$ with $ABC$.\\
\href{http://mathworld.wolfram.com/ContactTriangle.html}{Weisstein, Eric W. "Contact triangle" From MathWorld--A Wolfram Web Resource.}

We obtain the intersections of the bisectors with the sides.
\begin{tkzexample}[latex=7cm,small]
\begin{tikzpicture}[scale=.75]
 \tkzDefPoints{0/0/A,6/0/B,0.8/4/C}          
 \tkzDefSpcTriangle[intouch,name=X](A,B,C){_a,_b,_c}
 \tkzInCenter(A,B,C)\tkzGetPoint{I}
 \tkzDefCircle[in](A,B,C) \tkzGetPoints{I}{i}
 \tkzDrawCircle(I,i)
 \tkzDrawPolygon(A,B,C)
 \tkzDrawPolygon[new](X_a,X_b,X_c)
 \tkzDrawPoints(A,B,C)
 \tkzDrawPoints[new](X_a,X_b,X_c)
 \tkzLabelPoints[right](X_a)
 \tkzLabelPoints[left](X_b)
 \tkzLabelPoints[above](C)
 \tkzLabelPoints[below](A,B,X_c)
\end{tikzpicture} 
\end{tkzexample}

\subsubsection{Option \tkzname{extouch}}
The extouch triangle  $T_aT_bT_c$ is the triangle formed by the points of tangency of a triangle $ABC$ with its excircles $J_a$, $J_b$, and $J_c$. The points  $T_a$, $T_b$, and $T_c$ can also be constructed as the points which bisect the perimeter of $A_1A_2A_3$ starting at $A$, $B$, and $C$.\\
\href{http://mathworld.wolfram.com/ExtouchTriangle.html}{Weisstein, Eric W. "Extouch triangle" From MathWorld--A Wolfram Web Resource.}

We obtain the points of contact of the exinscribed circles as well as the triangle formed by the centers of the exinscribed circles.

\begin{tkzexample}[latex=8cm,small]
\begin{tikzpicture}[scale=.7]
\tkzDefPoints{0/0/A,6/0/B,0.8/4/C}
\tkzDefSpcTriangle[excentral,
                 name=J](A,B,C){_a,_b,_c}
\tkzDefSpcTriangle[extouch,
                  name=T](A,B,C){_a,_b,_c}
\tkzDefTriangleCenter[nagel](A,B,C)
\tkzGetPoint{N_a}
\tkzDefTriangleCenter[centroid](A,B,C)
\tkzGetPoint{G}
\tkzDrawPoints[new](J_a,J_b,J_c)
\tkzClipBB \tkzShowBB
\tkzDrawCircles[gray](J_a,T_a J_b,T_b J_c,T_c)
\tkzDrawLines[add=1 and 1](A,B B,C C,A)
\tkzDrawSegments[new](A,T_a B,T_b C,T_c)
\tkzDrawSegments[new](J_a,T_a J_b,T_b J_c,T_c)
\tkzDrawPolygon(A,B,C)
\tkzDrawPolygon[new](T_a,T_b,T_c)
\tkzDrawPoints(A,B,C,N_a)
\tkzDrawPoints[new](T_a,T_b,T_c)
\tkzLabelPoints[below left](A)
\tkzLabelPoints[below](N_a,B)
\tkzLabelPoints[above](C)
\tkzLabelPoints[new,below left](T_b)
\tkzLabelPoints[new,below right](T_c)
\tkzLabelPoints[new,right=6pt](T_a)
\tkzMarkRightAngles[fill=gray!15](J_a,T_a,B
 J_b,T_b,C J_c,T_c,A)
\end{tikzpicture}
\end{tkzexample}

\subsubsection{Option \tkzname{orthic}}

Given a triangle $ABC$, the triangle $H_AH_BH_C$ whose vertices are endpoints of the altitudes from each of the vertices of ABC is called the orthic triangle, or sometimes the altitude triangle. The three lines $AH_A$, $BH_B$, and $CH_C$ are concurrent at the orthocenter H of ABC.

\begin{tkzexample}[latex=7cm,small]
\begin{tikzpicture}[scale=.75]
\tkzDefPoints{1/5/A,0/0/B,7/0/C}
 \tkzDefSpcTriangle[orthic](A,B,C){H_A,H_B,H_C}
 \tkzDefTriangleCenter[ortho](B,C,A)
 \tkzGetPoint{H}
 \tkzDefPointWith[orthogonal,normed](H_A,B)
 \tkzGetPoint{a}
 \tkzDrawSegments[new](A,H_A B,H_B C,H_C)   
 \tkzMarkRightAngles[fill=gray!20,
         opacity=.5](A,H_A,C B,H_B,A C,H_C,A)
 \tkzDrawPolygon[fill=teal!20,opacity=.3](A,B,C)
 \tkzDrawPoints(A,B,C)
 \tkzDrawPoints[new](H_A,H_B,H_C)
 \tkzDrawPolygon[new,fill=orange!20,
                opacity=.3](H_A,H_B,H_C)
 \tkzLabelPoints(C)
 \tkzLabelPoints[left](B)
 \tkzLabelPoints[above](A)
 \tkzLabelPoints[new](H_A)
 \tkzLabelPoints[new,above left](H_C)
 \tkzLabelPoints[new,above right](H_B,H)
\end{tikzpicture}
\end{tkzexample}
    
\subsubsection{Option \tkzname{feuerbach}}
The Feuerbach triangle is the triangle formed by the three points of tangency of the nine-point circle with the excircles.\\
\href{http://mathworld.wolfram.com/FeuerbachTriangle.html}{Weisstein, Eric W. "Feuerbach triangle" From MathWorld--A Wolfram Web Resource.}

 The points of tangency define the Feuerbach triangle.

\begin{tkzexample}[latex=8cm,small]
\begin{tikzpicture}[scale=1]
  \tkzDefPoint(0,0){A}
  \tkzDefPoint(3,0){B}
  \tkzDefPoint(0.5,2.5){C}
  \tkzDefCircle[euler](A,B,C) \tkzGetPoint{N}
  \tkzDefSpcTriangle[feuerbach,
                       name=F](A,B,C){_a,_b,_c}
  \tkzDefSpcTriangle[excentral,
                       name=J](A,B,C){_a,_b,_c}
  \tkzDefSpcTriangle[extouch,
                        name=T](A,B,C){_a,_b,_c}
  \tkzLabelPoints[below left](J_a,J_b,J_c)  
  \tkzClipBB \tkzShowBB
  \tkzDrawCircle[purple](N,F_a)
  \tkzDrawPolygon(A,B,C)
  \tkzDrawPolygon[new](F_a,F_b,F_c)
  \tkzDrawCircles[gray](J_a,F_a J_b,F_b J_c,F_c)
  \tkzDrawPoints[blue](J_a,J_b,J_c,%
          F_a,F_b,F_c,A,B,C)
  \tkzLabelPoints(A,B,F_c)
  \tkzLabelPoints[above](C)    
  \tkzLabelPoints[right](F_a)
  \tkzLabelPoints[left](F_b)        
\end{tikzpicture}
\end{tkzexample}

\subsubsection{Option   \tkzname{tangential}} 
The tangential triangle is the triangle $T_aT_bT_c$ formed by the lines tangent to the circumcircle of a given triangle $ABC$ at its vertices. It is therefore antipedal triangle of $ABC$ with respect to the circumcenter $O$.\\ 
\href{http://mathworld.wolfram.com/TangentialTriangle.html}{Weisstein, Eric W. "Tangential Triangle." From MathWorld--A Wolfram Web Resource. }


\begin{tkzexample}[latex=8cm,small]
\begin{tikzpicture}[scale=.5,rotate=80]
  \tkzDefPoints{0/0/A,6/0/B,1.8/4/C}           
  \tkzDefSpcTriangle[tangential,
    name=T](A,B,C){_a,_b,_c}
  \tkzDrawPolygon(A,B,C)
  \tkzDrawPolygon[new](T_a,T_b,T_c)
  \tkzDrawPoints(A,B,C)
  \tkzDrawPoints[new](T_a,T_b,T_c)
  \tkzDefCircle[circum](A,B,C)  
  \tkzGetPoint{O} 
  \tkzDrawCircle(O,A)
  \tkzLabelPoints(A)
  \tkzLabelPoints[above](B)
  \tkzLabelPoints[left](C)
  \tkzLabelPoints[new](T_b,T_c)
  \tkzLabelPoints[new,left](T_a)
\end{tikzpicture} 
\end{tkzexample} 

\subsubsection{Option   \tkzname{euler}} 
The Euler triangle of a triangle $ABC$ is the triangle $E_AE_BE_C$ whose vertices are the midpoints of the segments joining the orthocenter $H$ with the respective vertices. The vertices of the triangle are known as the Euler points, and lie on the nine-point circle.
\\
\href{https://mathworld.wolfram.com/EulerTriangle.html}{Weisstein, Eric W. "Euler Triangle." From MathWorld--A Wolfram Web Resource.} 

\begin{tkzexample}[latex=7cm,small]
\begin{tikzpicture}[rotate=90,scale=1.25]
 \tkzDefPoints{0/0/A,6/0/B,0.8/4/C}
 \tkzDefSpcTriangle[medial,
     name=M](A,B,C){_A,_B,_C}
 \tkzDefTriangleCenter[euler](A,B,C)
     \tkzGetPoint{N} % I= N nine points
 \tkzDefTriangleCenter[ortho](A,B,C)
        \tkzGetPoint{H}
 \tkzDefMidPoint(A,H) \tkzGetPoint{E_A}
 \tkzDefMidPoint(C,H) \tkzGetPoint{E_C}
 \tkzDefMidPoint(B,H) \tkzGetPoint{E_B}
 \tkzDefSpcTriangle[ortho,name=H](A,B,C){_A,_B,_C}
 \tkzDrawPolygon(A,B,C)
 \tkzDrawCircle(N,E_A)
 \tkzDrawSegments[new](A,H_A B,H_B C,H_C)
 \tkzDrawPoints(A,B,C,N,H)
 \tkzDrawPoints[red](M_A,M_B,M_C)
 \tkzDrawPoints[blue]( H_A,H_B,H_C)
 \tkzDrawPoints[green](E_A,E_B,E_C)
 \tkzAutoLabelPoints[center=N,font=\scriptsize]%
(A,B,C,M_A,M_B,M_C,H_A,H_B,H_C,E_A,E_B,E_C)
\tkzLabelPoints[font=\scriptsize](H,N)
\tkzMarkSegments[mark=s|,size=3pt,
  color=blue,line width=1pt](B,E_B E_B,H)
   \tkzDrawPolygon[color=cyan](M_A,M_B,M_C)
\end{tikzpicture}
\end{tkzexample}

\subsubsection{Option  \tkzname{euler} and Option  \tkzname{orthic}} 
\begin{tkzexample}[vbox,small]
  \begin{tikzpicture}[scale=1.25]
    \tkzDefPoints{0/0/A,6/0/B,0.8/4/C}
    \tkzDefSpcTriangle[euler,name=E](A,B,C){a,b,c}
    \tkzDefSpcTriangle[orthic,name=H](A,B,C){a,b,c}
    \tkzDefExCircle(A,B,C) \tkzGetPoints{I}{i}
    \tkzDefExCircle(C,A,B) \tkzGetPoints{J}{j}
    \tkzDefExCircle(B,C,A) \tkzGetPoints{K}{k}
    \tkzDrawPoints[orange](I,J,K)
    \tkzLabelPoints[font=\scriptsize](A,B,C,I,J,K)
    \tkzClipBB
    \tkzInterLC(I,C)(I,i) \tkzGetSecondPoint{Fc}
    \tkzInterLC(J,B)(J,j) \tkzGetSecondPoint{Fb}
    \tkzInterLC(K,A)(K,k) \tkzGetSecondPoint{Fa}
    \tkzDrawLines[add=1.5 and 1.5](A,B A,C B,C)
    \tkzDefCircle[euler](A,B,C) \tkzGetPoints{E}{e}
    \tkzDrawCircle[orange](E,e)
    \tkzDrawSegments[orange](E,I E,J E,K)
    \tkzDrawSegments[dashed](A,Ha B,Hb C,Hc)
    \tkzDrawCircles(J,j I,i K,k)
    \tkzDrawPoints(A,B,C)
    \tkzDrawPoints[orange](E,I,J,K,Ha,Hb,Hc,Ea,Eb,Ec,Fa,Fb,Fc)
    \tkzLabelPoints[font=\scriptsize](E,Ea,Eb,Ec,Ha,Hb,Hc,Fa,Fb,Fc)  
  \end{tikzpicture}
\end{tkzexample}

\subsubsection{Option \tkzname{symmedial}}
The symmedial triangle$ K_AK_BK_C$ is the triangle whose vertices are the intersection points of the symmedians with the reference triangle $ABC$. 

\begin{tkzexample}[latex=7cm,small]
\begin{tikzpicture}
\tkzDefPoint(0,0){A}
\tkzDefPoint(5,0){B}
\tkzDefPoint(.75,4){C}
\tkzDefTriangleCenter[symmedian](A,B,C)\tkzGetPoint{K} 
\tkzDefSpcTriangle[symmedial,name=K_](A,B,C){A,B,C}
\tkzDrawPolygon(A,B,C)
\tkzDrawSegments[new](A,K_A B,K_B C,K_C)
\tkzDrawPoints(A,B,C,K,K_A,K_B,K_C)
\tkzLabelPoints(A,B,K,K_C)
\tkzLabelPoints[above](C)
\tkzLabelPoints[right](K_A)
\tkzLabelPoints[left](K_B)
\end{tikzpicture}
\end{tkzexample}

\subsection{Permutation of two points of a triangle}

\begin{NewMacroBox}{tkzPermute}{\parg{$pt1$,$pt2$,$pt3$}}%
\begin{tabular}{lll}%
arguments             & example & explanation                         \\
\midrule
\TAline{(pt1,pt2,pt3)} {\tkzcname{tkzPermute}(A,B,C)}{$A$, $\widehat{B,A,C}$ are unchanged, $B$, $C$ exchange their position}
\midrule
\end{tabular}

\medskip
\emph{The triangle is unchanged.}
\end{NewMacroBox}

\subsubsection{Modification of the \tkzname{school} triangle}
This triangle is constructed from the segment $[AB]$ on $[A,x)$.

If we want the segment $[AC]$ to be on $[A,x)$, we just have to swap $B$ and $C$.

\begin{tkzexample}[latex=7cm,small]
\begin{tikzpicture}
  \tkzDefPoints{0/0/A,4/0/B,6/0/x}
  \tkzDefTriangle[school](A,B)  
  \tkzGetPoint{C}
  \tkzPermute(A,B,C)
  \tkzDrawSegments(A,B C,x)
  \tkzDrawSegments(A,C B,C)
  \tkzDrawPoints(A,B,C)
  \tkzLabelPoints(A,C,x)
  \tkzLabelPoints[above](B)
  \tkzMarkRightAngles(C,B,A)
\end{tikzpicture}
\end{tkzexample}

Remark: Only the first point is unchanged. The order of the last two parameters is not important.

\endinput
\section{Definition of polygons}
\subsection{Defining the points of a square} \label{def_square}
We have seen the definitions of some triangles. Let us look at the definitions of some quadrilaterals and regular polygons.

\begin{NewMacroBox}{tkzDefSquare}{\parg{pt1,pt2}}%
The square is defined in the forward direction. From two points, two more points are obtained such that the four taken in order form a square. The square is defined in the forward direction.    The results are in \tkzname{tkzFirstPointResult} and \tkzname{tkzSecondPointResult}.\\
We can rename them with \tkzcname{tkzGetPoints}.

\medskip
\begin{tabular}{lll}%
\toprule
Arguments             & example & explication                         \\ 
\midrule
\TAline{\parg{pt1,pt2}}{\tkzcname{tkzDefSquare}\parg{A,B}}{The square is defined in the direct direction.}
\end{tabular}
\end{NewMacroBox}

\subsubsection{Using \tkzcname{tkzDefSquare} with two points}
Note the inversion of the first two points and the result.

\begin{tkzexample}[latex=4cm,small]
\begin{tikzpicture}[scale=.5]
  \tkzDefPoint(0,0){A} \tkzDefPoint(3,0){B}
  \tkzDefSquare(A,B)
  \tkzDrawPolygon[color=red](A,B,tkzFirstPointResult,%
               tkzSecondPointResult)
  \tkzDefSquare(B,A)
  \tkzDrawPolygon[color=blue](B,A,tkzFirstPointResult,%
               tkzSecondPointResult) 
\end{tikzpicture} 
\end{tkzexample}

 We may only need one point to draw an isosceles right-angled triangle so we use \tkzcname{tkzGetFirstPoint} or \tkzcname{tkzGetSecondPoint}.

\subsubsection{Use of \tkzcname{tkzDefSquare} to obtain an isosceles right-angled triangle}
\begin{tkzexample}[latex=7cm,small]
\begin{tikzpicture}[scale=1]
  \tkzDefPoint(0,0){A}
  \tkzDefPoint(3,0){B}
  \tkzDefSquare(A,B) \tkzGetFirstPoint{C}
  \tkzDrawPolygon[color=blue,fill=blue!30](A,B,C)
\end{tikzpicture}
\end{tkzexample}

\subsubsection{Pythagorean Theorem and \tkzcname{tkzDefSquare} }
\begin{tkzexample}[latex=8cm,small]
\begin{tikzpicture}[scale=.5]
\tkzInit
\tkzDefPoint(0,0){C}
\tkzDefPoint(4,0){A}
\tkzDefPoint(0,3){B} 
\tkzDefSquare(B,A)\tkzGetPoints{E}{F} 
\tkzDefSquare(A,C)\tkzGetPoints{G}{H} 
\tkzDefSquare(C,B)\tkzGetPoints{I}{J} 
\tkzFillPolygon[fill = red!50 ](A,C,G,H) 
\tkzFillPolygon[fill = blue!50 ](C,B,I,J) 
\tkzFillPolygon[fill = purple!50](B,A,E,F) 
\tkzFillPolygon[fill = orange,opacity=.5](A,B,C) 
\tkzDrawPolygon[line width = 1pt](A,B,C) 
\tkzDrawPolygon[line width = 1pt](A,C,G,H) 
\tkzDrawPolygon[line width = 1pt](C,B,I,J) 
\tkzDrawPolygon[line width = 1pt](B,A,E,F) 
\tkzLabelSegment[](A,C){$a$} 
\tkzLabelSegment[](C,B){$b$} 
\tkzLabelSegment[swap](A,B){$c$} 
\end{tikzpicture}
\end{tkzexample}

\subsection{Definition of parallelogram} 

\subsection{Defining the points of a parallelogram} 
It is a matter of completing three points in order to obtain a parallelogram.
\begin{NewMacroBox}{tkzDefParallelogram}{\parg{pt1,pt2,pt3}}%
From three points, another point is obtained such that the four taken in order form a parallelogram.  The result is in \tkzname{tkzPointResult}. \\
We can rename it with the name \tkzcname{tkzGetPoint}...

\begin{tabular}{lll}%
\toprule
arguments &  default & definition  \\ 
\midrule
\TAline{\parg{pt1,pt2,pt3}}{no default}{Three points are necessary}
\bottomrule
\end{tabular}
\end{NewMacroBox}

\subsubsection{Example of a parallelogram definition}

\begin{tkzexample}[latex=7 cm,small]
\begin{tikzpicture}[scale=1]
 \tkzDefPoints{0/0/A,3/0/B,4/2/C} 
 \tkzDefParallelogram(A,B,C) 
 \tkzGetPoint{D}
 \tkzDrawPolygon(A,B,C,D)
 \tkzLabelPoints(A,B) 
 \tkzLabelPoints[above right](C,D)
 \tkzDrawPoints(A,...,D)
\end{tikzpicture}
\end{tkzexample}



\subsubsection{Simple example}
Explanation of the definition of a parallelogram
\begin{tkzexample}[latex=7 cm,small]
\begin{tikzpicture}[scale=1]
  \tkzDefPoints{0/0/A,3/0/B,4/2/C} 
  \tkzDefPointWith[colinear= at C](B,A) 
  \tkzGetPoint{D}
  \tkzDrawPolygon(A,B,C,D)
  \tkzLabelPoints(A,B) 
  \tkzLabelPoints[above right](C,D)
  \tkzDrawPoints(A,...,D)
\end{tikzpicture}
\end{tkzexample}

\subsubsection{Construction of the golden rectangle }

\begin{tkzexample}[latex=8cm,small]
\begin{tikzpicture}[scale=.5]
  \tkzInit[xmax=14,ymax=10]
  \tkzClip[space=1]
  \tkzDefPoint(0,0){A}
  \tkzDefPoint(8,0){B}
  \tkzDefMidPoint(A,B)\tkzGetPoint{I}
  \tkzDefSquare(A,B)\tkzGetPoints{C}{D}
  \tkzDrawSquare(A,B)
  \tkzInterLC(A,B)(I,C)\tkzGetPoints{G}{E}
  \tkzDrawArc[style=dashed,color=gray](I,E)(D)
  \tkzDefPointWith[colinear= at C](E,B)
  \tkzGetPoint{F}
  \tkzDrawPoints(C,D,E,F)
  \tkzLabelPoints(A,B,C,D,E,F)
  \tkzDrawSegments[style=dashed,color=gray]%
(E,F C,F B,E)  
\end{tikzpicture}
\end{tkzexample}




\subsection{Drawing a square} 
\begin{NewMacroBox}{tkzDrawSquare}{\oarg{local options}\parg{pt1,pt2}}%
The macro draws a square but not the vertices. It is possible to color the inside. The order of the points is that of the direct direction of the trigonometric circle.

\medskip
\begin{tabular}{lll}%
\toprule
arguments             & example & explication                         \\ 
\midrule
\TAline{\parg{pt1,pt2}}{|\tkzcname{tkzDrawSquare}|\parg{A,B}}{|\tkzcname{tkzGetPoints\{C\}\{D\}}|}
\bottomrule
\end{tabular}

\medskip 
\begin{tabular}{lll}%
options             & example & explication                         \\ 
\midrule
\TOline{Options TikZ}{|red,line width=1pt|}{}
\end{tabular}
\end{NewMacroBox}

\subsubsection{The idea is to inscribe two squares in a semi-circle.}

\begin{tkzexample}[latex=6 cm,small]
\begin{tikzpicture}[scale=.75] 
   \tkzInit[ymax=8,xmax=8]
 \tkzClip[space=.25]    \tkzDefPoint(0,0){A}
 \tkzDefPoint(8,0){B}  \tkzDefPoint(4,0){I}
 \tkzDefSquare(A,B)    \tkzGetPoints{C}{D}
 \tkzInterLC(I,C)(I,B) \tkzGetPoints{E'}{E}
 \tkzInterLC(I,D)(I,B) \tkzGetPoints{F'}{F} 
 \tkzDefPointsBy[projection=onto A--B](E,F){H,G}
 \tkzDefPointsBy[symmetry   = center H](I){J}
 \tkzDefSquare(H,J)    \tkzGetPoints{K}{L}
 \tkzDrawSector[fill=yellow](I,B)(A)
 \tkzFillPolygon[color=red!40](H,E,F,G)
 \tkzFillPolygon[color=blue!40](H,J,K,L)
 \tkzDrawPolySeg[color=red](H,E,F,G) 
 \tkzDrawPolySeg[color=red](J,K,L)
 \tkzDrawPoints(E,G,H,F,J,K,L)
\end{tikzpicture}
\end{tkzexample}

\subsection{The golden rectangle} 
 \begin{NewMacroBox}{tkzDefGoldRectangle}{\parg{point,point}}%
The macro determines a rectangle whose size ratio is the number $\Phi$. The created points are in \tkzname{tkzFirstPointResult} and \tkzname{tkzSecondPointResult}. They can be obtained with the macro \tkzcname{tkzGetPoints}. The following macro is used to draw the rectangle.

\begin{tabular}{lll}%
\toprule
arguments             & example & explication                         \\
\midrule
\TAline{\parg{pt1,pt2}}{\parg{A,B}}{If C and D are created then $AB/BC=\Phi$.}
 \end{tabular}
\end{NewMacroBox}

 \begin{NewMacroBox}{tkzDrawGoldRectangle}{\oarg{local options}\parg{point,point}}
\begin{tabular}{lll}%
arguments             & example & explication                         \\
\midrule
\TAline{\parg{pt1,pt2}}{\parg{A,B}}{Draws the golden rectangle based on the segment $[AB]$}
\end{tabular}

\medskip 
\begin{tabular}{lll}%
options     & example & explication     \\ 
\midrule
\TOline{Options TikZ}{|red,line width=1pt|}{}
\end{tabular} 
\end{NewMacroBox}

\subsubsection{Golden Rectangles}
\begin{tkzexample}[latex=6 cm,small]
\begin{tikzpicture}[scale=.6]
 \tkzDefPoint(0,0){A}      \tkzDefPoint(8,0){B}
 \tkzDefGoldRectangle(A,B) \tkzGetPoints{C}{D}
 \tkzDefGoldRectangle(B,C) \tkzGetPoints{E}{F}
 \tkzDrawPolygon[color=red,fill=red!20](A,B,C,D)
 \tkzDrawPolygon[color=blue,fill=blue!20](B,C,E,F)
\end{tikzpicture}
\end{tkzexample}

\subsection{Drawing a polygon} 
 \begin{NewMacroBox}{tkzDrawPolygon}{\oarg{local options}\parg{points list}}%
Just give a list of points and the macro plots the polygon using the \TIKZ\ options present. You can  replace $(A,B,C,D,E)$ by $(A,...,E)$ and $(P_1,P_2,P_3,P_4,P_5)$ by $(P_1,P...,P_5)$

\begin{tabular}{lll}%
\toprule
arguments             & example & explication                         \\
\midrule
\TAline{\parg{pt1,pt2,pt3,...}}{|\BS tkzDrawPolygon[gray,dashed](A,B,C)|}{Drawing a triangle}
 \end{tabular}

\medskip
\begin{tabular}{lll}%
\toprule
options             & default & example                         \\
\midrule
\TOline{Options TikZ}{...}{|\BS tkzDrawPolygon[red,line width=2pt](A,B,C)|}
 \end{tabular} 
\end{NewMacroBox}

\subsubsection{\tkzcname{tkzDrawPolygon}}

\begin{tkzexample}[latex=7cm, small]  
\begin{tikzpicture} [rotate=18,scale=1.5]
 \tkzDefPoint(0,0){A}
 \tkzDefPoint(2.25,0.2){B}
 \tkzDefPoint(2.5,2.75){C}
 \tkzDefPoint(-0.75,2){D}
 \tkzDrawPolygon[fill=black!50!blue!20!](A,B,C,D)
 \tkzDrawSegments[style=dashed](A,C B,D) 
\end{tikzpicture}\end{tkzexample}

\subsection{Drawing a polygonal chain} 
 \begin{NewMacroBox}{tkzDrawPolySeg}{\oarg{local options}\parg{points list}}%
Just give a list of points and the macro plots the polygonal chain using the \TIKZ\ options present.

\begin{tabular}{lll}%
\toprule
arguments             & example & explication                         \\
\midrule
\TAline{\parg{pt1,pt2,pt3,...}}{|\BS tkzDrawPolySeg[gray,dashed](A,B,C)|}{Drawing a triangle}
 \end{tabular}

\medskip
\begin{tabular}{lll}%
\toprule
options             & default & example                         \\
\midrule
\TOline{Options TikZ}{...}{|\BS tkzDrawPolySeg[red,line width=2pt](A,B,C)|}
 \end{tabular} 
\end{NewMacroBox}

\subsubsection{Polygonal chain}

\begin{tkzexample}[latex=7cm, small]  
\begin{tikzpicture}
 \tkzDefPoints{0/0/A,6/0/B,3/4/C,2/2/D}					 
 \tkzDrawPolySeg(A,...,D)
 \tkzDrawPoints(A,...,D)
\end{tikzpicture}
\end{tkzexample}

\subsubsection{Polygonal chain: index notation}

\begin{tkzexample}[latex=7cm, small]  
\begin{tikzpicture}
\foreach \pt in {1,2,...,8}	{%
\tkzDefPoint(\pt*20:3){P_\pt}}		 
\tkzDrawPolySeg(P_1,P_...,P_8)
\tkzDrawPoints(P_1,P_...,P_8)
\end{tikzpicture}
\end{tkzexample}

\subsection{Clip a polygon} 
 \begin{NewMacroBox}{tkzClipPolygon}{\oarg{local options}\parg{points list}}%
This macro makes it possible to contain the different plots in the designated polygon.

\medskip
\begin{tabular}{lll}%
\toprule
arguments       & example & explication     \\ 
\midrule
\TAline{\parg{pt1,pt2}}{\parg{A,B}}{}
%\bottomrule
 \end{tabular}
\end{NewMacroBox}

\subsubsection{\tkzcname{tkzClipPolygon}} 
\begin{tkzexample}[latex=7 cm,small]
\begin{tikzpicture}[scale=1.25]
 \tkzInit[xmin=0,xmax=4,ymin=0,ymax=3] 
 \tkzClip[space=.5] 
 \tkzDefPoint(0,0){A} \tkzDefPoint(4,0){B}
 \tkzDefPoint(1,3){C} \tkzDrawPolygon(A,B,C)
 \tkzDefPoint(0,2){D}  \tkzDefPoint(2,0){E}
 \tkzDrawPoints(D,E) \tkzLabelPoints(D,E) 
 \tkzClipPolygon(A,B,C)
 \tkzDrawLine[color=red](D,E)
\end{tikzpicture}
\end{tkzexample}

\subsubsection{Example: use of "Clip" for Sangaku in a square} 
\begin{tkzexample}[latex=7cm, small]  
\begin{tikzpicture}[scale=.75]
 \tkzDefPoint(0,0){A} \tkzDefPoint(8,0){B}
 \tkzDefSquare(A,B) \tkzGetPoints{C}{D}
 \tkzDrawPolygon(B,C,D,A)
 \tkzClipPolygon(B,C,D,A)
 \tkzDefPoint(4,8){F}
 \tkzDefTriangle[equilateral](C,D) 
 \tkzGetPoint{I}
 \tkzDrawPoint(I)
 \tkzDefPointBy[projection=onto B--C](I) 
 \tkzGetPoint{J}
 \tkzInterLL(D,B)(I,J)  \tkzGetPoint{K}
 \tkzDefPointBy[symmetry=center K](B) 
 \tkzGetPoint{M}
 \tkzDrawCircle(M,I)
 \tkzCalcLength(M,I)   \tkzGetLength{dMI}
 \tkzFillPolygon[color = orange](A,B,C,D)
 \tkzFillCircle[R,color = yellow](M,\dMI pt)
 \tkzFillCircle[R,color = blue!50!black](F,4 cm)%
\end{tikzpicture}
\end{tkzexample}
 
\subsection{Color a polygon} 
 \begin{NewMacroBox}{tkzFillPolygon}{\oarg{local options}\parg{points list}}%
You can color by drawing the polygon, but in this case you color the inside of the polygon without drawing it.

\medskip
\begin{tabular}{lll}%
\toprule
arguments                & example & explication                         \\ 
\midrule
\TAline{\parg{pt1,pt2,\dots}}{\parg{A,B,\dots}}{}
%\bottomrule
 \end{tabular}
\end{NewMacroBox} 

\subsubsection{\tkzcname{tkzFillPolygon}} 
\begin{tkzexample}[latex=7cm, small]  
\begin{tikzpicture}[scale=0.7]
\tkzInit[xmin=-3,xmax=6,ymin=-1,ymax=6]
\tkzDrawX[noticks]
\tkzDrawY[noticks]    
\tkzDefPoint(0,0){O}  \tkzDefPoint(4,2){A}
\tkzDefPoint(-2,6){B}
\tkzPointShowCoord[xlabel=$x$,ylabel=$y$](A)
\tkzPointShowCoord[xlabel=$x'$,ylabel=$y'$,%
                   ystyle={right=2pt}](B) 
\tkzDrawSegments[->](O,A O,B)
\tkzLabelSegment[above=3pt](O,A){$\vec{u}$}
\tkzLabelSegment[above=3pt](O,B){$\vec{v}$}
\tkzMarkAngle[fill= yellow,size=1.8cm,%
              opacity=.5](A,O,B)
\tkzFillPolygon[red!30,opacity=0.25](A,B,O)
\tkzLabelAngle[pos = 1.5](A,O,B){$\alpha$} 
\end{tikzpicture}
\end{tkzexample}

\subsection{Regular polygon} 
 \begin{NewMacroBox}{tkzDefRegPolygon}{\oarg{local options}\parg{pt1,pt2}}%
From the number of sides, depending on the options, this macro determines a regular polygon according to its center or one side.

\begin{tabular}{lll}%
\toprule
arguments             & example & explication                         \\
\midrule
\TAline{\parg{pt1,pt2}}{\parg{O,A}}{with option "center", $O$ is the center of the polygon.}
\TAline{\parg{pt1,pt2}}{\parg{A,B}}{with option "side", $[AB]$ is a side.}
 \end{tabular}

\medskip
\begin{tabular}{lll}%
\toprule
options             & default & example                         \\
\midrule
\TOline{name}{P}{The vertices are named $P1$,$P2$,\dots}
\TOline{sides}{5}{number of sides.}
\TOline{center}{center}{The first point is the center.}
\TOline{side}{center}{The two points are vertices.}
\TOline{Options TikZ}{...}{}
\end{tabular} 
\end{NewMacroBox}

\subsubsection{Option \tkzname{center}}
\begin{tkzexample}[latex=7cm, small]   
\begin{tikzpicture}
	\tkzDefPoints{0/0/P0,0/0/Q0,2/0/P1}
	\tkzDefMidPoint(P0,P1) \tkzGetPoint{Q1}
  \tkzDefRegPolygon[center,sides=7](P0,P1)
	\tkzDefMidPoint(P1,P2) \tkzGetPoint{Q1}
  \tkzDefRegPolygon[center,sides=7,name=Q](P0,Q1)
	\tkzDrawPolygon(P1,P...,P7)
	\tkzFillPolygon[gray!20](Q0,Q1,P2,Q2)
	\foreach \j in {1,...,7} {\tkzDrawSegment[black](P0,Q\j)}
\end{tikzpicture}
\end{tkzexample}

\subsubsection{Option \tkzname{side}}
\begin{tkzexample}[latex=7cm, small]   
\begin{tikzpicture}[scale=1]
    \tkzDefPoints{-4/0/A, -1/0/B}
    \tkzDefRegPolygon[side,sides=5,name=P](A,B)
    \tkzDrawPolygon[thick](P1,P...,P5)
\end{tikzpicture}
\end{tkzexample}
\endinput
\newpage
\section{Circles}

Among the following macros, one will allow you to draw a circle, which is not a real feat. To do this, you will need to know the center of the circle and either the radius of the circle or a point on the circumference. It seemed to me that the most frequent use was to draw a circle with a given center passing through a given point. This will be the default method, otherwise you will have to use the \tkzname{R} option. There are a large number of special circles, for example the circle circumscribed by a triangle.

\begin{itemize}
  \item  I have created a first macro \tkzcname{tkzDefCircle} which allows, according to a particular circle, to retrieve its center and the measurement of the radius in cm. This recovery is done with the macros \tkzcname{tkzGetPoint} and \tkzcname{tkzGetLength};
 
 \item then a macro \tkzcname{tkzDrawCircle};
 
 \item then a macro that allows you to color in a disc, but without drawing the circle \tkzcname{tkzFillCircle};
 
 \item sometimes, it is necessary for a drawing to be contained in a disk, this is the role assigned to \tkzcname{tkzClipCircle};
  
 \item  it finally remains to be able to give a label to designate a circle and if several possibilities are offered, we will see here \tkzcname{tkzLabelCircle}.
\end{itemize} 

\subsection{Characteristics of a circle: \tkzcname{tkzDefCircle}}
 
This macro allows you to retrieve the characteristics (center and radius) of certain circles.

\begin{NewMacroBox}{tkzDefCircle}{\oarg{local options}\parg{A,B} or \parg{A,B,C}}%
\tkzHandBomb\ Attention the arguments are lists of two or three points. This macro is either used in partnership with \\ \tkzcname{tkzGetPoints} to obtain the center and a point on the circle, or by using \\ \tkzname{tkzFirstPointResult} and \tkzname{tkzSecondPointResult} if it is not necessary to keep the results. You can also use  \tkzcname{tkzGetLength} to get the radius.

\medskip
\begin{tabular}{lll}%
\toprule
arguments           & example & explanation                         \\
\midrule
\TAline{\parg{pt1,pt2} or \parg{pt1,pt2,pt3}}{\parg{A,B}} {$[AB]$ is radius $A$ is the center}
\bottomrule
\end{tabular} 

\medskip
\begin{tabular}{lll}%
\toprule
options             & default & definition                         \\ 
\midrule
\TOline{R}       {circum}{circle characterized by a center and a radius} 
\TOline{diameter}{circum}{circle characterized by two points defining a diameter}
\TOline{circum}       {circum}{circle circumscribed of a triangle} 
\TOline{in}           {circum}{incircle a triangle} 
\TOline{ex}           {circum}{excircle of a  triangle}
\TOline{euler or nine}{circum}{Euler's Circle}
\TOline{spieker}      {circum}{Spieker Circle}
\TOline{apollonius}   {circum}{circle of Apollonius}
\TOline{orthogonal from} {circum}{[orthogonal from = A ](O,M)}
\TOline{orthogonal through}{circum}{[orthogonal through = A and B](O,M)} 
\TOline{K} {1}{coefficient used for a circle of Apollonius} 
 \bottomrule
\end{tabular}

\medskip
\emph{In the following examples, I draw the circles with a macro not yet presented. You may only need the center and a point on the circle. }
\end{NewMacroBox} 

\subsubsection{Example with  option \tkzname{R}}  
We obtain with the macro \tkzcname{tkzGetPoint} a point of the circle which is the East pole.

\begin{tkzexample}[latex=7cm,small]  
\begin{tikzpicture}[scale=1]
  \tkzDefPoint(3,3){C}
  \tkzDefPoint(5,5){A}
   \tkzCalcLength(A,C) \tkzGetLength{rAC}
  \tkzDefCircle[R](C,\rAC) \tkzGetPoint{B}
  \tkzDrawCircle(C,B)
  \tkzDrawSegment(C,A)
  \tkzLabelSegment[above left](C,A){$2\sqrt{2}$}
  \tkzDrawPoints(A,B,C)
  \tkzLabelPoints(A,C,B)
\end{tikzpicture} 
\end{tkzexample}     


 \subsubsection{Example  with  option \tkzname{diameter}}  
 It is simpler here to search directly for the middle of $[AB]$. The result is the center and if necessary 
\begin{tkzexample}[latex=7cm,small]  
\begin{tikzpicture}
  \tkzDefPoint(0,0){O}
  \tkzDefPoint(2,2){B}
  \tkzDefCircle[diameter](O,B) \tkzGetPoint{A}
  \tkzDrawCircle(A,B)
  \tkzDrawPoints(O,A,B)
  \tkzDrawSegment(O,B)
  \tkzLabelPoints(O,A,B)
  \tkzLabelSegment[above left](O,A){$\sqrt{2}$}
  \tkzLabelSegment[above left](A,B){$\sqrt{2}$}
  \tkzMarkSegments[mark=s||](O,A A,B)
\end{tikzpicture}
\end{tkzexample}    

\subsubsection{Circles inscribed and circumscribed for a given triangle} 
 
\begin{tkzexample}[latex=7cm,small]  
\begin{tikzpicture}[scale=.75]
 \tkzDefPoint(2,2){A}  \tkzDefPoint(5,-2){B}
 \tkzDefPoint(1,-2){C}
 \tkzDefCircle[in](A,B,C)
 \tkzGetPoints{I}{x}   
 \tkzDefCircle[circum](A,B,C)
 \tkzGetPoint{K}  
 \tkzDrawCircles[new](I,x K,A) 
 \tkzLabelPoints[below](B,C)
 \tkzLabelPoints[above left](A,I,K)
 \tkzDrawPolygon(A,B,C)
 \tkzDrawPoints(A,B,C,I,K) 
\end{tikzpicture} 
\end{tkzexample}

\subsubsection{Example with option \tkzname{ex}}
We want to define an excircle of a  triangle relatively to point $C$

\begin{tkzexample}[latex=8cm,small]
\begin{tikzpicture}[scale=.75]
  \tkzDefPoints{ 0/0/A,4/0/B,0.8/4/C}
  \tkzDefCircle[ex](B,C,A)                   
  \tkzGetPoints{J_c}{h}
  \tkzDefPointBy[projection=onto A--C ](J_c)   
  \tkzGetPoint{X_c}
  \tkzDefPointBy[projection=onto A--B ](J_c)   
  \tkzGetPoint{Y_c}     
  \tkzDefCircle[in](A,B,C)    
  \tkzGetPoints{I}{y}
  \tkzDrawCircles[color=lightgray](J_c,h I,y)
  \tkzDefPointBy[projection=onto A--C ](I)
  \tkzGetPoint{F}
  \tkzDefPointBy[projection=onto A--B ](I)
  \tkzGetPoint{D}
  \tkzDrawPolygon(A,B,C)
  \tkzDrawLines[add=0 and 1.5](C,A C,B)
  \tkzDrawSegments(J_c,X_c I,D  I,F J_c,Y_c)
  \tkzMarkRightAngles(A,F,I B,D,I J_c,X_c,A J_c,Y_c,B)
  \tkzDrawPoints(B,C,A,I,D,F,X_c,J_c,Y_c)
  \tkzLabelPoints(B,A,J_c,I,D)
  \tkzLabelPoints[above](Y_c)
  \tkzLabelPoints[left](X_c)
  \tkzLabelPoints[above left](C)
  \tkzLabelPoints[left](F)
\end{tikzpicture}  
\end{tkzexample}

\subsubsection{Euler's circle for a given triangle with option \tkzname{euler}}
 
We verify that this circle passes through the middle of each side.
\begin{tkzexample}[latex=6cm,small]  
\begin{tikzpicture}[scale=.75]
   \tkzDefPoint(5,3.5){A} 
   \tkzDefPoint(0,0){B} \tkzDefPoint(7,0){C}
   \tkzDefCircle[euler](A,B,C)
   \tkzGetPoints{E}{e}
   \tkzDefSpcTriangle[medial](A,B,C){M_a,M_b,M_c}
   \tkzDrawCircle[new](E,e)
   \tkzDrawPoints(A,B,C,E,M_a,M_b,M_c)    
   \tkzDrawPolygon(A,B,C)    
   \tkzLabelPoints[below](B,C)  
   \tkzLabelPoints[left](A,E)   
\end{tikzpicture}
\end{tkzexample}

\subsubsection{Apollonius circles for a given segment option \tkzname{apollonius}} 
 
\begin{tkzexample}[latex=9cm,small]    
\begin{tikzpicture}[scale=0.75]
  \tkzDefPoint(0,0){A} 
  \tkzDefPoint(4,0){B}
  \tkzDefCircle[apollonius,K=2](A,B)
  \tkzGetPoints{K1}{x}
  \tkzDrawCircle[color = teal!50!black,
      fill=teal!20,opacity=.4](K1,x)
  \tkzDefCircle[apollonius,K=3](A,B)
  \tkzGetPoints{K2}{y}
  \tkzDrawCircle[color=orange!50,
      fill=orange!20,opacity=.4](K2,y) 
  \tkzLabelPoints[below](A,B,K1,K2)
  \tkzDrawPoints(A,B,K1,K2) 
  \tkzDrawLine[add=.2 and 1](A,B)  
\end{tikzpicture}
\end{tkzexample}  

 \subsubsection{Circles exinscribed to a given triangle option \tkzname{ex}}
 You can also get the center and the projection of it on one side of the triangle. 
 
 with \tkzcname{tkzGetFirstPoint\{Jb\}} and \tkzcname{tkzGetSecondPoint\{Tb\}}.
 
\begin{tkzexample}[latex=8cm,small]  
\begin{tikzpicture}[scale=.6]
  \tkzDefPoint(0,0){A}
  \tkzDefPoint(3,0){B}
  \tkzDefPoint(1,2.5){C}
  \tkzDefCircle[ex](A,B,C) \tkzGetPoints{I}{i}
  \tkzDefCircle[ex](C,A,B) \tkzGetPoints{J}{j}
  \tkzDefCircle[ex](B,C,A) \tkzGetPoints{K}{k}
  \tkzDefCircle[in](B,C,A) \tkzGetPoints{O}{o}
  \tkzDrawCircles[new](J,j I,i K,k O,o) 
  \tkzDrawLines[add=1.5 and 1.5](A,B A,C B,C)
  \tkzDrawPolygon[purple](I,J,K)
  \tkzDrawSegments[new](A,K B,J C,I)
  \tkzDrawPoints(A,B,C)
  \tkzDrawPoints[new](I,J,K)   
  \tkzLabelPoints(A,B,C,I,J,K)
\end{tikzpicture}
\end{tkzexample}
 
\subsubsection{Spieker circle with option \tkzname{spieker}}   
The incircle of the medial triangle $M_aM_bM_c$ is the Spieker circle:

\begin{tkzexample}[latex=6cm, small]
\begin{tikzpicture}[scale=1]
  \tkzDefPoints{ 0/0/A,4/0/B,0.8/4/C}
   \tkzDefSpcTriangle[medial](A,B,C){M_a,M_b,M_c}
   \tkzDefTriangleCenter[spieker](A,B,C) 
   \tkzGetPoint{S_p}
   \tkzDrawPolygon(A,B,C)
   \tkzDrawPolygon[cyan](M_a,M_b,M_c)
   \tkzDrawPoints(B,C,A)
   \tkzDefCircle[spieker](A,B,C)
   \tkzDrawPoints[new](M_a,M_b,M_c,S_p)
   \tkzDrawCircle[new](tkzFirstPointResult,tkzSecondPointResult)
   \tkzLabelPoints[right](M_a)
   \tkzLabelPoints[left](M_b)
   \tkzLabelPoints[below](A,B,M_c,S_p)
   \tkzLabelPoints[above](C)
\end{tikzpicture}
\end{tkzexample}
 
\subsection{Projection of excenters}

\begin{NewMacroBox}{tkzDefProjExcenter}{\oarg{local options}\parg{A,B,C}\parg{a,b,c}\marg{X,Y,Z}}%
Each excenter has three projections on the sides of the triangle ABC. We can do this with one macro\\ \tkzcname{tkzDefProjExcenter[name=J](A,B,C)(a,b,c)\{Y,Z,X\}}.

\medskip
\begin{tabular}{lll}%
\toprule
options             & default & definition                        \\
\midrule
\TOline{name} {no defaut}{used to name the vertices}
\bottomrule
\end{tabular}

\begin{tabular}{lll}%
arguments & default & definition \\
\midrule
\TAline{(pt1=$\alpha_1$,pt2=$\alpha_2$,\dots)}{no default}{Each point has a assigned weight}
\end{tabular}

\medskip
\end{NewMacroBox}

\subsubsection{\tkzname{Excircles}}

\begin{tkzexample}[vbox,small]
\begin{tikzpicture}[scale=.6]
\tikzset{line style/.append style={line width=.2pt}}
\tikzset{label style/.append style={color=teal,font=\footnotesize}}
\tkzDefPoints{0/0/A,5/0/B,0.8/4/C}
\tkzDefSpcTriangle[excentral,name=J](A,B,C){a,b,c} 
\tkzDefSpcTriangle[intouch,name=I](A,B,C){a,b,c}
\tkzDefProjExcenter[name=J](A,B,C)(a,b,c){X,Y,Z}
\tkzDefCircle[in](A,B,C)   \tkzGetPoint{I} \tkzGetSecondPoint{T}  
\tkzDrawCircles[red](Ja,Xa Jb,Yb Jc,Zc)
\tkzDrawCircle(I,T) 
\tkzDrawPolygon[dashed,color=blue](Ja,Jb,Jc)
\tkzDrawLines[add=1.5 and 1.5](A,C A,B B,C)
\tkzDrawSegments(Ja,Xa Ja,Ya Ja,Za
                 Jb,Xb Jb,Yb Jb,Zb
                 Jc,Xc Jc,Yc Jc,Zc
                 I,Ia I,Ib I,Ic)
\tkzMarkRightAngles[size=.2,fill=gray!15](Ja,Za,B Ja,Xa,B Ja,Ya,C Jb,Yb,C)
\tkzMarkRightAngles[size=.2,fill=gray!15](Jb,Zb,B Jb,Xb,C Jc,Yc,A Jc,Zc,B Jc,Xc,C I,Ia,B I,Ib,C I,Ic,A)
\tkzDrawSegments[blue](Jc,C Ja,A Jb,B)
\tkzDrawPoints(A,B,C,Xa,Xb,Xc,Ja,Jb,Jc,Ia,Ib,Ic,Ya,Yb,Yc,Za,Zb,Zc)
\tkzLabelPoints(A,Ya,Yb,Ja,I)
\tkzLabelPoints[left](Jb,Ib,Yc)
\tkzLabelPoints[below](Zb,Ic,Jc,B,Za,Xa)
\tkzLabelPoints[above right](C,Zc,Yb)
\tkzLabelPoints[right](Xb,Ia,Xc)
\end{tikzpicture}
\end{tkzexample}
 
\subsubsection{\tkzname{Orthogonal from}}
Orthogonal circle of given center. \tkzcname{tkzGetPoints\{z1\}\{z2\}} gives two points of the circle.

\begin{tkzexample}[latex=7cm,small]
\begin{tikzpicture}[scale=.75]
  \tkzDefPoints{0/0/O,1/0/A}
  \tkzDefPoints{1.5/1.25/B,-2/-3/C}
  \tkzDefCircle[orthogonal from=B](O,A)
  \tkzGetPoints{z1}{z2}
  \tkzDefCircle[orthogonal from=C](O,A)
  \tkzGetPoints{t1}{t2}
  \tkzDrawCircle(O,A)
  \tkzDrawCircles[new](B,z1 C,t1)
  \tkzDrawPoints(t1,t2,C)
  \tkzDrawPoints(z1,z2,O,A,B)
  \tkzLabelPoints[right](O,A,B,C)
\end{tikzpicture}
\end{tkzexample}

\subsubsection{\tkzname{Orthogonal through}}
Orthogonal circle passing through two given points.
\begin{tkzexample}[latex=6cm,small]
\begin{tikzpicture}[scale=1]
  \tkzDefPoint(0,0){O}
  \tkzDefPoint(1,0){A}
  \tkzDrawCircle(O,A)
  \tkzDefPoint(-1.5,-1.5){z1}
  \tkzDefPoint(1.5,-1.25){z2}
  \tkzDefCircle[orthogonal through=z1 and z2](O,A)
   \tkzGetPoint{c}
  \tkzDrawCircle[new](tkzPointResult,z1)
  \tkzDrawPoints[new](O,A,z1,z2,c)
  \tkzLabelPoints[right](O,A,z1,z2,c)
\end{tikzpicture}
\end{tkzexample}

\endinput
\subsection{Definition of circle by transformation; \tkzcname{tkzDefCircleBy} }
These transformations are:

\begin{itemize}
   \item translation;
   \item homothety;
   \item orthogonal reflection or symmetry;
   \item central symmetry;
   \item orthogonal projection;
   \item rotation (degrees);
   \item inversion.
\end{itemize}

The choice of transformations is made through the options. The macro is \tkzcname{tkzDefCircleBy} and the other for the transformation of a list of points \tkzcname{tkzDefCirclesBy}. For example, we'll write:
\begin{tkzltxexample}[]
\tkzDefCircleBy[translation= from A to A'](O,M) 
\end{tkzltxexample}
$O$ is the center and $M$ is a point on the circle.
The image is a circle. The new center is |tkzFirstPointResult| and |tkzSecondPointResult| is a point on the new circle. You can get the results with the macro \tkzcname{tkzGetPoints}.
\medskip
\begin{NewMacroBox}{tkzDefCircleBy}{\oarg{local options}\parg{pt1,pt2}}%
The argument is a couple of points. The results is a couple of points. If you want to keep these points then the macro \tkzcname{tkzGetPoints\{O'\}\{M'\}} allows you to assign the name \tkzname{O'} to the center and \tkzname{M'} to the point on the circle.

\begin{tabular}{lll}%
\toprule
arguments &  definition & examples               \\ 
\midrule
\TAline{pt1,pt2}   {existing points}   {$(O,M)$}
\bottomrule
\end{tabular}

\begin{tabular}{lll}%
options     &     & examples                         \\ 
\midrule
\TOline{translation}{= from \#1 to \#2}{[translation=from A to B](O,M)}
\TOline{homothety}  {= center \#1 ratio \#2}{[homothety=center A ratio .5](O,M)}
\TOline{reflection} {= over \#1--\#2}{[reflection=over A--B](O,M)}
\TOline{symmetry }  {= center \#1}{[symmetry=center A](O,M)}
\TOline{projection }{= onto \#1--\#2}{[projection=onto A--B](O,M)}
\TOline{rotation }  {= center \#1 angle \#2}{[rotation=center O angle 30](O,M)}
\TOline{inversion}{= center \#1 through \#2}{[inversion =center O through A](O,M)} 
% \TOline{inversion negative}{= center \#1 through \#2}{[inversion negative =center O through A](O,M)}
\bottomrule
\end{tabular}

\medskip
\emph{The image is only defined and not drawn.}
\end{NewMacroBox} 

\subsubsection{\tkzname{Translation}}
\begin{tkzexample}[latex=7cm,small]
\begin{tikzpicture}[>=latex] 
 \tkzDefPoint(0,0){A}  \tkzDefPoint(3,1){B}
 \tkzDefPoint(3,2){C}   \tkzDefPoint(4,3){D}
 \tkzDefCircleBy[translation= from B to A](C,D) 
 \tkzGetPoints{C'}{D'} 
 \tkzDrawPoints[teal](A,B,C,D,C',D')
 \tkzDrawSegments[orange,->](A,B)
 \tkzDrawCircles(C,D C',D')
 \tkzLabelPoints[color=teal](A,B,C,C') 
 \tkzLabelPoints[color=teal,above](D,D') 
\end{tikzpicture} 
\end{tkzexample}

\subsubsection{\tkzname{Reflection} (orthogonal symmetry)}

\begin{tkzexample}[latex=7cm,small]
\begin{tikzpicture}[>=latex] 
 \tkzDefPoint(0,0){A}  \tkzDefPoint(3,1){B}
 \tkzDefPoint(3,2){C}   \tkzDefPoint(4,3){D}
 \tkzDefCircleBy[reflection = over A--B](C,D)
 \tkzGetPoints{C'}{D'} 
 \tkzDrawPoints[teal](A,B,C,D,C',D')
 \tkzDrawLine[add =0 and 1][orange](A,B)
 \tkzDrawCircles(C,D C',D')
 \tkzLabelPoints[color=teal](A,B,C,C') 
 \tkzLabelPoints[color=teal,right](D,D') 
\end{tikzpicture} 
\end{tkzexample}

\subsubsection{\tkzname{Homothety}}

\begin{tkzexample}[latex=7cm,small]
\begin{tikzpicture}[scale=1.2]
 \tkzDefPoint(0,0){A}   \tkzDefPoint(3,1){B}
 \tkzDefPoint(3,2){C}   \tkzDefPoint(4,3){D}
 \tkzDefCircleBy[homothety=center A ratio .5](C,D)
 \tkzGetPoints{C'}{D'}
 \tkzDrawPoints[teal](A,C,D,C',D')
 \tkzDrawCircles(C,D C',D')
 \tkzLabelPoints[color=teal](A,C,C')
 \tkzLabelPoints[color=teal,right](D,D') 
\end{tikzpicture}
\end{tkzexample}

\subsubsection{\tkzname{Symmetry}}
\begin{tkzexample}[latex=7cm,small]
\begin{tikzpicture}[scale=1]
 \tkzDefPoint(0,0){A}   \tkzDefPoint(3,1){B}
 \tkzDefPoint(3,2){C}   \tkzDefPoint(4,3){D}
 \tkzDefCircleBy[symmetry=center B](C,D)
 \tkzGetPoints{C'}{D'}
 \tkzDrawPoints[teal](B,C,D,C',D')
 \tkzDrawLines[orange](C,C' D,D')
 \tkzDrawCircles(C,D C',D')
 \tkzLabelPoints[color=teal](A,C,C')
 \tkzLabelPoints[color=teal,above](D)
 \tkzLabelPoints[color=teal,below](D')
\end{tikzpicture}
\end{tkzexample}

\subsubsection{\tkzname{Rotation}}
\begin{tkzexample}[latex=7cm,small]
\begin{tikzpicture}[scale=0.5]
 \tkzDefPoint(3,-1){B}
 \tkzDefPoint(3,2){C}   \tkzDefPoint(4,3){D}
 \tkzDefCircleBy[rotation=center B angle 90](C,D)
 \tkzGetPoints{C'}{D'}
 \tkzDrawPoints[teal](B,C,D,C',D')
 \tkzLabelPoints[color=teal](B,C,D,C',D')
 \tkzDrawCircles(C,D C',D')
\end{tikzpicture}
\end{tkzexample}

\subsubsection{\tkzname{Inversion}}

\begin{tkzexample}[latex=7cm,small]
\begin{tikzpicture}[scale=1.5]
\tkzSetUpPoint[size=3,color=red,fill=red!20]
\tkzSetUpStyle[color=purple,ultra thin]{st1}
\tkzSetUpStyle[color=cyan,ultra thin]{st2}
\tkzDefPoint(2,0){A} \tkzDefPoint(3,0){B} 
\tkzDefPoint(3,2){C} \tkzDefPoint(4,2){D} 
\tkzDefCircleBy[inversion = center B through A](C,D) 
\tkzGetPoints{C'}{D'}
\tkzDrawPoints(A,B,C,D,C',D') 
\tkzLabelPoints(A,B,C,D,C',D')
\tkzDrawCircles(B,A)
\tkzDrawCircles[st1](C,D)
\tkzDrawCircles[st2](C',D')
\end{tikzpicture}
\end{tkzexample}

\endinput
\section{\tkzname{Intersections}}

It is possible to determine the coordinates of the points of intersection between two straight lines, a straight line and a circle, and two circles.

The associated commands have no optional arguments and the user must determine the existence of the intersection points himself.

\subsection{Intersection of two straight lines \tkzcname{tkzInterLL}}
\begin{NewMacroBox}{tkzInterLL}{\parg{$A,B$}\parg{$C,D$}}%
Defines the intersection point \tkzname{tkzPointResult} of the two lines $(AB)$ and $(CD)$. The known points are given in pairs (two per line) in brackets, and the resulting point can be retrieved with the macro \tkzcname{tkzDefPoint}.
\end{NewMacroBox}

\subsubsection{Example of intersection between two straight lines}

\begin{tkzexample}[latex=7cm,small]
\begin{tikzpicture}[rotate=-45,scale=.75]
  \tkzDefPoint(2,1){A}   
  \tkzDefPoint(6,5){B}
  \tkzDefPoint(3,6){C}   
  \tkzDefPoint(5,2){D}
  \tkzDrawLines(A,B C,D)
  \tkzInterLL(A,B)(C,D)  
     \tkzGetPoint{I}
  \tkzDrawPoints[color=blue](A,B,C,D)
   \tkzDrawPoint[color=red](I)
\end{tikzpicture}
\end{tkzexample}

\subsection{Intersection of a straight line and a circle  \tkzcname{tkzInterLC}}

As before, the line is defined by a couple of points. The circle
 is also defined by a couple:
\begin{itemize}
\item $(O,C)$ which is a pair of points, the first is the center and the second is any point on the circle.
\item $(O,r)$  The $r$ measure is the radius measure.
\end{itemize}

\begin{NewMacroBox}{tkzInterLC}{\oarg{options}\parg{$A,B$}\parg{$O,C$} or \parg{$O,r$} or \parg{$O,C,D$}}%
So the arguments are two couples. 

\medskip
\begin{tabular}{lll}%
\toprule
options            & default & definition                         \\ 
\midrule
\TOline{N}         {N}    {(O,C) determines the circle}
\TOline{R}         {N}    {(O, 1 ) unit 1 cm}  
\TOline{with nodes}{N}    {(O,C,D) CD is a radius}  
\TOline{common=pt} {}     {pt is common point; tkzFirstPoint gives the other point}
\TOline{near}      {}     {tkzFirstPoint is the closest point to the first point of the line}
\bottomrule
\end{tabular}

\medskip   
The macro defines the intersection points $I$ and $J$ of the line $(AB)$ and the center circle $O$ with radius $r$ if they exist; otherwise, an error will be reported in the |.log| file. \tkzname{with nodes} avoids you to calculate the radius which is the length of $[CD]$.
If \tkzname{common} and \tkzname{near} are not used then \tkzname{tkzFirstPoint} is the smallest angle (angle with \tkzname{tkzSecondPoint}  and the center of the circle). 
\end{NewMacroBox}

\begin{NewMacroBox}{tkzTestInterLC}{\parg{$O,A$}\parg{$O',B$}}%
So the arguments are two couples which define a line and a circle  with a center and a point on the circle. If there is a non empty intersection between these the line and the circle then the test \tkzcname{iftkzFlagLC} gives true.
\end{NewMacroBox}

\subsubsection{test line-circle intersection}

\begin{tkzexample}[latex=7cm,small]
  \begin{tikzpicture}[scale=1]
    \tkzDefPoints{% x   y   name
                    3    /4    /I,
                    3    /2    /P,
                    0    /2    /La,
                    8    /3    /Lb}
  \tkzDrawCircle(I,P)
  \foreach \i in {1,...,3}{%
     \coordinate  (Lb) at (8,\i);
     \tkzDrawLine(La,Lb)
     \tkzTestInterLC(La,Lb)(I,P)
      \iftkzFlagLC
      \tkzInterLC(La,Lb)(I,P)  
      \tkzGetPoints{a}{b}
      \tkzDrawPoints(a,b)
      \fi
     }
  \end{tikzpicture}
\end{tkzexample}


\subsubsection{Line-circle intersection}

In the following example, the drawing of the circle uses two points and the intersection of the straight line and the circle uses two pairs of points. We will compare the angles $\widehat{D,E,O}$ and $\widehat{E,D,O}$. These angles are in opposite directions. \tkzname{tkzFirstPoint} is assigned to the point that forms the angle with the smallest measure (counterclockwise direction). The counterclockwide angle  $\widehat{D,E,O}$   has a measure equal to  $360\circ$ minus the measure of  $\widehat{O,E,D}$.

\begin{tkzexample}[latex=7cm,small]
\begin{tikzpicture}[scale=.75]
 \tkzInit[xmax=5,ymax=4]
 \tkzDefPoint(1,1){O} 
 \tkzDefPoint(-2,4){La} 
 \tkzDefPoint(5,0){Lb} 
 \tkzDefPoint(3,3){C}
 \tkzInterLC(La,Lb)(O,C)  \tkzGetPoints{D}{E}  
 \tkzMarkAngle[->,size=1.5](E,D,O)
 \tkzDrawPolygons[new](O,D,E)
 \tkzMarkAngle[->,size=1.5](D,E,O)
 \tkzDrawCircle(O,C)
 \tkzDrawPoints[color=teal](O,La,Lb,C)
 \tkzDrawPoints[color=red](D,E)
 \tkzDrawLine(La,Lb)
 \tkzLabelPoints[above right](O,La,Lb,C,D,E)
\end{tikzpicture} 
\end{tkzexample}

\subsubsection{Line passing through the center option \tkzname{common}}
This case is special. You cannot compare the angles. In this case, the option \tkzname{near} must be used. \tkzname{tkzFirstPoint} is assigned to the point closest to the first point given for the line. Here we want $A$ to be closest to $Lb$.

\begin{tkzexample}[latex=7cm,small]
\begin{tikzpicture}
\tkzDefPoints{% x   y   name
             0    /1    /D,
             6    /0    /B,
             3    /3    /O,
             2    /2    /La,
             5    /5    /Lb}
  \tkzDrawCircle(O,D)
  \tkzDrawLine(La,Lb)
  \tkzInterLC[near](Lb,La)(O,D)  
  \tkzGetFirstPoint{A}
  \tkzDrawSegments(O,A)
  \tkzDrawPoints(O,D,A,La,Lb)
  \tkzLabelPoints(O,D,A,La,Lb)
\end{tikzpicture}
\end{tkzexample}

\subsubsection{Line-circle intersection with option \tkzname{common}}
A special case that we often meet, a point of the line is on the circle and we are looking for the other common point.
\begin{tkzexample}[latex=7cm,small]
\begin{tikzpicture}[scale=.5]
 \tkzDefPoints{0/0/O,-5/0/A,2/-2/B,0/5/D}
 \tkzInterLC[common=A](B,A)(O,D)
 \tkzGetFirstPoint{C}
 \tkzDrawPoints(O,A,B)
 \tkzDrawCircle(O,A)
 \tkzDrawLine(A,C)
 \tkzDrawPoint(C)
 \tkzLabelPoints(A,B,C)
\end{tikzpicture}
\end{tkzexample}


\subsubsection{Line-circle intersection order of points}
The idea is to compare the angles formed with the first defining point of the line, a resultant point and the center of the circle. The first point is the one that corresponds to the smallest angle.

As you can see $\widehat{BCO} < \widehat{BEO} $. To tell the truth,$ \widehat{BEO}$ is counterclockwise.

\begin{tkzexample}[latex=6cm,small]
\begin{tikzpicture}[scale=.5]
  \tkzDefPoints{0/0/O,5/1/A,2/2/B,3/1/D}
  \tkzInterLC[common=A](B,D)(O,A) \tkzGetPoints{C}{E}
  \tkzDrawPoints(O,A,B,D)
  \tkzDrawCircle(O,A) \tkzDrawLine(E,C)
  \tkzDrawSegments[dashed](B,O O,C)
  \tkzMarkAngle[->,size=1.5](B,C,O)
  \tkzDrawSegments[dashed](O,E)
  \tkzMarkAngle[->,size=1.5](B,E,O)
  \tkzDrawPoints(C,E)
  \tkzLabelPoints[above](O,E)
  \tkzLabelPoints[right](A,B,C,D)
\end{tikzpicture}
\end{tkzexample}

\subsubsection{Example with \tkzcname{foreach}}
\begin{tkzexample}[latex=7cm,small]
\begin{tikzpicture}[scale=3,rotate=180]
\tkzDefPoint(0,1){J} 
\tkzDefPoint(0,0){O}
\foreach \i in {0,-5,-10,...,-90}{
 \tkzDefPoint({2.5*cos(\i*pi/180)},{1+2.5*sin(\i*pi/180)}){P}
 \tkzInterLC[R](P,J)(O,1)\tkzGetPoints{N}{M}
 \tkzDrawSegment[color=orange](J,N)
 \tkzDrawPoints[red](N)} 
\foreach \i in {-90,-95,...,-175,-180}{
 \tkzDefPoint({2.5*cos(\i*pi/180)},{1+2.5*sin(\i*pi/180)}){P} 
 \tkzInterLC[R](P,J)(O,1)\tkzGetPoints{N}{M}
 \tkzDrawSegment[color=orange](J,M)
 \tkzDrawPoints[red](M)}   
\end{tikzpicture} 
\end{tkzexample}

\subsubsection{Line-circle intersection with option \tkzname{near}}
$D$ is the point closest to $b$.

\begin{tkzexample}[vbox,small]
  \begin{tikzpicture}
    \tkzDefPoints{0/0/A,12/0/C}
    \tkzDefGoldenRatio(A,C)                          \tkzGetPoint{B}
    \tkzDefMidPoint(A,C)                             \tkzGetPoint{O}
    \tkzDefMidPoint(A,B)                             \tkzGetPoint{O_1}
    \tkzDefMidPoint(B,C)                             \tkzGetPoint{O_2}
    \tkzDefPointBy[rotation=center O_2 angle 90](C)  \tkzGetPoint{P}
    \tkzDefPointBy[rotation=center O_1 angle 90](B)  \tkzGetPoint{Q}
    \tkzDefPointBy[rotation=center B angle 90](C)    \tkzGetPoint{b}
    \tkzInterLC[near](b,B)(O,A)                      \tkzGetFirstPoint{D}
    \tkzInterCC(D,B)(O,C)                            \tkzGetPoints{V}{U}
    \tkzDefPointBy[projection=onto U--V](O_1)        \tkzGetPoint{M}
    \tkzDefPointBy[projection=onto U--V](O_2)        \tkzGetPoint{N}  
    \tkzDrawPoints(A,B,C,O,O_1,O_2,D,U,V,M,N,b)
    \tkzDrawSemiCircles[teal](O,C O_1,B O_2,C)
    \tkzDrawSegments(A,C B,D U,V A,D C,D M,B B,N)
    \tkzDrawArc(D,U)(V)
    \tkzLabelPoints(A,B,C,O,O_1,O_2)
    \tkzLabelPoints[above](D,U,V,M,N)
  \end{tikzpicture}
\end{tkzexample}


\subsubsection{More complex example of a line-circle intersection}
Figure from  \url{http://gogeometry.com/problem/p190_tangent_circle}

\begin{tkzexample}[latex=7cm,small]
\begin{tikzpicture}[scale=.75]
 \tkzDefPoint(0,0){A}  
 \tkzDefPoint(8,0){B}
 \tkzDefMidPoint(A,B)              \tkzGetPoint{O}
 \tkzDefMidPoint(O,B)              \tkzGetPoint{O'}
 \tkzDefLine[tangent from=A](O',B) \tkzGetFirstPoint{E}
 \tkzInterLC(A,E)(O,B)             \tkzGetFirstPoint{D}
 \tkzDefPointBy[projection=onto A--B](D)  
 \tkzGetPoint{F}
 \tkzDrawCircles(O,B O',B)
 \tkzDrawSegments(A,D A,B D,F) 
 \tkzDrawSegments[color=red,line width=1pt,
     opacity=.4](A,O F,B)
 \tkzDrawPoints(A,B,O,O',E,D) 
 \tkzMarkRightAngle(D,F,B)
 \tkzLabelPoints[below right](A,B,O,O',E,D) 
\end{tikzpicture}
\end{tkzexample}

\subsubsection{Circle defined by a center and a measure, and special cases}
Let's look at some special cases like straight lines tangent to the circle.

\begin{tkzexample}[latex=7cm,small]
\begin{tikzpicture}[scale=.5]
 \tkzDefPoint(0,8){A}      \tkzDefPoint(8,0){B}
 \tkzDefPoint(8,8){C}      \tkzDefPoint(4,4){D}
 \tkzDefPoint(2,4){E}      \tkzDefPoint(4,2){F}
 \tkzDefPoint(8,4){G}
 \tkzInterLC(A,C)(D,G)     \tkzGetPoints{I1}{I2}
 \tkzInterLC(B,C)(D,G)     \tkzGetPoints{J1}{J2}
 \tkzInterLC[near](A,B)(D,G)  \tkzGetPoints{K1}{K2}
 \tkzInterLC(E,F)(D,G)     \tkzGetPoints{E1}{E2}
 \tkzDrawCircle(D,G)
 \tkzDrawPoints[color=red](I1,J1,K1,K2,E1,E2)
 \tkzDrawLines(A,B B,C A,C I2,J2 E1,E2)
 \tkzDrawPoints[color=blue](A,...,F)
 \tkzDrawPoints[color=red](I2,J2)
 \tkzLabelPoints[left](B,D,E,F)
 \tkzLabelPoints[below left](A,C)
 \tkzLabelPoints[below=4pt](I1,K1,K2,E2)
 \tkzLabelPoints[left](J1,E1)
\end{tikzpicture}

\end{tkzexample}

\subsubsection{Calculation of radius}
 With \tkzname{pgfmath} and \tkzcname{pgfmathsetmacro}

The radius measurement may be the result of a calculation that is not done within the intersection macro, but before.
A length can be calculated in several ways. It is possible of course,
 to use the module \tkzname{pgfmath} and the macro \tkzcname{pgfmathsetmacro}. In some cases, the results obtained are not precise enough, so the following calculation $0.0002 \div 0.0001$ gives $1.98$ with pgfmath while xfp will give $2$. 

With \tkzname{xfp} and \tkzcname{fpeval}:

\begin{tkzexample}[latex=7cm,small]
  \begin{tikzpicture}
  \tkzDefPoint(2,2){A}
  \tkzDefPoint(5,4){B}
  \tkzDefPoint(4,4){O}
  \pgfmathsetmacro\tkzLen{\fpeval{0.0002/0.0001}}
 % or \edef\tkzLen{\fpeval{0.0002/0.0001}}
  \tkzInterLC[R](A,B)(O, \tkzLen)
  \tkzGetPoints{I}{J}
  \tkzDrawCircle(O,I)
  \tkzDrawPoints[color=blue](A,B)
  \tkzDrawPoints[color=red](I,J)
  \tkzDrawLine(I,J)
\end{tikzpicture}
  \end{tkzexample}


\subsubsection{Option "with nodes"}
\begin{tkzexample}[latex=8cm,small]
\begin{tikzpicture}[scale=.75]
\tkzDefPoints{0/0/A,4/0/B,1/1/D,2/0/E}
\tkzDefTriangle[equilateral](A,B)
\tkzGetPoint{C}
\tkzInterLC[with nodes](D,E)(C,A,B)
\tkzGetPoints{F}{G}
\tkzDrawCircle(C,A)
\tkzDrawPolygon(A,B,C)
\tkzDrawPoints(A,...,G)
\tkzDrawLine(F,G)
\end{tikzpicture}
\end{tkzexample}

\subsection{Intersection of two circles  \tkzcname{tkzInterCC}}

The most frequent case is that of two circles defined by their center and a point, but as before the option \tkzname{R} allows to use the radius measurements.

\begin{NewMacroBox}{tkzInterCC}{\oarg{options}\parg{$O,A$}\parg{$O',A'$} or \parg{$O,r$}\parg{$O',r'$} or   \parg{$O,A,B$} \parg{$O',C,D$}}%
\begin{tabular}{lll}%
options       & default & definition                         \\
\midrule
\TOline{N}   {N}    {$OA$ and $O'A'$ are radii, $O$ and $O'$ are the centers.}
\TOline{R}   {N}    {$r$ and $r'$ are dimensions and measure the radii.}
\TOline{with nodes} {N}  {in (A,A,C)(C,B,F) AC and BF give the radii. }
\TOline{common=pt}  {}   {pt is common point; tkzFirstPoint gives the other point.}
\bottomrule
\end{tabular}

\medskip
This macro defines the intersection point(s) $I$ and $J$ of the two center circles $O$ and $O'$. If the two circles do not have a common point then the macro ends with an error that is not handled. If the centers are $O$ and $O'$ and the intersections are $A$ and $B$ then the angles $\widehat{O,A,O'}$ and $\widehat{O,B,O'}$ are in opposite directions. \tkzname{tkzFirstPoint} is assigned to the point that forms the "clockwise" angle.
\end{NewMacroBox}

\begin{NewMacroBox}{tkzTestInterCC}{\parg{$O,A$}\parg{$O',B$}}%
So the arguments are two couples which define two circles with a center and a point on the circle. If there is a non empty intersection between these two circles then the test \tkzcname{iftkzFlagCC} gives true.
\end{NewMacroBox}

\subsubsection{test circle-circle intersection}

\begin{tkzexample}[latex=7cm,small]
\begin{tikzpicture}[scale=.75]
  \tkzDefPoints{% x   y   name
                   0    /0    /A,
                   2    /0    /B,
                   4    /0    /I,
                   1    /0    /P}
\tkzDrawCircle(A,B)
\foreach \i in {1,...,3}{%
   \coordinate  (P) at (\i,0);
\tkzDrawCircle[new](I,P)
   \tkzTestInterCC(A,B)(I,P)
    \iftkzFlagCC
    \tkzInterCC(A,B)(I,P)  \tkzGetPoints{a}{b}
    \tkzDrawPoints(a,b)
    \fi}
  \end{tikzpicture}
\end{tkzexample}

\subsubsection{circle-circle intersection with \tkzname{common} point.}

\begin{tkzexample}[latex=7cm,small]
  \begin{tikzpicture}[scale=.5]
    \tkzDefPoints{0/0/O,5/-1/A,2/2/B}
    \tkzDrawPoints(O,A,B)
    \tkzDrawCircles(O,B A,B)
    \tkzInterCC[common=B](O,B)(A,B)\tkzGetFirstPoint{C}
    \tkzDrawPoint(C)
    \tkzLabelPoints[above](O,A,B,C)
  \end{tikzpicture}
\end{tkzexample}

\subsubsection{circle-circle intersection order of points.}
The idea is to compare the angles formed with the first center, a resultant point and the center of the second circle. The first point is the one that corresponds to the smallest angle.

As you can see $\widehat{ODB} < \widehat{OBE} $

\begin{tkzexample}[latex=7cm,small]
\begin{tikzpicture}[scale=.5]
   \pgfkeys{/pgf/number format/.cd,fixed relative,precision=4}
  \tkzDefPoints{0/0/O,5/-1/A,2/2/B,2/-1/C}
  \tkzDrawPoints(O,A,B)
  \tkzDrawCircles(O,A B,C)
  \tkzInterCC(O,A)(B,C)\tkzGetPoints{D}{E}
  \tkzDrawPoints(C,D,E)
  \tkzLabelPoints(O,A,B,C)
  \tkzLabelPoints[above](D,E) 
  \tkzDrawSegments[cyan](D,O D,B)
  \tkzMarkAngle[red,->,size=1.5](O,D,B)
  \tkzFindAngle(O,D,B)   \tkzGetAngle{an}
  \tkzLabelAngle(O,D,B){$ \pgfmathprintnumber{\an}$}
  \tkzDrawSegments[cyan](E,O E,B)
  \tkzMarkAngle[red,->,size=1.5](O,E,B)  
  \tkzFindAngle(O,E,B)   \tkzGetAngle{an}
  \tkzLabelAngle(O,E,B){$ \pgfmathprintnumber{\an}$}  
\end{tikzpicture}
\end{tkzexample}

  
  
\subsubsection{Construction of an equilateral triangle.}
$\widehat{A,C,B}$ is a clockwise angle
\begin{tkzexample}[latex=7cm,small]
\begin{tikzpicture}[trim left=-1cm,scale=.5]
 \tkzDefPoint(1,1){A}
 \tkzDefPoint(5,1){B}
 \tkzInterCC(A,B)(B,A)\tkzGetPoints{C}{D}
 \tkzDrawPoint[color=black](C)
 \tkzDrawCircles(A,B B,A)
 \tkzCompass[color=red](A,C)
 \tkzCompass[color=red](B,C)
 \tkzDrawPolygon(A,B,C)
 \tkzMarkSegments[mark=s|](A,C B,C)
 \tkzLabelPoints[](A,B)
 \tkzLabelPoint[above](C){$C$}
\end{tikzpicture}
\end{tkzexample}


\subsubsection{Segment trisection}
 The idea here is to divide a segment with a ruler and a compass into three segments of equal length.

\begin{tkzexample}[latex=7cm,small]
\begin{tikzpicture}[scale=.6]
 \tkzDefPoint(0,0){A}
 \tkzDefPoint(3,2){B}
 \tkzInterCC(A,B)(B,A)            \tkzGetSecondPoint{D}
 \tkzInterCC(D,B)(B,A)            \tkzGetPoints{A}{C}
 \tkzInterCC(D,B)(A,B)            \tkzGetPoints{E}{B}
 \tkzInterLC[common=D](C,D)(E,D)  \tkzGetFirstPoint{F}
 \tkzInterLL(A,F)(B,C)            \tkzGetPoint{O}
 \tkzInterLL(O,D)(A,B)            \tkzGetPoint{H}
 \tkzInterLL(O,E)(A,B)            \tkzGetPoint{G}
 \tkzDrawCircles(D,E A,B B,A E,A)
 \tkzDrawSegments[](O,F O,B O,D O,E)
 \tkzDrawPoints(A,...,H)
 \tkzDrawSegments(A,B B,D A,D A,E E,F C,F B,C)
 \tkzMarkSegments[mark=s|](A,G G,H H,B)
\end{tikzpicture}
\end{tkzexample}

\subsubsection{With the option "\tkzimp{with nodes}"}
\begin{tkzexample}[latex=6cm,small]
\begin{tikzpicture}[scale=.5]
 \tkzDefPoints{0/0/A,0/5/B,5/0/C}
 \tkzDefPoint(54:5){F}
 \tkzInterCC[with nodes](A,A,C)(C,B,F)
 \tkzGetPoints{a}{e}
 \tkzInterCC(A,C)(a,e) \tkzGetFirstPoint{b}
 \tkzInterCC(A,C)(b,a) \tkzGetFirstPoint{c}
 \tkzInterCC(A,C)(c,b) \tkzGetFirstPoint{d}
 \tkzDrawCircle[new](A,C)
 \tkzDrawPoints(a,b,c,d,e)
 \tkzDrawPolygon(a,b,c,d,e)
 \foreach \vertex/\num in {a/36,b/108,c/180,
                          d/252,e/324}{%
 \tkzDrawPoint(\vertex)
 \tkzLabelPoint[label=\num:$\vertex$](\vertex){}
 \tkzDrawSegment(A,\vertex)
 }
\end{tikzpicture}
\end{tkzexample}

\subsubsection{Mix of intersections}
\begin{tkzexample}[latex=8cm,small]
\begin{tikzpicture}[scale = .75]
  \tkzDefPoint(2,2){A}
  \tkzDefPoint(0,0){B}
  \tkzDefPoint(-2,2){C}
  \tkzDefPoint(0,4){D}
  \tkzDefPoint(4,2){E}
  \tkzCircumCenter(A,B,C)\tkzGetPoint{O}
  \tkzInterCC[R](O,2)(D,2) \tkzGetPoints{M1}{M2}
  \tkzInterCC(O,A)(D,O) \tkzGetPoints{1}{2}
  \tkzInterLC(A,E)(B,M1) \tkzGetSecondPoint{M3}
  \tkzInterLC(O,C)(M3,D) \tkzGetSecondPoint{L}
  \tkzDrawSegments(C,L)
  \tkzDrawPoints(A,B,C,D,E,M1,M2,M3,O,L)
  \tkzDrawSegments(O,E)
  \tkzDrawSegments[new](C,A D,B)
  \tkzDrawPoint(O)
  \tkzDrawCircles[new](M3,D B,M2 D,O)
  \tkzDrawCircle(O,A)
  \tkzLabelPoints[below right](A,B,C,D,E,M1,M2,M3,O,L)
\end{tikzpicture}
\end{tkzexample}


\subsubsection{Altshiller-Court's theorem}
  The two lines joining the points of intersection of two orthogonal circles to a point on one of the circles met the other circle in two diametricaly oposite points. Altshiller p 176


\begin{tkzexample}[vbox,small]
\begin{tikzpicture}
  \tkzDefPoints{0/0/P,5/0/Q,3/2/I}
  \tkzDefCircle[orthogonal from=P](Q,I) 
  \tkzGetFirstPoint{E}
  \tkzDrawCircles(P,E Q,E)
  \tkzInterCC[common=E](P,E)(Q,E) \tkzGetFirstPoint{F}
  \tkzDefPointOnCircle[through =  center P angle 80 point E] 
  \tkzGetPoint{A}
  \tkzInterLC[common=E](A,E)(Q,E)  \tkzGetFirstPoint{C}
  \tkzInterLL(A,F)(C,Q)  \tkzGetPoint{D}
  \tkzDrawLines[add=0 and 1](P,Q)
  \tkzDrawLines[add=0 and 2](A,E)
  \tkzDrawSegments(P,E E,F F,C A,F C,D)
  \tkzDrawPoints(P,Q,E,F,A,C,D)
  \tkzLabelPoints(P,Q,F)
  \tkzLabelPoints[above](E,A)
  \tkzLabelPoints[left](D)
  \tkzLabelPoints[above right](C)
\end{tikzpicture}
\end{tkzexample}


\endinput
\section{The angles} 

\subsection{Colour an angle: fill}

The simplest operation
\begin{NewMacroBox}{tkzFillAngle}{\oarg{local options}\parg{A,O,B}}%
$O$ is the vertex of the angle. $OA$ and $OB$ are the sides. Attention the angle is determined by the order of the points.

\medskip

\begin{tabular}{lll}%
\toprule
options             & default & definition                        \\ 
\midrule
\TOline{size}{1 cm}{this option determines the radius of the coloured angular sector.}

\bottomrule
\end{tabular} 

\medskip
Of course, you have to add all the styles of \TIKZ, like the use of fill and shade... 
\end{NewMacroBox}  

\subsubsection{Example with \tkzname{size}}  
\begin{tkzexample}[latex=7cm,small]
\begin{tikzpicture} 
   \tkzInit 
   \tkzDefPoints{0/0/O,2.5/0/A,1.5/2/B}
   \tkzFillAngle[size=2cm, fill=gray!10](A,O,B)
   \tkzDrawLines(O,A O,B)
   \tkzDrawPoints(O,A,B)
\end{tikzpicture}
\end{tkzexample}


\subsubsection{Changing the order of items} 
\begin{tkzexample}[latex=7cm,small]
\begin{tikzpicture} 
   \tkzInit 
   \tkzDefPoints{0/0/O,2.5/0/A,1.5/2/B}
   \tkzFillAngle[size=2cm,fill=gray!10](B,O,A)
   \tkzDrawLines(O,A O,B)
   \tkzDrawPoints(O,A,B)
\end{tikzpicture}
\end{tkzexample}

\begin{tkzexample}[latex=7cm,small]
\begin{tikzpicture} 
   \tkzInit 
   \tkzDefPoints{0/0/O,5/0/A,3/4/B}
   % Don't forget {} to get, () to use
   \tkzFillAngle[size=4cm,left color=white, 
                 right color=red!50](A,O,B)
   \tkzDrawLines(O,A O,B)
   \tkzDrawPoints(O,A,B)
\end{tikzpicture}
\end{tkzexample}

\begin{NewMacroBox}{tkzFillAngles}{\oarg{local options}\parg{A,O,B}\parg{A',O',B'}etc.}%
With common options, there is a macro for multiple angles.
  \end{NewMacroBox}  
  
\subsubsection{Multiples angles}  
\begin{tkzexample}[latex=7cm,small]
\begin{tikzpicture}[scale=0.75]
  \tkzDefPoint(0,0){B}
  \tkzDefPoint(8,0){C}
  \tkzDefPoint(0,8){A}
  \tkzDefPoint(8,8){D}
  \tkzDrawPolygon(B,C,D,A)
  \tkzDefTriangle[equilateral](B,C) 
  \tkzGetPoint{M}
  \tkzInterLL(D,M)(A,B) \tkzGetPoint{N}
  \tkzDefPointBy[rotation=center N angle -60](D) 
  \tkzGetPoint{L}
  \tkzInterLL(N,L)(M,B)     \tkzGetPoint{P}
  \tkzInterLL(M,C)(D,L)     \tkzGetPoint{Q}
  \tkzDrawSegments(D,N N,L L,D B,M M,C)
  \tkzDrawPoints(L,N,P,Q,M,A,D) 
  \tkzLabelPoints[left](N,P,Q)
  \tkzLabelPoints[above](M,A,D)
  \tkzLabelPoints(L,B,C)
  \tkzMarkAngles(C,B,M B,M,C M,C,B%
                 D,L,N L,N,D N,D,L)
  \tkzFillAngles[fill=red!20,opacity=.2](C,B,M%
      B,M,C M,C,B D,L,N L,N,D N,D,L)
\end{tikzpicture}
\end{tkzexample} 
 
\subsection{Mark an angle mark}
More delicate operation because there are many options. The symbols used for marking in addition to those of \TIKZ\ are defined in the file |tkz-lib-marks.tex| and designated by the following characters:\begin{tkzltxexample}[]
|, ||,|||, z, s, x, o, oo 
\end{tkzltxexample}

Their definitions are as follows

\begin{tkzltxexample}[]
\pgfdeclareplotmark{||}
  %double bar
{%
  \pgfpathmoveto{\pgfqpoint{2\pgflinewidth}{\pgfplotmarksize}}
  \pgfpathlineto{\pgfqpoint{2\pgflinewidth}{-\pgfplotmarksize}}
  \pgfpathmoveto{\pgfqpoint{-2\pgflinewidth}{\pgfplotmarksize}}
  \pgfpathlineto{\pgfqpoint{-2\pgflinewidth}{-\pgfplotmarksize}}
  \pgfusepathqstroke
}
\end{tkzltxexample}

\begin{tkzltxexample}[]
  %triple bar
  \pgfdeclareplotmark{|||}
  {%
    \pgfpathmoveto{\pgfqpoint{0 pt}{\pgfplotmarksize}}
    \pgfpathlineto{\pgfqpoint{0 pt}{-\pgfplotmarksize}}
    \pgfpathmoveto{\pgfqpoint{-3\pgflinewidth}{\pgfplotmarksize}}
    \pgfpathlineto{\pgfqpoint{-3\pgflinewidth}{-\pgfplotmarksize}}
    \pgfpathmoveto{\pgfqpoint{3\pgflinewidth}{\pgfplotmarksize}}
    \pgfpathlineto{\pgfqpoint{3\pgflinewidth}{-\pgfplotmarksize}}
    \pgfusepathqstroke
  } 
\end{tkzltxexample}

\begin{tkzltxexample}[]
  % An bar slant
  \pgfdeclareplotmark{s|}
  {%
    \pgfpathmoveto{\pgfqpoint{-.70710678\pgfplotmarksize}%
                             {-.70710678\pgfplotmarksize}}
    \pgfpathlineto{\pgfqpoint{.70710678\pgfplotmarksize}%
                             {.70710678\pgfplotmarksize}}
    \pgfusepathqstroke
  } 
\end{tkzltxexample}


\begin{tkzltxexample}[]
  % An double bar slant
  \pgfdeclareplotmark{s||}
  {%
   \pgfpathmoveto{\pgfqpoint{-0.75\pgfplotmarksize}{-\pgfplotmarksize}}
   \pgfpathlineto{\pgfqpoint{0.25\pgfplotmarksize}{\pgfplotmarksize}} 
   \pgfpathmoveto{\pgfqpoint{0\pgfplotmarksize}{-\pgfplotmarksize}}
   \pgfpathlineto{\pgfqpoint{1\pgfplotmarksize}{\pgfplotmarksize}} 
   \pgfusepathqstroke
  }   
\end{tkzltxexample}


\begin{tkzltxexample}[]
  % z
  \pgfdeclareplotmark{z}
  {%
    \pgfpathmoveto{\pgfqpoint{0.75\pgfplotmarksize}{-\pgfplotmarksize}} 
    \pgfpathlineto{\pgfqpoint{-0.75\pgfplotmarksize}{-\pgfplotmarksize}}
    \pgfpathlineto{\pgfqpoint{0.75\pgfplotmarksize}{\pgfplotmarksize}}
    \pgfpathlineto{\pgfqpoint{-0.75\pgfplotmarksize}{\pgfplotmarksize}}
    \pgfusepathqstroke
  }
\end{tkzltxexample}

\begin{tkzltxexample}[]
  % s
  \pgfdeclareplotmark{s}
  {%
     \pgfpathmoveto{\pgfqpoint{0pt}{0pt}} 
     \pgfpathcurveto
         {\pgfpoint{0pt}{0pt}}
         {\pgfpoint{-\pgfplotmarksize}{\pgfplotmarksize}}
         {\pgfpoint{\pgfplotmarksize}{\pgfplotmarksize}}
     \pgfpathmoveto{\pgfqpoint{0pt}{0pt}} 
      \pgfpathcurveto
         {\pgfpoint{0pt}{0pt}}
         {\pgfpoint{\pgfplotmarksize}{-\pgfplotmarksize}}
         {\pgfpoint{-\pgfplotmarksize}{-\pgfplotmarksize}} 
      \pgfusepathqstroke
  }  
\end{tkzltxexample}

\begin{tkzltxexample}[]
  % infinity
  \pgfdeclareplotmark{oo}
  {%
     \pgfpathmoveto{\pgfqpoint{0pt}{0pt}} 
     \pgfpathcurveto
         {\pgfpoint{0pt}{0pt}}
         {\pgfpoint{.5\pgfplotmarksize}{1\pgfplotmarksize}}
         {\pgfpoint{\pgfplotmarksize}{0pt}}
     \pgfpathmoveto{\pgfqpoint{0pt}{0pt}} 
      \pgfpathcurveto
         {\pgfpoint{0pt}{0pt}}
         {\pgfpoint{-.5\pgfplotmarksize}{1\pgfplotmarksize}}
         {\pgfpoint{-\pgfplotmarksize}{0pt}}  
     \pgfpathmoveto{\pgfqpoint{0pt}{0pt}}  
        \pgfpathcurveto
         {\pgfpoint{0pt}{0pt}}
         {\pgfpoint{.5\pgfplotmarksize}{-1\pgfplotmarksize}}
         {\pgfpoint{\pgfplotmarksize}{0pt}}
     \pgfpathmoveto{\pgfqpoint{0pt}{0pt}} 
      \pgfpathcurveto
         {\pgfpoint{0pt}{0pt}}
         {\pgfpoint{-.5\pgfplotmarksize}{-1\pgfplotmarksize}}
         {\pgfpoint{-\pgfplotmarksize}{0pt}}      
      \pgfusepathqstroke
  } 
\end{tkzltxexample}



%                \tkzMarkAngle(B, A, C)
%
% Marque d'angle
% arc de cercle (simple/double/triple) et marque d'églité.
%
% Par défaut: 
%                 arc       = simple
%                 mksize  = 1cm (rayon de l'arc)
%                 style traits pleins
%                 mkpos ?  position: 0.5 (position de la marque)
%                 mark rien du tout (ignoré si type est utilisé)
%
% Paramètres (optionnels)
%             arc     : l, ll, lll
%             mksize  : 1cm
%             gap     : 3pt
%             dist    : 1?
%             style   : type de traits
%             mkpos   : 0.5
%             mark    : none  , |, ||,|||, z, s, x, o, oo mais tous les 
%  % symboles de tikz sont permis

\begin{NewMacroBox}{tkzMarkAngle}{\oarg{local options}\parg{A,O,B}}%
$O$ is the vertex. Attention the arguments vary according to the options. Several markings are possible. You can simply draw an arc or  add a mark on this arc. The style of the arc is chosen with the option \tkzname{arc}, the radius of the arc is given by \tkzname{mksize}, the arc can, of course, be colored.

\medskip

\begin{tabular}{lll}%
\toprule
options             & default & definition                        \\ 
\midrule
\TOline{arc}{l}{choice of l, ll and lll (single, double or triple).}
\TOline{size}{1 cm}{arc radius.}
\TOline{mark}{none}{choice of mark.}
\TOline{mksize}{4pt}{symbol size (mark).}
\TOline{mkcolor}{black}{symbol color (mark).}
\TOline{mkpos}{0.5}{position of the symbol on the arc.}
\end{tabular} 
\end{NewMacroBox}  

\subsubsection{Example with \tkzname{mark = x}}
\begin{tkzexample}[latex=6cm,small]
    \begin{tikzpicture}[scale=.75]
        \tkzDefPoints{0/0/O,5/0/A,3/4/B}
        \tkzMarkAngle[size = 4cm,mark = x,
                      arc=ll,mkcolor = red](A,O,B)
        \tkzDrawLines(O,A O,B)
        \tkzDrawPoints(O,A,B)
    \end{tikzpicture}
\end{tkzexample}
\DeleteShortVerb{\|}
\subsubsection{Example with \tkzname{mark =||}}
\MakeShortVerb{\|}
\begin{tkzexample}[latex=6cm,small]
    \begin{tikzpicture}[scale=.75]
        \tkzDefPoints{0/0/O,5/0/A,3/4/B}
        \tkzMarkAngle[size = 4cm,mark = ||,
                    arc=ll,mkcolor = red](A,O,B)
        \tkzDrawLines(O,A O,B)
        \tkzDrawPoints(O,A,B)
    \end{tikzpicture}
\end{tkzexample}

\begin{NewMacroBox}{tkzMarkAngles}{\oarg{local options}\parg{A,O,B}\parg{A',O',B'}etc.}%
With common options, there is a macro for multiple angles.
  \end{NewMacroBox}  
  
  
\subsection{Label at an angle}

\begin{NewMacroBox}{tkzLabelAngle}{\oarg{local options}\parg{A,O,B}}%
There is only one option, dist (with or without unit), which can be replaced by the TikZ's pos option (without unit for the latter). By default, the value is in centimeters.

\begin{tabular}{lll}%
	\toprule
options             & default & definition                        \\ 
\midrule
\TOline{pos}{1}{ or dist, controls the distance from the top to the label.}
\bottomrule
\end{tabular} 

\medskip 
It is possible to move the label with all TikZ options : rotate, shift, below, etc.
\end{NewMacroBox}  

\subsubsection{Example with \tkzname{pos}} 
\begin{tkzexample}[latex=7cm,small]
\begin{tikzpicture}[scale=.75]
  \tkzDefPoints{0/0/O,5/0/A,3/4/B}
  \tkzMarkAngle[size = 4cm,mark = ||,
      arc=ll,color = red](A,O,B)%     
  \tkzDrawLines(O,A O,B)
  \tkzDrawPoints(O,A,B)
  \tkzLabelAngle[pos=2,draw,circle,
      fill=blue!10](A,O,B){$\alpha$} 
\end{tikzpicture}
\end{tkzexample}

\begin{tkzexample}[latex=7cm,small]
\begin{tikzpicture}[rotate=30]
  \tkzDefPoint(2,1){S} 
  \tkzDefPoint(7,3){T}
  \tkzDefPointBy[rotation=center S angle 60](T)
  \tkzGetPoint{P} 
  \tkzDefLine[bisector,normed](T,S,P)
  \tkzGetPoint{s}
  \tkzDrawPoints(S,T,P)   
  \tkzDrawPolygon[color=blue](S,T,P) 
  \tkzDrawLine[dashed,color=blue,add=0 and 3](S,s)  
  \tkzLabelPoint[above right](P){$P$}
  \tkzLabelPoints(S,T)
  \tkzMarkAngle[size = 1.8cm,mark = |,arc=ll,
                    color = blue](T,S,P)
  \tkzMarkAngle[size = 2.1cm,mark = |,arc=l,
                    color = blue](T,S,s)
  \tkzMarkAngle[size = 2.3cm,mark = |,arc=l,
                    color = blue](s,S,P)  
 \tkzLabelAngle[pos = 1.5](T,S,P){$60^{\circ}$}%    
 \tkzLabelAngles[pos = 2.7](T,S,s s,S,P){$30^{\circ}$}%   
\end{tikzpicture}
\end{tkzexample}

\begin{NewMacroBox}{tkzLabelAngles}{\oarg{local options}\parg{A,O,B}\parg{A',O',B'}etc.}%
With common options, there is a macro for multiple angles.
\end{NewMacroBox}  
  
\subsection{Marking a right angle}

\begin{NewMacroBox}{tkzMarkRightAngle}{\oarg{local options}\parg{A,O,B}}%
The \tkzname{german} option allows you to change the style of the drawing. The option \tkzname{size} allows to change the size of the drawing.

\medskip
\begin{tabular}{lll}%
\toprule
options             & default & definition         \\ 
\midrule
\TOline{german}{normal}{ german arc with inner point.}
\TOline{size}{0.2}{ side size.}
\end{tabular} 
\end{NewMacroBox}  

\subsubsection{Example of marking a right angle} 
\begin{tkzexample}[latex=6cm,small]
\begin{tikzpicture}
  \tkzDefPoints{0/0/A,3/1/B,0.9/-1.2/P}
  \tkzDefPointBy[projection = onto B--A](P)  \tkzGetPoint{H}
  \tkzDrawLines[add=.5 and .5](P,H)
  \tkzMarkRightAngle[fill=blue!20,size=.5,draw](A,H,P) 
  \tkzDrawLines[add=.5 and .5](A,B)
  \tkzMarkRightAngle[fill=red!20,size=.8](B,H,P)
  \tkzDrawPoints[](A,B,P,H)  
\end{tikzpicture}
\end{tkzexample}

\subsubsection{Example of marking a right angle, german style} 
\begin{tkzexample}[latex=6cm,small]
\begin{tikzpicture}
  \tkzDefPoints{0/0/A,3/1/B,0.9/-1.2/P}
  \tkzDefPointBy[projection = onto B--A](P)  \tkzGetPoint{H}
  \tkzDrawLines[add=.5 and .5](P,H)
  \tkzMarkRightAngle[german,size=.5,draw](A,H,P) 
  \tkzDrawPoints[](A,B,P,H) 
  \tkzDrawLines[add=.5 and .5,fill=blue!20](A,B)
  \tkzMarkRightAngle[german,size=.8](P,H,B) 
\end{tikzpicture}
\end{tkzexample}

\subsubsection{Mix of styles} 
\begin{tkzexample}[latex=6cm,small]
\begin{tikzpicture}[scale=.75]
  \tkzDefPoint(0,0){A}
  \tkzDefPoint(4,1){B}
  \tkzDefPoint(2,5){C}
  \tkzDefPointBy[projection=onto B--A](C) 
      \tkzGetPoint{H}
  \tkzDrawLine(A,B)
  \tkzDrawLine[add = .5 and .2,color=red](C,H)
  \tkzMarkRightAngle[,size=1,color=red](C,H,A)
  \tkzMarkRightAngle[german,size=.8,color=blue](B,H,C)
  \tkzFillAngle[opacity=.2,fill=blue!20,size=.8](B,H,C)
  \tkzLabelPoints(A,B,C,H)
  \tkzDrawPoints(A,B,C)
\end{tikzpicture}
\end{tkzexample}

\subsubsection{Full example} 

\begin{tkzexample}[latex=6cm,small]
\begin{tikzpicture}[rotate=-90]
\tkzDefPoint(0,1){A}
\tkzDefPoint(2,4){C}
\tkzDefPointWith[orthogonal normed,K=7](C,A)
\tkzGetPoint{B}
\tkzDrawSegment[green!60!black](A,C)
\tkzDrawSegment[green!60!black](C,B)
\tkzDrawSegment[green!60!black](B,A)
\tkzDrawLine[altitude,dashed,color=magenta](B,C,A)
\tkzGetPoint{P}
\tkzLabelPoint[left](A){$A$}
\tkzLabelPoint[right](B){$B$}
\tkzLabelPoint[above](C){$C$}
\tkzLabelPoint[left](P){$P$}
\tkzLabelSegment[auto](B,A){$c$}
\tkzLabelSegment[auto,swap](B,C){$a$}
\tkzLabelSegment[auto,swap](C,A){$b$}
\tkzMarkAngle[size=1cm,color=cyan,mark=|](C,B,A)
\tkzMarkAngle[size=1cm,color=cyan,mark=|](A,C,P)
\tkzMarkAngle[size=0.75cm,color=orange,mark=||](P,C,B)
\tkzMarkAngle[size=0.75cm,color=orange,mark=||](B,A,C)
\tkzMarkRightAngle[german](A,C,B)
\tkzMarkRightAngle[german](B,P,C)
\end{tikzpicture} 
\end{tkzexample} 

\subsection{\tkzcname{tkzMarkRightAngles}}
\begin{NewMacroBox}{tkzMarkRightAngles}{\oarg{local options}\parg{A,O,B}\parg{A',O',B'}etc.}%
With common options, there is a macro for multiple angles.
\end{NewMacroBox}

\section{Angles tools}

\subsection{Recovering an angle \tkzcname{tkzGetAngle}}
\begin{NewMacroBox}{tkzGetAngle}{\parg{name of macro}}%
Assigns the value in degree of an angle to a macro. This macro retrieves \tkzcname{tkzAngleResult} and stores the result in a new macro.

\medskip

\begin{tabular}{lll}%
\toprule
arguments             & example & explication             \\
\midrule
\TAline{name of macro} {\tkzcname{tkzGetAngle}\{ang\}}{\tkzcname{ang} contains the value of the angle.}
\end{tabular}
\end{NewMacroBox}

\subsection{Example of the use of \tkzcname{tkzGetAngle}}

 The point here is that $(AB)$ is the bisector of $\widehat{CAD}$, such that the $AD$ slope is zero. We recover the slope of $(AB)$ and then rotate twice.


\begin{tkzexample}[vbox,small]
\begin{tikzpicture}
  \tkzInit
  \tkzDefPoint(1,5){A} \tkzDefPoint(5,2){B}  
  \tkzDrawSegment(A,B)
  \tkzFindSlopeAngle(A,B)\tkzGetAngle{tkzang}
  \tkzDefPointBy[rotation= center A angle \tkzang ](B)
   \tkzGetPoint{C}
  \tkzDefPointBy[rotation= center A angle -\tkzang ](B) 
  \tkzGetPoint{D}
  \tkzCompass[length=1,dashed,color=red](A,C)
  \tkzCompass[delta=10,brown](B,C)  
   \tkzDrawPoints(A,B,C,D)
  \tkzLabelPoints(B,C,D)  
  \tkzLabelPoints[above left](A)
  \tkzDrawSegments[style=dashed,color=orange!30](A,C A,D)
\end{tikzpicture}
\end{tkzexample}



\subsection{Angle formed by three points}

\begin{NewMacroBox}{tkzFindAngle}{\parg{pt1,pt2,pt3}}%
The result is stored in a macro \tkzcname{tkzAngleResult}.

\medskip

\begin{tabular}{lll}%
\toprule
arguments     & example & explication     \\
\midrule
\TAline{(pt1,pt2,pt3)} {\tkzcname{tkzFindAngle}(A,B,C)}{\tkzcname{tkzAngleResult} gives the angle ($\overrightarrow{BA},\overrightarrow{BC}$)}
\bottomrule
\end{tabular}

\medskip
The result is between -180 degrees and +180 degrees. pt2 is the vertex and \tkzcname{tkzGetAngle} can retrieve the angle.
\end{NewMacroBox}
 
\subsubsection{Verification of angle measurement}
    
\begin{tkzexample}[latex=7cm,small]
\begin{tikzpicture}[scale=.75]
  \tkzDefPoint(-1,1){A}
  \tkzDefPoint(5,2){B}
  \tkzDefEquilateral(A,B)
  \tkzGetPoint{C}
  \tkzDrawPolygon(A,B,C)
  \tkzFindAngle(B,A,C) 
  \tkzGetAngle{angleBAC}
  \edef\angleBAC{\fpeval{round(\angleBAC)}}
  \tkzDrawPoints(A,B,C) 
  \tkzLabelPoints(A,B)
  \tkzLabelPoint[right](C){$C$}
  \tkzLabelAngle(B,A,C){\angleBAC$^\circ$}
  \tkzMarkAngle[size=1.5cm](B,A,C)
\end{tikzpicture}
\end{tkzexample}

\subsection{Example of the use of \tkzcname{tkzFindAngle} }

\begin{tkzexample}[vbox,small]
\begin{tikzpicture}
   \tkzInit[xmin=-1,ymin=-1,xmax=7,ymax=7]
   \tkzClip  
   \tkzDefPoint (0,0){O}  \tkzDefPoint (6,0){A}
   \tkzDefPoint (5,5){B}  \tkzDefPoint (3,4){M}
   \tkzFindAngle (A,O,M)  \tkzGetAngle{an}   
   \tkzDefPointBy[rotation=center O angle \an](A) 
   \tkzGetPoint{C}
   \tkzDrawSector[fill = blue!50,opacity=.5](O,A)(C)
   \tkzFindAngle(M,B,A)   \tkzGetAngle{am}
   \tkzDefPointBy[rotation = center O angle \am](A) 
   \tkzGetPoint{D} 
   \tkzDrawSector[fill = red!50,opacity = .5](O,A)(D) 
   \tkzDrawPoints(O,A,B,M,C,D)   
   \tkzLabelPoints(O,A,B,M,C,D) 
	\edef\an{\fpeval{round(\an,2)}}\edef\am{\fpeval{round(\am,2)}}
   \tkzDrawSegments(M,B B,A)
   \tkzText(4,2){$\widehat{AOC}=\widehat{AOM}=\an^{\circ}$} 
   \tkzText(1,4){$\widehat{AOD}=\widehat{MBA}=\am^{\circ}$}  
\end{tikzpicture}
\end{tkzexample}

\subsubsection{Determination of the three angles of a triangle}

\begin{tkzexample}[latex=7cm,small]
  \begin{tikzpicture}[scale=1.25,rotate=30]
  \tkzDefPoints{0.5/1.5/A, 3.5/4/B, 6/2.5/C}
  \tkzDrawPolygon(A,B,C) 
  \tkzDrawPoints(A,B,C) 
  \tkzLabelPoints[below](A,C)
  \tkzLabelPoints[above](B)
  \tkzMarkAngle[size=1cm](B,C,A)
  \tkzFindAngle(B,C,A) 
  \tkzGetAngle{angleBCA}
  \edef\angleBCA{\fpeval{round(\angleBCA,2)}}
  \tkzLabelAngle[pos = 1](B,C,A){$\angleBCA^{\circ}$}
  \tkzMarkAngle[size=1cm](C,A,B)
  \tkzFindAngle(C,A,B) 
  \tkzGetAngle{angleBAC}
  \edef\angleBAC{\fpeval{round(\angleBAC,2)}}
  \tkzLabelAngle[pos = 1.8](C,A,B){%
             $\angleBAC^{\circ}$} 
  \tkzMarkAngle[size=1cm](A,B,C)
  \tkzFindAngle(A,B,C) 
  \tkzGetAngle{angleABC}
  \edef\angleABC{\fpeval{round(\angleABC,2)}}
  \tkzLabelAngle[pos = 1](A,B,C){$\angleABC^{\circ}$}
  \end{tikzpicture}
\end{tkzexample}

 \subsection{Determining a slope}
It is a question of determining whether it exists, the slope of a straight line defined by two points. No verification of the existence is made.

\begin{NewMacroBox}{tkzFindSlope}{\parg{pt1,pt2}\marg{name of macro}}%
The result is stored in a macro.

\medskip

\begin{tabular}{lll}%
\toprule
arguments             & example & explication                         \\
\midrule
\TAline{(pt1,pt2){pt3}} {\tkzcname{tkzFindSlope}(A,B)\{slope\}}{\tkzcname{slope} will give the result of $\frac{y_B-y_A}{x_B-x_A}$} \\
\bottomrule
\end{tabular}

\medskip
\tkzHandBomb\ Careful not to have $x_B=x_A$.
\end{NewMacroBox}


\begin{tkzexample}[latex=7cm,small]
\begin{tikzpicture}[scale=1.5]
  \tkzInit[xmax=4,ymax=5]\tkzGrid[sub]
  \tkzDefPoint(1,2){A}    \tkzDefPoint(3,4){B}
  \tkzDefPoint(3,2){C}    \tkzDefPoint(3,1){D}
  \tkzDrawSegments(A,B A,C A,D)
  \tkzDrawPoints[color=red](A,B,C,D)  
  \tkzLabelPoints(A,B,C,D)
  \tkzFindSlope(A,B){SAB} \tkzFindSlope(A,C){SAC}
  \tkzFindSlope(A,D){SAD}
  \pgfkeys{/pgf/number format/.cd,fixed,precision=2}
  \tkzText[fill=Gold!50,draw=brown](1,4)%
  {The slope of (AB) is : $\pgfmathprintnumber{\SAB}$}     
  \tkzText[fill=Gold!50,draw=brown](1,3.5)%
  {The slope of (AC) is : $\pgfmathprintnumber{\SAC}$}    
  \tkzText[fill=Gold!50,draw=brown](1,3)%
  {The slope of (AD) is : $\pgfmathprintnumber{\SAD}$}
\end{tikzpicture}
\end{tkzexample}

\subsection{Angle formed by a straight line with the horizontal axis \tkzcname{tkzFindSlopeAngle}}
Much more interesting than the last one. The result is between -180 degrees and +180 degrees.

\begin{NewMacroBox}{tkzFindSlopeAngle}{\parg{A,B}}%
Determines the slope of the straight line (AB). The result is stored in a macro \tkzcname{tkzAngleResult}.

\medskip
\begin{tabular}{lll}%
\toprule
arguments  & example & explication     \\
\midrule
\TAline{(pt1,pt2)} {\tkzcname{tkzFindSlopeAngle}(A,B)}{}
\bottomrule
\end{tabular}

\medskip
\tkzcname{tkzGetAngle} can retrieve the result. If retrieval is not necessary, you can use \tkzcname{tkzAngleResult}.
\end{NewMacroBox}
 
 \subsubsection{Folding}
\begin{tkzexample}[latex=6cm,small]
\begin{tikzpicture} 
  \tkzDefPoint(1,5){A}
  \tkzDefPoint(5,2){B}  
  \tkzDrawSegment(A,B) 
  \tkzFindSlopeAngle(A,B)
  \tkzGetAngle{tkzang}
  \tkzDefPointBy[rotation= center A angle \tkzang ](B) 
  \tkzGetPoint{C} 
  \tkzDefPointBy[rotation= center A angle -\tkzang ](B) 
  \tkzGetPoint{D} 
  \tkzCompass[orange,length=1](A,C) 
  \tkzCompass[orange,delta=10](B,C)   
  \tkzDrawPoints(A,B,C,D) 
  \tkzLabelPoints(B,C,D)  
  \tkzLabelPoints[above left](A) 
  \tkzDrawSegments[style=dashed,color=orange](A,C A,D)
\end{tikzpicture}
\end{tkzexample}

\subsubsection{Example of the use of \tkzcname{tkzFindSlopeAngle}}
Here is another version of the construction of a mediator

\begin{tkzexample}[latex=6cm,small]
\begin{tikzpicture}
 \tkzInit
 \tkzDefPoint(0,0){A}        
 \tkzDefPoint(3,2){B}
 \tkzDefLine[mediator](A,B)  
 \tkzGetPoints{I}{J}
 \tkzCalcLength[cm](A,B)     
 \tkzGetLength{dAB}
 \tkzFindSlopeAngle(A,B)     
 \tkzGetAngle{tkzangle}
 \begin{scope}[rotate=\tkzangle]
   \tikzset{arc/.style={color=gray,delta=10}}
   \tkzDrawArc[orange,R,arc](B,3/4*\dAB)(120,240)
   \tkzDrawArc[orange,R,arc](A,3/4*\dAB)(-45,60)
   \tkzDrawLine(I,J)         
   \tkzDrawSegment(A,B)
  \end{scope}
  \tkzDrawPoints(A,B,I,J)    
  \tkzLabelPoints(A,B)
   \tkzLabelPoints[right](I,J)
\end{tikzpicture}
\end{tkzexample}
 
\endinput



\section{Random point definition}
%<--------------------------------------------------------------------------->
%           points random
%<--------------------------------------------------------------------------->
At the moment there are four possibilities:
\begin{enumerate}
  \item point in a rectangle;
  \item on a segment;
  \item on a straight line;
  \item on a circle.
\end{enumerate}

\subsection{Obtaining random points}
This is the new version that replaces  \tkzcname{tkzGetRandPointOn}.
\begin{NewMacroBox}{tkzDefRandPointOn}{\oarg{local options}}%
{The result is a point with a random position that can be named with the macro \tkzcname{tkzGetPoint}. It is possible to use \tkzname{tkzPointResult} if it is not necessary to retain the results.}

\medskip
\begin{tabular}{lll}%
\toprule
options             & default & definition                         \\ 
\midrule
\TOline{rectangle=pt1 and pt2}  {}{[rectangle=A and B]} 
\TOline{segment= pt1--pt2} {}{[segment=A--B]}
\TOline{line=pt1--pt2}{}{[line=A--B]} 
\TOline{circle =center pt1 radius dim}{}{[circle = center A radius 2 cm]} 
\TOline{circle through=center pt1 through pt2}{}{[circle through= center A through B]}
\TOline{disk through=center pt1 through pt2}{}{[disk through=center A through B]}
\end{tabular}
\end{NewMacroBox} 

\subsection{Random point in a rectangle} 

\begin{tkzexample}[latex=7cm,small]
\begin{tikzpicture}
  \tkzInit[xmax=5,ymax=5]\tkzGrid
  \tkzDefPoints{0/0/A,2/2/B,5/5/C}
  \tkzDefRandPointOn[rectangle = A and B]
  \tkzGetPoint{a}
  \tkzDefRandPointOn[rectangle = B and C]
  \tkzGetPoint{d}
  \tkzDrawLine(a,d)
  \tkzDrawPoints(A,B,C,a,d) 
  \tkzLabelPoints(A,B,C,a,d)  
\end{tikzpicture} 
\end{tkzexample} 

\subsection{Random point on a segment}  
\begin{tkzexample}[latex=7cm,small]
\begin{tikzpicture}  
  \tkzInit[xmax=5,ymax=5] \tkzGrid 
  \tkzDefPoints{0/0/A,2/2/B,3/3/C,5/5/D}  
  \tkzDefRandPointOn[segment = A--B]\tkzGetPoint{a}
  \tkzDefRandPointOn[segment = C--D]\tkzGetPoint{d}
  \tkzDrawPoints(A,B,C,D,a,d) 
  \tkzLabelPoints(A,B,C,D,a,d)
\end{tikzpicture}
\end{tkzexample}

\subsection{Random point on a straight line}
\begin{tkzexample}[latex=7cm,small]
\begin{tikzpicture}
  \tkzInit[xmax=5,ymax=5] \tkzGrid
  \tkzDefPoints{0/0/A,2/2/B,3/3/C,5/5/D}  
  \tkzDefRandPointOn[line = A--B]\tkzGetPoint{E}
  \tkzDefRandPointOn[line = C--D]\tkzGetPoint{F}
  \tkzDrawPoints(A,...,F)
  \tkzLabelPoints(A,...,F)
\end{tikzpicture}
\end{tkzexample}


\subsubsection{Example of random points}
\begin{tkzexample}[latex=7cm,small]
\begin{tikzpicture}
 \tkzDefPoints{0/0/A,2/2/B,-1/-1/C}
 \tkzDefCircle[through=](A,C) 
 \tkzGetLength{rAC}
 \tkzDrawCircle(A,C)
 \tkzDrawCircle(A,B)
 \tkzDefRandPointOn[rectangle=A and B]
 \tkzGetPoint{a}
 \tkzDefRandPointOn[segment=A--B]
 \tkzGetPoint{b}
 \tkzDefRandPointOn[circle=center A radius \rAC pt]
    \tkzGetPoint{d}
 \tkzDefRandPointOn[circle through= center A through B]
     \tkzGetPoint{c}
 \tkzDefRandPointOn[disk through=center A through B]
     \tkzGetPoint{e}
 \tkzLabelPoints[above right=3pt](A,B,C,a,b,...,e)
 \tkzDrawPoints[](A,B,C,a,b,...,e)
 \tkzDrawRectangle(A,B)
\end{tikzpicture}
\end{tkzexample}

\subsection{Random point on a circle}
\begin{tkzexample}[latex=7cm,small]
\begin{tikzpicture} 
  \tkzInit[xmax=5,ymax=5]  \tkzGrid   
  \tkzDefPoints{3/2/A,1/1/B}
  \tkzCalcLength[cm](A,B) \tkzGetLength{rAB}
  \tkzDrawCircle[R](A,\rAB cm) 
  \tkzDefRandPointOn[circle = center A radius
   \rAB cm]\tkzGetPoint{a}
  \tkzDrawSegment(A,a)
  \tkzDrawPoints(A,B,a) 
  \tkzLabelPoints(A,B,a)  
\end{tikzpicture}
\end{tkzexample}

\subsubsection{Random example and circle of Apollonius}
\begin{tkzexample}[latex=7cm,small]
\begin{tikzpicture}[scale=1]
 \tkzDefPoints{0/0/A,3/0/B}
 \def\coeffK{2}
 \tkzApolloniusCenter[K=\coeffK](A,B) 
 \tkzGetPoint{P}
 \tkzDefApolloniusPoint[K=\coeffK](A,B) 
 \tkzGetPoint{M}
 \tkzDefApolloniusRadius[K=\coeffK](A,B)
 \tkzDrawCircle[R,color = blue!50!black,
     fill=blue!20,
     opacity=.4](tkzPointResult,\tkzLengthResult pt)
 \tkzDefRandPointOn[circle through= center P through M]
 \tkzGetPoint{N}
 \tkzDrawPoints(A,B,P,M,N)
 \tkzLabelPoints(A,B,P,M,N)
 \tkzDrawSegments[red](N,A N,B)
 \tkzDrawPoints(A,B)
 \tkzDrawSegments[red](A,B)
 \tkzLabelCircle[R,draw,fill=green!10,%
     text width=3cm,%
     text centered](P,\tkzLengthResult pt-20pt)(-120)%
  { $MA/MB=\coeffK$\\$NA/NB=\coeffK$}
\end{tikzpicture}
\end{tkzexample}



\subsection{Middle of a compass segment}
 To conclude this section, here is a more complex example. It involves determining the middle of a segment, using only a compass. 

\begin{tkzexample}[latex=7cm,small]
\begin{tikzpicture}[scale=.75]
  \tkzDefPoint(0,0){A}
  \tkzDefRandPointOn[circle= center A radius 4cm]
  \tkzGetPoint{B}
  \tkzDrawPoints(A,B)
  \tkzDefPointBy[rotation= center A angle 180](B) 
  \tkzGetPoint{C}
  \tkzInterCC[R](A,4 cm)(B,4 cm)
  \tkzGetPoints{I}{I'}
  \tkzInterCC[R](A,4 cm)(I,4 cm)
  \tkzGetPoints{J}{B}
  \tkzInterCC(B,A)(C,B)
  \tkzGetPoints{D}{E}
  \tkzInterCC(D,B)(E,B)
  \tkzGetPoints{M}{M'}
  \tikzset{arc/.style={color=brown,style=dashed,delta=10}}
  \tkzDrawArc[arc](C,D)(E)
  \tkzDrawArc[arc](B,E)(D)
  \tkzDrawCircle[color=brown,line width=.2pt](A,B)
  \tkzDrawArc[arc](D,B)(M) 
  \tkzDrawArc[arc](E,M)(B)
  \tkzCompasss[color=red,style=solid](B,I I,J J,C)
  \tkzDrawPoints(B,C,D,E,M)
  \tkzLabelPoints(A,B,M)
 \end{tikzpicture}
 \end{tkzexample}
   
\endinput

 

\part{Drawing and Filling}
\section{Drawing}
\tkzname{\tkznameofpack} can draw 5 types of objects : point, line or line segment, circle, arc and sector.

%<---------------------------------------------------------------------------->
%    POINT(S)
%<---------------------------------------------------------------------------->
\subsection{Draw a point or some points}
There are two possibilities : \tkzcname{tkzDrawPoint} for a single point or \tkzcname{tkzDrawPoints} for one or more points.

\subsubsection{Drawing points \tkzcname{tkzDrawPoint}} \hypertarget{tdrp}{}

\begin{NewMacroBox}{tkzDrawPoint}{\oarg{local options}\parg{name}}%
\begin{tabular}{lll}%
arguments &  default & definition                 \\
\midrule
\TAline{name of point} {no default}  {Only one point name is accepted}
\bottomrule
\end{tabular}

\medskip
The argument is required. The disc takes the color of the circle, but  lighter. It is possible to change everything. The point is a node and therefore it is invariant if the drawing is modified by scaling.

\medskip
\begin{tabular}{lll}%
\toprule
options             & default & definition \\
\midrule
\TOline{\TIKZ\ options}{}{all \TIKZ\ options are valid.}
\TOline{shape}  {circle}{Possible \tkzname{cross} or \tkzname{cross out}}
\TOline{size}   {6}{$6 \times$ \tkzcname{pgflinewidth}}
\TOline{color}  {black}{the default color can be changed }
\bottomrule
\end{tabular}

\medskip
{We can create other forms such as \tkzname{cross}}
\end{NewMacroBox}

By default, \tkzname{point style } is defined  like this :

\begin{tkzltxexample}[]
  \tikzset{point style/.style = {%
           draw         = black,
           inner sep    = 0pt,
           shape        = circle,
           minimum size = 3 pt,
           fill         = black
                               }
        } 
\end{tkzltxexample}

\subsubsection{Example of point drawings}
Note that \tkzname{scale} does not affect the shape of the dots. Which is normal.  Most of the time, we are satisfied with a single point shape that we can define from the beginning, either with a macro or by modifying a configuration file.

\begin{tkzexample}[latex=5cm,small]
  \begin{tikzpicture}[scale=.5]
   \tkzDefPoint(1,3){A}
   \tkzDefPoint(4,1){B}
   \tkzDefPoint(0,0){O}
   \tkzDrawPoint[color=red](A)
   \tkzDrawPoint[fill=blue!20,draw=blue](B)
   \tkzDrawPoint[shape=cross,size=8pt,color=teal](O)
  \end{tikzpicture}
\end{tkzexample}

It is possible to draw several points at once but this macro is a little slower than the previous one. Moreover, we have to make do with the same options for all the points.
\newpage
\hypertarget{tdrps}{}
\begin{NewMacroBox}{tkzDrawPoints}{\oarg{local options}\parg{liste}}%
\begin{tabular}{lll}%
arguments &  default  & definition \\
\midrule
\TAline{points list}{no default}{example \tkzcname{tkzDrawPoints(A,B,C)}}
\bottomrule
\end{tabular}

\medskip
\begin{tabular}{lll}%
options             & default & definition \\
\midrule
\TOline{shape}  {circle}{Possible \tkzname{cross} or \tkzname{cross out}}
\TOline{size}  {6}{$6 \times$ \tkzcname{pgflinewidth}}
\TOline{color}  {black}{the default color can be changed }
\bottomrule
\end{tabular}

\medskip
\tkzHandBomb\ Beware of the final "s", an oversight leads to cascading errors if you try to draw multiple points. The options are the same as for the previous macro.
\end{NewMacroBox}

\subsubsection{Example}

\begin{tkzexample}[latex=7cm,small]
\begin{tikzpicture}
\tkzDefPoints{1/3/A,4/1/B,0/0/C}
\tkzDrawPoints[size=3,color=red,fill=red!50](A,B,C)
\end{tikzpicture}
\end{tkzexample}
%<---------------------------------------------------------------------------->
%    LINE(S)
%<---------------------------------------------------------------------------->
\section{Drawing the lines}
The following macros are simply used to draw, name lines.
\subsection{Draw a straight line}
To draw a normal straight line, just give a couple of points. You can  use the \tkzname{add} option to extend the line (This option is due to \tkzimp{Mark Wibrow}, see the code below). 

The style of a line is by default :

\begin{tkzltxexample}[]
  \tikzset{line style/.style = {%
    line width = 0.6pt,
    color      = black,
    style      = solid,
    add        = {.2} and  {.2}%
   }}
\end{tkzltxexample}
with
   
\begin{tkzltxexample}[]
  \tikzset{%
    add/.style args={#1 and #2}{
        to path={%
 ($(\tikztostart)!-#1!(\tikztotarget)$)--($(\tikztotarget)!-#2!(\tikztostart)$)%
  \tikztonodes}}}
\end{tkzltxexample}

You can modify this style with \tkzcname{tkzSetUpLine} see \ref{tkzsetupline}

\newpage
\begin{NewMacroBox}{tkzDrawLine}{\oarg{local options}\parg{pt1,pt2} }%
The arguments are a list of two points or three points. It would be possible, as for a half line, to create a style with \tkzcname{add}.

\begin{tabular}{lll}%
\toprule
options             & default & definition                         \\ 
\midrule
\TOline{\TIKZ\ options}{}{all \TIKZ\ options are valid.}
\TOline{add}{0.2 and 0.2}{add = $kl$ and $kr$, \dots}
\TOline{\dots}{\dots}{allows the segment to be extended to the left and right. }
\bottomrule
\end{tabular}

\tkzname{add} defines the length of the line passing through the points pt1 and pt2. Both numbers are percentages. The styles of \TIKZ\ are accessible for plots.
\end{NewMacroBox}

\subsubsection{Examples  with \tkzname{add}}
\begin{tkzexample}[latex=5cm,small]
\begin{tikzpicture}
 \tkzInit[xmin=-2,xmax=3,ymin=-2.25,ymax=2.25]
 \tkzClip[space=.25]
 \tkzDefPoint(0,0){A} \tkzDefPoint(2,0.5){B}
 \tkzDefPoint(0,-1){C}\tkzDefPoint(2,-0.5){D} 
 \tkzDefPoint(0,1){E} \tkzDefPoint(2,1.5){F} 
 \tkzDefPoint(0,-2){G} \tkzDefPoint(2,-1.5){H}
 \tkzDrawLine(A,B)    \tkzDrawLine[add = 0 and .5](C,D) 
 \tkzDrawLine[add = 1 and 0](E,F)
 \tkzDrawLine[add = 0 and 0](G,H) 
 \tkzDrawPoints(A,B,C,D,E,F,G,H)    
 \tkzLabelPoints(A,B,C,D,E,F,G,H)  
\end{tikzpicture}
\end{tkzexample} 

It is possible to draw several lines, but with the same options. 
\begin{NewMacroBox}{tkzDrawLines}{\oarg{local options}\parg{pt1,pt2 pt3,pt4 ...}}% 
Arguments are a list of pairs of  points separated by spaces.  The styles of \TIKZ\ are available for the draws. 
\end{NewMacroBox}      

\subsubsection{Example with \tkzcname{tkzDrawLines}}    

\begin{tkzexample}[latex=8cm,small]
\begin{tikzpicture}
  \tkzDefPoint(0,0){A}
  \tkzDefPoint(2,0){B}
  \tkzDefPoint(1,2){C}
  \tkzDefPoint(3,2){D}   
  \tkzDrawLines(A,B C,D A,C B,D)
  \tkzLabelPoints(A,B,C,D)
\end{tikzpicture}
\end{tkzexample}
%<---------------------------------------------------------------------------->
%    SEGMENT(S)
%<---------------------------------------------------------------------------->
\section{Drawing a segment}
There is, of course, a macro to simply draw a segment.

\subsection{Draw a segment \tkzcname{tkzDrawSegment}} 
\begin{NewMacroBox}{tkzDrawSegment}{\oarg{local options}\parg{pt1,pt2}}%
The arguments are a list of two points. The styles of \TIKZ\ are available for the drawings.
 
\medskip
\begin{tabular}{lll}%
argument    & example & definition    \\
\midrule
\TAline{(pt1,pt2)}{(A,B)}{draw the segment $[A,B]$}
\bottomrule 
\end{tabular}
 
\medskip
\begin{tabular}{lll}%
options    & example & definition    \\
\midrule
\TOline{\TIKZ\ options}{}{all \TIKZ\ options are valid.}
\TOline{dim}{no default}{dim = \{label,dim,option\}, \dots}
\TOline{\dots}{\dots}{allows you to add dimensions to a figure.}
\bottomrule 
\end{tabular}

This is of course equivalent to \tkzcname{draw (A)--(B);}. You can also use the option \tkzname{add}.
\end{NewMacroBox}

\subsubsection{Example with point references}     

\begin{tkzexample}[latex=6cm,small]
\begin{tikzpicture}[scale=1.5]
  \tkzDefPoint(0,0){A}
  \tkzDefPoint(2,1){B}
  \tkzDrawSegment[color=red,thin](A,B)
  \tkzDrawPoints(A,B)    
  \tkzLabelPoints(A,B)  
\end{tikzpicture}
\end{tkzexample}

\subsubsection{Example of extending an segment with option \tkzname{add}} 

\begin{tkzexample}[latex=7cm,small]
\begin{tikzpicture}
  \tkzDefPoints{0/0/A,6/0/B,0.8/4/C}
  \tkzDefTriangleCenter[euler](A,B,C) 
  \tkzGetPoint{E}
  \tkzDefCircle[euler](A,B,C)\tkzGetPoints{E}{e}
  \tkzDrawCircle[red](E,e)
  \tkzDrawLines[add=.5 and .5](A,B A,C B,C)
  \tkzDrawPoints(A,B,C,E)
  \tkzLabelPoints(A,B,C,E)
  \end{tikzpicture}
\end{tkzexample}

\subsubsection{Adding dimensions with option \tkzname{dim} new code from Muzimuzhi Z} 
This code comes from an answer to this question on tex.stackexchange.com 
(change-color-and-style-of-dimension-lines-in-tkz-euclide ).
The code of \tkzname{dim} is based on options of TikZ, you must add the units.
You can use now two styles : |dim style| and |dim fence style|. You have several ways to use them.
I'll let you look at the examples to see what you can do with these styles.

\begin{verbatim}
   \tikzset{dim style/.append style={dashed}} % append if you want to keep precedent style.
   or 
   \begin{scope}[ dim style/.append style={orange},
       dim fence style/.style={dashed}]
\end{verbatim}

\begin{tkzexample}[latex=7cm]
\begin{tikzpicture}[scale=.75]
  \tkzDefPoints{0/3/A, 1/-3/B}
  \tkzDrawPoints(A,B)
  \tkzDrawSegment[dim={\(l_0\),1cm,right=2mm}, 
    dim style/.append style={red, 
    dash pattern={on 2pt off 2pt}}](A,B)
  \tkzDrawSegment[dim={\(l_1\),2cm,right=2mm}, 
    dim style/.append style={blue}](A,B)
  \begin{scope}[ dim style/.style={orange},
      dim fence style/.style={dashed}]
    \tkzDrawSegment[dim={\(l_2\),3cm,right=2mm}](A,B)  
    \tkzDrawSegment[dim={\(l_3\),-2cm,right=2mm}](A,B)   
  \end{scope}  
  \tkzLabelPoints[left](A,B)
\end{tikzpicture}
\end{tkzexample}

\subsubsection{Adding dimensions with option \tkzname{dim} partI} 
\begin{tkzexample}[latex=7cm,small]
\begin{tikzpicture}[scale=2]
\pgfkeys{/pgf/number format/.cd,fixed,precision=2}
\tkzDefPoint(0,0){A}
\tkzDefPoint(3.07,0){B}
\tkzInterCC[R](A,2.37)(B,1.82)
\tkzGetPoints{C}{C'}
\tkzDefCircle[in](A,B,C) \tkzGetPoints{G}{g}
\tkzDrawCircle(G,g)
\tkzDrawPolygon(A,B,C)
\tkzDrawPoints(A,B,C)
\tkzCalcLength(A,B)\tkzGetLength{ABl}
\tkzCalcLength(B,C)\tkzGetLength{BCl}
\tkzCalcLength(A,C)\tkzGetLength{ACl}
\begin{scope}[dim style/.style={dashed,sloped,teal}]
  \tkzDrawSegment[dim={\pgfmathprintnumber\BCl,6pt,
                                          text=red}](C,B)
  \tkzDrawSegment[dim={\pgfmathprintnumber\ACl,6pt,}](A,C)
  \tkzDrawSegment[dim={\pgfmathprintnumber\ABl,-6pt,}](A,B)
\end{scope}
\tkzLabelPoints(A,B) \tkzLabelPoints[above](C)
\end{tikzpicture}
\end{tkzexample}

\subsubsection{Adding dimensions with option \tkzname{dim} part II} 
\begin{tkzexample}[latex=7cm,small]
\begin{tikzpicture}[scale=.75]
  \tkzDefPoints{0/0/O,-2/0/A,2/0/B,
                -2/4/C,2/4/D,2/-4/E,-2/-4/F}
  \tkzDrawPolygon(C,...,F)
  \tkzDrawSegments(A,B)
  \tkzDrawPoints(A,...,F,O)
  \tkzLabelPoints[below left](A,...,F,O)
  \tkzDrawSegment[dim={ $\sqrt{5}$,2cm,}](C,E)
  \tkzDrawSegment[dim={ $\frac{\sqrt{5}}{2}$,1cm,}](O,E)
  \tkzDrawSegment[dim={ $2$,2cm,left=8pt}](F,C)
  \tkzDrawSegment[dim={ $1$,1cm,left=8pt}](F,A)
\end{tikzpicture}
\end{tkzexample}

\subsection{Drawing segments \tkzcname{tkzDrawSegments}} 
If the options are the same we can plot several segments with the same macro. 

\begin{NewMacroBox}{tkzDrawSegments}{\oarg{local options}\parg{pt1,pt2 pt3,pt4 ...}}%
The arguments are a two-point couple list. The styles of \TIKZ\ are available for the plots.
\end{NewMacroBox}

\begin{tkzexample}[latex=6cm,small]
\begin{tikzpicture}
  \tkzInit[xmin=-1,xmax=3,ymin=-1,ymax=2]
  \tkzClip[space=1]
  \tkzDefPoint(0,0){A}
  \tkzDefPoint(2,1){B} 
  \tkzDefPoint(3,0){C} 
  \tkzDrawSegments(A,B B,C)
  \tkzDrawPoints(A,B,C)    
  \tkzLabelPoints(A,C) 
  \tkzLabelPoints[above](B)  
\end{tikzpicture}
\end{tkzexample}

\subsubsection{Place an arrow on segment}
\begin{tkzexample}[latex=6cm,small]
\begin{tikzpicture}
\tkzSetUpStyle[postaction=decorate,
    decoration={markings, 
    mark=at position .5 with {\arrow[thick]{#1}}
      }]{myarrow}
  \tkzDefPoint(0,0){A}
  \tkzDefPoint(4,-4){B}
  \tkzDrawSegments[myarrow=stealth](A,B)
  \tkzDrawPoints(A,B) 
\end{tikzpicture}
\end{tkzexample}

\subsection{Drawing line segment of a triangle}

\subsubsection{How to draw \tkzname{Altitude} } 
\begin{tkzexample}[latex=7cm,small]
  \begin{tikzpicture}[rotate=-90]
  \tkzDefPoint(0,1){A}
  \tkzDefPoint(2,4){C}
  \tkzDefPointWith[orthogonal normed,K=7](C,A)
  \tkzGetPoint{B}
  \tkzDefSpcTriangle[orthic,name=H](A,B,C){a,b,c}
  \tkzDrawLine[dashed,color=magenta](C,Hc)
  \tkzDrawSegment[green!60!black](A,C)
  \tkzDrawSegment[green!60!black](C,B)
  \tkzDrawSegment[green!60!black](B,A)
  \tkzLabelPoint[left](A){$A$}
  \tkzLabelPoint[right](B){$B$}
  \tkzLabelPoint[above](C){$C$}
  \tkzLabelPoint[left](Hc){$Hc$}
  \tkzLabelSegment[auto](B,A){$c$}
  \tkzLabelSegment[auto,swap](B,C){$a$}
  \tkzLabelSegment[auto,swap](C,A){$b$}
  \tkzMarkAngle[size=1,color=cyan,mark=|](C,B,A)
  \tkzMarkAngle[size=1,color=cyan,mark=|](A,C,Hc)
  \tkzMarkAngle[size=0.75,
                color=orange,mark=||](Hc,C,B)
  \tkzMarkAngle[size=0.75,
                color=orange,mark=||](B,A,C)
  \tkzMarkRightAngle(A,C,B)
  \tkzMarkRightAngle(B,Hc,C)
  \end{tikzpicture} 
\end{tkzexample}

\subsection{Drawing a polygon} 
 \begin{NewMacroBox}{tkzDrawPolygon}{\oarg{local options}\parg{points list}}%
Just give a list of points and the macro plots the polygon using the \TIKZ\ options present. You can  replace $(A,B,C,D,E)$ by $(A,...,E)$ and $(P_1,P_2,P_3,P_4,P_5)$ by $(P_1,P...,P_5)$

\begin{tabular}{lll}%
\toprule
arguments             & example & explanation                         \\
\midrule
\TAline{\parg{pt1,pt2,pt3,...}}{|\BS tkzDrawPolygon[gray,dashed](A,B,C)|}{Drawing a triangle}
\end{tabular}

\medskip
\begin{tabular}{lll}%
\toprule
options             & default & example                         \\
\midrule
\TOline{Options TikZ}{...}{|\BS tkzDrawPolygon[red,line width=2pt](A,B,C)|}
 \end{tabular} 
\end{NewMacroBox}

\subsubsection{\tkzcname{tkzDrawPolygon}}

\begin{tkzexample}[latex=7cm, small]  
\begin{tikzpicture} [rotate=18,scale=1]
 \tkzDefPoints{0/0/A,2.25/0.2/B,2.5/2.75/C,-0.75/2/D}
 \tkzDrawPolygon(A,B,C,D)
 \tkzDrawSegments[style=dashed](A,C B,D) 
\end{tikzpicture}
\end{tkzexample}

\subsubsection{Option \tkzname{two angles}}
\begin{tkzexample}[latex=6 cm,small]
\begin{tikzpicture}
\tkzDefPoint(0,0){A} 
\tkzDefPoint(6,0){B} 
\tkzDefTriangle[two angles = 50 and 70](A,B) \tkzGetPoint{C}
\tkzDrawPolygon(A,B,C)
\tkzLabelAngle[pos=1.4](B,A,C){$50^\circ$}
\tkzLabelAngle[pos=0.8](C,B,A){$70^\circ$}
\end{tikzpicture}
\end{tkzexample}

\subsubsection{Style of line}
\begin{tkzexample}[latex=8 cm,small]
\begin{tikzpicture}[scale=.6]
\tkzSetUpLine[line width=5mm,color=teal]
\tkzDefPoint(0,0){O}
\foreach \i in {0,...,5}{%
 \tkzDefPoint({30+60*\i}:4){p\i}}
\tkzDefMidPoint(p1,p3) \tkzGetPoint{m1}
\tkzDefMidPoint(p3,p5) \tkzGetPoint{m3}
\tkzDefMidPoint(p5,p1) \tkzGetPoint{m5}
\tkzDrawPolygon[line join=round](p1,p3,p5)
\tkzDrawPolygon[teal!80,
line join=round](p0,p2,p4)
\tkzDrawSegments(m1,p3 m3,p5 m5,p1)
\tkzDefCircle[R](O,4.8)\tkzGetPoint{o}
\tkzDrawCircle[teal](O,o)
\end{tikzpicture}
\end{tkzexample}

\subsection{Drawing a polygonal chain} 
 \begin{NewMacroBox}{tkzDrawPolySeg}{\oarg{local options}\parg{points list}}%
Just give a list of points and the macro plots the polygonal chain using the \TIKZ\ options present.

\begin{tabular}{lll}%
\toprule
arguments             & example & explanation                         \\
\midrule
\TAline{\parg{pt1,pt2,pt3,...}}{|\BS tkzDrawPolySeg[gray,dashed](A,B,C)|}{Drawing a triangle}
\end{tabular}

\medskip
\begin{tabular}{lll}%
\toprule
options             & default & example                         \\
\midrule
\TOline{Options TikZ}{...}{|\BS tkzDrawPolySeg[red,line width=2pt](A,B,C)|}
 \end{tabular} 
\end{NewMacroBox}

\subsubsection{Polygonal chain}

\begin{tkzexample}[latex=7cm, small]  
\begin{tikzpicture}
 \tkzDefPoints{0/0/A,6/0/B,3/4/C,2/2/D}          
 \tkzDrawPolySeg(A,...,D)
 \tkzDrawPoints(A,...,D)
\end{tikzpicture}
\end{tkzexample}

\subsubsection{The idea is to inscribe two squares in a semi-circle.}
A Sangaku look! It is a question of proving that one can inscribe in a half-disc, two squares, and to determine the length of their respective sides according to the radius.

\begin{tkzexample}[latex=7 cm,small]
\begin{tikzpicture}[scale=.75] 
  \tkzDefPoints{0/0/A,8/0/B,4/0/I}
  \tkzDefSquare(A,B)    \tkzGetPoints{C}{D} 
  \tkzInterLC(I,C)(I,B) \tkzGetPoints{E'}{E} 
  \tkzInterLC(I,D)(I,B) \tkzGetPoints{F'}{F} 
  \tkzDefPointsBy[projection=onto A--B](E,F){H,G} 
  \tkzDefPointsBy[symmetry = center H](I){J} 
  \tkzDefSquare(H,J)     \tkzGetPoints{K}{L} 
  \tkzDrawSector(I,B)(A) 
  \tkzDrawPolySeg(H,E,F,G) 
  \tkzDrawPolySeg(J,K,L) 
  \tkzDrawPoints(E,G,H,F,J,K,L)
\end{tikzpicture}
\end{tkzexample}

\subsubsection{Polygonal chain: index notation}

\begin{tkzexample}[latex=7cm, small]  
\begin{tikzpicture}
\foreach \pt in {1,2,...,8} {%
\tkzDefPoint(\pt*20:3){P_\pt}}     
\tkzDrawPolySeg(P_1,P_...,P_8)
\tkzDrawPoints(P_1,P_...,P_8)
\end{tikzpicture}
\end{tkzexample}
%<---------------------------------------------------------------------------->
%    CIRCLE
%<---------------------------------------------------------------------------->
\section{Draw a circle with \tkzcname{tkzDrawCircle}}

\subsection{Draw one circle}
\begin{NewMacroBox}{tkzDrawCircle}{\oarg{local options}\parg{A,B}}%
\tkzHandBomb\ Attention you need only two points to define a radius.  An additional option \tkzname{R} is available  to give a measure directly.

\medskip
\begin{tabular}{lll}%
\toprule
arguments           & example & explanation                         \\
\midrule
\TAline{\parg{pt1,pt2}}{\parg{A,B}} {A center through B}
 \bottomrule
\end{tabular}   

\medskip
Of course, you have to add all the styles of \TIKZ\ for the tracings...
\end{NewMacroBox}
 
 \subsubsection{Circles and styles, draw a circle and color the disc}
 We'll see that it's possible to colour in a disc while tracing the circle.
 
\begin{tkzexample}[latex=7cm,small]
\begin{tikzpicture}
  \tkzDefPoint(0,0){O} 
  \tkzDefPoint(3,0){A}
 % circle with center O and passing through A
  \tkzDrawCircle(O,A) 
 % diameter circle $[OA]$
 \tkzDefCircle[diameter](O,A) \tkzGetPoint{I}
 \tkzDrawCircle[new,fill=orange!10,opacity=.5](I,A)
 % circle with center O and radius = exp(1) cm
  \edef\rayon{\fpeval{0.25*exp(1)}}
  \tkzDefCircle[R](O,\rayon) \tkzGetPoint{o}
   \tkzDrawCircle[color=orange](O,o) 
\end{tikzpicture} 
\end{tkzexample}  

\subsection{Drawing circles}  
\begin{NewMacroBox}{tkzDrawCircles}{\oarg{local options}\parg{A,B C,D \dots}}%
\tkzHandBomb\ Attention, the arguments are lists of two points. The circles that can be drawn are the same as in the previous macro. An additional option \tkzname{R} is available to give  a measure directly.

\medskip
\begin{tabular}{lll}%
\toprule
arguments           & example & explanation                         \\
\midrule
\TAline{\parg{pt1,pt2 pt3,pt4 ...}}{\parg{A,B C,D}} {List of two points}
\bottomrule
\end{tabular}   

\medskip
\begin{tabular}{lll}%
\toprule
options             & default & definition                         \\ 
\midrule
\TOline{through}{through}{circle with two points defining a radius}
 \bottomrule
\end{tabular}

\medskip
You do not need to use the default option \tkzname{through}.
Of course, you have to add all the styles of \TIKZ\ for the tracings...
\end{NewMacroBox}

 \subsubsection{Circles defined by a triangle.} 
 
\begin{tkzexample}[latex=9cm,small]
\begin{tikzpicture}
  \tkzDefPoints{0/0/A,2/0/B,3/2/C}
  \tkzDrawPolygon(A,B,C)
  \tkzDrawCircles(A,B B,C C,A)
  \tkzDrawPoints(A,B,C)
  \tkzLabelPoints(A,B,C) 
\end{tikzpicture} 
\end{tkzexample}

\subsubsection{Concentric circles.} 
 
\begin{tkzexample}[latex=7cm,small]
\begin{tikzpicture}
   \tkzDefPoints{0/0/A,1/0/a,2/0/b,3/0/c}
   \tkzDrawCircles(A,a A,b A,c)
   \tkzDrawPoint(A)
   \tkzLabelPoints(A)
\end{tikzpicture}
\end{tkzexample}

\subsubsection{Exinscribed circles.} 

\begin{tkzexample}[latex=8cm,small] 
\begin{tikzpicture}[scale=1] 
\tkzDefPoints{0/0/A,4/0/B,1/2.5/C}
\tkzDrawPolygon(A,B,C)
\tkzDefCircle[ex](B,C,A) 
\tkzGetPoint{J_c} \tkzGetSecondPoint{T_c}
\tkzDrawCircle(J_c,T_c)
\tkzDrawLines[add=0 and 1](C,A C,B)
\tkzDrawSegment(J_c,T_c)
\tkzMarkRightAngle(J_c,T_c,B)
\tkzDrawPoints(A,B,C,J_c,T_c)
\end{tikzpicture}
\end{tkzexample}
 
\subsubsection{Cardioid}  
Based on an idea by O. Reboux made with pst-eucl (Pstricks module) by D. Rodriguez.

 Its name comes from the Greek \textit{kardia (heart)}, in reference to its shape, and was given to it by Johan Castillon (Wikipedia).     
 
\begin{tkzexample}[latex=7cm,small]
\begin{tikzpicture}[scale=.5]
  \tkzDefPoint(0,0){O} 
  \tkzDefPoint(2,0){A}
  \foreach \ang in {5,10,...,360}{%
     \tkzDefPoint(\ang:2){M}
     \tkzDrawCircle(M,A) 
   }  
\end{tikzpicture} 
\end{tkzexample}

\newpage

\subsection{Drawing semicircle}
\begin{NewMacroBox}{tkzDrawSemiCircle}{\oarg{local options}\parg{O,A}}%

\medskip
\begin{tabular}{lll}%
\toprule
arguments           & example & explanation                         \\
\midrule
\TAline{\parg{pt1,pt2}}{\parg{O,A}} {OA= radius}
\bottomrule
\end{tabular} 
    
$O$ center $A$ extremity of the semicircle
\end{NewMacroBox}  

\subsubsection{Use of \tkzcname{tkzDrawSemiCircle}}   

\begin{tkzexample}[latex=7cm,small]
\begin{tikzpicture}
   \tkzDefPoint(0,0){A} \tkzDefPoint(6,0){B}
   \tkzDefMidPoint(A,B)  \tkzGetPoint{O}
   \tkzDrawSemiCircle[blue](O,B)
   \tkzDrawSemiCircle[red](O,A)
   \tkzDrawPoints(O,A,B)
   \tkzLabelPoints[below right](O,A,B)
 \end{tikzpicture}
\end{tkzexample}

\subsection{Drawing semicircles}

\begin{NewMacroBox}{tkzDrawSemiCircles}{\oarg{local options}\parg{A,B C,D \dots}}%

\medskip
\begin{tabular}{lll}%
\toprule
arguments           & example & explanation                         \\
\midrule
\TAline{\parg{pt1,pt2 pt3,pt4 ...}}{\parg{A,B C,D}} {List of two points}
\bottomrule
\end{tabular} 
    
\end{NewMacroBox}  

\subsubsection{Use of \tkzcname{tkzDrawSemiCircles} : Golden arbelos}  

\begin{tkzexample}[latex=7cm,small]
\begin{tikzpicture}[scale=.75]
\tkzDefPoints{0/0/A,10/0/B}
\tkzDefGoldenRatio(A,B) \tkzGetPoint{C}
\tkzDefMidPoint(A,B)                     \tkzGetPoint{O_0}
\tkzDefMidPoint(A,C)                     \tkzGetPoint{O_1}
\tkzDefMidPoint(C,B)                     \tkzGetPoint{O_2}
\tkzLabelPoints(A,B,C)
\tkzDrawSegment(A,B)
\tkzDrawPoints(A,B,C)
\begin{scope}[local bounding box = graph]
  \tkzDrawSemiCircles[color=black](O_0,B)
\end{scope}
\useasboundingbox (graph.south west) rectangle (graph.north east);
\tkzClipCircle[out](O_1,C)\tkzClipCircle[out](O_2,B)
\tkzDrawSemiCircles[draw=none,fill=teal!15](O_0,B)
\tkzDrawSemiCircles[color=black](O_1,C O_2,B)
\end{tikzpicture}
\end{tkzexample}

%<---------------------------------------------------------------------------->
%    ARC
%<---------------------------------------------------------------------------->
\section{Drawing arcs} 
\subsection{Macro: \tkzcname{tkzDrawArc} }
\begin{NewMacroBox}{tkzDrawArc}{\oarg{local options}\parg{O,\dots}\parg{\dots}}%
This macro traces the arc of center $O$. Depending on the options, the arguments differ.   It is a question of determining a starting point and an end point. Either the starting point is given, which is the simplest, or the radius of the arc is given. In the latter case, it is necessary to have two angles. Either the angles can be given directly, or nodes associated with the center can be given to determine them. The angles are in degrees.

\medskip
\begin{tabular}{lll}%
\toprule
options             & default & definition                        \\ 
\midrule
\TOline{towards}{towards}{$O$ is the center and the arc from $A$ to $(OB)$} 
\TOline{rotate} {towards}{the arc starts from $A$ and the angle determines its length} 
\TOline{R}{towards}{We give the radius and two angles} 
\TOline{R with nodes}{towards}{We give the radius and two points}
\TOline{angles}{towards}{We give the radius and two points}
\TOline{delta}{0}{angle added on each side }
\TOline{reverse}{false}{inversion of the arc's path, interesting to inverse arrow} 
\bottomrule
\end{tabular}

\medskip
Of course, you have to add all the styles of \TIKZ\ for the tracings...

\medskip

\begin{tabular}{lll}%
\toprule
options             & arguments & example                         \\ 
\midrule
\TOline{towards}{\parg{pt,pt}\parg{pt}}{\tkzcname{tkzDrawArc[delta=10](O,A)(B)}} 
\TOline{rotate} {\parg{pt,pt}\parg{an}}{\tkzcname{tkzDrawArc[rotate,color=red](O,A)(90)}}
\TOline{R}{\parg{pt,$r$}\parg{an,an}}{\tkzcname{tkzDrawArc[R](O,2)(30,90)}}
\TOline{R with nodes}{\parg{pt,$r$}\parg{pt,pt}}{\tkzcname{tkzDrawArc[R with nodes](O,2)(A,B)}}
\TOline{angles}{\parg{pt,pt}\parg{an,an}}{\tkzcname{tkzDrawArc[angles](O,A)(0,90)}}
\end{tabular}
\end{NewMacroBox}

Here are a few examples: 

\subsubsection{Option \tkzname{towards}}
It's useless to put \tkzname{towards}. In this first example the arc starts from $A$ and goes to $B$. The arc going from $B$ to $A$ is different. The salient is obtained by going in the direct direction of the trigonometric circle.
\begin{tkzexample}[latex=6cm,small]
\begin{tikzpicture}[scale=.75]
  \tkzDefPoint(0,0){O}
  \tkzDefPoint(2,-1){A}
  \tkzDefPointBy[rotation= center O angle 90](A)
  \tkzGetPoint{B}
  \tkzDrawArc[color=orange,<->](O,A)(B) 
  \tkzDrawArc(O,B)(A)
  \tkzDrawLines[add = 0 and .5](O,A O,B)
  \tkzDrawPoints(O,A,B)
  \tkzLabelPoints[below](O,A,B)  
\end{tikzpicture}
\end{tkzexample}

\subsubsection{Option \tkzname{towards}}
In this one, the arc starts from A but stops on the right (OB).
 
\begin{tkzexample}[latex=6cm,small]
\begin{tikzpicture}[scale=0.75] 
  \tkzDefPoint(0,0){O}
  \tkzDefPoint(2,-1){A}
  \tkzDefPoint(1,1){B} 
  \tkzDrawArc[color=blue,->](O,A)(B)
  \tkzDrawArc[color=gray](O,B)(A)
  \tkzDrawArc(O,B)(A)
  \tkzDrawLines[add = 0 and .5](O,A O,B) 
  \tkzDrawPoints(O,A,B)
  \tkzLabelPoints[below](O,A,B)  
\end{tikzpicture}
\end{tkzexample}

\subsubsection{Option \tkzname{rotate}}
\begin{tkzexample}[latex=6cm,small] 
\begin{tikzpicture}[scale=0.75] 
  \tkzDefPoint(0,0){O}
  \tkzDefPoint(2,-2){A}
  \tkzDefPoint(60:2){B}
  \tkzDrawLines[add = 0 and .5](O,A O,B)
  \tkzDrawArc[rotate,color=red](O,A)(180)
  \tkzDrawPoints(O,A,B)
  \tkzLabelPoints[below](O,A,B) 
\end{tikzpicture}
\end{tkzexample} 

\subsubsection{Option \tkzname{R}} 
\begin{tkzexample}[latex=6cm,small]   
\begin{tikzpicture}[scale=0.75] 
  \tkzDefPoints{0/0/O}
  \tkzSetUpCompass[<->]
  \tkzDrawArc[R,color=teal,double](O,3)(270,360)
  \tkzDrawArc[R,color=orange,double](O,2)(0,270) 
  \tkzDrawPoint(O)
  \tkzLabelPoint[below](O){$O$}  
\end{tikzpicture} 
\end{tkzexample}

\subsubsection{Option \tkzname{R with nodes}} 
\begin{tkzexample}[latex=6cm,small]
\begin{tikzpicture}[scale=0.75] 
  \tkzDefPoint(0,0){O}
  \tkzDefPoint(2,-1){A}
  \tkzDefPoint(1,1){B}
  \tkzCalcLength(B,A)\tkzGetLength{radius}
  \tkzDrawArc[R with nodes](B,\radius)(A,O)
\end{tikzpicture}
\end{tkzexample}

\subsubsection{Option \tkzname{delta}}
This option allows a bit like \tkzcname{tkzCompass} to place an arc and overflow on either side. delta is a measure in degrees.

\begin{tkzexample}[latex=7cm,small] 
\begin{tikzpicture} 
 \tkzDefPoint(0,0){A}
 \tkzDefPoint(3,0){B}
 \tkzDefPointBy[rotation= center A angle 60](B)
 \tkzGetPoint{C} 
 \begin{scope}% style only local
   \tkzDefPointBy[symmetry= center C](A)
   \tkzGetPoint{D} 
   \tkzDrawSegments(A,B A,D)
   \tkzDrawLine(B,D)
   \tkzSetUpCompass[color=orange]
   \tkzDrawArc[orange,delta=10](A,B)(C)
   \tkzDrawArc[orange,delta=10](B,C)(A)
   \tkzDrawArc[orange,delta=10](C,D)(D)
 \end{scope}

 \tkzDrawPoints(A,B,C,D)
 \tkzLabelPoints[below right](A,B,C,D)
 \tkzMarkRightAngle(D,B,A)
\end{tikzpicture}
\end{tkzexample} 

\subsubsection{Option \tkzname{angles}: example 1}

\begin{tkzexample}[latex=6cm,small]
\begin{tikzpicture}[scale=.75]
  \tkzDefPoint(0,0){A}
  \tkzDefPoint(5,0){B}  
  \tkzDefPoint(2.5,0){O} 
  \tkzDefPointBy[rotation=center O angle 60](B)
  \tkzGetPoint{D}
  \tkzDefPointBy[symmetry=center D](O)
  \tkzGetPoint{E}
  \begin{scope}
    \tkzDrawArc[angles](O,B)(0,180)
    \tkzDrawArc[angles,](B,O)(100,180)  
    \tkzCompass[delta=20](D,E) 
    \tkzDrawLines(A,B O,E B,E)
    \tkzDrawPoints(A,B,O,D,E)
  \end{scope}
  \tkzLabelPoints[below right](A,B,O,D,E)
  \tkzMarkRightAngle(O,B,E) 
\end{tikzpicture} 
\end{tkzexample}

\subsubsection{Option \tkzname{angles}: example 2}

\begin{tkzexample}[latex=6cm,small]
  \begin{tikzpicture}
   \tkzDefPoint(0,0){O}
   \tkzDefPoint(5,0){I} 
   \tkzDefPoint(0,5){J}
   \tkzInterCC(O,I)(I,O)\tkzGetPoints{B}{C}  
   \tkzInterCC(O,I)(J,O)\tkzGetPoints{D}{A}
   \tkzInterCC(I,O)(J,O)\tkzGetPoints{L}{K}
   \tkzDrawArc[angles](O,I)(0,90)
   \tkzDrawArc[angles,color=gray,
               style=dashed](I,O)(90,180)
   \tkzDrawArc[angles,color=gray,
               style=dashed](J,O)(-90,0)
   \tkzDrawPoints(A,B,K)
   \foreach \point in {I,A,B,J,K}{%
               \tkzDrawSegment(O,\point)} 
  \end{tikzpicture} 
\end{tkzexample}

\subsubsection{Option \tkzname{reverse}: inversion of the arrow}

\begin{tkzexample}[latex=6cm,small]
  \begin{tikzpicture}
    \tkzDefPoints{0/0/O,3/0/U}
    \tkzDefPoint(10:1){A}
    \tkzDefPoint(90:1){B}
    \tkzLabelPoints(A,B)
    \tkzDrawArc[reverse,tkz arrow={Stealth}](O,A)(B)
    \tkzDrawPoints(A,B,O)
  \end{tikzpicture}
\end{tkzexample}
%<---------------------------------------------------------------------------->
%    SECTOR
%<---------------------------------------------------------------------------->
\section{Drawing a sector or sectors}
\subsection{\tkzcname{tkzDrawSector}} 
\tkzHandBomb\  Attention the arguments vary according to the options.
\begin{NewMacroBox}{tkzDrawSector}{\oarg{local options}\parg{O,\dots}\parg{\dots}}%
\begin{tabular}{SlSlSl}%
options             & default & definition                         \\ 
\midrule
\TOline{towards}{towards}{$O$ is the center and the arc from $A$ to $(OB)$}
\TOline{rotate} {towards}{the arc starts from $A$ and the angle determines its length } 
\TOline{R}{towards}{We give the radius and two angles}
\TOline{R with nodes}{towards}{We give the radius and two points}

\end{tabular} 

\medskip
\emph{You have to add, of course, all the styles of \TIKZ\ for tracings...}

\begin{tabular}{lll}%

options             & arguments & example                         \\ 
\midrule
\TOline{towards}{\parg{pt,pt}\parg{pt}}{\tkzcname{tkzDrawSector(O,A)(B)}}
\TOline{rotate} {\parg{pt,pt}\parg{an}}{\tkzcname{tkzDrawSector[rotate,color=red](O,A)(90)}} 
\TOline{R}{\parg{pt,$r$}\parg{an,an}}{\tkzcname{tkzDrawSector[R,color=teal](O,2)(30,90)}}
\TOline{R with nodes}{\parg{pt,$r$}\parg{pt,pt}}{\tkzcname{tkzDrawSector[R with nodes](O,2)(A,B)}}
\end{tabular}
\end{NewMacroBox}

Here are a few examples: 

\subsubsection{\tkzcname{tkzDrawSector} and \tkzname{towards}} 
There's no need to put \tkzname{towards}. You can use \tkzname{fill} as an option.

\begin{tkzexample}[latex=7cm,small]
\begin{tikzpicture}
  \tkzDefPoint(0,0){O}
  \tkzDefPoint(-30:1){A} 
  \tkzDefPointBy[rotation = center O angle -60](A) 
  \tkzDrawSector[teal](O,A)(tkzPointResult)
 \begin{scope}[shift={(-60:1)}]
  \tkzDefPoint(0,0){O}
  \tkzDefPoint(-30:1){A} 
  \tkzDefPointBy[rotation = center O angle -60](A) 
  \tkzDrawSector[red](O,tkzPointResult)(A)
  \end{scope}
\end{tikzpicture}   
\end{tkzexample}

\subsubsection{\tkzcname{tkzDrawSector} and \tkzname{rotate}}  
\begin{tkzexample}[latex=7cm,small]  
\begin{tikzpicture}[scale=2]
 \tkzDefPoints{0/0/O,2/2/A,2/1/B}
 \tkzDrawSector[rotate,orange](O,A)(20)
 \tkzDrawSector[rotate,teal](O,B)(-20)
\end{tikzpicture} 
\end{tkzexample}  

\subsubsection{\tkzcname{tkzDrawSector} and \tkzname{R}}  
\begin{tkzexample}[latex=7cm,small]
\begin{tikzpicture}[scale=1.25]
 \tkzDefPoint(0,0){O}
 \tkzDefPoint(2,-1){A}
 \tkzDrawSector[R](O,1)(30,90)
 \tkzDrawSector[R](O,1)(90,180)
 \tkzDrawSector[R](O,1)(180,270)
 \tkzDrawSector[R](O,1)(270,360) 
\end{tikzpicture}
\end{tkzexample}

\subsubsection{\tkzcname{tkzDrawSector} and \tkzname{R with nodes}}  
In this example I use the option \tkzname{fill} but \tkzcname{tkzFillSector} is possible.
\begin{tkzexample}[latex=7cm,small]
\begin{tikzpicture}[scale=1.25]
 \tkzDefPoint(0,0){O}
 \tkzDefPoint(4,-2){A}
 \tkzDefPoint(4,1){B}
 \tkzDefPoint(3,3){C}
 \tkzDrawSector[R with nodes,%
                fill=teal!20](O,1)(B,C)
 \tkzDrawSector[R with nodes,%
                fill=orange!20](O,1.25)(A,B)  
\tkzDrawSegments(O,A O,B O,C)
\tkzDrawPoints(O,A,B,C) 
\tkzLabelPoints(A,B,C) 
\tkzLabelPoints[left](O) 
\end{tikzpicture}
\end{tkzexample}

\subsubsection{\tkzcname{tkzDrawSector} and \tkzname{R with nodes}} 
\begin{tkzexample}[latex=6cm,small]
\begin{tikzpicture} [scale=.4]
 \tkzDefPoints{-1/-2/A,1/3/B}
 \tkzDefRegPolygon[side,sides=6](A,B) 
 \tkzGetPoint{O} 
 \tkzDrawPolygon[fill=black!10, draw=blue](P1,P...,P6) 
 \tkzLabelRegPolygon[sep=1.05](O){A,...,F}
 \tkzDrawCircle[dashed](O,A)
 \tkzLabelSegment[above,sloped,
                  midway](A,B){\(A B = 16m\)}
 \foreach \i  [count=\xi from 1]  in {2,...,6,1}
   {%
    \tkzDefMidPoint(P\xi,P\i)
    \path (O) to [pos=1.1] node {\xi} (tkzPointResult) ;
    }
  \tkzDefRandPointOn[segment = P3--P5] 
  \tkzGetPoint{S}
  \tkzDrawSegments[thick,dashed,red](A,S S,B)
  \tkzDrawPoints(P1,P...,P6,S)
  \tkzLabelPoint[left,above](S){$S$}
  \tkzDrawSector[R with nodes,fill=red!20](S,2)(A,B)
  \tkzLabelAngle[pos=1.5](A,S,B){$\alpha$}
\end{tikzpicture}
\end{tkzexample}

\endinput
\subsection{Coloring a disc}
This was possible with the macro \tkzcname{tkzDrawCircle}, but disk tracing was mandatory, this is no longer the case.
  
\begin{NewMacroBox}{tkzFillCircle}{\oarg{local options}\parg{A,B}}%
\begin{tabular}{lll}%
options             & default & definition                         \\ 
\midrule
\TOline{radius}  {radius}{two points define a radius}
\TOline{R} {radius}{a point and the measurement of a radius }
\bottomrule
\end{tabular}

\medskip
You don't need to put \tkzname{radius} because that's the default option. Of course, you have to add all the styles of \TIKZ\ for the plots.
\end{NewMacroBox}  


\subsubsection{Yin and Yang} 
\begin{tkzexample}[latex=8cm,small]
  \begin{tikzpicture}[scale=.75]
    \tkzDefPoint(0,0){O}
    \tkzDefPoint(-4,0){A}
    \tkzDefPoint(4,0){B}
    \tkzDefPoint(-2,0){I}
    \tkzDefPoint(2,0){J}
    \tkzDrawSector[fill=teal](O,A)(B)
    \tkzFillCircle[fill=white](J,B) 
    \tkzFillCircle[fill=teal](I,A) 
    \tkzDrawCircle(O,A) 
  \end{tikzpicture}
\end{tkzexample}


\subsubsection{From a sangaku} 

\begin{tkzexample}[latex=7cm,small]
\begin{tikzpicture}
   \tkzDefPoint(0,0){B}  \tkzDefPoint(6,0){C}%
   \tkzDefSquare(B,C)    \tkzGetPoints{D}{A} 
   \tkzClipPolygon(B,C,D,A) 
   \tkzDefMidPoint(A,D)  \tkzGetPoint{F}
   \tkzDefMidPoint(B,C)  \tkzGetPoint{E}
   \tkzDefMidPoint(B,D)  \tkzGetPoint{Q}           
   \tkzDefLine[tangent from = B](F,A) \tkzGetPoints{H}{G} 
   \tkzInterLL(F,G)(C,D) \tkzGetPoint{J}
   \tkzInterLL(A,J)(F,E) \tkzGetPoint{K}
   \tkzDefPointBy[projection=onto B--A](K)   
   \tkzGetPoint{M}  
   \tkzDrawPolygon(A,B,C,D)
   \tkzFillCircle[red!20](E,B)
   \tkzFillCircle[blue!20](M,A)
   \tkzFillCircle[green!20](K,Q)
  \tkzDrawCircles(B,A M,A E,B K,Q) 
\end{tikzpicture}
\end{tkzexample} 

\subsubsection{Clipping and filling part I} 
\begin{tkzexample}[latex=7cm,small]
\begin{tikzpicture}
\tkzDefPoints{0/0/A,4/0/B,2/2/O,3/4/X,4/1/Y,1/0/Z,
              0/3/W,3/0/R,4/3/S,1/4/T,0/1/U}
\tkzDefSquare(A,B)\tkzGetPoints{C}{D}
\tkzDefPointWith[colinear normed=at X,K=1](O,X)
 \tkzGetPoint{F}
\begin{scope}
  \tkzFillCircle[fill=teal!20](O,F)
  \tkzFillPolygon[white](A,...,D)
  \tkzClipPolygon(A,...,D)
  \foreach \c/\t in {S/C,R/B,U/A,T/D}
  {\tkzFillCircle[teal!20](\c,\t)}
\end{scope}
\foreach \c/\t in {X/C,Y/B,Z/A,W/D}
{\tkzFillCircle[white](\c,\t)}
  \foreach \c/\t in {S/C,R/B,U/A,T/D}
  {\tkzFillCircle[teal!20](\c,\t)}
\end{tikzpicture}
\end{tkzexample}

\subsubsection{Clipping and filling part II} 
\begin{tkzexample}[latex=7cm, small]  
\begin{tikzpicture}[scale=.75]
\tkzDefPoints{0/0/A,8/0/B,8/8/C,0/8/D}
\tkzDefMidPoint(A,B) \tkzGetPoint{F}
\tkzDefMidPoint(B,C) \tkzGetPoint{E}
\tkzDefMidPoint(D,B) \tkzGetPoint{I}
\tkzDefMidPoint(I,B) \tkzGetPoint{a}
\tkzInterLC(B,I)(B,C) \tkzGetSecondPoint{K}
\tkzDefMidPoint(I,K) \tkzGetPoint{b}
\begin{scope}
  \tkzFillSector[fill=blue!10](B,C)(A)
  \tkzDefMidPoint(A,B) \tkzGetPoint{x}
  \tkzDrawSemiCircle[fill=white](x,B)
  \tkzDefMidPoint(B,C) \tkzGetPoint{y}
  \tkzDrawSemiCircle[fill=white](y,C)
  \tkzClipCircle(E,B)
  \tkzClipCircle(F,B)
  \tkzFillCircle[fill=blue!10](B,A)
\end{scope}
\tkzDrawSemiCircle[thick](F,B)
\tkzDrawSemiCircle[thick](E,C)
\tkzDrawArc[thick](B,C)(A)
\tkzDrawSegments[thick](A,B B,C)
\tkzDrawPoints(A,B,C,E,F)
\tkzLabelPoints[centered](a,b)
\tkzLabelPoints(A,B,C,E,F)
\end{tikzpicture}
\end{tkzexample}

\subsubsection{Clipping and filling part III} 

\begin{tkzexample}[latex=7cm, small] 
\begin{tikzpicture}
  \tkzDefPoint(0,0){A} \tkzDefPoint(1,0){B}
  \tkzDefPoint(2,0){C} \tkzDefPoint(-3,0){a} 
  \tkzDefPoint(3,0){b}  \tkzDefPoint(0,3){c} 
  \tkzDefPoint(0,-3){d}
\begin{scope}
 \tkzClipPolygon(a,b,c,d)
 \tkzFillCircle[teal!20](A,C)
\end{scope}
 \tkzFillCircle[white](A,B)
 \tkzDrawCircle[color=red](A,C)
 \tkzDrawCircle[color=red](A,B)
\end{tikzpicture}
\end{tkzexample}

\subsection{Coloring a polygon} 
 \begin{NewMacroBox}{tkzFillPolygon}{\oarg{local options}\parg{points list}}%
You can color by drawing the polygon, but in this case you color the inside of the polygon without drawing it.

\medskip
\begin{tabular}{lll}%
\toprule
arguments                & example & explanation                         \\ 
\midrule
\TAline{\parg{pt1,pt2,\dots}}{\parg{A,B,\dots}}{}
%\bottomrule
 \end{tabular}
\end{NewMacroBox} 

\subsubsection{\tkzcname{tkzFillPolygon}} 
\begin{tkzexample}[latex=7cm, small]  
\begin{tikzpicture}[scale=.5]
   \tkzDefPoint(0,0){C} \tkzDefPoint(4,0){A}
   \tkzDefPoint(0,3){B}
   \tkzDefSquare(B,A)      \tkzGetPoints{E}{F}
   \tkzDefSquare(A,C)      \tkzGetPoints{G}{H}
   \tkzDefSquare(C,B)       \tkzGetPoints{I}{J}
   \tkzFillPolygon[color  =  orange!30   ](A,C,G,H)
   \tkzFillPolygon[color  =  teal!40  ](C,B,I,J)
   \tkzFillPolygon[color  =  purple!20](B,A,E,F)
   \tkzDrawPolygon[line width  =  1pt](A,B,C)
   \tkzDrawPolygon[line width  =  1pt](A,C,G,H)
   \tkzDrawPolygon[line width  =  1pt](C,B,I,J)
   \tkzDrawPolygon[line width  =  1pt](B,A,E,F)
   \tkzLabelSegment[above](C,A){$a$}
   \tkzLabelSegment[right](B,C){$b$}
   \tkzLabelSegment[below left](B,A){$c$}
\end{tikzpicture}
\end{tkzexample}

\subsection{\tkzcname{tkzFillSector}}
\tkzHandBomb\ Attention the arguments vary according to the options. 
\begin{NewMacroBox}{tkzFillSector}{\oarg{local options}\parg{O,\dots}\parg{\dots}}%
\begin{tabular}{lll}%
options          & default & definition      \\ 
\midrule
\TOline{towards}{towards}{$O$ is the center and the arc from $A$ to $(OB)$}
\TOline{rotate} {towards}{the arc starts from A and the angle determines its length } 
\TOline{R}{towards}{We give the radius and two angles}
\TOline{R with nodes}{towards}{We give the radius and two points}
\bottomrule
\end{tabular} 

\medskip
Of course, you have to add all the styles of \TIKZ\ for the tracings...

\medskip
\begin{tabular}{lll}%
\toprule
options             & arguments & example                         \\ 
\midrule
\TOline{towards}{\parg{pt,pt}\parg{pt}}{\tkzcname{tkzFillSector(O,A)(B)}}
\TOline{rotate} {\parg{pt,pt}\parg{an}}{\tkzcname{tkzFillSector[rotate,color=red](O,A)(90)}}
\TOline{R}{\parg{pt,$r$}\parg{an,an}}{\tkzcname{tkzFillSector[R,color=blue](O,2)(30,90)}} 
\TOline{R with nodes}{\parg{pt,$r$}\parg{pt,pt}}{\tkzcname{tkzFillSector[R with nodes](O,2)(A,B)}}
\end{tabular}   
\end{NewMacroBox} 

\subsubsection{\tkzcname{tkzFillSector} and \tkzname{towards}} 
It is useless to put \tkzname{towards} and you will notice that the contours are not drawn, only the surface is colored.
\begin{tkzexample}[latex=5.75cm,small]
  \begin{tikzpicture}[scale=.6] 
  \tkzDefPoint(0,0){O}
  \tkzDefPoint(-30:3){A}
  \tkzDefPointBy[rotation = center O angle -60](A)   
  \tkzFillSector[fill=purple!20](O,A)(tkzPointResult) 
    \begin{scope}[shift={(-60:1)}]
    \tkzDefPoint(0,0){O}
    \tkzDefPoint(-30:3){A}
    \tkzDefPointBy[rotation = center O angle -60](A)
    \tkzGetPoint{A'}
    \tkzFillSector[color=teal!40](O,A')(A)
      \end{scope}
  \end{tikzpicture}
\end{tkzexample}


\subsubsection{\tkzcname{tkzFillSector} and \tkzname{rotate}} 
\begin{tkzexample}[latex=5.75cm,small]
\begin{tikzpicture}[scale=1.5]
 \tkzDefPoint(0,0){O} \tkzDefPoint(2,2){A}
 \tkzFillSector[rotate,color=purple!20](O,A)(30)
 \tkzFillSector[rotate,color=teal!40](O,A)(-30)
\end{tikzpicture}    
\end{tkzexample} 

\subsection{Colour an angle: \tkzcname{tkzFillAngle}}

The simplest operation
\begin{NewMacroBox}{tkzFillAngle}{\oarg{local options}\parg{A,O,B}}%
$O$ is the vertex of the angle. $OA$ and $OB$ are the sides. Attention the angle is determined by the order of the points.

\medskip

\begin{tabular}{lll}%
\toprule
options             & default & definition                        \\ 
\midrule
\TOline{size}{1}{this option determines the radius of the coloured angular sector.}

\bottomrule
\end{tabular} 

\medskip
Of course, you have to add all the styles of \TIKZ, like the use of fill and shade... 
\end{NewMacroBox}  

\subsubsection{Example with \tkzname{size}}  
\begin{tkzexample}[latex=7cm,small]
\begin{tikzpicture} 
   \tkzInit 
   \tkzDefPoints{0/0/O,2.5/0/A,1.5/2/B}
   \tkzFillAngle[size=2, fill=gray!10](A,O,B)
   \tkzDrawLines(O,A O,B)
   \tkzDrawPoints(O,A,B)
\end{tikzpicture}
\end{tkzexample}


\subsubsection{Changing the order of items} 
\begin{tkzexample}[latex=7cm,small]
\begin{tikzpicture} 
   \tkzInit 
   \tkzDefPoints{0/0/O,2.5/0/A,1.5/2/B}
   \tkzFillAngle[size=2,fill=gray!10](B,O,A)
   \tkzDrawLines(O,A O,B)
   \tkzDrawPoints(O,A,B)
\end{tikzpicture}
\end{tkzexample}

\begin{tkzexample}[latex=7cm,small]
\begin{tikzpicture} 
   \tkzInit 
   \tkzDefPoints{0/0/O,5/0/A,3/4/B}
   % Don't forget {} to get, () to use
   \tkzFillAngle[size=4,left color=white, 
                 right color=red!50](A,O,B)
   \tkzDrawLines(O,A O,B)
   \tkzDrawPoints(O,A,B)
\end{tikzpicture}
\end{tkzexample}

\begin{NewMacroBox}{tkzFillAngles}{\oarg{local options}\parg{A,O,B}\parg{A',O',B'}etc.}%
With common options, there is a macro for multiple angles.
  \end{NewMacroBox}  
  
\subsubsection{Multiples angles}  
\begin{tkzexample}[latex=5cm,small]
\begin{tikzpicture}[scale=0.5]
  \tkzDefPoints{0/0/B,8/0/C,0/8/A,8/8/D}
  \tkzDrawPolygon(B,C,D,A)
  \tkzDefTriangle[equilateral](B,C) \tkzGetPoint{M}
  \tkzInterLL(D,M)(A,B) \tkzGetPoint{N}
  \tkzDefPointBy[rotation=center N angle -60](D) 
  \tkzGetPoint{L}
  \tkzInterLL(N,L)(M,B)     \tkzGetPoint{P}
  \tkzInterLL(M,C)(D,L)     \tkzGetPoint{Q}
  \tkzDrawSegments(D,N N,L L,D B,M M,C)
  \tkzDrawPoints(L,N,P,Q,M,A,D) 
  \tkzLabelPoints[left](N,P,Q)
  \tkzLabelPoints[above](M,A,D)
  \tkzLabelPoints(L,B,C)
  \tkzMarkAngles(C,B,M B,M,C M,C,B D,L,N L,N,D N,D,L)
  \tkzFillAngles[fill=red!20,opacity=.2](C,B,M%
      B,M,C M,C,B D,L,N L,N,D N,D,L)
\end{tikzpicture}
\end{tkzexample} 
\endinput

\section{Controlling Bounding Box}
From the \tkzimp{PgfManual} :"When you add the clip option, the current path is used for clipping subsequent drawings. Clipping never enlarges the clipping area. Thus, when you clip against a certain path and then clip again against another path, you clip against the intersection of both.
The only way to enlarge the clipping path is to end the {pgfscope} in which the clipping was done. At the end of a {pgfscope} the clipping path that was in force at the beginning of the scope is reinstalled."


First of all, you don't have to deal with \TIKZ\ the size of the bounding box. Early versions of \tkzNamePack{tkz-euclide} did not control the size of the bounding box, now with \tkzNamePack{\tkznameofpack} 4 the size of the bounding box is limited.

The initial bounding box after using the macro \tkzcname{tkzInit} is defined by the rectangle based on the points $(0,0)$ and $(10,10)$. The \tkzcname{tkzInit} macro allows this initial bounding box to be modified using the arguments (\tkzname{xmin}, \tkzname{xmax}, \tkzname{ymin}, and \tkzname{ymax}). Of course any external trace modifies the bounding box. \TIKZ\ maintains that bounding box. It is possible to influence this behavior either directly with commands or options in \TIKZ\ such as a command like \tkzcname{useasboundingbox} or the option \tkzname{use as bounding box}. A possible consequence is to reserve a box for a figure but the figure may overflow the box and spread over the main text.
The following command \tkzcname{pgfresetboundingbox} clears a bounding box and establishes a new one.

\subsection{Utility of \tkzcname{tkzInit}} 
 However, it is sometimes necessary to control the size of what will be displayed.
 To do this, you need to have prepared the bounding box you are going to work in, this is the role of the   macro \tkzNameMacro{tkzInit}.  For some drawings, it is interesting to fix the extreme values (xmin,xmax,ymin and ymax) and to "clip" the definition rectangle in order to control the size of the figure as well as possible.

The two macros that are useful for controlling the bounding box:
\begin{itemize}
   \item \tkzcname{tkzInit}
   \item \tkzcname{tkzClip}
\end{itemize}
\vspace{20pt}

To this, I added macros directly linked to the bounding box. You can now view it, backup it, restore it (see the  section Bounding Box).

\subsection{\tkzcname{tkzInit}}

\begin{NewMacroBox}{tkzInit}{\oarg{local options}}\hypertarget{init}{}%
\begin{tabular}{lll}%    
options  & default & definition             \\
\midrule    
\TOline{xmin} {0} {minimum value of the abscissae in cm}
\TOline{xmax} {10} {maximum value of the abscissae in cm}
\TOline{xstep}{1} {difference between two graduations in $x$}
\TOline{ymin} {0} {minimum y-axis value in cm }
\TOline{ymax} {10} {maximum y-axis value in cm}
\TOline{ystep}{1} {difference between two graduations in $y$}  
\bottomrule    
\end{tabular}

\medskip 

The role of \tkzcname{tkzInit} is to define a \textcolor{red}{orthogonal} coordinates system and a rectangular part of the plane in which you will place your drawings using Cartesian coordinates. 
This macro allows you to define your working environment as with a calculator. With \tkzname{\tkznameofpack} 4 \tkzcname{xstep}  and \tkzcname{ystep} are always 1. Logically it is no longer useful to use \tkzcname{tkzInit}, except for an action like "Clipping Out".
\end{NewMacroBox}


\subsection{\tkzcname{tkzClip}}

\begin{NewMacroBox}{tkzClip}{\oarg{local options}}
The role of this macro is to make invisible what is outside the rectangle defined by (xmin~;~ymin) and (xmax~;~ymax).

\medskip
\begin{tabular}{lll}
\hline
options  & default & definition             \\
\midrule
\TOline{space} {1} {added value on the right, left, bottom and top of the background}
\bottomrule
\end{tabular}

\medskip

The role of the \tkzname{space} option is to enlarge the visible part of the drawing. This part becomes the rectangle defined by (xmin-space~;~ymin-space) and (xmax+space~;~ymax+space).  \tkzname{space} can be negative!  The unit is cm and should not be specified.
\end{NewMacroBox}



The role of this macro is to "clip" the initial rectangle so that only the paths contained in this rectangle are drawn.

\begin{tkzexample}[latex=8cm,small]
\begin{tikzpicture}
 \tkzInit[xmax=4, ymax=3]
 \tkzDefPoints{-1/-1/A,5/2/B}
 \tkzDrawX \tkzDrawY 
 \tkzGrid
 \tkzClip
 \tkzDrawSegment(A,B)
\end{tikzpicture}
\end{tkzexample} 

It is possible to add a bit of space
\begin{tkzltxexample}[]
  \tkzClip[space=1]
\end{tkzltxexample} 

\subsection{\tkzcname{tkzClip} and the option \tkzname{space}} 
This option allows you to add some space around the "clipped" rectangle.
\begin{tkzexample}[latex=8cm,small]
\begin{tikzpicture}
 \tkzInit[xmax=4, ymax=3]
 \tkzDefPoints{-1/-1/A,5/2/B}
 \tkzDrawX \tkzDrawY 
 \tkzGrid
 \tkzClip[space=1]
 \tkzDrawSegment(A,B)
\end{tikzpicture}
\end{tkzexample}   
The dimensions of the "clipped" rectangle are \tkzname{xmin-1}, \tkzname{ymin-1}, \tkzname{xmax+1} and \tkzname{ymax+1}. 

%<--------------------------------------------------------------------------->
%              tkzShowBB
%<--------------------------------------------------------------------------->
\subsection{tkzShowBB}
The simplest macro. 
\begin{NewMacroBox}{tkzShowBB}{\oarg{local options}}% 
This macro displays the bounding box. A rectangular frame surrounds the bounding box. This macro accepts \TIKZ\ options.
\end{NewMacroBox} 


\subsubsection{Example with \tkzcname{tkzShowBB}}
\begin{tkzexample}[latex=8cm,small]
\begin{tikzpicture}[scale=.5]
  \tkzInit[ymax=5,xmax=8]
  \tkzGrid  
  \tkzDefPoint(3,0){A}
   \begin{scope}
    \tkzClipBB
    \tkzDefCircle[R](A,5) \tkzGetPoint{a}
    \tkzDrawCircle(A,a)
     \tkzShowBB[line width = 4pt,fill=teal!10,opacity=.4]
   \end{scope}
\tkzDefCircle[R](A,4) \tkzGetPoint{b}
\tkzDrawCircle[red](A,b)
\end{tikzpicture}
\end{tkzexample}
%<--------------------------------------------------------------------------->
%         tkzClipBB
%<--------------------------------------------------------------------------->
\subsection{tkzClipBB}
\begin{NewMacroBox}{tkzClipBB}{}%
The idea is to limit future constructions to the current bounding box.
\end{NewMacroBox}

\subsubsection{Example with \tkzcname{tkzClipBB} and the bisectors}

\begin{tkzexample}[latex=6cm,small]
  \begin{tikzpicture}
  \tkzInit[xmin=-3,xmax=6, ymin=-1,ymax=6]
  \tkzDefPoint(0,0){O}\tkzDefPoint(3,1){I}
  \tkzDefPoint(1,4){J}
  \tkzDefLine[bisector](I,O,J) \tkzGetPoint{i}
  \tkzDefLine[bisector out](I,O,J) \tkzGetPoint{j}
  \tkzDrawPoints(O,I,J,i,j)
  \tkzClipBB
  \tkzDrawLines[add = 1 and 2,color=orange](O,I O,J)
  \tkzDrawLines[add = 1 and 2](O,i O,j)
  \tkzShowBB
  \end{tikzpicture}
\end{tkzexample}


\newpage

\section{Clipping different objects}

\subsection{Clipping a polygon} 
 \begin{NewMacroBox}{tkzClipPolygon}{\oarg{local options}\parg{points list}}%
This macro makes it possible to contain the different plots in the designated polygon.

\medskip
\begin{tabular}{lll}%
\toprule
arguments       & example & explanation     \\ 
\midrule
\TAline{\parg{pt1,pt2,pt3,\dots}}{\parg{A,B,C}}{}
\midrule
options  & default & definition             \\
\midrule    
\TOline{out} {} {allows to clip the outside of the object}
 \end{tabular}
\end{NewMacroBox}

\subsubsection{\tkzcname{tkzClipPolygon}}

\begin{tkzexample}[latex=7cm,small]
  \begin{tikzpicture}[scale=1.25] 
  \tkzDefPoint(0,0){A} 
  \tkzDefPoint(4,0){B} 
  \tkzDefPoint(1,3){C} 
  \tkzDrawPolygon(A,B,C) 
  \tkzDefPoint(0,2){D} 
  \tkzDefPoint(2,0){E} 
  \tkzDrawPoints(D,E) 
  \tkzLabelPoints(D,E) 
  \tkzClipPolygon(A,B,C) 
  \tkzDrawLine[new](D,E)
\end{tikzpicture}
\end{tkzexample}

\subsubsection{\tkzcname{tkzClipPolygon[out]}}

\begin{tkzexample}[latex=7cm,small]
  \begin{tikzpicture}[scale=1]
  \tkzDefPoint(0,0){P1}
  \tkzDefPoint(4,0){P2}
  \tkzDefPoint(4,4){P3}
  \tkzDefPoint(0,4){P4}
  \tkzDefPoint(1,1){Q1}
  \tkzDefPoint(3,1){Q2}
  \tkzDefPoint(3,3){Q3}
  \tkzDefPoint(1,3){Q4}
  \tkzDrawPolygon(P1,P2,P3,P4)
  \begin{scope}
     \tkzClipPolygon[out](Q1,Q2,Q3,Q4)
    \tkzFillPolygon[teal!20](P1,P2,P3,P4)
  \end{scope}
  \tkzDrawPolygon(Q1,Q2,Q3,Q4)
  \end{tikzpicture}
\end{tkzexample}

\subsubsection{Example: use of "Clip" for Sangaku in a square} 
\begin{tkzexample}[latex=7cm, small]  
\begin{tikzpicture}[scale=.75]
 \tkzDefPoint(0,0){A} \tkzDefPoint(8,0){B}
 \tkzDefSquare(A,B)   \tkzGetPoints{C}{D}
 \tkzDefPoint(4,8){F}
 \tkzDefTriangle[equilateral](C,D) 
 \tkzGetPoint{I}
 \tkzDefPointBy[projection=onto B--C](I) 
 \tkzGetPoint{J}
 \tkzInterLL(D,B)(I,J)  \tkzGetPoint{K}
 \tkzDefPointBy[symmetry=center K](B)  
 \tkzGetPoint{M}
 \tkzClipPolygon(B,C,D,A)
 \tkzFillPolygon[color = orange](A,B,C,D)
 \tkzFillCircle[color = yellow](M,I)
 \tkzFillCircle[color = blue!50!black](F,D)
\end{tikzpicture}
\end{tkzexample}

\subsection{Clipping a disc}

\begin{NewMacroBox}{tkzClipCircle}{\oarg{local options}\parg{A,B}}%
\begin{tabular}{lll}%
\toprule
arguments           & example & explanation                         \\
\midrule
\TAline{\parg{A,B}}{\parg{A,B}} {AB radius}
\bottomrule
\end{tabular}  
 
\medskip
\begin{tabular}{lll}%
options             & default & definition                         \\ 
\midrule
\TOline{out} {} {allows to clip the outside of the object}
 \bottomrule
\end{tabular}

\medskip
It is not necessary to put \tkzname{radius} because that is the default option.
\end{NewMacroBox}

 \subsubsection{Simple clip} 
\begin{tkzexample}[latex=6cm,small] 
\begin{tikzpicture}[scale=.5]
  \tkzDefPoint(0,0){A} \tkzDefPoint(2,2){O}
  \tkzDefPoint(4,4){B} \tkzDefPoint(5,5){C}
  \tkzDrawPoints(O,A,B,C) 
  \tkzLabelPoints(O,A,B,C)
  \tkzDrawCircle(O,A) 
  \tkzClipCircle(O,A)
  \tkzDrawLine(A,C)
  \tkzDrawCircle[fill=teal!10,opacity=.5](C,O)
\end{tikzpicture} 
\end{tkzexample}

\subsection{Clip out}

\begin{tkzexample}[latex=6cm,small]
\begin{tikzpicture}
  \tkzInit[xmin=-3,ymin=-2,xmax=4,ymax=3]
   \tkzDefPoint(0,0){O}
   \tkzDefPoint(-4,-2){A}
   \tkzDefPoint(3,1){B}
   \tkzDefCircle[R](O,2) \tkzGetPoint{o}
   \tkzDrawPoints(A,B) % to have a good bounding box
   \begin{scope}
    \tkzClipCircle[out](O,o)
    \tkzDrawLines(A,B)
   \end{scope}
\end{tikzpicture}  
\end{tkzexample} 

\subsection{Intersection of disks}

\begin{tkzexample}[latex=6cm,small]
\begin{tikzpicture}
\tkzDefPoints{0/0/O,4/0/A,0/4/B}
\tkzDrawPolygon[fill=teal](O,A,B)
\tkzClipPolygon(O,A,B)
\tkzClipCircle(A,O)
\tkzClipCircle(B,O)
\tkzFillPolygon[white](O,A,B)
\end{tikzpicture}
\end{tkzexample} 

see a more complex example about clipping here : \ref{About clipping circles}

\subsection{Clipping a sector}
\tkzHandBomb\  Attention the arguments vary according to the options. 
\begin{NewMacroBox}{tkzClipSector}{\oarg{local options}\parg{O,\dots}\parg{\dots}}%
\begin{tabular}{lll}%
options             & default & definition                         \\ 
\midrule
\TOline{towards}{towards}{$O$ is the center and the sector starts from $A$ to $(OB)$}
\TOline{rotate} {towards}{The sector starts from $A$ and the angle determines its amplitude. } 
\TOline{R}{towards}{We give the radius and two angles} 
\bottomrule
\end{tabular} 

\medskip
You have to add, of course, all the styles of \TIKZ\ for tracings...

\medskip   
\begin{tabular}{lll}%
\toprule
options             & arguments & example                         \\ 
\midrule
\TOline{towards}{\parg{pt,pt}\parg{pt}}{\tkzcname{tkzClipSector(O,A)(B)}}
\TOline{rotate} {\parg{pt,pt}\parg{angle}}{\tkzcname{tkzClipSector[rotate](O,A)(90)}} 
\TOline{R}{\parg{pt,$r$}\parg{angle 1,angle 2}}{\tkzcname{tkzClipSector[R](O,2)(30,90)}}
\end{tabular}
\end{NewMacroBox}

\subsubsection{Example 1} 
\begin{tkzexample}[latex=7cm,small] 
\begin{tikzpicture}[scale=0.5]
\tkzDefPoint(0,0){a}
\tkzDefPoint(12,0){b}
\tkzDefPoint(4,10){c}
\tkzInterCC[R](a,6)(b,8) 
\tkzGetFirstPoint{AB1} \tkzGetSecondPoint{AB2}
\tkzInterCC[R](a,6)(c,6) 
\tkzGetFirstPoint{AC1} \tkzGetSecondPoint{AC2}
\tkzInterCC[R](b,8)(c,6) 
\tkzGetFirstPoint{BC1} \tkzGetSecondPoint{BC2}
\tkzDrawArc(a,AB2)(AB1)
\tkzDrawArc(b,AB1)(AB2)
\tkzDrawArc(a,AC2)(AC1)
\tkzDrawArc(c,AC1)(AC2)
\tkzDrawArc(b,BC2)(BC1)
\tkzDrawArc(c,BC1)(BC2)
\begin{scope}
\tkzClipSector(b,BC2)(BC1)
\tkzFillSector[teal!40!white](c,BC1)(BC2)
\end{scope}
\begin{scope}
\tkzClipSector(a,AB2)(AB1)
\tkzFillSector[teal!40!white](b,AB1)(AB2)
\end{scope}
\begin{scope}
\tkzClipSector(a,AC2)(AC1)
\tkzFillSector[teal!40!white](c,AC1)(AC2)
\end{scope}
\end{tikzpicture}
\end{tkzexample}

\subsubsection{Example 2} 
\begin{tkzexample}[latex=7cm,small] 
\begin{tikzpicture}[scale=1.5] 
  \tkzDefPoint(0,0){O}
  \tkzDefPoint(2,-1){A}
  \tkzDefPoint(1,1){B} 
  \tkzDrawSector[new,dashed](O,A)(B)
  \tkzDrawSector[new](O,B)(A)
\begin{scope}
\tkzClipSector(O,B)(A)
\tkzDefSquare(O,B) \tkzGetPoints{B'}{O'}
\tkzDrawPolygon[color=teal,fill=teal!20](O,B,B',O')
\end{scope}
\tkzDrawPoints(A,B,O) 
\end{tikzpicture} 
\end{tkzexample}


\subsection{Options from \TIKZ: trim left or right}
See the \tkzimp{pgfmanual}

\subsection{\TIKZ\ Controls \tkzcname{pgfinterruptboundingbox} and \tkzcname{endpgfinterruptboundingbox}}
This command temporarily interrupts the calculation of the box and configures a new box.
See the \tkzimp{pgfmanual}

\subsubsection{Example about contolling the bouding box} 
\begin{tkzexample}[latex=7cm,small] 
\begin{tikzpicture}
\tkzDefPoint(0,5){A}\tkzDefPoint(5,4){B}
\tkzDefPoint(0,0){C}\tkzDefPoint(5,1){D}
\tkzDrawSegments(A,B C,D A,C)
\pgfinterruptboundingbox
   \tkzInterLL(A,B)(C,D)\tkzGetPoint{I}
\endpgfinterruptboundingbox
\tkzClipBB
\tkzDrawCircle(I,B)
\end{tikzpicture}
\end{tkzexample}

\subsection{Reverse clip: tkzreverseclip}

In order to use this option, a bounding box must be defined. 

\begin{tkzltxexample}[]
  \tikzset{tkzreverseclip/.style={insert path={
     (current bounding box.south west) --(current bounding box.north west)
   --(current bounding box.north east) --  (current bounding box.south east)
   -- cycle} }}
\end{tkzltxexample}

\subsubsection{Example with \tkzcname{tkzClipPolygon[out]}}
\tkzcname{tkzClipPolygon[out]}, \tkzcname{tkzClipCircle[out]} use this option.
\begin{tkzexample}[vbox,small]
\begin{tikzpicture}[scale=1]
\tkzInit[xmin=-5,xmax=5,ymin=-4,ymax=6]
\tkzClip
\tkzDefPoints{-.5/0/P1,.5/0/P2}
\foreach \i [count=\j from 3] in {2,...,7}{%
    \tkzDefShiftPoint[P\i]({45*(\i-1)}:1){P\j}}  
\tkzClipPolygon[out](P1,P...,P8)
\tkzCalcLength(P1,P5)\tkzGetLength{r}
\begin{scope}[blend group=screen]
  \foreach \i in {1,...,8}{%
   \tkzDefCircle[R](P\i,\r) \tkzGetPoint{x}
   \tkzFillCircle[color=teal](P\i,x)}
  \end{scope}
\end{tikzpicture}
\end{tkzexample}

\endinput

\part{Marking}
\subsection{Mark a segment \tkzcname{tkzMarkSegment}}
\hypertarget{tms}{}  
  
 \begin{NewMacroBox}{tkzMarkSegment}{\oarg{local options}\parg{pt1,pt2}}% 
The macro allows you to place a mark on a segment.

\medskip
\begin{tabular}{lll}%
\toprule
options             & default & definition   \\
\midrule
\TOline{pos}{.5}{position of the mark} 
\TOline{color}{black}{color of the mark} 
\TOline{mark}{none}{choice of the mark} 
\TOline{size}{4pt}{size of the mark}
\bottomrule
\end{tabular}

Possible marks are those provided by \TIKZ, but other marks have been created based on an idea by Yves Combe.
\end{NewMacroBox} 

\subsubsection{Several marks }
\begin{tkzexample}[latex=5cm,small] 
\begin{tikzpicture}
  \tkzDefPoint(2,1){A}
  \tkzDefPoint(6,4){B}
  \tkzDrawSegment(A,B)
  \tkzMarkSegment[color=brown,size=2pt,pos=0.4, mark=z](A,B) 
  \tkzMarkSegment[color=blue,pos=0.2, mark=oo](A,B)
  \tkzMarkSegment[pos=0.8,mark=s,color=red](A,B) 
\end{tikzpicture}
\end{tkzexample}

\subsubsection{Use of \tkzname{mark}}      
\begin{tkzexample}[latex=5cm,small] 
\begin{tikzpicture}
  \tkzDefPoint(2,1){A} 
  \tkzDefPoint(6,4){B}
  \tkzDrawSegment(A,B)
  \tkzMarkSegment[color=gray,pos=0.2,mark=s|](A,B)
  \tkzMarkSegment[color=gray,pos=0.4,mark=s||](A,B)
  \tkzMarkSegment[color=brown,pos=0.6,mark=||](A,B)
  \tkzMarkSegment[color=red,pos=0.8,mark=|||](A,B)
\end{tikzpicture}
\end{tkzexample}


\subsection{Marking segments \tkzcname{tkzMarkSegments}}
\hypertarget{tmss}{} 
 
\begin{NewMacroBox}{tkzMarkSegments}{\oarg{local options}\parg{pt1,pt2 pt3,pt4 ...}}%
Arguments are a list of pairs of points separated by spaces. The styles of \TIKZ\ are available for plots.
\end{NewMacroBox} 

\subsubsection{Marks for an isosceles triangle}      
\begin{tkzexample}[latex=6cm,small]
\begin{tikzpicture}[scale=1]
 \tkzDefPoints{0/0/O,2/2/A,4/0/B,6/2/C}
 \tkzDrawSegments(O,A A,B)
 \tkzDrawPoints(O,A,B)
 \tkzDrawLine(O,B)   
 \tkzMarkSegments[mark=||,size=6pt](O,A A,B)
\end{tikzpicture}
\end{tkzexample} 

\subsection{Another marking}   
\begin{tkzexample}[latex=5cm,small] 
 \begin{tikzpicture}[scale=1]
  \tkzDefPoint(0,0){A}\tkzDefPoint(3,2){B} 
  \tkzDefPoint(4,0){C}\tkzDefPoint(2.5,1){P}
  \tkzDrawPolygon(A,B,C)
  \tkzDefEquilateral(A,P) \tkzGetPoint{P'}
  \tkzDefPointsBy[rotation=center A angle 60](P,B){P',C'}
  \tkzDrawPolygon(A,P,P')
  \tkzDrawPolySeg(P',C',A,P,B)
  \tkzDrawSegment(C,P)
  \tkzDrawPoints(A,B,C,C',P,P')
  \tkzMarkSegments[mark=s|,size=6pt,
  color=blue](A,P P,P' P',A) 
  \tkzMarkSegments[mark=||,color=orange](B,P P',C')
  \tkzLabelPoints(A,C) \tkzLabelPoints[below](P) 
  \tkzLabelPoints[above right](P',C',B) 
\end{tikzpicture} 
\end{tkzexample}  

\subsection{Mark an arc \tkzcname{tkzMarkArc}}
\hypertarget{tms}{}  
  
 \begin{NewMacroBox}{tkzMarkArc}{\oarg{local options}\parg{pt1,pt2,pt3}}% 
The macro allows you to place a mark on an arc. pt1 is the center, pt2 and pt3 are the endpoints of the arc.

\medskip
\begin{tabular}{lll}%
\toprule
options             & default & definition   \\
\midrule
\TOline{pos}{.5}{position of the mark} 
\TOline{color}{black}{color of the mark} 
\TOline{mark}{none}{choice of the mark} 
\TOline{size}{4pt}{size of the mark}
\bottomrule
\end{tabular}

Possible marks are those provided by \TIKZ, but other marks have been created based on an idea by Yves Combe.
\begin{tkzltxexample}[]
|, ||,|||, z, s, x, o, oo 
\end{tkzltxexample}
\end{NewMacroBox} 

\subsubsection{Several marks }
\begin{tkzexample}[latex=7cm,small] 
\begin{tikzpicture}
\tkzDefPoint(0,0){O}
\pgfmathsetmacro\r{2}
\tkzDefPoint(30:\r){A}
\tkzDefPoint(85:\r){B}
\tkzDrawCircle(O,A)
\tkzMarkArc[color=red,mark=||](O,A,B)
\tkzDrawPoints(B,A,O)
\end{tikzpicture}
\end{tkzexample}

 
\subsection{Mark an angle mark : {\tkzcname{tkzMarkAngle}}}
More delicate operation because there are many options. The symbols used for marking in addition to those of \TIKZ\ are defined in the file |tkz-lib-marks.tex| and designated by the following characters:\begin{tkzltxexample}[]
|, ||,|||, z, s, x, o, oo 
\end{tkzltxexample}

% Their definitions are as follows
%
% \begin{tkzltxexample}[]
% \pgfdeclareplotmark{||}
%   %double bar
% {%
%   \pgfpathmoveto{\pgfqpoint{2\pgflinewidth}{\pgfplotmarksize}}
%   \pgfpathlineto{\pgfqpoint{2\pgflinewidth}{-\pgfplotmarksize}}
%   \pgfpathmoveto{\pgfqpoint{-2\pgflinewidth}{\pgfplotmarksize}}
%   \pgfpathlineto{\pgfqpoint{-2\pgflinewidth}{-\pgfplotmarksize}}
%   \pgfusepathqstroke
% }
% \end{tkzltxexample}
%
% \begin{tkzltxexample}[]
%   %triple bar
%   \pgfdeclareplotmark{|||}
%   {%
%     \pgfpathmoveto{\pgfqpoint{0 pt}{\pgfplotmarksize}}
%     \pgfpathlineto{\pgfqpoint{0 pt}{-\pgfplotmarksize}}
%     \pgfpathmoveto{\pgfqpoint{-3\pgflinewidth}{\pgfplotmarksize}}
%     \pgfpathlineto{\pgfqpoint{-3\pgflinewidth}{-\pgfplotmarksize}}
%     \pgfpathmoveto{\pgfqpoint{3\pgflinewidth}{\pgfplotmarksize}}
%     \pgfpathlineto{\pgfqpoint{3\pgflinewidth}{-\pgfplotmarksize}}
%     \pgfusepathqstroke
%   }
% \end{tkzltxexample}
%
% \begin{tkzltxexample}[]
%   % An bar slant
%   \pgfdeclareplotmark{s|}
%   {%
%     \pgfpathmoveto{\pgfqpoint{-.70710678\pgfplotmarksize}%
%                              {-.70710678\pgfplotmarksize}}
%     \pgfpathlineto{\pgfqpoint{.70710678\pgfplotmarksize}%
%                              {.70710678\pgfplotmarksize}}
%     \pgfusepathqstroke
%   }
% \end{tkzltxexample}
%
%
% \begin{tkzltxexample}[]
%   % An double bar slant
%   \pgfdeclareplotmark{s||}
%   {%
%    \pgfpathmoveto{\pgfqpoint{-0.75\pgfplotmarksize}{-\pgfplotmarksize}}
%    \pgfpathlineto{\pgfqpoint{0.25\pgfplotmarksize}{\pgfplotmarksize}}
%    \pgfpathmoveto{\pgfqpoint{0\pgfplotmarksize}{-\pgfplotmarksize}}
%    \pgfpathlineto{\pgfqpoint{1\pgfplotmarksize}{\pgfplotmarksize}}
%    \pgfusepathqstroke
%   }
% \end{tkzltxexample}
%
%
% \begin{tkzltxexample}[]
%   % z
%   \pgfdeclareplotmark{z}
%   {%
%     \pgfpathmoveto{\pgfqpoint{0.75\pgfplotmarksize}{-\pgfplotmarksize}}
%     \pgfpathlineto{\pgfqpoint{-0.75\pgfplotmarksize}{-\pgfplotmarksize}}
%     \pgfpathlineto{\pgfqpoint{0.75\pgfplotmarksize}{\pgfplotmarksize}}
%     \pgfpathlineto{\pgfqpoint{-0.75\pgfplotmarksize}{\pgfplotmarksize}}
%     \pgfusepathqstroke
%   }
% \end{tkzltxexample}
%
% \begin{tkzltxexample}[]
%   % s
%   \pgfdeclareplotmark{s}
%   {%
%      \pgfpathmoveto{\pgfqpoint{0pt}{0pt}}
%      \pgfpathcurveto
%          {\pgfpoint{0pt}{0pt}}
%          {\pgfpoint{-\pgfplotmarksize}{\pgfplotmarksize}}
%          {\pgfpoint{\pgfplotmarksize}{\pgfplotmarksize}}
%      \pgfpathmoveto{\pgfqpoint{0pt}{0pt}}
%       \pgfpathcurveto
%          {\pgfpoint{0pt}{0pt}}
%          {\pgfpoint{\pgfplotmarksize}{-\pgfplotmarksize}}
%          {\pgfpoint{-\pgfplotmarksize}{-\pgfplotmarksize}}
%       \pgfusepathqstroke
%   }
% \end{tkzltxexample}
%
% \begin{tkzltxexample}[]
%   % infinity
%   \pgfdeclareplotmark{oo}
%   {%
%      \pgfpathmoveto{\pgfqpoint{0pt}{0pt}}
%      \pgfpathcurveto
%          {\pgfpoint{0pt}{0pt}}
%          {\pgfpoint{.5\pgfplotmarksize}{1\pgfplotmarksize}}
%          {\pgfpoint{\pgfplotmarksize}{0pt}}
%      \pgfpathmoveto{\pgfqpoint{0pt}{0pt}}
%       \pgfpathcurveto
%          {\pgfpoint{0pt}{0pt}}
%          {\pgfpoint{-.5\pgfplotmarksize}{1\pgfplotmarksize}}
%          {\pgfpoint{-\pgfplotmarksize}{0pt}}
%      \pgfpathmoveto{\pgfqpoint{0pt}{0pt}}
%         \pgfpathcurveto
%          {\pgfpoint{0pt}{0pt}}
%          {\pgfpoint{.5\pgfplotmarksize}{-1\pgfplotmarksize}}
%          {\pgfpoint{\pgfplotmarksize}{0pt}}
%      \pgfpathmoveto{\pgfqpoint{0pt}{0pt}}
%       \pgfpathcurveto
%          {\pgfpoint{0pt}{0pt}}
%          {\pgfpoint{-.5\pgfplotmarksize}{-1\pgfplotmarksize}}
%          {\pgfpoint{-\pgfplotmarksize}{0pt}}
%       \pgfusepathqstroke
%   }
% \end{tkzltxexample}
%


%                \tkzMarkAngle(B, A, C)
%
% Angle mark
% arc de cercle (simple/double/triple) et mark of equality.
%
% By default: 
%                 arc       = simple
%                 mksize  = 1 (radius of the arc)
%                 style traits pleins
%                 mkpos ?  position: 0.5 (mark position)
%                 mark   none
%
% Parameters (optional)
%             arc     : l, ll, lll
%             mksize  : 1
%             gap     : 3pt
%             dist    : 1?
%             style   : type of lines
%             mkpos   : 0.5
%             mark    : none  , |, ||,|||, z, s, x, o, oo mais tous les 
%  % tikz symbols are allowed

\begin{NewMacroBox}{tkzMarkAngle}{\oarg{local options}\parg{A,O,B}}%
$O$ is the vertex. Attention the arguments vary according to the options. Several markings are possible. You can simply draw an arc or  add a mark on this arc. The style of the arc is chosen with the option \tkzname{arc}, the radius of the arc is given by \tkzname{mksize}, the arc can, of course, be colored.

\medskip

\begin{tabular}{lll}%
\toprule
options             & default & definition                        \\ 
\midrule
\TOline{arc}{l}{choice of l, ll and lll (single, double or triple).}
\TOline{size}{1 (cm)}{arc radius.}
\TOline{mark}{none}{choice of mark.}
\TOline{mksize}{4pt}{symbol size (mark).}
\TOline{mkcolor}{black}{symbol color (mark).}
\TOline{mkpos}{0.5}{position of the symbol on the arc.}
\end{tabular} 
\end{NewMacroBox}  

\DeleteShortVerb{\|}
\subsubsection{Example with \tkzname{mark = x} and with \tkzname{mark =||}}

\begin{tkzexample}[latex=6cm,small]
\begin{tikzpicture}[scale=.75]
    \tkzDefPoints{0/0/O,5/0/A,3/4/B}
    \tkzMarkAngle[size = 4,mark = x,
                  arc=ll,mkcolor = red,mkpos=.33](A,O,B)
    \tkzMarkAngle[size = 2,mark = ||,
                arc=ll,mkcolor = blue,mkpos=.66](A,O,B)
    \tkzDrawLines(O,A O,B)
    \tkzDrawPoints(O,A,B)
\end{tikzpicture}
\end{tkzexample}

\MakeShortVerb{\|}
\begin{NewMacroBox}{tkzMarkAngles}{\oarg{local options}\parg{A,O,B}\parg{A',O',B'}etc.}%
With common options, there is a macro for multiple angles.
  \end{NewMacroBox}  

  
\subsection{Marking a right angle: {\tkzcname{tkzMarkRightAngle}}}

\begin{NewMacroBox}{tkzMarkRightAngle}{\oarg{local options}\parg{A,O,B}}%
The \tkzname{german} option allows you to change the style of the drawing. The option \tkzname{size} allows to change the size of the drawing.

\medskip
\begin{tabular}{lll}%
\toprule
options             & default & definition         \\ 
\midrule
\TOline{german}{normal}{ german arc with inner point.}
\TOline{size}{0.2}{ side size.}
\end{tabular} 
\end{NewMacroBox}  

\subsubsection{Example of marking a right angle} 
\begin{tkzexample}[latex=6cm,small]
\begin{tikzpicture}
  \tkzDefPoints{0/0/A,3/1/B,0.9/-1.2/P}
  \tkzDefPointBy[projection = onto B--A](P)  \tkzGetPoint{H}
  \tkzDrawLines[add=.5 and .5](P,H)
  \tkzMarkRightAngle[fill=blue!20,size=.5,draw](A,H,P) 
  \tkzDrawLines[add=.5 and .5](A,B)
  \tkzMarkRightAngle[fill=red!20,size=.8](B,H,P)
  \tkzDrawPoints[](A,B,P,H)  
\end{tikzpicture}
\end{tkzexample}

\subsubsection{Example of marking a right angle, german style} 
\begin{tkzexample}[latex=6cm,small]
\begin{tikzpicture}
  \tkzDefPoints{0/0/A,3/1/B,0.9/-1.2/P}
  \tkzDefPointBy[projection = onto B--A](P)  \tkzGetPoint{H}
  \tkzDrawLines[add=.5 and .5](P,H)
  \tkzMarkRightAngle[german,size=.5,draw](A,H,P) 
  \tkzDrawPoints[](A,B,P,H) 
  \tkzDrawLines[add=.5 and .5](A,B)
  \tkzMarkRightAngle[german,size=.8](P,H,B) 
\end{tikzpicture}
\end{tkzexample}

\subsubsection{Mix of styles} 
\begin{tkzexample}[latex=6cm,small]
\begin{tikzpicture}[scale=.75]
  \tkzDefPoint(0,0){A}
  \tkzDefPoint(4,1){B}
  \tkzDefPoint(2,5){C}
  \tkzDefPointBy[projection=onto B--A](C) 
      \tkzGetPoint{H}
  \tkzDrawLine(A,B)
  \tkzDrawLine[add = .5 and .2,color=red](C,H)
  \tkzMarkRightAngle[,size=1,color=red](C,H,A)
  \tkzMarkRightAngle[german,size=.8,color=blue](B,H,C)
  \tkzFillAngle[opacity=.2,fill=blue!20,size=.8](B,H,C)
  \tkzLabelPoints(A,B,C,H)
  \tkzDrawPoints(A,B,C,H)
\end{tikzpicture}
\end{tkzexample}

\subsubsection{Full example} 

\begin{tkzexample}[latex=6cm,small]
\begin{tikzpicture}[rotate=-90]
\tkzDefPoint(0,1){A}
\tkzDefPoint(2,4){C}
\tkzDefPointWith[orthogonal normed,K=7](C,A)
\tkzGetPoint{B}
\tkzDrawSegment[green!60!black](A,C)
\tkzDrawSegment[green!60!black](C,B)
\tkzDrawSegment[green!60!black](B,A)
\tkzDefSpcTriangle[orthic](A,B,C){N,O,P}
\tkzDrawLine[dashed,color=magenta](C,P)
\tkzLabelPoint[left](A){$A$}
\tkzLabelPoint[right](B){$B$}
\tkzLabelPoint[above](C){$C$}
\tkzLabelPoint[left](P){$P$}
\tkzLabelSegment[auto](B,A){$c$}
\tkzLabelSegment[auto,swap](B,C){$a$}
\tkzLabelSegment[auto,swap](C,A){$b$}
\tkzMarkAngle[size=1,color=cyan,mark=|](C,B,A)
\tkzMarkAngle[size=1,color=cyan,mark=|](A,C,P)
\tkzMarkAngle[size=0.75,color=orange,
    mark=||](P,C,B)
\tkzMarkAngle[size=0.75,color=orange,
   mark=||](B,A,C)
\tkzMarkRightAngle[german](A,C,B)
\tkzMarkRightAngle[german](B,P,C)
\end{tikzpicture} 
\end{tkzexample} 

\subsection{\tkzcname{tkzMarkRightAngles}}
\begin{NewMacroBox}{tkzMarkRightAngles}{\oarg{local options}\parg{A,O,B}\parg{A',O',B'}etc.}%
With common options, there is a macro for multiple angles.
\end{NewMacroBox}

\subsection{Angles Library} % (fold)
\label{sub:angles_library}

If you prefer to use  \TIKZ\ library \tkzname{angles}, you can mark angles with the macro \tkzcname{tkzPicAngle} and \tkzcname{tkzPicRightAngle}.

\begin{NewMacroBox}{tkzPicAngle}{\oarg{tikz options}\parg{A,O,B}}%
  
\medskip
\begin{tabular}{lll}%
\toprule
options             & example & definition         \\ 
\midrule
\TOline{tikz option}{see below}{drawing of the angle $\widehat{AOB}$.}
\end{tabular} 
\end{NewMacroBox}  

\begin{NewMacroBox}{tkzPicRightAngle}{\oarg{tikz options}\parg{A,O,B}}%
  
\medskip
\begin{tabular}{lll}%
\toprule
options             & example & definition         \\ 
\midrule
\TOline{tikz option}{see below}{drawing of the right angle $\widehat{AOB}$.}
\end{tabular} 

\medskip
\emph{You need to know possible options of the \tkzname{angles} library}
\end{NewMacroBox} 

\subsubsection{Angle with \TIKZ} % (fold)
\label{ssub:angle_with_tikz}


\begin{tkzexample}[latex=7cm,small]
  \begin{tikzpicture}
  \tkzDefPoints{0/0/A,4/0/B}
  \tkzDefTriangle[right,swap](A,B) \tkzGetPoint{C}
  \tkzDrawPolygon(A,B,C)
  \tkzDrawPoints(A,B,C)
  \tkzLabelPoints[below](B,A)
  \tkzLabelPoints[above right](C)
  \tkzPicAngle["$\alpha$",draw=orange,
               <->,angle eccentricity=1.2,
               angle radius=1cm](B,A,C)
  \tkzPicRightAngle[draw,red,thick,
                angle eccentricity=.5,
                pic text=.](C,B,A)
  \end{tikzpicture}
\end{tkzexample}

% subsubsection angle_with_tikz (end)
% subsection angles_library (end)
\endinput

\part{Labelling}
\section{Labelling} 
\subsection{Label for a point} 
\hypertarget{tlp}{}
It is possible to add several labels at the same point by using this macro several times.  

\begin{NewMacroBox}{tkzLabelPoint}{\oarg{local options}\parg{point}\var{label}}%
\begin{tabular}{lll}%
arguments &  example  &                  \\ 
\midrule
\TAline{point}{\tkzcname{tkzLabelPoint(A)\{\$A\_1\$\}}}{}
options  & default & definition\\
\midrule
\TOline{TikZ options}{}{colour, position etc.}
\bottomrule
\end{tabular}

\medskip
Optionally, we can use any style of \TIKZ, especially placement with above, right, dots...
\end{NewMacroBox}

\subsubsection{Example with \tkzcname{tkzLabelPoint}} 
\begin{tkzexample}[latex=7cm,small]  
\begin{tikzpicture}
  \tkzDefPoint(0,0){A}
  \tkzDefPoint(4,0){B}
  \tkzDefPoint(0,3){C}
  \tkzDrawSegments(A,B B,C C,A)
  \tkzDrawPoints(A,B,C)
  \tkzLabelPoint[left,red](A){$A$}
  \tkzLabelPoint[right,blue](B){$B$}
  \tkzLabelPoint[above,purple](C){$C$}  
\end{tikzpicture} 
\end{tkzexample} 

\subsubsection{Label and reference}
 The reference of a point is the object that allows to use the point, the label is the name of the point that will be displayed.
 
\begin{tkzexample}[latex=6cm,small]
 \begin{tikzpicture}
    \tkzDefPoint(2,0){A} 
    \tkzDrawPoint(A)
    \tkzLabelPoint[above](A){$A_1$}  
  \end{tikzpicture}
 \end{tkzexample}
 
\subsection{Add labels to points \tkzcname{tkzLabelPoints}}
It is possible to place several labels quickly when the point references are identical to the labels and when the labels are placed in the same way in relation to the points. By default, \tkzname{below right} is chosen.
\hypertarget{tlps}{}  

\begin{NewMacroBox}{tkzLabelPoints}{\oarg{local options}\parg{$A_1,A_2,...$}}%
\begin{tabular}{lll}
arguments &  example & result                 \\ 
\midrule
\TAline{list of points}{\tkzcname{tkzLabelPoints(A,B,C)}}{Display of $A$, $B$ and $C$}
\bottomrule
\end{tabular}

\medskip
This macro reduces the number of lines of code, but it is not obvious that all points need the same label positioning.
\end{NewMacroBox}

\subsubsection{Example with \tkzcname{tkzLabelPoints}}   
\begin{tkzexample}[latex = 6cm,small]  
\begin{tikzpicture}
  \tkzDefPoint(2,3){A}
  \tkzDefShiftPoint[A](30:2){B}
  \tkzDefShiftPoint[A](30:5){C}
  \tkzDrawPoints(A,B,C)
  \tkzLabelPoints(A,B,C) 
\end{tikzpicture} 
\end{tkzexample}
%<--------------------------------------------------------------------------->
%                       tkzAutoLabelPoints
%<--------------------------------------------------------------------------->
\subsection{Automatic position of labels \tkzcname{tkzAutoLabelPoints}}
The label of a point is placed in a direction defined by a center and a point \tkzname{center}. The distance to the point is determined by a percentage of the distance between the center and the point. This percentage is given by \tkzname{dist}.
\begin{NewMacroBox}{tkzLabelPoints}{\oarg{local options}\parg{$A_1,A_2,...$}}%
\begin{tabular}{lll}
arguments &  example & result                 \\ 
\midrule
\TAline{list of points}{\tkzcname{tkzLabelPoint(A,B,C)}}{Display of $A$, $B$ and $C$}
\end{tabular}
\end{NewMacroBox}

\subsubsection{Label for points with \tkzcname{tkzAutoLabelPoints}} 
Here the points are positioned relative to the center of gravity of $A,B,C \text{ and } O$.
\begin{tkzexample}[latex=4cm,small]
\begin{tikzpicture}[scale=1]
 \tkzDefPoint(2,1){O}
 \tkzDefRandPointOn[circle=center O radius 1.5]\tkzGetPoint{A}
 \tkzDefPointBy[rotation=center O angle 100](A)\tkzGetPoint{C}
 \tkzDefPointBy[rotation=center O angle 78](A)\tkzGetPoint{B}
 \tkzDrawCircle(O,A) 
 \tkzDrawPoints(O,A,B,C) 
 \tkzDrawSegments(C,B B,A A,O O,C)
 \tkzDefCentroid(A,B,C,O)
 \tkzDrawPoint(tkzPointResult)
 \tkzLabelPoints(O,A,C,B)
\end{tikzpicture}
\end{tkzexample}

\section{Label for a segment} 
\hypertarget{tls}{}  
\begin{NewMacroBox}{tkzLabelSegment}{\oarg{local options}\parg{pt1,pt2}\marg{label}}
This macro allows you to place a label along a segment or a line. The options are those of \TIKZ\ for example \tkzname{pos}.

\medskip
\begin{tabular}{lll}%%
argument    & example & definition    \\
\midrule
\TAline{label}{\tkzcname{tkzLabelSegment(A,B)\{$5$\}}}{label text} 
\TAline{(pt1,pt2)}{(A,B)}{label along $[AB]$} 
\bottomrule
\end{tabular}

\medskip
\begin{tabular}{lll}%
options  & default & definition    \\
\midrule
\TOline{pos}{.5}{label's position} 
\end{tabular}
\end{NewMacroBox}  

\subsubsection{First example}      
\begin{tkzexample}[latex=7 cm,small]
\begin{tikzpicture}
\tkzDefPoint(0,0){A}
\tkzDefPoint(6,0){B}
\tkzDrawSegment(A,B)
\tkzLabelSegment[above,pos=.8](A,B){$a$}
\tkzLabelSegment[below,pos=.2](A,B){$4$}
\end{tikzpicture} 
\end{tkzexample}  

\subsubsection{Example : blackboard}  
\begin{tkzexample}[latex=6cm,small]
\tikzstyle{background rectangle}=[fill=black]
\begin{tikzpicture}[show background rectangle,scale=.4]
  \tkzDefPoint(0,0){O}
  \tkzDefPoint(1,0){I}
  \tkzDefPoint(10,0){A}
  \tkzDefPointWith[orthogonal normed,K=4](I,A)
   \tkzGetPoint{H}
  \tkzDefMidPoint(O,A) \tkzGetPoint{M}
  \tkzInterLC(I,H)(M,A)\tkzGetPoints{B}{C}   
  \tkzDrawSegments[color=white,line width=1pt](I,H O,A)
  \tkzDrawPoints[color=white](O,I,A,B,M) 
  \tkzMarkRightAngle[color=white,line width=1pt](A,I,B) 
  \tkzDrawArc[color=white,line width=1pt,
              style=dashed](M,A)(O)    
  \tkzLabelSegment[white,right=1ex,pos=.5](I,B){$\sqrt{a}$} 
  \tkzLabelSegment[white,below=1ex,pos=.5](O,I){$1$}   
  \tkzLabelSegment[pos=.6,white,below=1ex](I,A){$a$} 
\end{tikzpicture} 
\end{tkzexample}

\subsubsection{Labels and option : \tkzname{swap}}
\begin{tkzexample}[latex=7cm,small]
\begin{tikzpicture}[rotate=-60]
\tkzSetUpStyle[red,auto]{label style}
\tkzDefPoint(0,1){A}
\tkzDefPoint(2,4){C}
\tkzDefPointWith[orthogonal normed,K=7](C,A)
\tkzGetPoint{B}
\tkzDefSpcTriangle[orthic](A,B,C){N,O,P}
\tkzDefTriangleCenter[circum](A,B,C)
\tkzGetPoint{O}
\tkzDrawPolygon[green!60!black](A,B,C)
\tkzDrawLine[dashed,color=magenta](C,P)
\tkzLabelSegment(B,A){$c$}
\tkzLabelSegment[swap](B,C){$a$}
\tkzLabelSegment[swap](C,A){$b$}
\tkzMarkAngles[size=1,
     color=cyan,mark=|](C,B,A A,C,P)
\tkzMarkAngle[size=0.75,
     color=orange,mark=||](P,C,B)
\tkzMarkAngle[size=0.75,
      color=orange,mark=||](B,A,C)
\tkzMarkRightAngles[german](A,C,B B,P,C)
\tkzAutoLabelPoints[center = O,dist= .1](A,B,C)
 \tkzLabelPoint[below left](P){$P$}
 \end{tikzpicture} 
\end{tkzexample}

\hypertarget{tlss}{} 
 \begin{NewMacroBox}{tkzLabelSegments}{\oarg{local options}\parg{pt1,pt2 pt3,pt4 ...}}%
The arguments are a two-point couple list. The styles of \TIKZ\ are available for plotting.
\end{NewMacroBox} 
 
\subsubsection{Labels for an isosceles triangle}      
\begin{tkzexample}[latex=6cm,small]
\begin{tikzpicture}[scale=1]
 \tkzDefPoints{0/0/O,2/2/A,4/0/B,6/2/C}
 \tkzDrawSegments(O,A A,B)
 \tkzDrawPoints(O,A,B)
 \tkzDrawLine(O,B)   
 \tkzLabelSegments[color=red,above=4pt](O,A A,B){$a$}
\end{tikzpicture}
\end{tkzexample}  

\section{Add labels on a straight line \tkzcname{tkzLabelLine}}% 

\begin{NewMacroBox}{tkzLabelLine}{\oarg{local options}\parg{pt1,pt2}\marg{label}}
\begin{tabular}{lll}%
arguments &  default & definition   \\ 
\midrule
\TAline{label}{}{\tkzcname{tkzLabelLine(A,B)}\{\$\tkzcname{Delta}\$\}}
\bottomrule
\end{tabular}

\begin{tabular}{lll}%
options             & default & definition   \\ 
\midrule
\TOline{pos}{.5}{\tkzname{pos} is an option for \TIKZ, but essential in this case\dots} 
\end{tabular}

As an option, and in addition to the \tkzname{pos}, you can use all styles of \TIKZ, especially the placement with \tkzname{above}, \tkzname{right}, \dots
\end{NewMacroBox}

\subsubsection{Example with \tkzcname{tkzLabelLine}}
An important option is \tkzname{pos}, it's the one that allows you to place the label along the right. The value of \tkzname{pos} can be greater than 1 or negative.

\begin{tkzexample}[latex=6cm,small]
\begin{tikzpicture}
   \tkzDefPoints{0/0/A,3/0/B,1/1/C}
   \tkzDefLine[perpendicular=through C,K=-1](A,B)
   \tkzGetPoint{c}
   \tkzDrawLines(A,B C,c)
   \tkzLabelLine[pos=1.25,blue,right](C,c){$(\delta)$} 
   \tkzLabelLine[pos=-0.25,red,left](C,c){again $(\delta)$} 
\end{tikzpicture}
\end{tkzexample}

\subsection{Label at an angle : \tkzcname{tkzLabelAngle}}

\begin{NewMacroBox}{tkzLabelAngle}{\oarg{local options}\parg{A,O,B}}%
There is only one option, dist (with or without unit), which can be replaced by the TikZ's pos option (without unit for the latter). By default, the value is in centimeters.

\begin{tabular}{lll}%
  \toprule
options             & default & definition                        \\ 
\midrule
\TOline{pos}{1}{ or dist, controls the distance from the top to the label.}
\bottomrule
\end{tabular} 

\medskip 
It is possible to move the label with all TikZ options : rotate, shift, below, etc.
\end{NewMacroBox}  

\subsubsection{Example author js bibra stackexchange} 
\begin{tkzexample}[latex=7cm,small]
\begin{tikzpicture}[scale=.75]
  \tkzDefPoint(0,0){C}
  \tkzDefPoint(20:9){B}
  \tkzDefPoint(80:5){A}
  \tkzDefPointsBy[projection=onto B--C](A){a}
  \tkzDrawPolygon[thick,fill=yellow!15](A,B,C)
  \tkzDrawSegment[dashed, red](A,a)   
  \tkzDrawSegment[style=red, dashed, 
  dim={$10$,15pt,midway,font=\scriptsize,
   rotate=90}](A,a) 
  \tkzMarkAngle(B,C,A)
  \tkzMarkRightAngle(A,a,C)    
  \tkzMarkRightAngle(C,A,B)
  \tkzFillAngle[fill=blue!20, opacity=0.5](B,C,A)
  \tkzFillAngle[fill=red!20, opacity=0.5](A,B,C)
  \tkzLabelAngle[pos=1.25](A,B,C){$\beta$}
  \tkzLabelAngle[pos=1.25](B,C,A){$\alpha$}
  \tkzMarkAngle(A,B,C)
  \tkzDrawPoints(A,B,C)
  \tkzLabelPoints(B,C)
  \tkzLabelPoints[above](A)
\end{tikzpicture}
\end{tkzexample}

\subsubsection{With \tkzname{pos}} 
\begin{tkzexample}[latex=7cm,small]
\begin{tikzpicture}[scale=.75]
  \tkzDefPoints{0/0/O,5/0/A,3/4/B}
  \tkzMarkAngle[size = 4,mark = ||,
      arc=ll,color = red](A,O,B)%     
  \tkzDrawLines(O,A O,B)
  \tkzDrawPoints(O,A,B)
  \tkzLabelAngle[pos=2,draw,circle,
      fill=blue!10](A,O,B){$\alpha$} 
\end{tikzpicture}
\end{tkzexample}

\subsubsection{\tkzname{pos} and \tkzcname{tkzLabelAngles}} 
\begin{tkzexample}[latex=7cm,small]
\begin{tikzpicture}[rotate=30]
  \tkzDefPoint(2,1){S} 
  \tkzDefPoint(7,3){T}
  \tkzDefPointBy[rotation=center S angle 60](T)
  \tkzGetPoint{P} 
  \tkzDefLine[bisector,normed](T,S,P)
  \tkzGetPoint{s}
  \tkzDrawPoints(S,T,P)   
  \tkzDrawPolygon[color=blue](S,T,P) 
  \tkzDrawLine[dashed,color=blue,add=0 and 3](S,s)  
  \tkzLabelPoint[above right](P){$P$}
  \tkzLabelPoints(S,T)
  \tkzMarkAngle[size = 1.8,mark = |,arc=ll,
                    color = blue](T,S,P)
  \tkzMarkAngle[size = 2.1,mark = |,arc=l,
                    color = blue](T,S,s)
  \tkzMarkAngle[size = 2.3,mark = |,arc=l,
                    color = blue](s,S,P)  
 \tkzLabelAngle[pos = 1.5](T,S,P){$60^{\circ}$}%    
 \tkzLabelAngles[pos = 2.7](T,S,s s,S,P){%
                            $30^{\circ}$}%   
\end{tikzpicture}
\end{tkzexample}


\begin{NewMacroBox}{tkzLabelAngles}{\oarg{local options}\parg{A,O,B}\parg{A',O',B'}etc.}%
With common options, there is a macro for multiple angles.
\end{NewMacroBox}  

It finally remains to be able to give a label to designate a circle and if several possibilities are offered, we will see here \tkzcname{tkzLabelCircle}.

\subsection{Giving a label to a circle}
\begin{NewMacroBox}{tkzLabelCircle}{\oarg{tikz options}\parg{O,A}\parg{angle}\marg{label}}%
\begin{tabular}{lll}%
options             & default & definition                         \\
\midrule
\TOline{tikz options} {}{circle $O$ center  through $A$}
\bottomrule
\end{tabular} 

\medskip
\emph{ We can use the styles from \TIKZ. The label is created and therefore "passed" between braces.}
\end{NewMacroBox} 

\subsubsection{Example}  
\begin{tkzexample}[latex=5cm,small] 
\begin{tikzpicture}
 \tkzDefPoint(0,0){O} \tkzDefPoint(2,0){N}
 \tkzDefPointBy[rotation=center O angle 50](N) 
     \tkzGetPoint{M}
 \tkzDefPointBy[rotation=center O angle -20](N) 
      \tkzGetPoint{P}
 \tkzDefPointBy[rotation=center O angle 125](N) 
      \tkzGetPoint{P'}
 \tkzLabelCircle[above=4pt](O,N)(120){$\mathcal{C}$}
 \tkzDrawCircle(O,M) 
 \tkzFillCircle[color=blue!10,opacity=.4](O,M) 
 \tkzLabelCircle[draw,
       text width=2cm,text centered,left=24pt](O,M)(-120)%
          {The circle\\ $\mathcal{C}$}  
 \tkzDrawPoints(M,P)\tkzLabelPoints[right](M,P)   
\end{tikzpicture}      
\end{tkzexample} 

\section{Label for an arc} 
\hypertarget{tls}{}  
\begin{NewMacroBox}{tkzLabelArc}{\oarg{local options}\parg{pt1,pt2,pt3}\marg{label}}
This macro allows you to place a label along an arc. The options are those of \TIKZ\ for example \tkzname{pos}.

\medskip
\begin{tabular}{lll}%%
argument    & example & definition    \\
\midrule
\TAline{label}{\tkzcname{tkzLabelArc(A,B)\{$5$\}}}{label text} 
\TAline{(pt1,pt2,pt3)}{(O,A,B)}{label along the arc $\widearc{AB}$} 
\bottomrule
\end{tabular}

\medskip
\begin{tabular}{lll}%
options  & default & definition    \\
\midrule
\TOline{pos}{.5}{label's position} 
\end{tabular}
\end{NewMacroBox}  

\subsubsection{Label on arc}      
\begin{tkzexample}[latex=7 cm,small]
\begin{tikzpicture}
\tkzDefPoint(0,0){O}
\pgfmathsetmacro\r{2}
\tkzDefPoint(30:\r){A}
\tkzDefPoint(85:\r){B}
\tkzDrawCircle(O,A)
\tkzDrawPoints(B,A,O)
\tkzLabelArc[right=2pt](O,A,B){$\widearc{AB}$}
\tkzLabelPoints(A,B,O)
\end{tikzpicture}
\end{tkzexample}

\endinput

\part{Complements}
\section{Using the compass}    

\subsection{Main macro \tkzcname{tkzCompass}} 
\begin{NewMacroBox}{tkzCompass}{\oarg{local options}\parg{A,B}}%
This macro allows you to leave a compass trace, i.e. an arc at a designated point. The center must be indicated. Several specific options will modify the appearance of the arc as well as TikZ options such as style, color, line thickness etc.

You can define the length of the arc with the option |length| or the option |delta|.

\medskip
\begin{tabular}{lll}%
\toprule
options             & default & definition                        \\ 
\midrule
\TOline{delta} {0 (deg)}{Increases the angle of the arc symmetrically} 
\TOline{length}{1 (cm)}{Changes the length (in cm)}
\end{tabular}
\end{NewMacroBox} 

\subsubsection{Option \tkzname{length}} 
\begin{tkzexample}[latex=7cm,small]
\begin{tikzpicture}
  \tkzDefPoint(1,1){A}
  \tkzDefPoint(6,1){B}
  \tkzInterCC[R](A,4)(B,3)
  \tkzGetPoints{C}{D}
  \tkzDrawPoint(C)
  \tkzCompass[length=1.5](A,C)
  \tkzCompass(B,C)
  \tkzDrawSegments(A,B A,C B,C)
\end{tikzpicture}
\end{tkzexample}

\subsubsection{Option \tkzname{delta}} 
\begin{tkzexample}[latex=7cm,small]
\begin{tikzpicture} 
  \tkzDefPoint(0,0){A} 
  \tkzDefPoint(5,0){B}
  \tkzInterCC[R](A,4)(B,3)
  \tkzGetPoints{C}{D}
  \tkzDrawPoints(A,B,C) 
  \tkzCompass[delta=20](A,C)
  \tkzCompass[delta=20](B,C) 
  \tkzDrawPolygon(A,B,C)  
  \tkzMarkAngle(A,C,B)
\end{tikzpicture}
\end{tkzexample} 

\subsection{Multiple constructions \tkzcname{tkzCompasss}} 
\begin{NewMacroBox}{tkzCompasss}{\oarg{local options}\parg{pt1,pt2 pt3,pt4,\dots}}%
\tkzHandBomb\ Attention the arguments are lists of two points. This saves a few lines of code.

\medskip
\begin{tabular}{lll}%
\toprule
options             & default & definition                        \\ 
\midrule
\TOline{delta} {0}{Modifies the angle of the arc by increasing it symmetrically} 
\TOline{length}{1}{Changes the length} 
\end{tabular}
\end{NewMacroBox} 

\subsubsection{Use \tkzcname{tkzCompasss}} % (fold)
\label{ssub:use_tkzcname_tkzcompasss}

% subsubsection use_tkzcname_tkzcompasss (end)
\begin{tkzexample}[latex=7cm,small]
\begin{tikzpicture}[scale=.75]
 \tkzDefPoint(2,2){A}  \tkzDefPoint(5,-2){B}
 \tkzDefPoint(3,4){C}  \tkzDrawPoints(A,B) 
 \tkzDrawPoint[shape=cross out](C)    
 \tkzCompasss[new](A,B A,C B,C C,B) 
 \tkzShowLine[mediator,new,dashed,length = 2](A,B)
 \tkzShowLine[parallel = through C,
                     color=purple,length=2](A,B)
 \tkzDefLine[mediator](A,B)  
  \tkzGetPoints{i}{j}
 \tkzDefLine[parallel=through C](A,B) 
   \tkzGetPoint{D}
 \tkzDrawLines[add=.6 and .6](C,D A,C B,D)
 \tkzDrawLines(i,j) \tkzDrawPoints(A,B,C,i,j,D)  
 \tkzLabelPoints(A,B,C,i,j,D)
\end{tikzpicture}
\end{tkzexample} 

\endinput
\section{The Show}

\subsection{Show the constructions of some lines \tkzcname{tkzShowLine}}

 \begin{NewMacroBox}{tkzShowLine}{\oarg{local options}\parg{pt1,pt2} or \parg{pt1,pt2,pt3}}%
These constructions concern mediatrices, perpendicular or parallel lines passing through a given point and bisectors. The arguments are therefore lists of two or three points. Several options allow the adjustment of the constructions. The idea of this macro comes from \tkzimp{Yves Combe}.
  

\medskip 
\begin{tabular}{lll}%
\toprule
options      & default & definition    \\ 
\midrule
\TOline{mediator}{mediator}{displays the constructions of a mediator} 
\TOline{perpendicular}{mediator}{constructions for a perpendicular} 
\TOline{orthogonal}{mediator}{idem}
\TOline{bisector}{mediator}{constructions for a bisector}
\TOline{K}{1}{circle within a triangle }
\TOline{length}{1}{in cm, length of a arc}
\TOline{ratio} {.5}{arc length ratio}
\TOline{gap}{2}{placing the point of construction}
\TOline{size}{1}{radius of an arc (see bisector)}
 \bottomrule
\end{tabular}

\medskip
You have to add, of course, all the styles of \TIKZ\ for tracings\dots
\end{NewMacroBox}

\subsubsection{Example of \tkzcname{tkzShowLine} and \tkzname{parallel}} 
\begin{tkzexample}[latex=7cm,small]
\begin{tikzpicture}
 \tkzDefPoints{-1.5/-0.25/A,1/-0.75/B,-1.5/2/C}
 \tkzDrawLine(A,B)
 \tkzDefLine[parallel=through C](A,B) \tkzGetPoint{c} 
 \tkzShowLine[parallel=through C](A,B)
 \tkzDrawLine(C,c) \tkzDrawPoints(A,B,C,c)
\end{tikzpicture}
\end{tkzexample}

\subsubsection{Example of \tkzcname{tkzShowLine} and \tkzname{perpendicular}} 
\begin{tkzexample}[latex=5cm,small]
\begin{tikzpicture}
\tkzDefPoints{0/0/A, 3/2/B, 2/2/C} 
\tkzDefLine[perpendicular=through C,K=-.5](A,B) \tkzGetPoint{c}
\tkzShowLine[perpendicular=through C,K=-.5,gap=3](A,B)
\tkzDefPointBy[projection=onto A--B](c)\tkzGetPoint{h} 
\tkzMarkRightAngle[fill=lightgray](A,h,C) 
\tkzDrawLines[add=.5 and .5](A,B C,c) 
\tkzDrawPoints(A,B,C,h,c)
\end{tikzpicture}
\end{tkzexample}

\subsubsection{Example of \tkzcname{tkzShowLine} and \tkzname{bisector}} 
\begin{tkzexample}[latex=7 cm,small]
\begin{tikzpicture}[scale=1.25]
 \tkzDefPoints{0/0/A, 4/2/B, 1/4/C}
 \tkzDrawPolygon(A,B,C) 
 \tkzSetUpCompass[color=brown,line width=.1 pt]
 \tkzDefLine[bisector](B,A,C)  \tkzGetPoint{a}
 \tkzDefLine[bisector](C,B,A)  \tkzGetPoint{b}
 \tkzInterLL(A,a)(B,b) \tkzGetPoint{I}
 \tkzDefPointBy[projection = onto A--B](I) 
   \tkzGetPoint{H}
 \tkzShowLine[bisector,size=2,gap=3,blue](B,A,C)
 \tkzShowLine[bisector,size=2,gap=3,blue](C,B,A)   
 \tkzDrawCircle[color=blue,%
 line width=.2pt](I,H) 
 \tkzDrawSegments[color=red!50](I,tkzPointResult)
 \tkzDrawLines[add=0 and -0.3,color=red!50](A,a B,b) 
\end{tikzpicture}
\end{tkzexample}

\subsubsection{Example of \tkzcname{tkzShowLine} and \tkzname{mediator}} 
\begin{tkzexample}[latex=7 cm,small]
\begin{tikzpicture}
\tkzDefPoint(2,2){A}
\tkzDefPoint(5,4){B}
\tkzDrawPoints(A,B) 
\tkzShowLine[mediator,color=orange,length=1](A,B) 
\tkzGetPoints{i}{j}
\tkzDrawLines[add=-0.1 and -0.1](i,j)
\tkzDrawLines(A,B)
\tkzLabelPoints[below =3pt](A,B)
\end{tikzpicture}
\end{tkzexample}

\subsection{Constructions of certain transformations \addbs{tkzShowTransformation}}
\begin{NewMacroBox}{tkzShowTransformation}{\oarg{local options}\parg{pt1,pt2} or \parg{pt1,pt2,pt3}}%
These constructions concern orthogonal symmetries, central symmetries, orthogonal projections and translations. Several options allow the adjustment of the constructions. The idea of this macro comes from \tkzimp{Yves Combe}.
  
\medskip 
\begin{tabular}{lll}%
\toprule
options             & default & definition                        \\ 
\midrule
\TOline{reflection= over pt1--pt2}{reflection}{constructions of orthogonal symmetry} 
\TOline{symmetry=center pt}{reflection}{constructions of central symmetry} 
\TOline{projection=onto pt1--pt2}{reflection}{constructions of a projection}
\TOline{translation=from pt1 to pt2}{reflection}{constructions of a translation}
\TOline{K}{1}{circle within a triangle }
\TOline{length}{1}{arc length}
\TOline{ratio} {.5}{arc length ratio}
\TOline{gap}{2}{placing the point of construction}
\TOline{size}{1}{radius of an arc (see bisector)}
\end{tabular}
\end{NewMacroBox}

\subsubsection{Example of the use of \tkzcname{tkzShowTransformation}} 

\begin{tkzexample}[latex=6cm,small]
\begin{tikzpicture}[scale=.6]
  \tkzDefPoint(0,0){O} \tkzDefPoint(2,-2){A}
  \tkzDefPoint(70:4){B} \tkzDrawPoints(A,O,B)
  \tkzLabelPoints(A,O,B)
  \tkzDrawLine[add= 2 and 2](O,A)
  \tkzDefPointBy[translation=from O to A](B) 
  \tkzGetPoint{C}
  \tkzDrawPoint[color=orange](C)  \tkzLabelPoints(C)
  \tkzShowTransformation[translation=from O to A,%
             length=2](B) 
  \tkzDrawSegments[->,color=orange](O,A B,C)  
  \tkzDefPointBy[reflection=over O--A](B) \tkzGetPoint{E}
  \tkzDrawSegment[blue](B,E)
  \tkzDrawPoint[color=blue](E)\tkzLabelPoints(E) 
  \tkzShowTransformation[reflection=over O--A,size=2](B)   
  \tkzDefPointBy[symmetry=center O](B) \tkzGetPoint{F} 
  \tkzDrawSegment[color=green](B,F)
  \tkzDrawPoint[color=green](F)\tkzLabelPoints(F)
  \tkzShowTransformation[symmetry=center O,%
                      length=2](B) 
  \tkzDefPointBy[projection=onto O--A](C) 
  \tkzGetPoint{H}    
  \tkzDrawSegments[color=magenta](C,H)
  \tkzDrawPoint[color=magenta](H)\tkzLabelPoints(H)
  \tkzShowTransformation[projection=onto O--A,%
                         color=red,size=3,gap=-2](C)   
\end{tikzpicture}
\end{tkzexample}

\subsubsection{Another example of the use of \tkzcname{tkzShowTransformation}} 

You'll find this figure again, but without the construction features.
\begin{tkzexample}[latex=7cm,small]  
\begin{tikzpicture}[scale=.6]
  \tkzDefPoints{0/0/A,8/0/B,3.5/10/I}
  \tkzDefMidPoint(A,B) \tkzGetPoint{O} 
  \tkzDefPointBy[projection=onto A--B](I) 
     \tkzGetPoint{J}
  \tkzInterLC(I,A)(O,A)  \tkzGetPoints{M}{M'}
  \tkzInterLC(I,B)(O,A)  \tkzGetPoints{N}{N'}
  \tkzDefMidPoint(A,B) \tkzGetPoint{M}    
  \tkzDrawSemiCircle(M,B)
  \tkzDrawSegments(I,A I,B A,B B,M A,N) 
  \tkzMarkRightAngles(A,M,B A,N,B)  
  \tkzDrawSegment[style=dashed,color=blue](I,J)
  \tkzShowTransformation[projection=onto A--B,
                  color=red,size=3,gap=-3](I)
  \tkzDrawPoints[color=red](M,N)
  \tkzDrawPoints[color=blue](O,A,B,I,M) 
  \tkzLabelPoints(O)  
  \tkzLabelPoints[above right](N,I) 
  \tkzLabelPoints[below left](M,A) 
\end{tikzpicture} 
\end{tkzexample} 

\endinput
\section{Protractor}
Based on an idea by Yves Combe, the following macro allows you to draw a protractor. 
The operating principle is even simpler. Just name a half-line (a ray). The protractor will be placed on the origin $O$, the direction of the half-line is given by $A$. The angle is measured in the direct direction of the trigonometric circle.
\subsection{The macro \tkzcname{tkzProtractor}} % (fold)
\label{sub:the_macro_tkzcname_tkzprotractor}

% subsection the_macro_tkzcname_tkzprotractor (end)
\begin{NewMacroBox}{tkzProtractor}{\oarg{local options}\parg{$O,A$}}%
\begin{tabular}{lll}%
options    & default & definition     \\ 
\midrule
\TOline{lw}  {0.4 pt} {line thickness}
\TOline{scale}  {1} {ratio: adjusts the size of the protractor} 
\TOline{return} {false} {trigonometric circle indirect}
\end{tabular}
\end{NewMacroBox}

\subsubsection{The circular protractor} 
Measuring in the forward direction

\begin{tkzexample}[latex=7cm,small] 
\begin{tikzpicture}[scale=.5]
\tkzDefPoint(2,0){A}\tkzDefPoint(0,0){O}
\tkzDefShiftPoint[A](31:5){B}
\tkzDefShiftPoint[A](158:5){C}
\tkzDrawPoints(A,B,C)
\tkzDrawSegments[color = red,
    line width = 1pt](A,B A,C)
  \tkzProtractor[scale = 1](A,B) 
\end{tikzpicture}
\end{tkzexample}  

\subsubsection{The circular protractor, transparent and returned}

\begin{tkzexample}[latex=7cm,small] 
\begin{tikzpicture}[scale=.5]
  \tkzDefPoint(2,3){A}
  \tkzDefShiftPoint[A](31:5){B}  
   \tkzDefShiftPoint[A](158:5){C}   
  \tkzDrawSegments[color=red,line width=1pt](A,B A,C)
  \tkzProtractor[return](A,C) 
\end{tikzpicture}
\end{tkzexample}
\endinput
\section{Miscellaneous tools and mathematical tools}
\subsection{Duplicate a segment} 
This involves constructing a segment on a given half-line of the same length as a given segment.

\begin{NewMacroBox}{tkzDuplicateSegment}{\parg{pt1,pt2}\parg{pt3,pt4}\marg{pt5}}%
This involves creating a segment on a given half-line of the same length as a given segment . It is in fact the definition of a point.
\tkzcname{tkzDuplicateSegment} is the new name of \tkzcname{tkzDuplicateLen}.

\medskip  
\begin{tabular}{lll}%
\toprule
arguments             & example & explanation                         \\ 

\midrule
\TAline{(pt1,pt2)(pt3,pt4)\{pt5\}} {\tkzcname{tkzDuplicateSegment}(A,B)(E,F)\{C\}}{AC=EF and $C \in [AB)$} \\  
\bottomrule
\end{tabular}

\medskip
\emph{The macro \tkzcname{tkzDuplicateLength} is identical to this one. }
\end{NewMacroBox}

\subsubsection{Use of\tkzcname{tkzDuplicateSegment}} 

\begin{tkzexample}[latex=6cm,small]
\begin{tikzpicture}[scale=.5]
 \tkzDefPoints{0/0/A,2/-3/B,2/5/C}
 \tkzDuplicateSegment(A,B)(A,C)  
 \tkzGetPoint{D}
 \tkzDrawSegments[new](A,B A,C)
 \tkzDrawSegment[teal](A,D)
 \tkzDrawPoints[new](A,B,C,D) 
 \tkzLabelPoints[above right=3pt](A,B,C,D)
\end{tikzpicture} 
\end{tkzexample} 

\subsubsection{Proportion of gold with \tkzcname{tkzDuplicateSegment}} 
\begin{tkzexample}[latex=7cm,small]
\begin{tikzpicture}[rotate=-90,scale=.4]
 \tkzDefPoints{0/0/A,10/0/B}
 \tkzDefMidPoint(A,B)   
 \tkzGetPoint{I}
 \tkzDefPointWith[orthogonal,K=-.75](B,A)
 \tkzGetPoint{C}
 \tkzInterLC(B,C)(B,I)  \tkzGetSecondPoint{D}
 \tkzDuplicateSegment(B,D)(D,A) \tkzGetPoint{E}
 \tkzInterLC(A,B)(A,E)   \tkzGetPoints{N}{M}
 \tkzDrawArc[orange,delta=10](D,E)(B)
 \tkzDrawArc[orange,delta=10](A,M)(E)
 \tkzDrawLines(A,B B,C A,D)
 \tkzDrawArc[orange,delta=10](B,D)(I)
 \tkzDrawPoints(A,B,D,C,M,I)
 \tkzLabelPoints[below left](A,B,D,C,M,I)
\end{tikzpicture}
\end{tkzexample}

\subsubsection{Golden triangle or sublime triangle} 
\begin{tkzexample}[latex=7cm,small]
\begin{tikzpicture}[scale=.75] 
  \tkzDefPoints{0/0/A,5/0/C,0/5/B} 
  \tkzDefMidPoint(A,C)\tkzGetPoint{H} 
  \tkzDuplicateSegment(H,B)(H,A)\tkzGetPoint{D} 
  \tkzDuplicateSegment(A,D)(A,B)\tkzGetPoint{E} 
  \tkzDuplicateSegment(A,D)(B,A)\tkzGetPoint{G} 
  \tkzInterCC(A,C)(B,G)\tkzGetSecondPoint{F}
  \tkzDrawLine(A,C)
  \tkzDrawArc(A,C)(B) 
  \begin{scope}[arc style/.style={color=gray,%
                               style=dashed}]
    \tkzDrawArc(H,B)(D) 
    \tkzDrawArc(A,D)(B) 
    \tkzDrawArc(B,G)(F) 
  \end{scope}
  \tkzDrawSegment[dashed](H,B) 
  \tkzCompass(B,F) 
  \tkzDrawPolygon[new](A,B,F)
  \tkzDrawPoints(A,...,H)
  \tkzLabelPoints[below left](A,...,H)
\end{tikzpicture}
\end{tkzexample}

\subsection{Segment length \tkzcname{tkzCalcLength}}
There's an option in \TIKZ\  named \tkzname{veclen}. This option
 is used to calculate AB if A and B are two points.

The only problem for me is that the version of \TIKZ\ is not accurate enough in some cases. My version uses the \tkzNamePack{xfp} package and is slower, but more accurate.

\begin{NewMacroBox}{tkzCalcLength}{\oarg{local options}\parg{pt1,pt2}}%
You can store the result with the macro \tkzcname{tkzGetLength} for example \tkzcname{tkzGetLength\{dAB\}} \\
defines the macro \tkzcname{dAB}.

\medskip
\begin{tabular}{lll}%
\toprule
arguments    & example & explanation       \\
\midrule
\TAline{(pt1,pt2)\{name of macro\}} {\tkzcname{tkzCalcLength}(A,B)}{\tkzcname{dAB} gives $AB$ in cm}
\bottomrule
\end{tabular}

\medskip
Only one option

\begin{tabular}{lll}%
   
\toprule
 options    & default & example       \\
\midrule
\TOline{cm}  {true}{\tkzcname{tkzCalcLength}(A,B) After \tkzcname{tkzGetLength\{dAB\}} \tkzcname{dAB} gives $AB$ in cm}
\end{tabular}
\end{NewMacroBox}

\subsubsection{Compass square construction}

\begin{tkzexample}[latex=7cm,small]
\begin{tikzpicture}[scale=1]
  \tkzDefPoint(0,0){A} \tkzDefPoint(4,0){B}
  \tkzCalcLength(A,B)\tkzGetLength{dAB}
  \tkzDefLine[perpendicular=through A](A,B)
  \tkzGetPoint{D}
  \tkzDefPointWith[orthogonal,K=-1](B,A)    
    \tkzGetPoint{F}
  \tkzGetPoint{C}
  \tkzDrawLine[add= .6 and .2](A,B)
  \tkzDrawLine(A,D) 
  \tkzShowLine[orthogonal=through A,gap=2](A,B)
  \tkzMarkRightAngle(B,A,D)
  \tkzCompasss(A,D D,C)
  \tkzDrawArc[R](B,\dAB)(80,110)
  \tkzDrawPoints(A,B,C,D)
  \tkzDrawSegments[color=gray,style=dashed](B,C C,D)
  \tkzLabelPoints[below left](A,B,C,D)
\end{tikzpicture}
\end{tkzexample}


\subsubsection{Example}
The macro \tkzcname{tkzDefCircle[radius](A,B)} defines the radius that we retrieve with \tkzcname{tkzGetLength},  this result is in \tkzname{cm}.

\begin{tkzexample}[latex=6cm,small]
\begin{tikzpicture}[scale=.5]
 \tkzDefPoint(0,0){A}
 \tkzDefPoint(3,-4){B}
 \tkzDefMidPoint(A,B) \tkzGetPoint{M}
 \tkzCalcLength(M,B)\tkzGetLength{rAB}
 \tkzDrawCircle(A,B)
 \tkzDrawPoints(A,B)
 \tkzLabelPoints(A,B)
 \tkzDrawSegment[dashed](A,B)
 \tkzLabelSegment(A,B){$\pgfmathprintnumber{\rAB}$}
\end{tikzpicture}
\end{tkzexample}


\subsection{Transformation from pt to cm or cm to pt}
Not sure if this is necessary and it is only a division by 28.45274 and a multiplication by the same number. The macros are:

\begin{NewMacroBox}{tkzpttocm}{\parg{number}\marg{name of macro}}%
The result is stored in a macro.

\medskip
\begin{tabular}{lll}%
\toprule
arguments             & example & explanation                         \\
\midrule
\TAline{(number)\{name of macro\}} {\tkzcname{tkzpttocm}(120)\{len\}}{\tkzcname{len} gives a number of tkzname{cm}}
\bottomrule
\end{tabular}

\medskip
You'll have to use \tkzcname{len} along with \tkzname{cm}.
\end{NewMacroBox}

\subsection{Change of unit} 
\begin{NewMacroBox}{tkzcmtopt}{\parg{number}\marg{name of macro}}%
The result is stored in a macro.

\medskip
\begin{tabular}{lll}
\toprule
arguments             & example & explanation                         \\
\midrule
\TAline{(number)\{name of macro\}}{\tkzcname{tkzcmtopt}(5)\{len\}}{\tkzcname{len} length in \tkzname{pts}}
\bottomrule
\end{tabular}

\medskip
\emph{The result can be used with \tkzcname{len}\ \tkzname{pt}}
\end{NewMacroBox}


\subsection{Get point coordinates}
%<--------------------------------------------------------------------------–>
%                    Coordonnées d'un point 
%    result in #2x and #2y    #1 is the point and we get its coordinates
% use either $A$ one point \tkzGetPointCoord(A){V} then \Vx = xA and \Vy = yA
% in cm 
% tkzGetPointCoord with [#1] cm or  pt ?? todo
%<--------------------------------------------------------------------------–>
\begin{NewMacroBox}{tkzGetPointCoord}{\parg{$A$}\marg{name of macro}}%
\begin{tabular}{lll}%
arguments             & example & explanation                         \\
\midrule
\TAline{(point)\{name of macro\}} {\tkzcname{tkzGetPointCoord}(A)\{A\}}{\tkzcname{Ax} and \tkzcname{Ay} give coordinates for $A$}
\end{tabular}

\medskip
\emph{Stores in two macros the coordinates of a point. If the name of the macro is \tkzname{p}, then \tkzcname{px} and \tkzcname{py} give the coordinates of the chosen point with the cm as unit.}
\end{NewMacroBox}

\subsubsection{Coordinate transfer with \tkzcname{tkzGetPointCoord}}

\begin{tkzexample}[width=8cm,small]
\begin{tikzpicture}
 \tkzInit[xmax=5,ymax=3]
 \tkzGrid[sub,orange]
 \tkzDrawX \tkzDrawY
 \tkzDefPoint(1,0){A}
 \tkzDefPoint(4,2){B}
 \tkzGetPointCoord(A){a}
 \tkzGetPointCoord(B){b}
 \tkzDefPoint(\ax,\ay){C}
 \tkzDefPoint(\bx,\by){D}
 \tkzDrawPoints[color=red](C,D)
\end{tikzpicture}
\end{tkzexample}

\subsubsection{Sum of vectors with \tkzcname{tkzGetPointCoord}}
\begin{tkzexample}[width=6cm,small]
\begin{tikzpicture}[>=latex]
  \tkzDefPoint(1,4){a}
  \tkzDefPoint(3,2){b}
  \tkzDefPoint(1,1){c}
  \tkzDrawSegment[->,red](a,b)
  \tkzGetPointCoord(c){c}
  \draw[color=blue,->](a) -- ([shift=(b)]\cx,\cy) ;
  \draw[color=purple,->](b) -- ([shift=(b)]\cx,\cy) ;
  \tkzDrawSegment[->,blue](a,c)
  \tkzDrawSegment[->,purple](b,c)
\end{tikzpicture}
\end{tkzexample}

\subsection{Swap labels of points}

\begin{NewMacroBox}{tkzSwapPoints}{\parg{$pt1$,$pt2$}}%
\begin{tabular}{lll}%
arguments             & example & explanation                         \\
\midrule
\TAline{(pt1,pt2)} {\tkzcname{tkzSwapPoints}(A,B)}{now $A$ has the coordinates of $B$ }
\end{tabular}

\emph{The points have exchanged their coordinates.}
\end{NewMacroBox}

\subsubsection{Use of \tkzcname{tkzSwapPoints}}

\begin{tkzexample}[width=6cm,small]
\begin{tikzpicture}
  \tkzDefPoints{0/0/O,5/-1/A,2/2/B}
   \tkzSwapPoints(A,B)
   \tkzDrawPoints(O,A,B)
   \tkzLabelPoints(O,A,B)
\end{tikzpicture}
\end{tkzexample}

\subsection{Dot Product}
In Euclidean geometry, the dot product of the Cartesian coordinates of two vectors is widely used.

\begin{NewMacroBox}{tkzDotProduct}{\parg{$pt1$,$pt2$,$pt3$}}%
  The dot product of two vectors $\overrightarrow{u} = [a,b]$ and  $\overrightarrow{v} = [a',b']$ is defined as: $\overrightarrow{u}\cdot \overrightarrow{v} = aa' + bb'$

$\overrightarrow{u} = \overrightarrow{pt1pt2}$ $\overrightarrow{v} = \overrightarrow{pt1pt3}$
  
\begin{tabular}{lll}%
arguments             & example & explanation                         \\
\midrule
\TAline{(pt1,pt2,pt3)} {\tkzcname{tkzDotProduct}(A,B,C)}{the result is $\overrightarrow{AB}\cdot \overrightarrow{AC}$}
\end{tabular}

\emph{The result is a number that can be retrieved with \tkzcname{tkzGetResult}.}
\end{NewMacroBox}

\subsubsection{Simple example} % (fold)
\label{ssub:simple_example}

\begin{tkzexample}[small,latex=7cm]
\begin{tikzpicture}
  \tkzDefPoints{-2/-3/A,4/0/B,1/3/C}
  \tkzDefPointBy[projection= onto A--B](C)  
  \tkzGetPoint{H}
  \tkzDrawSegment(C,H)
  \tkzMarkRightAngle(C,H,A)
  \tkzDrawSegments[vector style](A,B A,C)
  \tkzDrawPoints(A,H) \tkzLabelPoints(A,B,H)
  \tkzLabelPoints[above](C)
  \tkzDotProduct(A,B,C) \tkzGetResult{pabc}
  \pgfmathparse{round(10*\pabc)/10}
  \let\pabc\pgfmathresult
  \node at (1,-3) {%
  $\overrightarrow{PA}\cdot \overrightarrow{PB}=\pabc$};
  \tkzDotProduct(A,H,B) \tkzGetResult{phab}
  \pgfmathparse{round(10*\phab)/10}
  \let\phab\pgfmathresult
  \node at (1,-4) {$PA \times PH = \phab $};
\end{tikzpicture}
\end{tkzexample}
% subsubsection simple_example (end)


\subsubsection{Cocyclic points} % (fold)
\label{ssub:cocyclicpts}

\begin{tkzexample}[small,latex=7cm]
\begin{tikzpicture}[scale=.75]
  \tkzDefPoints{1/2/O,5/2/B,2/2/P,3/3/Q}
  \tkzInterLC[common=B](O,B)(O,B) \tkzGetFirstPoint{A}
  \tkzInterLC[common=B](P,Q)(O,B) \tkzGetPoints{C}{D}
  \tkzDrawCircle(O,B)
  \tkzDrawSegments(A,B C,D)
  \tkzDrawPoints(A,B,C,D,P)
  \tkzLabelPoints(P)
  \tkzLabelPoints[below left](A,C)
  \tkzLabelPoints[above right](B,D)
  \tkzDotProduct(P,A,B) \tkzGetResult{pab}
  \pgfmathparse{round(10*\pab)/10}
  \let\pab\pgfmathresult
  \tkzDotProduct(P,C,D) \tkzGetResult{pcd}
  \pgfmathparse{round(10*\pcd)/10}
  \let\pcd\pgfmathresult
  \node at (1,-3) {%
  $\overrightarrow{PA}\cdot \overrightarrow{PB} =
   \overrightarrow{PC}\cdot \overrightarrow{PD}$};
    \node at (1,-4)%
    {$\overrightarrow{PA}\cdot \overrightarrow{PB} =\pab$};
 \node at (1,-5){%
 $\overrightarrow{PC}\cdot \overrightarrow{PD} =\pcd$};
\end{tikzpicture}
\end{tkzexample}
% subsubsection cocyclicpts (end)

\newpage
\subsection{Power of a point with respect to a circle}

\begin{NewMacroBox}{tkzPowerCircle}{\parg{$pt1$}\parg{$pt2$,$pt3$}}%
\begin{tabular}{lll}%
arguments             & example & explanation                         \\
\midrule
\TAline{(pt1)(pt2,pt3)} {\tkzcname{tkzPowerCircle}(A)(O,M)}{power of $A$ with respect to the circle (O,A)}
\end{tabular}

\emph{The result is a number that represents the power of a point with respect to a circle.}
\end{NewMacroBox}

\subsubsection{Power from the radical axis} % (fold)
\label{ssub:power}

In this example, the radical axis $(EF)$ has been drawn. A point $H$ has been chosen on $(EF)$ and the power of the point $H$ with respect to the circle of center $A$ has been calculated as well as $PS^2$. You can check that the power of $H$ with respect to the circle of center $C$ as well as $HS'^2, HT^2, HT'^2$ give the same result.  

\begin{tkzexample}[small,latex=7cm]
\begin{tikzpicture}[scale=.5]
  \tkzDefPoints{-1/0/A,0/5/B,5/-1/C,7/1/D}
  \tkzDrawCircles(A,B C,D)
  \tkzDefRadicalAxis(A,B)(C,D) \tkzGetPoints{E}{F}
  \tkzDrawLine[add=1 and 2](E,F)
  \tkzDefPointOnLine[pos=1.5](E,F) \tkzGetPoint{H}
  \tkzDefLine[tangent from = H](A,B)\tkzGetPoints{T}{T'}
  \tkzDefLine[tangent from = H](C,D)\tkzGetPoints{S}{S'}
  \tkzDrawSegments(H,T H,T' H,S H,S')
  \tkzDrawPoints(A,B,C,D,E,F,H,T,T',S,S')
  \tkzPowerCircle(H)(A,B) \tkzGetResult{pw}
  \tkzDotProduct(H,S,S) \tkzGetResult{phtt}
  \node {Power $\approx \pw \approx \phtt$};
\end{tikzpicture}
\end{tkzexample}
% subsubsection power (end)

\subsection{Radical axis}

In geometry, the radical axis of two non-concentric circles is the set of points whose power with respect to the circles are equal. Here |\tkzDefRadicalAxis(A,B)(C,D)| gives the radical axis of the two circles $\mathcal{C}(A,B)$ and $\mathcal{C}(C,D)$. 

\begin{NewMacroBox}{tkzDefRadicalAxis}{\parg{$pt1$,$pt2$}\parg{$pt3$,$pt4$}}%
\begin{tabular}{lll}%
arguments             & example & explanation                         \\
\midrule
\TAline{(pt1,pt2)(pt3,pt4)} {\tkzcname{tkzDefRadicalAxis}(A,B)(C,D)}{Two circles with centers $A$ and $C$}
\midrule
\end{tabular}


\emph{The result is two points of the radical axis.}
\end{NewMacroBox}

\subsubsection{Two circles disjointed} % (fold)
\label{ssub:two_circles_disjointed}


\begin{tkzexample}[small,latex=8cm]
\begin{tikzpicture}[scale=.75]
    \tkzDefPoints{-1/0/A,0/2/B,4/-1/C,4/0/D}
    \tkzDrawCircles(A,B C,D)
    \tkzDefRadicalAxis(A,B)(C,D)
     \tkzGetPoints{E}{F}
    \tkzDrawLine[add=1 and 2](E,F)
    \tkzDrawLine[add=.5 and .5](A,C)
\end{tikzpicture}
\end{tkzexample}
% subsubsection two_circles_disjointed (end)

\subsection{Two intersecting circles} % (fold)
\label{sub:two_intersecting_circles}
\begin{tkzexample}[small,latex=8cm]
\begin{tikzpicture}[scale=.5]
  \tkzDefPoints{-1/0/A,0/2/B,3/-1/C,3/-2/D}
  \tkzDrawCircles(A,C B,D)
  \tkzDefRadicalAxis(A,C)(B,D)
  \tkzGetPoints{E}{F}
  \tkzDrawPoints(A,B,C,D,E,F)
    \tkzLabelPoints(A,B,C,D,E,F)
  \tkzDrawLine[add=.5 and 1](E,F)
  \tkzDrawLine[add=.25 and .25](A,B)
\end{tikzpicture}
\end{tkzexample}
% subsection two_intersecting_circles (end)


\subsection{Two externally tangent circles} % (fold)
\label{sub:two_externally_tangent_circles}
\begin{tkzexample}[small,latex=8cm]
\begin{tikzpicture}[scale=.5]
  \tkzDefPoints{0/0/A,4/0/B,6/0/C}
  \tkzDrawCircles(A,B C,B)
  \tkzDefRadicalAxis(A,B)(C,B)
  \tkzGetPoints{E}{F}
  \tkzDrawPoints(A,B,C,E,F)
    \tkzLabelPoints(A,B,C,E,F)
  \tkzDrawLine[add=1 and 1](E,F)
  \tkzDrawLine[add=.5 and .5](A,B)
\end{tikzpicture}
\end{tkzexample}
% subsection two_externally_tangent_circles (end)


\subsection{Two circles tangent internally} % (fold)
\label{sub:deux_cercles_tangents_interieurement}
\begin{tkzexample}[small,latex=8cm]
\begin{tikzpicture}[scale=.5]
  \tkzDefPoints{0/0/A,3/0/B,5/0/C}
  \tkzDrawCircles(A,C B,C)
  \tkzDefRadicalAxis(A,C)(B,C)
  \tkzGetPoints{E}{F}
  \tkzDrawPoints(A,B,C,E,F)
  \tkzLabelPoints[below right](A,B,C,E,F)
  \tkzDrawLine[add=1 and 1](E,F)
  \tkzDrawLine[add=.5 and .5](A,B)
\end{tikzpicture}
\end{tkzexample}
% subsection deux_cercles_tangents_interieurement (end)

\subsubsection{Three circles} % (fold)
\label{ssub:threecircles}



\begin{tkzexample}[small,latex=8cm]
\begin{tikzpicture}[scale=.4]
  \tkzDefPoints{0/0/A,5/0/a,7/-1/B,3/-1/b,5/-4/C,2/-4/c}
  \tkzDrawCircles(A,a B,b C,c)
  \tkzDefRadicalAxis(A,a)(B,b) \tkzGetPoints{i}{j}
  \tkzDefRadicalAxis(A,a)(C,c) \tkzGetPoints{k}{l}
  \tkzDefRadicalAxis(C,c)(B,b) \tkzGetPoints{m}{n}
  \tkzDrawLines[new](i,j k,l m,n)
\end{tikzpicture}
\end{tkzexample}
% subsubsection threecircles (end)
  
\subsection{\tkzcname{tkzIsLinear}, \tkzcname{tkzIsOrtho}}
 \begin{NewMacroBox}{tkzIsLinear}{\parg{$pt1$,$pt2$,$pt3$}}%
 \begin{tabular}{lll}%
 arguments             & example & explanation                         \\
 \midrule
 \TAline{(pt1,pt2,pt3)} {\tkzcname{tkzIsLinear}(A,B,C)}{$A,B,C$ aligned ?}
 \midrule
 \end{tabular}
 
 \emph{\tkzcname{tkzIsLinear} allows to test the alignment of the three points $pt1$,$pt2$,$pt3$. }
 \end{NewMacroBox}
 
 \begin{NewMacroBox}{tkzIsOrtho}{\parg{$pt1$,$pt2$,$pt3$}}%
 \begin{tabular}{lll}%
 arguments             & example & explanation                         \\
 \midrule
 \TAline{(pt1,pt2,pt3)} {\tkzcname{tkzIsOrtho}(A,B,C)}{$(AB)\perp (AC)$ ? }
 \midrule
 \end{tabular}
 
 \emph{\tkzcname{tkzIsOrtho} allows to test the orthogonality of lines $(pt1pt2)$ and $(pt1pt3)$. }
 \end{NewMacroBox}
 
 \subsubsection{Use of \tkzcname{tkzIsOrtho} and \tkzcname{tkzIsLinear}}
   
\begin{tkzexample}[small,latex=7cm]
  \begin{tikzpicture}
  \tkzDefPoints{1/-2/A,5/0/B}
  \tkzDefCircle[diameter](A,B) \tkzGetPoint{O}
  \tkzDrawCircle(O,A)
  \tkzDefPointBy[rotation= center O angle 60](B) 
  \tkzGetPoint{C}
  \tkzDefPointBy[rotation= center O angle 60](A) 
  \tkzGetPoint{D}
  \tkzDrawCircle(O,A)
  \tkzDrawPoints(A,B,C,D,O)
  \tkzIsOrtho(C,A,B)
  \iftkzOrtho
    \tkzDrawPolygon[blue](A,B,C)
  \tkzDrawPoints[blue](A,B,C,D)
  \else
  \tkzDrawPoints[red](A,B,C,D)
  \fi
   \tkzIsLinear(O,C,D)
    \iftkzLinear
    \tkzDrawSegment[orange](C,D)
    \fi
\end{tikzpicture}
  
\end{tkzexample}

  
\endinput

\part{Working with style}
\section{Predefined styles}\label{custom}
The way to proceed will depend on your use of the package. A method that seems to me to be correct is to use as much as possible predefined styles in order to separate the content from the form. This method will be the right one if you plan to create a document (like this documentation) with many figures. We will see how to define a global style for a document. We will see how to use a style locally.

The file \tkzname{tkz-euclide.cfg} contains the predefined styles of the main objects. Among these the most important are points, lines, segments, circles, arcs and compass traces.
 If you always use the same styles and if you create many figures then it is interesting to create your own styles . To do this you need to know what features you can modify. It will be necessary to know some notions of \TIKZ.
 
 The predefined styles are global styles. They exist before the creation of the figures. It is better to avoid changing them between two figures. On the other hand these styles can be modified in a figure temporarily. There the styles are defined locally and do not influence the other figures.
 
 For the document you are reading here is how I defined the different styles.

\begin{tkzltxexample}[]
  \tkzSetUpColors[background=white,text=black]  
  \tkzSetUpPoint[size=2,color=teal]
  \tkzSetUpLine[line width=.4pt,color=teal]
  \tkzSetUpCompass[color=orange, line width=.4pt,delta=10]
  \tkzSetUpArc[color=gray,line width=.4pt]
  \tkzSetUpStyle[orange]{new}
\end{tkzltxexample}

The macro \tkzcname{tkzSetUpColors} allows you to set the background color as well as the text color. If you don't use it, the colors of your document will be used as well as the fonts. Let's see how to define the styles of the main objects.

\section{Points style}
This is how the points  are defined :
\begin{tkzltxexample}[]
\tikzset{point style/.style = {%
       draw         = \tkz@euc@pointcolor,
       inner sep    = 0pt,
       shape        = \tkz@euc@pointshape,
       minimum size = \tkz@euc@pointsize,
       fill         = \tkz@euc@pointcolor}}  
\end{tkzltxexample}

It is of course possible to use \tkzcname{tikzset} but you can use a macro provided by the package. You can use the macro \tkzcname{tkzSetUpPoint} globally or locally, \\ Let's look at this possibility.

\subsection{Use of \tkzcname{tkzSetUpPoint}}

\begin{NewMacroBox}{tkzSetUpPoint}{\oarg{local options}}%
\begin{tabular}{lll}%
options &  default & definition                 \\ 
\midrule
\TOline{color}{black}{point color} 
\TOline{size}{3}{point size} 
\TOline{fill}{black!50}{inside point color} 
\TOline{shape}{circle}{point shape circle, cross or cross out} 
\end{tabular}
\end{NewMacroBox}



\subsubsection{Global style or local style}
First of all here is a figure created with the styles of my documentation, then the style of the points is modified within the environment \tkzNameEnv{tikzspicture}. 

You can use the macro \tkzcname{tkzSetUpPoint} globally or locally, If you place this macro in your preamble or before your first figure then the point style will be valid for all figures in your document. It will be possible to use another style locally by using this command within an environment \tkzNameEnv{tikzpicture}.\\ Let's look at this possibility.


\begin{tkzexample}[latex=7cm,small]
\begin{tikzpicture}
  \tkzDefPoints{0/0/A,5/0/B,3/2/C,3/1/D}
  \tkzDrawPolygon(A,B,C)
  \tkzDrawPoints(A,B,C)
  \tkzLabelPoints(A,B) 
  \tkzLabelPoints[above right](C) 
\end{tikzpicture}
\end{tkzexample}

\subsubsection{Local style}
The style of the points is modified locally in the second figure 
\begin{tkzexample}[latex=7cm,small]
\begin{tikzpicture}
  \tkzSetUpPoint[size=4,color=red,fill=red!20]
  \tkzDefPoints{0/0/A,5/0/B,3/2/C,3/1/D}
  \tkzDrawPolygon(A,B,C)
  \tkzDrawPoints(A,B,C)
  \tkzDrawPoint[shape=cross out,thick](D)
  \tkzLabelPoints(A,B) 
  \tkzLabelPoints[above right](C) 
\end{tikzpicture}
\end{tkzexample}

\subsubsection{\tkzname{Style} and \tkzname{scope}}
The points get back the initial style. Point D has a new style limited by the environment \tkzNameEnv{scope}. It is also possible to use |{...}| orThe points get back the initial style. Point $D$ has a new style limited by the environment \tkzNameEnv{scope}. It is also possible to use |{...}| or |\begingoup  ... \endgroup|.

\begin{tkzexample}[latex=7cm,small]
\begin{tikzpicture}
  \tkzDefPoints{0/0/A,5/0/B,3/2/C,3/1/D}
  \tkzDrawPolygon(A,B,C)
  \tkzDrawPoints(A,B,C)
  \begin{scope}
    \tkzSetUpPoint[size=4,color=red,fill=red!20]
    \tkzDrawPoint(D)
  \end{scope}
  \tkzLabelPoints(A,B) 
  \tkzLabelPoints[above right](C) 
\end{tikzpicture}
\end{tkzexample}

\subsubsection{Simple example with \tkzcname{tkzSetUpPoint}}

\begin{tkzexample}[latex=5cm,small]
\begin{tikzpicture}
  \tkzSetUpPoint[shape = cross out,color=blue]
  \tkzDefPoint(2,1){A}
  \tkzDefPoint(4,0){B}
  \tkzDrawLine(A,B)
  \tkzDrawPoints(A,B)
\end{tikzpicture}
\end{tkzexample}

\subsubsection{Use of \tkzcname{tkzSetUpPoint} inside a group}
\begin{tkzexample}[latex=8cm,small]
\begin{tikzpicture}
  \tkzDefPoints{0/0/A,2/4/B,4/0/C,3/2/D}
  \tkzDrawSegments(A,B A,C A,D)
  {\tkzSetUpPoint[shape=cross out,
            fill= teal!50,
            size=4,color=teal]
  \tkzDrawPoints(A,B)}
  \tkzSetUpPoint[fill= teal!50,size=4,
               color=teal]
   \tkzDrawPoints(C,D)
  \tkzLabelPoints(A,B,C,D)
\end{tikzpicture}
\end{tkzexample}

\section{Lines style}

You have several possibilities to change the style of a line. You can modify the style of a line with \tkzcname{tkzSetUpLine} or directly modify the style of the lines with |\tikzset{line style/.style = ... }|

Reminder about \tkzname{line width} : There are a number of predefined styles that provide more “natural” ways of setting the line width. You can also redefine these styles.


\medskip
\begin{tabular}{cc}
predefined style & value of \tkzname{line width} \cr
\midrule
ultra thin    &  0.1 pt \cr
very thin     &  0.2 pt \cr
thin          &  0.4 pt \cr
semithick     &  0.6 pt \cr
thick         &  0.8 pt \cr
very thick    &  1.2 pt \cr
ultra thick   &  1.6 pt \cr
\midrule
\end{tabular}


\subsection{Use of \tkzcname{tkzSetUpLine}} \label{tkzsetupline}
It is a macro that allows you to define the style of all the lines.

\begin{NewMacroBox}{tkzSetUpLine}{\oarg{local options}}%
\begin{tabular}{lll}%
options &  default & definition                 \\
\midrule
\TOline{color}{black}{colour of the construction lines}
\TOline{line width}{0.4pt}{thickness of the construction lines}
\TOline{style}{solid}{style of construction lines}
\TOline{add}{.2 and .2}{changing the length of a line segment}
\end{tabular}
\end{NewMacroBox}

\subsubsection{Change line width}
\begin{tkzexample}[latex=8cm,small]
\begin{tikzpicture}[scale=.75]
\tkzSetUpLine[line width=1pt]
\begin{scope}[rotate=-90]
    \tkzDefPoints{0/6/A,10/0/B,10/6/C}
    \tkzDefPointBy[projection = onto B--A](C)
    \tkzGetPoint{H}
    \tkzMarkRightAngle[size=.4,
                       fill=teal!20](B,C,A)
    \tkzMarkRightAngle[size=.4,
                       fill=orange!20](B,H,C)
    \tkzDrawPolygon(A,B,C)
    \tkzDrawSegment[new](C,H)
\end{scope}
 \tkzLabelSegment[below](C,B){$a$}
 \tkzLabelSegment[right](A,C){$b$}
 \tkzLabelSegment[left](A,B){$c$}
 \tkzLabelSegment[color=red](C,H){$h$}
 \tkzDrawPoints(A,B,C)
 \tkzLabelPoints[above left](H)
 \tkzLabelPoints(B,C)
 \tkzLabelPoints[above](A)
\end{tikzpicture}
\end{tkzexample}

\subsubsection{Change style of line}

\begin{tkzexample}[latex=7cm,small]
\begin{tikzpicture}[scale=.5]
\tikzset{line style/.style = {color = gray,
                             style=dashed}}
\tkzDefPoints{1/0/A,4/0/B,1/1/C,5/1/D}
\tkzDefPoints{1/2/E,6/2/F,0/4/A',3/4/B'}
\tkzCalcLength(C,D)
\tkzGetLength{rCD}
\tkzCalcLength(E,F)
\tkzGetLength{rEF}
\tkzInterCC[R](A',\rCD)(B',\rEF)
\tkzGetPoints{I}{J}
\tkzDrawLine(A',B')
\tkzCompass(A',B')
\tkzDrawSegments(A,B C,D E,F)
\tkzDefCircle[R](A',\rCD) \tkzGetPoint{a'}
\tkzDefCircle[R](B',\rEF)\tkzGetPoint{b'}
\tkzDrawCircles(A',a' B',b')
\begin{scope}
  \tkzSetUpLine[color=red]
  \tkzDrawSegments(A',I B',I)
\end{scope}
\tkzDrawPoints(A,B,C,D,E,F,A',B',I,J)
\tkzLabelPoints(A,B,C,D,E,F,A',B',I,J)
\end{tikzpicture}
\end{tkzexample}

\subsubsection{Example 3: extend lines}
\begin{tkzexample}[latex=7cm,small]
  \begin{tikzpicture}[scale=.75]
  \tkzSetUpLine[add=.5 and .5]
  \tkzDefPoints{0/0/A,4/0/B,1/3/C}
  \tkzDrawLines(A,B B,C A,C)
  \tkzDrawPolygon[red,thick](A,B,C)
  \tkzSetUpPoint[size=4,circle,color=red,fill=red!20]
  \tkzDrawPoints(A,B,C)
  \end{tikzpicture}
\end{tkzexample}

\section{Arc style}

\subsection{The macro \tkzcname{tkzSetUpArc}}
\begin{NewMacroBox}{tkzSetUpArc}{\oarg{local options}}%
\begin{tabular}{lll}%
options &  default & definition                 \\
\midrule
\TOline{color}{black}{colour of the  lines}
\TOline{line width}{0.4pt}{thickness of the lines}
\TOline{style}{solid}{style of construction lines}
\end{tabular}
\end{NewMacroBox}

\subsubsection{Use of \tkzcname{tkzSetUpArc}}
\begin{tkzexample}[latex=7cm,small]
\begin{tikzpicture}
\def\r{3} \def\angle{200}
\tkzSetUpArc[delta=10,color=purple,line width=.2pt]
\tkzSetUpLabel[font=\scriptsize,red]
\tkzDefPoint(0,0){O}
\tkzDefPoint(\angle:\r){A}
\tkzInterCC(O,A)(A,O) \tkzGetPoints{C'}{C}
\tkzInterCC(O,A)(C,O) \tkzGetPoints{D'}{D}
\tkzInterCC(O,A)(D,O) \tkzGetPoints{X'}{X}
\tkzDrawCircle(O,A)
\tkzDrawArc(A,C')(C)
\tkzDrawArc(C,O)(D)
\tkzDrawArc(D,O)(X)
\tkzDrawLine[add=.1 and .1](A,X)
\tkzDrawPoints(O,A)
\tkzSetUpPoint[size=3,color=purple,fill=purple!10]
\tkzDrawPoints(C,C',D,X)
\tkzLabelPoints[below left](O,A)
\tkzLabelPoints[below](C')
\tkzLabelPoints[below right](X)
\tkzLabelPoints[above](C,D)
\end{tikzpicture}
\end{tkzexample}

\section{Compass style, configuration macro \tkzcname{tkzSetUpCompass}}
The following macro will help to understand the construction of a figure by showing the compass traces necessary to obtain certain points. 

\subsection{The macro \tkzcname{tkzSetUpCompass} }
\begin{NewMacroBox}{tkzSetUpCompass}{\oarg{local options}}%
\begin{tabular}{lll}%
options &  default & definition                 \\
\midrule
\TOline{color}{black}{colour of the construction lines}
\TOline{line width}{0.4pt}{thickness of the construction lines}
\TOline{style}{solid}{style of  lines : solid, dashed,dotted,...}
\TOline{delta}{0}{changes the length of the arc }
\end{tabular}
\end{NewMacroBox}

\subsubsection{Use of \tkzcname{tkzSetUpCompass}}

\begin{tkzexample}[latex=7cm,small]
\begin{tikzpicture}
  \tkzSetUpCompass[color=red,delta=15]
  \tkzDefPoint(1,1){A}
  \tkzDefPoint(6,1){B}
  \tkzInterCC[R](A,4)(B,4) \tkzGetPoints{C}{D}
  \tkzCompass(A,C)
  \tkzCompass(B,C)
  \tkzDrawPolygon(A,B,C)
  \tkzDrawPoints(A,B,C)
\end{tikzpicture} 
\end{tkzexample}

\subsubsection{Use of \tkzcname{tkzSetUpCompass} with \tkzcname{tkzShowLine}}

\begin{tkzexample}[latex=7cm,small]
\begin{tikzpicture}[scale=.75]
\tkzSetUpStyle[bisector,size=2,gap=3]{showbi}
\tkzSetUpCompass[color=teal,line width=.3 pt] 
\tkzDefPoints{0/1/A, 8/3/B, 3/6/C} 
\tkzDrawPolygon(A,B,C) 
\tkzDefLine[bisector](B,A,C) \tkzGetPoint{a} 
\tkzDefLine[bisector](C,B,A) \tkzGetPoint{b} 
\tkzShowLine[showbi](B,A,C) 
\tkzShowLine[showbi](C,B,A) 
\tkzInterLL(A,a)(B,b) \tkzGetPoint{I} 
\tkzDefPointBy[projection= onto A--B](I)
\tkzGetPoint{H} 
\tkzDrawCircle[new](I,H) 
\tkzDrawSegments[new](I,H) 
\tkzDrawLines[add=0 and .2,new](A,I B,I)
\end{tikzpicture}
\end{tkzexample} 





\section{Label style}

\subsection{The macro \tkzcname{tkzSetUpLabel} }
The macro \tkzcname{tkzSetUpLabel} is used to define the style of the point labels.
\begin{NewMacroBox}{tkzSetUpStyle}{\oarg{local options}}%
  The options are the same as those of \TIKZ
\end{NewMacroBox}

\subsubsection{Use of  \tkzcname{tkzSetUpLabel}}
\begin{tkzexample}[latex=6cm,small]
\begin{tikzpicture}[scale=.75]
  \tkzSetUpLabel[font=\scriptsize,red]
  \tkzSetUpStyle[line width=1pt,teal]{XY}
  \tkzInit[xmin=-3,xmax=3,ymin=-3,ymax=3]
  \tkzDrawX[noticks,XY]
  \tkzDrawY[noticks,XY]
  \tkzDefPoints{1/0/A,0/1/B,-1/0/C,0/-1/D}
  \tkzDrawPoints[teal,fill=teal!30,size=6](A,...,D)
  \tkzLabelPoint[above right](A){$A(1,0)$}
  \tkzLabelPoint[above right](B){$B(0,1)$}
  \tkzLabelPoint[above left](C){$C(-1,0)$}
  \tkzLabelPoint[below left](D){$D(0,-1)$}
\end{tikzpicture}
\end{tkzexample}


\section{Own style}
You can set  your own style with \tkzcname{tkzSetUpStyle}

\subsection{The macro \tkzcname{tkzSetUpStyle} }
\begin{NewMacroBox}{tkzSetUpStyle}{\oarg{local options}}%
  The options are the same as those of \TIKZ
\end{NewMacroBox}

\subsubsection{Use of \tkzcname{tkzSetUpStyle}}
\begin{tkzexample}[latex=2cm,small]
\begin{tikzpicture}
  \tkzSetUpStyle[color=blue!20!black,fill=blue!20]{mystyle}
  \tkzDefPoint(0,0){O}
  \tkzDefPoint(0,1){A}
  \tkzDrawPoints(O) % general style
  \tkzDrawPoints[mystyle,size=4](A) % my style
  \tkzLabelPoints(O,A)
\end{tikzpicture}
\end{tkzexample}

\section{How to use \tkzname{arrows}}

In some countries, arrows are used to indicate the parallelism of lines,
to represent half-lines or the sides of an angle (rays).

Here are some examples of how to place these arrows.
\tkzname{ tkz-euclide} loads a library called \tkzname{arrows.meta}.
 
|\usetikzlibrary{arrows.meta}|

This  library is used to produce different styles of arrow heads. The next examples use some of them. 

\subsection{Arrows at endpoints on segment, ray or line}
\tkzname{Stealth}, \tkzname{Triangle}, \tkzname{To}, \tkzname{Latex} and \dots  which can be combined with \tkzname{reversed}. That's easy to place an arrow at one or two endpoints.

\begin{enumerate}
\item \tkzname{Triangle} and \tkzname{Ray}
 \begin{tkzexample}[latex=6cm,small]
    \begin{tikzpicture}
      \tkzDefPoints{0/0/A,4/0/B}
      \tkzDrawSegment[-Triangle](A,B)
    \end{tikzpicture}
  \end{tkzexample}
\item \tkzname{Stealth} and \tkzname{Segment}
  \begin{tkzexample}[latex=6cm,small]
    \begin{tikzpicture}
      \tkzDefPoints{0/0/A,4/0/B}
      \tkzDrawSegment[Stealth-Stealth](A,B)
    \end{tikzpicture}
  \end{tkzexample}
\item \tkzname{Latex}   and \tkzname{Line}
  \begin{tkzexample}[latex=6cm,small]
    \begin{tikzpicture}
      \tkzDefPoints{0/0/A,4/0/B}
      \tkzDrawLine[red,Latex-Latex](A,B)
      \tkzDrawPoints(A,B)
    \end{tikzpicture}
  \end{tkzexample}
\item \tkzname{To} and \tkzname{Segment}
  \begin{tkzexample}[latex=6cm,small]
    \begin{tikzpicture}
      \tkzDefPoints{0/0/A,4/0/B}
      \tkzDrawSegment[To-To](A,B)
    \end{tikzpicture}
  \end{tkzexample}
\item \tkzname{Latex}  and \tkzname{Segment}
  \begin{tkzexample}[latex=6cm,small]
    \begin{tikzpicture}
      \tkzDefPoints{0/0/A,4/0/B}
      \tkzDrawSegment[Latex-Latex](A,B)
    \end{tikzpicture}
  \end{tkzexample}
\item \tkzname{Latex}  and \tkzname{Ray}
  \begin{tkzexample}[latex=6cm,small]
    \begin{tikzpicture}
      \tkzDefPoints{0/0/A,4/0/B}
      \tkzDrawSegment[Latex-](A,B)
    \end{tikzpicture}
  \end{tkzexample}
\item \tkzname{Latex}  and \tkzname{Several rays}
  \begin{tkzexample}[latex=6cm,small]
\begin{tikzpicture}
 \tkzDefPoints{0/0/A,4/0/B,5/-2/C}
 \tkzDrawSegments[-Latex](A,B A,C)
\end{tikzpicture}
\end{tkzexample}
\end{enumerate}

\subsubsection{Scaling an arrow head}

\begin{tkzexample}[latex=6cm,small]
\begin{tikzpicture}
 \tkzDefPoints{0/0/A,4/0/B}
 \tkzDrawSegment[{Latex[scale=2]}-{Latex[scale=2]}](A,B)
\end{tikzpicture}
\end{tkzexample}

\subsubsection{Using vector style}
|\tikzset{vector style/.style={>=Latex,->}}|

You can redefine this style.
\begin{tkzexample}[latex=6cm,small]
\begin{tikzpicture}
 \tkzDefPoints{0/0/A,4/0/B}
\tkzDrawSegment[vector style](A,B)
\end{tikzpicture}
\end{tkzexample}
  
\subsection{Arrows on  middle point of a line segment}

Arrows on lines are used to indicate that those lines are parallel. It depends on the country, in France we prefer to indicate outside the figure that $(A,B) \parallel (D,C)$. The code is an adaptation of an answer by \tkzname{muzimuzhi Z} on the site \href{https://tex.stackexchange.com/questions/632596/how-to-manage-argument-pattern-keys-and-subways}{tex.stackexchange.com}.

\medskip
 Syntax: \\

 \begin{itemize}
\item |tkz arrow| (\tkzname{Latex} by default)
\item |tkz arrow=<arrow end tip>|
\item |tkz arrow=<arrow end tip> at <pos> (<pos> = .5 by default)|
\item |tkz arrow={<arrow end tip>[<arrow options>] at <pos>}| option possible \tkzname{scale}
 \end{itemize}

Example usages: \\

|\tkzDrawSegment[tkz arrow=Stealth] (A,B)|\\
|\tkzDrawSegment[tkz arrow={To[scale=3] at .4}](A,B)|\\
|\tkzDrawSegment[tkz arrow={Latex[scale=5,blue] at .6}](A,B)|

\subsubsection{In a parallelogram}
\begin{tkzexample}[latex=7cm,small]
\begin{tikzpicture}
 \tkzDefPoints{0/0/A,3/0/B,4/2/C} 
 \tkzDefParallelogram(A,B,C) 
 \tkzGetPoint{D}
 \tkzDrawSegments[tkz arrow](A,B D,C)
 \tkzDrawSegments(B,C D,A)
 \tkzLabelPoints(A,B) 
 \tkzLabelPoints[above right](C,D)
 \tkzDrawPoints(A,...,D)
\end{tikzpicture}
\end{tkzexample}

\subsubsection{A line parallel to another one}
\begin{tkzexample}[latex=7cm,small]
\begin{tikzpicture}
 \tkzDefPoints{0/0/A,3/0/B,1/2/C} 
 \tkzDefPointWith[colinear= at C](A,B) 
 \tkzGetPoint{D}
 \tkzDrawSegments[tkz arrow=Triangle](A,B C,D)
 \tkzLabelPoints(A,B,C) 
 \tkzDrawPoints(A,...,C)
\end{tikzpicture}
\end{tkzexample}

\subsubsection{Arrow on a circle}
It is possible to place an arrow on the first quarter of a circle. A rotation allows you to move the arrow.
\begin{tkzexample}[latex=7cm,small]
\begin{tikzpicture}
\tkzDefPoints{0/0/A,3/0/B} 
\begin{scope}[rotate=150]
 \tkzDrawCircle[tkz arrow={Latex[scale=2,red]}](A,B)
\end{scope}
\end{tikzpicture}
\end{tkzexample}

\subsection{Arrows on  all segments of a polygon}
Some users of my package have asked me to be able to place an arrow on each side of a polygon. I used a style proposed by Paul Gaborit on the site 
\href{https://tex.stackexchange.com/questions/3161/tikz-how-to-draw-an-arrow-in-the-middle-of-the-line}{tex.stackexchange.com}.

|\tikzset{tkz arrows/.style=|\\
|{postaction={on each path={tkz arrow={Latex[scale=2,color=black]}}}}}|   

You can change this style. With \tkzname{tkz arrows} you can an arrow on each segment of a polygon

\subsubsection{Arrow on each segment with \tkzname{tkz arrows} }
\begin{tkzexample}[latex=7cm,small]
\begin{tikzpicture}
 \tkzDefPoints{0/0/A,3/0/B}  
 \tkzDefSquare(A,B) \tkzGetPoints{C}{D}
 \tkzDrawPolygon[tkz arrows](A,...,D)
\end{tikzpicture}
\end{tkzexample}

\subsubsection{Using \tkzname{tkz arrows} with a circle}
\begin{tkzexample}[latex=7cm,small]
\begin{tikzpicture}
 \tkzDefPoints{0/0/A,3/0/B} 
 \tkzDrawCircle[tkz arrows](A,B)
\end{tikzpicture}
\end{tkzexample}

\endinput

\part{Examples}
\section{Different authors}

\subsection{Code from Andrew Swan}

\begin{tkzexample}[latex=7cm]
\begin{tikzpicture}[scale=1.25]
\def\radius{4}
\def\angle{40}
\pgfmathsetmacro{\htan}{tan(\angle)}
\tkzDefPoint(0,0){A} \tkzDefPoint(0,\radius){F}
\tkzDefPoint(\radius,0){B}
\tkzDefPointBy[rotation= center A angle \angle](B)
\tkzGetPoint{C}
\tkzDefLine[perpendicular=through B,K=1](A,B)
\tkzGetPoint{b}
\tkzInterLL(A,C)(B,b) \tkzGetPoint{D}
\tkzDefLine[perpendicular=through C,K=-1](A,B)
\tkzGetPoint{c}
\tkzInterLL(C,c)(A,B) \tkzGetPoint{E}
\tkzDrawSector[fill=blue,opacity=0.1](A,B)(C)
\tkzDrawArc[thin](A,B)(F)
\tkzMarkAngle(B,A,C)
\tkzLabelAngle[pos=0.8](B,A,C){$x$}
\tkzDrawPolygon(A,B,D)
\tkzDrawSegments(C,B)
\tkzDrawSegments[dashed,thin](C,E)
\tkzLabelPoints[below left](A)
\tkzLabelPoints[below right](B)
\tkzLabelPoints[above](C)
\tkzLabelPoints[above right](D)
\begin{scope}[/pgf/decoration/raise=5pt]
\draw [decorate,decoration={brace,mirror,
   amplitude=10pt},xshift=0pt,yshift=-4pt]
(A) -- (B) node [black,midway,yshift=-20pt]
{\footnotesize $1$};
\draw [decorate,decoration={brace,amplitude=10pt},
       xshift=4pt,yshift=0pt]
(D) -- (B) node [black,midway,xshift=27pt]
{\footnotesize $\tan x$};
\draw [decorate,decoration={brace,amplitude=10pt},
       xshift=4pt,yshift=0pt]
(E) -- (C) node [black,midway,xshift=-27pt]
{\footnotesize $\sin x$};
\end{scope}
\end{tikzpicture}
\end{tkzexample}


\subsection{Example: Dimitris Kapeta}

You need in this example to use \tkzname{mkpos=.2} with \tkzcname{tkzMarkAngle} because the measure of $ \widehat{CAM}$ is too small.
Another possiblity is to use \tkzcname{tkzFillAngle}.


\begin{tkzexample}[latex=7cm,small]
\begin{tikzpicture}[scale=1]
  \tkzDefPoint(0,0){O}
  \tkzDefPoint(2.5,0){N}
  \tkzDefPoint(-4.2,0.5){M}
  \tkzDefPointBy[rotation=center O angle 30](N)
  \tkzGetPoint{B}
  \tkzDefPointBy[rotation=center O angle -50](N)
  \tkzGetPoint{A}
  \tkzInterLC[common=B](M,B)(O,B) \tkzGetFirstPoint{C}
  \tkzInterLC[common=A](M,A)(O,A) \tkzGetFirstPoint{A'}
  \tkzMarkAngle[mkpos=.2, size=0.5](A,C,B)
  \tkzMarkAngle[mkpos=.2, size=0.5](A,M,C)
  \tkzDrawSegments(A,C M,A M,B A,B)
  \tkzDrawCircle(O,N)
  \tkzLabelCircle[above left](O,N)(120){%
                 $\mathcal{C}$}
  \begin{scope}[xfp]
    \tkzMarkAngle[mkpos=.2, size=1.2](C,A,M)
  \end{scope}

  \tkzDrawPoints(O, A, B, M, B, C)
  \tkzLabelPoints[right](O,A,B)
  \tkzLabelPoints[above left](M,C)
  \tkzLabelPoint[below left](A'){$A'$}
\end{tikzpicture}
\end{tkzexample}


\subsection{Example :  John Kitzmiller }
Prove that $\dfrac{AC}{CE}=\dfrac{BD}{DF}$.

Another interesting example from John, you can see how to use some extra options like\\\ \tkzname{decoration} and \tkzname{postaction}  from \TIKZ\ with \tkzname{tkz-euclide}.

\begin{tkzexample}[vbox,small]
\begin{tikzpicture}[scale=1.5,decoration={markings,
  mark=at position 3cm with {\arrow[scale=2]{>}}}]
  \tkzDefPoints{0/0/E, 6/0/F, 0/1.8/P, 6/1.8/Q, 0/3/R, 6/3/S}
  \tkzDrawLines[postaction={decorate}](E,F P,Q R,S)
  \tkzDefPoints{3.5/3/A, 5/3/B}
  \tkzDrawSegments(E,A F,B)
  \tkzInterLL(E,A)(P,Q) \tkzGetPoint{C}
  \tkzInterLL(B,F)(P,Q) \tkzGetPoint{D}
  \tkzLabelPoints[above right](A,B)
  \tkzLabelPoints[below](E,F)
  \tkzLabelPoints[above left](C)
  \tkzDrawSegments[style=dashed](A,F)
  \tkzInterLL(A,F)(P,Q) \tkzGetPoint{G}
  \tkzLabelPoints[above right](D,G)
  \tkzDrawSegments[color=teal, line width=3pt, opacity=0.4](A,C A,G)
  \tkzDrawSegments[color=magenta, line width=3pt, opacity=0.4](C,E G,F)
  \tkzDrawSegments[color=teal, line width=3pt, opacity=0.4](B,D)
  \tkzDrawSegments[color=magenta, line width=3pt, opacity=0.4](D,F)
\end{tikzpicture}
\end{tkzexample}


\subsection{Example 1: from Indonesia}

\begin{tkzexample}[vbox,small]
\begin{tikzpicture}[scale=3]
   \tkzDefPoints{0/0/A,2/0/B}
   \tkzDefSquare(A,B) \tkzGetPoints{C}{D}
   \tkzDefPointBy[rotation=center D angle 45](C)\tkzGetPoint{G}
   \tkzDefSquare(G,D)\tkzGetPoints{E}{F}
   \tkzInterLL(B,C)(E,F)\tkzGetPoint{H}
   \tkzFillPolygon[gray!10](D,E,H,C,D)
   \tkzDrawPolygon(A,...,D)\tkzDrawPolygon(D,...,G)
   \tkzDrawSegment(B,E)
   \tkzMarkSegments[mark=|,size=3pt,color=gray](A,B B,C C,D D,A E,F F,G G,D D,E)
   \tkzMarkSegments[mark=||,size=3pt,color=gray](B,E E,H)
   \tkzLabelPoints[left](A,D)
   \tkzLabelPoints[right](B,C,F,H)
   \tkzLabelPoints[above](G)\tkzLabelPoints[below](E)
   \tkzMarkRightAngles(D,A,B D,G,F)
\end{tikzpicture}
\end{tkzexample}

\subsection{Example 2: from Indonesia}
\begin{tkzexample}[vbox,overhang,small]
  \begin{tikzpicture}[pol/.style={fill=brown!40,opacity=.2},
      seg/.style={tkzdotted,color=gray}, hidden pt/.style={fill=gray!40},
       mra/.style={color=gray!70,tkzdotted,/tkzrightangle/size=.2},scale=2]
  \tkzDefPoints{0/0/A,2.5/0/B,1.33/0.75/D,0/2.5/E,2.5/2.5/F}
  \tkzDefLine[parallel=through D](A,B)  \tkzGetPoint{I1}
  \tkzDefLine[parallel=through B](A,D)  \tkzGetPoint{I2}
  \tkzInterLL(D,I1)(B,I2)               \tkzGetPoint{C}
  \tkzDefLine[parallel=through E](A,D)  \tkzGetPoint{I3}
  \tkzDefLine[parallel=through D](A,E)  \tkzGetPoint{I4}
  \tkzInterLL(E,I3)(D,I4)               \tkzGetPoint{H}
  \tkzDefLine[parallel=through F](E,H)  \tkzGetPoint{I5}
  \tkzDefLine[parallel=through H](E,F)  \tkzGetPoint{I6}
  \tkzInterLL(F,I5)(H,I6)               \tkzGetPoint{G}
  \tkzDefMidPoint(G,H) \tkzGetPoint{P}  \tkzDefMidPoint(G,C) \tkzGetPoint{Q}
  \tkzDefMidPoint(B,C) \tkzGetPoint{R}  \tkzDefMidPoint(A,B) \tkzGetPoint{S}
  \tkzDefMidPoint(A,E) \tkzGetPoint{T}  \tkzDefMidPoint(E,H) \tkzGetPoint{U}
  \tkzDefMidPoint(A,D) \tkzGetPoint{M}  \tkzDefMidPoint(D,C) \tkzGetPoint{N}
  \tkzInterLL(B,D)(S,R)\tkzGetPoint{L} \tkzInterLL(H,F)(U,P) \tkzGetPoint{K}
  \tkzDefLine[parallel=through K](D,H)  \tkzGetPoint{I7}
  \tkzInterLL(K,I7)(B,D)                \tkzGetPoint{O}
  \tkzFillPolygon[pol](P,Q,R,S,T,U)
  \tkzDrawSegments[seg](K,O K,L P,Q R,S T,U C,D H,D A,D M,N B,D)
  \tkzDrawSegments(E,H B,C G,F G,H G,C Q,R S,T U,P H,F)
  \tkzDrawPolygon(A,B,F,E)
  \tkzDrawPoints(A,B,C,E,F,G,H,P,Q,R,S,T,U,K) \tkzDrawPoints[hidden pt](M,N,O,D)
  \tkzMarkRightAngle[mra](L,O,K)
  \tkzMarkSegments[mark=|,size=1pt,thick,color=gray](A,S B,S B,R C,R
                    Q,C Q,G G,P H,P E,U H,U E,T A,T)
  \tkzLabelAngle[pos=.3](K,L,O){$\alpha$}
  \tkzLabelPoints[below](O,A,S,B)    \tkzLabelPoints[above](H,P,G)
  \tkzLabelPoints[left](T,E)         \tkzLabelPoints[right](C,Q)
  \tkzLabelPoints[above left](U,D,M) \tkzLabelPoints[above right](L,N)
  \tkzLabelPoints[below right](F,R)  \tkzLabelPoints[below left](K)
\end{tikzpicture}
\end{tkzexample}
\newpage

\subsection{Illustration of  the Morley theorem by Nicolas François}
\begin{tkzexample}[vbox,small]
  \begin{tikzpicture}
    \tkzInit[ymin=-3,ymax=5,xmin=-5,xmax=7]
    \tkzClip
    \tkzDefPoints{-2.5/-2/A,2/4/B,5/-1/C}
    \tkzFindAngle(C,A,B) \tkzGetAngle{anglea}
    \tkzDefPointBy[rotation=center A angle 1*\anglea/3](C) \tkzGetPoint{TA1}
    \tkzDefPointBy[rotation=center A angle 2*\anglea/3](C) \tkzGetPoint{TA2}
    \tkzFindAngle(A,B,C) \tkzGetAngle{angleb}
    \tkzDefPointBy[rotation=center B angle 1*\angleb/3](A) \tkzGetPoint{TB1}
    \tkzDefPointBy[rotation=center B angle 2*\angleb/3](A) \tkzGetPoint{TB2}
    \tkzFindAngle(B,C,A) \tkzGetAngle{anglec}
    \tkzDefPointBy[rotation=center C angle 1*\anglec/3](B) \tkzGetPoint{TC1}
    \tkzDefPointBy[rotation=center C angle 2*\anglec/3](B) \tkzGetPoint{TC2}
    \tkzInterLL(A,TA1)(B,TB2) \tkzGetPoint{U1}
    \tkzInterLL(A,TA2)(B,TB1) \tkzGetPoint{V1}
    \tkzInterLL(B,TB1)(C,TC2) \tkzGetPoint{U2}
    \tkzInterLL(B,TB2)(C,TC1) \tkzGetPoint{V2}
    \tkzInterLL(C,TC1)(A,TA2) \tkzGetPoint{U3}
    \tkzInterLL(C,TC2)(A,TA1) \tkzGetPoint{V3}
    \tkzDrawPolygons(A,B,C U1,U2,U3 V1,V2,V3)
    \tkzDrawLines[add=2 and 2,very thin,dashed](A,TA1 B,TB1 C,TC1 A,TA2 B,TB2 C,TC2)
    \tkzDrawPoints(U1,U2,U3,V1,V2,V3)
    \tkzLabelPoint[left](V1){$s_a$} \tkzLabelPoint[right](V2){$s_b$}
    \tkzLabelPoint[below](V3){$s_c$} \tkzLabelPoint[above left](A){$A$}
    \tkzLabelPoints[above right](B,C) \tkzLabelPoint(U1){$t_a$}
    \tkzLabelPoint[below left](U2){$t_b$} \tkzLabelPoint[above](U3){$t_c$}
  \end{tikzpicture}
  \end{tkzexample}

\newpage
\subsection{Gou gu theorem / Pythagorean Theorem by  Zhao Shuang}
\begin{tikzpicture}
\node [mybox,title={Gou gu theorem / Pythagorean Theorem by  Zhao Shuang}] (box){%
\begin{minipage}{0.90\textwidth}
  {\emph{Pythagoras was not the first person who discovered this theorem around the world. Ancient China discovered this theorem much earlier than him. So there is another name for the Pythagorean theorem in China, the Gou-Gu theorem.
Zhao Shuang was an ancient Chinese mathematician. He rediscovered the “Gou gu therorem”, which is actually the Chinese version of the “Pythagorean theorem”. Zhao Shuang used a method called the “cutting and compensation principle”, he  created a picture of “Pythagorean Round Square”
Below the figure used to illustrate the proof of the “Gou gu theorem.”  (code from Nan Geng)
}} 
\end{minipage}
};
\end{tikzpicture}
  
\begin{tkzexample}[latex=7cm,small]
\begin{tikzpicture}[scale=.8]
  \tkzDefPoint(0,0){A} \tkzDefPoint(4,0){A'}
  \tkzInterCC[R](A, 5)(A', 3)
  \tkzGetSecondPoint{B}
  \tkzDefSquare(A,B)   \tkzGetPoints{C}{D}
  \tkzCalcLength(A,A') \tkzGetLength{lA}
  \tkzCalcLength(A',B) \tkzGetLength{lB}
  \pgfmathparse{\lA-\lB}
  \tkzInterLC[R](A,A')(A',\pgfmathresult)
  \tkzGetFirstPoint{D'}
  \tkzDefSquare(D',A')\tkzGetPoints{B'}{C'}
  \tkzDefLine[orthogonal=through D](D,D') 
   \tkzGetPoint{d}
  \tkzDefLine[orthogonal=through A](A,A')
   \tkzGetPoint{a}
  \tkzDefLine[orthogonal=through C](C,C') 
   \tkzGetPoint{c}  
  \tkzInterLL(D,d)(C,c) \tkzGetPoint{E}
  \tkzInterLL(D,d)(A,a) \tkzGetPoint{F}
  \tkzDefSquare(E,F)\tkzGetPoints{G}{H}
  \tkzDrawPolygons[fill=teal!10](A,B,A' B,C,B'
     C,D,C' A,D',D)  
  \tkzDrawPolygons(A,B,C,D E,F,G,H)
  \tkzDrawPolygon[fill=green!10](A',B',C',D')
  \tkzDrawSegment[dim={$a$,-10pt,}](D,C')
  \tkzDrawSegment[dim={$b$,-10pt,}](C,C')
  \tkzDrawSegment[dim={$c$,-10pt,}](C,D)
  \tkzDrawPoints[size=2](A,B,C,D,A',B',C',D')
  \tkzLabelPoints[left](A)
  \tkzLabelPoints[below](B)
  \tkzLabelPoints[right](C)
  \tkzLabelPoints[above](D)
  \tkzLabelPoints[right](A')
  \tkzLabelPoints[below right](B')
  \tkzLabelPoints[below left](C') 
  \tkzLabelPoints[below](D')
 \end{tikzpicture}
\end{tkzexample}

\newpage
\subsection{Reuleaux-Triangle}
\begin{tikzpicture}
\node [mybox,title={Reuleaux-triangle by  Stefan Kottwitz}] (box){%
\begin{minipage}{0.90\textwidth}
  {\emph{A well-known classic field of mathematics is geometry.
You may know Euclidean geometry from school, with constructions
by compass and ruler. Math teachers may be very interested in
drawing geometry constructions and explanations. Underlying
constructions can help us with general drawings where we would
need intersections and tangents of lines and circles, even if
it does not look like geometry.
So, here, we will remember school geometry drawings.
We will use the tkz-euclide package, which works on top of TikZ.
We will construct an equilateral triangle.
Then we extend it to get a Reuleaux triangle, and add annotations.
The code is fully explained in the LaTeX Cookbook, Chapter 10,
Advanced Mathematics, Drawing geometry pictures.
 Stefan Kottwitz
}} 
\end{minipage}
};
\end{tikzpicture}

\begin{tkzexample}[vbox,small]
  \begin{tikzpicture}
    \tkzDefPoint(0,0){A} \tkzDefPoint(4,1){B}
    \tkzInterCC(A,B)(B,A) \tkzGetPoints{C}{D}
    \tkzInterLC(A,B)(B,A) \tkzGetPoints{F}{E}
    \tkzDrawCircles[dashed](A,B B,A)
    \tkzDrawPolygons(A,B,C A,E,D)
    \tkzCompasss[color=red, very thick](A,C B,C A,D B,D)
    \begin{scope}
      \tkzSetUpArc[thick,delta=0]
      \tkzDrawArc[fill=blue!10](A,B)(C)
      \tkzDrawArc[fill=blue!10](B,C)(A)
      \tkzDrawArc[fill=blue!10](C,A)(B)
    \end{scope}
    \tkzMarkAngles(D,A,E A,E,D)
    \tkzFillAngles[fill=yellow,opacity=0.5](D,A,E A,E,D) 
    \tkzMarkRightAngle[size=0.65,fill=red!20,opacity=0.2](A,D,E) 
    \tkzLabelAngle[pos=0.7](D,A,E){$\alpha$}
    \tkzLabelAngle[pos=0.8](A,E,D){$\beta$}
    \tkzLabelAngle[pos=0.5,xshift=-1.4mm](A,D,D){$90^\circ$}
    \begin{scope}[font=\small]
      \tkzLabelSegment[below=0.6cm,align=center](A,B){Reuleaux\\triangle}
      \tkzLabelSegment[above right,sloped](A,E){hypotenuse}
      \tkzLabelSegment[below,sloped](D,E){opposite}
      \tkzLabelSegment[below,sloped](A,D){adjacent}
      \tkzLabelSegment[below right=4cm](A,E){Thales circle}
    \end{scope}
    \tkzLabelPoints[below left](A)
    \tkzLabelPoints(B,D)
    \tkzLabelPoint[above](C){$C$}
    \tkzLabelPoints(E)
    \tkzDrawPoints(A,...,E)

  \end{tikzpicture}
\end{tkzexample}




\endinput
\section{Some interesting examples}

\subsection{Square root of the integers}
\begin{tikzpicture}
\node [mybox,title={Square root of the integers}] (box){%
\begin{minipage}{0.90\textwidth}
  {\emph{How to get $1$, $\sqrt{2}$, $\sqrt{3}$ with a rule and a compass.
}} 
\end{minipage}
};
\end{tikzpicture}

\begin{tkzexample}[latex=7cm,small]
\begin{tikzpicture}
  \tkzDefPoint(0,0){O}
  \tkzDefPoint(1,0){a0}
   \tkzDrawSegment(O,a0)
  \foreach \i [count=\j] in {0,...,16}{%
    \tkzDefPointWith[orthogonal normed](a\i,O)
    \tkzGetPoint{a\j}
       \pgfmathsetmacro{\c}{5*\i} 
    \tkzDrawPolySeg[fill=teal!\c](a\i,a\j,O)}
 \end{tikzpicture}
\end{tkzexample}

\subsection{About right triangle}
\begin{tikzpicture}
\node [mybox,title={About right triangle}] (box){%
\begin{minipage}{0.90\textwidth}
  {\emph{We have a segment $[AB]$ and we want to determine a point $C$ such that $AC=8$~cm    and $ABC$ is a right triangle in $B$.
}} 
\end{minipage}
};
\end{tikzpicture}

\begin{tkzexample}[latex=7cm,small]
\begin{tikzpicture}[scale=.5]
  \tkzDefPoint["$A$" left](2,1){A}
  \tkzDefPoint["$B$" right](6,4){B}
  \tkzDefPointWith[orthogonal,K=-1](B,A)
  \tkzDrawLine[add = .5 and .5](B,tkzPointResult)
  \tkzInterLC[R](B,tkzPointResult)(A,8)
  \tkzGetPoints{J}{C}
  \tkzDrawSegment(A,B)
  \tkzDrawPoints(A,B,C)
  \tkzCompass(A,C)
  \tkzMarkRightAngle(A,B,C)
  \tkzDrawLine[color=gray,style=dashed](A,C)
  \tkzLabelPoint[above](C){$C$}
\end{tikzpicture}
\end{tkzexample}

\subsection{Archimedes}
\begin{tikzpicture}
\node [mybox,title={Archimedes}] (box){%
\begin{minipage}{0.90\textwidth}
  {\emph{This is an ancient problem   proved by the great Greek mathematician Archimedes .
The figure below shows a semicircle, with diameter $AB$. A tangent line is drawn and  touches the semicircle at $B$.   An other tangent line at a point, $C$, on the semicircle is drawn. We project the point $C$ on the line segment $[AB]$  on a point $D$. The two tangent lines intersect at the point $T$. Prove that the line $(AT)$ bisects $(CD)$
}} 
\end{minipage}
};
\end{tikzpicture}

\begin{tkzexample}[]
\begin{tikzpicture}[scale=1]
  \tkzDefPoint(0,0){A}\tkzDefPoint(6,0){D}
  \tkzDefPoint(8,0){B}\tkzDefPoint(4,0){I}
  \tkzDefLine[orthogonal=through D](A,D)
  \tkzInterLC[R](D,tkzPointResult)(I,4) \tkzGetSecondPoint{C}
  \tkzDefLine[orthogonal=through C](I,C)    \tkzGetPoint{c}
  \tkzDefLine[orthogonal=through B](A,B)    \tkzGetPoint{b}
  \tkzInterLL(C,c)(B,b) \tkzGetPoint{T}
  \tkzInterLL(A,T)(C,D) \tkzGetPoint{P}
  \tkzDrawArc(I,B)(A)
  \tkzDrawSegments(A,B A,T C,D I,C) \tkzDrawSegment[new](I,C)
  \tkzDrawLine[add = 1 and 0](C,T)   \tkzDrawLine[add = 0 and 1](B,T)
  \tkzMarkRightAngle(I,C,T)
  \tkzDrawPoints(A,B,I,D,C,T)
  \tkzLabelPoints(A,B,I,D)  \tkzLabelPoints[above right](C,T)
  \tkzMarkSegment[pos=.25,mark=s|](C,D) \tkzMarkSegment[pos=.75,mark=s|](C,D)
\end{tikzpicture}
\end{tkzexample}

\newpage
\subsubsection{Square and rectangle of same area; Golden section}

\begin{tikzpicture}
\node [mybox,title={Book II, proposition XI  \_Euclid's Elements\_}] (box){%
\begin{minipage}{0.90\textwidth}
{\emph{To construct Square and rectangle of same area.}
} 
\end{minipage}
};
\end{tikzpicture}% 

\begin{tkzexample}[vbox,small]
\begin{tikzpicture}[scale=.75]
 \tkzDefPoint(0,0){D} \tkzDefPoint(8,0){A}
 \tkzDefSquare(D,A) \tkzGetPoints{B}{C}
 \tkzDefMidPoint(D,A) \tkzGetPoint{E}
 \tkzInterLC(D,A)(E,B)\tkzGetSecondPoint{F}
 \tkzInterLC(A,B)(A,F)\tkzGetSecondPoint{G}    
 \tkzDefSquare(A,F)\tkzGetFirstPoint{H}
 \tkzInterLL(C,D)(H,G)\tkzGetPoint{I} 
 \tkzFillPolygon[teal!10](I,G,B,C)
 \tkzFillPolygon[teal!10](A,F,H,G)
 \tkzDrawArc[angles](E,B)(0,120)
 \tkzDrawSemiCircle(A,F)
 \tkzDrawSegments(A,F E,B H,I F,H)
 \tkzDrawPolygons(A,B,C,D)
 \tkzDrawPoints(A,...,I)
 \tkzLabelPoints[below right](A,E,D,F,I)
 \tkzLabelPoints[above right](C,B,G,H)
\end{tikzpicture}
\end{tkzexample}

\newpage

\subsubsection{Steiner Line and Simson Line}

\begin{tikzpicture}
\node [mybox,title={Steiner Line and Simson Line}] (box){%
\begin{minipage}{0.90\textwidth}
{\emph{Consider the triangle ABC and a point M on its circumcircle. The projections  of M on the sides of the triangle are on a line (Steiner Line),  The three closest points to M on lines AB, AC, and BC are collinear. It's the Simson Line.
}} 
\end{minipage}
};
\end{tikzpicture}%

\begin{tkzexample}[latex=7cm,small]
\begin{tikzpicture}[scale=.75,rotate=-20]
  \tkzDefPoint(0,0){B} 
  \tkzDefPoint(2,4){A} \tkzDefPoint(7,0){C}
  \tkzDefCircle[circum](A,B,C)  
  \tkzGetPoint{O}
  \tkzDrawCircle(O,A) 
  \tkzCalcLength(O,A)  
  \tkzGetLength{rOA} 
  \tkzDefShiftPoint[O](40:\rOA){M}
  \tkzDefShiftPoint[O](60:\rOA){N}  
  \tkzDefTriangleCenter[orthic](A,B,C)
  \tkzGetPoint{H}
  \tkzDefSpcTriangle[orthic,name=H](A,B,C){a,b,c}
  \tkzDefPointsBy[reflection=over A--B](M,N){P,P'}
  \tkzDefPointsBy[reflection=over A--C](M,N){Q,Q'}
  \tkzDefPointsBy[reflection=over C--B](M,N){R,R'}
  \tkzDefMidPoint(M,P)\tkzGetPoint{I}
  \tkzDefMidPoint(M,Q)\tkzGetPoint{J}
  \tkzDefMidPoint(M,R)\tkzGetPoint{K} 
  \tkzDrawSegments[new](P,R M,P M,Q M,R N,P'%
   N,Q' N,R' P',R' I,K)
  \tkzDrawPolygons(A,B,C)
  \tkzDrawPoints(A,B,C,H,M,N,P,Q,R,P',Q',R',I,J,K)
  \tkzLabelPoints(A,B,C,H,M,N,P,Q,R,P',Q',R',I,J,K)
\end{tikzpicture}
\end{tkzexample}

\newpage
\subsection{Lune of Hippocrates}

\begin{tikzpicture}
\node [mybox,title={Lune of Hippocrates}] (box){%
\begin{minipage}{0.90\textwidth}
  { \emph{From wikipedia : In geometry, the lune of Hippocrates, named after Hippocrates of Chios, is a lune bounded by arcs of two circles, the smaller of which has as its diameter a chord spanning a right angle on the larger circle.In the first figure, the area of the lune is equal to the area of the triangle ABC. Hippocrates of Chios (ancient Greek mathematician,)
}} 
\end{minipage}
};
\end{tikzpicture}% 

\begin{tkzexample}[latex=7cm,small]
\begin{tikzpicture}
 \tkzInit[xmin=-2,xmax=5,ymin=-1,ymax=6]
 \tkzClip % allows you to define a bounding box 
   % large enough
  \tkzDefPoint(0,0){A}\tkzDefPoint(4,0){B}
  \tkzDefSquare(A,B) 
  \tkzGetFirstPoint{C} 
  \tkzDrawPolygon[fill=green!5](A,B,C)
   \begin{scope}
     \tkzClipCircle[out](B,A)
     \tkzDefMidPoint(C,A) \tkzGetPoint{M}
     \tkzDrawSemiCircle[fill=teal!5](M,C)
   \end{scope}
   \tkzDrawArc[delta=0](B,C)(A)
\end{tikzpicture}
\end{tkzexample}

\subsection{Lunes of Hasan Ibn al-Haytham}

\begin{tikzpicture}
\node [mybox,title={Lune of Hippocrates}] (box){%
\begin{minipage}{0.90\textwidth}
  { \emph{From wikipedia : the Arab mathematician Hasan Ibn al-Haytham (Latinized name Alhazen) showed that two lunes, formed on the two sides of a right triangle, whose outer boundaries are semicircles and whose inner boundaries are formed by the circumcircle of the triangle, then the areas of these two lunes added together are equal to the area of the triangle. The lunes formed in this way from a right triangle are known as the lunes of Alhazen.
}} 
\end{minipage}};
\end{tikzpicture}% 

\begin{tkzexample}[latex=7cm,small]
\begin{tikzpicture}[scale=.5,rotate=180]
  \tkzInit[xmin=-1,xmax=11,ymin=-4,ymax=7]
  \tkzClip
  \tkzDefPoints{0/0/A,8/0/B}
  \tkzDefTriangle[pythagore,swap](A,B) 
  \tkzGetPoint{C}
  \tkzDrawPolygon[fill=green!5](A,B,C)
  \tkzDefMidPoint(C,A) \tkzGetPoint{I}
  \begin{scope}
    \tkzClipCircle[out](I,A)
    \tkzDefMidPoint(B,A) \tkzGetPoint{x}
    \tkzDrawSemiCircle[fill=teal!5](x,A)
    \tkzDefMidPoint(B,C) \tkzGetPoint{y}
    \tkzDrawSemiCircle[fill=teal!5](y,B)
  \end{scope}
  \tkzSetUpCompass[/tkzcompass/delta=0]
      \tkzDefMidPoint(C,A) \tkzGetPoint{z}
  \tkzDrawSemiCircle(z,A)
\end{tikzpicture}
\end{tkzexample}

\newpage
\subsection{About clipping circles}\label{About clipping circles}
\begin{tikzpicture}
\node [mybox,title={About clipping circles}] (box){%
\begin{minipage}{0.90\textwidth}
  { \emph{The problem is the management of the bounding box. First you have to define a rectangle in which the figure will be inserted. This is done with the first two lines.
}} 
\end{minipage}
};
\end{tikzpicture}% 

\begin{tkzexample}[latex=7cm,small]
\begin{tikzpicture}
  \tkzInit[xmin=0,xmax=6,ymin=0,ymax=6]
  \tkzClip
  \tkzDefPoints{0/0/A, 6/0/B}
  \tkzDefSquare(A,B)      \tkzGetPoints{C}{D}
  \tkzDefMidPoint(A,B)        \tkzGetPoint{M}
  \tkzDefMidPoint(A,D)        \tkzGetPoint{N}
  \tkzDefMidPoint(B,C)        \tkzGetPoint{O}
  \tkzDefMidPoint(C,D)        \tkzGetPoint{P}
 \begin{scope}
  \tkzClipCircle[out](M,B) \tkzClipCircle[out](P,D)
  \tkzFillPolygon[teal!20](M,N,P,O)
 \end{scope}
 \begin{scope}
   \tkzClipCircle[out](N,A) \tkzClipCircle[out](O,C)
   \tkzFillPolygon[teal!20](M,N,P,O)
 \end{scope}
 \begin{scope}
   \tkzClipCircle(P,C) \tkzClipCircle(N,A)      
   \tkzFillPolygon[teal!20](N,P,D)
 \end{scope}
 \begin{scope}
     \tkzClipCircle(O,C) \tkzClipCircle(P,C) 
     \tkzFillPolygon[teal!20](P,C,O)
 \end{scope}
 \begin{scope}
     \tkzClipCircle(M,B)  \tkzClipCircle(O,B)
     \tkzFillPolygon[teal!20](O,B,M)
 \end{scope}
 \begin{scope}
     \tkzClipCircle(N,A) \tkzClipCircle(M,A)  
     \tkzFillPolygon[teal!20](A,M,N)
 \end{scope}
 \tkzDrawSemiCircles(M,B N,A O,C P,D)
 \tkzDrawPolygons(A,B,C,D M,N,P,O)
 \end{tikzpicture}
 \end{tkzexample}

\newpage
\subsection{Similar isosceles triangles}

\begin{tikzpicture}
\node [mybox,title={Similar isosceles triangles}] (box){%
\begin{minipage}{0.90\textwidth}
  { \emph{The following is from the excellent site \textbf{Descartes et les Mathématiques}. I did not modify the text and I am only the author of the programming of the figures.
\url{http://debart.pagesperso-orange.fr/seconde/triangle.html}
}} 
\end{minipage}
};
\end{tikzpicture}% 

The following is from the excellent site \textbf{Descartes et les Mathématiques}. I did not modify the text and I am only the author of the programming of the figures.

\url{http://debart.pagesperso-orange.fr/seconde/triangle.html}

Bibliography:

\begin{itemize}
\item   Géométrie au Bac - Tangente, special issue no. 8 - Exercise 11, page 11

\item   Elisabeth Busser and Gilles Cohen: 200 nouveaux problèmes du "Monde" - POLE 2007 (200 new problems of "Le Monde")

\item   Affaire de logique n° 364 - Le Monde February 17, 2004
\end{itemize}


Two statements were proposed, one by the magazine \textit{Tangente} and the other by \textit{Le Monde}.

\vspace*{2cm}
\emph{Editor of the magazine "Tangente"}: \textcolor{orange}{Two similar isosceles triangles $AXB$ and $BYC$ are constructed with main vertices $X$ and $Y$, such that $A$, $B$ and $C$ are aligned and that these triangles are "indirect". Let $\alpha$ be the angle at vertex $\widehat{AXB}$ = $\widehat{BYC}$. We then construct a third isosceles triangle $XZY$ similar to the first two, with main vertex $Z$ and "indirect".
We ask to demonstrate that point $Z$ belongs to the straight line $(AC)$.}

\vspace*{2cm}
\emph{Editor of  "Le Monde"}: \textcolor{orange}{We construct two similar isosceles triangles $AXB$ and $BYC$ with principal vertices $X$ and $Y$, such that $A$, $B$ and $C$ are aligned and that these triangles are "indirect". Let $\alpha$ be the angle at vertex $\widehat{AXB}$ = $\widehat{BYC}$. The point Z of the line segment $[AC]$ is equidistant from the two vertices $X$ and $Y$.\\
At what angle does he see these two vertices?}

\vspace*{2cm} The constructions and their associated codes are on the next two pages, but you can search before looking. The programming respects (it seems to me ...) my reasoning in both cases.

\subsection{Revised version of "Tangente"}
\begin{tkzexample}[]
\begin{tikzpicture}[scale=.8,rotate=60]
  \tkzDefPoint(6,0){X}   \tkzDefPoint(3,3){Y}
  \tkzDefShiftPoint[X](-110:6){A}    \tkzDefShiftPoint[X](-70:6){B}
  \tkzDefShiftPoint[Y](-110:4.2){A'} \tkzDefShiftPoint[Y](-70:4.2){B'}
  \tkzDefPointBy[translation= from A' to B ](Y) \tkzGetPoint{Y}
  \tkzDefPointBy[translation= from A' to B ](B') \tkzGetPoint{C}
  \tkzInterLL(A,B)(X,Y) \tkzGetPoint{O}
  \tkzDefMidPoint(X,Y) \tkzGetPoint{I}
  \tkzDefPointWith[orthogonal](I,Y)
  \tkzInterLL(I,tkzPointResult)(A,B) \tkzGetPoint{Z}
  \tkzDefCircle[circum](X,Y,B) \tkzGetPoint{O}
  \tkzDrawCircle(O,X)
  \tkzDrawLines[add = 0 and 1.5](A,C) \tkzDrawLines[add = 0 and 3](X,Y)
  \tkzDrawSegments(A,X B,X B,Y C,Y)   \tkzDrawSegments[color=red](X,Z Y,Z)
  \tkzDrawPoints(A,B,C,X,Y,O,Z)
  \tkzLabelPoints(A,B,C,Z)   \tkzLabelPoints[above right](X,Y,O)
\end{tikzpicture}
\end{tkzexample}

\subsection{"Le Monde" version}

\begin{tkzexample}[]
\begin{tikzpicture}[scale=1.25]
  \tkzDefPoint(0,0){A}
  \tkzDefPoint(3,0){B}
  \tkzDefPoint(9,0){C}
  \tkzDefPoint(1.5,2){X}
  \tkzDefPoint(6,4){Y}
  \tkzDefCircle[circum](X,Y,B) \tkzGetPoint{O}
  \tkzDefMidPoint(X,Y)               \tkzGetPoint{I}
  \tkzDefPointWith[orthogonal](I,Y)  \tkzGetPoint{i}
  \tkzDrawLines[add = 2 and 1,color=orange](I,i)
  \tkzInterLL(I,i)(A,B)              \tkzGetPoint{Z}
  \tkzInterLC(I,i)(O,B)              \tkzGetFirstPoint{M}
  \tkzDefPointWith[orthogonal](B,Z)  \tkzGetPoint{b}
  \tkzDrawCircle(O,B)
  \tkzDrawLines[add = 0 and 2,color=orange](B,b)
  \tkzDrawSegments(A,X B,X B,Y C,Y A,C X,Y)
  \tkzDrawSegments[color=red](X,Z Y,Z)
  \tkzDrawPoints(A,B,C,X,Y,Z,M,I)
  \tkzLabelPoints(A,B,C,Z)
  \tkzLabelPoints[above right](X,Y,M,I)
\end{tikzpicture}
\end{tkzexample}

\subsection{Triangle altitudes}

\begin{tikzpicture}
\node [mybox,title={Triangle altitudes}] (box){%
\begin{minipage}{0.90\textwidth}
  { \emph{From Wikipedia : The following is again from the excellent site \textbf{Descartes et les Mathématiques} (Descartes and the Mathematics).
\url{http://debart.pagesperso-orange.fr/geoplan/geometrie_triangle.html}.
The three altitudes of a triangle intersect at the same H-point. 
}} 
\end{minipage}
};
\end{tikzpicture}% 

\begin{tkzexample}[vbox,small]
\begin{tikzpicture}
   \tkzDefPoint(0,0){C} \tkzDefPoint(7,0){B}
   \tkzDefPoint(5,6){A}
   \tkzDefMidPoint(C,B) \tkzGetPoint{I}
   \tkzInterLC(A,C)(I,B)
   \tkzGetFirstPoint{B'}
   \tkzInterLC(A,B)(I,B)
   \tkzGetSecondPoint{C'}
   \tkzInterLL(B,B')(C,C') \tkzGetPoint{H}
   \tkzInterLL(A,H)(C,B) \tkzGetPoint{A'}
   \tkzDefCircle[circum](A,B',C') \tkzGetPoint{O}
   \tkzDrawArc(I,B)(C)
   \tkzDrawPolygon(A,B,C)
   \tkzDrawCircle[color=red](O,A)
   \tkzDrawSegments[color=orange](B,B' C,C' A,A')
   \tkzMarkRightAngles(C,B',B B,C',C C,A',A)
   \tkzDrawPoints(A,B,C,A',B',C',H)
   \tkzLabelPoints[above right](A,B',C',H)
   \tkzLabelPoints[below right](B,C,A')
\end{tikzpicture}
\end{tkzexample}

\subsection{Altitudes - other construction}

\begin{tkzexample}[vbox,small]
\begin{tikzpicture}
\tkzDefPoint(0,0){A} \tkzDefPoint(8,0){B} 
\tkzDefPoint(5,6){C} 
\tkzDefMidPoint(A,B)\tkzGetPoint{O} 
\tkzDefPointBy[projection=onto A--B](C) \tkzGetPoint{P}
\tkzInterLC[common=A](C,A)(O,A)
\tkzGetFirstPoint{M}
\tkzInterLC(C,B)(O,A)
\tkzGetSecondPoint{N}
\tkzInterLL(B,M)(A,N)\tkzGetPoint{I}
\tkzDefCircle[diameter](A,B)\tkzGetPoint{x}
\tkzDefCircle[diameter](I,C)\tkzGetPoint{y}
\tkzDrawCircles(x,A y,C)
\tkzDrawSegments(C,A C,B A,B B,M A,N)
\tkzMarkRightAngles[fill=brown!20](A,M,B A,N,B A,P,C)
\tkzDrawSegment[style=dashed,color=orange](C,P)
\tkzLabelPoints(O,A,B,P)
\tkzLabelPoint[left](M){$M$} 
\tkzLabelPoint[right](N){$N$} 
\tkzLabelPoint[above](C){$C$} 
\tkzLabelPoint[above right](I){$I$} 
\tkzDrawPoints[color=red](M,N,P,I) 
\tkzDrawPoints[color=brown](O,A,B,C)
\end{tikzpicture}
\end{tkzexample}

\newpage
\subsection{Three circles  in an Equilateral Triangle }
\begin{tikzpicture}
\node [mybox,title={Three circles  in an Equilateral Triangle}] (box){%
\begin{minipage}{0.90\textwidth}
  { \emph{From Wikipedia : In geometry, the Malfatti circles are three circles inside a given triangle such that each circle is tangent to the other two and to two sides of the triangle. They are named after Gian Francesco Malfatti, who made early studies of the problem of constructing these circles in the mistaken belief that they would have the largest possible total area of any three disjoint circles within the triangle. Below is a study of a particular case with an equilateral triangle and three identical circles. 
}} 
\end{minipage}
};
\end{tikzpicture}% 
\begin{tkzexample}[latex=7cm,small]
\begin{tikzpicture}[scale=.8]
  \tkzDefPoints{0/0/A,8/0/B,0/4/a,8/4/b,8/8/c}
  \tkzDefTriangle[equilateral](A,B) \tkzGetPoint{C}
  \tkzDefMidPoint(A,B) \tkzGetPoint{M}
  \tkzDefMidPoint(B,C) \tkzGetPoint{N}
  \tkzDefMidPoint(A,C) \tkzGetPoint{P}
  \tkzInterLL(A,N)(M,a) \tkzGetPoint{Ia}
  \tkzDefPointBy[projection = onto A--B](Ia)
  \tkzGetPoint{ha}
  \tkzInterLL(B,P)(M,b) \tkzGetPoint{Ib}
  \tkzDefPointBy[projection = onto A--B](Ib)
  \tkzGetPoint{hb}
  \tkzInterLL(A,c)(M,C) \tkzGetPoint{Ic}
  \tkzDefPointBy[projection = onto A--C](Ic)
  \tkzGetPoint{hc}
  \tkzInterLL(A,Ia)(B,Ib) \tkzGetPoint{G}
  \tkzDefSquare(A,B) \tkzGetPoints{D}{E}
  \tkzDrawPolygon(A,B,C)
  \tkzClipBB
  \tkzDrawSemiCircles[gray,dashed](M,B A,M 
  A,B B,A G,Ia)
  \tkzDrawCircles[gray](Ia,ha Ib,hb Ic,hc)
  \tkzDrawPolySeg(A,E,D,B)
  \tkzDrawPoints(A,B,C,G,Ia,Ib,Ic)
  \tkzDrawSegments[gray,dashed](C,M A,N B,P
   M,a M,b A,a a,b b,B A,D Ia,ha)
\end{tikzpicture}
\end{tkzexample}

\newpage
\subsection{Law of sines}
\begin{tikzpicture}
\node [mybox,title={Law of sines}] (box){%
\begin{minipage}{0.90\textwidth}
  {From wikipedia : \emph{In trigonometry, the law of sines, sine law, sine formula, or sine rule is an equation relating the lengths of the sides of a triangle (any shape) to the sines of its angles.
}} 
\end{minipage}
};
\end{tikzpicture}% 

\begin{tkzexample}[latex=7cm,small]
  \begin{tikzpicture}
  \tkzDefPoints{0/0/A,5/1/B,2/6/C}
  \tkzDefTriangleCenter[circum](A,B,C)
   \tkzGetPoint{O} 
  \tkzDefPointBy[symmetry= center O](B) 
   \tkzGetPoint{D} 
  \tkzDrawPolygon[color=brown](A,B,C)
  \tkzDrawCircle(O,A)
  \tkzDrawPoints(A,B,C,D,O)
  \tkzDrawSegments[dashed](B,D A,D)
  \tkzLabelPoint[left](D){$D$}
  \tkzLabelPoint[below](A){$A$}
  \tkzLabelPoint[above](C){$C$}
  \tkzLabelPoint[right](B){$B$}
  \tkzLabelPoint[below](O){$O$}
  \tkzLabelSegment(B,C){$a$}
  \tkzLabelSegment[left](A,C){$b$}
  \tkzLabelSegment(A,B){$c$}
  \end{tikzpicture}
\end{tkzexample}

In the triangle $ABC$ 

\begin{equation}
\frac{a}{\sin A} = \frac{b}{\sin B} =\frac{c}{\sin C}
\end{equation}

\[\widehat{C} = \widehat{D}\] 
\begin{equation}
\frac{c}{2R} = \sin D = \sin C 
\end{equation}

Then \[ \frac{c}{\sin C} = 2R\]

\newpage
\subsection{Flower of Life}
\begin{tikzpicture}
\node [mybox,title={Book IV, proposition XI  \_Euclid's Elements\_}] (box){%
\begin{minipage}{0.90\textwidth}
  {\emph{Sacred geometry can be described as a belief system attributing a religious or cultural value to many of the fundamental forms of space and time. According to this belief system, the basic patterns of existence are perceived as sacred because in contemplating them one is contemplating the origin of all things. By studying the nature of these forms and their relationship to each other, one may seek to gain insight into the scientific, philosophical, psychological, aesthetic and mystical laws of the universe.
The Flower of Life is considered to be a symbol of sacred geometry, said to contain ancient, religious value depicting the fundamental forms of space and time. In this sense, it is a visual expression of the connections life weaves through all mankind, believed by some to contain a type of Akashic Record of basic information of all living things.
}} 
\end{minipage}
};
\end{tikzpicture}% 

One of the beautiful arrangements of circles found at the Temple of Osiris at Abydos, Egypt (Rawles 1997). \\
Weisstein, Eric W. "Flower of Life." From MathWorld--A Wolfram Web Resource.\\ \url{http://mathworld.wolfram.com/FlowerofLife.html}
 
\begin{tkzexample}[vbox,small]
\begin{tikzpicture}[scale=.75]
  \tkzSetUpLine[line width=2pt,color=teal!80!black]
  \tkzSetUpCompass[line width=2pt,color=teal!80!black]
   \tkzDefPoint(0,0){O}  \tkzDefPoint(2.25,0){A}
   \tkzDrawCircle(O,A)
\foreach \i in {0,...,5}{
   \tkzDefPointBy[rotation= center O angle 30+60*\i](A)\tkzGetPoint{a\i}
   \tkzDefPointBy[rotation= center {a\i} angle  120](O)\tkzGetPoint{b\i}
   \tkzDefPointBy[rotation= center {a\i} angle  180](O)\tkzGetPoint{c\i}
   \tkzDefPointBy[rotation= center {c\i} angle  120](a\i)\tkzGetPoint{d\i}
   \tkzDefPointBy[rotation= center {c\i} angle   60](d\i)\tkzGetPoint{f\i}
   \tkzDefPointBy[rotation= center {d\i} angle   60](b\i)\tkzGetPoint{e\i} 
   \tkzDefPointBy[rotation= center {f\i} angle   60](d\i)\tkzGetPoint{g\i} 
   \tkzDefPointBy[rotation= center {d\i} angle   60](e\i)\tkzGetPoint{h\i}
   \tkzDefPointBy[rotation= center {e\i} angle  180](b\i)\tkzGetPoint{k\i}   
   \tkzDrawCircle(a\i,O)
   \tkzDrawCircle(b\i,a\i)
   \tkzDrawCircle(c\i,a\i)
   \tkzDrawArc[rotate](f\i,d\i)(-120)
   \tkzDrawArc[rotate](e\i,d\i)(180)
   \tkzDrawArc[rotate](d\i,f\i)(180)
   \tkzDrawArc[rotate](g\i,f\i)(60)
   \tkzDrawArc[rotate](h\i,d\i)(60)
   \tkzDrawArc[rotate](k\i,e\i)(60) 
}
   \tkzClipCircle(O,f0)
\end{tikzpicture}
\end{tkzexample}


\newpage
\subsection{Pentagon in a circle}
\begin{tikzpicture}
\node [mybox,title={Book IV, proposition XI  \_Euclid's Elements\_}] (box){%
\begin{minipage}{0.90\textwidth}
  {\emph{To inscribe an equilateral and equiangular pentagon in a given circle.
}} 
\end{minipage}
};
\end{tikzpicture}% 

\begin{tkzexample}[vbox,small]
\begin{tikzpicture}[scale=.75]
   \tkzDefPoint(0,0){O} 
   \tkzDefPoint(5,0){A}
   \tkzDefPoint(0,5){B}
   \tkzDefPoint(-5,0){C} 
   \tkzDefPoint(0,-5){D}
   \tkzDefMidPoint(A,O)             \tkzGetPoint{I}
   \tkzInterLC(I,B)(I,A)            \tkzGetPoints{F}{E}
   \tkzInterCC(O,C)(B,E)            \tkzGetPoints{D3}{D2}
   \tkzInterCC(O,C)(B,F)            \tkzGetPoints{D4}{D1}
   \tkzDrawArc[angles](B,E)(180,360)
   \tkzDrawArc[angles](B,F)(220,340)
   \tkzDrawLine[add=.5 and .5](B,I)
   \tkzDrawCircle(O,A)
   \tkzDefCircle[diameter](O,A)     \tkzGetPoint{x}
   \tkzDrawCircle(x,A)
   \tkzDrawSegments(B,D C,A) 
   \tkzDrawPolygon[new](D,D1,D2,D3,D4)
   \tkzDrawPoints(A,...,D,O)
   \tkzDrawPoints[new](E,F,I,D1,D2,D4,D3)
   \tkzLabelPoints[below left](A,...,D,O)
   \tkzLabelPoints[new,below right](I,E,F,D1,D2,D4,D3)  
\end{tikzpicture}
\end{tkzexample}

 \newpage
 \subsection{Pentagon in a square}
 \begin{tikzpicture}
 \node [mybox,title={Pentagon in a square}] (box){%
 \begin{minipage}{0.90\textwidth}
   {: \emph{To inscribe an equilateral and equiangular pentagon in a given square.
 }} 
 \end{minipage}
 };
 \end{tikzpicture}%
    
\begin{tkzexample}[vbox,small]
\begin{tikzpicture}[scale=.75]
  \tkzDefPoints{0/0/O,-5/-5/A,5/-5/B}
  \tkzDefSquare(A,B)   \tkzGetPoints{C}{D}
  \tkzDefMidPoint(A,B) \tkzGetPoint{F}
  \tkzDefMidPoint(C,D) \tkzGetPoint{E}
  \tkzDefMidPoint(B,C) \tkzGetPoint{G}
  \tkzDefMidPoint(A,D) \tkzGetPoint{K}
  \tkzInterLC(D,C)(E,B)                    \tkzGetSecondPoint{T}
  \tkzDefMidPoint(D,T)                     \tkzGetPoint{I}
  \tkzInterCC[with nodes](O,D,I)(E,D,I)    \tkzGetSecondPoint{H}
  \tkzInterLC(O,H)(O,E)                    \tkzGetSecondPoint{M}
  \tkzInterCC(O,E)(E,M)                    \tkzGetFirstPoint{Q}
  \tkzInterCC[with nodes](O,O,E)(Q,E,M)    \tkzGetFirstPoint{P}
  \tkzInterCC[with nodes](O,O,E)(P,E,M)    \tkzGetFirstPoint{N}
  \tkzCompasss(O,H E,H)
  \tkzDrawArc(E,B)(T)
  \tkzDrawPolygons[purple](A,B,C,D M,E,Q,P,N) 
  \tkzDrawCircle(O,E)
  \tkzDrawSegments(T,I O,H E,H E,F G,K)
  \tkzDrawPoints(T,M,Q,P,N,I)
  \tkzLabelPoints(A,B,O,N,P,Q,M,H)
  \tkzLabelPoints[above right](C,D,E,I,T)
\end{tikzpicture} 
\end{tkzexample}

\newpage
 \subsection{Hexagon Inscribed}
 \begin{tikzpicture}
 \node [mybox,title={Hexagon Inscribed}] (box){%
 \begin{minipage}{0.90\textwidth}
   {\emph{To inscribe a regular hexagon in a given equilateral triangle  perfectly inside it (no boarders).
 }} 
 \end{minipage}
 };
 \end{tikzpicture}%
 
\subsubsection{Hexagon Inscribed version 1} % (fold)
\label{ssub:hexagon_inscribed_version_1}
\begin{tkzexample}[latex=7cm,small]
  \begin{tikzpicture}[scale=.5]
   \pgfmathsetmacro{\c}{6} 
   \tkzDefPoints{0/0/A,\c/0/B}
   \tkzDefTriangle[equilateral](A,B)\tkzGetPoint{C}
   \tkzDefTriangleCenter[centroid](A,B,C) 
   \tkzGetPoint{I}
   \tkzDefPointBy[homothety=center A ratio 1./3](B) 
   \tkzGetPoint{c1}
   \tkzInterLC(B,C)(I,c1) \tkzGetPoints{a1}{a2}
   \tkzInterLC(A,C)(I,c1) \tkzGetPoints{b1}{b2}
   \tkzInterLC(A,B)(I,c1) \tkzGetPoints{c1}{c2}
   \tkzDrawPolygon(A,B,C)
   \tkzDrawCircle[thin,orange](I,c1)
   \tkzDrawPolygon[red,thick](a2,a1,b2,b1,c2,c1)
 \end{tikzpicture} 
\end{tkzexample}
% subsubsection hexagon_inscribed_version_1 (end)

\subsubsection{Hexagon Inscribed version 2} % (fold)
\label{ssub:hexagon_inscribed_version_2}
\begin{tkzexample}[latex=7cm,small]
\begin{tikzpicture}[scale=.5]
 \pgfmathsetmacro{\c}{6} 
 \tkzDefPoints{0/0/A,\c/0/B}
 \tkzDefTriangle[equilateral](A,B)\tkzGetPoint{C}
 \tkzDefTriangleCenter[centroid](A,B,C) 
 \tkzGetPoint{I}
 \tkzDefPointsBy[rotation= center I%
                 angle 60](A,B,C){a,b,c}
 \tkzDrawPolygon[fill=teal!20,opacity=.5](A,B,C)
 \tkzDrawPolygon[fill=purple!20,opacity=.5](a,b,c)
\end{tikzpicture} 
\end{tkzexample}
% subsubsection hexagon_inscribed_version_2 (end)

\newpage
\subsection{Power of a point with respect to a circle}

\begin{tikzpicture}
\node [mybox,title={Power of a point with respect to a circle}] (box){%
\begin{minipage}{0.90\textwidth}
  {\emph{$\overline{MA} \times \overline{MB}={MT}^2={MO}^2-{OT}^2$} } 
\end{minipage}
};
\end{tikzpicture}% 

\begin{tkzexample}[vbox,small]
\begin{tikzpicture}
 \pgfmathsetmacro{\r}{2}%
 \pgfmathsetmacro{\xO}{6}%
 \pgfmathsetmacro{\xE}{\xO-\r}%
 \tkzDefPoints{0/0/M,\xO/0/O,\xE/0/E}
 \tkzDefCircle[diameter](M,O)
 \tkzGetPoint{I}
 \tkzInterCC(I,O)(O,E) \tkzGetPoints{T}{T'}
 \tkzDefShiftPoint[O](45:2){B}
 \tkzInterLC(M,B)(O,E) \tkzGetPoints{A}{B}
 \tkzDrawCircle(O,E)
 \tkzDrawSemiCircle[dashed](I,O)
 \tkzDrawLine(M,O)
 \tkzDrawLines(M,T O,T M,B)
 \tkzDrawPoints(A,B,T)
 \tkzLabelPoints[above](A,B,O,M,T)
\end{tikzpicture}
\end{tkzexample}

\newpage
\subsection{Radical axis of two non-concentric circles}
\begin{tikzpicture}
\node [mybox,title={Radical axis of two non-concentric circles}] (box){%
\begin{minipage}{0.90\textwidth}
  {From Wikipedia : \emph{In geometry, the radical axis of two non-concentric circles is the set of points whose power with respect to the circles are equal. For this reason the radical axis is also called the power line or power bisector of the two circles.  The notation radical axis was used by the French mathematician M. Chasles as axe radical.
}} 
\end{minipage}
};
\end{tikzpicture}% 

\begin{tkzexample}[vbox,small]
\begin{tikzpicture}
\tkzDefPoints{0/0/A,4/2/B,2/3/K}
\tkzDefCircle[R](A,1)\tkzGetPoint{a}
\tkzDefCircle[R](B,2)\tkzGetPoint{b}
\tkzDefCircle[R](K,3)\tkzGetPoint{k}
\tkzDrawCircles(A,a B,b)
\tkzDrawCircle[dashed,new](K,k)
\tkzInterCC(A,a)(K,k) \tkzGetPoints{a}{a'}
\tkzInterCC(B,b)(K,k) \tkzGetPoints{b}{b'}
\tkzDrawLines[new,add=2 and 2](a,a')
\tkzDrawLines[new,add=1 and 1](b,b')
\tkzInterLL(a,a')(b,b') \tkzGetPoint{X}
\tkzDefPointBy[projection= onto A--B](X) \tkzGetPoint{H}
\tkzDrawPoints(A,B,H,X,a,b,a',b')
\tkzDrawLine(A,B)
\tkzDrawLine[add= 1 and 2,new](X,H)
\tkzLabelPoints(A,B,H,X,a,b,a',b')
\end{tikzpicture}
\end{tkzexample}

\newpage
\subsection{External homothetic center}
\begin{tikzpicture}
\node [mybox,title={External homothetic center}] (box){%
\begin{minipage}{0.90\textwidth}
  {From Wikipedia : \emph{ Given two nonconcentric circles, draw radii parallel and in the same direction. Then the line joining the extremities of the radii passes through a fixed point on the line of centers which divides that line externally in the ratio of radii. This point is called the external homothetic center, or external center of similitude (Johnson 1929, pp. 19-20 and 41).
}} 
\end{minipage}
};
\end{tikzpicture}% 

\begin{tkzexample}[vbox,small]
\begin{tikzpicture}
\tkzDefPoints{0/0/A,4/2/B,2/3/K}
\tkzDefCircle[R](A,1)\tkzGetPoint{a}
\tkzDefCircle[R](B,2)\tkzGetPoint{b}
\tkzDrawCircles(A,a B,b)
\tkzDrawLine(A,B)
\tkzDefShiftPoint[A](60:1){M}
\tkzDefShiftPoint[B](60:2){M'}
\tkzInterLL(A,B)(M,M') \tkzGetPoint{O}
\tkzDefLine[tangent from = O](B,M') \tkzGetPoints{X}{T'}
\tkzDefLine[tangent from = O](A,M) \tkzGetPoints{X}{T}
\tkzDrawPoints(A,B,O,T,T',M,M')
\tkzDrawLines[new](O,B O,T' O,M')
\tkzDrawSegments[new](A,M B,M')
\tkzLabelPoints(A,B,O,T,T',M,M')
\end{tikzpicture}
\end{tkzexample}

\newpage
\subsection{Tangent lines to two circles}

\begin{tikzpicture}
\node [mybox,title={Tangent lines to two circles}] (box){%
\begin{minipage}{0.90\textwidth}
  {\emph{For two circles, there are generally four distinct lines that are tangent to both  if the two circles are outside each other.  For two of these, the external tangent lines, the circles fall on the same side of the line; the external tangent lines intersect in the external homothetic center}}
\end{minipage}
};
\end{tikzpicture}%

\begin{tkzexample}[vbox,small]
\begin{tikzpicture}
 \pgfmathsetmacro{\r}{1}%
 \pgfmathsetmacro{\R}{2}%
 \pgfmathsetmacro{\rt}{\R-\r}%
 \tkzDefPoints{0/0/A,4/2/B,2/3/K}
 \tkzDefMidPoint(A,B) \tkzGetPoint{I}
 \tkzInterLC[R](A,B)(B,\rt) \tkzGetPoints{E}{F}
 \tkzInterCC(I,B)(B,F) \tkzGetPoints{a}{a'}
 \tkzInterLC[R](B,a)(B,\R) \tkzGetPoints{X'}{T'}
 \tkzDefLine[tangent at=T'](B) \tkzGetPoint{h}
 \tkzInterLL(T',h)(A,B) \tkzGetPoint{O}
 \tkzInterLC[R](O,T')(A,\r) \tkzGetPoints{T}{T}
 \tkzDefCircle[R](A,\r)  \tkzGetPoint{a}         
 \tkzDefCircle[R](B,\R)  \tkzGetPoint{b}
 \tkzDefCircle[R](B,\rt)  \tkzGetPoint{c}
 \tkzDrawCircles(A,a)  
 \tkzDrawCircles[orange](B,b B,c)           
 \tkzDrawCircle[orange,dashed](I,B)
 \tkzDrawPoints(O,A,B,a,a',E,F,T',T)
 \tkzDrawLines(O,B A,a B,T' A,T)
 \tkzDrawLines[add= 1 and 8](T',h)
 \tkzLabelPoints(O,A,B,a,a',E,F,T,T')
\end{tikzpicture}
\end{tkzexample}

\newpage
\subsection{Tangent lines to two circles with radical axis}

\begin{tikzpicture}
\node [mybox,title={Tangent lines to two circles with radical axis}] (box){%
\begin{minipage}{0.90\textwidth}
  {\emph{As soon as two circles are not concentric, we can construct their radical axis, the set of points of equal power with respect to the two circles. We know that the radical axis is a line orthogonal to the line of the centers. Note that if we specify $P$ and $Q$ as the points of contact of one of the common exterior tangents with the two circles and $D$ and $E$ as the points of the circles outside [AB], then (DP) and (EQ) intersect on the radical axis of the two circles. We will show that this property is always true and that it allows us to construct common tangents, even when the circles have the same radius. }}
\end{minipage}
};
\end{tikzpicture}%


\begin{tkzexample}[vbox,small]
\begin{tikzpicture}
\tkzDefPoints{0/0/A,4/2/B,2/3/K}
\tkzDefCircle[R](A,1) \tkzGetPoint{a}
\tkzDefCircle[R](B,3) \tkzGetPoint{b}
\tkzInterCC[R](A,1)(K,3) \tkzGetPoints{a}{a'}
\tkzInterCC[R](B,3)(K,3) \tkzGetPoints{b}{b'}
\tkzInterLL(a,a')(b,b')  \tkzGetPoint{X}
\tkzDefPointBy[projection= onto A--B](X) \tkzGetPoint{H}
\tkzGetPoint{C}
\tkzInterLC[R](A,B)(B,3) \tkzGetPoints{b1}{E}
\tkzInterLC[R](A,B)(A,1) \tkzGetPoints{D}{a2}
\tkzDefMidPoint(D,E) \tkzGetPoint{I}
\tkzDrawCircle[orange](I,D)
\tkzInterLC(X,H)(I,D) \tkzGetPoints{M}{M'}
\tkzInterLC(M,D)(A,D) \tkzGetPoints{P}{P'}
\tkzInterLC(M,E)(B,E) \tkzGetPoints{Q'}{Q}
\tkzInterLL(P,Q)(A,B) \tkzGetPoint{O}
\tkzDrawCircles(A,a B,b)
\tkzDrawSegments[orange](A,P I,M B,Q)
\tkzDrawPoints(A,B,D,E,M,I,O,P,Q,X,H)
\tkzDrawLines(O,E M,D M,E O,Q)
\tkzDrawLine[add= 3 and 4,orange](X,H)
\tkzLabelPoints(A,B,D,E,M,I,O,P,Q,X,H)
\end{tikzpicture}
\end{tkzexample}

\newpage
\subsection{Middle of a  segment with a compass}

\begin{tikzpicture}
\node [mybox,title={Tangent lines to two circles with radical axis}] (box){%
\begin{minipage}{0.90\textwidth}
  {\emph{This example involves determining the middle of a segment, using only a compass.}}
\end{minipage}
};
\end{tikzpicture}%

\begin{tkzexample}[vbox,small]
\begin{tikzpicture}
\tkzDefPoint(0,0){A}
\tkzDefRandPointOn[circle= center A radius 4]    \tkzGetPoint{B}
\tkzDefPointBy[rotation= center A angle 180](B)  \tkzGetPoint{C}
\tkzInterCC(A,B)(B,A)                            \tkzGetPoints{I}{I'}
\tkzInterCC(A,I)(I,A)                            \tkzGetPoints{J}{B}
\tkzInterCC(B,A)(C,B)                            \tkzGetPoints{D}{E}
\tkzInterCC(D,B)(E,B)                            \tkzGetPoints{M}{M'}
\tkzSetUpArc[color=orange,style=solid,delta=10]
\tkzDrawArc(C,D)(E)
\tkzDrawArc(B,E)(D)
\tkzDrawCircle[color=teal,line width=.2pt](A,B)
\tkzDrawArc(D,B)(M) 
\tkzDrawArc(E,M)(B)
\tkzCompasss[color=orange,style=solid](B,I I,J J,C)
\tkzDrawPoints(A,B,C,D,E,M)
\tkzLabelPoints(A,B,M)
\end{tikzpicture}
\end{tkzexample}
 
\newpage

\subsection{Definition of a circle  \_Apollonius\_}

\begin{tikzpicture}
\node [mybox,title={Definition of a circle  \_Apollonius\_}] (box){%
\begin{minipage}{0.90\textwidth}
  {From Wikipedia : \emph{Apollonius showed that a circle can be defined as the set of points in a plane that have a specified ratio of distances to two fixed points, known as foci. This Apollonian circle is the basis of the Apollonius pursuit problem. ... The solutions to this problem are sometimes called the circles of Apollonius.}} 
\end{minipage}
};
\end{tikzpicture}% 

Explanation

A circle is the set of points in a plane that are equidistant from a given point O. The distance r from the center is called the radius, and the point O is called the center. It is the simplest definition but it is not the only one. Apollonius of Perga gives another definition :
The set of all points whose distances from two fixed points are in a constant ratio is a circle.

With \pkg{tkz-euclide} is easy to show you the last definition

\begin{tkzexample}[vbox, small]
\begin{tikzpicture}[scale=1.5]
    % Firstly we defined two fixed point. 
    % The figure depends of these points and the ratio K
\tkzDefPoint(0,0){A}
\tkzDefPoint(4,0){B}
    % tkz-euclide.sty knows about the apollonius's circle
    % with K=2 we search some points like  I such as IA=2 x IB
\tkzDefCircle[apollonius,K=2](A,B) \tkzGetPoints{K1}{k}
\tkzDefPointOnCircle[through=  center K1 angle 30 point k]
\tkzGetPoint{I}
\tkzDefPointOnCircle[through= center K1 angle 280  point k]
\tkzGetPoint{J}
\tkzDrawSegments[new](A,I I,B A,J J,B)  
\tkzDrawCircle[color = teal,fill=teal!20,opacity=.4](K1,k)
\tkzDrawPoints(A,B,K1,I,J)
\tkzDrawSegment(A,B)
\tkzLabelPoints[below,font=\scriptsize](A,B,K1,I,J)
\end{tikzpicture}
\end{tkzexample}

\subsection{Application of Inversion : \tkzname{Pappus chain} }\label{pappus}
\begin{tikzpicture}
\node [mybox,title={Pappus chain}] (box){%
\begin{minipage}{0.90\textwidth}
From Wikipedia  {\emph{In geometry, the Pappus chain is a ring of circles between two tangent circles investigated by Pappus of Alexandria in the 3rd century AD.}}
\end{minipage}
};
\end{tikzpicture}%

\begin{tkzexample}[vbox,small]
\begin{tikzpicture}[ultra thin]
  \pgfmathsetmacro{\xB}{6}%
  \pgfmathsetmacro{\xC}{9}%
  \pgfmathsetmacro{\xD}{(\xC*\xC)/\xB}%
  \pgfmathsetmacro{\xJ}{(\xC+\xD)/2}%
  \pgfmathsetmacro{\r}{\xD-\xJ}%
  \pgfmathsetmacro{\nc}{16}%
  \tkzDefPoints{0/0/A,\xB/0/B,\xC/0/C,\xD/0/D}
  \tkzDefCircle[diameter](A,C) \tkzGetPoint{x}
  \tkzDrawCircle[fill=teal!30](x,C)
  \tkzDefCircle[diameter](A,B) \tkzGetPoint{y}
  \tkzDrawCircle[fill=teal!30](y,B)
  \foreach \i in {-\nc,...,0,...,\nc}
  {\tkzDefPoint(\xJ,2*\r*\i){J}
   \tkzDefPoint(\xJ,2*\r*\i-\r){H}
   \tkzDefCircleBy[inversion = center A through C](J,H)
   \tkzDrawCircle[fill=teal](tkzFirstPointResult,tkzSecondPointResult)}
\end{tikzpicture}
\end{tkzexample}

\subsection{Book of lemmas proposition 1 Archimedes}
\begin{tikzpicture}
\node [mybox,title={Book of lemmas proposition 1 Archimedes}] (box){%
\begin{minipage}{0.90\textwidth}
  {\emph{If two circles touch at $A$, and if $[CD]$, $[EF]$ be parallel diameters in them, $A$, $C$ and $E$ are aligned.}}
\end{minipage}
};
\end{tikzpicture}%

\begin{tkzexample}[latex=7cm,small]
\begin{tikzpicture}
  \tkzDefPoints{0/0/O_1,0/1/O_2,0/3/A}
  \tkzDefPoint(15:3){F}
  \tkzInterLC(F,O_1)(O_1,A) \tkzGetSecondPoint{E}
  \tkzDefLine[parallel=through O_2](E,F) 
  \tkzGetPoint{x}   
  \tkzInterLC(x,O_2)(O_2,A) \tkzGetPoints{D}{C} 
  \tkzDrawCircles(O_1,A O_2,A)
  \tkzDrawSegments[new](O_1,A E,F C,D)
  \tkzDrawSegments[purple](A,E A,F)
  \tkzDrawPoints(A,O_1,O_2,E,F,C,D)
  \tkzLabelPoints(A,O_1,O_2,E,F,C,D)
\end{tikzpicture}
\end{tkzexample}

$(CD) \parallel (EF)$ $(AO_1)$ is secant to these two lines so
$\widehat{A0_2C} = \widehat{A0_1E}$.

Since the triangles $AO_2C$ and $AO_1E$ are isosceles the angles at the base are equal $widehat{AC0_2} = \widehat{AE0_1} = \widehat{CA0_2} = \widehat{EA0_1}$. Thus $A$,$C$ and $E$ are aligned

\subsection{Book of lemmas proposition 6 Archimedes}
\begin{tikzpicture}
\node [mybox,title={Book of lemmas proposition 6 Archimedes}] (box){%
\begin{minipage}{0.90\textwidth}
  {\emph{Let $AC$, the diameter of a semicircle, be divided at $B$ so that $AC/AB =\phi$ or in any ratio. Describe semicircles within the first semicircle and on $AB$, $BC$ as diameters, and suppose a circle drawn touching the all three semicircles. If $GH$ be the diameter of this circle, to find relation between $GH$ and $AC$.}}
\end{minipage}
};
\end{tikzpicture}%


\begin{tkzexample}[vbox,overhang,small]
\begin{tikzpicture}
\tkzDefPoints{0/0/A,12/0/C}
\tkzDefGoldenRatio(A,C)                  \tkzGetPoint{B}
\tkzDefMidPoint(A,C)                     \tkzGetPoint{O}
\tkzDefMidPoint(A,B)                     \tkzGetPoint{O_1}
\tkzDefMidPoint(B,C)                     \tkzGetPoint{O_2}
\tkzDefExtSimilitudeCenter(O_1,A)(O_2,B) \tkzGetPoint{M_0}
\tkzDefIntSimilitudeCenter(O,A)(O_1,A)   \tkzGetPoint{M_1}
\tkzDefIntSimilitudeCenter(O,C)(O_2,C)   \tkzGetPoint{M_2}
\tkzInterCC(O_1,A)(M_2,C)                \tkzGetFirstPoint{E}
\tkzInterCC(O_2,C)(M_1,A)                \tkzGetSecondPoint{F}
\tkzInterCC(O,A)(M_0,B)                  \tkzGetFirstPoint{D}
\tkzInterLL(O_1,E)(O_2,F)                \tkzGetPoint{O_3}
\tkzDefCircle[circum](E,F,B)             \tkzGetPoint{0_4}
\tkzInterLC(A,D)(O_1,A)                  \tkzGetFirstPoint{I}
\tkzInterLC(C,D)(O_2,B)                  \tkzGetSecondPoint{K}
\tkzInterLC[common=D](A,D)(O_3,D)        \tkzGetFirstPoint{G}
\tkzInterLC[common=D](C,D)(O_3,D)        \tkzGetFirstPoint{H}
\tkzInterLL(C,G)(B,K)                    \tkzGetPoint{M}
\tkzInterLL(A,H)(B,I)                    \tkzGetPoint{L}
\tkzInterLL(L,G)(A,C)                    \tkzGetPoint{N}
\tkzInterLL(M,H)(A,C)                    \tkzGetPoint{P}  
\tkzDrawCircles[red,thin](O_3,F)
\tkzDrawCircles[new,thin](0_4,B)
\tkzDrawSemiCircles[teal](O,C O_1,B O_2,C)
\tkzDrawSemiCircles[green](M_2,C)
\tkzDrawSemiCircles[green,swap](M_1,A)
\tkzDrawSegment(A,C)
\tkzDrawSegments[new](O_1,O_3 O_2,O_3)
\tkzDrawSegments[new,very thin](B,H C,G A,H G,N H,P)
\tkzDrawSegments[new,very thin](B,D A,D C,D G,H I,B K,B B,G)
\tkzDrawPoints(A,B,C,M_1,M_2,E,O_3,F,D,0_4,O_1,O_2,I,K,G,H,L,P,N,M)  
\tkzLabelPoints[font=\scriptsize](A,B,C,M_1,M_2,F,O_1,O_2,I,K,G,H,L,M,N)
\tkzLabelPoints[font=\scriptsize,right](E,O_3,D,0_4,P)
\end{tikzpicture}
\end{tkzexample}

Let $GH$ be the diameter of the circle which is parallel to $AC$, and let the circle touch the semicircles on $AC$, $AB$, $BC$ in $D$, $E$, $F$ respectively.

Then, by Prop. 1 $A$,$G$ and $D$ are aligned, ainsi que $D$, $H$ and $C$.\\
 For a like reason $A$ $E$ and $H$ are aligned, $C$ $F$ and $G$are aligned, as also are $B$ $E$ and $G$, $B$ $F$ and $H$.
 
Let $(AD)$ meet the semicircle on $[AC]$ at $I$, and let $(BD)$ meet the semicircle on $[BC]$ in $K$. Join CI, CK meeting AE, BF in L, M, and let GL, HM produced meet AB in N, P respectively.

Now, in the triangle $AGB$, the perpendiculars from $A$, $C$ on the opposite sides meet in $L$. Therefore by the properties of triangles, $(GN)$ is perpendicular to $(AC)$.
Similarly $(HP)$ is perpendicular to $(BC)$.\\
Again, since the angles at $I$, $K$, $D$ are right, $(CK)$ is parallel to $(AD)$, and $(CI)$ to $(BD)$.

 Therefore\\
\[\frac{AB}{BC} = \frac{AL}{LH}    =  \frac{AN}{NP}  \quad\text{and} \quad \frac{BC}{AB} = \frac{CM}{MG}    =  \frac{PC}{NP} \]

hence

\[ \frac{AN}{NP}    =  \frac{NP}{PC} \quad\text{so} \quad {NP}^2 = AN \times PC  \]

Now suppose that $B$ divides $[AC]$ according to the divine proportion that is :
\[\phi = \frac{AB}{BC} =  \frac{AC}{AB} \quad\text{then}  \quad AN = \phi NP \text{and}\quad  NP = \phi PC \]

We have 
\[ AC = AN + NP + PC\quad \text{either} \quad AB + BC = = AN + NP + PC \quad \text{or} \quad (\phi + 1) BC = AN + NP + PC \]

we get 

\[ (\phi + 1) BC = \phi NP + NP + PC =(\phi + 1)NP + PC = \phi(\phi + 1)PC + PC = {\phi}^2 + \phi + 1)PC \]

as 
\[ {\phi}^2 = \phi + 1 \quad \text{then} \quad (\phi + 1) BC = 2(\phi + 1) PC \quad\text{i.e.}\quad BC = 2 PC \]

That is,
$p$ is the middle of the segment $BC$.

Part of the proof from \url{https://www.cut-the-knot.org}


\subsection{ "The" Circle of APOLLONIUS}

\begin{tikzpicture}
\node [mybox,title={The Apollonius circle of a triangle  \_Apollonius\_}] (box){%
\begin{minipage}{0.90\textwidth}
  {\emph{The circle which touches all three excircles of a triangle and encompasses them is often known as "the" Apollonius circle (Kimberling 1998, p. 102)}}
\end{minipage}
};
\end{tikzpicture}%

Explanation

The purpose of the first  examples was to show the simplicity with which we could recreate these propositions. With TikZ you need to do calculations and use trigonometry while with \pkg{tkz-euclide} you only need to build simple objects

But don't forget that behind or far above \pkg{tkz-euclide} there is TikZ. I'm only creating an interface between TikZ and the user of my package.

The last example is very complex and it is to show you all that we can do with \pkg{tkz-euclide}.


\begin{tkzexample}[vbox,small]
\begin{tikzpicture}[scale=.6]
\tkzDefPoints{0/0/A,6/0/B,0.8/4/C}
\tkzDefTriangleCenter[euler](A,B,C)        \tkzGetPoint{N} 
\tkzDefTriangleCenter[circum](A,B,C)       \tkzGetPoint{O} 
\tkzDefTriangleCenter[lemoine](A,B,C)      \tkzGetPoint{K}
\tkzDefTriangleCenter[ortho](A,B,C)        \tkzGetPoint{H}
\tkzDefSpcTriangle[excentral,name=J](A,B,C){a,b,c}
\tkzDefSpcTriangle[centroid,name=M](A,B,C){a,b,c}
\tkzDefCircle[in](Ma,Mb,Mc)                \tkzGetPoint{Sp}  % Sp Spieker center
\tkzDefProjExcenter[name=J](A,B,C)(a,b,c){Y,Z,X}
\tkzDefLine[parallel=through Za](A,B)      \tkzGetPoint{Xc}
\tkzInterLL(Za,Xc)(C,B)                    \tkzGetPoint{C'}
\tkzDefLine[parallel=through Zc](B,C)      \tkzGetPoint{Ya}
\tkzInterLL(Zc,Ya)(A,B)                    \tkzGetPoint{A'}
\tkzDefPointBy[reflection= over Ja--Jc](C')\tkzGetPoint{Ab}
\tkzDefPointBy[reflection= over Ja--Jc](A')\tkzGetPoint{Cb}
\tkzInterLL(K,O)(N,Sp)                     \tkzGetPoint{Q}
\tkzInterLC(A,B)(Q,Cb)                     \tkzGetFirstPoint{Ba}
\tkzInterLC(A,C)(Q,Cb)                     \tkzGetPoints{Ac}{Ca}
\tkzInterLC(B,C')(Q,Cb)                    \tkzGetFirstPoint{Bc}
\tkzInterLC[next to=Ja](Ja,Q)(Q,Cb)        \tkzGetFirstPoint{F'a}
\tkzInterLC[next to=Jc](Jc,Q)(Q,Cb)        \tkzGetFirstPoint{F'c}
\tkzInterLC[next to=Jb](Jb,Q)(Q,Cb)        \tkzGetFirstPoint{F'b}
\tkzInterLC[common=F'a](Sp,F'a)(Ja,F'a)    \tkzGetFirstPoint{Fa}
\tkzInterLC[common=F'b](Sp,F'b)(Jb,F'b)    \tkzGetFirstPoint{Fb}
\tkzInterLC[common=F'c](Sp,F'c)(Jc,F'c)    \tkzGetFirstPoint{Fc}
\tkzInterLC(Mc,Sp)(Q,Cb)                   \tkzGetFirstPoint{A''}
\tkzDefCircle[euler](A,B,C)                \tkzGetPoints{E}{e}
\tkzDefCircle[ex](C,A,B)                   \tkzGetPoints{Ea}{a}
\tkzDefCircle[ex](A,B,C)                   \tkzGetPoints{Eb}{b}
\tkzDefCircle[ex](B,C,A)                   \tkzGetPoints{Ec}{c}
% Calculations are done, now you can draw, mark and label
\tkzDrawCircles(Q,Cb E,e)%
\tkzDrawCircles(Eb,b Ea,a Ec,c)
\tkzDrawPolygon(A,B,C)
\tkzDrawSegments[dashed](A,A' C,C' A',Zc Za,C' B,Cb B,Ab A,Ca)
\tkzDrawSegments[dashed](C,Ac Ja,Xa Jb,Yb Jc,Zc)
\begin{scope}
   \tkzClipCircle(Q,Cb) % We limit the drawing of the lines
   \tkzDrawLine[add=5 and 12,orange](K,O)
   \tkzDrawLine[add=12 and 28,red!50!black](N,Sp)
\end{scope}
\tkzDrawPoints(A,B,C,K,Ja,Jb,Jc,Q,N,O,Sp,Mc,Xa,Xb,Yb,Yc,Za,Zc)
\tkzDrawPoints(A',C',A'',Ab,Cb,Bc,Ca,Ac,Ba,Fa,Fb,Fc,F'a,F'b,F'c)
\tkzLabelPoints(Ja,Jb,Jc,Q,Xa,Xb,Za,Zc,Ab,Cb,Bc,Ca,Ac,Ba,F'b)
\tkzLabelPoints[above](O,K,F'a,Fa,A'')
\tkzLabelPoints[below](B,F'c,Yc,N,Sp,Fc,Mc)
\tkzLabelPoints[left](A',C',Fb)
\tkzLabelPoints[right](C)
\tkzLabelPoints[below right](A)
\tkzLabelPoints[above right](Yb)
\tkzDrawSegments(Fc,F'c Fb,F'b Fa,F'a)
\tkzDrawSegments[color=green!50!black](Mc,N Mc,A'' A'',Q)
\tkzDrawSegments[color=red,dashed](Ac,Ab Ca,Cb Ba,Bc Ja,Jc A',Cb C',Ab)
\tkzDrawSegments[color=red](Cb,Ab Bc,Ac Ba,Ca A',C')
\tkzMarkSegments[color=red,mark=|](Cb,Ab Bc,Ac Ba,Ca)
\tkzMarkRightAngles(Jc,Zc,A Ja,Xa,B Jb,Yb,C)
\tkzDrawSegments[green,dashed](A,F'a B,F'b C,F'c)
\end{tikzpicture}
\end{tkzexample}

\endinput

\part{FAQ}
\section{FAQ} 

\subsection{Most common errors}
 For the moment, I'm basing myself on my own, because having changed syntax several times, I've made a number of mistakes. This section is going to be expanded.
 
\begin{itemize}\setlength{\itemsep}{10pt}
\item \tkzcname{tkzDrawPoint(A,B)} when you need  \tkzcname{tkzDrawPoints}.

\item  \tkzcname{tkzGetPoint(A)} When defining an object, use braces and not brackets, so write: \tkzcname{tkzGetPoint\{A\}}.
  
\item \tkzcname{tkzGetPoint\{A\}} in place of \tkzcname{tkzGetFirstPoint\{A\}}. When a macro gives two points as results, either we retrieve these points using \tkzcname{tkzGetPoints\{A\}\{B\}}, or we retrieve only one of the two points, using \tkzcname{tkzGetFirstPoint\{A\}} or 
\tkzcname{tkzGetSecondPoint\{A\}}. These two points can be used with the reference \tkzname{tkzFirstPointResult} or 
\tkzname{tkzSecondPointResult}. It is possible that a third point is given as \tkzname{tkzPointResult}.  
     
\item \tkzcname{tkzDrawSegment(A,B A,C)} when you need \tkzcname{tkzDrawSegments}. It is possible to use only the versions with an "s" but it is less efficient!

\item Mixing options and arguments; all macros that use a circle need to know the radius of the circle. If the radius is given by a measure then the option includes a \tkzname{R}.

\item  \tkzcname{tkzDrawSegments[color = gray,style=dashed]\{B,B' C,C'\}} is a mistake. Only macros that define an object use braces.   

\item The angles are given in degrees, more rarely in radians.  
\item If an error occurs in a calculation when passing parameters, then it is better to make these calculations before calling the macro.
 
\item Do not mix the syntax of \tkzNamePack{pgfmath} and \tkzNamePack{xfp}. I've often chosen \tkzNamePack{xfp} but if you prefer pgfmath then do your calculations before passing parameters.

\item Use of \tkzcname{tkzClip}: In order to get accurate results, I avoided using normalized vectors. The advantage of normalization is to control the dimension of the manipulated objects, the disadvantage is that with TeX, this implies inaccuracies. These inaccuracies are often small, in the order of a thousandth, but they lead to disasters if the drawing is enlarged. Not normalizing implies that some points are far away from the working area and \tkzcname{tkzClip} allows you to reduce the size of the drawing. 

\item  An error occurs if you use the macro \tkzcname{tkzDrawAngle}
 with too small an angle. The error is produced by the \NameLib{decoration} library when you want to place a mark on an arc. Even if the mark is absent, the error is still present. It is possible to get around this difficulty with the option \tkzname{mkpos=.2} for example, which will place the mark before the arc. Another possibility is to use the macro \tkzcname{tkzFillAngle}.

\end{itemize}    
\endinput

\clearpage\newpage
\small\printindex
\end{document}