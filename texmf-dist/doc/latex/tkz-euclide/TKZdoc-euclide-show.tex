\section{The Show}

\subsection{Show the constructions of some lines \tkzcname{tkzShowLine}}

 \begin{NewMacroBox}{tkzShowLine}{\oarg{local options}\parg{pt1,pt2} or \parg{pt1,pt2,pt3}}%
These constructions concern mediatrices, perpendicular or parallel lines passing through a given point and bisectors. The arguments are therefore lists of two or three points. Several options allow the adjustment of the constructions. The idea of this macro comes from \tkzimp{Yves Combe}.
  

\medskip 
\begin{tabular}{lll}%
\toprule
options      & default & definition    \\ 
\midrule
\TOline{mediator}{mediator}{displays the constructions of a mediator} 
\TOline{perpendicular}{mediator}{constructions for a perpendicular} 
\TOline{orthogonal}{mediator}{idem}
\TOline{bisector}{mediator}{constructions for a bisector}
\TOline{K}{1}{circle within a triangle }
\TOline{length}{1}{in cm, length of a arc}
\TOline{ratio} {.5}{arc length ratio}
\TOline{gap}{2}{placing the point of construction}
\TOline{size}{1}{radius of an arc (see bisector)}
 \bottomrule
\end{tabular}

\medskip
You have to add, of course, all the styles of \TIKZ\ for tracings\dots
\end{NewMacroBox}

\subsubsection{Example of \tkzcname{tkzShowLine} and \tkzname{parallel}} 
\begin{tkzexample}[latex=7cm,small]
\begin{tikzpicture}
 \tkzDefPoints{-1.5/-0.25/A,1/-0.75/B,-1.5/2/C}
 \tkzDrawLine(A,B)
 \tkzDefLine[parallel=through C](A,B) \tkzGetPoint{c} 
 \tkzShowLine[parallel=through C](A,B)
 \tkzDrawLine(C,c) \tkzDrawPoints(A,B,C,c)
\end{tikzpicture}
\end{tkzexample}

\subsubsection{Example of \tkzcname{tkzShowLine} and \tkzname{perpendicular}} 
\begin{tkzexample}[latex=5cm,small]
\begin{tikzpicture}
\tkzDefPoints{0/0/A, 3/2/B, 2/2/C} 
\tkzDefLine[perpendicular=through C,K=-.5](A,B) \tkzGetPoint{c}
\tkzShowLine[perpendicular=through C,K=-.5,gap=3](A,B)
\tkzDefPointBy[projection=onto A--B](c)\tkzGetPoint{h} 
\tkzMarkRightAngle[fill=lightgray](A,h,C) 
\tkzDrawLines[add=.5 and .5](A,B C,c) 
\tkzDrawPoints(A,B,C,h,c)
\end{tikzpicture}
\end{tkzexample}

\subsubsection{Example of \tkzcname{tkzShowLine} and \tkzname{bisector}} 
\begin{tkzexample}[latex=7 cm,small]
\begin{tikzpicture}[scale=1.25]
 \tkzDefPoints{0/0/A, 4/2/B, 1/4/C}
 \tkzDrawPolygon(A,B,C) 
 \tkzSetUpCompass[color=brown,line width=.1 pt]
 \tkzDefLine[bisector](B,A,C)  \tkzGetPoint{a}
 \tkzDefLine[bisector](C,B,A)  \tkzGetPoint{b}
 \tkzInterLL(A,a)(B,b) \tkzGetPoint{I}
 \tkzDefPointBy[projection = onto A--B](I) 
   \tkzGetPoint{H}
 \tkzShowLine[bisector,size=2,gap=3,blue](B,A,C)
 \tkzShowLine[bisector,size=2,gap=3,blue](C,B,A)   
 \tkzDrawCircle[color=blue,%
 line width=.2pt](I,H) 
 \tkzDrawSegments[color=red!50](I,tkzPointResult)
 \tkzDrawLines[add=0 and -0.3,color=red!50](A,a B,b) 
\end{tikzpicture}
\end{tkzexample}

\subsubsection{Example of \tkzcname{tkzShowLine} and \tkzname{mediator}} 
\begin{tkzexample}[latex=7 cm,small]
\begin{tikzpicture}
\tkzDefPoint(2,2){A}
\tkzDefPoint(5,4){B}
\tkzDrawPoints(A,B) 
\tkzShowLine[mediator,color=orange,length=1](A,B) 
\tkzGetPoints{i}{j}
\tkzDrawLines[add=-0.1 and -0.1](i,j)
\tkzDrawLines(A,B)
\tkzLabelPoints[below =3pt](A,B)
\end{tikzpicture}
\end{tkzexample}

\subsection{Constructions of certain transformations \addbs{tkzShowTransformation}}
\begin{NewMacroBox}{tkzShowTransformation}{\oarg{local options}\parg{pt1,pt2} or \parg{pt1,pt2,pt3}}%
These constructions concern orthogonal symmetries, central symmetries, orthogonal projections and translations. Several options allow the adjustment of the constructions. The idea of this macro comes from \tkzimp{Yves Combe}.
  
\medskip 
\begin{tabular}{lll}%
\toprule
options             & default & definition                        \\ 
\midrule
\TOline{reflection= over pt1--pt2}{reflection}{constructions of orthogonal symmetry} 
\TOline{symmetry=center pt}{reflection}{constructions of central symmetry} 
\TOline{projection=onto pt1--pt2}{reflection}{constructions of a projection}
\TOline{translation=from pt1 to pt2}{reflection}{constructions of a translation}
\TOline{K}{1}{circle within a triangle }
\TOline{length}{1}{arc length}
\TOline{ratio} {.5}{arc length ratio}
\TOline{gap}{2}{placing the point of construction}
\TOline{size}{1}{radius of an arc (see bisector)}
\end{tabular}
\end{NewMacroBox}

\subsubsection{Example of the use of \tkzcname{tkzShowTransformation}} 

\begin{tkzexample}[latex=6cm,small]
\begin{tikzpicture}[scale=.6]
  \tkzDefPoint(0,0){O} \tkzDefPoint(2,-2){A}
  \tkzDefPoint(70:4){B} \tkzDrawPoints(A,O,B)
  \tkzLabelPoints(A,O,B)
  \tkzDrawLine[add= 2 and 2](O,A)
  \tkzDefPointBy[translation=from O to A](B) 
  \tkzGetPoint{C}
  \tkzDrawPoint[color=orange](C)  \tkzLabelPoints(C)
  \tkzShowTransformation[translation=from O to A,%
             length=2](B) 
  \tkzDrawSegments[->,color=orange](O,A B,C)  
  \tkzDefPointBy[reflection=over O--A](B) \tkzGetPoint{E}
  \tkzDrawSegment[blue](B,E)
  \tkzDrawPoint[color=blue](E)\tkzLabelPoints(E) 
  \tkzShowTransformation[reflection=over O--A,size=2](B)   
  \tkzDefPointBy[symmetry=center O](B) \tkzGetPoint{F} 
  \tkzDrawSegment[color=green](B,F)
  \tkzDrawPoint[color=green](F)\tkzLabelPoints(F)
  \tkzShowTransformation[symmetry=center O,%
                      length=2](B) 
  \tkzDefPointBy[projection=onto O--A](C) 
  \tkzGetPoint{H}    
  \tkzDrawSegments[color=magenta](C,H)
  \tkzDrawPoint[color=magenta](H)\tkzLabelPoints(H)
  \tkzShowTransformation[projection=onto O--A,%
                         color=red,size=3,gap=-2](C)   
\end{tikzpicture}
\end{tkzexample}

\subsubsection{Another example of the use of \tkzcname{tkzShowTransformation}} 

You'll find this figure again, but without the construction features.
\begin{tkzexample}[latex=7cm,small]  
\begin{tikzpicture}[scale=.6]
  \tkzDefPoints{0/0/A,8/0/B,3.5/10/I}
  \tkzDefMidPoint(A,B) \tkzGetPoint{O} 
  \tkzDefPointBy[projection=onto A--B](I) 
     \tkzGetPoint{J}
  \tkzInterLC(I,A)(O,A)  \tkzGetPoints{M}{M'}
  \tkzInterLC(I,B)(O,A)  \tkzGetPoints{N}{N'}
  \tkzDefMidPoint(A,B) \tkzGetPoint{M}    
  \tkzDrawSemiCircle(M,B)
  \tkzDrawSegments(I,A I,B A,B B,M A,N) 
  \tkzMarkRightAngles(A,M,B A,N,B)  
  \tkzDrawSegment[style=dashed,color=blue](I,J)
  \tkzShowTransformation[projection=onto A--B,
                  color=red,size=3,gap=-3](I)
  \tkzDrawPoints[color=red](M,N)
  \tkzDrawPoints[color=blue](O,A,B,I,M) 
  \tkzLabelPoints(O)  
  \tkzLabelPoints[above right](N,I) 
  \tkzLabelPoints[below left](M,A) 
\end{tikzpicture} 
\end{tkzexample} 

\endinput