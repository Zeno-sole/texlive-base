\section{Triangles}

\subsection{Definition of triangles \tkzcname{tkzDefTriangle}}
The following macros will allow you to define or construct a triangle from \tkzname{at least} two points.

 At the moment, it is possible to define the following triangles:
 \begin{itemize}
\item  \tkzname{two angles}  determines a triangle with two angles;
\item  \tkzname{equilateral}  determines an equilateral triangle;
\item  \tkzname{isosceles right}  determines an isoxsceles right triangle;
\item \tkzname{half} determines a right-angled triangle such that the ratio of the measurements of the two adjacent sides to the right angle is equal to $2$;
\item \tkzname{pythagore} determines a right-angled triangle whose side measurements are proportional to 3, 4 and 5;
\item \tkzname{school} determines a right-angled triangle whose angles are 30, 60 and 90 degrees;
\item \tkzname{golden} determines a right-angled triangle such that the ratio of the measurements on the two adjacent sides to the right angle is equal to $\Phi=1.618034$, I chose "golden triangle" as the denomination because it comes from the golden rectangle and I kept the denomination "gold triangle" or "Euclid's triangle" for the isosceles triangle whose angles at the base are 72 degrees;

\item  \tkzname{euclid} or \tkzname{gold} for the gold triangle; in the previous version the option was "euclide" with an "e".

\item \tkzname{cheops} determines a third point such that the triangle is isosceles with side measurements proportional to $2$, $\Phi$ and $\Phi$.
\end{itemize}

\newpage
\begin{NewMacroBox}{tkzDefTriangle}{\oarg{local options}\parg{A,B}}%
The points are ordered because the triangle is constructed following the direct direction of the trigonometric circle. This macro is either used in partnership with \tkzcname{tkzGetPoint} or by using \tkzname{tkzPointResult} if it is not necessary to keep the name.

\medskip
\begin{tabular}{lll}%
\toprule
options             & default & definition                        \\
\midrule
\TOline{two angles= \#1 and \#2}{no defaut}{triangle knowing two angles}
\TOline{equilateral} {equilateral}{equilateral triangle }
\TOline{half} {equilateral}{B rectangle  $AB=2BC$ $AC$ hypothenuse }
\TOline{isosceles right} {equilateral}{isosceles right triangle }
\TOline{pythagore}{equilateral}{proportional to the pythagorean triangle 3-4-5}
\TOline{pythagoras}{equilateral}{same as above}
\TOline{egyptian}{equilateral}{same as above}
\TOline{school} {equilateral}{angles of 30, 60 and 90 degrees }
\TOline{gold}{equilateral}{B rectangle and $AB/AC = \Phi$}
\TOline{euclid} {equilateral}{angles of 72, 72 and 36 degrees, $A$ is the apex}
\TOline{golden} {equilateral}{angles of 72, 72 and 36 degrees, $C$ is the apex}
\TOline{sublime} {equilateral}{angles of 72, 72 and 36 degrees, $C$ is the apex}
\TOline{cheops} {equilateral}{AC=BC, AC and BC are proportional to $2$ and $\Phi$.}
\TOline{swap} {false}{gives the symmetric point with respect to $AB$}
\bottomrule
\end{tabular}

\medskip
\emph{\tkzcname{tkzGetPoint} allows you to store the point otherwise \tkzname{tkzPointResult} allows for immediate use.}
\end{NewMacroBox}

\subsubsection{Option \tkzname{equilateral}}
\begin{tkzexample}[latex=7 cm,small]
\begin{tikzpicture}
  \tkzDefPoint(0,0){A}
  \tkzDefPoint(4,0){B}
  \tkzDefTriangle[equilateral](A,B)
  \tkzGetPoint{C}
  \tkzDrawPolygons(A,B,C)
  \tkzDefTriangle[equilateral](B,A)
  \tkzGetPoint{D}
  \tkzDrawPolygon(B,A,D)
  \tkzMarkSegments[mark=s|](A,B B,C A,C A,D B,D)
\end{tikzpicture}
\end{tkzexample}


\subsubsection{Option \tkzname{two angles}}
\begin{tkzexample}[latex=6 cm,small]
\begin{tikzpicture}
\tkzDefPoint(0,0){A} 
\tkzDefPoint(5,0){B} 
\tkzDefTriangle[two angles = 50 and 70](A,B)
\tkzGetPoint{C} 
\tkzDrawSegment(A,B) 
\tkzDrawPoints(A,B) 
\tkzLabelPoints(A,B) 
\tkzDrawSegments[new](A,C B,C) 
\tkzDrawPoints[new](C)
\tkzLabelPoints[above,new](C)
\tkzLabelAngle[pos=1.4](B,A,C){$50^\circ$}
\tkzLabelAngle[pos=0.8](C,B,A){$70^\circ$}
\end{tikzpicture}
\end{tkzexample}

\subsubsection{Option \tkzname{school}}
The angles are 30, 60 and 90 degrees.

\begin{tkzexample}[latex=6 cm,small]
\begin{tikzpicture}
  \tkzDefPoints{0/0/A,4/0/B}
  \tkzDefTriangle[school](A,B)  
  \tkzGetPoint{C}
  \tkzMarkRightAngles(C,B,A)
  \tkzLabelAngle[pos=0.8](B,A,C){$30^\circ$}
  \tkzLabelAngle[pos=0.8](C,B,A){$90^\circ$}
  \tkzLabelAngle[pos=0.8](A,C,B){$60^\circ$} 
  \tkzDrawSegments(A,B)
  \tkzDrawSegments[new](A,C B,C)
  \tkzLabelPoints(A,B)
  \tkzLabelPoints[above](C)
\end{tikzpicture}
\end{tkzexample}

\subsubsection{Option \tkzname{pythagore}}
This triangle has sides whose lengths are proportional to 3, 4 and 5.

\begin{tkzexample}[latex=6 cm,small]
\begin{tikzpicture} 
  \tkzDefPoints{0/0/A,4/0/B} 
  \tkzDefTriangle[pythagore](A,B) 
  \tkzGetPoint{C} 
  \tkzDrawSegments(A,B)
  \tkzDrawSegments[new](A,C B,C)
  \tkzMarkRightAngles(A,B,C)
  \tkzDrawPoints[new](C) 
  \tkzDrawPoints(A,B) 
  \tkzLabelPoints[above](A,B)
  \tkzLabelPoints[new](C)  
\end{tikzpicture}
\end{tkzexample}

\subsubsection{Option \tkzname{pythagore} and \tkzname{swap}}
This triangle has sides whose lengths are proportional to 3, 4 and 5.

\begin{tkzexample}[latex=6 cm,small]
\begin{tikzpicture} 
  \tkzDefPoints{0/0/A,4/0/B} 
  \tkzDefTriangle[pythagore,swap](A,B) 
  \tkzGetPoint{C} 
  \tkzDrawSegments(A,B)
  \tkzDrawSegments[new](A,C B,C)
  \tkzMarkRightAngles(A,B,C)
  \tkzLabelPoint[above,new](C){$C$} 
  \tkzDrawPoints[new](C) 
  \tkzDrawPoints(A,B) 
  \tkzLabelPoints(A,B) 
\end{tikzpicture}
\end{tkzexample}

\subsubsection{Option \tkzname{golden}}
\begin{tkzexample}[latex=6 cm,small]
\begin{tikzpicture}[scale=.8]
\tkzDefPoint(0,0){A} \tkzDefPoint(4,0){B} 
\tkzDefTriangle[golden](A,B)\tkzGetPoint{C} 
\tkzDefSpcTriangle[in,name=M](A,B,C){a,b,c}
\tkzDrawPolygon(A,B,C) 
\tkzDrawPoints(A,B) 
\tkzDrawSegment(C,Mc) 
\tkzDrawPoints[new](C)
\tkzLabelPoints(A,B) 
\tkzLabelPoints[above,new](C)
\end{tikzpicture}
\end{tkzexample}

\subsubsection{Option \tkzname{euclid}}
\tkzimp{Euclid} and \tkzimp{golden} are identical but the segment AB is a base in one and a side in the other. 

\begin{tkzexample}[latex=7 cm,small]
\begin{tikzpicture}[scale=.75]
 \tkzDefPoint(0,0){A} \tkzDefPoint(4,0){B}
 \tkzDefTriangle[euclid](A,B)\tkzGetPoint{C}
 \tkzDrawPolygon(A,B,C)
 \tkzDrawPoints(A,B,C)
 \tkzLabelPoints(C)
 \tkzLabelPoints[above](A,B)
 \tkzLabelAngle[pos=0.8](A,B,C){$72^\circ$}
 \tkzLabelAngle[pos=0.8](B,C,A){$72^\circ$}
 \tkzLabelAngle[pos=0.8](C,A,B){$36^\circ$}
\end{tikzpicture}
\end{tkzexample}

\subsubsection{Option \tkzname{isosceles right}}
\begin{tkzexample}[latex=7 cm,small]
\begin{tikzpicture}
  \tkzDefPoint(0,0){A}
  \tkzDefPoint(4,0){B}
  \tkzDefTriangle[isosceles right](A,B)
  \tkzGetPoint{C}
  \tkzDrawPolygons(A,B,C)
  \tkzDrawPoints(A,B,C)
  \tkzMarkRightAngles(A,C,B)
  \tkzLabelPoints(A,B)
  \tkzLabelPoints[above](C)
\end{tikzpicture}
\end{tkzexample}

\subsubsection{Option \tkzname{gold} }
\begin{tkzexample}[latex=6 cm,small]
\begin{tikzpicture}
 \tkzDefPoints{0/0/A,4/0/B} 
 \tkzDefTriangle[gold](A,B)
 \tkzGetPoint{C}
 \tkzDrawPolygon(A,B,C)
 \tkzDrawPoints(A,B,C)
 \tkzLabelPoints[above](A,B) 
 \tkzLabelPoints[below](C)
 \tkzMarkRightAngle(A,B,C)
 \tkzText(0,-2){$\dfrac{AC}{AB}=\varphi$}
\end{tikzpicture}
\end{tkzexample}

\clearpage
\subsection{Specific triangles with \tkzcname{tkzDefSpcTriangle}}

The centers of some triangles have been defined in the "points" section, here it is a question of determining the three vertices of specific triangles.

\begin{NewMacroBox}{tkzDefSpcTriangle}{\oarg{local options}\parg{p1,p2,p3}\marg{r1,r2,r3}}
The order of the points is important! p1p2p3 defines a triangle then the result is a triangle whose vertices have as reference a combination with \tkzname{name} and r1,r2, r3. If \tkzname{name} is empty then the references are  r1,r2 and r3.

\medskip
\begin{tabular}{lll}%
\toprule
options             & default & definition                        \\
\midrule
\TOline{orthic} {centroid}{determined by endpoints of the altitudes ...}
\TOline{centroid or medial}{centroid}{intersection of the triangle's three triangle medians}
\TOline{in or incentral}{centroid}{determined with the angle bisectors}
\TOline{ex or excentral} {centroid}{determined with the excenters}
\TOline{extouch}{centroid}{formed by the points of tangency  with the excircles}
\TOline{intouch or contact} {centroid}{formed by the points of tangency of the incircle}
\TOline{} {}{each of the vertices}
\TOline{euler} {centroid}{formed by Euler points on the  nine-point circle}
\TOline{symmedial} {centroid}{intersection points of the symmedians}
\TOline{tangential}{centroid}{formed by the lines tangent to the circumcircle}
\TOline{feuerbach} {centroid}{formed by the points of tangency of the nine-point ...}
\TOline{} {} {circle with the excircles}
\TOline{name} {empty}{used to name the vertices}
\midrule
\end{tabular}
\end{NewMacroBox}

\subsubsection{How to name the vertices}

With \tkzcname{tkzDefSpcTriangle[medial,name=M](A,B,C)\{\_A,\_B,\_C\}} you get three vertices named $M_A$, $M_B$ and $M_C$.

With \tkzcname{tkzDefSpcTriangle[medial](A,B,C)\{a,b,c\}} you get three vertices named and labeled $a$, $b$ and $c$.

Possible \tkzcname{tkzDefSpcTriangle[medial,name=M\_](A,B,C)\{A,B,C\}} you get three vertices named $M_A$, $M_B$ and $M_C$.

\subsection{Option \tkzname{medial} or \tkzname{centroid} }
The geometric centroid  of the polygon vertices of a triangle is the point $G$ (sometimes also denoted $M$) which is also the intersection of the triangle's three triangle medians. The point is therefore sometimes called the median point. The centroid is always in the interior of the triangle.
\\

\href{http://mathworld.wolfram.com/TriangleCentroid.html}{Weisstein, Eric W. "Centroid triangle" From MathWorld--A Wolfram Web Resource.}

In the following example, we obtain the Euler circle which passes through the previously defined points.

\begin{tkzexample}[latex=7cm,small]
  \begin{tikzpicture}[rotate=90,scale=.75]
   \tkzDefPoints{0/0/A,6/0/B,0.8/4/C}
   \tkzDefTriangleCenter[centroid](A,B,C)
   \tkzGetPoint{M}
   \tkzDefSpcTriangle[medial,name=M](A,B,C){_A,_B,_C}
   \tkzDrawPolygon(A,B,C)
   \tkzDrawSegments[dashed,new](A,M_A B,M_B C,M_C)
   \tkzDrawPolygon[new](M_A,M_B,M_C)
   \tkzDrawPoints(A,B,C)
   \tkzDrawPoints[new](M,M_A,M_B,M_C)
   \tkzLabelPoints[above](B)
   \tkzLabelPoints[below](A,C,M_B)
   \tkzLabelPoints[right](M_C)
   \tkzLabelPoints[left](M_A)
   \tkzLabelPoints[font=\scriptsize](M)
  \end{tikzpicture}
\end{tkzexample}

\subsubsection{Option \tkzname{in} or \tkzname{incentral} }

The incentral triangle is the triangle whose vertices are determined by
the intersections of the reference triangle’s angle bisectors with the
respective opposite sides.
\\
\href{http://mathworld.wolfram.com/ContactTriangle.html}{Weisstein, Eric W. "Incentral triangle" From MathWorld--A Wolfram Web Resource.}


\begin{tkzexample}[latex=7cm,small]
\begin{tikzpicture}[scale=1]
  \tkzDefPoints{ 0/0/A,5/0/B,2/3/C}
  \tkzDefSpcTriangle[in,name=I](A,B,C){_a,_b,_c}
  \tkzDefCircle[in](A,B,C) \tkzGetPoints{I}{a}
  \tkzDrawCircle(I,a)
  \tkzDrawPolygon(A,B,C)
  \tkzDrawPolygon[new](I_a,I_b,I_c)
  \tkzDrawSegments[dashed,new](A,I_a B,I_b C,I_c)
  \tkzDrawPoints(A,B,C,I,I_a,I_b,I_c) 
  \tkzLabelPoints[below](A,B,I_c)
  \tkzLabelPoints[above left](I_b)
  \tkzLabelPoints[above right](C,I_a)
\end{tikzpicture}
\end{tkzexample}

\subsubsection{Option \tkzname{ex} or \tkzname{excentral} }

The excentral triangle of a triangle $ABC$ is the triangle $J_aJ_bJ_c$ with vertices corresponding to the excenters of $ABC$.

\begin{tkzexample}[latex=7cm,small]
\begin{tikzpicture}[scale=.6]
 \tkzDefPoints{0/0/A,6/0/B,0.8/4/C}
 \tkzDefSpcTriangle[excentral,name=J](A,B,C){_a,_b,_c}
 \tkzDefSpcTriangle[extouch,name=T](A,B,C){_a,_b,_c}
 \tkzDrawPolygon(A,B,C)
 \tkzDrawPolygon[new](J_a,J_b,J_c)
 \tkzClipBB
 \tkzDrawPoints(A,B,C)
 \tkzDrawPoints[new](J_a,J_b,J_c)
 \tkzLabelPoints(A,B,C)
 \tkzLabelPoints[new](J_b,J_c)
 \tkzLabelPoints[new,above](J_a)
 \tkzDrawCircles[gray](J_a,T_a J_b,T_b J_c,T_c) 
\end{tikzpicture}
\end{tkzexample}


\subsubsection{Option \tkzname{intouch} or \tkzname{contact}}
The contact triangle of a triangle $ABC$, also called the intouch triangle, is the triangle  formed by the points of tangency of the incircle of $ABC$ with $ABC$.\\
\href{http://mathworld.wolfram.com/ContactTriangle.html}{Weisstein, Eric W. "Contact triangle" From MathWorld--A Wolfram Web Resource.}

We obtain the intersections of the bisectors with the sides.
\begin{tkzexample}[latex=7cm,small]
\begin{tikzpicture}[scale=.75]
 \tkzDefPoints{0/0/A,6/0/B,0.8/4/C}          
 \tkzDefSpcTriangle[intouch,name=X](A,B,C){_a,_b,_c}
 \tkzInCenter(A,B,C)\tkzGetPoint{I}
 \tkzDefCircle[in](A,B,C) \tkzGetPoints{I}{i}
 \tkzDrawCircle(I,i)
 \tkzDrawPolygon(A,B,C)
 \tkzDrawPolygon[new](X_a,X_b,X_c)
 \tkzDrawPoints(A,B,C)
 \tkzDrawPoints[new](X_a,X_b,X_c)
 \tkzLabelPoints[right](X_a)
 \tkzLabelPoints[left](X_b)
 \tkzLabelPoints[above](C)
 \tkzLabelPoints[below](A,B,X_c)
\end{tikzpicture} 
\end{tkzexample}

\subsubsection{Option \tkzname{extouch}}
The extouch triangle  $T_aT_bT_c$ is the triangle formed by the points of tangency of a triangle $ABC$ with its excircles $J_a$, $J_b$, and $J_c$. The points  $T_a$, $T_b$, and $T_c$ can also be constructed as the points which bisect the perimeter of $A_1A_2A_3$ starting at $A$, $B$, and $C$.\\
\href{http://mathworld.wolfram.com/ExtouchTriangle.html}{Weisstein, Eric W. "Extouch triangle" From MathWorld--A Wolfram Web Resource.}

We obtain the points of contact of the exinscribed circles as well as the triangle formed by the centers of the exinscribed circles.

\begin{tkzexample}[latex=8cm,small]
\begin{tikzpicture}[scale=.7]
\tkzDefPoints{0/0/A,6/0/B,0.8/4/C}
\tkzDefSpcTriangle[excentral,
                 name=J](A,B,C){_a,_b,_c}
\tkzDefSpcTriangle[extouch,
                  name=T](A,B,C){_a,_b,_c}
\tkzDefTriangleCenter[nagel](A,B,C)
\tkzGetPoint{N_a}
\tkzDefTriangleCenter[centroid](A,B,C)
\tkzGetPoint{G}
\tkzDrawPoints[new](J_a,J_b,J_c)
\tkzClipBB \tkzShowBB
\tkzDrawCircles[gray](J_a,T_a J_b,T_b J_c,T_c)
\tkzDrawLines[add=1 and 1](A,B B,C C,A)
\tkzDrawSegments[new](A,T_a B,T_b C,T_c)
\tkzDrawSegments[new](J_a,T_a J_b,T_b J_c,T_c)
\tkzDrawPolygon(A,B,C)
\tkzDrawPolygon[new](T_a,T_b,T_c)
\tkzDrawPoints(A,B,C,N_a)
\tkzDrawPoints[new](T_a,T_b,T_c)
\tkzLabelPoints[below left](A)
\tkzLabelPoints[below](N_a,B)
\tkzLabelPoints[above](C)
\tkzLabelPoints[new,below left](T_b)
\tkzLabelPoints[new,below right](T_c)
\tkzLabelPoints[new,right=6pt](T_a)
\tkzMarkRightAngles[fill=gray!15](J_a,T_a,B
 J_b,T_b,C J_c,T_c,A)
\end{tikzpicture}
\end{tkzexample}

\subsubsection{Option \tkzname{orthic}}

Given a triangle $ABC$, the triangle $H_AH_BH_C$ whose vertices are endpoints of the altitudes from each of the vertices of ABC is called the orthic triangle, or sometimes the altitude triangle. The three lines $AH_A$, $BH_B$, and $CH_C$ are concurrent at the orthocenter H of ABC.

\begin{tkzexample}[latex=7cm,small]
\begin{tikzpicture}[scale=.75]
\tkzDefPoints{1/5/A,0/0/B,7/0/C}
 \tkzDefSpcTriangle[orthic](A,B,C){H_A,H_B,H_C}
 \tkzDefTriangleCenter[ortho](B,C,A)
 \tkzGetPoint{H}
 \tkzDefPointWith[orthogonal,normed](H_A,B)
 \tkzGetPoint{a}
 \tkzDrawSegments[new](A,H_A B,H_B C,H_C)   
 \tkzMarkRightAngles[fill=gray!20,
         opacity=.5](A,H_A,C B,H_B,A C,H_C,A)
 \tkzDrawPolygon[fill=teal!20,opacity=.3](A,B,C)
 \tkzDrawPoints(A,B,C)
 \tkzDrawPoints[new](H_A,H_B,H_C)
 \tkzDrawPolygon[new,fill=orange!20,
                opacity=.3](H_A,H_B,H_C)
 \tkzLabelPoints(C)
 \tkzLabelPoints[left](B)
 \tkzLabelPoints[above](A)
 \tkzLabelPoints[new](H_A)
 \tkzLabelPoints[new,above left](H_C)
 \tkzLabelPoints[new,above right](H_B,H)
\end{tikzpicture}
\end{tkzexample}
    
\subsubsection{Option \tkzname{feuerbach}}
The Feuerbach triangle is the triangle formed by the three points of tangency of the nine-point circle with the excircles.\\
\href{http://mathworld.wolfram.com/FeuerbachTriangle.html}{Weisstein, Eric W. "Feuerbach triangle" From MathWorld--A Wolfram Web Resource.}

 The points of tangency define the Feuerbach triangle.

\begin{tkzexample}[latex=8cm,small]
\begin{tikzpicture}[scale=1]
  \tkzDefPoint(0,0){A}
  \tkzDefPoint(3,0){B}
  \tkzDefPoint(0.5,2.5){C}
  \tkzDefCircle[euler](A,B,C) \tkzGetPoint{N}
  \tkzDefSpcTriangle[feuerbach,
                       name=F](A,B,C){_a,_b,_c}
  \tkzDefSpcTriangle[excentral,
                       name=J](A,B,C){_a,_b,_c}
  \tkzDefSpcTriangle[extouch,
                        name=T](A,B,C){_a,_b,_c}
  \tkzLabelPoints[below left](J_a,J_b,J_c)  
  \tkzClipBB \tkzShowBB
  \tkzDrawCircle[purple](N,F_a)
  \tkzDrawPolygon(A,B,C)
  \tkzDrawPolygon[new](F_a,F_b,F_c)
  \tkzDrawCircles[gray](J_a,F_a J_b,F_b J_c,F_c)
  \tkzDrawPoints[blue](J_a,J_b,J_c,%
          F_a,F_b,F_c,A,B,C)
  \tkzLabelPoints(A,B,F_c)
  \tkzLabelPoints[above](C)    
  \tkzLabelPoints[right](F_a)
  \tkzLabelPoints[left](F_b)        
\end{tikzpicture}
\end{tkzexample}

\subsubsection{Option   \tkzname{tangential}} 
The tangential triangle is the triangle $T_aT_bT_c$ formed by the lines tangent to the circumcircle of a given triangle $ABC$ at its vertices. It is therefore antipedal triangle of $ABC$ with respect to the circumcenter $O$.\\ 
\href{http://mathworld.wolfram.com/TangentialTriangle.html}{Weisstein, Eric W. "Tangential Triangle." From MathWorld--A Wolfram Web Resource. }


\begin{tkzexample}[latex=8cm,small]
\begin{tikzpicture}[scale=.5,rotate=80]
  \tkzDefPoints{0/0/A,6/0/B,1.8/4/C}           
  \tkzDefSpcTriangle[tangential,
    name=T](A,B,C){_a,_b,_c}
  \tkzDrawPolygon(A,B,C)
  \tkzDrawPolygon[new](T_a,T_b,T_c)
  \tkzDrawPoints(A,B,C)
  \tkzDrawPoints[new](T_a,T_b,T_c)
  \tkzDefCircle[circum](A,B,C)  
  \tkzGetPoint{O} 
  \tkzDrawCircle(O,A)
  \tkzLabelPoints(A)
  \tkzLabelPoints[above](B)
  \tkzLabelPoints[left](C)
  \tkzLabelPoints[new](T_b,T_c)
  \tkzLabelPoints[new,left](T_a)
\end{tikzpicture} 
\end{tkzexample} 

\subsubsection{Option   \tkzname{euler}} 
The Euler triangle of a triangle $ABC$ is the triangle $E_AE_BE_C$ whose vertices are the midpoints of the segments joining the orthocenter $H$ with the respective vertices. The vertices of the triangle are known as the Euler points, and lie on the nine-point circle.
\\
\href{https://mathworld.wolfram.com/EulerTriangle.html}{Weisstein, Eric W. "Euler Triangle." From MathWorld--A Wolfram Web Resource.} 

\begin{tkzexample}[latex=7cm,small]
\begin{tikzpicture}[rotate=90,scale=1.25]
 \tkzDefPoints{0/0/A,6/0/B,0.8/4/C}
 \tkzDefSpcTriangle[medial,
     name=M](A,B,C){_A,_B,_C}
 \tkzDefTriangleCenter[euler](A,B,C)
     \tkzGetPoint{N} % I= N nine points
 \tkzDefTriangleCenter[ortho](A,B,C)
        \tkzGetPoint{H}
 \tkzDefMidPoint(A,H) \tkzGetPoint{E_A}
 \tkzDefMidPoint(C,H) \tkzGetPoint{E_C}
 \tkzDefMidPoint(B,H) \tkzGetPoint{E_B}
 \tkzDefSpcTriangle[ortho,name=H](A,B,C){_A,_B,_C}
 \tkzDrawPolygon(A,B,C)
 \tkzDrawCircle(N,E_A)
 \tkzDrawSegments[new](A,H_A B,H_B C,H_C)
 \tkzDrawPoints(A,B,C,N,H)
 \tkzDrawPoints[red](M_A,M_B,M_C)
 \tkzDrawPoints[blue]( H_A,H_B,H_C)
 \tkzDrawPoints[green](E_A,E_B,E_C)
 \tkzAutoLabelPoints[center=N,font=\scriptsize]%
(A,B,C,M_A,M_B,M_C,H_A,H_B,H_C,E_A,E_B,E_C)
\tkzLabelPoints[font=\scriptsize](H,N)
\tkzMarkSegments[mark=s|,size=3pt,
  color=blue,line width=1pt](B,E_B E_B,H)
   \tkzDrawPolygon[color=cyan](M_A,M_B,M_C)
\end{tikzpicture}
\end{tkzexample}

\subsubsection{Option  \tkzname{euler} and Option  \tkzname{orthic}} 
\begin{tkzexample}[vbox,small]
  \begin{tikzpicture}[scale=1.25]
    \tkzDefPoints{0/0/A,6/0/B,0.8/4/C}
    \tkzDefSpcTriangle[euler,name=E](A,B,C){a,b,c}
    \tkzDefSpcTriangle[orthic,name=H](A,B,C){a,b,c}
    \tkzDefExCircle(A,B,C) \tkzGetPoints{I}{i}
    \tkzDefExCircle(C,A,B) \tkzGetPoints{J}{j}
    \tkzDefExCircle(B,C,A) \tkzGetPoints{K}{k}
    \tkzDrawPoints[orange](I,J,K)
    \tkzLabelPoints[font=\scriptsize](A,B,C,I,J,K)
    \tkzClipBB
    \tkzInterLC(I,C)(I,i) \tkzGetSecondPoint{Fc}
    \tkzInterLC(J,B)(J,j) \tkzGetSecondPoint{Fb}
    \tkzInterLC(K,A)(K,k) \tkzGetSecondPoint{Fa}
    \tkzDrawLines[add=1.5 and 1.5](A,B A,C B,C)
    \tkzDefCircle[euler](A,B,C) \tkzGetPoints{E}{e}
    \tkzDrawCircle[orange](E,e)
    \tkzDrawSegments[orange](E,I E,J E,K)
    \tkzDrawSegments[dashed](A,Ha B,Hb C,Hc)
    \tkzDrawCircles(J,j I,i K,k)
    \tkzDrawPoints(A,B,C)
    \tkzDrawPoints[orange](E,I,J,K,Ha,Hb,Hc,Ea,Eb,Ec,Fa,Fb,Fc)
    \tkzLabelPoints[font=\scriptsize](E,Ea,Eb,Ec,Ha,Hb,Hc,Fa,Fb,Fc)  
  \end{tikzpicture}
\end{tkzexample}

\subsubsection{Option \tkzname{symmedial}}
The symmedial triangle$ K_AK_BK_C$ is the triangle whose vertices are the intersection points of the symmedians with the reference triangle $ABC$. 

\begin{tkzexample}[latex=7cm,small]
\begin{tikzpicture}
\tkzDefPoint(0,0){A}
\tkzDefPoint(5,0){B}
\tkzDefPoint(.75,4){C}
\tkzDefTriangleCenter[symmedian](A,B,C)\tkzGetPoint{K} 
\tkzDefSpcTriangle[symmedial,name=K_](A,B,C){A,B,C}
\tkzDrawPolygon(A,B,C)
\tkzDrawSegments[new](A,K_A B,K_B C,K_C)
\tkzDrawPoints(A,B,C,K,K_A,K_B,K_C)
\tkzLabelPoints(A,B,K,K_C)
\tkzLabelPoints[above](C)
\tkzLabelPoints[right](K_A)
\tkzLabelPoints[left](K_B)
\end{tikzpicture}
\end{tkzexample}

\subsection{Permutation of two points of a triangle}

\begin{NewMacroBox}{tkzPermute}{\parg{$pt1$,$pt2$,$pt3$}}%
\begin{tabular}{lll}%
arguments             & example & explanation                         \\
\midrule
\TAline{(pt1,pt2,pt3)} {\tkzcname{tkzPermute}(A,B,C)}{$A$, $\widehat{B,A,C}$ are unchanged, $B$, $C$ exchange their position}
\midrule
\end{tabular}

\medskip
\emph{The triangle is unchanged.}
\end{NewMacroBox}

\subsubsection{Modification of the \tkzname{school} triangle}
This triangle is constructed from the segment $[AB]$ on $[A,x)$.

If we want the segment $[AC]$ to be on $[A,x)$, we just have to swap $B$ and $C$.

\begin{tkzexample}[latex=7cm,small]
\begin{tikzpicture}
  \tkzDefPoints{0/0/A,4/0/B,6/0/x}
  \tkzDefTriangle[school](A,B)  
  \tkzGetPoint{C}
  \tkzPermute(A,B,C)
  \tkzDrawSegments(A,B C,x)
  \tkzDrawSegments(A,C B,C)
  \tkzDrawPoints(A,B,C)
  \tkzLabelPoints(A,C,x)
  \tkzLabelPoints[above](B)
  \tkzMarkRightAngles(C,B,A)
\end{tikzpicture}
\end{tkzexample}

Remark: Only the first point is unchanged. The order of the last two parameters is not important.

\endinput