%!TEX root = tzplot-doc.tex
%\begin{document}

%%%%%=====CNONTINUED============
%%%%%%==========================
%%%\part{Points, Lines, and Curves}
%%%%%%==========================
%%%\label{p:linesandcurves}


%%==================================
\chapter{Connecting Points}
\label{c:tzto}

%%%------------------------------------------------------------
\section{\protect\cmd{\tzto}: Two points}
\label{s:tzto}

\icmd{\tzto} connects two points with a straight or curved line, using \Tikz's |to| operation.
So |\tzto| is more general than |\tzline|, which only connects points only with a straight line.

\begin{tzdef}
% syntax: minimum
\tzto(<coor>)(<coor>)
% syntax: medium
\tzto[<opt>](<coor>){<text>}[<node opt>](<coor>){<text>}[<node opt>]
% syntax: full
\tzto[<opt>]<shift coor>"<path name>"
     (<coor>){<text>}[<node opt>]
     (<coor>){<text>}[<node opt>]<code.append>
% defaults
  []<>""(<m>){}[above](<m>){}[]<>
\end{tzdef}


\begin{tztikz}
\tzto[dashed](1,1)(3,2) % works like:
  \draw [dashed] (1,1) to (3,2);
\end{tztikz}


\paragraph{Line styles}
You can control line styles with the first optional argument |[<opt>]|.

\begin{tzcode}{.3}
% \tzto
\begin{tikzpicture}
\tzhelplines(4,3)
\tzto[blue,thick](0,1)(3,3)
\tzto[out=0](0,0)(4,2)
\tzto[bend right,dashed](2,0)(4,1)
\settzmidarrow{o}[very thick,red]
\tzarc(2,2)(180:45:1cm)
\end{tikzpicture}
\end{tzcode}

\paragraph{Adding text}

You can add text \xem{next to the line} (|[midway,above]|, by default) by specifying the options |{<text>}| and |[<node opt>]| \xem{in-between} the two coordinates.

\begin{tztikz}
\tzto[->,bend right](1,1){A}[near start](3,2) % works like:
  \draw [->,bend right] (1,1) to node [near start] {A} (3,2);

\tzto[->,bend right](1,1){A}[near start](3,2){B}[right] % works like:
  \draw [->,bend right] (1,1) to node [near start] {A} 
                        (3,2)    node [right]      {B};
\end{tztikz}


You can also add text \xem{at or around} the last coordinate by the options |{<text>}| and |[<node opt>]| \xem{immediately after} the last coordinate.

\begin{tzcode}{.3}
% \tzto: adding text
\begin{tikzpicture}
\tzhelplines(4,3)
\tzto[blue,thick,->](0,1){line}[sloped](3,3)
\tzto[out=0](0,0){curve}[pos=.8](4,2){line B}[r]
\tzto[bend right,dashed](2,0)(4,1){line C}[red,r]
\end{tikzpicture}
\end{tzcode}

\paragraph{Shift}
You can move the line by specifying the option |<shift coor>| before the first coordinate or immediately before the option |"<path name>"| if it exists.
(The \xem{empty} shift option |<>| is \xem{not allowed}.)

\begin{tzcode}{.3}
% \tzto: shift
\begin{tikzpicture}
\tzhelplines(4,3)
\tzto[blue,thick,->]<0,1>(0,1){line}[sloped](3,3)
\tzto[out=0]<1,1>(0,0){curve}[pos=.8](4,2){line B}[r]
\tzto[bend right,dashed]<-1,0>(2,0)(4,1){line C}[red,r]
\end{tikzpicture}
\end{tzcode}

\paragraph{Naming paths: Intersections}
You can name the path of |\tzto| by specifying the option |"<path name>"| immediately before the first coordinate.

\begin{tzcode}{.3}
% \tzto: name path
\begin{tikzpicture}
\tzhelplines(4,3)
\tzto[bend right=45,blue,thick,->]"curveA"(1,3)(4,0)
\tzto[out=0]"curveB"(0,0)(4,2){line B}[r]
% intersection point
\tzXpoint*{curveA}{curveB}
\end{tikzpicture}
\end{tzcode}

\paragraph{Extending the path}
You can extend the path of |\tzto| by writing \Tikz\ code in the last optional argument |<code.append>|.

\begin{tzcode}{.3}
% \tzto: extending path
\begin{tikzpicture}
\tzhelplines(4,3)
\tzto[bend right=45,blue,thick,->](1,3)(4,0)
  <--([turn]0:1cm) node [b] {extended}>
\tzto[out=0](0,0)(4,2)
  <--([turn]0:1cm) node [r] {extended}>
\end{tikzpicture}
\end{tzcode}

You can also use \icmd{\tztoAtBegin} and \icmd{\tztoAtEnd} to extend a path of |\tzto| at the beginning and at the end, respectively.
Specifying |<code.append>| extends the path after |\tztoAtEnd|.

\begin{tzcode}{.3}
% \tzto: extending path
\begin{tikzpicture}
\tzhelplines(4,3)
\tztoAtBegin{(0,2) to [bend left]}
\tztoAtEnd{--([turn]0:1cm) node [b] {extended}}
\tzto[bend right=45,blue,thick,->](1,3)(4,0)
\tztoAtEnd{--([turn]0:1cm) node [r] {extended}}
\tzto[out=0](0,0)(4,2)
\end{tikzpicture}
\end{tzcode}


%%------------------------------------------------------------
\section{\protect\cmd{\tzto+}: Relative coordinates}
\label{s:tzto+}

The \iisw{plus version} \icmd{\tzto+} uses the second coordinate relative to the first coordinate.

\xem{Everything else is the same as in} |\tzto|.

\begin{tztikz}
\tzto+(1,1)(3,2) % works like:
  \draw (1,1) to ++(3,2);
\end{tztikz}

\begin{tzcode}{.3}
% \tzto+
\begin{tikzpicture}
\tzhelplines(4,3)
\tzto+[blue,thick,->](0,1){line}[sloped](3,2)
\tzto+[out=0](0,0){curve}[pos=.8](4,2){line B}[r]
\tzto+[bend right,dashed](2,0)(2,1){line C}[red,r]
\end{tikzpicture}
\end{tzcode}

\begin{tzcode}{.3}
% \tzto+
\begin{tikzpicture}
\tzhelplines(4,3)
\tzto+[bend right=45,blue,thick,->](1,3)(3,-3)
  <--([turn]0:1cm) node [b] {extended}>
\tzto+[out=0](0,0)(4,2)
  <--([turn]0:1cm) node [r] {extended}>
\end{tikzpicture}
\end{tzcode}


\begin{tzcode}{.3}
% \tzto: extending path
\begin{tikzpicture}
\tzhelplines(4,3)
\tztoAtBegin{(0,2) to [bend left]}
\tztoAtEnd{--([turn]0:1cm) node [b] {extended}}
\tzto+[bend right=45,blue,thick,->](1,3)(3,-3)
\tztoAtEnd{--([turn]0:1cm) node [r] {extended}}
\tzto+[out=0](0,0)(4,2)
\end{tikzpicture}
\end{tzcode}




%%------------------------------------------------------------
\section{\protect\cmd{\tztos}: Multiple points: Semicolon version}
\label{s:tztos}

\icmd{\tztos} takes an arbitrary number of coordinates as arguments to connect them by the \Tikz's |to| operation. |\tztos| is much more flexible than |\tzlines|. Since this is a \xem{semicolon version}, you need to enter a \xem{semicolon} to indicate when the coordinate iteration ends.

\begin{tzdef}
% syntax
\tztos[<opt>]<shift coor>"<path name>"
      (<coor1>)[<to opt>]{<text>}[<node opt>]..repeated..()[]{}[] ; <code.append>
% defaults
  []<>""(<m>)[]{}[]..repeated..()[]{}[];<>
\end{tzdef}

The quadruple |(<coor>)[<to opt>]{<text>}[<node opt>]| is 
the whole repeating pattern.
Here, |[<to opt>]| is for the options of \Tikz's |to| operation such as |[bend right]|, |[bend left]|, |[bend left=<angle>]|, |[out=<angle>,in=<angle>]| and so on.

\begin{tztikz}
\tztos(0,0)(1,2)(3,1); % works like:
  \draw (0,0) to (1,2) to (3,1);
\end{tztikz}

\begin{tztikz}
\tztos[blue]  (0,0)    [bend right] (1,2)    [out=-60,in=45] (3,1); % works like:
  \draw[blue] (0,0) to [bend right] (1,2) to [out=-60,in=45] (3,1);
\end{tztikz}

\paragraph{How to connect coordinates}
You can use the options of \Tikz's |to| operation to connect the coordinates with different types of curves.

\begin{tzcode}{.3}
% \tztos
\begin{tikzpicture}
\tzhelplines(4,3)
\tztos[blue,<->](0,1)(1,2)(3,1)(4,3);
\tztos[red,thick]
      (0,0)[bend right]
      (1,1)
      (2,1)[out=-135,in=45]
      (3,0);
\end{tikzpicture}
\end{tzcode}

\begin{tzcode}{.3}
% \tztos
\begin{tikzpicture}
\tzhelplines(4,3)
\tzcoors*(0,0)(A)(1,2)(B){B}[b](3,3)(C)(4,2)(D);
\tztos(A)[out=80,in=180]
      (B)[out=0,in=180]
      (C)[out=0,in=180]
      (D){End!}[r];
\end{tikzpicture}
\end{tzcode}

\paragraph{Adding text}
You can add text next to lines or curves by specifying the options |{<text>}| and |[<node opt>]| \xem{in-between} coordinates or \xem{after} the option |[<to opt>]|, if it exists.
You can also add text at or around the last coordinate by the last options |{<text>}| and |[<node opt>]|.

\begin{tzcode}{.3}
% \tztos: adding text
\begin{tikzpicture}
\tzhelplines(4,3)
\tztos(0,1)[bend right]{A}
      (1,2)            {Ben}[red,sloped,a]
      (3,1)[bend right]{Cate}[sloped,near end]
      (4,3)            {Name}[draw,blue,a] ;
\end{tikzpicture}
\end{tzcode}

\paragraph{Shift}
You can move the line or curve of |\tztos|
using the option |<shift coor>| before the first coordinate or immediately before the option |"<path name">|, if any.
(The empty shift option |<>| is not allowed.)

\begin{tzcode}{.3}
% \tztos: shift
\begin{tikzpicture}
\tzhelplines(4,3)
\tzdot(0,0)
\tztos[blue,<->](0,1)(1,2)(3,1)(4,3);
\tztos[red,thick]
  <1,1>(0,0)[bend right]
       (1,1)
       (2,1)[out=-135,in=45]
       (3,0);
\end{tikzpicture}
\end{tzcode}



\paragraph{Naming paths: intersections}
You can name the path of |\tztos| by specifying |"<path name>"| immediately before the first mandatory coordinate.

\begin{tzcode}{.3}
% \tztos: name path, intersection points
\begin{tikzpicture}
\tzhelplines(4,3)
\tzdot(0,0)
\tztos[blue,<->]"AA"(0,1)(1,2)(3,1)(4,3);
\tztos[red,thick]
  <1,1>"BB"(0,0)[bend right]
       (1,1)
       (2,1)[out=-135,in=45]
       (3,0);
% intersection points
\tzXpoint*{AA}{BB}(X)
\tzdot(X-2)(4pt)
\end{tikzpicture}
\end{tzcode}


\paragraph{Extending paths}
You can extend the path of |\tztos| from the last coordinate,
by writing \Tikz\ code in the last (after the semicolon) optional argument |<code.append>|.

\begin{tzcode}{.3}
% \tztos
\begin{tikzpicture}
\tzhelplines(4,3)
\tztos[blue](0,1)(1,2)(3,1)(4,3);<--cycle>
\tztos[red,thick]
      (0,0)[bend right]
      (1,1)
      (2,1)[out=-135,in=45]
      (3,0); < to ([turn]0:1.5cm) >
\end{tikzpicture}
\end{tzcode}

You can also use \icmd{\tztosAtBegin} and \icmd{\tztosAtEnd} to extend a path of |\tztos| at the beginning and at the end, respectively.
Specifying the option |<code.append>| extends the path after |\tztosAtEnd|.

\begin{tzcode}{.3}
% \tztos
\begin{tikzpicture}
\tzhelplines(4,3)
\tztos[blue](0,1)(1,2)(3,1)(4,3);<--cycle>
\tztosAtBegin{ (1,3) -| }
\tztosAtEnd{ to ([turn]0:1.5cm) }
\tztos[red,thick]
      (0,0)[bend right]
      (1,1)
      (2,1)[out=-135,in=45]
      (3,0); 
\end{tikzpicture}
\end{tzcode}

%%------------------------------------------------------------
\section{\protect\cmd{\tztos+}: Relative coordinates: Semicolon version}
\label{s:tztos+}

The \iisw{plus version} \icmd{\tztos+} takes each coordinate (except the first coordinate) relative (with |++|) to the previous coordinate.

\xem{Everything else is the same as in} |\tztos|.

\begin{tzcode}{.3}
% \tztos+
\begin{tikzpicture}
\tzhelplines(4,3)
\tztos+[blue](0,1)(1,1)(2,-1)(1,2);<--cycle>
\tztos+[red,thick]
      (0,0)[bend right]
      (1,1)
      (1,0)[out=-135,in=45]
      (1,-1); < to ([turn]0:1.5cm) >
\end{tikzpicture}
\end{tzcode}



\section{\protect\cmd{\tzlink}: Two points}
\label{s:tzlink}

\icmd{\tzlink} is a \xem{generalized version} of |\tzline|.
You can decide the style that links two coordinates.

The default link style is |to|, which can be changed with the \xem{second} optional argument |[<code1>]| \xem{between two coordinates} like, for example, \verb+(<coor1>)[-|](<coor2>)+.
Another way is to use the \xem{first} curly brace option |{<link style>}|.
You can also change it by \icmd{\settzlinkstyle} like, for example, \verb+\settzlinkstyle{-|}+, which is effective until the end of the |tikzpicture| environment. 

\begin{tzdef}
% syntax: minimum
\tzlink(<coor1>)(<coor2>)                % default: to
% syntax: medium
\tzlink[<opt>](<coor1>)[<code1>](coor2>) % change link style by [<code1>]
% syntax: full
\tzlink{<link style>}[<opt>]<shift coor>"<path name>"
       (<coor1>)[<code1>]{<text1>}[<node opt1>]<code2> 
       (<coor2>){<text2>}[<node opt2>]<code.append>
% remark:
  - {<link style>}: % default: to
  - [<code1>]: --, -|, to, edge, etc. (optionally more appropriate tikz code)
  - <code2>: + or ++                  (possibly with other tikz code)
  - {<text1>}: node text next to line
  - {<text2>}: node text at the last coordinate
% defaults
 {to}[]<>""(<m>)[]{}[above]<>(<m>){}[]<>
\end{tzdef}

\begin{tztikz}
\tzlink(1,0)(3,1) % works like:
  \draw (1,0) to (3,1);
\end{tztikz}

\begin{tztikz}
\tzlink(1,0)[-|]<++>(3,1) % works like:
  \draw (1,0) -| ++ (3,1);
\end{tztikz}


\paragraph{Link styles} You can change the link style (|to| by default) using the optional argument |[<code1>]| \xem{immediately after} the first coordinate.

\begin{tzcode}{.3}
% \tzlink: to (default link style)
\begin{tikzpicture}
\tzhelplines(4,3)
\tzlink[->](0,0)(1,0)
\tzlink[->,bend right](0,2)(3,0)
\tzlink[->](1,2)[-|](4,1)
\end{tikzpicture}
\end{tzcode}

You can also use \icmd{\settzlinkstyle} to change the link style. Its effect is valid until the end of the |tikzpicture| environment, unless changed again.

\begin{tzcode}{.3}
% \settzlinkstyle
\begin{tikzpicture}
\tzhelplines(4,3)
\settzlinkstyle{to[bend left]}  %%
\tzlink[->](0,0)(1,0)
\tzlink[->](0,2)[--](3,0)       % local change
\tzlink[->](1,2)(4,1)
\end{tikzpicture}
\end{tzcode}

The \iisw{plus version} \icmd{\tzlink+} uses the second coordinate relative (with |++|, by default) to the first one.
Everything else is the same as in |\tzlink|.

         
\begin{tztikz}
\tzlink (1,0)[to[bend right]]{A}[a]<++>(3,1) % or
\tzlink+(1,0)[to[bend right]]{A}[a]    (3,1) % works like:
  \draw (1,0) to[bend right] node [above] {A} ++ (3,1);
\end{tztikz}


\begin{tzcode}{.3}
% \tzlink(+): various link styles
\begin{tikzpicture}
\tzhelplines(4,3)
\tzlink[->,blue,out=90,in=180](0,2)(3,0)
\tzlink+[->,dashed,blue](1,2)[to[bend left]](3,-1)
\tzlink[->](0,0)[..controls (1,1) and (3,0)..](4,2)
\tzlink+[red,thick](1,0)[edge[bend left]](2,3)
\end{tikzpicture}
\end{tzcode}

\paragraph{Adding text}

You can add text next to the line or curve of |\tzlink| by specifying the option |{<text1>}| followed by the optional argument |[<node opt1>]| \xem{between} two coordinates (or after the option |[<code1>]| if it exists).

You can name the line or curve of |\tzlink| by specifying the optional arguments |{<text2>}| and |[<node opt2>]| \xem{after} the second coordinate. The option |[<node opt2>]| is often for the position of node text at or around the second coordinate.

\begin{tzcode}{.3}
% \tzlink
\begin{tikzpicture}
\tzhelplines(4,3)
\tzlink[->](1,3){line A}(4,3)
\tzlink[blue](0,3){line B}[sloped](3,0){E}[r]
\tzlink[->,red](1,0){Start}[at start,l](4,2){Line C}[r]
\end{tikzpicture}
\end{tzcode}

\begin{tzcode}{.3}
% \tzlink(+)
\begin{tikzpicture}
\tzhelplines(4,3)
\tzlink[->](0,0){A}(1,0)
\tzlink+[bend right](0,2){curve}[sloped](3,-2){End}[r]
\tzlink[->,auto](1,2)[-|]{B}(4,1){Goal!}[b,draw,blue]
\end{tikzpicture}
\end{tzcode}


\paragraph{Shift}
You can move the line or curve by specifying the option |<shift coor>|  \xem{before} the first coordinate (or ever before the option |"<path name>"| if it exists).

\begin{tzcode}{.3}
% \tzlink(+): shift
\begin{tikzpicture}
\tzhelplines(4,3)
\tzlink+[bend right](0,2)(3,-2){A}[r]
\tzlink+[bend right]<.5,.5>(0,2)(3,-2){A-shifted}[r]
\tzlink[->](1,2)[-|](4,1)
\tzlink[->,dashed]<-.2,.3>(1,2)[-|](4,1)
\end{tikzpicture}
\end{tzcode}

\paragraph{Naming paths: Intersections}
You can name the path of |\tzlink| by specifying the quote option |"<path name>"| \xem{immediately before} the first coordinate.
You can use the path names of two paths to find an intersection point.

\begin{tzcode}{.3}
% \tzlink(+): path name: intersection
\begin{tikzpicture}
\tzhelplines(4,3)
\tzlink[blue]"BB"(0,3){line B}[sloped](3,0)
\tzlink+[->,red,bend right]"CC"(1,0)(3,2){Line C}[r]
\tzXpoint{BB}{CC}(E) % instersection
\tzdot*(E){E}[a]
\end{tikzpicture}
\end{tzcode}


\paragraph{Extending paths}
You can extend the path of |\tzlink| by specifying \Tikz\ code in the last optional argument |<code.append>|.
You can also use the macros \icmd{\tzlinkAtBegin} and \icmd{\tzlinkAtEnd} to extend the path at the beginning and at the end, respectively.
Specifying the option |<code.append>| extends the path after |\tzlinkAtEnd|.

\begin{tzcode}{.3}
% \tzlink(+): path name: intersection
\begin{tikzpicture}
\tzhelplines(4,3)
\tzlinkAtBegin{(1,1) to[bend left]}
\tzlinkAtEnd{arc (180:90:1cm) node [r,draw] {BB}}
\tzlink[blue]"BB"(0,3){line B}[sloped](3,0)
\tzlink+[->,red,bend right]"CC"(1,0)(3,2){Line C}[r]
  < -| ++(-1,1) node [a] {End!} >
\tzXpoint{BB}{CC}(E)
\tzdot*(E){E}[a]
\end{tikzpicture}
\end{tzcode}


%%------------------------------------------------------------
\section{\protect\cmd{\tzlinks}: All in one: Semicolon versions}
\label{s:tzlinks}

\subsection{\protect\cmd{\tzlinks}: Standard version}
\label{ss:tzlinks}

The macro \icmd{\tzlinks} is a \iisw{semicolon version} of |\tzlink|. |\tzlinks| accepts any number of coordinates to connect them with connected line segments or curves.
You can change how two adjacent points are connected.
|\tzlinks| is quite \xem{flexible} that you can think of it as a \xem{generalized version} of |\tzlines|.


\begin{tzdef}
% syntax: minimum
\tzlinks (<coor>)(<coor>) ..repeated.. (<coor>) ;
% syntax: full
\tzlinks{<path style>}[<opt>]<shift coor>"<path name>"
        (<coor>)[<code1>]{<text>}[<node opt>]<code2>
        ..repeated.. ()[]{}[]<> ; {<fill opacity>} <code.append>
% remark:
  - <code1> must be link style, such as --, to, -|, etc.
           (possibly followed by other code)
  - <code2> is especially for + or ++
           (possibly with other code)
  - repetition MUST be ended by ; (semicolon)
  - full repeating pattern is ()[]{}[]<>..repeated..()[]{}[]<>
  - {<fill opacity>} works only in \tzlinks*
% defaults: \tzlinks, \tzlinks*, \tzlinks+, \tzlinks*+
  {to}[]<>"" (<m>)[]{}[]<>   ..repeated.. ()[]{}[]<> ; {  }<>
 *{to}[fill=black!50,fill opacity=.3,text opacity=1]
        <>"" (<m>)[]{}[]<>   ..repeated.. ()[]{}[]<> ; {.3}<>
 +{to}[]<>"" (<m>)[]{}[]<++> ..repeated.. ()[]{}[]<> ; {  }<>
\end{tzdef}

\paragraph{How to change link styles}
There are three ways of changing link styles:
\begin{itemize}\firmlist
\item The fist curly brace option |{<link style>}| controls the link style of connecting any two coordinates.
\item The second bracket option |[<code1>]| (locally) changes the link style of two adjacent coordinates. |[<code1>]| (locally) overrides |{<link style>}|.
\item The effect of the macro |\settzlinkstyle| remains until the end of |tikzpicture| environment, unless changed again.
\end{itemize}

\begin{tztikz}
% standard version
\tzlinks(1,1)(2,2)(3,1)(4,3); % works like:
  \draw (1,1) to (2,2) to (3,1) to (4,3);
\end{tztikz}

\begin{tzcode}{.3}
% \tzlinks
\begin{tikzpicture}
\tzhelplines(4,3)
\tzlinks(0,1)(1,2)(3,1)(4,3);
\end{tikzpicture}
\end{tzcode}

The default path style is |to|, which can be by \icmd{\settzlinkstyle}.
The effect remains valid until the end of |tikzpicture| environment unless changed again.

\begin{tzcode}{.3}
% \tzlinks: \settzlinkstyle
\begin{tikzpicture}
\tzhelplines(4,3)
\settzlinkstyle{to[bend right]} %%
\tzlinks[dashed](0,1)(1,2)[-|](3,1)(4,3);
\tzlinks[blue]
      (0,0)
      (1,1)
      (2,1)[to[out=-135,in=45]]
      (3,0);
\end{tikzpicture}
\end{tzcode}


\begin{tzcode}{.3}
% \tzlinks: various link styles
\begin{tikzpicture}
\tzhelplines(4,3)
\tzlinks[->](1,2)[-|]
            (0,0)[..controls (1,2) and (2,0)..]
            (3,1)[to[bend left]]<++>
            (1,2);
\end{tikzpicture}
\end{tzcode}


\begin{tzcode}{.3}
% \tzlinks: various link styles
\begin{tikzpicture}
\tzhelplines(4,3)
\tzlinks[->](1,1)<++>
            (1,1)[[rounded corners=10pt]--]
            (0,3)
            (0,0)[[sharp corners]--]
            (4,0)<++>
            (-1,2);
\end{tikzpicture}
\end{tzcode}


\begin{tzcode}{.3}
% \tzlinks: edge
\begin{tikzpicture}
\tzhelplines(4,3)
\tzlinks(0,1)(1,2)[edge[dashed]]<(3,1)>(3,1)(4,3);
\end{tikzpicture}
\end{tzcode}

\remark
In \Tikz, |edge| is not part of the main path.
So you need to be careful when you move the path or find intersection points.



\subsection{\protect\cmd{\tzlinks+}, \protect\cmd{\tzlinks*}, \protect\cmd{\tzlinks*+}: Variants}
\label{ss:tzlinks-variants}

The \iisw{plus version} \icmd{\tzlinks+} treats a coordinate as relative (with |++| by default) to the previous coordinate, except the first coordinate.

\begin{tztikz}
% plus (+) version
\tzlinks+(1,1)(2,2){--}(3,1)(4,3); % works like:
  \draw (1,1) to ++(2,2) -- ++(3,1) to ++(4,3);
\end{tztikz}

\begin{tzcode}{.3}
% \tzlinks+
\begin{tikzpicture}
\tzhelplines(4,3)
\tzlinks+(0,1)(1,1)[to[bend right]](2,-1)(1,2);
\end{tikzpicture}
\end{tzcode}

The \xem{starred version} \icmd{\tzlinks*} is to fill the closed area formed by |\tzlinks| with color or pattern. The related default value is  |[fill=black!50, fill opacity=.3, text opacity=1]|.
In fact, |\tzlinks*[draw=none]| is (almost) the same as |\tzpath*| (See Section \ref{s:tzpath*} on page \pageref{s:tzpath*} for more details).

\begin{tztikz}
% starred (*) version
\tzlinks*(1,1)(2,2)(3,1)(4,3); % works like:
  \draw [fill=black!50,fill opacity=.3,text opacity=1] 
        (1,1) to (2,2) to (3,1) to (4,3);
\end{tztikz}

\icmd{\tzlinks*+} is the plus version of |\tzlinks*|.

\begin{tzcode}{.3}
% \tzlinks*+
\begin{tikzpicture}
\tzhelplines(4,3)
\tzlinks*+[thick](0,1)(1,1)[to[bend right]](2,-1)(1,2);
\end{tikzpicture}
\end{tzcode}

\begin{tzcode}{.3}
% \tzlinks: fill, fill opacity
\begin{tikzpicture}
\tzhelplines(4,3)
\tzline[red,very thick](0,3)(4,0)
\tzlinks[fill](0,0)(3,1)(2,3)(0,2); % default opacity=1
\tzlinks[fill=green,fill opacity=.5]
  <1,-1>(0,0)(3,1)(2,3)(0,2);
\end{tikzpicture}
\end{tzcode}

You can also change the fill opacity by specifying the \xem{last} curly brace optional argument |{<fill opacity>}|, \xem{after the semicolon} (or before the option |<code.append>| it it exists).

\begin{tzcode}{.3}
% \tzlinks*
\begin{tikzpicture}
\tzhelplines(4,3)
\settzfillcolor{green}
\tzline[red,very thick](0,3)(4,0)
\tzlinks*[blue](0,0)(3,1)(2,3)(0,2);{1}
\tzlinks*%[green]
  <1,-1>(0,0)(3,1)(2,3)(0,2); {.5} %%
\end{tikzpicture}
\end{tzcode}

You can also use the macros |\settzfillcolor| and |\settzfillopacity| to change the defaults.
The effect remains valid until the end of the |tikzpicture| environment, unless changed again.

%%------------------------------------------------------------
\subsection{Putting text, shift, intersections, and extending paths}
\label{ss:tzlinks-add}


\paragraph{Adding text}
You can add text next to connected line segment or around the last coordinate by specifying options |{<text>}[<node option>]|.

\begin{tztikz}
\tzlinks+[dashed]"AA"(1,1){A}(2,2)[--]{B}[a]<+>(3,1){C}[b]; % works like
  \draw [dashed,name path=AA] 
        (1,1) to node         {A} 
      ++(2,2) -- node [above] {B}  
       +(3,1) to node [below] {C} ;
\end{tztikz}


\begin{tzcode}{.3}
% \tzlinks: adding text
\begin{tikzpicture}
\tzhelplines(4,3)
\tzlinks[blue](0,0){Ace}
              (1,1){Beth}[sloped,a]
              (3,0){C}   [near start]
              (4,2){End!}[draw,a=3pt];
\end{tikzpicture}
\end{tzcode}

\begin{tzcode}{.3}
% \tzlinks: adding text
\begin{tikzpicture}
\tzhelplines(4,3)
\tzlinks[->](1,1){start}[sloped,b]<++>
    (1,1)[[rounded corners=10pt]--]{K}[at end,a]
    (0,3){Rounded}[scale=0.7,font=\scshape,at end,ar]
    (0,0)[[sharp corners]--]
    (4,0)<++>
    (-1,2){End!}[draw,blue,a=3pt];
\end{tikzpicture}
\end{tzcode}

\begin{tzcode}{.3}
% \tzlinks: edge
\begin{tikzpicture}
\tzhelplines(4,3)
\tzlinks(0,1){start}[at start,b]
  (1,2)[edge[dashed,bend right]]{Go!}[sloped,b]<(3,1)>
  (3,1)
  (4,3){End\\here!}[align=center,r,draw,red] ;
\end{tikzpicture}
\end{tzcode}

\remark Note that |edge| is not part of the main path.

\paragraph{Shift}
You can move the path of |\tzlinks| with the option |<shift coor>|.
An \xem{empty} option |<>| is \xem{not allowed}.


\begin{tzcode}{.3}
% \tzlinks:
\begin{tikzpicture}
\tzhelplines(4,3)
\tzlinks<.5,-1>(0,1){start}[at start,b] % shift
  (1,2)[to[dashed,bend right]]{Go!}[sloped,b]
  (3,1)
  (4,3){End\\here!}[align=center,r,draw,red] ;
\end{tikzpicture}
\end{tzcode}


\paragraph{Naming paths: intersections}
You can name the path of |\tzlinks| \xem{immediately befor} the first coordinate and use it to find intersection points.

\begin{tzcode}{.3}
% \tzlinks: edge
\begin{tikzpicture}
\tzhelplines(4,3)
\tzlinks"AA"(0,1){start}[at start,b]
  (1,2)[edge[dashed,bend right]]{Go!}[sloped,b]<(3,1)>
  (3,1)
  (4,3){End\\here!}[align=center,r,draw,red] ;
\tzhXpointat{AA}{1.5}(A) % horizontal intersections
\tzdots*(A){\texttt{(A)}}(A-2){\texttt{(A-2)}}[0];
\end{tikzpicture}
\end{tzcode}

In the previous example, since |edge| is not part of the main path of |\tzlinks|,
the second horizontal intersection point |(A-2)| is not on the |edge| curve.

\paragraph{Extending paths}

You can extend the path of |\tzlinks| by writing \Tikz\ code in the last (after the semicolon) optional argument |<code.append>|.
So |<--cycle>| closes the path with a straight line.

\begin{tzcode}{.3}
% \tzlinks* <code.append>
\begin{tikzpicture}
\tzhelplines(4,3)
\tzlinks*[blue](0,1)(1,2)[-|](3,1)(4,3); 
             < to [bend right=60] (0,1) -- cycle >
\tzlinks*[fill=red]
      (0,0)[to[bend right]]
      (1,1)
      (2,1)[to[out=-135,in=45]]
      (3,0); % not closed
\end{tikzpicture}
\end{tzcode}

You can also use \icmd{\tzlinksAtBegin} and \icmd{\tzlinksAtEnd} to extend the path of |\tzlinks| at the beginning and at the end, respectively.
Specifying the option |<code.append>| extends the path after |\tzlinksAtEnd|.

\begin{tzcode}{.3}
% \tzlinks*: extending paths
\begin{tikzpicture}
\tzhelplines(4,3)
\tzlinksAtBegin{(0,2) to[bend left]}
\tzlinksAtEnd{ to [out=210] (1,3) }
\tzlinks*[blue](0,1)(1,2)[-|](3,1)(4,3); 
             < -| (0,2) >
\tzlinks*[fill=red]
      (0,0)[to[bend right]]
      (1,1)
      (2,1)[to[out=-135,in=45]]
      (3,0); <--cycle>
\end{tikzpicture}
\end{tzcode}




%%==================================
\chapter{Filling Area}
\label{c:fillingarea}

%%------------------------------------------------------------
\section{\protect\cmd{\tzpath}: Semicolon version}
\label{s:tzpath}


The macro \icmd{\tzpath} is the same as |\tzlinks[draw=none]|. In other words, |\tzpath| creates a path connecting an arbitrary number of coordinates, but it does not stroke the path. Since |\tzpath| is a \iisw{semicolon version} macro, you need to enter a \xem{semicolon} `|;|' to indicate where the coordinate iteration ends.

You can visualize the path with |\tzpath[draw]|, which is the same as |\tzlinks|.


\begin{tzdef}
% syntax: minimal
\tzpath (<coor>)(<coor>) ..repeated.. (<coor>) ;
% syntax: full
\tzpath {<path style>}[<opt>]<shift coor>"<path name>"
        (<coor>)[<code1>]{<text>}[<node opt>]<code2>
        ..repeated.. ()[]{}[]<> ; <code.append>
% remark:
  - <code1> must be link style, such as --, to, -|, etc.
           (possibly followed by other codes)
  - <code2> is especially for + and ++
           (possibly with other codes)
  - repetition MUST be ended by ; (semicolon)
% defaults: \tzpath
  {to}[]<>"" (<m>)[]{}[]<> ..repeated.. ()[]{}[]<> ; <>
\end{tzdef}

\remark
\begin{itemize}
\item The main purpose of |\tzpath| is to fill an enclosed area with colors or patterns.
You can use |\tzpath[fill]| or |\tzpath[pattern=<...>]| to do it.
\item Use \icmd{\settzpathlayer}, like |\settzpathlayre{behind}|, to change the layer of |\tzpath| (default: |main|).
\end{itemize}

\paragraph{Path construction operation}
(If you are not an experienced user of \Tikz\, just skip this part.)

%|\tzpath| is similar to |\tztos|, but it is more flexible in constructing paths.
|\tzpath| allows you to choose how to construct a path using the |[<path style>]| option \xem{in-between} coordinates. \xem{Path extension operation} can be selected from `|--|', `|to|', `\verb+|-+', `\verb+-|+', etc.
You can use \xem{the first brace option} |{<path style>}| to change all the |<path style>| in-between coordinates.


\begin{tzcode}{.3}
% \tzpath: path construction
\begin{tikzpicture}
\tzhelplines(4,3)
\tzpath[draw](0,1)(1,2)[-|](3,1)(4,3);
\tzpath[draw]
      (0,0)[to[bend right]]
      (1,1)
      (2,1)[to[out=-135,in=45]]
      (3,0);
\end{tikzpicture}
\end{tzcode}

The default path style is `|to|' and can also be changed by \icmd{\settzpathstyle}, like, for example, |\settzpathstyle{--}|. |\settzpathstyle| is an alias of |\settzlinkstyle|. The effect remains valid until the end of |tikzpicture| environment unless changed again.

\begin{tzcode}{.3}
% \tzpath: \settzpathstyle
\begin{tikzpicture}
\tzhelplines(4,3)
\settzpathstyle{to[bend right]}
\tzpath[draw](0,1)(1,2)[-|](3,1)(4,3);
\tzpath[draw]
      (0,0)[to[bend right]]
      (1,1)
      (2,1)[to[out=-135,in=45]]
      (3,0);
\end{tikzpicture}
\end{tzcode}

\begin{tzcode}{.3}
% \tzpath: \settzpathstyle
\begin{tikzpicture}
\tzhelplines(4,3)
\settzpathstyle{to[bend right]}
\tzlinks(0,1)(1,2)[-|](3,1)(4,3);
\tzlinks
      (0,0)[to[bend right]]
      (1,1)
      (2,1)[to[out=-135,in=45]]
      (3,0);
\end{tikzpicture}
\end{tzcode}

You can extend the path of |\tzpath| by writing \Tikz\ code in the last (after the semicolon) optional argument |<code.append>|.
So |<--cycle>| closes the path with a straight line.

\begin{tzcode}{.3}
% \tzpath: <code.append>
\begin{tikzpicture}
\tzhelplines(4,3)
\tzpath[draw](0,1)(1,2)[-|](3,1)(4,3); 
             < to [bend right=60] (0,1) >
\tzpath[fill]
      (0,0)[to[bend right]]
      (1,1)
      (2,1)[to[out=-135,in=45]]
      (3,0); <--cycle>
\end{tikzpicture}
\end{tzcode}

\section{\protect\cmd{\tzpath*}: Semicolon version}
\label{s:tzpath*}

The starred version \icmd{\tzpath*} is the (almost) same as |\tzlinks*[draw=none]|. It fills the interior of |\tzpath| with |fill=black!50| with |fill opacity=.3| and |text opacity=1|, by default.
The only difference between |\tzpath*| and |\tzlinks*[draw=none]| is that the layer of |\tzpath*| can be changed by |\settzpathlayer|, like |\settzpathlayer{behind}|.

|\tzpath*| works like |\tzpath[fill=black!50,fill opacity=.3,text opacity=1]|.
You can change the defaults by \icmd{\settzfillcolor} and \icmd{\settzfillopacity}.

\begin{tzdef}
% syntax: minimal
\tzpath*(<coor>)(<coor>) ..repeated.. (<coor>) ; {<fill opacity>}
% syntax: full
\tzpath*{<path style>}[<opt>]<shift coor>"<path name>"
        (<coor>)[<code1>]{<text>}[<node opt>]<code2>
        ..repeated.. ()[]{}[]<> ; {<fill opacity>} <code.append>
% remark:
  - <code1> must be link style, such as --, to, -|, etc.
           (possibly followed by other codes)
  - <code2> is especially for + and ++
           (possibly with other codes)
  - repetition MUST be ended by ; (semicolon)
% defaults: \tzpath*
 *{to}[fill=black!50,fill opacity=.3,text opacity=1]<>""
  (<m>)[]{}[]<> ..repeated.. ()[]{}[]<> ; {.3}<>
\end{tzdef}



\paragraph{Filling the interior}

You can optionally change the opacity of fill color using the \Tikz\ option |fill opacity|.

\begin{tzcode}{.3}
% \tzpath: fill, fill opacity
\begin{tikzpicture}
\tzhelplines(4,3)
\tzline[red,very thick](0,3)(4,0)
\tzpath[fill](0,0)(3,1)(2,3)(0,2); % default opacity=1
\tzpath[fill,green,fill opacity=.5]
  <1,-1>(0,0)(3,1)(2,3)(0,2);
\end{tikzpicture}
\end{tzcode}

You can also change the fill opacity by specifying the \xem{last} curly brace optional argument |{<fill opacity>}|, \xem{after the semicolon}.

\begin{tzcode}{.3}
% \tzpath*
\begin{tikzpicture}
\tzhelplines(4,3)
\tzline[red,very thick](0,3)(4,0)
\tzpath*(0,0)(3,1)(2,3)(0,2);
\tzpath*[green]
  <1,-1>(0,0)(3,1)(2,3)(0,2); {.5} %%
\end{tikzpicture}
\end{tzcode}

You can also use the macros |\settzfillcolor| and |\settzfillopacity| to change the defaults.
The effect remains valid until the end of the |tikzpicture| environment, unless changed again.

\begin{tzcode}{.3}
% \tzpath*: \settzfillcolor
\begin{tikzpicture}
\tzhelplines(4,3)
\settzfillcolor{red}
\tzpath*[blue](0,1)(1,2)[-|](3,1)(4,3); 
             < to [bend right=60] (0,1) >
\tzpath*
      (0,0)[to[bend right]]
      (1,1)
      (2,1)[to[out=-135,in=45]]
      (3,0); <--cycle>
\end{tikzpicture}
\end{tzcode}


%%------------------------------------------------------------
\section{\protect\cmd{\tzpath+} and \protect\cmd{\tzpath*+}: Relative coordinates: Semicolon versions}
\label{s:tzpath+}

The \iisw{plus version} \icmd{\tzpath+} uses each coordinate (except for the first coordinate) relative (with |++|) to the previous coordinate.

\xem{Everything else is the same as in} |\tzpath|.

\icmd{\tzpath*+} is simply the plus version of |\tzpath*|.

\begin{tzcode}{.3}
% \tzpath+
\begin{tikzpicture}
\tzhelplines(4,3)
\tzline[red,very thick](0,3)(4,0)
\tzpath+[pattern=bricks,preaction={fill=brown}]
        (0,0)(3,1)(-1,2)(-2,-1);
\tzpath+[pattern=north east lines,opacity=.5]
  <1,-1>(0,0)(3,1)(-1,2)(-2,-1);
\end{tikzpicture}
\end{tzcode}

You can change the layer (|main| by default) using |\settzpathlayer|.

\begin{tzcode}{.3}
% \settzpathlayer
\begin{tikzpicture}
\tzhelplines(4,3)
\settzpathlayer{behind} %% layer
\tzline[red,very thick](0,3)(4,0)
\tzpath+[pattern=bricks,preaction={fill=brown}]
        (0,0)(3,1)(-1,2)(-2,-1);
\tzpath+[pattern=north east lines,opacity=.5]
  <1,-1>(0,0)(3,1)(-1,2)(-2,-1);
\end{tikzpicture}
\end{tzcode}

\begin{tzcode}{.3}
% \tzpath*+
\begin{tikzpicture}
\tzhelplines(4,3)
\tzline[red,very thick](0,3)(4,0)
\tzpath*+(0,0)(3,1)(-1,2)(-2,-1);
\tzpath*+[fill=green]
   <1,-1>(0,0)(3,1)(-1,2)(-2,-1); {.5} %%
\end{tikzpicture}
\end{tzcode}


%%=======================================
%%==================================
\chapter{Curves}
\label{c:curves}

There are many ways to draw curves.

%%-----------------------------------
%%------------------------------------------------------------
\section{B\'{e}zier curves}
\label{s:beziercurves}

\subsection{\protect\cmd{\tzbezier}}
\label{ss:tzbezier}

\icmd{\tzbezier} accepts \xem{three or four} coordinates to draw a B\'{e}zier curve from the first coordinate to the last coordinate, with \xem{one or two} control points.

\begin{tzdef}
% syntax: minimal
\tzbezier(<start-coor>)(cntl-coor>)(<last-coor>)
\tzbezier(<start-coor>)(cntl-coor>)(<cntl-coor>)(<last-coor>)
% syntax: full
\tzbezier[<draw opt>]<shift coor>"<name path>"
         (<start-coor>)(cntl-coor>)(<cntl-coor>)(<last-coor>)
         {<text>}[<node opt>]<code.append>
% defaults
  []<>""(<m>)(<m>)()(<m>){}[]<>
\end{tzdef}

\paragraph{Control points}

You can specify one or two control points, |(<cntl-coor>)|.

\begin{tztikz}
% three coordinates: one control point
\tzbezier(0,1)(1,0)(4,3) % works like:
  \draw (0,0) ..controls (1,0).. (4,3);
\end{tztikz}

\begin{tztikz}
% four coordinates: two control points
\tzbezier(0,1)(1,0)(2,4)(4,3) % works like:
  \draw (0,0) ..controls (1,0) and (2,4).. (4,3);
\end{tztikz}

\begin{tzcode}{.3}
% \tzbezier
\begin{tikzpicture}
\tzhelplines(4,4)
\tzbezier[red](0,3)(1,2)(2,4)(4,3)
\tzbezier[blue](0,2)(1,.5)(4,0)
\end{tikzpicture}
\end{tzcode}

The style \ixxw{tzshowcontrols} displays the control points by drawing dotted lines, by default.
You can also change the dotted line style, like |tzshowcontrols={dashed,green}|.


\begin{tzcode}{.3}
% \tzbezier: tzshowcontrols
\begin{tikzpicture}
\tzhelplines(4,4)
\tzbezier[blue,tzshowcontrols](0,2)(1,.5)(4,0)
\tzdots*[blue](0,2)(1,.5)(4,0);
\tzbezier[red,tzshowcontrols={green,dashed}]
         (0,3)(1,2)(2,4)(4,3)
\tzdots*[red](0,3)(1,2)(2,4)(4,3);
\end{tikzpicture}
\end{tzcode}

\paragraph{Adding text}
You can add text next to the curve or at the last coordinate
by specifying the options |{<text>}| and |[<node opt>]| \xem{immediately after the last coordinate}.


\begin{tzcode}{.3}
% \tzbezier: adding text
\begin{tikzpicture}
\tzhelplines(4,4)
\tzbezier[blue](0,2)(1,.5)(4,0){curve}[draw,black,r]
\tzbezier[red](0,3)(1,2)(2,4)(4,3){bezier}[near end,a]
\end{tikzpicture}
\end{tzcode}

\paragraph{Shift}
You can move the curve of |\tzbezier| by specifying the option |<shift coor>| before the first coordinate or immediately before the option |"<path name>"|, if it exists.
The \xem{empty} shift option |<>| is \xem{not allowed}.

\begin{tzcode}{.3}
% \tzbezier: shift
\begin{tikzpicture}
\tzhelplines(4,4)
\tzbezier[blue]
         (0,2)(1,.5)(4,0){curve}[draw,black,r]
\tzbezier[blue,dashed]
  <1,1>(0,2)(1,.5)(4,0){curve}[r]
\tzbezier[red]
         (0,3)(1,2)(2,4)(4,3){bezier}[near end,a]
\tzbezier[red,dashed]
  <0,-1>(0,3)(1,2)(2,4)(4,3){bezier}[blue,near end,a]
\end{tikzpicture}
\end{tzcode}

\paragraph{Naming paths: intersections}
You can name the path of |\tzbezier| by specifying the option |"<path name>"| \xem{immediately before} the first coordinate.

\begin{tzcode}{.3}
% \tzbezier: name path, intersection
\begin{tikzpicture}
\tzhelplines(4,4)
\tzbezier[blue,dashed,thick]<1,1>"curve"
  (0,2)(1,.5)(4,0){curve}[r]
\tzbezier[red,dashed,thick]<0,-1>"bezier"
  (0,3)(1,2)(2,4)(4,3){bezier}[blue,near end,a]
% intersection point
\tzXpoint*{curve}{bezier}(X){X}[-90]
\end{tikzpicture}
\end{tzcode}

\paragraph{Extending paths}
You can extend the path of |\tzbezier| from the last coordinate, by writing \Tikz\ code
in the last optional argument |<code.append>|.

\begin{tzcode}{.3}
% \tzbezier: extending path
\begin{tikzpicture}
\tzhelplines(4,4)
\tzbezier[red](0,3)(1,2)(2,4)(4,3){bezier}[near end,a]
  < to [bend right] ++(-1,-1) node [below] {A} >
\tzbezier[blue,->](0,2)(1,.5)(4,0)
  < |- ++(-2,1) node [below] {B} >
\end{tikzpicture}
\end{tzcode}

You can also use \icmd{\tzbezierAtBegin} and \icmd{\tzbezierAtEnd} to extend the path of |\tzbezier| at the beginning and at the end, respectively.
Specifying the option |<code.append>| extends the path after |\tzbezierAtEnd|.

\begin{tzcode}{.3}
% \tzbezier: extending path
\begin{tikzpicture}
\tzhelplines(4,4)
\tzbezierAtEnd
  { to [bend right] ++(-1,-1) node [below] {A} }
\tzbezier[red](0,3)(1,2)(2,4)(4,3){bezier}[near end,a]
\tzbezierAtBegin{ (0,0) to [out=45,in=-30] }
\tzbezierAtEnd{ |- ++(-2,1) }
\tzbezier[blue,->](0,2)(1,.5)(4,0){B}[b]
  < arc (0:90:1) node [a] {End!} >
\end{tikzpicture}
\end{tzcode}


\subsection{\protect\cmd{\tzbezier+}: Relative coordinates}
\label{ss:tzbezier+}

For the \iisw{plus version} \icmd{\tzbezier+}, the last coordinate is \xem{relative} to the first coordinate.
And the first control point is \xem{relative} to the first coordinate.

\begin{tztikz}
% three coordinates
\tzbezier+(0,1)(1,-1)(4,3) % works like:
  \draw (0,1) ..controls +(1,-1).. ($(0,1)+(4,3)$);
\end{tztikz}

In \Tikz, the second control point is \xem{relative} to the last coordinate.

Therefore, for |\tzbezier+(A)(B)(C)(D)|, |(B)| and |(D)| are \xem{relative} to |(A)|, and the second control point |(C)| is \xem{relative} to the last coordinate |(D)|.

\begin{tztikz}
% four coordinates
\tzbezier+(0,1)(1,-1)(-2,1)(4,3) % works like:
  \draw (0,1) ..controls +(1,-1) and +(-2,1).. ($(0,1)+(4,3)$);
\end{tztikz}

\begin{tzcode}{.3}
% \tzbezier+
\begin{tikzpicture}
\tzhelplines(4,4)
\tzbezier+(0,3)(1,-1)(-2,1)(4,0){bezier}[near end,a]
  < to [bend right] ++(-1,-1) node [below] {A} >
\tzbezier+[blue,->](0,2)(1,-1.5)(4,-2)
  < |- ++(-2,1) node [above] {B} >
\end{tikzpicture}
\end{tzcode}


\begin{tzcode}{.3}
% \tzbezierAtBegin, \tzbezierAtEnd
\begin{tikzpicture}
\tzhelplines(4,4)
\tzbezierAtEnd
  { to [bend right] ++(-1,-1) node [below] {A} }
\tzbezier+(0,3)(1,-1)(-2,1)(4,0){bezier}[near end,a]
\tzbezierAtBegin{ (1,0) to [out=150,in=30] }
\tzbezierAtEnd{ |- ++(-2,1) node [a] {here!} }
\tzbezier+[blue,->](0,2)(1,-1.5)(4,-2){B}[b]
\end{tikzpicture}
\end{tzcode}


%%-----------------------------------
%%------------------------------------------------------------
\section{Parabolas}
\label{s:parabolas}


\subsection{\protect\cmd{\tzparabola}}
\label{ss:tzparabola}

\icmd{\tzparabola} accepts \xem{two or three} coordinates to draw a parabola from the first coordinate to the last coordinate.
In the case of three coordinates, the parabola bends at the second coordinate.

\begin{tztikz}
% two coordinates
\tzparabola(0,0)(3,2) % works like:
  \draw (0,0) parabola (3,2);
\end{tztikz}

\begin{tztikz}
% three coordinates
\tzparabola(0,0)(1,1)(3,2) % works like:
  \draw (0,0) parabola bend (1,1) (3,2);
\end{tztikz}


\begin{tzdef}
% syntax: minimal
\tzparabola(<coor>)(<coor>)
\tzparabola(<coor>)(<coor>)(<coor>)
% syntax: full
\tzparabola[<opt1>]<shift coor>"<path name>"
           (<coor>)(<coor>)(<coor>){<text>}[<node opt>]<code.append>
% default
  []<>""(<m>)()(<m>){}[]<>
\end{tzdef}

\paragraph{Parabolas}
|\tzplot| draws the graph of a quadratic function $f(x)=ax^2+bx+c$ for appropriate values of $a$, $b$, and $c$.

\begin{tzcode}{.3}
% \tzparabola: two coordinates
\begin{tikzpicture}
\tzhelplines(4,3)
\tzparabola[thick](0,0)(2,2)
\tzparabola[bend at end,dashed](0,0)(2,2)
\tzparabola[red,thick,bend pos=.5](2,0)(4,3)
\tzparabola[blue,parabola height=-2cm](2,3)(4,2)
\end{tikzpicture}
\end{tzcode}

\begin{tzcode}{.3}
% \tzparabola: three coordinates
\begin{tikzpicture}
\tzhelplines(4,3)
\tzparabola[thick](0,2)(1,0)(2,2)
\tzparabola[blue](0,3)(2,1)(4,2)
\tzparabola[red,thick](2,0)(3,2)(4,3)
\end{tikzpicture}
\end{tzcode}

\paragraph{Adding text}
You can add text at or around the \xem{last coordinate} by specifying the options |{<text>}| and |[<node opt>]| \xem{immediately after the last coordinate}.

\begin{tzcode}{.3}
% \tzparabola: adding text
\begin{tikzpicture}
\tzhelplines(4,3)
\tzparabola[thick](0,0)(2,2){parabola}[blue,draw,a]
\tzparabola[blue](0,3)(2,1)(4,2){AC}[r]
\tzparabola[red,thick,bend pos=.5](2,0)(4,3){C}[ar]
\end{tikzpicture}
\end{tzcode}

\paragraph{Shift}
You can move the parabola by specifying the option |<shift coor>| before the first coordinate or immediately before the option |"<path name>"|, if it exist.
The \xem{empty} shift option |<>| is \xem{not allowed}.

\begin{tzcode}{.3}
% \tzparabola: shift
\begin{tikzpicture}
\tzhelplines(4,3)
\tzparabola[blue]         (0,3)(2,1)(4,2){AC}[r]
\tzparabola[dashed]<0,-.5>(0,3)(2,1)(4,2){AC}[r]
\tzparabola[dotted]<0,-1> (0,3)(2,1)(4,2){AC}[r]
\end{tikzpicture}
\end{tzcode}

\paragraph{Naming paths: intersections}
You can name the path of |\tzparabola| by specifying the option |"<path name>"| \xem{immediately before} the first coordinate.

\begin{tzcode}{.3}
% \tzparabola: name path, intersection
\begin{tikzpicture}
\tzhelplines(4,3)
\tzparabola[blue]"AC"(0,3)(2,1)(4,2){AC}[r]
\tzparabola"BB"(1,0)(2,2)(3,0)
% intersection points
\tzXpoint*{AC}{BB}(X)
\tzdot(X-2)(3pt)
\end{tikzpicture}
\end{tzcode}

\paragraph{Extending paths}
You can extend the path of |\tzparabola| from the last coordinate by writing \Tikz\ code in the last optional argument |<code.append>|.

\begin{tzcode}{.3}
% \tzparabola: extending path
\begin{tikzpicture}
\tzhelplines(4,3)
\tzparabola[blue,->]"AC"(0,3)(2,1)(4,2)
  < -- ++ (-1,1) >
\tzparabola"BB"(1,0)(2,2)(3,0)
  < -| ++ (1,1) node [right] {K!} >
\end{tikzpicture}
\end{tzcode}

You can also use the macros \icmd{\tzparabolaAtBegin} and \icmd{\tzparabolaAtEnd} to extend the path of |\tzparabola| at the beginning and at the and, respectively.
Specifying the option |<code.append>| extends the path after |\tzparabolaAtEnd|.

\begin{tzcode}{.3}
% \tzparabola: extending path
\begin{tikzpicture}
\tzhelplines(4,3)
\tzparabolaAtEnd{ -- ++ (-1,1) }
\tzparabola[blue,->]"AC"(0,3)(2,1)(4,2)
\tzparabolaAtBegin{ (0,1) to [bend right ] }
\tzparabolaAtEnd{ -| ++ (1,1) node [right] {End!} }
\tzparabola"BB"(1,0)(2,2)(3,0){K!}[bl]
  < arc (-90:0:1) node [a] {here!} >
\end{tikzpicture}
\end{tzcode}



\subsection{\protect\cmd{\tzparabola+}: Relative coordinates}
\label{ss:tzparabola+}

The \iisw{plus version} \icmd{\tzparabola+} uses the second and the third coordinates \xem{relative} to the first coordinate.

\xem{Everything else is the same as in} |\tzparabola|.

\begin{tztikz}
% two coordinates
\tzparabola+(0,1)(1,2) % works like:
  \draw (0,1) parabola ($(0,1)+(1,2)$);
\end{tztikz}


\begin{tztikz}
% three coordinates
\tzparabola+(0,1)(1,1)(3,-1) % works like:
  \draw (0,1) parabola bend +(1,1) ($(0,1)+(3,-1)$);
\end{tztikz}

\begin{tzcode}{.3}
% \tzparabola+
\begin{tikzpicture}
\tzhelplines(4,3)
\tzparabola+[thick](0,0)(2,2){parabola}[blue,draw,a]
\tzparabola+[blue](0,3)(2,-2)(4,-1){AC}[r]
\tzparabola+[red,thick,bend pos=.5](2,0)(2,3){C}[ar]
\end{tikzpicture}
\end{tzcode}



%%------------------------------------------------------------
\section{Edges}
\label{s:edge}

%%------------------------------------------------------------
\subsection{\protect\cmd{\tzedge(+)}}
\label{ss:tzedge}

The macro \icmd{\tzedge} connects two coordinates with a straight or curved line, using the \Tikz\ |edge| operation.


\begin{tzdef}
% syntax
\tzedge[<opt>]<shift coor>
       (<coor>){<text>}[<opt>](<coor>){<text>}[<opt>]<code.append>
% defaults
  []<>(<m>){}[](<m>){}[]<>
\end{tzdef}

\remark In \Tikz, the |edge| operation works like the |to| operation, but it is independently drawn after a main path is drawn and does not form a main path. So the option |"<path name>"| (for finding intersections) is not provided in |\tzedge|.


\begin{tztikz}
\tzedge(1,1)(3,2) % works like:
  \draw (1,1) edge (3,2);
\end{tztikz}

\begin{tztikz}
\tzedge[->,bend right](1,1){A}[near start](3,2) % works like:
  \draw (1,1) edge [->,bend right] node [near start] {A} (3,2);
\end{tztikz}

\begin{tzcode}{.3}
% \tzedge
\begin{tikzpicture}
\tzhelplines(4,3)
\tzedge[->](0,1)(2,3)
\tzedge[out=0,dashed]
       (0,0){edge}[near start,sloped,a](4,2)
\tzedge[in=90,blue](2,0)(4,1){C}[r]
\end{tikzpicture}
\end{tzcode}

The \xem{plus version} \icmd{\tzedge+} use the second coordinate \xem{relative} to the first coordinate.

\begin{tztikz}
\tzedge+(1,1)(3,2) % works like:
  \draw (1,1) edge ++(3,2);
\end{tztikz}

\begin{tztikz}
\tzedge+[->,bend right](1,1){A}[near start](3,2){B}[right] % works like:
  \draw (1,1) edge [->,bend right] node [near start] {A} ++(3,2)
      ++(3,2) node [right] {B};
\end{tztikz}

\begin{tzcode}{.3}
% \tzedge+: shift
\begin{tikzpicture}
\tzhelplines(4,3)
\tzedge+[->](0,1)(2,2)
\tzedge+[out=0,dashed]
       (0,0){edge}[near start,sloped,a](4,2)
\tzedge+[in=90,blue]<-.5,.5>(2,0)(2,1){shifted}[b,draw]
\end{tikzpicture}
\end{tzcode}


\remark |edge| in the option |<code.append>| works with the second coordinate, but it works with the last node if the second option |{<text>}| is not empty.
(In \Tikz, if |edge| is preceded by |node|, the node is its start point.)

\begin{tzcode}{.3}
% <code.append>
\begin{tikzpicture}
\tzhelplines(4,3)
\tzedge[->](0,1)(2,3)<edge[bend left,red] ++(1,-2)>
\tzedge[out=0,dashed]
       (0,0){edge}[near start,sloped,a]
       (4,2){A}[draw]
       <edge[bend left,thick] ++ (-1,-2)>
\end{tikzpicture}
\end{tzcode}

%%------------------------------------------------------------
\subsection{\protect\cmd{\tzedges(+)}: Semicolon versions}
\label{ss:tzedges}

The macro \icmd{\tzedges} accepts any number of coordinates to draw edges from the first coordinate to each of the next using the \Tikz\ |edge| operation. That is, the first coordinate is the unique start coordinate, and all others are target coordinates.
|\tzedges| is a semicolon version, so you need to type |;| to indicate when the coordinate iteration ends.


\begin{tzdef}
% syntax: minimum
\tzedges(<coor>)(<coor>)..repeated..(<coor>);
% syntax: full
\tzedges[<opt>]<shift coor>
        (<coor>)[<edge opt>]{<text>}[<opt>]..repeated..()[]{}[] ; <code.append>
% defaults
  [](<m>)[]{}[]..repeated..()[]{}[];
\end{tzdef}

\remark In \Tikz, the |edge| operation works like |to| operation, but it is added after main path is formed, like |node| does.
\begin{itemize}\firmlist
\item Each |edge| is drawn independently from a main path as well as any other |edge|'s. (See \Tikz\ manual for more details.)
\item The |edge| operation of \Tikz\ does not change change anything about a main path, so the current point is not changed by |\tzedges|. This means that the last node (with |{<text>}| and |[<node opt>]|) works with the first (namely, start) coordinate.
\end{itemize}





\begin{tzcode}{.3}
% \tzedges
\begin{tikzpicture}
\tzhelplines(4,3)
\tzedges(0,1){A}                  (1,2)
             [blue,thick]{edge}[b](3,1)
                                  (4,3);
\end{tikzpicture}
\end{tzcode}

\begin{tztikz}
\tzedges(0,1){A}(1,2)[blue,thick]{edge}[a](3,1)(4,3); % works like
  \draw (0,1) edge              node         {A}    (1,2)
              edge [blue,thick] node [above] {edge} (3,1)
              edge                                  (4,3) ;
\end{tikzpicture}
\end{tztikz}

\begin{tzcode}{.3}
% \tzedges: shift
\begin{tikzpicture}
\tzhelplines(4,3)
\tzedges[bend right]<1,-1> % shift
        (0,1)[->,thick,blue]{A}
        (1,2)[dotted]{Beta}[b]
        (3,1)[dashed]{Cate}[near end,sloped,a]
        (4,3);
\end{tikzpicture}
\end{tzcode}


The \xem{plus version} \icmd{\tzedges+} uses the second and next coordinates \xem{relative} to the first coordinate.

\begin{tztikz}
\tzedges+(0,1)(2,1)(3,0)(4,2); % works like
  \draw (0,1) edge ++ (2,1)
              edge ++ (3,0)
              edge ++ (4,2);
\end{tztikz}

\begin{tzcode}{.3}
% \tzedges+
\begin{tikzpicture}
\tzhelplines(4,3)
\tzedges+[bend right](0,1)[bend right=0]{A}
         (1,1)[dotted]{Beta}[b]
         (3,0)[dashed]{Cate}[near end,sloped,a]
         (4,2);
\end{tikzpicture}
\end{tzcode}


\remark The option |<code.append>| works with the first coordinate because |edge| and |node| do not change the current point.


\begin{tzcode}{.3}
% <code.append>
\begin{tikzpicture}
\tzhelplines(4,3)
\tzedges+(0,1)[bend right=0,->]{A}[fill=white]
         (1,1)[bend right,dotted]{Beta}[midway]
         (3,0)[bend right,dashed]{Cate}[near end]
         (4,2);
         < edge [bend left,thick,blue] (1,3) >
\end{tikzpicture}
\end{tzcode}



%%------------------------------------------------------------
\section{More curves}

\subsection{\protect\cmd{\tzplotcurve}, \protect\cmd{\tzplot}}

You can draw curves with |\tzplotcurve| and |\tzplot|.

\begin{tzcode}{.3}
% \tzplotcurve(*)
\begin{tikzpicture}
\tzhelplines(4,3)
\tzplotcurve*[red,text=blue]
  (0,0)(1,2){B}(2,2)(3,3)(4,1){E}[0];
\tzplotcurve(1,.5)(2,3)(3,2)(2,1)(3,1);
\end{tikzpicture}
\end{tzcode}


See Section \ref{s:tzplotcurve} on page \pageref{s:tzplotcurve}, for more details on |\tzplotcurve|.

See Section \ref{s:tzplot} on page \pageref{s:tzplot}, for more details on |\tzplot|.


\subsection{\protect\cmd{\tzto}, \protect\cmd{\tztos}}
\label{ss:curves:tzto}

You can draw curves with |\tzto| and |\tztos|.
|\tztos| are quite useful to draw various curves.

\begin{tzcode}{.3}
% \tztos
\begin{tikzpicture}
\tzhelplines(4,3)
\tzcoors*(0,0)(A)(1,2)(B){B}[b](3,3)(C)(4,2)(D);
\tztos(A)[out=80,in=180]
      (B)[out=0,in=180]
      (C)[out=0,in=180]
      (D){End!}[r];
\end{tikzpicture}
\end{tzcode}

See Section \ref{s:tzto} on page \pageref{s:tzto}, for more details on |\tzto|.

See Section \ref{s:tztos} on page \pageref{s:tztos}, for more details on |\tztos|.

\subsection{\protect\cmd{\tzlink}, \protect\cmd{\tzlinks}}
\label{ss:curves:tzlink}

You can also draw curves with |\tzlink| and |\tzlinks|.

\begin{tzcode}{.3}
% \tztos
\begin{tikzpicture}
\tzhelplines(4,3)
\tzcoors*(0,0)(A)(1,2)(B){B}[b](3,3)(C)(4,2)(D);
\tzlinks(A)[to[out=80,in=180]]
      (B)[to[out=0,in=180]]
      (C)[to[out=0,in=180]]
      (D){End!}[r];
\end{tikzpicture}
\end{tzcode}

See Section \ref{s:tzlink} on page \pageref{s:tzlink}, for more details on |\tzlink|.

See Section \ref{s:tzlink} on page \pageref{s:tzlinks}, for more details on |\tztos|.

\subsection{\protect\cmd{\tzfn}}
\label{ss:curves:tzfn}

With \icmd{\tzfn}, you can plot functions such as $f(x)=\frac13(x-1)^3+1$, $g(x)=\sin x$, $h(x)=\sqrt{x-1}$, and so on.
See Section \ref{s:tzfn} on page \pageref{s:tzfn}, for more details on |\tzfn|.






