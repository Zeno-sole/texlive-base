\label{func}

%\subsection{Courbe de Bezier}
\SbSSCT{Courbe de Bezier}{Bezier curve}

\begin{tabular}{|c|c|c|} \hline  
\begin{psgraph*}[axesstyle=none,xticksize= -2 2 ,yticksize=-2 2, subticks=0](0,0)(-2,-2)(2,2){2.5cm}{2.5cm } 
 \psBezier1[showpoints=true]{<->}(-2,-1)(0,2)
\end{psgraph*}
&  
\begin{psgraph*}[axesstyle=none,xticksize= -2 2 ,yticksize=-2 2, subticks=0](0,0)(-2,-2)(2,2){2.5cm}{2.5cm } 
 \psBezier2[showpoints=true]{<->}(-2,-1)(0,2)(2,1)
\end{psgraph*}
&  
\begin{psgraph*}[axesstyle=none,xticksize= -2 2 ,yticksize=-2 2, subticks=0](0,0)(-2,-2)(2,2){2.5cm}{2.5cm } 
 \psBezier3[showpoints=true]{<->}(-2,-1)(0,2)(2,1)(1,-1)
\end{psgraph*}
\\ \hline 
 \BSS{psBezier1} &   \BSS{psBezier2} & \BSS{psBezier3}  
\\ \hline 
\begin{psgraph*}[axesstyle=none,xticksize= -2 2 ,yticksize=-2 2, subticks=0](0,0)(-2,-2)(2,2){2.5cm}{2.5cm } 
 \psBezier4[showpoints=true]{<->}(-2,-1)(0,2)(2,1)(2,-1)(0,-2)
\end{psgraph*}
&  
\begin{psgraph*}[axesstyle=none,xticksize= -2 2 ,yticksize=-2 2, subticks=0](0,0)(-2,-2)(2,2){2.5cm}{2.5cm } 
 \psBezier5[showpoints=true]{<->}(-2,-1)(0,2)(2,1)(2,-1)(0,-2)(-1,-1)
\end{psgraph*}
&  
\begin{psgraph*}[axesstyle=none,xticksize= -2 2 ,yticksize=-2 2, subticks=0](0,0)(-2,-2)(2,2){2.5cm}{2.5cm } 
 \psBezier6[showpoints=true]{<->}(-2,-1)(0,2)(2,1)(2,-1)(0,-2)(-1,-1)(-2,1)
\end{psgraph*}
\\ \hline
 \BSS{psBezier4} &   \BSS{psBezier5} & \BSS{psBezier6} 
\\ \hline 
\begin{psgraph*}[axesstyle=none,xticksize= -2 2 ,yticksize=-2 2, subticks=0](0,0)(-2,-2)(2,2){2.5cm}{2.5cm } 
 \psBezier7[showpoints=true]{<->}(-2,-1)(0,2)(2,1)(2,-1)(0,-2)(-1,-1)(-2,1)(-1,2)
\end{psgraph*}
&  
\begin{psgraph*}[axesstyle=none,xticksize= -2 2 ,yticksize=-2 2, subticks=0](0,0)(-2,-2)(2,2){2.5cm}{2.5cm } 
 \psBezier8[showpoints=true]{<->}(-2,-1)(0,2)(2,1)(2,-1)(0,-2)(-1,-1)(-2,1)(-1,2)(1,1)
\end{psgraph*}
&  
\begin{psgraph*}[axesstyle=none,xticksize= -2 2 ,yticksize=-2 2, subticks=0](0,0)(-2,-2)(2,2){2.5cm}{2.5cm } 
 \psBezier9[showpoints=true]{<->}(-2,-1)(0,2)(2,1)(2,-1)(0,-2)(-1,-1)(-2,1)(-1,2)(1,1)(1,0)
\end{psgraph*}
\\ \hline
 \BSS{psBezier7} &   \BSS{psBezier8} & \BSS{psBezier9} 
 \\ \hline  
\end{tabular} 

\newpage
%============================
%\subsection{Polynôme de Chebyshev }
\SbSSCT{Polynôme de Chebyshev }{Chebyshev polynomial}

\subsubsection{Polynôme de première espèce}

\begin{tabular}{|c|c|c|} \hline
\multicolumn{3}{|c|}{ \BS{psplot}\AC{-1}\AC{1}\AC{1 x \BSS{ChebyshevT} }  \BSI{ChebyshevT}{pst-func} } 
\\ \hline   
\begin{psgraph*}[axesstyle=none,xticksize= -1.5 1.5 ,yticksize=-1.5 1.5 , subticks=0 ](0,0)(-1.5,-1.5)(1.5,1.5){3.5cm}{3.5cm}  
 \psplot{-1}{1}{1 x \ChebyshevT}
\end{psgraph*}
&  
\begin{psgraph*}[axesstyle=none,xticksize= -1.5 1.5 ,yticksize=-1.5 1.5 , subticks=0](0,0)(-1.5,-1.5)(1.5,1.5){3.5cm}{3.5cm} 
 \psplot{-1}{1}{3 x \ChebyshevT}
\end{psgraph*}
&  
\begin{psgraph*}[axesstyle=none,xticksize= -1.5 1.5 ,yticksize=-1.5 1.5 , subticks=0](0,0)(-1.5,-1.5)(1.5,1.5){3.5cm}{3.5cm}   
 \psplot{-1}{1}{6 x \ChebyshevT}
\end{psgraph*}
\\ \hline  
1 x \BSS{ChebyshevT} & 3 x \BSS{ChebyshevT} & 6 x \BSS{ChebyshevT} \\ 
\hline 
\end{tabular} 

\subsubsection{Polynôme de deuxième espèce }

\begin{tabular}{|c|c|c|} \hline
\multicolumn{3}{|c|}{ \BS{psplot}\AC{-1}\AC{1}\AC{1 x \BSS{ChebyshevU} }  \BSI{ChebyshevU}{pst-func} } 
\\ \hline   
\begin{psgraph*}[axesstyle=none,xticksize= -1.5 1.5 ,yticksize=-1.5 1.5 , subticks=0 ](0,0)(-1.5,-1.5)(1.5,1.5){3.5cm}{3.5cm} 
 \psplot{-1}{1}{1 x \ChebyshevU}
\end{psgraph*}
&  
\begin{psgraph*}[axesstyle=none,xticksize= -1.5 1.5 ,yticksize=-1.5 1.5 , subticks=0](0,0)(-1.5,-1.5)(1.5,1.5){3.5cm}{3.5cm} 
 \psplot{-1}{1}{3 x \ChebyshevU}
\end{psgraph*}
&  
\begin{psgraph*}[axesstyle=none,xticksize= -1.5 1.5 ,yticksize=-1.5 1.5 , subticks=0](0,0)(-1.5,-1.5)(1.5,1.5){3.5cm}{3.5cm}  
 \psplot{-1}{1}{6 x \ChebyshevU}
\end{psgraph*}
\\ \hline  
1 x \BSS{ChebyshevU} & 3 x \BSS{ChebyshevU} & 6 x \BSS{ChebyshevU} \\ 
\hline 
\end{tabular}

%ù+++++++++++++

\newpage

%\subsection{Fonction polynomiale}
\SbSSCT{Fonction polynomiale}{Function plynomial}

\begin{tabular}{|c|c|c|c|}	\hline 
\multicolumn{4}{|c|}{ \BSS{psPolynomial}[coeff= 1 ]\AC{-2}\AC{2} \BSI{psPolynomial}{pst-func} }
\\ \hline 
\begin{psgraph*}[axesstyle=none,xticksize= -2 2 ,yticksize=-2 2, subticks=0](0,0)(-2,-2)(2,2){2.5cm}{2.5cm } 
  \psPolynomial[coeff= 1 ]{-2}{2}
\end{psgraph*}
&
\begin{psgraph*}[axesstyle=none,xticksize= -2 2 ,yticksize=-2 2, subticks=0](0,0)(-2,-2)(2,2){2.5cm}{2.5cm } 
  \psPolynomial[coeff= 0 1 ]{-2}{4}
\end{psgraph*}
&
\begin{psgraph*}[axesstyle=none,xticksize= -2 2 ,yticksize=-2 2, subticks=0](0,0)(-2,-2)(2,2){2.5cm}{2.5cm } 
  \psPolynomial[coeff=0 0 1 ]{-2}{4}
\end{psgraph*}
&
\begin{psgraph*}[axesstyle=none,xticksize= -2 2 ,yticksize=-2 2, subticks=0](0,0)(-2,-2)(2,2){2.5cm}{2.5cm } 
  \psPolynomial[coeff=0 0 0 1 ]{-2}{4}
\end{psgraph*}
\\ 	\hline  
\RDD{coeff}= 1 & \RDD{coeff}=0  1 & \RDD{coeff}=0  1 & \RDD{coeff}=0 0 01
 \RDI{coeff}{pst-func} \\
$f(x)=1$ & $f(x)=x$ & $f(x)=x^2$ & $f(x)=x^3 $
\\ 	\hline 
\end{tabular}


 
 \bigskip

\begin{tabular}{|c|} \hline  
\begin{psgraph*}[axesstyle=none,xticksize= -2 2 ,yticksize=-2 4, subticks=0](0,0)(-2,-2)(4,2){4cm}{3cm } 
  \psPolynomial[coeff=0 0 0 1 ]{-2}{4}
  \psPolynomial[coeff=0 0 0 1 ,linecolor=red,xShift=2 ]{-2}{4}
\end{psgraph*}
\\ \hline  
\BS{psPolynomial}[coeff=0 0 0 1 ,linecolor=red,\RDD{xShift}=2 ]\AC{-2}\AC{4}
\\ \hline 
\end{tabular}  

 \bigskip


\begin{tabular}{|c|c|c|} \hline 
\multicolumn{3}{|c|}{ \BSS{psPolynomial}[coeff=0 0 0 0 0 1 ,linecolor=red,\RDD{Derivation}=1 ]\AC{-2}\AC{2}  \RDI{Derivation}{pst-func} } \\ \hline 
\begin{psgraph*}[axesstyle=none,xticksize= -2 2 ,yticksize=-2 2, subticks=0](0,0)(-2,-2)(2,2){3.5cm}{3.5cm} 
  \psPolynomial[coeff=0 0 0 0 0 1 ]{-2}{4}
  \psPolynomial[coeff=0 0 0 0 0 1 ,linecolor=red,Derivation=1 ]{-2}{4}
\end{psgraph*}
&
\begin{psgraph*}[axesstyle=none,xticksize= -2 2 ,yticksize=-2 2, subticks=0](0,0)(-2,-2)(2,2){3.5cm}{3.5cm} 
  \psPolynomial[coeff=0 0 0 0 0 1 ]{-2}{4}
  \psPolynomial[coeff=0 0 0 0 0 1 ,linecolor=red,Derivation= 2 ]{-2}{4}
\end{psgraph*}
&
\begin{psgraph*}[axesstyle=none,xticksize= -2 2 ,yticksize=-2 2, subticks=0](0,0)(-2,-2)(2,2){3.5cm}{3.5cm} 
  \psPolynomial[coeff=0 0 0 0 0 1 ]{-2}{4}
  \psPolynomial[coeff=0 0 0 0 0 1 ,linecolor=red,Derivation= 3 ]{-2}{4}
\end{psgraph*}
\\ \hline 
\RDD{Derivation}= 1 & \RDD{Derivation}= 2 & \RDD{Derivation}= 3
\\ \hline 
\end{tabular}  



\bigskip
\begin{tabular}{|c|} \hline 
\begin{psgraph*}[axesstyle=none,xticksize= -2 2 ,yticksize=-3 3, subticks=0](0,0)(-3,-2)(3,2){6cm}{4cm } 
 \psPolynomial[markZeros,dotscale=3,coeff=1 1 -1 -.5 0.15]{-3}{3}%
\end{psgraph*}
\\ \hline
 \BS{psPolynomial}[\RDD{markZeros},dotscale=3,coeff=1 1 -1 -.5 0.15]\AC{-3}\AC{3}%
\\ \hline 
\end{tabular}

%----------------------

\bigskip
\begin{tabular}{|c|} \hline 
\begin{psgraph*}[axesstyle=none,xticksize= -3 2 ,yticksize=-3 3, subticks=0](0,0)(-3,-3)(3,2){8cm}{5cm } 
 \psPolynomial[markZeros,dotscale=2,zeroLineTo=1,coeff=1 1 -1 -.5 0.15]{-3}{3}%
 \psPolynomial[linestyle=dotted,Derivation=1,coeff=1 1 -1 -.5 0.15]{-3}{3}% 
\end{psgraph*}
\\ \hline
 \BS{psPolynomial}[markZeros,\RDD{zeroLineTo}=1,coeff=1 1 -1 -.5 0.15]\AC{-3}\AC{3} \\
  \BS{psPolynomial}[linestyle=dotted,Derivation=1,coeff=1 1 -1 -.5 0.15]\AC{-3}\AC{3}
\\ \hline 
\end{tabular}

\bigskip
\begin{tabular}{|c|} \hline 
\begin{psgraph*}[axesstyle=none,xticksize= -1 2 ,yticksize=-2 2, subticks=0](0,0)(-2,-1)(2,2){8cm}{4cm } 
 \psPolynomial[coeff=1 1 -1 -.5 0.15]{-3}{3}%
 \psPolynomial[markZeros,linestyle=dotted,Derivation=1,zeroLineTo=0,zeroLineTo=0,zeroLineStyle=solid,zeroLineColor=red,zeroLineWidth=3pt,coeff=1 1 -1 -.5 0.15]{-3}{3}% 
\end{psgraph*}
\\ \hline
 \BS{psPolynomial}[coeff=1 1 -1 -.5 0.15]\AC{-3}\AC{3} \\
  \BS{psPolynomial}[markZeros,linestyle=dotted,Derivation=1,zeroLineTo=0,\\
 \RDD{zeroLineStyle}=solid,\RDD{zeroLineColor}=red,\RDD{zeroLineWidth}=3pt,\\
 coeff=1 1 -1 -.5 0.15]\AC{-3}\AC{3}
\\ \hline 
\end{tabular}


\bigskip
\begin{tabular}{|c|} \hline 
\begin{psgraph*}[axesstyle=none,xticksize= -1 2 ,yticksize=-3 3, subticks=0](0,0)(-2,-1)(1,2){8cm}{4cm } 
 \psPolynomial[coeff=1 1 -1 -.5 0.15]{-2}{1}%
 \psPolynomial[markZeros,linestyle=dotted,Derivation=2,zeroLineTo=0,zeroLineStyle=solid,zeroLineColor=red,zeroLineWidth=3pt,coeff=1 1 -1 -.5 0.15]{-2}{1}% 
\end{psgraph*}
\\ \hline
 \BS{psPolynomial}[coeff=1 1 -1 -.5 0.15]\AC{-3}\AC{3} \\
  \BS{psPolynomial}[markZeros,linestyle=dotted,Derivation=2,zeroLineTo=0,\\
\hspace{1cm} \RDD{zeroLineStyle}=solid,\RDD{zeroLineColor}=red,\RDD{zeroLineWidth}=3pt,\\
 coeff=1 1 -1 -.5 0.15]\AC{-3}\AC{3}
\\ \hline 
\end{tabular}

\newpage

% \subsection{Polynôme de Bernstein}
 \SbSSCT{Polynôme de Bernstein}{Bernstein polynomial}
 
 
\begin{tabular}{|c|c|c|} \hline  
\begin{psgraph*}[axesstyle=none,xticksize= -.5 1.5 ,yticksize=-.5 1.5,xticksize= -.5  1.5 , dx=.5,Dx=.5, dy=.5,Dy=.5 , subticks=0] (0,0)(-.5,-.5)(1.5,1.5){3.5cm}{3.5cm }  
\psBernstein(0,0)
\end{psgraph*}
&  
\begin{psgraph*}[axesstyle=none,xticksize= -.5 1.5 ,yticksize=-.5 1.5,xticksize= -.5  1.5 , dx=.5,Dx=.5, dy=.5,Dy=.5 , subticks=0] (0,0)(-.5,-.5)(1.5,1.5){3.5cm}{3.5cm }    
\psBernstein(0,1)
\end{psgraph*}
&  
\begin{psgraph*}[axesstyle=none,xticksize= -.5 1.5 ,yticksize=-.5 1.5,xticksize= -.5  1.5 , dx=.5,Dx=.5, dy=.5,Dy=.5 , subticks=0] (0,0)(-.5,-.5)(1.5,1.5){3.5cm}{3.5cm }   
\psBernstein(1,1)
\end{psgraph*}
\\ \hline  
\BSS{psBernstein}(0,0) \BSI{psBernstein}{pst-func} & \BSS{psBernstein}(0,1) & \BSS{psBernstein}(1,1)
\\ \hline 
\begin{psgraph*}[axesstyle=none,xticksize= -.5 1.5 ,yticksize=-.5 1.5,xticksize= -.5  1.5 , dx=.5,Dx=.5, dy=.5,Dy=.5 , subticks=0] (0,0)(-.5,-.5)(1.5,1.5){3.5cm}{3.5cm }    
\psBernstein(0,2)
\end{psgraph*}
&  
\begin{psgraph*}[axesstyle=none,xticksize= -.5 1.5 ,yticksize=-.5 1.5,xticksize= -.5  1.5 , dx=.5,Dx=.5, dy=.5,Dy=.5 , subticks=0] (0,0)(-.5,-.5)(1.5,1.5){3.5cm}{3.5cm }   
\psBernstein(1,2)
\end{psgraph*}
&  
\begin{psgraph*}[axesstyle=none,xticksize= -.5 1.5 ,yticksize=-.5 1.5,xticksize= -.5  1.5 , dx=.5,Dx=.5, dy=.5,Dy=.5 , subticks=0] (0,0)(-.5,-.5)(1.5,1.5){3.5cm}{3.5cm }    
\psBernstein(2,2)
\end{psgraph*}  
\\ \hline
\BSS{psBernstein}(0,2) & \BSS{psBernstein}(1,2) & \BSS{psBernstein}(2,2)
\\ \hline   
\begin{psgraph*}[axesstyle=none,xticksize= -.5 1.5 ,yticksize=-.5 1.5,xticksize= -.5  1.5 , dx=.5,Dx=.5, dy=.5,Dy=.5 , subticks=0] (0,0)(-.5,-.5)(1.5,1.5){3.5cm}{3.5cm } 
\psBernstein(0,3) 
\end{psgraph*}
&  
\begin{psgraph*}[axesstyle=none,xticksize= -.5 1.5 ,yticksize=-.5 1.5,xticksize= -.5  1.5 , dx=.5,Dx=.5, dy=.5,Dy=.5 , subticks=0] (0,0)(-.5,-.5)(1.5,1.5){3.5cm}{3.5cm }     
\psBernstein(1,3)
\psBernstein[linestyle=dotted](2,3)
\end{psgraph*}
&  
\begin{psgraph*}[axesstyle=none,xticksize= -.5 1.5 ,yticksize=-.5 1.5,xticksize= -.5  1.5 , dx=.5,Dx=.5, dy=.5,Dy=.5 , subticks=0] (0,0)(-.5,-.5)(1.5,1.5){3.5cm}{3.5cm }   
\psBernstein(3,3)
\end{psgraph*} 
\\ \hline 
\BSS{psBernstein}(0,3) & \BSS{psBernstein}(1,3) & \BSS{psBernstein}(3,3)\\
& \BSS{psBernstein}(2,3) & 
\\ \hline
\end{tabular}

\bigskip 

\begin{tabular}{|c|c|c|} \hline  
\begin{psgraph*}[axesstyle=none,xticksize= 0 1 ,yticksize=0 1 , dx=.2,Dx=.2, dy=.2,Dy=.2 , subticks=0](0,0)(0,0)(1,1){12cm}{4cm }
\psBernstein[linestyle=dotted](5,5)
\psBernstein[linestyle=dotted](4,5)
\psBernstein[linestyle=dotted](3,5)
\psBernstein[linestyle=dotted](2,5)
\psBernstein[linestyle=dotted](1,5)  
\psBernstein[linestyle=dotted](0,5) 
\psBernstein[envelope](0.1,5)
\end{psgraph*}
\\ \hline 
\BS{psBernstein}[\RDD{envelope}](0,5) \RDI{envelope}{pst-func}  
\\ \hline
\end{tabular} 

\newpage

%\subseZeroction{Zéros d'une fonction ou  point d'intersection de deux
fonction}
 \SbSSCT{Zéros d'une fonction ou  point d'intersection de deux
 fonction}{Zeros or intersections}
 
\begin{tabular}{|c|} \hline  
\begin{psgraph*}[axesstyle=none,xticksize= -1 2 ,yticksize=0 10](0,0)(0,-1)(10,2){10cm}{3cm} 
 \psplot[plotpoints=500,algebraic,linewidth=0.8pt]{0.001}{9.75}{cos(x)+.5}
 \psZero[algebraic](0.5,5){cos(x)+.5}{A}
% \psComment{->}(3,1)(N1){noeud N1}
\rput[B](3,1.5){ \ovalnode{B}{n\oe ud A}}
\ncline{->}{B}{A}
\end{psgraph*}
\\ \hline  
 \BSS{psZero}[algebraic](0.5,5)\AC{cos(x)+.5}\AC{A}
\\ \hline 
\end{tabular} 

\bigskip

\begin{tabular}{|c|} \hline  
\begin{psgraph*}[axesstyle=none,xticksize= -1 2 ,yticksize=0 10](0,0)(0,-1)(10,2){10cm}{3cm} 
 \psplot[plotpoints=500,algebraic,linestyle=dotted]{0.001}{9.75}{sin(x)}
 \psplot[plotpoints=500,algebraic,linewidth=0.8pt]{0.001}{9.75}{cos(x)+.5} 
 \psZero[algebraic](3,5){cos(x)+.5}[sin(x)]{N1}
% \psComment{->}(3,1)(N1){noeud N1}
\rput[B](3,1.5){ \ovalnode{B}{n\oe ud N1}}
\ncline{->}{B}{N1}
\end{psgraph*}
\\ \hline  
 \BSS{psZero}[algebraic](0.5,5)\AC{cos(x)+.5}[sin(x)]\AC{N1}
\\ \hline 
\end{tabular}

\bigskip

\begin{tabular}{|c|c|} \hline  
\multicolumn{2}{|c|}{ \BSS{psZero}[algebraic,\RDD{markZeros}](0.5,5)\AC{cos(x)+.5[sin(x)]}\AC{A} } 
\\ \hline 
\begin{psgraph*}[axesstyle=none,xticksize= -1 2 ,yticksize=0 4](0,0)(0,-1)(4,2){4cm}{3cm}
 \psplot[plotpoints=500,algebraic,linewidth=.1pt]{0.001}{9.75}{sin(x)} 
 \psplot[plotpoints=500,algebraic,linewidth=0.8pt]{0.001}{9.75}{cos(x)+.5}
 \psZero[algebraic,markZeros](0.5,5){cos(x)+.5}[sin(x)]{A}
\end{psgraph*}
&
\begin{psgraph*}[axesstyle=none,xticksize= -1 2 ,yticksize=0 4](0,0)(0,-1)(4,2){4cm}{3cm} 
 \psplot[plotpoints=500,algebraic,linewidth=.1pt]{0.001}{9.75}{sin(x)}
 \psplot[plotpoints=500,algebraic,linewidth=0.8pt]{0.001}{9.75}{cos(x)+.5}
 \psZero[algebraic,markZeros,onlyNode](0.5,5){cos(x)+.5}[sin(x)]{A}
\end{psgraph*}
\\ \hline 
\RDD{ markZeros} \RDI{markZeros}{pst-func} & \RDD{onlyNode}  \RDI{onlyNode}{pst-func}
\\ \hline 
\end{tabular} 

\bigskip

\begin{tabular}{|c|c|} \hline  
%\multicolumn{2}{|c|}{ \BSS{psZero}[algebraic,\RDD{markZeros}](0.5,5)\AC{cos(x)+.5[sin(x)]}\AC{A} } 
%\\ \hline 
\begin{psgraph*}[axesstyle=none,xticksize= -1 2 ,yticksize=0 4](0,0)(0,-1)(4,2){4cm}{3cm}
 \psplot[plotpoints=500,algebraic,linewidth=.1pt]{0.001}{9.75}{sin(x)} 
 \psplot[plotpoints=500,algebraic,linewidth=0.8pt]{0.001}{9.75}{cos(x)+.5}
 \psZero[algebraic,PrintCoord](0.5,5){cos(x)+.5}[sin(x)]{A}
\end{psgraph*}
&
\begin{psgraph*}[axesstyle=none,xticksize= -1 2 ,yticksize=0 4](0,0)(0,-1)(4,2){4cm}{3cm} 
 \psplot[plotpoints=500,algebraic,linewidth=.1pt]{0.001}{9.75}{sin(x)}
 \psplot[plotpoints=500,algebraic,linewidth=0.8pt]{0.001}{9.75}{cos(x)+.5}
 \psZero[algebraic,markZeros,onlyYVal](0.5,5){cos(x)+.5}[sin(x)]{A}
\end{psgraph*}
\\ \hline 
\RDD{PrintCoord}  \RDI{PrintCoord}{pst-func} & \RDD{onlyYVal}  \RDI{onlyYVal}{pst-func}
\\ \hline 
\end{tabular} 

\bigskip

\begin{tabular}{|c|c|} \hline  
%\multicolumn{2}{|c|}{ \BSS{psZero}[algebraic,\RDD{markZeros}](0.5,5)\AC{cos(x)+.5[sin(x)]}\AC{A} } 
%\\ \hline 
\begin{psgraph*}[axesstyle=none,xticksize= -1 2 ,yticksize=0 4](0,0)(0,-1)(4,2){4cm}{3cm}
 \psplot[plotpoints=500,algebraic,linewidth=.1pt]{0.001}{9.75}{sin(x)} 
 \psplot[plotpoints=500,algebraic,linewidth=0.8pt]{0.001}{9.75}{cos(x)+.5}
 \psZero[algebraic,PointName=Point,PrintCoord](0.5,5){cos(x)+.5}[sin(x)]{A}
\end{psgraph*}
&
\begin{psgraph*}[axesstyle=none,xticksize= -1 2 ,yticksize=0 4](0,0)(0,-1)(4,2){4cm}{3cm} 
 \psplot[plotpoints=500,algebraic,linewidth=.1pt]{0.001}{9.75}{sin(x)}
 \psplot[plotpoints=500,algebraic,linewidth=0.8pt]{0.001}{9.75}{cos(x)+.5}
 \psZero[algebraic,markZeros,originV,PrintCoord](0.5,5){cos(x)+.5}[sin(x)]{A}
\end{psgraph*}
\\ \hline 
\RDD{PointName},PrintCoord  \RDI{PointName}{pst-func} & \RDD{originV},PrintCoord  \RDI{originV}{pst-func}
\\ \hline 
\dft: PointName= I &
\\ \hline 
\end{tabular} 

\bigskip

\begin{tabular}{|c|c|} \hline  
%\multicolumn{2}{|c|}{ \BSS{psZero}[algebraic,\RDD{markZeros}](0.5,5)\AC{cos(x)+.5[sin(x)]}\AC{A} } 
%\\ \hline 
\begin{psgraph*}[axesstyle=none,xticksize= -1 2 ,yticksize=0 4](0,0)(0,-1)(4,2){4cm}{3cm}
 \psplot[plotpoints=500,algebraic,linewidth=.1pt]{0.001}{9.75}{sin(x)} 
 \psplot[plotpoints=500,algebraic,linewidth=0.8pt]{0.001}{9.75}{cos(x)+.5}
 \psZero[algebraic,PrintCoord,decimals=3](0.5,5){cos(x)+.5}[sin(x)]{A}
\end{psgraph*}
&
\begin{psgraph*}[axesstyle=none,xticksize= -1 2 ,yticksize=0 4](0,0)(0,-1)(4,2){4cm}{3cm} 
 \psplot[plotpoints=500,algebraic,linewidth=.1pt]{0.001}{9.75}{sin(x)}
 \psplot[plotpoints=500,algebraic,linewidth=0.8pt]{0.001}{9.75}{cos(x)+.5}
 \psZero[algebraic,markZeros,PrintCoord,ydecimals=4](0.5,5){cos(x)+.5}[sin(x)]{A}
\end{psgraph*}
\\ \hline 
\RDD{decimals}=3,PrintCoord  \RDI{originV}{pst-func}& \RDD{ydecimals}=4,PrintCoord \RDI{ydecimals}{pst-func}
%\\ \hline 
%\dft: PointName= I &
\\ \hline 
\end{tabular}


\bigskip

\begin{tabular}{|c|c|} \hline  
%\multicolumn{2}{|c|}{ \BSS{psZero}[algebraic,\RDD{markZeros}](0.5,5)\AC{cos(x)+.5[sin(x)]}\AC{A} } 
%\\ \hline 
\begin{psgraph*}[axesstyle=none,xticksize= -1 2 ,yticksize=0 4](0,0)(0,-1)(4,2){4cm}{3cm}
 \psplot[plotpoints=500,algebraic,linewidth=.1pt]{0.001}{9.75}{sin(x)} 
 \psplot[plotpoints=500,algebraic,linewidth=0.8pt]{0.001}{9.75}{cos(x)+.5}
 \psZero[algebraic,PrintCoord,xShift=.5](0.5,5){cos(x)+.5}[sin(x)]{A}
\end{psgraph*}
&
\begin{psgraph*}[axesstyle=none,xticksize= -1 2 ,yticksize=0 4](0,0)(0,-1)(4,2){4cm}{3cm} 
 \psplot[plotpoints=500,algebraic,linewidth=.1pt]{0.001}{9.75}{sin(x)}
 \psplot[plotpoints=500,algebraic,linewidth=0.8pt]{0.001}{9.75}{cos(x)+.5}
 \psZero[algebraic,markZeros,PrintCoord,yShift=.5](0.5,5){cos(x)+.5}[sin(x)]{A}
\end{psgraph*}
\\ \hline 
\RDD{xShift}=.5,PrintCoord  \RDI{xShift}{pst-func} & \RDD{yShift}=.5,PrintCoord  \RDI{yShift}{pst-func}
%\\ \hline 
%\dft: PointName= I &
\\ \hline 
\end{tabular} 


\bigskip

\begin{tabular}{|c|c|} \hline  
%\multicolumn{2}{|c|}{ \BSS{psZero}[algebraic,\RDD{markZeros}](0.5,5)\AC{cos(x)+.5[sin(x)]}\AC{A} } 
%\\ \hline 
\begin{psgraph*}[axesstyle=none,xticksize= -1 2 ,yticksize=0 4](0,0)(0,-1)(4,2){4cm}{3cm}
 \psplot[plotpoints=500,algebraic,linewidth=.1pt]{0.001}{9.75}{sin(x)} 
 \psplot[plotpoints=500,algebraic,linewidth=0.8pt]{0.001}{9.75}{cos(x)+.5}
 \psZero[algebraic,PrintCoord,yShift=.5,,postString=123](0.5,5){cos(x)+.5}[sin(x)]{A}
\end{psgraph*}
&
%\begin{psgraph*}[axesstyle=none,xticksize= -1 2 ,yticksize=0 4](0,0)(0,-1)(4,2){4cm}{3cm} 
%% \psplot[plotpoints=500,algebraic,linewidth=.1pt]{0.001}{9.75}{sin(x)}
% \psplot[plotpoints=500,algebraic,linewidth=0.8pt]{0.001}{9.75}{cos(x)+.5}
% \psZero[xShift=-0.2,yShift=0.15,postString=1,Newton](0.5,5){ x cos }{A}
%\end{psgraph*}
\\ \hline 
\RDD{postString}=123,PrintCoord  \RDI{postString}{pst-func} &
% \RDD{yShift}=.5,PrintCoord  \RDI{yShift}{pst-func}
%\\ \hline 
%\dft: PointName= I &
\\ \hline 
\end{tabular} 

\newpage

%\subsection{Fonction de Fourier}
 \SbSSCT{Fonction de Fourier}{Fourrier}
 
\begin{tabular}{|c|} \hline  
\begin{psgraph*}[axesstyle=none,xticksize= -3 3 ,yticksize=-10 10 , subticks=0 ](0,0)(-10,-3)(10,3){10cm}{6cm } 
 \psFourier[cosCoeff=0 1 -1 ]{-10}{10}
\end{psgraph*}
\\ \hline  
\BSS{psFourier}[\RDD{cosCoeff}=0 1 -1 ]\AC{-10}\AC{10}
\BSI{psFourier}{pst-func} \RDI{cosCoeff}{pst-func}
\\ \hline 
\dft : cosCoeff =0
\\ \hline 

\end{tabular}   


\bigskip
\begin{tabular}{|c|} \hline  
\begin{psgraph*}[axesstyle=none,xticksize= -2 2 ,yticksize=-5 5 , subticks=0 ](0,0)(-5,-2)(5,2){10cm}{4cm } 
 \psFourier[sinCoeff=1 .5 .33 .25 .2 .165 .14  .125 ]{-5}{5}
\end{psgraph*}
\\ \hline  
\BSS{psFourier}[\RDD{sinCoeff}=1 .5 .33 .25 .2 .165 .14  .125 ]\AC{-5}\AC{5}  \RDI{sinCoeff}{pst-func}
\\ \hline 
\dft : sinCoeff =1
\\ \hline 
\end{tabular}

\newpage
%\subsection{Fonction de Bessel}
 \SbSSCT{Fonction de Bessel}{Bessel}
 
\begin{tabular}{|c|} \hline  
  $\displaystyle J_n(x)=\frac{1}{\pi} \int_0^\pi \cos  (x \sin t-nt)  dt$ 
\\ \hline  
\begin{psgraph*}[axesstyle=none,xticksize= -1.5 1.5 ,yticksize=-20 20 , subticks=0,Dx=5 ](0,0)(-20,-1.5)(20,1.5){10cm}{4cm } 
 \psBessel{0}{-20}{20}
\end{psgraph*}  
\\ \hline  
n= 0 \hspace{1cm} \BSS{psBessel}\AC{0}\AC{-20}\AC{20} 
\\ \hline 
\begin{psgraph*}[axesstyle=none,xticksize= -1.5 1.5 ,yticksize=-20 20 , subticks=0,Dx=5 ](0,0)(-20,-1.5)(20,1.5){10cm}{4cm } 
 \psBessel{2}{-20}{20}
\end{psgraph*}  
\\ \hline  
n= 2 \hspace{1cm} \BSS{psBessel}\AC{2}\AC{-20}\AC{20} 
\\ \hline 
\end{tabular} 

\bigskip

\begin{tabular}{|c|} \hline  
  $\displaystyle f(x)=2.5 J_0(x) +sin(t)$ 
\\ \hline  
\begin{psgraph*}[axesstyle=none,xticksize= -3 3 ,yticksize=-20 20 , subticks=0,Dx=5 ](0,0)(-20,-3)(20,3){10cm}{4cm } 
 \psBessel[constI=2.5,constII={ t k  sin }]{0}{-20}{20}%
\end{psgraph*} 
\\ \hline 
\BS{psBessel}[\RDD{constI}=2.5,\RDD{constII}=\AC{ t k  sin }]\AC{0}\AC{-20}\AC{20} \RDI{constI}{pst-func}  \RDI{constII}{pst-func}
\\ \hline 
\end{tabular} 


\newpage
%\subsection{Fonction de Bessel modifiée}
\SbSSCT{Fonction de Bessel modifiée}{modified Bessel}

\begin{tabular}{|c|c|c|} \hline 
\multicolumn{3}{|c|}{\BSS{psModBessel}[yMaxValue=5,\RDD{nue}=0]\AC{0}\AC{5} \BSI{psModBessel}{pst-func} \RDI{nue}{pst-func}} \\ \hline
\begin{psgraph*}[axesstyle=none,xticksize= 0 5 ,yticksize=0 5](0,0)(0,0)(4,5){3.5cm}{5cm} 
 \psModBessel[yMaxValue=5,nue=0]{0}{5}
\end{psgraph*}
&  
\begin{psgraph*}[axesstyle=none,xticksize= 0 5 ,yticksize=0 5](0,0)(0,0)(4,5){3.5cm}{5cm} 
 \psModBessel[yMaxValue=5,nue=1]{0}{5}
\end{psgraph*}
&  
\begin{psgraph*}[axesstyle=none,xticksize= 0 5 ,yticksize=0 5](0,0)(0,0)(4,5){3.5cm}{5cm} 
 \psModBessel[yMaxValue=5,nue=2]{0}{5}
\end{psgraph*}
\\ \hline 
\RDD{nue}=0 & \RDD{nue}=1 & \RDD{nue}= 2
\\ \hline 
\multicolumn{3}{|c|}{\dft : nue=0} \\ \hline
\end{tabular} 

\newpage


%\subsection{Sinus intégral}
\SbSSCT{Sinus intégral}{Integral sinus}

\begin{tabular}{|c|} \hline  
\begin{psgraph*}[axesstyle=none,xticksize= -2 2 ,yticksize=-14 14, dx=2,Dx=2](0,0)(-14,-2)(14,2){10cm}{3cm} 
 \psSi{-14.5}{14.5}
\end{psgraph*}
\\ \hline  
 \BSS{psSi}\AC{-14.5}\AC{14.5}  \BSI{psSi}{pst-func}
\\ \hline 
\end{tabular} 

\bigskip

\begin{tabular}{|c|} \hline  
\begin{psgraph*}[axesstyle=none,xticksize= -4 1 ,yticksize=-14 14, dx=2,Dx=2](0,0)(-14,-4)(14,1){10cm}{3cm} 
 \pssi{-14.5}{14.5}
\end{psgraph*}
\\ \hline  
 \BSS{pssi}\AC{-14.5}\AC{14.5}  \BSI{pssi}{pst-func}
\\ \hline 
\end{tabular} 

%\subsection{Cosinus intégral}
\SbSSCT{Cosinus intégral}{Integral cosinus}

\begin{tabular}{|c|} \hline  
\begin{psgraph*}[axesstyle=none,xticksize= -4 1 ,yticksize=-12 12 , dx=2,Dx=2](0,0)(-12,-4)(12,1){10cm}{3cm} 
 \psCi[plotpoints=500]{-11.5}{11.5}
\end{psgraph*}
\\ \hline  
 \BSS{psCi}\AC{-11.5}\AC{11.5}  \BSI{psCi}{pst-func}
\\ \hline 
\end{tabular} 

\bigskip

\begin{tabular}{|c|} \hline  
\begin{psgraph*}[axesstyle=none,xticksize= 0 4 ,yticksize=-12 12 , dx=2,Dx=2](0,0)(-12,0)(12,4){10cm}{3cm} 
 \psci[plotpoints=500]{-11.5}{11.5}
\end{psgraph*}
\\ \hline  
 \BSS{psci}\AC{-11.5}\AC{11.5} \BSI{psci}{pst-func}
\\ \hline 
\end{tabular}

\newpage

%\subsection{Intégration et Convolution}
\SbSSCT{Intégration et Convolution}{Integration and convolution }

\begin{tabular}{|c|} \hline  
\begin{psgraph*}[axesstyle=none,xticksize= 0 1 ,yticksize=-6 6 , dx=2,Dx=2, dy=.5,Dy=.5](0,0)(-6,0)(6,1){10cm}{3cm} 
 \psplot[linestyle=dotted]{-6}{6}{x 0 2 GAUSS}
 \psCumIntegral{-6}{6}{0 2 GAUSS}
\end{psgraph*}
\\ \hline 
 \BS{psplot}[linestyle=dotted]\AC{-6}\AC{6}\AC{x 0 2 GAUSS} \\
 \BSS{psCumIntegral}\AC{-10}\AC{10}\AC{0 2 GAUSS}
 \BSI{psCumIntegral}{pst-func} 
\\ \hline 
\end{tabular}

\bigskip


\begin{tabular}{|c|} \hline  
\begin{psgraph*}[axesstyle=none,xticksize= 0 1 ,yticksize=-6 6 , dx=2,Dx=2, dy=.5,Dy=.5](0,0)(-6,0)(6,1){10cm}{3cm} 
 \psplot[linestyle=dotted]{-6}{6}{x 0 2 GAUSS}
 \psCumIntegral{0}{6}{0 2 GAUSS}
\end{psgraph*}
\\ \hline  
 \BSS{psCumIntegral}\AC{0}\AC{6}\AC{0 2 GAUSS}
\\ \hline 
\end{tabular}

\bigskip


\begin{tabular}{|c|} \hline  
\begin{psgraph*}[axesstyle=none,xticksize= 0 1 ,yticksize=-6 6 , dx=2,Dx=2, dy=.5,Dy=.5](0,0)(-6,0)(6,1){10cm}{3cm}  
 \psplot[linestyle=dotted]{-6}{6}{x 0 .5 GAUSS}
 \psIntegral{-6}{6}(-2,4){.5 GAUSS}
\end{psgraph*}
\\ \hline  
 \BSS{psIntegral}\AC{-2}\AC{4}\AC{.5 GAUSS} 
  \BSI{psIntegral}{pst-func} 
\\ \hline 
\end{tabular}

\bigskip

\begin{tabular}{|c|} \hline  
\begin{psgraph*}[axesstyle=none,xticksize= 0 1 ,yticksize=-6 6 , dx=2,Dx=2, dy=.5,Dy=.5](0,0)(-6,0)(6,1){10cm}{3cm}  
 \psplot[linestyle=dotted]{-6}{6}{x 0 .5 GAUSS}
 \psIntegral[Simpson=10]{-6}{6}(-2,4){.5 GAUSS}
\end{psgraph*}
\\ \hline  
 \BS{psIntegral}[\RDD{Simpson}=10]\AC{-2}\AC{4}\AC{.5 GAUSS}
\\ \hline 
\end{tabular}

\bigskip


\begin{tabular}{|c|} \hline  
\begin{psgraph*}[axesstyle=none,xticksize= 0 1.5 ,yticksize=-6 6 , dx=2,Dx=2, dy=.5,Dy=.5](0,0)(-6,0)(6,1.5){10cm}{3cm}
 \psplot[linestyle=dashed]{-5}{5}{x abs 2 le {0.5}{0} ifelse}
  \psplot[linestyle=dotted]{-5}{5}{x abs 1 le {0.75}{0} ifelse}
 \psConv{-5}{5}(-6,6) {abs 2 le {0.5}{0} ifelse}{abs 2 le {0.75}{0} ifelse}
\end{psgraph*}
\\ \hline  
 \BS{psplot}[linestyle=dashed]\AC{-5}\AC{5}\AC{x abs 2 le {0.5}{0} ifelse} \\ 
 \BS{psplot}[linestyle=dotted]\AC{-5}\AC{5}\AC{x abs 1 le {0.75}{0} ifelse} \\
 \BSS{psConv}\AC{-5}\AC{5}\AC(-6,6) \AC{abs 2 le {0.5}{0} ifelse}\AC{abs 2 le {0.75}{0} ifelse}
   \BSI{psConv}{pst-func} 
\\ \hline 
\end{tabular}

%====================


\newpage

%\subsection{Loi de Gauss}
\SbSSCT{Loi de Gauss}{Gauss Distribution}

\begin{tabular}{|c|c|} \hline  
\begin{psgraph*}[axesstyle=none,xticksize= -5 1 ,yticksize=-12 12 , subticks=0 ](0,0)(-2,0)(2,1){6cm}{2cm } 
 \psGauss{-2}{2}%
\end{psgraph*}
&  
\begin{psgraph*}[axesstyle=none,xticksize= -5 1 ,yticksize=-12 12 , subticks=0 ](0,0)(-2,0)(2,1){6cm}{2cm } 
 \psGaussI{-2}{2}%
\end{psgraph*}
\\ \hline  
 \BSS{psGauss}\AC{-2}\AC{2} \BSI{psGauss}{pst-func}
&  
 \BSS{psGaussI}\AC{-2}\AC{2} \BSI{psGaussI}{pst-func}
\\ 
\hline 
\end{tabular} 
\bigskip

\begin{tabular}{|c|c|} \hline  
\begin{psgraph*}[axesstyle=none,xticksize= -5 1 ,yticksize=-12 12 , subticks=0 ](0,0)(-2,0)(2,1){6cm}{2cm } 
 \psGauss[linestyle=dotted]{-2}{2}%
 \psGauss[mue=0.5]{-2}{2}%
\end{psgraph*}
&  
\begin{psgraph*}[axesstyle=none,xticksize= -5 1 ,yticksize=-12 12 , subticks=0 ](0,0)(-2,0)(2,1){6cm}{2cm }
 \psGauss[linestyle=dotted]{-2}{2}% 
 \psGauss[mue=-.5]{-2}{2}%
\end{psgraph*}
\\ \hline  
 \BSS{psGauss}[\RDD{mue}=0.5]\AC{-2}\AC{2}  \RDI{mue}{pst-func}
&  
 \BSS{psGauss}[\RDD{mue}=0.5]\AC{-2}\AC{2}
\\ 
\hline 
\end{tabular}

\bigskip

\begin{tabular}{|c|c|} \hline  
\begin{psgraph*}[axesstyle=none,xticksize= 0 2 ,yticksize=-2 2 , subticks=0 ](0,0)(-2,0)(2,2){6cm}{2cm } 
 \psGauss[linestyle=dotted]{-2}{2}%
 \psGauss[sigma=.25]{-2}{2}%
\end{psgraph*}
&  
\begin{psgraph*}[axesstyle=none,xticksize= 0 2 ,yticksize=-2 2 , subticks=0 ](0,0)(-2,0)(2,2){6cm}{2cm }
 \psGauss[linestyle=dotted]{-2}{2}% 
 \psGauss[sigma=1]{-2}{2}%
\end{psgraph*}
\\ \hline 
 
 \BSS{psGauss}[\RDD{sigma}=0.25]\AC{-2}\AC{2} \RDI{sigma}{pst-func}
&  
 \BSS{psGauss}[\RDD{sigma}=1]\AC{-2}\AC{2}
\\ \hline 
\end{tabular}

\newpage 

%\subsection{Loi binomiale}
\SbSSCT{Loi binomiale}{Binomial Distribution}

\begin{tabular}{|c|c|c|} \hline  
\begin{psgraph*}[axesstyle=none,xticksize= 0 1 ,yticksize=-1 3 , subticks=0, dy=.2,Dy=.2](0,0)(-1,0)(3,1){3cm}{2cm }
\psBinomial{2}{0.5}
\end{psgraph*}
&  
\begin{psgraph*}[axesstyle=none,xticksize= 0 1 ,yticksize=-1 3 , subticks=0, dy=.2,Dy=.2 ](0,0)(-1,0)(3,1){3cm}{2cm }
\psBinomial{2}{0.25}
\end{psgraph*}
&  
\begin{psgraph*}[axesstyle=none,xticksize= 0 1 ,yticksize=-1 3 , subticks=0, dy=.2,Dy=.2 ](0,0)(-1,0)(3,1){3cm}{2cm }
\psBinomial{2}{0.75}
\end{psgraph*}
\\ \hline  
\BSS{psBinomial}\AC{2}\AC{0.5}  \BSI{psBinomial}{pst-func}
&  
\BSS{psBinomial}\AC{2}\AC{0.25}
&  
\BSS{psBinomial}\AC{2}\AC{0.75}
\\ \hline 
\end{tabular} 

\bigskip

\begin{tabular}{|c|c|} \hline  
  
\begin{psgraph*}[axesstyle=none,xticksize= 0 1 ,yticksize=-1 5 , subticks=0, dy=.2,Dy=.2 ](0,0)(-1,0)(5,1){5cm}{2cm }
\psBinomial{3}{0.5}
\end{psgraph*}
&  
\begin{psgraph*}[axesstyle=none,xticksize= 0 1 ,yticksize=-1 5 , subticks=0, dy=.2,Dy=.2 ](0,0)(-1,0)(5,1){5cm}{2cm }
\psBinomial{4}{0.5}
\end{psgraph*}
\\ \hline  
 
\BSS{psBinomial}\AC{3}\AC{0.5}
&  
\BSS{psBinomial}\AC{4}\AC{0.5}
\\ \hline 
\end{tabular}

\bigskip

\begin{tabular}{|c|c|} \hline  
  
\begin{psgraph*}[axesstyle=none,xticksize= 0 1 ,yticksize=-1 5 , subticks=0, dy=.2,Dy=.2 ](0,0)(-1,0)(5,1){5cm}{2cm }
\psBinomial[linestyle=dotted]{4}{0.5}
\psBinomial{2,4}{0.5}
\end{psgraph*}
&  
\begin{psgraph*}[axesstyle=none,xticksize= 0 1 ,yticksize=-1 5 , subticks=0, dy=.2,Dy=.2 ](0,0)(-1,0)(5,1){5cm}{2cm }
\psBinomial[linestyle=dotted]{4}{0.5}
\psBinomial{1,2,4}{0.5}
\end{psgraph*}
\\ \hline  
 
\BSS{psBinomial}\AC{2,4}\AC{0.5}
&  
\BSS{psBinomial}\AC{1,2,4}\AC{0.5}
\\ \hline 
\end{tabular}

\bigskip

\begin{tabular}{|c|c|} \hline  
  
\begin{psgraph*}[axesstyle=none,xticksize= 0 1 ,yticksize=-3 3 , subticks=0, dy=.2,Dy=.2 ](0,0)(-3,0)(3,1){5cm}{2cm }
\psBinomialN{3}{0.5}
\end{psgraph*}
&  
\begin{psgraph*}[axesstyle=none,xticksize= 0 1 ,yticksize=-3 3 , subticks=0, dy=.2,Dy=.2 ](0,0)(-3,0)(3,1){5cm}{2cm }
\psBinomialN{4}{0.5}
\end{psgraph*}
\\ \hline  
 
\BSS{psBinomialN}\AC{3}\AC{0.5} \BSI{psBinomialN}{pst-func}
&  
\BSS{psBinomialN}\AC{4}\AC{0.5}
\\ \hline 
\end{tabular}

\newpage

\subsubsection{paramètres}

\begin{tabular}{|c|c|} \hline  
  
\begin{psgraph*}[axesstyle=none,xticksize= 0 1 ,yticksize=-1 5 , subticks=0, dy=.2,Dy=.2 ](0,0)(-1,0)(5,1){5cm}{2cm }
\psBinomial[printValue]{3}{0.5}
\end{psgraph*}
&  
\begin{psgraph*}[axesstyle=none,xticksize= 0 1 ,yticksize=-1 5 , subticks=0, dy=.2,Dy=.2 ](0,0)(-1,0)(5,1){5cm}{2cm }
\psBinomial[markZeros]{4}{0.5}
\end{psgraph*}
\\ \hline  
 
\BSS{psBinomial}[\RDD{printValue}]\AC{3}\AC{0.5} \RDI{printValue}{pst-func}
&  
\BS{psBinomial}[\RDD{markZeros}]\AC{4}\AC{0.5}
\RDI{markZeros}{pst-func}
\\ \hline 
\end{tabular}


\bigskip

\begin{tabular}{|c|c|} \hline  
  
\begin{psgraph*}[axesstyle=none,xticksize= 0 1 ,yticksize=-1 5 , subticks=0, dy=.2,Dy=.2 ](0,0)(-1,0)(5,1){5cm}{2cm }
\psBinomial[fillcolor=yellow,fillstyle=solid,]{3}{0.5}
\end{psgraph*}
&  
\begin{psgraph*}[axesstyle=none,xticksize= 0 1 ,yticksize=-1 5 , subticks=0, dy=.2,Dy=.2 ](0,0)(-1,0)(5,1){5cm}{2cm }
\psBinomial[barwidth=0.5]{4}{0.5} 
\end{psgraph*}
\\ \hline  
 
\BSS{psBinomial}[fillcolor=yellow]\AC{3}\AC{0.5}
&  
\BS{psBinomial}[\RDD{barwidth}=0.5]\AC{4}\AC{0.5} \RDI{barwidth}{pst-func}
\\ \hline 
\end{tabular}

\bigskip

\begin{tabular}{|c|c|} \hline  
  
\begin{psgraph*}[axesstyle=none,xticksize= 0 1 ,yticksize=-1 5 , subticks=0, dy=.2,Dy=.2 ](0,0)(-1,0)(5,1){5cm}{2cm }
\psBinomial[fillstyle=vlines,markZeros]{3}{0.5}
\end{psgraph*}
&  
\begin{psgraph*}[axesstyle=none,xticksize= 0 1 ,yticksize=-1 5 , subticks=0, dy=.2,Dy=.2 ](0,0)(-1,0)(5,1){5cm}{2cm }
\psBinomial[barwidth=0.5,markZeros]{4}{0.5}
\end{psgraph*}
\\ \hline  
 
[fillstyle=vlines,,markZeros]
&  
[barwidth=0.5,markZeros]
\\ \hline 
\end{tabular}

\newpage
%\subsection{Loi de Poisson}
\SbSSCT{Loi de Poisson}{Poisson Distribution}

\begin{tabular}{|c|c|c|} \hline  
\begin{psgraph*}[axesstyle=none,xticksize= 0 1 ,yticksize=-1 5 , subticks=0, dy=.2,Dy=.2 ](0,0)(-1,0)(3,1){3cm}{2cm }
 \psPoisson{2}{1}
\end{psgraph*}
&  
\begin{psgraph*}[axesstyle=none,xticksize= 0 1 ,yticksize=-1 5 , subticks=0, dy=.2,Dy=.2 ](0,0)(-1,0)(5,1){3cm}{2cm }
 \psPoisson{3}{1}
\end{psgraph*}
&  
\begin{psgraph*}[axesstyle=none,xticksize= 0 1 ,yticksize=-1 5 , subticks=0, dy=.2,Dy=.2 ](0,0)(-1,0)(5,1){3cm}{2cm }
 \psPoisson{4}{1}
\end{psgraph*}
\\ \hline  
\BSS{psPoisson}\AC{2}\AC{1} \BSI{psPoisson}{pst-func}  & \BSS{psPoisson}\AC{3}\AC{1}   &  \BSS{psPoisson}\AC{4}\AC{1} \\ 
\hline 
\end{tabular} 

\bigskip

\begin{tabular}{|c|c|c|} \hline  
\begin{psgraph*}[axesstyle=none,xticksize= 0 1 ,yticksize=-1 5 , subticks=0, dy=.2,Dy=.2 ](0,0)(-1,0)(5,1){3cm}{2cm }
 \psPoisson{4}{2}
\end{psgraph*}
&  
\begin{psgraph*}[axesstyle=none,xticksize= 0 1 ,yticksize=-1 5 , subticks=0, dy=.2,Dy=.2 ](0,0)(-1,0)(5,1){3cm}{2cm }
 \psPoisson{4}{3}
\end{psgraph*}
&  
\begin{psgraph*}[axesstyle=none,xticksize= 0 1 ,yticksize=-1 5 , subticks=0, dy=.2,Dy=.2 ](0,0)(-1,0)(5,1){3cm}{2cm }
 \psPoisson{4}{4}
\end{psgraph*}
\\ \hline  
\BSS{psPoisson}\AC{4}\AC{2}   & \BSS{psPoisson}\AC{4}\AC{3}   &  \BSS{psPoisson}\AC{4}\AC{4} \\ 
\hline 
\end{tabular} 


\bigskip

\begin{tabular}{|c|c|c|} \hline  
\begin{psgraph*}[axesstyle=none,xticksize= 0 1 ,yticksize=-1 5 , subticks=0, dy=.2,Dy=.2 ](0,0)(-1,0)(5,1){3cm}{2cm }
 \psPoisson[linestyle=dotted]{4}{2}
 \psPoisson{1,4}{2}
\end{psgraph*}
&  
\begin{psgraph*}[axesstyle=none,xticksize= 0 1 ,yticksize=-1 5 , subticks=0, dy=.2,Dy=.2 ](0,0)(-1,0)(5,1){3cm}{2cm }
 \psPoisson[linestyle=dotted]{4}{2}
 \psPoisson{2,4}{2}
\end{psgraph*}
&  
\begin{psgraph*}[axesstyle=none,xticksize= 0 1 ,yticksize=-1 5 , subticks=0, dy=.2,Dy=.2 ](0,0)(-1,0)(5,1){3cm}{2cm }
 \psPoisson[linestyle=dotted]{4}{2}
 \psPoisson{3,4}{2}
\end{psgraph*}
\\ \hline  
\BSS{psPoisson}\AC{1,4}\AC{2}   & \BSS{psPoisson}\AC{2,4}\AC{2}   &  \BSS{psPoisson}\AC{3,4}\AC{2} \\ 
\hline 
\end{tabular}

\bigskip

\begin{tabular}{|c|c|c|} \hline  
\begin{psgraph*}[axesstyle=none,xticksize= 0 1 ,yticksize=-1 5 , subticks=0, dy=.2,Dy=.2 ](0,0)(-1,0)(5,1){3cm}{2cm }
 \psPoisson[markZeros]{4}{2}
\end{psgraph*}
&  
\begin{psgraph*}[axesstyle=none,xticksize= 0 1 ,yticksize=-1 5 , subticks=0, dy=.2,Dy=.2 ](0,0)(-1,0)(5,1){3cm}{2cm }
 \psPoisson[printValue]{4}{2}
\end{psgraph*}
&  
\begin{psgraph*}[axesstyle=none,xticksize= 0 1 ,yticksize=-1 5 , subticks=0, dy=.2,Dy=.2 ](0,0)(-1,0)(5,1){3cm}{2cm }
 \psPoisson[barwidth=0.5]{4}{2}
\end{psgraph*}
\\ \hline  
\BS{psPoisson}[\RDD{markZeros}]\AC{4}\AC{2}   & \BS{psPoisson}[\RDD{printValue}]\AC{4}\AC{2}   &  \BS{psPoisson}[\RDD{barwidth}=0.5]\AC{4}\AC{2} \\ 
\hline 
\end{tabular}

\newpage
%\subsection{Loi Gamma }
\SbSSCT{Loi Gamma}{Gamma Distribution}

\begin{center}
\begin{tabular}{|c|} \hline  
\begin{psgraph*}[axesstyle=none,xticksize= 0 1 ,yticksize=0 3.5 , subticks=0, dy=.2,Dy=.2 ](0,0)(0,0)(3.5,1){5cm}{3cm }
 \psGammaDist{0.1}{3}
\end{psgraph*}
\\ \hline  
\BSS{psGammaDist}\AC{0.1}\AC{3}  \BSI{psGammaDist}{pst-func}
\\ \hline 
\end{tabular} 
\end{center}



 
\bigskip
\begin{tabular}{|c|c|} \hline  
\begin{psgraph*}[axesstyle=none,xticksize= 0 1 ,yticksize=0 3.5 , subticks=0, dy=.2,Dy=.2 ](0,0)(0,0)(3.5,1){5cm}{3cm }
 \psGammaDist[linestyle=dotted]{0.1}{3}
 \psGammaDist[alpha=0.25]{0.1}{3}
\end{psgraph*}

&  
\begin{psgraph*}[axesstyle=none,xticksize= 0 1 ,yticksize=0 3.5 , subticks=0, dy=.2,Dy=.2 ](0,0)(0,0)(3.5,1){5cm}{3cm }
 \psGammaDist[linestyle=dotted]{0.1}{3}
 \psGammaDist[alpha=0.75]{0.1}{3}
\end{psgraph*}
\\ \hline 
 \BS{psGammaDist}[\RDD{alpha}=0.25]\AC{0.1}\AC{3} \RDI{alpha}{pst-func} &
  \BS{psGammaDist}[\RDD{alpha}=0.75]\AC{0.1}\AC{3}  \\ 
\hline 
\end{tabular} 

\bigskip
\begin{tabular}{|c|c|} \hline  
\begin{psgraph*}[axesstyle=none,xticksize= 0 1 ,yticksize=0 3.5 , subticks=0, dy=.2,Dy=.2 ](0,0)(0,0)(3.5,1){5cm}{3cm }
 \psGammaDist[linestyle=dotted]{0.1}{3}
 \psGammaDist[beta=0.25]{0.1}{3}
\end{psgraph*}

&  
\begin{psgraph*}[axesstyle=none,xticksize= 0 1 ,yticksize=0 3.5 , subticks=0, dy=.2,Dy=.2 ](0,0)(0,0)(3.5,1){5cm}{3cm }
 \psGammaDist[linestyle=dotted]{0.1}{3}
 \psGammaDist[beta=0.75]{0.1}{3}
\end{psgraph*}
\\ \hline 
 \BS{psGammaDist}[\RDD{beta}=0.25]\AC{0.1}\AC{3} \RDI{beta}{pst-func} &
  \BS{psGammaDist}[\RDD{beta}=0.75]\AC{0.1}\AC{3}  \\ 
\hline 
\end{tabular} 


\bigskip
\begin{tabular}{|c|c|} \hline  
\begin{psgraph*}[axesstyle=none,xticksize= 0 1 ,yticksize=0 3.5 , subticks=0, dy=.2,Dy=.2 ](0,0)(0,0)(3.5,1){5cm}{3cm }
 \psGammaDist[linestyle=dotted]{0.1}{3}
 \psGammaDist[alpha=0.25,beta=0.75]{0.1}{3}
\end{psgraph*}

&  
\begin{psgraph*}[axesstyle=none,xticksize= 0 1 ,yticksize=0 3.5 , subticks=0, dy=.2,Dy=.2 ](0,0)(0,0)(3.5,1){5cm}{3cm }
 \psGammaDist[linestyle=dotted]{0.1}{3}
 \psGammaDist[alpha=0.75,beta=0.25]{0.1}{3}
\end{psgraph*}
\\ \hline 
[alpha=0.25,beta=0.75] &
[alpha=0.75,beta=0.25]  \\ 
\hline 
\end{tabular} 

\newpage
%\subsection{Loi du $\chi^2$}
\SbSSCT{Loi du $\chi^2$}{$\chi^2$ Distribution}

\begin{center}
\begin{tabular}{|c|} \hline  
\begin{psgraph*}[axesstyle=none,xticksize= 0 1 ,yticksize=0 5 , subticks=0, dy=.2,Dy=.2 ](0,0)(0,0)(5,1){10cm}{3cm }
\psChiIIDist{0.01}{5}
\end{psgraph*}
\\ \hline  
\BSS{psChiIIDist}\AC{0.01}\AC{5} \BSI{psChiIIDist}{pst-func}
\\ \hline 
\end{tabular} 
\end{center}

\begin{tabular}{|c|} \hline  
\begin{psgraph*}[axesstyle=none,xticksize= 0 1 ,yticksize=0 5 , subticks=0, dy=.2,Dy=.2 ](0,0)(0,0)(5,1){10cm}{3cm }
\psChiIIDist[linestyle=dotted]{0.01}{5}
\psChiIIDist[nue=.5]{0.01}{5}
\end{psgraph*}
\\ \hline  
\BS{psChiIIDist}[\RDD{nue}=.5]\AC{0.01}\AC{5} \RDI{nue}{pst-func}
\\ \hline 
\end{tabular}

\bigskip
\begin{tabular}{|c|} \hline  
\begin{psgraph*}[axesstyle=none,xticksize= 0 1 ,yticksize=0 5 , subticks=0, dy=.2,Dy=.2 ](0,0)(0,0)(5,1){10cm}{3cm }
\psChiIIDist[linestyle=dotted]{0.01}{5}
\psChiIIDist[nue=2]{0.01}{5}
\end{psgraph*}
\\ \hline  
\BS{psChiIIDist}[\RDD{nue}=2]\AC{0.01}\AC{5}
\\ \hline 
\end{tabular}

\bigskip
\begin{tabular}{|c|} \hline  
\begin{psgraph*}[axesstyle=none,xticksize= 0 1 ,yticksize=0 5 , subticks=0, dy=.2,Dy=.2 ](0,0)(0,0)(5,1){10cm}{3cm }
\psChiIIDist[linestyle=dotted]{0.01}{5}
\psChiIIDist[nue=3]{0.01}{5}
\end{psgraph*}
\\ \hline  
\BS{psChiIIDist}[\RDD{nue}=3]\AC{0.01}\AC{5}
\\ \hline 
\end{tabular}

%\subsection{Loi de Student}
\SbSSCT{Loi de Student}{Student Distribution}

\begin{tabular}{|c|} \hline  
\begin{psgraph*}[axesstyle=none,xticksize= 0 1 ,yticksize=-4 4 , subticks=0, dy=.2,Dy=.2 ](0,0)(-4,0)(4,1){10cm}{3cm }
 \psTDist{-4}{4}
\end{psgraph*}
\\ \hline  
\BSS{psTDist}\AC{4}\AC{4} \BSI{psTDist}{pst-func}
\\ \hline 
\end{tabular} 

\bigskip
\begin{tabular}{|c|} \hline  
\begin{psgraph*}[axesstyle=none,xticksize= 0 1 ,yticksize=-4 4 , subticks=0, dy=.2,Dy=.2 ](0,0)(-4,0)(4,1){10cm}{3cm }
 \psTDist[linestyle=dotted]{-4}{4}
 \psTDist[nue=.5]{-4}{4}
\end{psgraph*}
\\ \hline  
\BSS{psTDist}[\RDD{nue}=.5]\AC{4}\AC{4} \RDI{nue}{pst-func}
\\ \hline 
\end{tabular}

\bigskip
\begin{tabular}{|c|} \hline  
\begin{psgraph*}[axesstyle=none,xticksize= 0 1 ,yticksize=-4 4 , subticks=0, dy=.2,Dy=.2 ](0,0)(-4,0)(4,1){10cm}{3cm }
 \psTDist[linestyle=dotted]{-4}{4}
 \psTDist[nue=10]{-4}{4}
\end{psgraph*}
\\ \hline  
\BSS{psTDist}[\RDD{nue}=10]\AC{4}\AC{4} \\ \hline 
\end{tabular}



\newpage
%\subsection{Loi de F}
\SbSSCT{Loi de F}{F Distribution}

\begin{tabular}{|c|} \hline  
\begin{psgraph*}[axesstyle=none,xticksize= 0 1 ,yticksize=0 5 , subticks=0, dy=.2,Dy=.2 ](0,0)(0,0)(5,1){10cm}{3cm }
\psFDist{0.1}{5}
\end{psgraph*}
\\ \hline  
\BSS{psFDist}\AC{0.1}\AC{5} \BSI{psFDist}{pst-func}
\\ \hline 
\end{tabular} 


\bigskip

\begin{tabular}{|c|} \hline  
\begin{psgraph*}[axesstyle=none,xticksize= 0 1 ,yticksize=0 5 , subticks=0, dy=.2,Dy=.2 ](0,0)(0,0)(5,1){10cm}{3cm }
\psFDist[linestyle=dotted]{0.1}{5}
\psFDist[nue=3]{0.01}{5}
\end{psgraph*}

\\ \hline  
\BSS{psFDist}[\RDD{nue}=3]\AC{0.1}\AC{5} \RDI{nue}{pst-func}
\\ \hline 
\end{tabular}



\bigskip

\begin{tabular}{|c|} \hline  
\begin{psgraph*}[axesstyle=none,xticksize= 0 1 ,yticksize=0 5 , subticks=0, dy=.2,Dy=.2 ](0,0)(0,0)(5,1){10cm}{3cm }
\psFDist[linestyle=dotted]{0.1}{5}
\psFDist[mue=12]{0.01}{5}
\end{psgraph*}
\\ \hline  
\BSS{psFDist}[\RDD{mue}=12]\AC{0.1}\AC{5} \RDI{mue}{pst-func}
\\ \hline 
\end{tabular}



\bigskip

\begin{tabular}{|c|} \hline  
\begin{psgraph*}[axesstyle=none,xticksize= 0 1 ,yticksize=0 5 , subticks=0, dy=.2,Dy=.2 ](0,0)(0,0)(5,1){10cm}{3cm }
\psFDist[linestyle=dotted]{0.1}{5}
\psFDist[nue=3,mue=12]{0.01}{5}
\end{psgraph*}
\\ \hline  
\BSS{psFDist}[nue=3,mue=12]\AC{0.1}\AC{5} \RDI{mue}{pst-func}
\\ \hline 
\end{tabular}

\newpage

%\subsection{Loi de Beta}
\SbSSCT{Loi de Beta}{Beta Distribution}

\begin{center}
\begin{tabular}{|c|} \hline  
\begin{psgraph*}[axesstyle=none,xticksize= 0 2 ,yticksize=0 1 , subticks=0, dy=.2,Dy=.2 ](0,0)(0,0)(1,2){5cm}{3cm }
 \psBetaDist{0.01}{0.99}
\end{psgraph*}
\\ \hline  
\BSS{psBetaDist}\AC{0.01}\AC{0.99} \BSI{psBetaDist}{pst-func}
\\ \hline 
\end{tabular}
\end{center}



\bigskip

\begin{tabular}{|c|c|c|} \hline 
\multicolumn{3}{|c|}{ \BSS{psBetaDist}[alpha=0.1]\AC{0.01}\AC{0.99} }
\\ \hline 
\begin{psgraph*}[axesstyle=none,xticksize= 0 2 ,yticksize=0 1 , subticks=0, dy=.2,Dy=.2 ](0,0)(0,0)(1,2){3cm}{3cm }
 \psBetaDist[alpha=0.1]{0.01}{0.99}
\end{psgraph*}
&  
\begin{psgraph*}[axesstyle=none,xticksize= 0 2 ,yticksize=0 1 , subticks=0, dy=.2,Dy=.2 ](0,0)(0,0)(1,2){3cm}{3cm }
 \psBetaDist[alpha=.5]{0.01}{0.99}
\end{psgraph*}
&  
\begin{psgraph*}[axesstyle=none,xticksize= 0 2 ,yticksize=0 1 , subticks=0, dy=.2,Dy=.2 ](0,0)(0,0)(1,2){3cm}{3cm }
 \psBetaDist[alpha=.9]{0.01}{0.99}
\end{psgraph*}
\\ \hline 
[alpha=0.1] & [alpha=0.5] & [alpha=0.9] \\ 
\hline 
\multicolumn{3}{|c|}{ \dft : alpha= 1 }
\\ \hline
\end{tabular} 

\bigskip

\begin{tabular}{|c|c|c|} \hline 
\multicolumn{3}{|c|}{ \BSS{psBetaDist}[beta=0.1]\AC{0.01}\AC{0.99} }
\\ \hline 
\begin{psgraph*}[axesstyle=none,xticksize= 0 2 ,yticksize=0 1 , subticks=0, dy=.2,Dy=.2 ](0,0)(0,0)(1,2){3cm}{3cm }
 \psBetaDist[beta=0.1]{0.01}{0.99}
\end{psgraph*}
&  
\begin{psgraph*}[axesstyle=none,xticksize= 0 2 ,yticksize=0 1 , subticks=0, dy=.2,Dy=.2 ](0,0)(0,0)(1,2){3cm}{3cm }
 \psBetaDist[beta=.5]{0.01}{0.99}
\end{psgraph*}
&  
\begin{psgraph*}[axesstyle=none,xticksize= 0 2 ,yticksize=0 1 , subticks=0, dy=.2,Dy=.2 ](0,0)(0,0)(1,2){3cm}{3cm }
 \psBetaDist[beta=.9]{0.01}{0.99}
\end{psgraph*}
\\ \hline 
[beta=0.1] & [beta=0.5] & [beta=0.9] \\ 
\hline 
\multicolumn{3}{|c|}{ \dft : beta= 1 }
\\ \hline
\end{tabular} 

\bigskip

\begin{tabular}{|c|c|c|} \hline 
\multicolumn{3}{|c|}{ \BSS{psBetaDist}[beta=0.1]\AC{0.01}\AC{0.99} }
\\ \hline 
\begin{psgraph*}[axesstyle=none,xticksize= 0 2 ,yticksize=0 1 , subticks=0, dy=.2,Dy=.2 ](0,0)(0,0)(1,2){3cm}{3cm }
 \psBetaDist[alpha=.1,beta=0.1]{0.01}{0.99}
\end{psgraph*}
&  
\begin{psgraph*}[axesstyle=none,xticksize= 0 2 ,yticksize=0 1 , subticks=0, dy=.2,Dy=.2 ](0,0)(0,0)(1,2){3cm}{3cm }
 \psBetaDist[alpha=.1,beta=.5]{0.01}{0.99}
\end{psgraph*}
&  
\begin{psgraph*}[axesstyle=none,xticksize= 0 2 ,yticksize=0 1 , subticks=0, dy=.2,Dy=.2 ](0,0)(0,0)(1,2){3cm}{3cm }
 \psBetaDist[alpha=.1,beta=.9]{0.01}{0.99}
\end{psgraph*}
\\ \hline 
[alpha=.1,beta=0.1] & [alpha=.1,beta=0.5] & [alpha=.1,beta=0.9] \\ 
\hline 
\end{tabular}

\newpage

%\subsection{Loi de Cauchy}
\SbSSCT{Loi de Cauchy}{Cauchy Distribution}

\begin{center}
\begin{tabular}{|c|} \hline  
\begin{psgraph*}[axesstyle=none,xticksize= 0 1 ,yticksize=-3 3 , subticks=0, dy=.2,Dy=.2 ](0,0)(-3,0)(3,1){6cm}{2cm }
\psCauchy{-3}{3}
\end{psgraph*}
\\ \hline  
\BSS{psCauchy}\AC{-3}\AC{3} \BSI{psCauchy}{pst-func}
\\ \hline 
\end{tabular}
\end{center}

\bigskip

\begin{tabular}{|c|c|c|} \hline 
\multicolumn{3}{|c|}{ \BSS{psCauchy}[b=0.1]\AC{-3}\AC{3} }
\\ \hline 
\begin{psgraph*}[axesstyle=none,xticksize= 0 1 ,yticksize=-3  3 , subticks=0, dy=.2,Dy=.2 ](0,0)(-3,0)(3,1){3cm}{3cm }
\psCauchy[b=0.1]{-3}{3}
\end{psgraph*}
&  
\begin{psgraph*}[axesstyle=none,xticksize= 0 1 ,yticksize=-3  3 , subticks=0, dy=.2,Dy=.2 ](0,0)(-3,0)(3,1){3cm}{3cm }
\psCauchy[b=.5]{-3}{3}
\end{psgraph*}
&  
\begin{psgraph*}[axesstyle=none,xticksize= 0 1 ,yticksize=-3  3 , subticks=0, dy=.2,Dy=.2 ](0,0)(-3,0)(3,1){3cm}{3cm }
\psCauchy[b=1]{-3}{3}
\end{psgraph*}
\\ \hline 
[b=0.1]] & [b=0.5] & [b=1] \\ 
\hline 
\multicolumn{3}{|c|}{ \dft : b = 1 }
\\ \hline
\end{tabular}


\bigskip

\begin{tabular}{|c|c|c|} \hline 
\multicolumn{3}{|c|}{ \BSS{psCauchy}[m=0.1]\AC{-3}\AC{3} }
\\ \hline 
\begin{psgraph*}[axesstyle=none,xticksize= 0 1 ,yticksize=-3  3 , subticks=0, dy=.2,Dy=.2 ](0,0)(-3,0)(3,1){3cm}{3cm }
\psCauchy[m=-1]{-3}{3}
\end{psgraph*}
&  
\begin{psgraph*}[axesstyle=none,xticksize= 0 1 ,yticksize=-3  3 , subticks=0, dy=.2,Dy=.2 ](0,0)(-3,0)(3,1){3cm}{3cm }
\psCauchy[m=0]{-3}{3}
\end{psgraph*}
&  
\begin{psgraph*}[axesstyle=none,xticksize= 0 1 ,yticksize=-3  3 , subticks=0, dy=.2,Dy=.2 ](0,0)(-3,0)(3,1){3cm}{3cm }
\psCauchy[m=1]{-3}{3}
\end{psgraph*}
\\ \hline 
[m=-1]] & [m=0] & [m=1] \\ 
\hline 
\multicolumn{3}{|c|}{ \dft : m = 0 }
\\ \hline
\end{tabular}

\newpage

\begin{center}
\begin{tabular}{|c|} \hline  
\begin{psgraph*}[axesstyle=none,xticksize= 0 1 ,yticksize=-3 3 , subticks=0, dy=.2,Dy=.2 ](0,0)(-3,0)(3,1){6cm}{2cm }
\psCauchyI{-3}{3}
\end{psgraph*}
\\ \hline  
\BSS{psCauchyI}\AC{-3}\AC{3} \BSI{psCauchyI}{pst-func}
\\ \hline 
\end{tabular}
\end{center}




\bigskip

\begin{tabular}{|c|c|c|} \hline 
\multicolumn{3}{|c|}{ \BSS{psCauchyI}[b=0.1]\AC{-3}\AC{3} }
\\ \hline 
\begin{psgraph*}[axesstyle=none,xticksize= 0 1 ,yticksize=-3  3 , subticks=0, dy=.2,Dy=.2 ](0,0)(-3,0)(3,1){3cm}{3cm }
\psCauchyI[linestyle=dotted]{-2.5}{2.5}
\psCauchyI[b=0.1]{-3}{3}
\end{psgraph*}
&  
\begin{psgraph*}[axesstyle=none,xticksize= 0 1 ,yticksize=-3  3 , subticks=0, dy=.2,Dy=.2 ](0,0)(-3,0)(3,1){3cm}{3cm }
\psCauchyI[linestyle=dotted]{-3}{3}
\psCauchyI[b=.5]{-3}{3}
\end{psgraph*}
&  
\begin{psgraph*}[axesstyle=none,xticksize= 0 1 ,yticksize=-3  3 , subticks=0, dy=.2,Dy=.2 ](0,0)(-3,0)(3,1){3cm}{3cm }
\psCauchyI[linestyle=dotted]{-2.5}{2.5}
\psCauchyI[b=1]{-3}{3}
\end{psgraph*}
\\ \hline 
[b=0.1]] & [b=0.5] & [b=1] \\ 
\hline 
\multicolumn{3}{|c|}{ \dft : b = 1 }
\\ \hline
\end{tabular}


\bigskip

\begin{tabular}{|c|c|c|} \hline 
\multicolumn{3}{|c|}{ \BSS{psCauchyI}[m=0.1]\AC{-3}\AC{3} }
\\ \hline 
\begin{psgraph*}[axesstyle=none,xticksize= 0 1 ,yticksize=-3  3 , subticks=0, dy=.2,Dy=.2 ](0,0)(-3,0)(3,1){3cm}{3cm }
\psCauchyI[linestyle=dotted]{-3}{3}
\psCauchyI[m=-1]{-3}{3}
\end{psgraph*}
&  
\begin{psgraph*}[axesstyle=none,xticksize= 0 1 ,yticksize=-3  3 , subticks=0, dy=.2,Dy=.2 ](0,0)(-3,0)(3,1){3cm}{3cm }
\psCauchyI[linestyle=dotted]{-2.5}{2.5}
\psCauchyI[m=0]{-3}{3}
\end{psgraph*}
&  
\begin{psgraph*}[axesstyle=none,xticksize= 0 1 ,yticksize=-3  3 , subticks=0, dy=.2,Dy=.2 ](0,0)(-3,0)(3,1){3cm}{3cm }
\psCauchyI[linestyle=dotted]{-3}{3}
\psCauchyI[m=1]{-3}{3}
\end{psgraph*}
\\ \hline 
[m=-1] & [m=0] & [m=1] \\ 
\hline 
\multicolumn{3}{|c|}{ \dft : m = 0 }
\\ \hline
\end{tabular}

\newpage

%\subsection{Loi de Weibull}
\SbSSCT{Loi de Weibull}{Weibull Distribution}

\begin{center}
\begin{tabular}{|c|} \hline  
\begin{psgraph*}[axesstyle=none,xticksize= 0 1 ,yticksize=0 3 , subticks=0, dy=.2,Dy=.2 ](0,0)(0,0)(3,1){6cm}{2cm }
\psWeibull{0}{3}
\end{psgraph*}
\\ \hline  
\BSS{psWeibull}\AC{0}\AC{3} \BSI{psWeibull}{pst-func} 
\\ \hline 
\end{tabular}
\end{center}

\bigskip

\begin{tabular}{|c|c|} \hline  
\begin{psgraph*}[axesstyle=none,xticksize= 0 1 ,yticksize=0 3 , subticks=0, dy=.2,Dy=.2 ](0,0)(0,0)(3,1){6cm}{2cm }
\psWeibull[linestyle=dotted]{0}{3}
\psWeibull[alpha=.5]{0}{3}
\end{psgraph*}
& 
\begin{psgraph*}[axesstyle=none,xticksize= 0 1 ,yticksize=0 3 , subticks=0, dy=.2,Dy=.2 ](0,0)(0,0)(3,1){6cm}{2cm }
\psWeibull[linestyle=dotted]{0}{3}
\psWeibull[alpha=2]{0}{3}
\end{psgraph*}
 \\ \hline  
\BS{psWeibull}[\RDD{alpha}=.5]\AC{0}\AC{3} 
&  \BS{psWeibull}\RDD{alpha}=2]\AC{0}\AC{3} \\ 
\hline 
\multicolumn{2}{|c|}{ \dft : alpha=1 }
\\ \hline 
\end{tabular} 

\bigskip

\begin{tabular}{|c|c|} \hline  
\begin{psgraph*}[axesstyle=none,xticksize= 0 1 ,yticksize=0 3 , subticks=0, dy=.2,Dy=.2 ](0,0)(0,0)(3,1){6cm}{2cm }
\psWeibull[linestyle=dotted]{0}{3}
\psWeibull[beta=.5]{0}{3}
\end{psgraph*}
& 
\begin{psgraph*}[axesstyle=none,xticksize= 0 1 ,yticksize=0 3 , subticks=0, dy=.2,Dy=.2 ](0,0)(0,0)(3,1){6cm}{2cm }
\psWeibull[linestyle=dotted]{0}{3}
\psWeibull[beta=2]{0}{3}
\end{psgraph*}
 \\ \hline  
\BS{psWeibull}[\RDD{beta}=.5]\AC{0}\AC{3} 
&  \BS{psWeibull}\RDD{beta}=2]\AC{0}\AC{3} \\ 
\hline 
\multicolumn{2}{|c|}{ \dft :beta=1 }
\\ \hline
\end{tabular} 

\bigskip

\begin{tabular}{|c|c|} \hline  
\begin{psgraph*}[axesstyle=none,xticksize= 0 2 ,yticksize=0 3 , subticks=0, dy=.2,Dy=.2 ](0,0)(0,0)(3,2){6cm}{4cm }
\psWeibull[linestyle=dotted]{0}{3}
\psWeibull[alpha=2,beta=.5]{0}{3}
\end{psgraph*}
& 
\begin{psgraph*}[axesstyle=none,xticksize= 0 2 ,yticksize=0 3 , subticks=0, dy=.2,Dy=.2 ](0,0)(0,0)(3,2){6cm}{4cm }
\psWeibull[linestyle=dotted]{0}{3}
\psWeibull[alpha=2,beta=2]{0}{3}
\end{psgraph*}
 \\ \hline  
\BS{psWeibull}[alpha=2,beta=.5]\AC{0}\AC{3} 
&  \BS{psWeibull}[alpha=2,beta=2]\AC{0}\AC{3} \\ 
\hline 
\end{tabular} 


\newpage

\begin{center}
\begin{tabular}{|c|} \hline  
\begin{psgraph*}[axesstyle=none,xticksize= 0 1 ,yticksize=0 3 , subticks=0, dy=.2,Dy=.2 ](0,0)(0,0)(3,1){6cm}{2cm }
\psWeibullI{0}{3}
\end{psgraph*}
\\ \hline  
\BSS{psWeibullI}\AC{0}\AC{3}  \BSI{psWeibullI}{pst-func}
\\ \hline 
\end{tabular}
\end{center}

\bigskip


\begin{tabular}{|c|c|} \hline  
\begin{psgraph*}[axesstyle=none,xticksize= 0 1 ,yticksize=0 3 , subticks=0, dy=.2,Dy=.2 ](0,0)(0,0)(3,1){6cm}{2cm }
\psWeibullI[linestyle=dotted]{0}{3}
\psWeibullI[alpha=.5]{0}{3}
\end{psgraph*}
& 
\begin{psgraph*}[axesstyle=none,xticksize= 0 1 ,yticksize=0 3 , subticks=0, dy=.2,Dy=.2 ](0,0)(0,0)(3,1){6cm}{2cm }
\psWeibullI[linestyle=dotted]{0}{3}
\psWeibullI[alpha=2]{0}{3}
\end{psgraph*}
 \\ \hline  
\BS{psWeibullI}[\RDD{alpha}=.5]\AC{0}\AC{3}  \RDI{alpha}{pst-func}
&  \BS{psWeibullI}\RDD{alpha}=2]\AC{0}\AC{3} \\ 
\hline 
\multicolumn{2}{|c|}{ \dft : alpha=1 }
\\ \hline
\end{tabular} 


\bigskip


\begin{tabular}{|c|c|} \hline  
\begin{psgraph*}[axesstyle=none,xticksize= 0 1 ,yticksize=0 3 , subticks=0, dy=.2,Dy=.2 ](0,0)(0,0)(3,1){6cm}{2cm }
\psWeibullI[linestyle=dotted]{0}{3}
\psWeibullI[beta=.5]{0}{3}
\end{psgraph*}
& 
\begin{psgraph*}[axesstyle=none,xticksize= 0 1 ,yticksize=0 3 , subticks=0, dy=.2,Dy=.2 ](0,0)(0,0)(3,1){6cm}{2cm }
\psWeibullI[linestyle=dotted]{0}{3}
\psWeibullI[beta=2]{0}{3}
\end{psgraph*}
 \\ \hline  
\BS{psWeibullI}[\RDD{beta}=.5]\AC{0}\AC{3} \RDI{beta}{pst-func}
&  \BS{psWeibullI}\RDD{beta}=2]\AC{0}\AC{3} \\ 
\hline
\multicolumn{2}{|c|}{ \dft : beta=1 }
\\ \hline 
\end{tabular} 


\bigskip

\begin{tabular}{|c|c|} \hline  
\begin{psgraph*}[axesstyle=none,xticksize= 0 1 ,yticksize=0 3 , subticks=0, dy=.2,Dy=.2 ](0,0)(0,0)(3,1){6cm}{2cm }
\psWeibullI[linestyle=dotted]{0}{3}
\psWeibullI[alpha=2,beta=.5]{0}{3}
\end{psgraph*}
& 
\begin{psgraph*}[axesstyle=none,xticksize= 0 1 ,yticksize=0 3 , subticks=0, dy=.2,Dy=.2 ](0,0)(0,0)(3,1){6cm}{2cm }
\psWeibullI[linestyle=dotted]{0}{3}
\psWeibullI[alpha=2,beta=2]{0}{3}
\end{psgraph*}
 \\ \hline  
\BS{psWeibullI}[alpha=2,beta=.5]\AC{0}\AC{3} 
&  \BS{psWeibullI}[alpha=2,beta=2]\AC{0}\AC{3} \\ 
\hline 
\end{tabular}

\newpage
%\subsection{Loi de Vasicek}
\SbSSCT{Loi de Vasicek}{Vasicek Distribution}

\begin{center}
\begin{tabular}{|c|} \hline  
\begin{psgraph*}[axesstyle=none,xticksize= 0 5, dx=.2,Dx=.2 ,yticksize=0 1, subticks=0 ](0,0)(0,0)(1,5){6cm}{3cm }
 \psVasicek{0}{0.9999}
\end{psgraph*}
\\ \hline  
\BSS{psVasicek}\AC{0}\AC{3}  \BSI{psVasicek}{pst-func}
\\ \hline 
\end{tabular}
\end{center}

\bigskip


\begin{tabular}{|c|c|} \hline  
\begin{psgraph*}[axesstyle=none,xticksize= 0 10, dx=.2,Dx=.2 ,yticksize=0 1, subticks=0 ](0,0)(0,0)(1,10){6cm}{3cm }
\psVasicek[linestyle=dotted]{0}{0.9999}
\psVasicek[pd=.1]{0}{0.9999}
\end{psgraph*}
& 
\begin{psgraph*}[axesstyle=none,xticksize= 0 5, dx=.2,Dx=.2 ,yticksize=0 1, subticks=0 ](0,0)(0,0)(1,5){6cm}{3cm }
\psVasicek[linestyle=dotted]{0}{0.9999}
\psVasicek[pd=.5]{0}{0.9999}
\end{psgraph*}
 \\ \hline  
\BS{psVasicek}[\RDD{pd}=.1]\AC{0}\AC{3}  \RDI{pd}{pst-func}
&  \BS{psVasicek}[\RDD{pd}=.5]\AC{0}\AC{3} \\ 
\hline 
\multicolumn{2}{|c|}{ \dft : pd = 0.22 }
\\ \hline
\end{tabular} 

\bigskip


\begin{tabular}{|c|c|} \hline  
\begin{psgraph*}[axesstyle=none,xticksize= 0 10, dx=.2,Dx=.2 ,yticksize=0 1, subticks=0 ](0,0)(0,0)(1,10){6cm}{3cm }
\psVasicek[linestyle=dotted]{0}{0.9999}
\psVasicek[R2=.05]{0}{0.9999}
\end{psgraph*}
& 
\begin{psgraph*}[axesstyle=none,xticksize= 0 5, dx=.2,Dx=.2 ,yticksize=0 1, subticks=0 ](0,0)(0,0)(1,5){6cm}{3cm }
\psVasicek[linestyle=dotted]{0}{0.9999}
\psVasicek[R2=.2]{0}{0.9999}
\end{psgraph*}
 \\ \hline  
\BS{psVasicek}[\RDD{R2}=.05]\AC{0}\AC{3}  \RDI{R2}{pst-func}
&  \BS{psVasicek}[\RDD{R2}=.2]\AC{0}\AC{3} \\ 
\hline 
\multicolumn{2}{|c|}{ \dft : R2 = 0.11 }
\\ \hline
\end{tabular}


\bigskip


\begin{tabular}{|c|c|} \hline  
\begin{psgraph*}[axesstyle=none,xticksize= 0 5, dx=.2,Dx=.2 ,yticksize=0 1, subticks=0 ](0,0)(0,0)(1,5){6cm}{3cm }
\psVasicek[linestyle=dotted,pd=.5]{0}{0.9999}
\psVasicek[pd=.5,R2=.05]{0}{0.9999}
\end{psgraph*}
& 
\begin{psgraph*}[axesstyle=none,xticksize= 0 5, dx=.2,Dx=.2 ,yticksize=0 1, subticks=0 ](0,0)(0,0)(1,5){6cm}{3cm }
\psVasicek[linestyle=dotted,pd=.5]{0}{0.9999}
\psVasicek[pd=.5,R2=.2]{0}{0.9999}
\end{psgraph*}
 \\ \hline  
\BS{psVasicek}[pd=.5,R2=.05]\AC{0}\AC{3}  \RDI{R2}{pst-func}
&  \BS{psVasicek}[pd=.5,R2=.2]\AC{0}\AC{3} \\ 
\hline 
\end{tabular}

\newpage
%\subsection{Courbe de Lorenz}
\SbSSCT{Courbe de Lorenz}{Lorenz curve}

\begin{tabular}{|c|c|} \hline  
\begin{psgraph*}[axesstyle=none,xticksize= 0 1,dx=.2,Dx=.2 , dy=.2,Dy=.2 ,yticksize=0 1, subticks=0 ](0,0)(0,0)(1,1){5cm}{3cm }
 \psLorenz{0.1 0.2 0.3 }
\end{psgraph*}
&
\begin{psgraph*}[axesstyle=none,xticksize= 0 1,dx=.2,Dx=.2 , dy=.2,Dy=.2 ,yticksize=0 1, subticks=0 ](0,0)(0,0)(1,1){5cm}{3cm }
 \psLorenz*{.1 .2 .3 }
\end{psgraph*}
\\ \hline  
\BSS{psLorenz}\AC{0.1 0.2 0.3} \BSI{psLorenz}{pst-func}
&
\BSS{psLorenz}*\AC{0.1 0.2 0.3 }
\\ \hline 
\end{tabular}


\bigskip

\begin{tabular}{|c|c|} \hline  
\begin{psgraph*}[axesstyle=none,xticksize= 0 1, dx=.2,Dx=.2 ,yticksize=0 1, subticks=0,lly=-13mm ](0,0)(0,0)(1,1){5cm}{3cm }
 \psLorenz[linestyle=dotted]{.1 .2 .3 }
 \psLorenz[plotstyle=bezier]{.1 .2 .3 }
\end{psgraph*}
&  
\begin{psgraph*}[axesstyle=none,xticksize= 0 1, dx=.2,Dx=.2 ,yticksize=0 1, subticks=0,lly=-13mm ](0,0)(0,0)(1,1){5cm}{3cm }
 \psLorenz[linestyle=dotted]{.1 .2 .3 }
 \psLorenz[Gini]{.1 .2 .3 }
\end{psgraph*}
\\ \hline  
\BS{psLorenz}[plotstyle=bezier]\AC{.1 .2 .3}  
&
\BS{psLorenz}[\RDD{Gini}]\AC{.1 .2 .3 }
 \\ \hline 
\end{tabular} 

\newpage
%\subsection{Courbe de Lamé : superellipses}
\SbSSCT{Courbe de Lamé : superellipses}{Lame curve}


 \begin{tabular}{|c|c|c|c|}  \hline  
\begin{psgraph*}[axesstyle=none,xticksize= -1 1 , yticksize=-1 1, subticks=0 ](0,0)(-1,-1)(1,1){2cm}{2cm}
\psLame{.5}
\end{psgraph*} 
 &  
\begin{psgraph*}[axesstyle=none,xticksize= -1 1 , yticksize=-1 1, subticks=0 ](0,0)(-1,-1)(1,1){2cm}{2cm}
\psLame{.75}
\end{psgraph*} 
 &  
\begin{psgraph*}[axesstyle=none,xticksize= -1 1 , yticksize=-1 1, subticks=0 ](0,0)(-1,-1)(1,1){2cm}{2cm}
\psLame{2}
\end{psgraph*} 
 &  
\begin{psgraph*}[axesstyle=none,xticksize= -1 1 , yticksize=-1 1, subticks=0 ](0,0)(-1,-1)(1,1){2cm}{2cm}
\psLame{5}
\end{psgraph*} 
 \\  \hline 
 \BSS{psLame}\AC{.5} \BSI{psLame}{pst-func} &  \BSS{psLame}\AC{.75} &  \BSS{psLame}\AC{2} &   \BSS{psLame}\AC{5}\\ 
 \hline 
 \end{tabular} 

\bigskip

\begin{tabular}{|c|c|} \hline  
\begin{psgraph*}[axesstyle=none,xticksize= -2 2 ,yticksize=-2 2,subticks=0 ](0,0)(-2,-2)(2,2){4cm}{4cm }
\psLame[radiusA=2]{.5}
\end{psgraph*}
&  
\begin{psgraph*}[axesstyle=none,xticksize= -2 2 ,yticksize=-2 2,subticks=0 ](0,0)(-2,-2)(2,2){4cm}{4cm }
\psLame[radiusB=2]{.5}
\end{psgraph*}
\\ \hline 
\BS{psLame}[\RDD{radiusA}=2]\AC{.5} \RDI{radiusA}{pst-func}&
\BS{psLame}[\RDD{radiusB}=2]\AC{.5}  \RDI{radiusB}{pst-func}\\ 
\hline 
\end{tabular} 


%\subsection{Fonction de Thomae}
\SbSSCT{Fonction de Thomae}{Thomae curve}

 \psset{unit=4cm}


\begin{tabular}{|c|c|c|c|} \hline  
\begin{psgraph*}[axesstyle=none,xticksize= 0 1, dx=.2,Dx=.2, dy=.2,Dy=.2  ,yticksize=0 1, subticks=0,lly=-13mm ](0,0)(0,0)(1,1){2.5cm}{2.5cm }
 \psThomae[dotsize=5pt](0,1){1}
\end{psgraph*}
&  
\begin{psgraph*}[axesstyle=none,xticksize= 0 1, dx=.2,Dx=.2, dy=.2,Dy=.2  ,yticksize=0 1, subticks=0,lly=-13mm ](0,0)(0,0)(1,1){2.5cm}{2.5cm }
 \psThomae[dotsize=5pt](0,1){2}
\end{psgraph*}
&  
\begin{psgraph*}[axesstyle=none,xticksize= 0 1, dx=.2,Dx=.2, dy=.2,Dy=.2  ,yticksize=0 1, subticks=0,lly=-13mm ](0,0)(0,0)(1,1){2.5cm}{2.5cm }
 \psThomae[dotsize=5pt](0,1){3}
\end{psgraph*}
&  
\begin{psgraph*}[axesstyle=none,xticksize= 0 1, dx=.2,Dx=.2, dy=.2,Dy=.2  ,yticksize=0 1, subticks=0,lly=-13mm ](0,0)(0,0)(1,1){2.5cm}{2.5cm }
 \psThomae[dotsize=5pt](0,1){10}
\end{psgraph*}
\\ 
\hline 
\BSS{psThomae}(0,1)\AC{1}  \BSI{psThomae}{pst-func} & \BSS{psThomae}(0,1)\AC{2}  & \BSS{psThomae}(0,1)\AC{3} & \BSS{psThomae}(0,1)\AC{10}\\ 
\hline 
\end{tabular} 

\bigskip

\begin{tabular}{|c|c|} \hline  
\begin{psgraph*}[axesstyle=none,xticksize= 0 1, dx=.5,Dx=.5, dy=.5,Dy=.5  ,yticksize=0 2, subticks=0 ](0,0)(0,0)(2,1){5cm}{3cm }
 \psThomae[dotsize=5pt](0,2){10}
\end{psgraph*}
&  
\begin{psgraph*}[axesstyle=none,xticksize= 0 1, dx=.5,Dx=.5, dy=.5,Dy=.5  ,yticksize=0 2.5, subticks=0 ](0,0)(0,0)(2.5,1){7cm}{3cm }
 \psThomae[dotsize=5pt](0.5,2.5){10}
\end{psgraph*}
\\ \hline 
\BSS{psThomae}(0,2){10}(0,2)\AC{10} & \BSS{psThomae}(0,2){10}(0.5,2)\AC{10} \\ 
\hline 
\end{tabular}


%\subsection{Fonction de Weierstrass}
\SbSSCT{Fonction de Weierstrass}{Weierstrass curve}

\begin{tabular}{|c|c|} \hline  
\begin{psgraph*}[axesstyle=none,xticksize= -.5  .5, dx=.5,Dx=.5, dy=.5,Dy=.5  ,yticksize=0 2, subticks=0 ](0,0)(0,-.5)(2,.5){5cm}{3cm }
 \psWeierstrass(0,2){2}
\end{psgraph*}
&  
\begin{psgraph*}[axesstyle=none,xticksize= -.5  .5, dx=.5,Dx=.5, dy=.5,Dy=.5  ,yticksize=0 2, subticks=0 ](0,0)(0,-.5)(2,.5){5cm}{3cm }
% \psWeierstrass[linestyle=dotted](0,2){2}
 \psWeierstrass(0,2){5}
\end{psgraph*}
\\ \hline  
\BSS{psWeierstrass}(0,2)\AC{2} & \BSS{psWeierstrass}(0,2)\AC{5}  \BSI{psWeierstrass}{pst-func}\\ 
\hline 
\end{tabular} 

\bigskip

 \begin{tabular}{|c|c|} \hline  
 \begin{psgraph*}[axesstyle=none,xticksize= 0  .5, dx=.1,Dx=.1, dy=.1,Dy=.1  ,yticksize=.5 1, subticks=0 ](0,0)(0.5,0)(1,.5){5cm}{3cm }
  \psWeierstrass(.5,1){2}
 \end{psgraph*}
 &  
 \begin{psgraph*}[axesstyle=none,xticksize= 0  .05, dx=.01,Dx=.01, dy=.01,Dy=.01  ,yticksize=.95 1, subticks=0 ](0,0)(0.95,0)(1,.05){5cm}{3cm }
  \psWeierstrass(.95,1){2}
 \end{psgraph*}

 \\ \hline  
 \BSS{psWeierstrass}(.5,1)\AC{2} & \BSS{psWeierstrass}(.95,1)\AC{5}  \BSI{psWeierstrass}{pst-func}\\ 
 \hline 
 \end{tabular} 
 
 \bigskip
 
\begin{tabular}{|c|c|} \hline  
\begin{psgraph*}[axesstyle=none,xticksize= -.5  .5, dx=.5,Dx=.5, dy=.5,Dy=.5  ,yticksize=0 2, subticks=0 ](0,0)(0,-.5)(2,.5){5cm}{3cm }
 \psWeierstrass[linestyle=dotted](0,2){2}
 \psWeierstrass[epsilon=1.e-1](0,5){2}
\end{psgraph*}
&  
\begin{psgraph*}[axesstyle=none,xticksize= -.5  .5, dx=.5,Dx=.5, dy=.5,Dy=.5  ,yticksize=0 2, subticks=0 ](0,0)(0,-.5)(2,.5){5cm}{3cm }
 \psWeierstrass[linestyle=dotted](0,2){2}
 \psWeierstrass[epsilon=1](0,5){2}
\end{psgraph*}
\\ \hline  
 \BSS{psWeierstrass}[\RDD{epsilon}=1.e-1](0,5)\AC{2} &   \BSS{psWeierstrass}[\RDD{epsilon}=1](0,5)\AC{2}
 \RDI{epsilon}{pst-func} \\ \hline 
\multicolumn{2}{|c|}{ \dft : epsilon=1.e-18  } 
\\ \hline 
 
\end{tabular} 
 
\newpage
%\subsection{Fonction définie implicitement}
\SbSSCT{Fonction définie implicitement}{implicit defined functions}

\begin{tabular}{|c|} \hline  
\begin{psgraph*}[axesstyle=none,xticksize= -6  3 ,yticksize=-5 5, subticks=0 ](0,0)(-5,-6)(5,3){10cm}{5cm }
 \psplotImp[linewidth=2pt](-6,-7)(4,3){4 x 3 exp y 3 exp add 4 x y mul mul sub }
\end{psgraph*}
\\ \hline  
 \BSS{psplotImp}(-6,-7)(4,3)\AC{4 x 3 exp y 3 exp add 4 x y mul mul sub }
\\ \hline 
\end{tabular}

\bigskip

\begin{tabular}{|c|} \hline   
\begin{psgraph*}[axesstyle=none,xticksize= -6  3 ,yticksize=-5 5, subticks=0 ](0,0)(-5,-6)(5,3){10cm}{5cm }
 \psplotImp[algebraic,linewidth=2pt](-6,-7)(4,3){x^3 +y^3 -4*x*y}
\end{psgraph*}
\\ \hline  
 \BSS{psplotImp}[\RDD{algebraic}](-6,-7)(4,3)\AC{x\^{}3 +y\^{}3 -4*x*y }
\\ \hline 
\end{tabular}

\bigskip
\begin{tabular}{|c|} \hline   
\begin{psgraph*}[axesstyle=none,xticksize= -6  3 ,yticksize=-5 5, subticks=0 ](0,0)(-5,-6)(5,3){10cm}{5cm }
 \psplotImp[algebraic,linewidth=2pt,stepFactor=2](-6,-7)(4,3){x^3 +y^3 -4*x*y}
\end{psgraph*}
\\ \hline  
 \BSS{psplotImp}[algebraic,\RDD{stepFactor}=2](-6,-7)(4,3)\AC{x\^{}3 +y\^{}3 -4*x*y }
\\ \hline 
\end{tabular}

\bigskip
\begin{tabular}{|c|} \hline   
\begin{psgraph*}[axesstyle=none,xticksize= -1  1 ,yticksize=-1 1, subticks=0 ](0,0)(-1,-1)(1,1){5cm}{5cm }
 \psplotImp[algebraic,polarplot,linewidth=1pt](-1,-1)(1,1){ r + cos(10*phi) }
\end{psgraph*}
\\ \hline  
 \BSS{psplotImp}[algebraic,\RDD{polarplot}](-1,-1)(1,1)\AC{r + cos(10*phi) }
\\ \hline 
\end{tabular}

\bigskip

\begin{tabular}{|c|} \hline   
\begin{psgraph*}[axesstyle=none,xticksize= -1  1 ,yticksize=-1 1, subticks=0 ](0,0)(-1,-1)(1,1){5cm}{5cm }
 \psplotImp[algebraic,polarplot,linewidth=1pt,stepFactor=1](-1,-1)(1,1){ r + cos(10*phi) }
\end{psgraph*}
\\ \hline  
 \BSS{psplotImp}[algebraic,polarplot,\RDD{stepFactor}=1](-1,-1)(1,1)\AC{r + cos(10*phi) }
\\ \hline 
\end{tabular}

\newpage

%\subsection{Fonction de rotation}
\SbSSCT{Fonction de rotation}{Rotating functions}

 
\begin{tabular}{|c|}\hline  
\begin{psgraph*}[axesstyle=none,xticksize= -2  2 ,yticksize=0 5, subticks=0 ](0,0)(0,-2)(5,2){10cm}{5cm }
 \psVolume[fillstyle=solid,fillcolor=blue!40](0,4){4}{x sqrt}
\end{psgraph*}

\\ \hline  
 \BSS{psVolume}[fillstyle=solid,fillcolor=blue!40](0,4)\AC{4}\AC{x sqrt}  \BSI{psVolume}{pst-func}
\\ \hline 
\end{tabular} 

\bigskip

\begin{tabular}{|c|}\hline  
\begin{psgraph*}[axesstyle=none,xticksize= -1  1 ,yticksize=0 7, subticks=0 ](0,0)(0,-1)(7,1){10cm}{5cm }
 \psVolume[fillstyle=solid,fillcolor=yellow,algebraic](0,6.28){20}{cos(x)}
\end{psgraph*}
\\ \hline  
 \BSS{psVolume}[fillstyle=solid,fillcolor=yellow,algebraic](0,6.28)\AC{20}\AC{cos(x)}
\\ \hline 
\end{tabular}