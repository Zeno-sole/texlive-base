%%%%%%%%%%%%%%%%%%%%%%%%%%%%%%%%%%%%%%%%%%%%%%%%%%%%%%%%%%%%%%%%%%%%%%%%%%%%%%
%  Molecular Coding Format manual                by  Akira Yamaji 2021.09.20
%%%%%%%%%%%%%%%%%%%%%%%%%%%%%%%%%%%%%%%%%%%%%%%%%%%%%%%%%%%%%%%%%%%%%%%%%%%%%%
\documentclass[a4paper]{article}
\usepackage[pdftex]{graphicx}
\usepackage[pdftex]{hyperref}
\usepackage{makeidx}
\makeindex
\hypersetup{colorlinks=true,linkcolor=blue}
\usepackage{mcf_setup}
\edef\MCFjobname{mcf_man_soc}%
%----------------------------------------------------------------------------
%%%%\pdfpkresolution=1200
%----------------------------------------------------------------------------
%%%%\edef\fext{pk}%   **** for proof print (fast, not complete output)
\edef\fext{mps}%  **** for final print (it takes long time)
%----------------------------------------------------------------------------
%%%%\edef\fext{png}%
%%%%\edef\fext{svg}%
%%%%\edef\fext{eps}%
%----------------------------------------------------------------------------
\topmargin=-18mm
\textheight=254mm
\textwidth=168mm
\oddsidemargin=0mm
%%%%\oddsidemargin=7mm
%%%%\evensidemargin=-7mm
\unitlength=1mm%
%----------------------------------------------------------------------------
\newcount \fnum%
\newdimen \htman%
\newdimen \wdman%
\newdimen \htmans%
\newbox \fbox%
%----------------------------------------------------------------------------
\htman=45mm%
\wdman=94mm%
\htmans=42mm%
\fnum=1%
%----------------------------------------------------------------------------
\makeatletter
%----------------------------------------------------------------------------
\font\@strufont=\MCFjobname\relax%
%----------------------------------------------------------------------------
\def\MCFgraph{%
\ifx\fext\@pk{\@strufont\char\fnum}%
\else%
\edef\file@name{\MCFjobname-\z@num\fnum.\fext}%
\setbox\fbox=\hbox{\@strufont\char\fnum}%
\includegraphics[width=\wd\fbox,height=\ht\fbox]{\file@name}%
%%%%%%\includegraphics{\file@name}%
\fi%
\global\advance\fnum\@ne\relax%
}%
%----------------------------------------------------------------------------
\def\put@char{%
  \begin{picture}(75,100)%
     \put(0,95){\bf [\NO]\EN}%
     \put(5,90){\small\tt FM:\fm{ }MW:\mw}%
     \put(5,0){\MCFgraph}%
  \end{picture}%
}%
%----------------------------------------------------------------------------
\def\INFO#1{\@for\@temp:=#1\do{\tag@var\@temp}\put@char}%
%----------------------------------------------------------------------------
\begin{document}
\title{\Huge\sf Molecular Coding Format manual}
\author{Akira Yamaji}
\date{\today}
\maketitle
\begin{center} Located at http://www.ctan.org/pkg/mcf2graph \end{center}
\begin{center} Suggestion or request mail to: mcf2graph@gmail.com \end{center}
%-----------------------------------------------------------------------------
\thispagestyle{empty}
\vspace{5mm}%
\MCFgraph\MCFgraph\MCFgraph\MCFgraph\\
\MCFgraph\MCFgraph\MCFgraph\MCFgraph\\
\MCFgraph\MCFgraph\MCFgraph\MCFgraph\\
\MCFgraph\MCFgraph\MCFgraph\MCFgraph\\
\MCFgraph\MCFgraph\MCFgraph\MCFgraph\\
\MCFgraph\MCFgraph\MCFgraph\MCFgraph\\
\MCFgraph\MCFgraph\MCFgraph\MCFgraph
%-----------------------------------------------------------------------------
\twocolumn
\thispagestyle{empty}
\tableofcontents
%-----------------------------------------------------------------------------
\linethickness{0.08mm}%
%----------------------------------------------------------------------------
\newpage
\twocolumn
\section{Introduction}
Molecular Coding Format(MCF) is new linear notation
represent chemical structure diagrams.
This Coding is named from programing technique
such as operator, array, scope, macro, adressing, etc.
mcf2graph convert from MCF to pk font, PNG, SVG, EPS, MOL file.
It is also able to calculate molecular weight,
exact mass, molecular formula.
%-----------------------------------------------------------------------------
\section{MCF syntax}
\subsection{Make bond}
\subsubsection{Chain}
\begin{verbatim}
real number plus (+): counterclockwize 
real number minus(-): clockwize
$n (0<=n<360): absolute angle

<10,-30,45,-45,60,$300,$0
\end{verbatim}
\MCFgraph
%-----------------------------------------------------------------------------
\subsubsection{Chain use !,!n}
\index{"!}%
\begin{verbatim}
!  : take value 60 or -60 depend on
     current angle and environment
!6 : !,!,!,!,!,!

<-30,!6
\end{verbatim}
\MCFgraph
%-----------------------------------------------------------------------------
\subsubsection{Jump to atom}
\index{"@}%
\begin{verbatim}
@n : Jump to An
** An: atom number(-999<=n<=4095)

<-30,!6,@3,0,!,@5,-30
\end{verbatim}
\MCFgraph
%------------------------------------
\subsubsection{Branch bond}
\index{\textbackslash}%
\begin{verbatim}
\ : 0

<-30,!6,@3,\,!
\end{verbatim}
\MCFgraph
%-----------------------------------------------------------------------------
\subsubsection{Branch modified bond}
\index{*\textbackslash}%
\index{\textbackslash*}%
\index{\textbackslash\textbackslash}%
\index{*\textbackslash*}%
\begin{verbatim}
\   : 0
*\  : 0~wf
\*  : 0~zf
\\  : 0~dm
*\* : 0~wv

<30,!8,
@2,\,!,@4,*\,!,@6,\*,!,@8,\\,!,@10,*\*,!
\end{verbatim}
\MCFgraph
%-----------------------------------------------------------------------------
\index{\textasciicircum}% ^
\index{\textasciitilde}% ~
\index{`}%
\begin{verbatim}
<30,!6,
\~dr,!,   : 0~dr,!
\`1.5,-90 : 0`1.5,-90
\^15,-60  : 0^15,-60
\end{verbatim}
\MCFgraph
%-----------------------------------------------------------------------------
\subsubsection{Connect atom}
\index{\&}%
\begin{verbatim}
&n : Connect to An

<-30,!6,@3,\,!3,&6~bd,@9,&4~bz
\end{verbatim}
\MCFgraph
%-----------------------------------------------------------------------------
\subsubsection{Ring}
\index{?}%
\begin{verbatim}
?n : n membered ring(3<=n<=20)
?6 : <-120,60,60,60,60,60,&1
?6
\end{verbatim}
\MCFgraph
%-----------------------------------------------------------------------------
\subsubsection{Rotate current angle}
\index{\textgreater}%
\begin{verbatim}
<angle : rotate current angle

0,0,<90,0,<-90,0,<$315,0,<$90,0,<$0,0 
\end{verbatim}
\MCFgraph
%-----------------------------------------------------------------------------
\subsection{Change bond type}
\subsubsection{Double,triple,wedge,vector}
\index{\textasciitilde}%
\index{\textasciitilde\textasciitilde}%
\index{"!"!}%
\index{"!"!"!}%
\index{dm}%
\index{dl}%
\index{dr}%
\index{db}%
\index{tm}%
\begin{verbatim}

(Double,triple)
a~type : ~~type,a
dm  : double middle
dl  : double left side
dr  : double right side
db  : double left or right side
tm  : triple
!!  : !~db  / !!! : !~tm

<-30,!~dm,!,!~dl,!,!~dr,!~db,!~db,!,!~tm
<-30,!~dm,!,!~dl,!,!~dr,!!  ,!!  ,!,!!!
\end{verbatim}
\MCFgraph
\vspace{-3mm}%
%-----------------------------------------------------------------------------
%%\subsubsection{Wedge}
\index{wf}%
\index{wb}%
\index{zf}%
\index{zb}%
\index{vf}%
\index{vb}%
\begin{verbatim}

(Wedge,Vector)
wf: wedge forward
wb: wedge backward
zf: hashed(zebra stripe) wedge foward
zb: hashed(zebra stripe) wedge backward
vf:vector forward
vb:vector backward

<-30,
 !~wf,!,!~wb,!,!~zf,!,!~zb,!,!~vf,!~vb
\end{verbatim}
\MCFgraph
\vspace{-3mm}%
%-----------------------------------------------------------------------------
%%\subsubsection{Dotted,wave}
\index{dt}%
\index{wv}%
\index{bd}%
\index{bz}%
\begin{verbatim}

(Dotted,wave)
Bn=bond type : change bond type at Bn
dt : dotted   /  wv : wave
bd : broad    /  bz : broad dotted 

<-30,!7,1=dt,3=wv,5=bd,7=bz
\end{verbatim}
\MCFgraph
\vspace{-3mm}%
%-----------------------------------------------------------------------------
\subsubsection{Over line}
\index{si\_}%
\index{wf\_}%
\index{wb\_}%
\index{zf\_}%
\index{zb\_}%
\index{bd\_}%
\index{dl\_}%
\index{dr\_}%
\index{dm\_}%
\begin{verbatim}
si_ : single over line 
wf_ : wedge forward over line 
wb_ : wedge backward over line 
zf_ : hashed wedge forward over line 
zb_ : hashed wedge backward over line 
bd_ : broad over line 
dl_ : duble left over line 
dr_ : duble right over line 
dm_ : duble over line 

<30,!8,!,60,90`18,
{2~si_,4~wf_,6~wb_,8~zf_,10~zb_,
  12~bd_,14~dl_,16~dr_,18~dm_}:/_`2
\end{verbatim}
\MCFgraph
%-----------------------------------------------------------------------------
\subsubsection{Steric ring}
\index{wf\_r}%
\index{wb\_r}%
\index{bd\_r}%
\begin{verbatim}
wf_r : wedge foward (half width)
bd_r : broad (half width, rounded)
wb_r : wedge backward (half width)

#1.25,-30~wf_r,30~bd_r`1,30~wb_r,
 120,O,30,&1,##,6^$90:/!OH`-.5,
 {1^$-90,2^$90,3^$-90,4^$90}:/OH`-.5,
\end{verbatim}
\MCFgraph
%-----------------------------------------------------------------------------
\subsubsection{Change multiple bond type}
\index{vf}%
\begin{verbatim}
{2,4,6}=dr : 2=dr,4=dr,6=dr

<30,!7,{2,4,6}=dr
\end{verbatim}
\MCFgraph
%-----------------------------------------------------------------------------
\subsection{Change bond length}
\subsubsection{Chain length}
\index{`}%
\begin{verbatim}
(!,!n)`length : change length of !,!n

<-30,!2,!4`1.2,!2
\end{verbatim}
\MCFgraph
%-----------------------------------------------------------------------------
\index{\#}%
\index{\#\#}%
\begin{verbatim}
#n : bond length=n
## : reset bond length

<-30,!2,#1.2,!4,##,!2
\end{verbatim}
\MCFgraph
%-----------------------------------------------------------------------------
\subsubsection{Ring length}
\begin{verbatim}
?n`length : change ring length

?6,@4,\,?6`1.2
\end{verbatim}
\MCFgraph
%-----------------------------------------------------------------------------
\subsection{Change atom}
\subsubsection{Insert atom}
\begin{verbatim}
Insert hetero atom

<-30,!2,O,!2,N,!2
\end{verbatim}
\MCFgraph
%-----------------------------------------------------------------------------
\subsubsection{Addressed atom}
\index{:}%
\begin{verbatim}
2:O : change A2 C to O
{3,4}:N : change  A3,A4 C to N

<30,!4,2:O,{3,4}:N
\end{verbatim}
\MCFgraph
%-----------------------------------------------------------------------------
\subsubsection{Brock address}
\index{\textbar}% |
\begin{verbatim}
| : divide brock

?6,@4,\,|,?6,2:O
\end{verbatim}
\MCFgraph
%-----------------------------------------------------------------------------
\subsubsection{Reset brock address}
\index{\textbar\textbar}% ||
\begin{verbatim}
|| : reset brock adress

?6,@4,\,|,?6,||,2:N
\end{verbatim}
\MCFgraph
%-----------------------------------------------------------------------------
\subsubsection{Absolute address}
\index{\$}% $
\begin{verbatim}
$2:N : change A$2 C to N  **1<=n<=3095

?6,@4,\,|,?6,$2:N
\end{verbatim}
\MCFgraph
%-----------------------------------------------------------------------------
\subsubsection{Relative address}
\begin{verbatim}
-2:N : change A(-2) C to N **-999<=n<=-1

?6,@4,\,?6,-2:N
\end{verbatim}
\MCFgraph
%-----------------------------------------------------------------------------
\subsubsection{Charged atom}
\begin{verbatim}
p_ : positive / n_ : negative

<-30,!2,N,??,p_,!2,S,n_^180,
!6,7:N,7:??,9:S,7:n_,9:n_^180
\end{verbatim}
\MCFgraph
%-----------------------------------------------------------------------------
\newpage
%-----------------------------------------------------------------------------
\subsection{Fuse ring}
%%%\subsubsection{Attached 1 bond}
\begin{verbatim}
(Attached 1 bond)

?6,3=?6 : fuse ?6 at B3
** Bn(n:-999<=n<=4095): bond number

?6,3=?6
\end{verbatim}
\MCFgraph
%-----------------------------------------------------------------------------
\begin{verbatim}
** fused ring size depend on 
attached bond length

?6,@4,\,?6`1.2,5=?6,11=?6
\end{verbatim}
\MCFgraph
%-----------------------------------------------------------------------------
\begin{verbatim}
?6,3=?6[13] : fuse ?6[13] at B3
?6[13]: 6 membered ring scaled 13/10
** ?m[n] (5<=m<=8,11<=n<=15)

?6,3=?6[13]
\end{verbatim}
\MCFgraph
%-----------------------------------------------------------------------------
\begin{verbatim}
?6,{-3,-4,-4,-2,-2,-4,-4}=?6
?6,{4,8,13,20,25,28,33}=?6
\end{verbatim}
\MCFgraph
%-----------------------------------------------------------------------------
%%%\subsubsection{Attached 2 bond}
\begin{verbatim}
(Attached 2 bond)

(4,11)=?6[4]  : fuse 4/6 ring to B11..B4
(4,11)=?5[3]  : fuse 3/5 ring to B11..B4
(4,11)=?4[2]  : fuse 2/4 ring to B11..B4
** ?m[n] (4<=m<=6,n=m-2)

1:<30,?6,3=?6,(11,4)=?6[4]
2:<30,?6,3=?6,(11,4)=?5[3]
3:<30,?6,3=?6,(11,4)=?4[2]
\end{verbatim}
\MCFgraph
\vspace{-3mm}%
\begin{verbatim}
\end{verbatim}
%-----------------------------------------------------------------------------
%%%\subsubsection{Attached 3 bond}
\begin{verbatim}
(Attached 3 bond)

(16,4)=?6[3] : fuse 3/6 ring to B16..B4
(16,4)=?5[2] : fuse 2/5 ring to B16..B4
** ?m[n] (5<=m<=6,n=m-3)

1:?6,{3,10}=?6,(16,4)=?6[3]
2:?6,{3,10}=?6,(16,4)=?5[2]
\end{verbatim}
\MCFgraph
\vspace{-3mm}%
\begin{verbatim}
\end{verbatim}
%-----------------------------------------------------------------------------
%%%%\subsubsection{Attached 4 bond}
\begin{verbatim}
(Attached 4 bond)

(21,4)=?6[2] : fuse 2/6 ring to B21..B4

MC(<-30,?6,{3,10,15}=?6,(21,4)=?6[2])

** ?m[n] (m=6,n=2)
\end{verbatim}
\MCFgraph
%-----------------------------------------------------------------------------
\subsection{Spiro ring}
\begin{verbatim}
@4,?5 : add ?5 at A4

<30,!6,@4,?5
\end{verbatim}
\MCFgraph
%-----------------------------------------------------------------------------
\subsection{Group}
\subsubsection{Insert group}
\index{/}%
\index{Ph}%
\begin{verbatim}
/ : group start single bond

/_   : methyl
/!   : ethyl
/!2  : propyl
/?!  : isopropyl
/??! : tert-butyl
/Ph  : phenyl

<30,!,/_,!2,/!,!2,/!2,!4,/?!,
 !4,/??!,!2,/Ph^-60,!
\end{verbatim}
\MCFgraph
%-----------------------------------------------------------------------------
\subsubsection{Insert modified group}
\index{//}%
\index{*/}%
\index{/*}%
\index{*/*}%
\index{**}%
\begin{verbatim}
//  : double (double middle)
*/  : wedge forward
/*  : hashed wedge forward
*/* : wave
**  : direct

<30,!,//O,!2,*/H,!2,/*H,!2,*/*H,!2,**?3,!
\end{verbatim}
\MCFgraph
%-----------------------------------------------------------------------------
\index{\textasciicircum}% ^
\index{\textasciitilde}% ~
\index{`}%
\index{\textless}%
\begin{verbatim}
~ : change type
^ : change angle
` : change length
> : change environment

<-30,``1,!,
  /_`2^30,!2,/!2>lr,!2,/!2>rl,!)
\end{verbatim}
\MCFgraph
%-----------------------------------------------------------------------------
\subsubsection{Add group}
\begin{verbatim}
<30,!17,2:/_,4:/!,6:/!2,
 10:/?!,14:/??!,16:/Ph^-60
\end{verbatim}
\MCFgraph
%-----------------------------------------------------------------------------
\subsubsection{Add modified group}
\begin{verbatim}
~,^,` : change type,angle,length

<30,!6,{2~wf,4~zf,6^-30,8^$120}:/_
\end{verbatim}
\MCFgraph
%-----------------------------------------------------------------------------
\begin{verbatim}

^,`,> : change angle,length,environment

<-30,!7`1,3:/_`2^30,5:/!2>lr,7:/!2>rl
\end{verbatim}
\MCFgraph
%-----------------------------------------------------------------------------
\newpage
%-----------------------------------------------------------------------------
\subsection{Chain environment}
\subsubsection{Horizontal,vertical}
\index{hz}%
\index{vt}%
\index{"'}%
\begin{verbatim}
>hz : horizontal environment (default)
>vt : vertical environment

?4,
{3^-90,3^-30,3^90}:/'(!3,"{hz}")>hz,
{1^-60,1,1^60}:/'(!3,"{vt}")>vt
\end{verbatim}
\MCFgraph
%-----------------------------------------------------------------------------
\subsubsection{Left-right,right-left}
\index{lr}%
\index{rl}%
\begin{verbatim}
>lr : left-right environment
>rl : right-left environment

<-30,!6,
{3^-30,3,3^30}:/'(!3,"{lr}")>lr,
{5^-30,5,5^30}:/'(!3,"{rl}")>rl
\end{verbatim}
\MCFgraph
%-----------------------------------------------------------------------------
\subsubsection{Fixed rotate angle}
\index{\textgreater}%
\begin{verbatim}
>n : rotate n

<30,!4,
2:/!6>30,  % 2:\,30,30,30,30,30,30
4:/!4>-45  % 4:\,-45,-45,-45,-45

\end{verbatim}
\MCFgraph
%-----------------------------------------------------------------------------
\subsubsection{Multi rotate angle}
\begin{verbatim}
>'(90,-90,...) : rotate 90,-90,...

<30,!6,6>'(90,-90,90,-90,90):/!5
\end{verbatim}
\MCFgraph
%-----------------------------------------------------------------------------
\newpage
\subsection{Miscellaneous}
%-----------------------------------------------------------------------------
\subsubsection{Abbreviated parts}
\index{NH}%
\index{N"!}%
\index{N"!2}%
\index{SO}%
\index{SOO}%
\begin{verbatim}
NH  : N,/H~nl    N! : N,/_     N!2 : N,/!
SO  : S,//O    SOO : S,//O^35,//^-35

<-30,!2,NH,!2,N!,!2,N!2,SO,!2,SOO,!
\end{verbatim}
\MCFgraph
\index{?"!}%
\index{??}%
\index{??"!}%
\index{N?"!}%
\begin{verbatim}
?!   : /_,!           ??   : /_^35,/_-35
/?!  : isopropyl      /??! : tert-butyl
/N?! : dimethylamino

<30,!9`1,?!,!,??,!,2:??,4:/??,6:/??!,8:/N?!
\end{verbatim}
\MCFgraph
%-----------------------------------------------------------------------------
\subsubsection{Parts definition}
\begin{verbatim}
'(..) : user defined parts

iBuOH:='(!,/_,!,OH);
MC(<30,?6,{4,6}:/iBuOH)
\end{verbatim}
\MCFgraph
%-----------------------------------------------------------------------------
\subsubsection{Parts inline definition}
\begin{verbatim}
<30,!8,{2,6}:/'(!,/_,!,OH)
\end{verbatim}
\MCFgraph
%-----------------------------------------------------------------------------
\subsubsection{Move position}
\index{"@()}%
\begin{verbatim}
@(x,y) : Move (l*x,l*y) from current position
@$(x,y): Move (l*x,l*y) from origin(@1)
          ** l=bond length of ring

<30,?6,@3,!4,//O,!,O,n_^60,@$(6,1),H,p_^15
\end{verbatim}
\MCFgraph
%-----------------------------------------------------------------------------
\subsubsection{Serial number}
\index{--}%
\begin{verbatim}
6--10 : 6,7,8,9,10
<30,!14,{2,6--10,14}:/_~bd_r`0.5
\end{verbatim}
\MCFgraph
\newpage
%-----------------------------------------------------------------------------
\subsubsection{Change color}
\index{red}%
\index{blue}%
\index{green}%
\begin{verbatim}
(use with metapost only)
beginfont()
  MC(<30,Ph,{2,5}:N,3:/NH2,4:/COOH,
    %---------------------
    2:red,     % red   A2
    5:blue,    % blue  A5
    3=green    % green B3
    %---------------------
  )
endfont
\end{verbatim}
\MCFgraph
%-----------------------------------------------------------------------------
\subsubsection{Change font}
\index{atomfont}%
\begin{verbatim}
(use with metapost only)
beginfont()
  %----------------
  atomfont:="cmr8";
  %----------------
  MC(<30,Ph,{2,5}:N,3:/NH2,4:/COOH)
endfont
\end{verbatim}
\MCFgraph
%%%%%%%%%%%%%%%%%%%%%%%%%%%%%%%%%%%%%%%%%%%%%%%%%%%%%%%%%%%%%%%%%%%%%%%%%%%%%%%
\section{Option parameter}
%------------------------------------------------------------------------------
\subsection{Angle parameter}
\index{mangle}%
\begin{verbatim}
mangle=0   ** default

MCat(0.2,0.5)(Ph)
mangle:=30;
MCat(0.8,0.5)(Ph)
\end{verbatim}
\MCFgraph
%------------------------------------------------------------------------------
\subsection{Size/Ratio parameter}
%-----------------------------------------------------------------------------
\subsubsection{Bond length}
\index{blength}%
\begin{verbatim}
(fit to font size)
blength=0   ** default
\end{verbatim}
\MCFgraph
%-----------------------------------------
\begin{verbatim}
(ratio bond/font width)
blength=0.1  ** (0<blength<=1)
blength=60mm(width)*0.1=6mm
\end{verbatim}
\MCFgraph
%-----------------------------------------
\begin{verbatim}
(bond length)
blength=9mm
** (blength>1) ignore msize(w,h)
\end{verbatim}
\MCFgraph
%------------------------------------------------------------------------------
\subsubsection{Molecular size}
\index{msize}%
\begin{verbatim}
msize=(1,1)  ** default
\end{verbatim}
\MCFgraph
%-----------------------------------------------------------
\begin{verbatim}
msize=(0.25,1)
msize=40mm-4mm*0.25=9mm
\end{verbatim}
\MCFgraph
%-----------------------------------------------------------
\begin{verbatim}
msize=(11mm,11mm)
\end{verbatim}
\MCFgraph
%------------------------------------------------------------------------------
\subsubsection{Molecular position}
\index{mposition}%
\begin{verbatim}
mposition=(0.5,0.5) ** default
\end{verbatim}
\MCFgraph
%--------------------------------------------------------------
\begin{verbatim}
mposition=(1,0)
\end{verbatim}
\MCFgraph
%--------------------------------------------------------------
\begin{verbatim}
mposition=(10mm,4mm)
\end{verbatim}
\MCFgraph
%------------------------------------------------------------------------------
\subsection{Size parameter}
%-----------------------------------------------------------------------------
\subsubsection{Font size}
\index{fsize}%
\begin{verbatim}
fsize=(font width,font height)
** default: (30mm,20mm)

fsize=(40mm,15mm)
\end{verbatim}
\MCFgraph
%-----------------------------------------------------------------------------
\subsubsection{Font margin}
\index{fmargin}%
\begin{verbatim}
fmargin=(margin left rigth,top bottom)
** default: (0.4mm,0.4mm)

fmargin=(10mm,2mm)
\end{verbatim}
\MCFgraph
%-----------------------------------------------------------------------------
\subsubsection{Offset thickness of bond}
\index{offset\_thickness}%
\begin{verbatim}
default: offset_thickness=0.2pt
\end{verbatim}
\MCFgraph
%-----------------------------------------------------------------------------
\subsubsection{Offset of double bond gap}
\index{offset\_bond\_gap}%
\begin{verbatim}
default: offset_bond_gap=0.3pt
\end{verbatim}
\MCFgraph
%-----------------------------------------------------------------------------
\subsubsection{Offset of atom width}
\index{offset\_atom}%
\begin{verbatim}
default: offset_atom=0.8pt
\end{verbatim}
\MCFgraph
%-----------------------------------------------------------------------------
\subsubsection{Offset of wedge width}
\index{offset\_wedge}%
\begin{verbatim}
default:  offset_wedge=0.4pt
\end{verbatim}
\MCFgraph
%-----------------------------------------------------------------------------
\subsubsection{Max bond length}
\index{max\_blength}%
\begin{verbatim}
default:  max_blength=10mm
\end{verbatim}
\MCFgraph
%-----------------------------------------------------------------------------
\subsection{Ratio parameter}
%-----------------------------------------------------------------------------
\subsubsection{Thickness/bond length}
\index{ratio\_thickness\_bond}%
\begin{verbatim}
default:  ratio_thickness_bond=0.015
\end{verbatim}
\MCFgraph
%-----------------------------------------------------------------------------
\subsubsection{Char/bond thickness}
\index{ratio\_char\_bond}%
\begin{verbatim}
default:  ratio_char_bond=1.5
\end{verbatim}
\MCFgraph
%-----------------------------------------------------------------------------
\subsubsection{Bond gap/bond length}
\index{ratio\_bondgap\_bond}%
\begin{verbatim}
default:  ratio_bondgap_bond= 0.15
\end{verbatim}
\MCFgraph
%-----------------------------------------------------------------------------
\subsubsection{Atom/bond length}
\index{ratio\_atom\_bond}%
\begin{verbatim}
default:  ratio_atom_bond= 0.36
\end{verbatim}
\MCFgraph
%-----------------------------------------------------------------------------
\subsubsection{Wedge/bond length}
\index{ratio\_wedge\_bond}%
\begin{verbatim}
default:  ratio_wedge_bond=0.12
\end{verbatim}
\MCFgraph
%-----------------------------------------------------------------------------
\subsubsection{Font atom gap/atom length}
\index{ratio\_atomgap\_atom}%
\begin{verbatim}
default:  ratio_atomgap_atom= 0.050
\end{verbatim}
\MCFgraph
%-----------------------------------------------------------------------------
\subsubsection{Chain/ring length}
\index{ratio\_chain\_ring}%
\begin{verbatim}
default:  ratio_chain_ring= 0.66
\end{verbatim}
\MCFgraph
%-----------------------------------------------------------------------------
\subsubsection{Hash gap/bond length}
\index{ratio\_hashgap\_bond}%
\begin{verbatim}
default:  ratio_hashgap_bond=0.12
\end{verbatim}
\MCFgraph
%-----------------------------------------------------------------------------
%%%%\newpage
%-----------------------------------------------------------------------------
\subsection{Drawing mode}
%-----------------------------------------------------------------------------
\subsubsection{Numbering atom}
\index{sw\_numbering}%
\index{Atom}%
\index{Brock}%
\index{Inverse}%
\index{numbering\_start}%
\index{numbering\_end}%
\begin{verbatim}
sw_numbering=Atom
numbering_start:=3; numbering_end:=8;
default: sw_numbering=0 :
\end{verbatim}
\MCFgraph
\begin{picture}(5,20)
\put(0,14){\makebox[9mm]{\tt Atom}}
\put(0, 8){\makebox[12mm]{\tt +Brock}}
\put(0, 2){\makebox[16mm]{\tt +Inverse}}
\end{picture}
%-----------------------------------------------------------------------------
\subsubsection{Numbering bond}
\index{numbering\_start}%
\index{numbering\_end}%
\index{Bond}%
\index{Brock}%
\index{Inverse}%
\begin{verbatim}
sw_numbering=Bond
numbering_start:=3; numbering_end:=8;
default: sw_numbering=0 :
\end{verbatim}
\MCFgraph
\begin{picture}(5,20)
\put(0,14){\makebox[9mm]{\tt Bond}}
\put(0, 8){\makebox[12mm]{\tt +Brock}}
\put(0, 2){\makebox[16mm]{\tt +Inverse}}
\end{picture}
%-----------------------------------------------------------------------------
\subsubsection{Trimming mode}
\index{sw\_trimming}%
\begin{verbatim}
sw_trimming:=0;  ** default
msize:=(1,0.7);
MCat(0.2,0.3)(Ph)
MCat(0.8,0.7)(Ph)
\end{verbatim}
\MCFgraph
\begin{verbatim}
sw_trimming:=1;
MCat(0.2,0.3)(Ph)
MCat(0.8,0.7)(Ph)
\end{verbatim}
\MCFgraph
%-----------------------------------------------------------------------------
\subsubsection{Expand mode}
\index{sw\_expand}%
\begin{verbatim}
MCat(0, .5)(<30,Ph,4:/COOH,3:/NH2)
sw_expand:=1;
MCat(1, .5)(<30,Ph,4:/COOH,3:/NH2)
** default: sw_expand=0
\end{verbatim}
\MCFgraph
%-----------------------------------------------------------------------------
\subsubsection{Abbreviate group}
\index{Group}%
\index{sw\_abbreviate}%
\begin{verbatim}
** default: sw_abbreviate=Group
\end{verbatim}
\MCFgraph
%-----------------------------------------------------------------------------
\subsubsection{Abbreviate bond type}
\index{Bond}%
\index{sw\_abbreviate}%
\begin{verbatim}
** default: sw_abbreviate=Bond
\end{verbatim}
\MCFgraph
%-----------------------------------------------------------------------------
\subsection{Frame}
%-----------------------------------------------------------------------------
\subsubsection{Font frame}
\index{sw\_frame}%
\index{Bothside}%
\index{Inside}%
\index{Outside}%
\begin{verbatim}
** default:sw_frame=0
(Draw font frame)
fmargin:=(5mm,2mm);
sw_frame=Outside
\end{verbatim}
\MCFgraph
\begin{verbatim}
(Frame inside margin)
sw_frame=Inside
\end{verbatim}
\MCFgraph
\begin{verbatim}
(Draw both frame)
sw_frame=Bothside=Inside+Outside
\end{verbatim}
\MCFgraph
%-----------------------------------------------------------------------------
\subsubsection{Molecular frame}
\index{Mol}%
\begin{verbatim}
sw_frame=Mol
** default:sw_frame=0
\end{verbatim}
\MCFgraph
%-----------------------------------------------------------------------------
\subsubsection{Atom frame}
\index{Atom}%
\begin{verbatim}
sw_frame=Atom
** default: sw_frame=0

MC(<30,COOH,!,COOH)
\end{verbatim}
\MCFgraph
%-----------------------------------------------------------------------------
\subsection{Parameter setting}
\subsubsection{Local parameter setting}
\index{beginfont()}%
\index{endfont}%
\begin{verbatim}
beginfont()
  MC(Ph)
endfont
beginfont()
  %--------------------------
  ratio_thickness_bond:=0.05;
  %--------------------------
  MC(Ph)
endfont
beginfont()
  MC(Ph)
endfont
\end{verbatim}
\MCFgraph\MCFgraph\MCFgraph
%-----------------------------------------------------------------------------
\subsubsection{Global parameter setting}
\begin{verbatim}
beginfont()
  MC(Ph)
endfont
%--------------------------
ratio_thickness_bond:=0.05;
%--------------------------
beginfont()
  MC(Ph)
endfont
beginfont()
  MC(Ph)
endfont
\end{verbatim}
\MCFgraph\MCFgraph\MCFgraph
%%%%%%%%%%%%%%%%%%%%%%%%%%%%%%%%%%%%%%%%%%%%%%%%%%%%%%%%%%%%%%%%%%%%%%%%%%%%%%%
\section{Function}
%-----------------------------------------------------------------------------
\subsection{Function MC()}
\index{MC()}%
\begin{verbatim}
(Draw molecule)

msize=(a,b)        **default (1,1)
mposition=(c,d)    **default (0.5,0.5)

a: ratio molecular width/font width
b: ratio molecular hight/font hight
c: x axis position
d: y axis position

beginfont()
  MC(<30,Ph,3:/F,4:/Cl)
endfont
\end{verbatim}
\MCFgraph
%-----------------------------------------------------------------------------
\subsection{Function MCat()}
\index{MCat()}%
\begin{verbatim}
(Draw molecule at mposition)

MCat(c,d)(....) :

mposition:=(c,d); MC(....)

c: x axis position
d: y axis position

defaultsize:=5bp;
fsize:=(60mm,40mm); fmargin:=(3mm,3mm);
blength:=0.07; sw_frame:=Outside;
mangle:=0;
for i=1 step -0.5 until 0:
  for j=0 step 0.33 until 1:
    MCat(j,i)(Ph,4:N)
    add(drawarrow((A1+A1up**aw)..A1);
        label(decimal(mangle),
              p0+(0.5w,0.5h));
    )
    mangle:=mangle+30;
  endfor
endfor

\end{verbatim}
\MCFgraph
%-----------------------------------------------------------------------------
\subsection{Function check()}
\index{check()}%
\begin{verbatim}
(immediately compile)

beginfont("EN:Pyridine")
  MC(<30,Ph,2:N)
endfont

(check mcf and compile)

** check(mc) : error count

beginfont("EN:Pyridine",
  ":<30,Ph,}2:N") % ** extra '}'
  if check(mc)=0: MC(scantokens(mc)) fi
endfont

\end{verbatim}
\MCFgraph\hspace{22mm}\MCFgraph
\begin{verbatim}
check(mc)=0   check(mc)>=1
\end{verbatim}
%===============================================================================
\newpage
\subsection{Function add()}
\index{add()}%
\index{plus}%
\index{minus}%
\index{lonepair}%
\index{lonepairdiam}%
\index{lonepairspace}%
\index{circlediam}%
\index{circlepen}%
\index{w}%
\index{h}%
\index{aw}%
\index{em}%
\index{p0}%
\index{l}%
\index{/*}%
\index{**}%
\index{\textgreater\textgreater}%
\index{An}%
\index{A[]}%
\index{A[]ang}%
\index{A[]up}%
\index{A[]left}%
\index{A[]right}%
\index{A[]down}%
\index{Bn}%
\index{B[]}%
\index{B[]s}%
\index{B[]m}%
\index{B[]e}%
\index{B[]ang}%
\index{B[]up}%
\index{B[]left}%
\index{B[]right}%
\index{B[]down}%
\index{defaultscale}%
\index{labeloffset}%
\begin{verbatim}
(Add label to molecule)

w:       molecular width
h:       molecular height
aw:      atom font size
em:      label font size
p0:      origin of molecular structure
l:       bond length

An:      atom number
A[m]:    atom position
A[m]ang: branch angle of A[m]
A[m]up:    dir A[m]ang
A[m]left:  dir A[m]ang+90
A[m]right: dir A[m]ang-90
A[m]down:  dir A[m]ang+180

Bn:      bond number
B[m]:    bond(path)
B[m]s:   bond start position
B[m]m:   bond middle position
B[m]e:   bond end position
B[m]ang: bond angle
B[m]up:    dir B[m]ang
B[m]left:  dir B[m]ang+90
B[m]right: dir B[m]ang-90
B[m]down:  dir B[m]ang+180

plus  : '+' circled
minus : '-' circled
  circlediam = 0.6aw (default)
  circlepen =  0.2bp (default)

lonepair r: ':' rotated r
  lonepairdiam  = 0.3aw (default)
  lonepairspace = 0.7aw (default)

** : scaled
<< : rotated
a /* b : point b of a

beginfont("EN:add() 1")
 fsize:=(70mm,40mm);
 sw_frame:=Bothside;
 max_blength:=10mm;
 msize:=(.91,.9);
 MCat(.5,.85)(<30,?6,{2,5}:O)
 add(
  defaultscale:=.8;
  labeloffset:=.3aw;
  dotlabel.lft("p0",p0);
  dotlabel.rt( "p0+(w,h)",p0+(w,h));
  dotlabel.ulft("A1",A1);
  drawarrow A1..A1+__*l<<A1ang;
  dotlabel.lrt( "B3s",B3s);
  dotlabel.rt("B3m",B3m);
  drawarrow B3m..B3m+__*l<<(B3ang+90);
  dotlabel.ulft("A6",A6);
  drawarrow A1{A1down}..A6;
  dotlabel.urt( "B3e",B3e);
  label.rt(  "An="&decimal(An)&
           "  Bn="&decimal(Bn)&
           "  aw="&decimal(aw)&
           "  em="&decimal(em),
           p0+(-9em,-1.5em));
  label.rt(  "w="&decimal(w)&
           "  h="&decimal(h)&
           "  l="&decimal(l),
           p0+(-9em,-3em));
 )
endfont
\end{verbatim}
\MCFgraph
\begin{verbatim}
beginfont("EN:add() 2")
 fsize:=(60mm,20mm);
 msize:=(1,0.85);
 %---------------------------------------
 MCat(0,0)(<30,Ph,3=dl,4:/NH2)
 %---------------------------------------
 add(
  labeloffset:=.7aw;
  label.top(lone_pair 90,A7);
  drawarrow 
    (A7+up**1.2aw){A7left}
     ..{B7right}B7/*0.3;
  drawarrow
    B3m..A3+B2up**1.5aw..{A3down}A3;
 )
 %---------------------------------------
 MCat(1,0)(<30,?6,{1,5}=dl,4://NH2)
 %---------------------------------------
 add(
  labeloffset:=.7aw;
  label.top(plus,A7);
  label.urt(minus,A3);
  label(lonepair A3ang,A3+A3up**.7aw);
 )
 %---------------------------------------
 ext(drawdblarrow (.4w,.4h)..(.55w,.4h);)
 %---------------------------------------
endfont
\end{verbatim}
\MCFgraph
%-----------------------------------------------------------------------------
\newpage
\subsection{Function ext()}
\index{ext()}%
\index{w0}%
\index{h0}%
\index{aw}%
\index{em}%
\index{n}%
\index{ratio\_thickness\_char}%
\index{defaultscale}%
\begin{verbatim}
(Extra label to font)
 
w:    font width
h:    font height
w0:   font width-2xpart(fmargin)
h0:   font height-2ypart(fmargin)
aw:   atom font size
em:   label font size
p0:   fmargin

n:    molecular number
p[m]: molecular origin position
w[m]: molecular width
h[m]: molecular height

ratio_thickness_char:
pen thickness / char width
%----------------------------------------
beginfont()
 fsize:=(70mm,30mm;);
 blength:=0.065;
 %---------------------------------------
 MCat(0.1,0.5)(
   <-210,60`1,60`1,60`1,{1,3}=dl,
   1:/R1,4:/R2^-60
   )
   add(
     defaultscale:=0.6;
     label.bot("Diene",p0+(0.5w,0));
   )
 MCat(0.4,0.5)(
   <-30,-60`1,1=dl,1:/R3,2:/R4^60)
   add(defaultscale:=0.6;
   label.bot("Dienophile",p0+(.5w,0));
  )
 MCat(0.9,0.5)(
   <30,?6,6=dl,2:/R2,3:/R4,4:/R3,5:/R1
 )
 %---------------------------------------
 ext(
  drawarrow (.52w,.5h)..(.6w,.5h);
  defaultscale:=0.7;
  label("+",(0.25w,0.5h));
  ratio_thickness_char:=0.125;
  label.bot("Diels-Alder Reaction",
            (.5w,h));
 )
 %---------------------------------------
endfont
\end{verbatim}
\MCFgraph
%-----------------------------------------------------------------------------
\subsubsection{Local ext() setting}
\begin{verbatim}
beginfont("EN:?3")
  fsize:=(12mm,15mm);
  MCat(0.5,1)(<30,?3)
endfont
beginfont("EN:?4")
  fsize:=(12mm,15mm);
  MCat(0.5,1)(?4)
  %-------------------------------
  ext(label.top(inf_EN,(0.5w,0));)
  %-------------------------------
endfont
beginfont("EN:?5")
  fsize:=(12mm,15mm);
  MCat(0.5,1)(?5)
endfont
beginfont("EN:?5")
  fsize:=(12mm,15mm);
  MCat(0.5,1)(?6)
endfont
\end{verbatim}
\MCFgraph\MCFgraph\MCFgraph\MCFgraph
\subsubsection{Global ext() setting}
\index{ext\_clear}%
\begin{verbatim}
ext_clear: reset global ext()

beginfont("EN:?3")
  fsize:=(12mm,15mm);
  MCat(0.5,1)(<30,?3)
endfont
%-------------------------------
ext(label.top(inf_EN,(0.5w,0));)
%-------------------------------
beginfont("EN:?4")
  fsize:=(12mm,15mm);
  MCat(0.5,1)(?4)
endfont
beginfont("EN:?5")
  fsize:=(12mm,15mm);
  MCat(0.5,1)(?5)
endfont
%---------
ext_clear;
%---------
beginfont("EN:?6")
  fsize:=(12mm,15mm);
  MCat(0.5,1)(?6)
endfont
\end{verbatim}
\MCFgraph\MCFgraph\MCFgraph\MCFgraph
%-----------------------------------------------------------------------------
\newpage
%-----------------------------------------------------------------------------
\section{MCF example}
%-----------------------------------------------------------------------------
\subsection{Luciferin}
\index{check()}%
\begin{verbatim}
(use data base file 'mcf_data_base')
beginfont("f:mcf_data_base",
          "t:EN","v:Luciferin")
  fsize:=(50mm,15mm);
  if check(mc)=0: MC(scantokens(mc)) fi
endfont
\end{verbatim}
\MCFgraph
%-----------------------------------------------------------------------------
\subsection{Colchicine}
\begin{verbatim}
beginfont("EN:Colchicine","MW:385.41",
  %-------------------------------------
  ": <30,Ph,{1,2,6}:/O!,-4=?7,-5=?7,  ",
  ": {-1,-4,-6}=dl,-2://O,-3:/O!,     ",
  ": @9,\,NH,!,//O,!                  ")
  %-------------------------------------
  fsize:=(50mm,20mm);
  if check(mc)=0: MC(scantokens(mc)) fi
endfont
\end{verbatim}
\MCFgraph
%-----------------------------------------------------------------------------
\subsection{Paclitaxel}
\begin{verbatim}
beginfont("EN:Paclitaxel","MW:853.91",
  %------------------------------------
  ": ?6,5=dl,@3,#1,36,45,45,45,45,##, ",
  ": &5",                             ",
  ": -4=?6,-4=?4,-1=wb,-3=wf,-1:O,||, ",
  ": 4:??,6:/_,{3^-60,15}:*/OH,       ",
  ": 8:/*H^-60,",                     ",
  ": 9:*/_^60,10://O,                 ",
  ": @1,\,O,!,//O,!,*/OH,!,/Ph,60~wf, ",
  ":   NH,-60,//O,60,Ph,              ",
  ": @7,\*,O,-45,//O,60,Ph,           ",
  ": @11,*\,O,-60,//O,60,             ",
  ": @12,\*^-15,O,60,//O,-60          ")
  %------------------------------------
  fsize:=(50mm,25mm);
  if check(mc)=0: MC(scantokens(mc)) fi
endfont
\end{verbatim}
\MCFgraph
%-----------------------------------------------------------------------------
\subsection{Maltose}
\index{arc\_lb}
\index{arc\_br}
\begin{verbatim}
(bond type for glycan)
arc_lb : arc left > bottom
arc_br : arc bottom right

beginfont("EN:Maltose","MW:342.3",
  %-----------------------------------------
  ": #1.25,-30~wf_r,30~bd_r`1,30~wb_r,    ",
  ":  120,O,30,&1,##,                     ",
  ": {1^$-90,2^$90,3^$-90}:/OH`-.5,       ",
  ": 6^$90:/!OH`-.5,                      ",
  ": @4,$-50~arc_lb`1,O,$50~arc_br`1,<$0, ",
  ": |,#1.25,-30~wf_r,30~bd_r`1,30~wb_r,  ",
  ":    120,O,30,&1,##,                   ",
  ": {2^$90,3^$-90,4^$-90}:/OH`-.5,       ",
  ": 6^$90:/!OH`-.5                       ")
  %-----------------------------------------
  fsize:=(50mm,20mm);
  if check(mc)=0: MC(scantokens(mc)) fi
endfont
\end{verbatim}
\MCFgraph
%-----------------------------------------------------------------------------
\subsection{Cellobiose}
\index{arc\_lbr}
\index{arc\_ltr}
\begin{verbatim}
(bond type for glycan)
arc_lbr : arc left > bottom > right
arc_ltr : arc left > top > right

beginfont("EN:Cellobiose","MW:342.3",
  %-----------------------------------------
  ": #1.25,-30~wf_r,30~bd_r`1,30~wb_r,    ",
  ":  120,O,30,&1,##,                     ",
  ": {1^$-90,2^$90,3^$-90}:/OH`-.5,       ",
  ": 6^$90:/!OH`-.5,                      ",
  ": @4,$0~arc_ltr,O,$0~arc_lbr,          ",
  ": |,#1.25,-30~wf_r,30~bd_r`1,30~wb_r,  ",
  ":    120,O,30,&1,##,                   ",
  ": {2^$90,3^$-90,4^$-90}:/OH`-.5,       ",
  ": 6^$90:/!OH`-.5                       ")
  %-----------------------------------------
  fsize:=(50mm,20mm);
  if check(mc)=0: MC(scantokens(mc)) fi
endfont
\end{verbatim}
\MCFgraph
%------------------------------------------------------------------------
\onecolumn
\section{Example to use mcf2graph}
\subsection{Metafont/Metapost souce file}
\index{mcf2graph.mf}%
\index{sw\_output}%
\index{tag}%
\index{var}%
\begin{verbatim}
%-------------------------------------------------------------------------
input mcf2graph.mf;                                     > input main macro
%-------------------------------------------------------------------------
sw_output:=Info;      % aux(information) file output on > global setting
%%%% sw_output:=Report;                                 > report output
%%%% sw_output:=MOL2k;                                  > MOL file output
fsize:=(60mm,40mm);   %  (font width,font height)       >
tag1:="J";                                              > jobname
tag2:="C";                                              > char No
tag3:="mw";           % calculated molecular weight     > 
tag4:="fm";           % calculated molecular formula    >
outputformat:="png"; hppp:=vppp:=0.1;                   > PNG output
outputtemplate:="%j-%3c.png";                           >
%-------------------------------------------------------------------------
beginfont("EN:Ampicillin","MW:349.405")                 > information
  MC(<45,?4,-3=?5,2:N,7:S,                              > immediately compile
    3^45:/*H,1://O^15,5:/*COOH^-18,6:??,                >
    @4,*\^15,NH,!,//O,!,/*NH2,!,Ph)                     >
endfont                                                 >
%------------------------------------------------------------------------
beginfont("EN:Cholesterol","MW:386.65",                 >information
  %----------------------------------------             >
  ": <30,?6,{-4,-2}=?6,-4=?5,7=dl,      ",              > mc1
  ": 10:/*H^180,11:/*H^-60,17:/*H^-54,  ",              > mc2
  ": {4,12}:*/_^60,                     ",              > mc3
  ": @-1,18,/*_,-60,!3,?!               ")              > mc4
  %----------------------------------------             >
  if check(mc)=0: MC(scantokens(mc)) fi                 > mc=mc1 - mc4
endfont                                                 >
%------------------------------------------------------------------------------
beginfont("f:mcf_data_base.mcf","t:EN","v:Adenine")     > from mcf_data_base.mcf
  if check(mc)=0: MC(scantokens(mc)) fi                 > select EN="Adenine"
endfont                                                 >
%------------------------------------------------------------------------------
beginfont("t:EN","v:Guanine")                           > select EN="Guanine"
  if check(mc)=0: MC(scantokens(mc)) fi                 
endfont
%------------------------------------------------------------------------------
beginfont("t:EN","v:Cytosine")                          > select EN="Cytosine"
  if check(mc)=0: MC(scantokens(mc)) fi                 >
endfont                                                 >
%------------------------------------------------------------------------------
beginfont("t:n","v+:4")                                 > v+:4 = select No.4
  if check(mc)=0: MC(scantokens(mc)) fi                 >        keep file open
endfont                                                 >
%------------------------------------------------------------------------------
forever:
%%%%%%%%%%  beginfont("f:mcf_data_base","v+:*")          > select all
  beginfont("f:mcf_data_base","t:EXA","v+:1")            > 'v+:1'= select EXA=1
    if f_EOF=0: if check(mc)=0: MC(scantokens(mc)) fi fi >       keep file open
  endfont                                                >    
  exitif (f_EOF=1)or(f_close=1);                         > exit if file end
endfor
%------------------------------------------------------------------------------
bye
\end{verbatim}
%------------------------------------------------------------------------
\noindent%
\newpage
\subsection{Molecular data base file}
\begin{verbatim}
%%%%%%%%%%%%%%%%%%%%%%%%%%%%%%%%%%%%%%%%%%%%%%%%%%%%%%%%%%%%%%%%%%%%%%%%%%%%%%%
% molecular data base file mcf_data_base.mcf  by Akira Yamaji  2021.04.18
%%%%%%%%%%%%%%%%%%%%%%%%%%%%%%%%%%%%%%%%%%%%%%%%%%%%%%%%%%%%%%%%%%%%%%%%%%%%%%%
%  tag1:var1;tag2:var2;tag3:var3 ....
%  first character of line '%' comment out
%  first character of line '+' begin MCF , end MCF
%------------------------------------------------------------------------------
Cat:biological;EN:Adenine;MW:135.13
+
<30,?6,3=?5,{1,3,5,9}=dl,{2,6,9}:N,5:/NH2,7:NH
+------------------------------------------------------------------------------
Cat:biological;EN:Guanine;MW:151.13
+
<30,?6,3=?5,{1,3,9}=dl,{2,9}:N,{6,7}:NH,5://O,1:/NH2
+------------------------------------------------------------------------------
Cat:biological;EN:Cytosine;MW:111.10
+
<30,?6,{4,6}=dl,4:N,3://O,2:NH,5:/NH2
+------------------------------------------------------------------------------
Cat:biological;EN:Thymine;MW:126.11
+
<30,?6,3=dl,{2,6}:NH,{1,5}://O,4:/_
+------------------------------------------------------------------------------
Cat:biological;EN:Uracil;MW:112.09
+
<30,?6,6=dl,{3,5}://O,{2,4}:NH
+------------------------------------------------------------------------------
Cat:biological;EN:Geraniol;MW:154.25
+
<30,!8,OH,{2,6}=dr,{2,6}:/_
+------------------------------------------------------------------------------
Cat:biological;EN:Limonene;MW:136.24
+
<30,?6,2=dl,2:/_,@5,*\,/_,!!
+------------------------------------------------------------------------------
Cat:biological;EN:l-Menthol;MW:156.27
+
<30,?6,2:/*?!,5:*/_,3:*/OH
+------------------------------------------------------------------------------
Cat:biological;EN:Vanillin;MW:152.15
+
<30,Ph,2:/OH,3:/O!,5:/CHO
+------------------------------------------------------------------------------
Cat:biological;EN:Allicin;MW:162.28
+
<-30,!!,!2,SO,!,S,!2,!!
+------------------------------------------------------------------------------
Cat:biological;EN:Stearic acid;MW:284.48
+
<30,!17,COOH
+------------------------------------------------------------------------------
Cat:biological;EN:Linoleic acid;MW:280.45
+
<30,!5,-30,-30,!,-30,-30,!7,COOH,{6,9}=dr
+------------------------------------------------------------------------------
\end{verbatim}
%------------------------------------------------------------------------------
\noindent%
\newpage
\subsection{Function query()}
\index{query()}%
\paragraph{(Example)}
\begin{verbatim}
%--------------------------------------------------------------
% query()
%
% "f:filename" : input file name  (default "mcf_data_base.mcf")
% "o:filename" : output file name (default "temp.mcf")
% "s:sort-key" : sort by sort-key 
%
% operator  :  = , <> , <= , >= , < , >
%
% filter 1  : Cat=biological
% filter 2  : MW>=285
% filter 3  : MW<=295
%--------------------------------------------------------------
query("s:EN",
%%%%%    "f:mcf_data_base.mcf","o:temp.mcf","s:EN",
      "Cat=biological","MW>=285","MW<=295");
%--------------------------------------------------------------
forever:
  beginfont("f:temp","v+:*")    % use file temp.mcf / select all
    if f_EOF=0: if check(mc)=0: MC(scantokens(mc)) fi fi
  endfont
  exitif f_EOF=1;
endfor
%---------------------------------------------------------------
\end{verbatim}
%---------------------------------------------------------------
\paragraph{(output)}
\begin{verbatim}
Cat:biological;EN:Atoropin;MW:289.375;EXA:1
+
<30,O,!,//O,!,!,Ph,@$1,\~zb^-60,|,?7`1.1,@6,*\^190`1.25,N!,&3~wb,$3:/!OH~wv 
+------------------------------------------------------------------------------
Cat:biological;EN:Cianidanol;MW:290.27;EXA:1
+
<30,Ph,3=?6,@8,*\,Ph,7:O,{1,5,13,14}:/OH,9:/*OH
+------------------------------------------------------------------------------
Cat:biological;EN:Lycorine;MW:287.315;EXA:1
+
<30,Ph,-4=?6,-2=?6,6=?5,(9,12)=?5[3],13=dl,8:N,{15,17}:O,
 9:/*H^180,10:*/H^60,13:*/OH,14:/*OH
+------------------------------------------------------------------------------
Cat:biological;EN:Morphine;MW:285.343;EXA:1
+
<30,Ph,2=?6,-4=?6,(1,12)=?5[2],-1:O,-1=zb,
 @7,60~wf`0.75,70~si_`1.3,45,N!,&9~wb,15=dl,6:/OH,8^180:*/H,12:/*OH
+------------------------------------------------------------------------------
Cat:biological;EN:Piperine;MW:285.343;EXA:1
+
<30,Ph,|,-1=?5,{1,3}:O,@$4,\,!!,!,!!,!,//O,!,|,?6,1:N
+------------------------------------------------------------------------------
\end{verbatim}
%------------------------------------------------------------------------------
\noindent%
\newpage
\subsection{Information aux file output}
\paragraph{(Insert option parameter setting)}
\index{J}%
\index{C}%
\index{NO}%
\index{MW}%
\index{MI}%
\index{EN}%
\index{JN}%
\index{FM}%
\index{USE}%
\index{mw}%
\index{fm}%
\index{mi}%
\index{w}%
\index{h}%
\index{Info}%
\index{Table}%
\index{Temp}%
\begin{verbatim}
  sw_output:=Info;             %% tag1:var1;tag2:var2
  sw_output:=Info+Table;       %% tag1;tag2 var1;var2
  sw_output:=Info+Temp;        %% tag1:var1;tag2:var2 / output 'temp-info.aux'
  sw_output:=Info+Mcode;       %% output jobname&'.aux'
  sw_output:=Info+Mcode+Temp;  %% output 'temp-info.aux','temp-mc.aux'
  sw_output:=Font+Info+Temp;   %% output font,'temp-info.aux','temp-mc.aux'
\end{verbatim}
\paragraph{(Command line)}
\begin{verbatim}
  >mpost -s ahlength=1 FILENAME  (sw_output=Info)
  >mpost -s ahlength=2 FILENAME  (sw_output=Info+Table)
\end{verbatim}
\paragraph{(Sourse)}
\begin{verbatim}
beginfont("EN:Ampicillin")    .... endfont
beginfont("EN:Cholesterol")   .... endfont
beginfont("EN:Limonin")       .... endfont
beginfont("EN:beta-Carotene") .... endfont
\end{verbatim}
\paragraph{(Setting)}
\begin{verbatim}
tag1:="J"; tag2:="C"; tag3:="mw"; tag4:="fm"; tag5:="EN";
\end{verbatim}
\paragraph{(Output)}
\index{aux\_delimiter}%
\begin{verbatim}
(sw_output=Info)
F:mcf_man_soc;C:1;mw:349.40462;fm:C16H19N3O4S;EN:Ampicillin
F:mcf_man_soc;C:2;mw:386.6532;fm:C27H46O;EN:Cholesterol
F:mcf_exa_soc;C:3;mw:470.5113;fm:C26H30O8;EN:Limonin
F:mcf_exa_soc;C:4;mw:536.8722;fm:C40H56;EN:beta-Carotene

(sw_output=Info+Table)
F;C;mw;fm
mcf_man_soc;1;349.40462;C16H19N3O4S;Ampicillin
mcf_man_soc;2;386.6532;C27H46O;Cholesterol
mcf_exa_soc;3;470.5113;C26H30O8;Limonin
mcf_exa_soc;4;536.8722;C40H56;beta-Carotene

(aux_delimiter="/")
F:mcf_man_soc/C:1/mw:349.40462/fm:C16H19N3O4S/EN:Ampicillin
F:mcf_man_soc/C:2/mw:386.6532/fm:C27H46O/EN:Cholesterol
F:mcf_exa_soc/C:3/mw:470.5113/fm:C26H30O8/EN:Limonin
F:mcf_exa_soc/C:4/mw:536.8722/fm:C40H56/EN:beta-Carotene
\end{verbatim}
\paragraph{(Tag)}
\begin{verbatim}
J   : jobname                           
C   : char number
NO  : serial number
EN  : english name
JN  : japanese name
FM  : formula from literature data
MW  : molecular weight from literature data
MI  : monoisotopic mass from literature data
USE : the use
mw  : molecular weight calculated
mi  : monoisotopic mass calculated
fm  : molecular formula calculated
w   : font width
h   : font height
\end{verbatim}
%------------------------------------------------------------------------
\noindent%
\newpage
\subsection{Metafont aux file output}
\index{Mfont}%
\paragraph{(Insert option parameter setting)}
\begin{verbatim}
  sw_output:=Mfont;
\end{verbatim}
\paragraph{(Command line)}
\begin{verbatim}
  >mpost -s ahlength=7 FILENAME  (sw_output=Mfont)
\end{verbatim}
\paragraph{(Output)}
\begin{verbatim}
beginfont("Cat:biological","EN:Adenine","MW:135.13",
": <30,?6,3=?5,{1,3,5,9}=dl,{2,6,9}:N,5:/NH2,7:NH")
if check(mc)=0: MC(scantokens(mc)) fi
endfont
beginfont("Cat:biological","EN:Guanine","MW:151.13",
": <30,?6,3=?5,{1,3,9}=dl,{2,9}:N,{6,7}:NH,5://O,1:/NH2")
if check(mc)=0: MC(scantokens(mc)) fi
endfont
beginfont("Cat:biological","EN:Cytosine","MW:111.10",
": <30,?6,{4,6}=dl,4:N,3://O,2:NH,5:/NH2")
if check(mc)=0: MC(scantokens(mc)) fi
endfont
beginfont("Cat:biological","EN:Thymine","MW:126.11",
": <30,?6,3=dl,{2,6}:NH,{1,5}://O,4:/_")
if check(mc)=0: MC(scantokens(mc)) fi
endfont
beginfont("Cat:biological","EN:Adenine","MW:135.13",
": <30,?6,3=?5,{1,3,5,9}=dl,{2,6,9}:N,5:/NH2,7:NH")
if check(mc)=0: MC(scantokens(mc)) fi
endfont
beginfont("Cat:biological","EN:Guanine","MW:151.13",
": <30,?6,3=?5,{1,3,9}=dl,{2,9}:N,{6,7}:NH,5://O,1:/NH2")
if check(mc)=0: MC(scantokens(mc)) fi
endfont
beginfont("Cat:biological","EN:Cytosine","MW:111.10",
": <30,?6,{4,6}=dl,4:N,3://O,2:NH,5:/NH2")
if check(mc)=0: MC(scantokens(mc)) fi
endfont
beginfont("Cat:biological","EN:Thymine","MW:126.11",
": <30,?6,3=dl,{2,6}:NH,{1,5}://O,4:/_")
if check(mc)=0: MC(scantokens(mc)) fi
endfont
beginfont("Cat:biological","EN:Uracil","MW:112.09",
": <30,?6,6=dl,{3,5}://O,{2,4}:NH")
if check(mc)=0: MC(scantokens(mc)) fi
endfont
beginfont("Cat:biological","EN:Geraniol","MW:154.25",
": <30,!8,OH,{2,6}=dr,{2,6}:/_")
if check(mc)=0: MC(scantokens(mc)) fi
endfont
beginfont("Cat:biological","EN:Limonene","MW:136.24",
": <30,?6,2=dl,2:/_,@5,*\,/_,!!")
if check(mc)=0: MC(scantokens(mc)) fi
endfont
beginfont("Cat:biological","EN:l-Menthol","MW:156.27",
": <30,?6,2:/*?!,5:*/_,3:*/OH")
if check(mc)=0: MC(scantokens(mc)) fi
endfont
\end{verbatim}
%------------------------------------------------------------------------
\noindent%
\newpage
\subsection{MCF aux file output}
\paragraph{(Insert option parameter setting)}
\index{Mcode}%
\index{Temp}%
\begin{verbatim}
sw_output:=Mcode;                  %%  output 'jobname-nnn-EN-mc.aux'
sw_output:=Mcode+Temp;             %%  output 'temp-mc.aux'
sw_output:=Info+Mcode;             %%  output 'jobname-data.aux'
sw_output:=Info+Mcode+Temp;        %%  output 'temp-info.aux','temp-mc.aux'
sw_output:=Font+Mcode+Temp;        %%  output font,'temp-mc.aux'
sw_output:=Font+Info+Mcode+Temp;   %%  output font,'temp-info.aux','temp-mc.aux'
\end{verbatim}
\paragraph{(Command line)}
\begin{verbatim}
  >mpost -s ahlength=8 FILENAME  (sw_output=Info+Mcode)
\end{verbatim}
\paragraph{(Output temporary file)}
\begin{verbatim}
sw_output=Mcode        ** file name = 'jobname-nnn-EN-mc.aux'
sw_output=Mcode+Temp   ** file name = 'temp-mc.aux'

(result)
<30,?6,3=?5,{1,3,5,9}=dl,{2,6,9}:N,5:/NH2,7:NH

\end{verbatim}
\paragraph{(Output data-base file)}
\begin{verbatim}
sw_output=Mcode+Info   ** file name = 'jobname-data.aux'

(result)
Cat:biological;EN:Adenine;MW:135.13;EXA:1
+
<30,?6,3=?5,{1,3,5,9}=dl,{2,6,9}:N,5:/NH2,7:NH
+--------------------------------------------------

\end{verbatim}
%------------------------------------------------------------------------
\paragraph{(Lualatex example)}
%-----------------------------------------------------------------------
\begin{verbatim}
beginfont("t:EN","v:Adenine")
  sw_output:=Mcode+Temp;
endfont
\end{verbatim}
%-----------------------------------------------------------------------
\begin{verbatim}
%-----------------------------------------------------------------------
\begin{mplibcode}
  beginfont("t:EN","v:Vancomycin")
    sw_output:=Mcode+Temp;     %%%% output temp-mc.aux %%%%
  endfont;
\end{mplibcode}
%-----------------------------------------------------------------------
\verbatiminput{temp-mc.aux}
%-----------------------------------------------------------------------
\end{verbatim}
%-----------------------------------------------------------------------
\begin{verbatim}
(result)
file name = 'temp-mc.aux'

<30,?6,@4,?6,@-4,\,!3,<-12,?5,@-3,<-12,?6,-3=?6,@-3,*\,!3,
 ?6,@-4,?6,@6,\,!,/*Me^-40,*/OH^20,!,//O,!1,OH,
 3=wb,11=dl,15=dr,17=wf,19=wf,38=wb,{5,7,16,24,25,33,42}:O,
 32:*/H^60,10:/Me,{12,31}:*/_,27://_,37:/*_,28:/OH,{3,29}:/*OH
\end{verbatim}
%------------------------------------------------------------------------
\newpage
\noindent%
\subsection{Report output}
\paragraph{(Insert option parameter setting)}
\index{sw\_output}%
\index{Report}%
\begin{verbatim}
  sw_output:=Report;        ** file name = 'jobname-report.aux'
  sw_output:=Report+Temp;   ** file name = 'temp-report.aux'
\end{verbatim}
\paragraph{(Command line)}
\begin{verbatim}
  >mpost -s ahlength=3 FILENAME
\end{verbatim}
\paragraph{(Output)}
\begin{verbatim}
===========================================================================
 No.   3 / Name = Cytosine
---------------------------------------------------------------------------
 <30,?6,{4,6}=dl,4:N,3://O,2:NH,5:/NH2
---------------------------------------------------------------------------
 row=  1 / length=  37 / commands=  7
 {}=X =  1 / {}:X =  0 / '() =  0 / @ =  0 / & =  0 / < =  1
---------------------------------------------------------------------------
 Warnings =   0 / Code= 60
 Width * Height =   34.68852 *    47.4036
 Shift width * height  =          0 *  -14.46167
 Bond length = 12.75589   Atom size   = 5.38914
 Atom count=  9 Bond count=  9 Ring count=  1 Hide H count=  2
---------------------------------------------------------------------------
< NO. ><atom(s) >(  x axis   ,   y axis   )<bond><hideH><chg>
 A1     C        (         0 ,          0 )    3     1
 A2     N        (     0.866 ,       -0.5 )    3        
 A3     C        (     1.732 ,          0 )    4        
 A4     N        (     1.732 ,          1 )    3        
 A5     C        (     0.866 ,        1.5 )    4        
 A6     C        (         0 ,          1 )    3     1
 A7     O        (     2.508 ,     -0.448 )    2        
 A8     H        (     0.866 ,     -0.922 )    1        
 A9     NH2      (     0.866 ,      2.371 )    1        
---------------------------------------------------------------------------
< NO. ><  bond   (sdt)><angle +(  +-  )><length (   pt   )>
 B1     1 ->   2 (  1)     330 (   -30)       1 (   12.76)
 B2     2 ->   3 (  1)      30 (    30)       1 (   12.76)
 B3     3 ->   4 (  1)      90 (    90)       1 (   12.76)
 B4     4 ->   5 (  2)     150 (   150)       1 (   12.76)
 B5     5 ->   6 (  1)     210 (  -150)       1 (   12.76)
 B6     6 ->   1 (  2)     270 (   -90)       1 (   12.76)
 B7     3 ->   7 (  2)     330 (   -30)    0.66 (    8.42)
 B8     2 ->   8 (  1)     270 (   -90)    0.36 (    4.59)
 B9     5 ->   9 (  1)      90 (    90)    0.66 (    8.42)
---------------------------------------------------------------------------
<atom>( atom wt )[ mi wt   ]  < cnt > < sum wt   >[ sum mi wt  ]
 C    (  12.0107)[       12] *    4       48.04279[          48]
 H    (  1.00793)[  1.00783] *    5        5.03967[     5.03914]
 N    (  14.0067)[ 14.00307] *    3        42.0201[     42.0092]
 O    (  15.9994)[ 15.99492] *    1        15.9994[    15.99492]
 Molecular Weight [Mono Isotopic] =       111.1019[   111.04326]
---------------------------------------------------------------------------
 Weight  Calc: 111.1019 / Input: 111.10 / weight gap= 0.00195
 Fomula  Calc: C4H5N3O / Input: 
===========================================================================
\end{verbatim}%
\newpage
%------------------------------------------------------------------------
\noindent%
\subsection{MOL file output}
\paragraph{(Insert option parameter setting)}
\index{sw\_output}%
\index{MOL2k}%
\index{MOL3k}%
\begin{verbatim}
  sw_output:=MOL2k;     % MOL(V2000)
  sw_output:=MOL3k;     % MOL(V3000)
\end{verbatim}
\paragraph{(Command line)}
\begin{verbatim}
  >mpost -s ahlength=5  FILENAME     % MOL(V2000)
  >mpost -s ahlength=6  FILENAME     % MOL(V3000)
\end{verbatim}
\paragraph{(Output)}
\begin{verbatim}
%%%%%%%%%%%%%%%%%%%%%%%%%%%%%%%%%%%%%%%%%%%%%%%%%%%%%%%%%%%%%%%%%%%%%
  -MCFtoMOL- EN:Caffeine         

 14 15  0  0  0  0  0  0  0  0999 V2000
         0         0         0 C   0  0  0  0
   0.86603      -0.5         0 N   0  0  0  0
   1.73206         0         0 C   0  0  0  0
   1.73206         1         0 C   0  0  0  0
   0.86603       1.5         0 C   0  0  0  0
         0         1         0 N   0  0  0  0
    2.6831  -0.30902         0 N   0  0  0  0
   3.27089       0.5         0 C   0  0  0  0
    2.6831   1.30902         0 N   0  0  0  0
   0.86603  -1.36383         0 C   0  0  0  0
  -0.76894   1.44394         0 C   0  0  0  0
  -0.76894  -0.44394         0 O   0  0  0  0
   0.86603   2.36383         0 O   0  0  0  0
   2.95299    2.1396         0 C   0  0  0  0
  1  2  1  0     0  0
  2  3  1  0     0  0
  3  4  2  0     0  0
  4  5  1  0     0  0
  5  6  1  0     0  0
  6  1  1  0     0  0
  3  7  1  0     0  0
  7  8  2  0     0  0
  8  9  1  0     0  0
  9  4  1  0     0  0
  2 10  1  0     0  0
  6 11  1  0     0  0
  1 12  2  0     0  0
  5 13  2  0     0  0
  9 14  1  0     0  0
M  END
%%%%%%%%%%%%%%%%%%%%%%%%%%%%%%%%%%%%%%%%%%%%%%%%%%%%%%%%%%%%%%%%%%%%%
\end{verbatim}%
%----------------------------------------------------------------------------
\newpage
\subsection{LuaTeX file example}
\index{Font}%
%############################################################################
\begin{verbatim}
\documentclass{article}
\usepackage{luamplib}%
\usepackage[T1]{fontenc}%
\usepackage{textcomp}%
\mplibcodeinherit{enable}%
\mplibverbatim{enable}%
\mplibnumbersystem{double}%
\everymplib{%
  if unknown Ph1: input mcf2graph.mf; fi
  sw_output:=Font; max_blength:=4.5mm;
  defaultfont:="uhvr8r"; defaultsize:=8bp; defaultscale:=1;
}%
\begin{document}
\noindent%
%--------------------------------------------------------------------
\begin{mplibcode}
  fsize:=(50mm,50mm);
  beginfont("NO:1","EN:Limonin","MW:470.51",
    %----------------------------------------
    ": <30,?6,{-3,-4}=?6,                  ",
    ": -5=?3,-2=wf,-1=wb,6=?5,-4=?6,-5=wf, ",
    ": {13,15,17,20}:O,{3,12,21}://O,      ",
    ": {4~wf^60,8~zf^60,18^35,18^-35}:/_,  ",
    ": {1^60,5^180,16^60}:/*H,             ",
    ": @14,\*,|,?5,{1,4}=dl,3:O            ")
    %----------------------------------------
  if check(mc)=0: MC(scantokens(mc)) fi 
  endfont
\end{mplibcode}\\
%--------------------------------------------------------------------
\begin{mplibcode}
  fsize:=(80mm,50mm);
  beginfont("NO:2","EN:beta-carotene","MW:536.87",
    %------------------------------------------
    ": <30,?6,3=dl,{3,5^35,5^-35}:/_,         ",
    ": @4,\,|,!18,{1,3,5,7,9,11,13,15,17}=dr, ",
    ": {3,7,12,16}:/_,                        ",
    ": |,?6,6=dl,{6,2^35,2^-35}:/_            ")
    %------------------------------------------
  if check(mc)=0: MC(scantokens(mc)) fi 
  endfont
\end{mplibcode}\\
%--------------------------------------------------------------------
\begin{mplibcode}
  fsize:=(50mm,50mm);
  beginfont("NO:3","EN:Gibberellin A3","MW:346.37",
    %------------------------------------
    ": <18,?5,3=?7,5=?6[12],           ",
    ": @8,160`1.3,&3,13=dl,6=wf,8=wb,  ",
    ": @5,40~zf`1,O,60,//O^180,&14~zb, ",
    ": 2:/COOH,7://_,13:*/OH,8:/*OH,   ",
    ": 14:*/_,{1^60,4^60}:*/H          ")
    %------------------------------------
  if check(mc)=0: MC(scantokens(mc)) fi 
endfont;
%--------------------------------------------------------------------
\end{mplibcode}\\
\end{document}
\end{verbatim}%
%############################################################################
%------------------------------------------------------------------------
\newpage
\subsection{LaTeX file example}
%############################################################################
\index{mcf\_setup.sty}%
\begin{verbatim}
%--------------------------------------------------------------------
\documentclass[a4paper]{article}
\usepackage{graphicx}
\makeatletter%
%---------------------
\usepackage{mcf_setup}
%---------------------
\pagestyle{empty}
%--------------------------------------------------------------------
\def\put@char{%
  \begin{picture}(84,42)%
     \put(0,38){\bf [\MOLnum]\EN{ }\small\tt/FM:\fm/MW:\mw}%
     \put(10,0){\font\@strufont=\File\relax%
               \hbox{\@strufont\char\Char}}%
  \end{picture}%
}%
\def\INFO#1{\@for\@temp:=#1\do{\tag@var\@temp}\put@char}%
\makeatother
%--------------------------------------------------------------------
\begin{document}
\unitlength=1mm%
\INFO{J:mcf_man_soc,C:141,NO:1,mw:349.40462,fm:C16H19N3O4S,EN:Ampicillin}%
\INFO{J:mcf_man_soc,C:142,NO:2,mw:386.6532,fm:C27H46O,EN:Cholesterol}%
\end{document}
%--------------------------------------------------------------------
\end{verbatim}%
%############################################################################
%------------------------------------------------------------------------
\INFO{J:mcf_man_soc,C:141,NO:1,mw:349.40462,fm:C16H19N3O4S,EN:Ampicillin}%
\INFO{J:mcf_man_soc,C:142,NO:2,mw:386.6532,fm:C27H46O,EN:Cholesterol}%
%------------------------------------------------------------------------
\texttt{\printindex}
%------------------------------------------------------------------------
\end{document}
