%%%%%%%%%%%%%%%%%%%%%%%%%%%%%%%%%%%%%%%%%%%%%%%%%%%%%%%%%%%%%%%%%%%%%%%%%%%%%%%
% Example of  MCF typest with LuaLaTeX(luamplib)    by A.Yamaji    2021.09.20
%%%%%%%%%%%%%%%%%%%%%%%%%%%%%%%%%%%%%%%%%%%%%%%%%%%%%%%%%%%%%%%%%%%%%%%%%%%%%%%
% ** mcf2graph.mf must be version 4.72
% ** use mcf_data_base.mcf
\documentclass{article}
%------------------------------------------------------------------------------
\usepackage{luamplib}%
\usepackage[T1]{fontenc}%
\usepackage{textcomp,verbatim,mcf_setup}%
\mplibcodeinherit{enable}%
\mplibverbatim{enable}%
\mplibnumbersystem{double}%
\everymplib{%
  if unknown Ph1: input mcf2graph.mf; fi
  outputformat:="eps";
  sw_output:=Font+Info+Temp;
  tag1:="J"; tag2:="C"; tag3:="fm"; tag4:="mw"; tag5:="EN"; tag6:="MW";
  max_blength:=4.5mm;
  defaultfont:="uhvr8r";
  defaultsize:=8bp;
  defaultscale:=1;
}%
%------------------------------------------------------------------------------
\pagestyle{empty}
\topmargin=-20mm
\oddsidemargin=0mm
\textwidth=170mm
\textheight=280mm
\unitlength=1mm
%------------------------------------------------------------------------------------
\begin{document}
\begin{center}
 {\Huge\sf MCF example for luamplib(Lua\LaTeX)}\vspace{5mm}\\
 Author : Akira Yamaji \quad Date : \today\\
 Located at : http://www.ctan.org/pkg/mcf2graph\vspace{2mm}\\
\end{center}
%------------------------------------------------------------------------------------
\section{MCF example}
\noindent%
%------------------------------------------------------------------------------------
 use molecular data base file 'mcf\_exa\_data.mcf'\\
{{\tt FM(C) :} molecular formula calculated by mcf2graph\\
{{\tt MW(C) :} molecular weight calculated by mcf2graph\\
{{\tt MW(D) :} molecular weight from literature data
%%%%%%%%%%%%%%%%%%%%%%%%%%%%%%%%%%%%%%%%%%%%%%%%%%%%%%%%%%%%%%%%%%%%%%%%%%%%%
\subsection{Chlorophyll a}
\noindent%
\begin{verbatim}
beginfont("f:mcf_data_base.mcf","t:EN","v:Chlorophyll a","NO:-",
          ":,||,{4,11,17,23}:red,27:green")   %%%% change color of atoms %%%%
  fsize:=(100mm,45mm);
  if check(mc)=0: MC(scantokens(mc)) fi
  ext(defaultscale:=0.8;
      label.lrt("FM(C): "&cal_FM,(0,h-5mm));
      label.lrt("MW(C): "&cal_MW,(0,h-9mm));
      label.lrt("MW(D): "&inf_MW,(0,h-13mm));)
endfont
\end{verbatim}
%------------------------------------------------------------------------------------
\begin{mplibcode}
beginfont("f:mcf_data_base.mcf","t:EN","v:Chlorophyll a","NO:-",
          ":,||,{4,11,17,23}:red,27:green")   %%%% change color of atoms %%%%
  fsize:=(100mm,45mm);
  if check(mc)=0: MC(scantokens(mc)) fi
  ext(defaultscale:=0.8;
      label.lrt("FM(C): "&cal_FM,(0,h-5mm));
      label.lrt("MW(C): "&cal_MW,(0,h-9mm));
      label.lrt("MW(D): "&inf_MW,(0,h-13mm));)
endfont
\end{mplibcode}
%%%%%%%%%%%%%%%%%%%%%%%%%%%%%%%%%%%%%%%%%%%%%%%%%%%%%%%%%%%%%%%%%%%%%%%%%%%%%%%%%%%%%
\subsection{Dinophysistoxin-1}
\noindent%
\begin{verbatim}
beginfont("t:EN","v:Okadaic acid","EN:Dinophysistoxin-1",
          "MW:819",":,38:*/_,65=red")  %%%% add methyl group (color red) %%%%
  fsize:=(150mm,35mm);
  if check(mc)=0: MC(scantokens(mc)) fi
  ext(defaultscale:=0.8;
      label.lrt("FM(C): "&cal_FM,(0,h-5mm));
      label.lrt("MW(C): "&cal_MW,(0,h-9mm));
      label.lrt("MW(D): "&inf_MW,(0,h-13mm));)
endfont;
\end{verbatim}
%----------------------------------------------------------------------------
\begin{mplibcode}
beginfont("t:EN","v:Okadaic acid","EN:Dinophysistoxin-1",
          "MW:819",":,38:*/_,65=red")  %%%% add methyl group (color red) %%%%
  fsize:=(150mm,35mm);
  if check(mc)=0: MC(scantokens(mc)) fi
  ext(defaultscale:=0.8;
      label.lrt("FM(C): "&cal_FM,(0,h-5mm));
      label.lrt("MW(C): "&cal_MW,(0,h-9mm));
      label.lrt("MW(D): "&inf_MW,(0,h-13mm));)
endfont;
\end{mplibcode}
%%%%%%%%%%%%%%%%%%%%%%%%%%%%%%%%%%%%%%%%%%%%%%%%%%%%%%%%%%%%%%%%%%%%%%%%%%%%%
\newpage
\subsection{Erythromycin}
\noindent%
%----------------------------------------------------------------------------
\begin{verbatim}
beginfont("EN:Erythromycin","MW:733.93",
  %--------------------------------------------------------------------
  ": <30,#1,<-120,60,60,60,-60,60,60,-60,60,60,60,-60,60,60,##,&1,    ",
  ": 14:O,13:/*Et,{1,9}://O,{2,10}:*/_,{4,6^-35,8,12^35}:/*_,         ",
  ": {6^35,11,12^-35}:*/OH,                                           ",
  ": @$3,\*,O,30,|,?6`.7,2:O,{3,5^35}:/_`.5,4:/OH`.-.5,5^-35`-.5:/O!, ",
  ": @$5,\*^30`1.7,O,!,|,?6`.7,6:O,5:/_`.5,2:/OH`-.5,3:/N?!`-.5       ")
  %--------------------------------------------------------------------
  fsize:=(120mm,30mm);
  if check(mc)=0: MC(scantokens(mc)) fi
endfont;
\end{verbatim}
%----------------------------------------------------------------------------
\begin{mplibcode}
beginfont("EN:Erythromycin","MW:733.93",
  %--------------------------------------------------------------------
  ": <30,#1,<-120,60,60,60,-60,60,60,-60,60,60,60,-60,60,60,##,&1,    ",
  ": 14:O,13:/*Et,{1,9}://O,{2,10}:*/_,{4,6^-35,8,12^35}:/*_,         ",
  ": {6^35,11,12^-35}:*/OH,                                           ",
  ": @$3,\*,O,30,|,?6`.7,2:O,{3,5^35}:/_`.5,4:/OH`.-.5,5^-35`-.5:/O!, ",
  ": @$5,\*^30`1.7,O,!,|,?6`.7,6:O,5:/_`.5,2:/OH`-.5,3:/N?!`-.5       ")
  %--------------------------------------------------------------------
  fsize:=(120mm,30mm);
  if check(mc)=0: MC(scantokens(mc)) fi
  ext(defaultscale:=0.8;
      label.lrt("FM(C): "&cal_FM,(0,h-5mm));
      label.lrt("MW(C): "&cal_MW,(0,h-9mm));
      label.lrt("MW(D): "&inf_MW,(0,h-13mm));)
endfont;
\end{mplibcode}
%%%%%%%%%%%%%%%%%%%%%%%%%%%%%%%%%%%%%%%%%%%%%%%%%%%%%%%%%%%%%%%%%%%%%%%%%%%%%
\subsection{Paclitaxel}
\noindent%
\begin{verbatim}
beginfont("EN:Paclitaxel","MW:853.918",
  %---------------------------------------------------------------------
  ": ?6,5=dl,@3,#1,36,45,45,45,45,##,&5,-4=?6,-4=?4,-1=wb,-3=wf,-1:O, ",
  ": 4:??,6:/_,{3^-60,15}:*/OH,8:/*H^-60,9:*/_^60,10://O,             ",
  ": @1,\,O,!,//O,!,*/OH,!,/Ph,60~wf,NH,-60,//O,60,Ph,                ",
  ": @7,\*,O,-45,//O,60,Ph,11:*/OCO!>rl,12:/*OCO!^-15>lr              ")
  %---------------------------------------------------------------------
  fsize:=(120mm,30mm);
  if check(mc)=0: MC(scantokens(mc)) fi
endfont
\end{verbatim}
%----------------------------------------------------------------------------
\begin{mplibcode}
beginfont("EN:Paclitaxel","MW:853.918",
  %---------------------------------------------------------------------
  ": ?6,5=dl,@3,#1,36,45,45,45,45,##,&5,-4=?6,-4=?4,-1=wb,-3=wf,-1:O, ",
  ": 4:??,6:/_,{3^-60,15}:*/OH,8:/*H^-60,9:*/_^60,10://O,             ",
  ": @1,\,O,!,//O,!,*/OH,!,/Ph,60~wf,NH,-60,//O,60,Ph,                ",
  ": @7,\*,O,-45,//O,60,Ph,11:*/OCO!>rl,12:/*OCO!^-15>lr              ")
  %---------------------------------------------------------------------
  fsize:=(120mm,30mm);
  if check(mc)=0: MC(scantokens(mc)) fi
  ext(defaultscale:=0.8;
      label.lrt("FM(C): "&cal_FM,(0,h-5mm));
      label.lrt("MW(C): "&cal_MW,(0,h-9mm));
      label.lrt("MW(D): "&inf_MW,(0,h-13mm));)
endfont
\end{mplibcode}
%----------------------------------------------------------------------------
\subsection{Kekulene}
\noindent%
\begin{verbatim}
beginfont("EN:Kekulene","MW:600.7",
  %-----------------------------------------------------------------------
  ": <30,Ph,{3,-2,-3,-2,-3,-2,-3,-2,-3,-2}=?6,(-3,6)=?6[3],@-4,&6,      ",
  ": {8,12,14,16,18,22,24,26,28,32,34,36,38,42,44,46,48,52,54,56,58}=dl ")
  %-----------------------------------------------------------------------
  fsize:=(120mm,25mm);
  if check(mc)=0: MC(scantokens(mc)) fi
endfont
\end{verbatim}
%----------------------------------------------------------------------------
\begin{mplibcode}
beginfont("EN:Kekulene","MW:600.7",
  ": <30,Ph,{3,-2,-3,-2,-3,-2,-3,-2,-3,-2}=?6,(-3,6)=?6[3],@-4,&6,      ",
  ": {8,12,14,16,18,22,24,26,28,32,34,36,38,42,44,46,48,52,54,56,58}=dl ")
  fsize:=(120mm,25mm);
  if check(mc)=0: MC(scantokens(mc)) fi
  ext(defaultscale:=0.8;
      label.lrt("FM(C): "&cal_FM,(0,h-5mm));
      label.lrt("MW(C): "&cal_MW,(0,h-9mm));
      label.lrt("MW(D): "&inf_MW,(0,h-13mm));)
endfont
\end{mplibcode}
%%%%%%%%%%%%%%%%%%%%%%%%%%%%%%%%%%%%%%%%%%%%%%%%%%%%%%%%%%%%%%%%%%%%%%%%%%%%%
\newpage
%----------------------------------------------------------------------------
\subsection{Maitotoxin}
\noindent%
%--------------------------------------------------------------------------------
\begin{verbatim}
%--------------------------------------------------------------------------------
\begin{mplibcode}
  beginfont("t:EN","v:Maitotoxin")
    sw_output:=Info+Mcode+Temp;           %%%% output temp-info,temp-mc.aux %%%%
  endfont;
\end{mplibcode}
\verbatiminput{temp-mc.aux}                  %%%% input temp-mc.aux %%%%
\begin{mplibcode}
  beginfont("t:EN","v:Maitotoxin")
    fsize:=(120mm,60mm); fmargin:=(0,3mm); sw_frame:=Outside;
    if check(mc)=0: MC(scantokens(mc)) fi    %%%% output font %%%%
  endfont
\end{mplibcode}
%--------------------------------------------------------------------------------
\newread\auxfile%
\openin\auxfile=temp-info.aux                %%%% input temp-info.aux %%%%
\read\auxfile to \info%
\infotovar{\info}                            %%%% info to variables %%%%
\closein\auxfile%
{\tt ** EN:\EN \quad MW(C):\MW \quad MW(D):\mw \quad FM(C):\fm}%
%--------------------------------------------------------------------------------
\end{verbatim}
%--------------------------------------------------------------------------------
\begin{mplibcode}
  beginfont("t:EN","v:Maitotoxin")
    sw_output:=Info+Mcode+Temp;                       %%%% output temp-info.aux %%%%
    if check(mc)=0: MC(scantokens(mc)) fi
  endfont
\end{mplibcode}
\begin{mplibcode}
  beginfont("t:EN","v:Maitotoxin")
    fsize:=(120mm,60mm); fmargin:=(0,3mm); sw_frame:=Outside;
    if check(mc)=0: MC(scantokens(mc)) fi    %%%% output font %%%%
  endfont
\end{mplibcode}
%--------------------------------------------------------------------------------
\newread\auxfile%
\openin\auxfile=temp-info.aux                %%%% input temp-info.aux %%%%
\read\auxfile to \info%
\infotovar{\info}                            %%%% info to variables %%%%
\closein\auxfile%
\verbatiminput{temp-mc.aux}                  %%%% input temp-mc.aux %%%%
{\tt ** EN:\EN \quad MW(C):\MW \quad MW(D):\mw \quad FM(C):\fm}%
%--------------------------------------------------------------------------------
%%%%%%%%%%%%%%%%%%%%%%%%%%%%%%%%%%%%%%%%%%%%%%%%%%%%%%%%%%%%%%%%%%%%%%%%%%%%%
\newpage
\subsection{TCA cycle}
\noindent%
\begin{mplibcode}
beginfont("EN:TCA cycle")
fsize:=(160mm,75mm);
max_blength:=5mm;
%--------------------------------------------------------------------------------
COOH:='(//O,!,OH);
HOCO:='(OH,!,//O,);
MCat(0.33,   1)(<30,HOCO,!,//O,!2,COOH)                    % Oxaloacetate
MCat(0.66,   1)(<30,HOCO,!4,COOH,@-4`1,\,COOH,4:/OH^-165)  % Citrate
MCat(1,      1)(<30,HOCO,!2,!~dr,!,COOH,@-4`1,\,COOH)      % cis-Aconitate
MCat(1,   0.58)(<30,HOCO,!4,COOH,@-4,\`1,COOH,5:/OH)       % Isocitrate
MCat(1,   0.05)(<30,HOCO,!3,//O,!,COOH,@-4,\`1,COOH)       % Oxalosuccinate
MCat(0.66,0.05)(<30,HOCO,!3,//O,!,COOH)                    % alfa-Ketoglutarate
MCat(0.33,0.05)(<30,HOCO,!3,//O,!,"{S-CoA}")               % Succinyl-CoA
MCat(0,   0.05)(<30,HOCO,!3,COOH)                          % Succinate
MCat(0,   0.55)(<30,HOCO,!,!~dr,!,COOH)                    % Fumarate
MCat(0,      1)(<30,HOCO,!3,COOH,3:/OH)                    % L-Malate
%--------------------------------------------------------------------------------
ext(
defaultfont:="uhvr8r";
defaultscale:=0.75;
ext_setup;
def sel_dir(expr rot)=
  save tx; nA:=rot; if nA>=360: nA:=nA-360; fi
  if     (nA<30)or(nA>330):   def tx=top enddef; elseif (nA>=30)and(nA<=150):  def tx=lft enddef;
  elseif (nA>150)and(nA<210): def tx=bot enddef; elseif (nA>=210)and(nA<=330): def tx=rt enddef; fi
enddef;
def reaction_arrow(expr a)(expr r)(expr p)(expr sa,la)(expr sb,lb)(expr sc,lc)(expr sd,ld)=
  drawarrow ((0,0)..(a,0)) rotated r shifted p;
  if sa<>"": draw ((0.5a,0){dir 180}..{dir 90}(0,.5a*la)) rotated r shifted p;
             sel_dir(r); label.tx(sa,p+((0,0.5a*la) rotated r)); fi
  if sb<>"": draw ((0.5a,0){dir 0}..{dir 90}(a,.5a*lb)) rotated r shifted p;
             sel_dir(r); label.tx(sb,p+((a,0.5a*lb) rotated r)); fi
  if sc<>"": draw ((0.5a,0){dir 180}..{dir -90}(0,-.5a*lc)) rotated r shifted p;
             sel_dir(r+180); label.tx(sc,p+((0,-0.5a*lc) rotated r)); fi
  if sd<>"": draw ((0.5a,0){dir 0}..{dir -90}(a,-.5a*ld)) rotated r shifted p;
             sel_dir(r+180); label.tx(sd,p+((a,-0.5a*ld) rotated r)); fi
enddef;
def r_arrow(expr a)(expr r)(expr p)(expr sc,lc)(expr sd,ld)=
   reaction_arrow(a)(r)(p)("",0)("",0)(sc,lc)(sd,ld)
enddef;
%------------------------------------------------------------------------
save dx; pair dx; dx:=(12mm,0);
label.bot("Oxaloacetate",p1+dx);   label.bot("Citrate",p2+dx);
label.bot("cis-Aconitate",p3+dx);  label.bot("Isocitrate",p4+dx);
label.bot("Oxalosuccinate",p5+dx); label.bot("alfa-Ketoglutarate",p6+dx);
label.bot("Succinyl-CoA",p7+dx);   label.bot("Succinate",p8+dx);
label.bot("Fumarate",p9+dx);       label.bot("L-Malate",p10+dx);
sw_label_emu:=1;
ext_setup;
r_arrow(10mm)(  0)(p1+ ( 1.1w1, 0.3h1))("Acetyl-CoA",1.5)(" CoA-SH",1);
r_arrow(10mm)(  0)(p2+ ( 1.1w2, 0.4h2))("",0)("H2O",1);
r_arrow( 8mm)(270)(p3+ ( 0.5w3,-0.4h3))("H2O",1)("",0);
r_arrow( 8mm)(270)(p4+ ( 0.5w4,-0.4h4))("NAD+",1)("NADH2+",1);
r_arrow(10mm)(180)(p5+ (-0.1w5, 0.4h5))("",0)("CO2",1);
r_arrow(10mm)(180)(p6+ (-0.1w6, 0.5h6))("NAD+,CoA-SH",1.7)("NADH2+,CO2",1);
r_arrow(10mm)(180)(p7+ (-0.1w7, 0.5h7))("GDP,Pi",1.7)("GTP,CoA-SH",1);
r_arrow( 8mm)( 90)(p8+ ( 0.4w8, 1.2h8))("FAD",1)("FADH2",1);
r_arrow( 8mm)( 90)(p9+ ( 0.4w9, 1.2h9))("H2O",1)("",0);
r_arrow(10mm)(  0)(p10+( 1.1w10,0.3h10))("NAD+",1)("NADH2+",1.5);
defaultscale:=1.5;
label("TCA-cycle",(0.5w,0.5h));
)
endfont
\end{mplibcode}
%------------------------------------------------------------------------
\begin{verbatim}
beginfont("EN:TCA cycle")
fsize:=(160mm,75mm);
max_blength:=5mm;
COOH:='(//O,!,OH);
HOCO:='(OH,!,//O,);
MCat(0.33,   1)(<30,HOCO,!,//O,!2,COOH)                    % Oxaloacetate
MCat(0.66,   1)(<30,HOCO,!4,COOH,@-4`1,\,COOH,4:/OH^-165)  % Citrate
MCat(1,      1)(<30,HOCO,!2,!~dr,!,COOH,@-4`1,\,COOH)      % cis-Aconitate
MCat(1,   0.58)(<30,HOCO,!4,COOH,@-4,\`1,COOH,5:/OH)       % Isocitrate
MCat(1,   0.05)(<30,HOCO,!3,//O,!,COOH,@-4,\`1,COOH)       % Oxalosuccinate
MCat(0.66,0.05)(<30,HOCO,!3,//O,!,COOH)                    % alfa-Ketoglutarate
MCat(0.33,0.05)(<30,HOCO,!3,//O,!,"{S-CoA}")               % Succinyl-CoA
MCat(0,   0.05)(<30,HOCO,!3,COOH)                          % Succinate
MCat(0,   0.55)(<30,HOCO,!,!~dr,!,COOH)                    % Fumarate
MCat(0,      1)(<30,HOCO,!3,COOH,3:/OH)                    % L-Malate
ext(
defaultfont:="uhvr8r";
defaultscale:=0.75;
ext_setup;
save dx; pair dx; dx:=(12mm,0);
label.bot("Oxaloacetate",p1+dx);   label.bot("Citrate",p2+dx);
label.bot("cis-Aconitate",p3+dx);  label.bot("Isocitrate",p4+dx);
label.bot("Oxalosuccinate",p5+dx); label.bot("alfa-Ketoglutarate",p6+dx);
label.bot("Succinyl-CoA",p7+dx);   label.bot("Succinate",p8+dx);
label.bot("Fumarate",p9+dx);       label.bot("L-Malate",p10+dx);
sw_label_emu:=1;
ext_setup;
r_arrow(10mm)(  0)(p1+ ( 1.1w1, 0.3h1))("Acetyl-CoA",1.5)(" CoA-SH",1);
r_arrow(10mm)(  0)(p2+ ( 1.1w2, 0.4h2))("",0)("H2O",1);
r_arrow( 8mm)(270)(p3+ ( 0.5w3,-0.4h3))("H2O",1)("",0);
r_arrow( 8mm)(270)(p4+ ( 0.5w4,-0.4h4))("NAD+",1)("NADH2+",1);
r_arrow(10mm)(180)(p5+ (-0.1w5, 0.4h5))("",0)("CO_2_",1);
r_arrow(10mm)(180)(p6+ (-0.1w6, 0.5h6))("NAD+,CoA-SH",1.7)("NADH2+,CO2",1);
r_arrow(10mm)(180)(p7+ (-0.1w7, 0.5h7))("GDP,Pi",1.7)("GTP,CoA-SH",1);
r_arrow( 8mm)( 90)(p8+ ( 0.4w8, 1.2h8))("FAD",1)("FADH2",1);
r_arrow( 8mm)( 90)(p9+ ( 0.4w9, 1.2h9))("H2O",1)("",0);
r_arrow(10mm)(  0)(p10+( 1.1w10,0.3h10))("NAD+",1)("NADH2+",1.5);
defaultscale:=1.5;
label("TCA-cycle",(0.5w,0.5h));
)
endfont
\end{verbatim}
%----------------------------------------------------------------------------
\end{document}
