% change texinfo fonts
% usage: put the following code early in your file,
% but after \input texinfo

%%%%%%%%%%%%%%%%
% @tex
% \input pstexi
% @end tex
%
% @textfonts
% @rm
%%%%%%%%%%%%%%%%

% Set the font macro #1 to the font named #2, adding on the
% specified font prefix (normally `cm').
% #3 is the font's design size, #4 is a scale factor
% \def\setfont#1#2#3#4{\font#1=\fontprefix#2#3 scaled #4}
% UPDATE: a new implementation uses a fifth encoding(?) parameter,
% which we simply ignore.
\newcount\atsize
\def\setfont#1#2#3#4#5{%
  \def\mag{#4}
  \global\atsize=\mag
  \global\multiply\atsize #3
  \global\divide\atsize 1000
  \global\font#1=#2 at \atsize pt}

% \fontprefix not usable outside cm/lm/ec
\gdef\fontprefix{}

\gdef\rmshape{pplr7t}
\gdef\rmbshape{pplb7t}               %where the normal face is bold
\gdef\bfshape{pplb7t}
\gdef\bxshape{pplb7t}
\gdef\ttshape{texnansi-lmtt10}
\gdef\ttbshape{texnansi-lmtk10}
\gdef\ttslshape{texnansi-lmtto10}
\gdef\itshape{pplri7t}
\gdef\itbshape{pplbi7t}
\gdef\slshape{pplri7t}
\gdef\slbshape{pplbi7t}
\gdef\sfshape{texnansi-lmss10}
\gdef\sfbshape{texnansi-lmssbx10}
\gdef\scshape{pplrc7t}
\gdef\scbshape{pplbc7t}

\definetextfontsizexi

\global\let\subtitlerm=\tenrm
