\documentclass{article}

\usepackage{polyglossia}
\setdefaultlanguage{russian}

\usepackage{fontspec}
\usepackage{natbib}

\setmainfont[Mapping=tex-text]{CMU Serif}

\let\K=\textit
\let\J=\textbf
\def\NN#1 {\textit{#1 }}
\def\NN#1 #2 {\textit{#1 #2 }}

\begin{document}

\section*{Подорожники: жизненная форма}

Подорожники~--- преимущественно розеточные растения, то есть
ос\-нов\-ная особенность их побеговой системы~--- укороченные
междоузлия вегетативных побегов. Даже у растений, относящихся к
роду \K{Psyllium}, в пазухах листьев развиваются боковые
укороченные вегетативные и удлиненные генеративные побеги. У
растений из рода \K{Plantago} стебель с удлиненными междоузлиями
и очередным (в отличие от \K{Psyllium}) листорасположением
формируется достаточно редко: у некоторых тихоокеанских
древовидных видов секции \K{Palaeopsyllium}, у средиземноморских
\NN P. lagopus L. и \NN P. amplexicaulis Cav., а также у
ближневосточных и американских представителей секций
\K{Albicans} Barn. и \K{Gnaphaloides} Barn. На нашей территории
эти виды не встречаются. У большинства растений из рода
\K{Plantago} ось первого порядка длительное время растет
моноподиально, образуя (при помощи контрактильной деятельности
боковых корней~--- \cite{pl:59}) неспециализированное
эпигеогенное корневище, в состав которого у многих подорожников
входит верхняя часть главного корня, гипокотиль и нижние
междоузлия побега, а генеративные почки закладываются в пазухах
отмерших прошлогодних листьев или самых первых листьев текущего
года \cite{pl:30}. В наших условиях розеточные побеги
большинства многолетних видов зимуют с зелеными листьями.

Виды подорожника различаются степенью ветвления вегетативных
побегов. Так, у \NN P. major ветвление корневища наблюдается
крайне редко, поскольку, как правило, вслед за этим происходит
распадение растения на отдельные партикулы \cite{pl:18}.  Мы
наблюдали растения \NN P. uliginosa с ветвящимся корневищем в
сборах с территории Ленинградской области и Республики Коми
(LEU), причем исключительно с окультуренных местообитаний.
\NN P.  media и в особенности \NN P.  lanceolata ветвятся
значительно сильнее, а \NN P. maritima subsp. \K{maritima} и
subsp.  \K{subpolaris} (Andreev) Tzvel. (в отличие от \NN P.
schrenkii C. Koch) часто образует громадные (площадью до 2
м$^2$) клоны, возникшие, по-видимому, за счет ветвления и
последующего распада одной особи. Однолетние подорожники (\K{P.
tenuiflora, P. minuta}) практически не ветвятся.

Анатомические признаки строения древесины у подорожников с
одревесневающими стеблями довольно однообразны
\cite{pl:197,pl:124} и могут использоваться, по-видимому, лишь
для уточнения положения всего семейства в системе цветковых
растений.

У представителей рода \K{Psyllium} листорасположение
перекрестнопарное. Другие виды подорожников, по нашим данным,
также могут отличаться признаками расположения листьев на
побеге, в частности, числом развитых зеленых листьев и формулой
листорасположения (2/5 у \K{P. major}; 1/3 у \K{P.
media}\footnote{Иногда также 3/8 \cite{pl:254}.}, \K{P.
maritima} и \K{P. lanceolata}).

Отмирающие в течение сезона нижние листья у степных многолетних
видов (например, у \NN P. maxima Juss. ex Jacq. и у
тетраплоидных растений \K{P. media}) не перегнивают полностью, а
образуют своими основаниями своеобразную <<муфту>>, которая, по
нашим наблюдениям в Оренбургской области, может препятствовать
серьезному повреждению точки роста во время степных пожаров.

Вегетативное размножение не отличается среди подорожников
большим разнообразием. Как правило, оно происходит путем
партикуляции старых особей \cite{pl:17}. В этой связи ст\'оит
отметить только \NN P. lanceolata, способный к образованию
придаточных почек на корнях \cite{pl:241} и \K{Littorella
uniflora}, которая, как и некоторые другие водные растения,
образует ползучие укореняющиеся боковые побеги-<<усы>> --- без
сомнения, хорошая адаптация к водному образу жизни. В некоторых
местообитаниях (например, на оз.  Высокинское Карельского
перешейка), мы наблюдали исключительно вегетативное размножение
\K{Littorella}.

\bibliography{rubibtex-ex}
\bibliographystyle{ugost2008ns}

\end{document}