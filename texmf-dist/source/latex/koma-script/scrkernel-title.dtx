% \iffalse meta-comment
% ======================================================================
% scrkernel-title.dtx
% Copyright (c) Markus Kohm, 2002-2022
%
% This file is part of the LaTeX2e KOMA-Script bundle.
%
% This work may be distributed and/or modified under the conditions of
% the LaTeX Project Public License, version 1.3c of the license.
% The latest version of this license is in
%   http://www.latex-project.org/lppl.txt
% and version 1.3c or later is part of all distributions of LaTeX 
% version 2005/12/01 or later and of this work.
%
% This work has the LPPL maintenance status "author-maintained".
%
% The Current Maintainer and author of this work is Markus Kohm.
%
% This work consists of all files listed in MANIFEST.md.
% ======================================================================
%%% From File: $Id: scrkernel-title.dtx 3874 2022-06-05 10:38:02Z kohm $
%<option>%%%            (run: option)
%<body>%%%            (run: body)
%<*dtx>
\ifx\ProvidesFile\undefined\def\ProvidesFile#1[#2]{}\fi
\begingroup
  \def\filedate$#1: #2-#3-#4 #5${\gdef\filedate{#2/#3/#4}}
  \filedate$Date: 2022-06-05 12:38:02 +0200 (So, 05. Jun 2022) $
  \def\filerevision$#1: #2 ${\gdef\filerevision{r#2}}
  \filerevision$Revision: 3874 $
  \edef\reserved@a{%
    \noexpand\endgroup
    \noexpand\ProvidesFile{scrkernel-title.dtx}%
                          [\filedate\space\filerevision\space
                           KOMA-Script source (title pages)]
  }%
\reserved@a
\documentclass[USenglish]{koma-script-source-doc}
\usepackage{babel}
\setcounter{StandardModuleDepth}{3}
\begin{document}
\DocInput{scrkernel-title.dtx}
\end{document}
%</dtx>
% \fi
%
% \changes{v2.95}{2002/06/26}{first version after splitting \file{scrclass.dtx}}
% \changes{v3.36}{2022/03/08}{switch over from \cls*{scrdoc} to
%   \cls*{koma-script-source-doc}}
% \changes{v3.36}{2022/03/08}{whole implementation documentation in English}
%
% \GetFileInfo{scrkernel-title.dtx}
% \title{The Code to Implement the Extended Titles of
%   \href{https://komascript.de}{\KOMAScript} Classes}
% \author{\href{mailto:komascript@gmx.info}{Markus Kohm}}
% \date{Revision \fileversion{} of \filedate}
% \maketitle
% \begin{abstract}
%   \file{scrkernel-title.dtx} implements on-page titles and title pages, that
%   are extended a lot comparing with the titles of the standard classes.
% \end{abstract}
% \tableofcontents
%
% \section{User Manual}
%
% You can find the user documentation the commands implemented here in the
% \KOMAScript{} manual, either the German \file{scrguide} or the English
% \file{scrguien}.
%
% \MaybeStop{\PrintIndex}
%
%
% \section{Implementation of Titles}
%
% This kind of titles are not needed for letters.
%
%    \begin{macrocode}
%<*!letter>
%    \end{macrocode}
% 
% \begin{option}{titlepage}
% \changes{v2.95c}{2006/08/21}{using \pkg*{scrkbase}}
% \changes{v3.12}{2013/11/21}{value \opt{\quotechar=firstiscover} added}
% \changes{v3.17}{2015/03/10}{using internal value storage}
% \changes{v3.28}{2019/11/18}{\cs{ifstr} renamed to \cs{Ifstr}}
% \begin{macro}{\@titlepagetrue,\@titlepagefalse,\if@titlepage}
% \begin{macro}{\@titlepageiscoverpagetrue,\@titlepageisvoerpagefalse,
%               \if@titlepageiscoverpage}
% \changes{v3.12}{2013/11/21}{added}
% There is support for two kinds of titles: in-page titles also known as title
% heads and title pages. Usually title pages are the title pages of the inner
% book. But some users prefer to interpret them as cover instead. With
% \KOMAScript{} this does in fact makes some sense, but only, if the user uses
% the mock title and wants to interpret it not as mock title but as cover. So
% we provide the following values:
% \begin{description}
% \item[\opt{=false}:] use a in-page title (\cs{@titlepagefalse},
%   \cs{@titlepageiscoverpagefalse}),
% \item[\opt{=true}:] use several title pages (\cs{@titlepagetrue},
%   \cs{@titlepageiscoverpagefalse}),
% \item[\opt{=firstiscover}:] use several title pages and the first one is the
%   cover (\cs{@titlepagetrue}, \cs{@titlepageiscoverpagetrue}). We do not
%   test, if this is was originally a mock title of a main title.
% \end{description}
% With package \pkg*{scrextend} this feature is defined as activatable. And
% the initial values differ from class to class.
%    \begin{macrocode}
%<*option>
%<*class>
\newif\if@titlepage
%<article>\@titlepagefalse
%<report|book>\@titlepagetrue
%</class>
%<*package&extend>
\scr@ext@activateable{title}{%
  \scr@ifundefinedorrelax{if@titlepage}{%
    \expandafter\newif\csname if@titlepage\endcsname
    \@titlepagefalse
  }{}%
  \scr@ifundefinedorrelax{if@titlepageiscoverpage}{%
    \expandafter\newif\csname if@titlepageiscoverpage\endcsname
    \@titlepageiscoverpagefalse
  }{}%
%</package&extend>
%<*class|package>
%<class>\newif\if@titlepageiscoverpage
\KOMA@key{titlepage}[true]{%
%<!(package&extend)>  \Ifstr{#1}{firstiscover}{%
%<package&extend>  \Ifstr{##1}{firstiscover}{%
    \@titlepagetrue
    \@titlepageiscoverpagetrue
    \FamilyKeyStateProcessed
    \KOMA@kav@replacevalue{.%
%<class>      \KOMAClassFileName
%<package&extend>      scrextend.\scr@pkgextension
    }{titlepage}{firstiscover}%
  }{%
    \def\FamilyElseValue{, `firstiscover'}%
%<!(package&extend)>     \KOMA@set@ifkey{titlepage}{@titlepage}{#1}%
%<package&extend>     \KOMA@set@ifkey{titlepage}{@titlepage}{##1}%
    \ifx\FamilyKeyState\FamilyKeyStateProcessed
      \KOMA@kav@remove{.%
%<class>      \KOMAClassFileName
%<package&extend>      scrextend.\scr@pkgextension
      }{titlepage}{firstiscover}%
      \KOMA@kav@replacebool{.%
%<class>      \KOMAClassFileName
%<package&extend>      scrextend.\scr@pkgextension
      }{titlepage}{@titlepage}%
      \@titlepageiscoverpagefalse
    \fi
  }%
}
\KOMA@kav@xadd{.%
%<class>  \KOMAClassFileName
%<package&extend>  scrextend.\scr@pkgextension
}{titlepage}{\if@titlepage true\else false\fi}
%<*extend>
%    \end{macrocode}
% For \pkg*{scrextend} we allow to activate the feature at load time.
%    \begin{macrocode}
  \def\scr@ext@immediate@title{%
    \scr@ext@activate{title}%
    \let\scr@ext@immediate@title\relax
  }%
}
%</extend>
%</class|package>
%    \end{macrocode}
% \begin{option}{notitlepage}
% \changes{v2.95c}{2006/08/21}{obsolete}
% \changes{v3.01a}{2008/11/21}{standard not obsolete}
%    \begin{macrocode}
\KOMA@DeclareStandardOption%
%<package>  [scrextend]%
  {notitlepage}{titlepage=false}
%</option>
%    \end{macrocode}
% \end{option}^^A notitlepage
% \end{macro}^^A \if@titlepageiscoverpage
% \end{macro}^^A \if@titlepage
% \end{option}^^A titlepage
%
% \begin{command}{\extratitle}
% \begin{macro}{\@extratitle}
% \begin{command}{\frontispiece}
% \changes{v3.25}{2017/11/21}{added}
% \begin{macro}{\@frontispiece}
% \changes{v3.25}{2017/11/21}{added}
% \begin{command}{\titlehead}
% \begin{macro}{\@titlehead}
% \begin{command}{\subject}
% \begin{macro}{\@subject}
% \begin{command}{\subtitle}
% \changes{v2.97c}{2007/06/20}{neue Möglichkeit}
% \begin{macro}{\@subtitle}
% \changes{v2.97c}{2007/06/20}{added}
% \begin{command}{\publishers}
% \begin{macro}{\@publishers}
% \begin{command}{\uppertitleback}
% \begin{macro}{\@uppertitleback}
% \begin{command}{\lowertitleback}
% \begin{macro}{\@lowertitleback}
% \begin{command}{\dedication}
% \begin{macro}{\@dedication}
% The \LaTeX{} kernel already provides user commands \cs{title}, \cs{author},
% \cs{date}, which store their arguments globally into macros \cs{@title},
% \cs{@author} and \cs{@date}. But for \KOMAScript{} these are not enough for
% several pages of title and even not for the main title. So several more
% similar commands and macros are defined. All additional user commands but
% \cs{subject} are long.
% \begin{description}
% \item[ToDo:] Make them all robust as \LaTeX{} from 2018/10/01 does. However
%   this could be incompatible with \pkg{uni-titlepage}, so when doing the
%   change, don't forget to also change \pkg{uni-titlepage} if needed.
% \end{description}
% For package \pkg*{scrextend} all the code is added to the activatable
% \texttt{title}.
%    \begin{macrocode}
%<*body>
%<package&extend>\scr@ext@addto@activateable{title}{%
\newcommand*{\@extratitle}{}%
\newcommand{\extratitle}[1]{\gdef\@extratitle{%
%<extend>    ##1%
%<!extend>    #1%
}}%
\newcommand*{\@frontispiece}{}%
\newcommand{\frontispiece}[1]{\gdef\@frontispiece{%
%<extend>    ##1%
%<!extend>    #1%
}}%
\newcommand*{\@titlehead}{}%
\newcommand{\titlehead}[1]{\gdef\@titlehead{%
%<extend>    ##1%
%<!extend>    #1%
}}%
\newcommand*{\@subject}{}%
\newcommand*{\subject}[1]{\gdef\@subject{%
%<extend>    ##1%
%<!extend>    #1%
}}%
\newcommand*{\subtitle}[1]{\gdef\@subtitle{%
%<extend>    ##1%
%<!extend>    #1%
}}%
\newcommand*{\@subtitle}{}%
\newcommand*{\@publishers}{}%
\newcommand{\publishers}[1]{\gdef\@publishers{%
%<extend>    ##1%
%<!extend>    #1%
}}%
\newcommand*{\@uppertitleback}{}%
\newcommand{\uppertitleback}[1]{\gdef\@uppertitleback{%
%<extend>    ##1%
%<!extend>    #1%
}}%
\newcommand*{\@lowertitleback}{}%
\newcommand{\lowertitleback}[1]{\gdef\@lowertitleback{%
%<extend>    ##1%
%<!extend>    #1%
}}%
\newcommand*{\@dedication}{}%
\newcommand{\dedication}[1]{\gdef\@dedication{%
%<extend>    ##1%
%<!extend>    #1%
}}%
%    \end{macrocode}
% \end{macro}
% \end{command}
% \end{macro}
% \end{command}
% \end{macro}
% \end{command}
% \end{macro}
% \end{command}
% \end{macro}
% \end{command}
% \end{macro}
% \end{command}
% \end{macro}
% \end{command}
% \end{macro}
% \end{command}
% \end{macro}
% \end{command}
%
% \begin{macro}{\next@tpage}
% \changes{v2.3b}{1995/07/24}{\cs{null} removed}
% \changes{v2.3g}{1996/01/14}{\cs{newpage} replaced by \cs{clearpage}}
% \changes{v3.11c}{2013/02/14}{output \cs{@thanks}}
% \changes{v3.11c}{2013/02/14}{initialize \cs{@thanks}}
% \changes{v3.20}{2015/10/14}{\cs{setparsizes} resp. similar code added}
% \begin{macro}{\next@tdpage}
% \changes{v3.11c}{2013/02/14}{added}
% \changes{v3.12}{2013/11/21}{\cs{titlepage@restore} added}
% Because we have more than one title page but all title pages inside one
% \env{titlepage} environment, we need a special kind of
% \cs{newpage}/\cs{clearpage}.
% \begin{description}
% \item[ToDo:] Replace the original \cs{clearpage} and \cs{cleardoublepage}
%   inside \env{titlepage} by these two macros. With this, users could also
%   have more than one page inside a \env{titlepage} environment.
% \end{description}
%    \begin{macrocode}
\newcommand*{\next@tpage}{%
  \@thanks\global\let\@thanks\@empty
  \clearpage
  \csname titlepage@restore\endcsname
%<!extend>      \setparsizes{\z@}{\z@}{\z@\@plus 1fil}\par@updaterelative
%<extend>      \parskip\z@ \parindent\z@ \parfillskip\z@\@plus 1fil
  \thispagestyle{empty}%
  \let\footnote\thanks
  \setcounter{footnote}{0}%
}
\newcommand*{\next@tdpage}{%
  \next@tpage\if@twoside \ifodd\value{page}\else\null\next@tpage\fi\fi
}
%    \end{macrocode}
% \end{macro}^^A \next@tdpage
% \end{macro}^^A \next@tpage
%
% \begin{command}{\maketitle}
% \changes{v2.3a}{1995/07/08}{\texttt{plus} replaced by \cs{@plus}}
% \changes{v2.3a}{1995/07/08}{\texttt{fill} replaced by \cs{fill}}
% \changes{v2.3a}{1995/07/08}{\cs{vfil} replaced by \cs{vfill}}
% \changes{v2.3a}{1995/07/08}{using \cs{@makefnmark} in \cs{@makefntext}}
% \changes{v2.3a}{1995/07/08}{\cs{@makefnmark} and \cs{@makefntext}
%   independent from math mode}
% \changes{v2.3b}{1995/07/24}{page break inside \cs{@extratitle} in
%   double-sided mode fixed}
% \changes{v2.3d}{1995/08/19}{\cs{fill} replaced by \texttt{fill}}
% \changes{v2.3e}{1995/08/30}{support but ignore optional argument in
%   single-sided mode}
% \changes{v2.3g}{1996/01/14}{several \cs{null} to \cs{next@tpage} added}
% \changes{v2.3g}{1996/01/14}{\cs{footnote} in titles fixed}
% \changes{v2.3g}{1996/01/14}{deleting \cs{@date}, \cs{title} etc. to save memory}
% \changes{v2.3g}{1996/01/14}{\cs{gdef} replaced by \cs{global}\cs{let}}
% \changes{v2.4}{1996/02/25}{\cs{footnote} in titles fixed again}
% \changes{v2.4h}{1996/11/09}{always use \cs{sectfont} followed by the side
%   for the \cs{@title}}
% \changes{v2.4l}{1997/02/06}{using footnote symbols with width 0}
% \changes{v2.4l}{1997/02/06}{redefining \cs{\@makefnmark} instead of
%   \cs{@makefntext} only}
% \changes{v2.8d}{2001/07/05}{using \cs{titlepagestyle} for in-page title}
% \changes{v2.8p}{2001/09/22}{\cs{sectfont} replaced by \cs{titlefont}}
% \changes{v2.95c}{2006/08/21}{setting paragraph base values}
% \changes{v2.97c}{2007/08/31}{interleaf pages with page style
%   \pstyle{emmpty}}
% \changes{v3.04a}{2009/07/14}{distinguishing in-page titles and title pages
%   inside \cs{maketitle} instead of loading time}
% \changes{v3.11c}{2013/02/14}{mock title moved to new macro}
% \changes{v3.11c}{2013/02/14}{using \cs{next@tdpage}}
% \changes{v3.12}{2013/05/29}{using font elements \fnt{titlehead},
%   \fnt{author}, \fnt{date}, \fnt{publishers}, \fnt{dedication}}
% \changes{v3.12}{2013/11/21}{support for
%   \opt{titlepage\quotechar=firstiscover}}
% \changes{v3.12}{2013/11/21}{retro page only if needed}
% \changes{v3.12}{2013/11/21}{kill the title macros only in compatibility mode}
% \changes{v3.12a}{2014/03/03}{orphan spaced fixed}
% \changes{v3.12a}{2014/03/03}{output extra/mock title in one-side mode only
%   if not empty}
% \changes{v3.12a}{2014/03/03}{\cs{next@tdpage} removed from \cs{@maketitle}
%   and \cs{@makeextrattitle}}
% \changes{v3.13}{2013/10/31}{\cs{next@tdpage} replaced by
%   \cs{cleardoubleemptypage} after dedication (last title page)}
% \changes{v3.19}{2015/08/25}{don't use empty \cs{titlepagestyle}}
% \changes{v3.25}{2017/11/21}{support for \cs{@frontispiece}}
% \changes{v3.27}{2019/01/01}{warning for not empty retro page in single-sided
%   mode}
% \changes{v3.27}{2019/03/18}{empty \cs{@thanks} after usage}
% \changes{v3.27}{2019/03/18}{deleting \cs{@thanks} always globally}
% The user command that prints either the in-page title or the title pages
% with all the title macros filled before.
%    \begin{macrocode}
%<package&extend>\let\maketitle\relax\let\@maketitle\relax
\newcommand*\maketitle[1][1]{%
%    \end{macrocode}
% To support several \cs{maketitle} in the same document, we have to take care
% to use the correct \cs{and} every time. So we define it here (instead of
% outside).
%    \begin{macrocode}
  \expandafter\ifnum \csname scr@v@3.12\endcsname>\scr@compatibility\relax
  \else
    \def\and{%
      \end{tabular}%
      \hskip 1em \@plus.17fil%
      \begin{tabular}[t]{c}%
    }%
  \fi
%    \end{macrocode}
% Depending on in-page titles or title pages, we have to use different
% code. First the title pages.
%    \begin{macrocode}
  \if@titlepage
    \begin{titlepage}
      \setcounter{page}{%
%<extend>        ##1%
%<!extend>        #1%
      }%
%    \end{macrocode}
% From \KOMAScript~3.12 we need special treatment for the case this is the
% first title page and the first one should be the cover. The changes here has
% to be restored later. So we locally define \cs{titlepage@restore} to restore
% the original parameters.
%    \begin{macrocode}
      \if@titlepageiscoverpage
        \edef\titlepage@restore{%
          \noexpand\endgroup
          \noexpand\global\noexpand\@colht\the\@colht
          \noexpand\global\noexpand\@colroom\the\@colroom
          \noexpand\global\vsize\the\vsize
          \noexpand\global\noexpand\@titlepageiscoverpagefalse
          \noexpand\let\noexpand\titlepage@restore\noexpand\relax
        }%
        \begingroup
        \topmargin=\dimexpr \coverpagetopmargin-1in\relax
        \oddsidemargin=\dimexpr \coverpageleftmargin-1in\relax
        \evensidemargin=\dimexpr \coverpageleftmargin-1in\relax
        \textwidth=\dimexpr
        \paperwidth-\coverpageleftmargin-\coverpagerightmargin\relax
        \textheight=\dimexpr
        \paperheight-\coverpagetopmargin-\coverpagebottommargin\relax
        \headheight=0pt
        \headsep=0pt
        \footskip=\baselineskip
        \@colht=\textheight
        \@colroom=\textheight
        \vsize=\textheight
        \columnwidth=\textwidth
        \hsize=\columnwidth
        \linewidth=\hsize
      \else
        \let\titlepage@restore\relax
      \fi
      \let\footnotesize\small
      \let\footnoterule\relax
      \let\footnote\thanks
      \renewcommand*\thefootnote{\@fnsymbol\c@footnote}%
      \let\@oldmakefnmark\@makefnmark
      \renewcommand*{\@makefnmark}{\rlap\@oldmakefnmark}%
      \ifx\@extratitle\@empty
        \ifx\@frontispiece\@empty
        \else
          \if@twoside\mbox{}\next@tpage\fi
          \noindent\@frontispiece\next@tdpage
        \fi
      \else
        \noindent\@extratitle
        \ifx\@frontispiece\@empty
        \else
          \next@tpage
          \noindent\@frontispiece
        \fi
        \next@tdpage  
      \fi
%<!extend>      \setparsizes{\z@}{\z@}{\z@\@plus 1fil}\par@updaterelative
%<extend>      \parskip\z@ \parindent\z@ \parfillskip\z@\@plus 1fil
      \ifx\@titlehead\@empty \else
        \begin{minipage}[t]{\textwidth}%
          \usekomafont{titlehead}{\@titlehead\par}%
        \end{minipage}\par
      \fi
      \null\vfill
      \begin{center}
        \ifx\@subject\@empty \else
          {\usekomafont{subject}{\@subject\par}}%
          \vskip 3em
        \fi
        {\usekomafont{title}{\huge \@title\par}}%
        \vskip 1em
        {\ifx\@subtitle\@empty\else\usekomafont{subtitle}{\@subtitle\par}\fi}%
        \vskip 2em
        {%
          \usekomafont{author}{%
            \lineskip 0.75em
            \begin{tabular}[t]{c}
              \@author
            \end{tabular}\par
          }%
        }%
        \vskip 1.5em
        {\usekomafont{date}{\@date \par}}%
        \vskip \z@ \@plus3fill
        {\usekomafont{publishers}{\@publishers \par}}%
        \vskip 3em
      \end{center}\par
      \@thanks\global\let\@thanks\@empty
      \vfill\null
      \if@twoside
%    \end{macrocode}
% From \KOMAScript~3.12 we show the retro page only if needed. It is needed in
% double-sided mode, if the upper and lower elements are not empty.
%    \begin{macrocode}
        \@tempswatrue
        \expandafter\ifnum \@nameuse{scr@v@3.12}>\scr@compatibility\relax
        \else
          \ifx\@uppertitleback\@empty\ifx\@lowertitleback\@empty
            \@tempswafalse
          \fi\fi
        \fi
        \if@tempswa
          \next@tpage
          \begin{minipage}[t]{\textwidth}
            \@uppertitleback
          \end{minipage}\par
          \vfill
          \begin{minipage}[b]{\textwidth}
            \@lowertitleback
          \end{minipage}\par
          \@thanks\global\let\@thanks\@empty
        \fi
%    \end{macrocode}
% In single-sided mode, we warn for used but never printed upper and lower
% retro page variables.
%    \begin{macrocode}
      \else
        \ifx\@uppertitleback\@empty\else
%<package>          \PackageWarning{scrextend}{%
%<class>          \ClassWarning{\KOMAClassName}{%
            non empty \string\uppertitleback\space ignored
            by \string\maketitle\MessageBreak
            in `twoside=false' mode%
          }%
        \fi
        \ifx\@lowertitleback\@empty\else
%<package>          \PackageWarning{scrextend}{%
%<class>          \ClassWarning{\KOMAClassName}{%
            non empty \string\lowertitleback\space ignored
            by \string\maketitle\MessageBreak
            in `twoside=false' mode%
          }%
        \fi          
      \fi
      \ifx\@dedication\@empty 
      \else
        \next@tdpage\null\vfill
        {\centering\usekomafont{dedication}{\@dedication \par}}%
        \vskip \z@ \@plus3fill
%    \end{macrocode}
% \begin{description}
% \item[ToDo:] Why can't we use \cs{next@tdpage} here? Would this be a
%   problem, if we redefine \cs{cleardoublepage} to \cs{next@tdpage}?
% \end{description}
%    \begin{macrocode}
        \@thanks\global\let\@thanks\@empty
        \cleardoubleemptypage
      \fi
%    \end{macrocode}
% If we are still at a cover page, we have to restore the title page.
%    \begin{macrocode}
      \ifx\titlepage@restore\relax\else\clearpage\titlepage@restore\fi
    \end{titlepage}
  \else
%    \end{macrocode}
% Now, the in-page title.
%    \begin{macrocode}
    \par
    \@tempcnta=%
%<extend>    ##1%
%<!extend>    #1%
    \relax\ifnum\@tempcnta=1\else
%<class>      \ClassWarning{\KOMAClassName}{%
%<package>      \PackageWarning{scrextend}{%
        Optional argument of \string\maketitle\space ignored\MessageBreak
        in `titlepage=false' mode%
      }%
    \fi
    \ifx\@uppertitleback\@empty\else
%<package>     \PackageWarning{scrextend}{%
%<class>      \ClassWarning{\KOMAClassName}{%
        non empty \string\uppertitleback\space ignored
        by \string\maketitle\MessageBreak
        in `titlepage=false' mode%
      }%
    \fi
    \ifx\@lowertitleback\@empty\else
%<package>      \PackageWarning{scrextend}{%
%<class>      \ClassWarning{\KOMAClassName}{%
        non empty \string\lowertitleback\space ignored
        by \string\maketitle\MessageBreak
        in `titlepage=false' mode%
      }%
    \fi          
    \begingroup
      \let\titlepage@restore\relax
      \renewcommand*\thefootnote{\@fnsymbol\c@footnote}%
      \let\@oldmakefnmark\@makefnmark
      \renewcommand*{\@makefnmark}{\rlap\@oldmakefnmark}%
      \next@tdpage
      \if@twocolumn
        \ifnum \col@number=\@ne
          \ifx\@extratitle\@empty
            \ifx\@frontispiece\@empty\else\if@twoside\mbox{}\fi\fi
          \else
            \@makeextratitle
          \fi
          \ifx\@frontispiece\@empty
            \ifx\@extratitle\@empty\else\next@tdpage\fi
          \else
            \next@tpage
            \@makefrontispiece
            \next@tdpage
          \fi
          \@maketitle
        \else
          \ifx\@extratitle\@empty
            \ifx\@frontispiece\@empty\else\if@twoside\mbox{}\fi\fi
          \else
            \twocolumn[\@makeextratitle]%
          \fi
          \ifx\@frontispiece\@empty
            \ifx\@extratitle\@empty\else\next@tdpage\fi
          \else
            \next@tpage
            \twocolumn[\@makefrontispiece]%
            \next@tdpage
          \fi
          \twocolumn[\@maketitle]%
        \fi
      \else
        \ifx\@extratitle\@empty
          \ifx\@frontispiece\@empty\else \mbox{}\fi
        \else
          \@makeextratitle
        \fi
        \ifx\@frontispiece\@empty
          \ifx\@extratitle\@empty\else\next@tdpage\fi
        \else
          \next@tpage
          \@makefrontispiece
          \next@tdpage
        \fi
        \@maketitle
      \fi
      \ifx\titlepagestyle\@empty\else\thispagestyle{\titlepagestyle}\fi
      \@thanks\global\let\@thanks\@empty
    \endgroup
  \fi
  \setcounter{footnote}{0}%
  \expandafter\ifnum \csname scr@v@3.12\endcsname>\scr@compatibility\relax
    \let\thanks\relax
    \let\maketitle\relax
    \let\@maketitle\relax
    \global\let\@thanks\@empty
    \global\let\@author\@empty
    \global\let\@date\@empty
    \global\let\@title\@empty
    \global\let\@subtitle\@empty
    \global\let\@extratitle\@empty
    \global\let\@frontispiece\@empty    
    \global\let\@titlehead\@empty
    \global\let\@subject\@empty
    \global\let\@publishers\@empty
    \global\let\@uppertitleback\@empty
    \global\let\@lowertitleback\@empty
    \global\let\@dedication\@empty
    \global\let\author\relax
    \global\let\title\relax
    \global\let\extratitle\relax
    \global\let\titlehead\relax
    \global\let\subject\relax
    \global\let\publishers\relax
    \global\let\uppertitleback\relax
    \global\let\lowertitleback\relax
    \global\let\dedication\relax
    \global\let\date\relax
  \fi
  \global\let\and\relax
}%
%    \end{macrocode}
% \begin{macro}{\@maketitle}
% \changes{v2.95c}{2006/08/21}{default paragraph settings}
% \changes{v3.11c}{2012/02/14}{extra/mock title moved to \cs{@makeextratitle}}
% \changes{v3.12a}{2014/03/03}{\cs{next@tdpage} removed}
% \begin{macro}{\@makeextratitle}
% \changes{v3.11c}{2013/02/14}{added}
% \changes{v3.12a}{2014/03/03}{\cs{next@tdpage} removed}
% \begin{macro}{\@makefrontispiece}
% \changes{v3.25}{2017/11/21}{added}
% These do the printing of the main title, the extra or mock title and the
% frontispiece (retro page of the mock title).
%    \begin{macrocode}
\newcommand*{\@makeextratitle}{%
  \ifx\@extratitle\@empty \else
    \noindent\@extratitle\par
  \fi
}
\newcommand*{\@makefrontispiece}{%
  \ifx\@frontispiece\@empty \else
    \noindent\@frontispiece\par
  \fi
}
\newcommand*{\@maketitle}{%
  \global\@topnum=\z@
%<!extend>  \setparsizes{\z@}{\z@}{\z@\@plus 1fil}\par@updaterelative
%<extend>  \parskip\z@ \parindent\z@ \parfillskip\z@\@plus 1fil
  \ifx\@titlehead\@empty \else
    \begin{minipage}[t]{\textwidth}
      \usekomafont{titlehead}{\@titlehead\par}%
    \end{minipage}\par
  \fi
  \null
  \vskip 2em%
  \begin{center}%
    \ifx\@subject\@empty \else
      {\usekomafont{subject}{\@subject \par}}%
      \vskip 1.5em
    \fi
    {\usekomafont{title}{\huge \@title \par}}%
    \vskip .5em
    {\ifx\@subtitle\@empty\else\usekomafont{subtitle}\@subtitle\par\fi}%
    \vskip 1em
    {%
      \usekomafont{author}{%
        \lineskip .5em%
        \begin{tabular}[t]{c}
          \@author
        \end{tabular}\par
      }%
    }%
    \vskip 1em%
    {\usekomafont{date}{\@date \par}}%
    \vskip \z@ \@plus 1em
    {\usekomafont{publishers}{\@publishers \par}}%
    \ifx\@dedication\@empty \else
      \vskip 2em
      {\usekomafont{dedication}{\@dedication \par}}%
    \fi
  \end{center}%
  \par
  \vskip 2em
}%
%    \end{macrocode}
% \end{macro}^^A \@makefrontispiece
% \end{macro}^^A \@makeextratitle
% \end{macro}^^A \@maketitle
% \end{command}^^A \maketitle
%
% \begin{command}{\coverpagetopmargin}
% \changes{v3.12}{2013/11/21}{added}
% Top margin of a cover page (\opt{titlepage=firstiscover}).
%    \begin{macrocode}
\newcommand*{\coverpagetopmargin}{%
  \dimexpr \topmargin+1in\relax
}
%    \end{macrocode}
% \end{command}
%
% \begin{command}{\coverpagebottommargin}
% \changes{v3.12}{2013/11/21}{added}
% Bottom margin of a cover page (\opt{titlepage=firstiscover}).
%    \begin{macrocode}
\newcommand*{\coverpagebottommargin}{%
  2\dimexpr\coverpagetopmargin\relax
}
%    \end{macrocode}
% \end{command}
%
% \begin{command}{\coverpageleftmargin}
% \changes{v3.12}{2013/11/21}{added}
% Left margin of a cover page (\opt{titlepage=firstiscover}).
%    \begin{macrocode}
\newcommand*{\coverpageleftmargin}{%
  \dimexpr \evensidemargin+1in\relax
}
%    \end{macrocode}
% \end{command}
%
% \begin{command}{\coverpagerightmargin}
% \changes{v3.12}{2013/11/21}{added}
% Right margin of a cover page (\opt{titlepage=firstiscover}).
%    \begin{macrocode}
\newcommand*{\coverpagerightmargin}{\coverpageleftmargin}
%    \end{macrocode}
% \end{command}
%
% \begin{environment}{titlepage}
% The environment to be used for title pages. Title pages are always
% one-columned and use page style \pstyle{empty}. We have to take care to
% reset this at the end of the title page.
% \begin{description}
% \item[ToDo:] Wouldn't it be better to use \cs{pagestyle} instead of
%   \cs{thispagestyle}, because \cs{pagestyle} would also be valid, if a user
%   uses more than one page inside a \env{titlepage} environment?
% \item[ToDo:] See the note about \cs{next@tpage} above.
% \end{description}
%    \begin{macrocode}
%<package&extend>\scr@ifundefinedorrelax{titlepage}{%
\newenvironment{titlepage}{%
%<report|book>  \cleardoublepage
  \if@twocolumn
    \@restonecoltrue\onecolumn
  \else
    \@restonecolfalse\newpage
  \fi
  \thispagestyle{empty}%
  \if@compatibility
    \setcounter{page}{0}%
  \fi
}{%
  \if@restonecol\twocolumn \else \newpage \fi
}%
%<package&extend>}{}%
%    \end{macrocode}
% \end{environment}
%
% In case of package \pkg*{scrextend} the code of activatable \texttt{title}
% is complete and should be used if activated.
%    \begin{macrocode}
%<package&extend>}\csname scr@ext@immediate@title\endcsname
%    \end{macrocode}
%
% \begin{fontelement}{subtitle}
% \changes{v2.97c}{2007/06/20}{added}
%    \begin{macrocode}
\newkomafont{subtitle}{\usekomafont{title}\large}%
%    \end{macrocode}
% \end{fontelement}
%
% \begin{fontelement}{titlehead}
% \changes{v3.12}{2013/05/29}{added}
%    \begin{macrocode}
\newkomafont{titlehead}{}%
%    \end{macrocode}
% \end{fontelement}
%
% \begin{fontelement}{author}
% \changes{v3.12}{2013/05/29}{added}
%    \begin{macrocode}
\newkomafont{author}{\Large}
%    \end{macrocode}
% \end{fontelement}
%
% \begin{fontelement}{date}
% \changes{v3.12}{2013/05/29}{added}
%    \begin{macrocode}
\newkomafont{date}{\Large}
%    \end{macrocode}
% \end{fontelement}
%
% \begin{fontelement}{publishers}
% \changes{v3.12}{2013/05/29}{added}
%    \begin{macrocode}
\newkomafont{publishers}{\Large}
%    \end{macrocode}
% \end{fontelement}
%
% \begin{fontelement}{dedication}
% \changes{v3.12}{2013/05/29}{added}
%    \begin{macrocode}
\newkomafont{dedication}{\Large}
%    \end{macrocode}
% \end{fontelement}
%
% \begin{fontelement}{title}
% \changes{v2.8o}{2001/09/14}{added}
% \begin{macro}{\titlefont}
% \changes{v2.8p}{2001/09/22}{added}
% Font to be used for the title. The font element has to be defined manually,
% because of compatibility with \KOMAScript{} up to version 2.8n and old
% \LaTeX~2.09 style \textsc{Script 2.0}. \cs{titlefont} is a user command for
% compatibility only. So we index it a internal macro.
%    \begin{macrocode}
\newcommand*\titlefont{\sectfont}%
\newcommand*{\scr@fnt@title}{\titlefont}%
%    \end{macrocode}           
% \end{macro}
% \end{fontelement}
%
% \begin{fontelement}{subject}
% \changes{v2.8q}{2002/01/14}{added}
% \changes{v2.95}{2002/06/26}{also with \cls*{scrbook}, \cls*{scrreprt},
%   \cls*{scrartcl}}
% \begin{macro}{\subject@font}
% \changes{v2.8q}{2002/01/14}{added}
% \changes{v2.95}{2002/06/26}{also defined with \cls*{scrbook},
%   \cls*{scrreprt}, \cls*{scrartcl}}
% This has to be defined manually, because of compatibility with \KOMAScript{}
% up to version 2.8n and old \LaTeX~2.09 style \textsc{Script 2.0}.
%    \begin{macrocode}
\newcommand*{\subject@font}{\normalfont\normalcolor\bfseries\Large}%
\newcommand*{\scr@fnt@subject}{\subject@font}%
%</body>
%    \end{macrocode}
% \end{macro}
% \end{fontelement}
%
%    \begin{macrocode}
%</!letter>
%    \end{macrocode}
%
% \section{Implementation of the Abstract}
%
% The abstract is not part of the title but related to it. But maybe it is
% also a kind of section.
% \begin{description}
% \item[ToDo:] Should this be moved to \file{scrkernel-sections.dtx} or a new
%   file \file{scrkernel-abstract.dtx}?
% \end{description}
% It is only defined for the classes \cls*{scrartcl} and \cls*{scrreprt} but
% not for \cls*{scrbook}, \cls*{scrlttr2} or package \pkg*{scrextend}.
%    \begin{macrocode}
%<*class&(article|report)>           
%    \end{macrocode}
%
% \begin{option}{abstract}
% \changes{v2.95c}{2006/08/21}{added}
% \begin{macro}{\@abstrttrue,\@abstrtfalse,\if@abstrt}
% \changes{v3.17}{2015/03/10}{using \pkg*{scrkbase}}
% \KOMAScript{} provides abstracts with and without titles and only for the
% classes \cls*{scrartcl} and \cls*{scrreprt}. Books usually use a chapter for
% the abstract. We just need a simple boolean option, that sets a boolean
% switch.
%    \begin{macrocode}
%<*option>
\KOMA@ifkey{abstract}{@abstrt}
%    \end{macrocode}
% \begin{option}{abstracton,abstractoff}
% \changes{v2.95c}{2006/08/21}{obsolete}
% \changes{v3.01a}{2008/11/21}{standard not obsolete}
% \changes{v3.27}{2019/03/13}{deprecated}
%    \begin{macrocode}
\KOMA@DeclareDeprecatedOption{abstracton}{abstract=true}
\KOMA@DeclareDeprecatedOption{abstractoff}{abstract=false}
%</option>
%    \end{macrocode}
% \end{option}^^A abstracton,abstractoff
% \end{macro}^^A \if@abstrt
% \end{option}^^A abstract
%
% \begin{environment}{abstract}
% \changes{v2.3a}{1995/07/08}{\cs{@endparpenalty} added}
% \changes{v2.3g}{1996/01/14}{\cs{@beginparpenalty} added}
% \changes{v2.7a}{2001/01/04}{\cs{section*} replaced by \cs{addsec*}}
% \changes{v3.04a}{2009/07/14}{\cs{if@titlepage} moved inside environment}
% The abstract is different for documents with title page. We provide an
% abstract with or without title.
%    \begin{macrocode}
%<*body>
\newenvironment{abstract}{%
  \if@titlepage
    \titlepage
    \null\vfil
    \@beginparpenalty\@lowpenalty
    \if@abstrt
      \begin{center}
        \normalfont\sectfont\nobreak\abstractname
        \@endparpenalty\@M
      \end{center}
    \fi
  \else
    \if@twocolumn\if@abstrt
        \addsec*{\abstractname}
      \fi
    \else
      \if@abstrt
        \small
        \begin{center}
          {\normalfont\sectfont\nobreak\abstractname
            \vspace{-.5em}\vspace{\z@}}%
        \end{center}
      \fi
      \quotation
    \fi
  \fi
}{%
  \if@titlepage
    \par\vfil\null\endtitlepage
  \else
    \if@twocolumn\else\endquotation\fi
  \fi
}
%    \end{macrocode}
% \end{environment}^^A abstract
%
% \begin{command}{\abstractname}
% \changes{v3.36}{2022/03/08}{explicit definition for English languages}
% The name of the abstract (if a title is used).
%    \begin{macrocode}
\newcommand*\abstractname{Abstract}
\providecaptionname{american,australian,british,canadian,english,newzealand,%
  UKenglish,ukenglish,USenglish,usenglish}\abstractname{Abstract}
%</body> 
%    \end{macrocode}
% \end{command}^^A \abstractname
%
%    \begin{macrocode}
%</class&(article|report)>           
%    \end{macrocode}
%
%
% \Finale
% \PrintChanges
% 
\endinput
% Local Variables:
% mode: doctex
% ispell-local-dictionary: "en_US"
% eval: (flyspell-mode 1)
% TeX-master: t
% TeX-engine: luatex-dev
% eval: (setcar (or (cl-member "Index" (setq-local TeX-command-list (copy-alist TeX-command-list)) :key #'car :test #'string-equal) (setq-local TeX-command-list (cons nil TeX-command-list))) '("Index" "mkindex %s" TeX-run-index nil t :help "makeindex for dtx"))
% End:
