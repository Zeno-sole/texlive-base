% \iffalse meta-comment
% ======================================================================
% scrkernel-variables.dtx
% Copyright (c) Markus Kohm, 2006-2022
%
% This file is part of the LaTeX2e KOMA-Script bundle.
%
% This work may be distributed and/or modified under the conditions of
% the LaTeX Project Public License, version 1.3c of the license.
% The latest version of this license is in
%   http://www.latex-project.org/lppl.txt
% and version 1.3c or later is part of all distributions of LaTeX 
% version 2005/12/01 or later and of this work.
%
% This work has the LPPL maintenance status "author-maintained".
%
% The Current Maintainer and author of this work is Markus Kohm.
%
% This work consists of all files listed in MANIFEST.md.
% ======================================================================
%%% From File: $Id: scrkernel-variables.dtx 3874 2022-06-05 10:38:02Z kohm $
%<option>%%%            (run: option)
%<body>%%%            (run: body)
%<*dtx>
\ifx\ProvidesFile\undefined\def\ProvidesFile#1[#2]{}\fi
\begingroup
  \def\filedate$#1: #2-#3-#4 #5${\def\filedate{#2/#3/#4}}
  \filedate$Date: 2022-06-05 12:38:02 +0200 (So, 05. Jun 2022) $
  \def\filerevision$#1: #2 ${\def\filerevision{r#2}}
  \filerevision$Revision: 3874 $
  \xdef\fileinfo{\filedate\space\filerevision\space}
  \edef\reserved@a{%
    \noexpand\endgroup
    \noexpand\ProvidesFile{scrkernel-variables.dtx}%
                          [\filedate\space\filerevision\space
                           KOMA-Script source(koma variables)]
  }%
\reserved@a
\documentclass[USenglish]{koma-script-source-doc}
\usepackage{babel}
\setcounter{StandardModuleDepth}{2}
\begin{document}
\DocInput{scrkernel-variables.dtx}
\end{document}
%</dtx>
% \fi
%
% \changes{v2.95}{2006/03/20}{new by splitting \file{scrclass.dtx}}
% \changes{v3.36}{2022/02/15}{switch over from \cls*{scrdoc} to
%   \cls*{koma-script-source-doc}}
% \changes{v3.36}{2022/02/15}{whole implementation documentation in English}
%
% \GetFileInfo{scrkernel-variables.dtx}
% \title{Using Variables in
%   \href{https://komascript.de}{\KOMAScript} Letter Classes and Package}
% \author{\href{mailto:komascript@gmx.info}{Markus Kohm}}
% \date{Revision \fileversion{} of \filedate}
% \maketitle
% \begin{abstract}
%   Letters have several variables like the name and address of the sender and
%   the addressee, for fields like customer ids phone numbers, email addresses
%   etc. Depending on the purpose of a letter or a corporate identity much
%   more field and so variables are possible. For all of them
%   \file{scrkernel-variables.dtx} provides a general mechanism and user
%   interface.
% \end{abstract}
% \tableofcontents
%
% \section{User Manual}
%
% You can find the user documentation the commands implemented here in the
% \KOMAScript{} manual, either the German \file{scrguide} or the English
% \file{scrguien}.
% 
% \MaybeStop{\PrintIndex}
%
% \section{Implementation of Variables}
%
%    \begin{macrocode}
%<*letter>
%    \end{macrocode}
%
% \subsection{Options}
% The mechanism does not depend on options.
%
%
% \subsection{Commands and macros for variables}
%
% \begin{command}{\newkomavar*,\newkomavar}
% \changes{v2.8q}{2002/01/14}{added}
% \changes{v3.20}{2016/04/12}{\cs{@ifstar} replaced \cs{kernel@ifstar} by}
% We have two kinds of fields and so variables:
% \begin{itemize}
% \item to be printed in the reference line,
% \item for other purpose, but not printed in the reference line.
% \end{itemize}
% But it is also possible to add variables later to (or
% even remove them from the reference line. But the star variant
% \cs{newkomavar*} automatically adds a variable to (the end of) the reference
% line.
% \begin{macro}{\@newkomavar}
% Because of the star variant (and because we currently do not use
% \pkg{xparse}), we need an internal command.
% Both \cs{newkomavar} and \cs{newkomavar*} have one optional argument
% (\texttt{\#1}), the \meta{output name}. It is intended to be printed. In
% opposite to this the first mandatory argument (\texttt{\#2}) is the
% \meta{name} of the variable. It is not indented to be printed but used for access
% only. Only \emph{command sequence} characters are allowed here. It is
% recommended to use only letters for the \meta{name}. Internally
% macro \cs{scr@\meta{name}@name} and \cs{scr@\meta{name}@var} are
% defined. The second one is initialized to be empty.
%    \begin{macrocode}
%<*body>
\newcommand*{\newkomavar}{%
  \kernel@ifstar {\@tempswatrue\@newkomavar}{\@tempswafalse\@newkomavar}}
\newcommand*{\@newkomavar}[2][\relax]{%
  \@ifundefined{scr@#2@name}{%
    \@ifundefined{scr@#2@var}{%
      \begingroup
        \def\@tempa{#1}\def\@tempb{\relax}%
        \ifx\@tempa\@tempb\endgroup\else
        \endgroup\@namedef{scr@#2@name}{#1}%
      \fi%
      \expandafter\let\csname scr@#2@var\endcsname=\@empty
      \if@tempswa\addtoreffields{#2}\fi
    }{%
%<class>      \ClassError{scrlttr2%
%<package>      \PackageError{scrletter%
      }{%
        This should never happen%
      }{%
        The contents of the KOMA-Script variable `#2'\MessageBreak
        is undefined, but the name of the variable is
        defined.\MessageBreak
        This should never happen. So someone crashs me!%
      }%
    }%
  }{%
%<class>    \ClassError{scrlttr2%
%<package>    \PackageError{scrletter%
    }{%
      Variable `#2' already defined%
    }{%
      I'll ignore this command, if you'll continue.%
    }%
  }%
}
%    \end{macrocode}
% \end{macro}
% \end{command}
%
% \begin{command}{\setkomavar*,\setkomavar}
% \changes{v2.8q}{2002/01/14}{added}
% \changes{v3.08}{2010/10/28}{\cs{scr@\emph{Name}@postsetname} and
%   \cs{scr@\emph{Name}@postsetvar} added}
% \changes{v3.20}{2016/04/12}{\cs{@ifstar} replaced by \cs{kernel@ifstar}}
% After defining a new variable we also need a command to set the
% \meta{value}. This is the second mandatory argument (\texttt{\#3}) of
% \cs{setkomavar}. The first mandatory argument (\texttt{\#2}) once again is
% the \meta{name}. The \meta{output name} can be set with the
% optional first argument (\texttt{\#1}).  But for the case only the
% \meta{output name} should be changed but not the \meta{value} there is an
% additional star variant \cs{setkomavar*}. This has only two mandatory argument:
% \meta{name} (\texttt{\#1}) and \meta{output name} (\texttt{\#2}). Both also
% executes two hooks \cs{scr@\meta{name}@postsetname} and
% \cs{scr@\meta{name}@postsetvar} after setting the \meta{output name}
% resp. the \meta{value}. So package authors can define additional actions for
% the case a variable changes.
% \begin{macro}{\@setkomavar}
% \changes{v2.8q}{2002/01/14}{added}
% \changes{v3.20}{2016/04/12}{\cs{@ifnextchar} replaced by \cs{kernel@ifnextchar}}
% \begin{macro}{\@@setkomavar}
% \changes{v2.8q}{2002/01/14}{added}
% \begin{macro}{\@setkomaname}
% \changes{v2.8q}{2002/01/14}{added}
% Because of the star variant and the optional argument additional helper
% macros are needed.
%    \begin{macrocode}
\newcommand*{\setkomavar}{%
  \kernel@ifstar {\@setkomaname}{\@setkomavar}}
\newcommand*{\@setkomavar}[1]{%
  \kernel@ifnextchar [%]
  {\@@setkomavar{#1}}{\@@setkomavar{#1}[\relax]}}
\newcommand*{\@setkomaname}[2]{%
  \@ifundefined{scr@#1@var}{%
    \@komavar@err{set}{#1}%
  }{%
    \@namedef{scr@#1@name}{#2}%
    \csname scr@#1@postsetname\endcsname
  }%
}
\newcommand*{\@@setkomavar}{}
\long\def\@@setkomavar#1[#2]#3{%
  \@ifundefined{scr@#1@var}{%
    \@komavar@err{set}{#1}%
  }{%
    \@namedef{scr@#1@var}{#3}%
    \csname scr@#1@postsetvar\endcsname
    \begingroup
      \def\@tempa{#2}\def\@tempb{\relax}%
      \ifx\@tempa\@tempb\endgroup\else
      \endgroup\@namedef{scr@#1@name}{#2}%
      \csname scr@#1@postsetname\endcsname
    \fi
  }%
}
%    \end{macrocode}
% \end{macro}^^A \@setkomaname
% \end{macro}^^A \@@setkomavar
% \end{macro}^^A \@setkomavar
% \end{command}^^A \setkomavar*,\setkomavar
%
% \begin{macro}{\@komavar@err}
% \changes{v2.8q}{2002/01/14}{added}
% Error message for using or setting (\texttt{\#1}) unknown variables
% (\texttt{\#2}), used by \cs{setkomavar} or \cs{usekomavar}.
%    \begin{macrocode}
\newcommand*{\@komavar@err}[2]{%
%<class>  \ClassError{scrlttr2%
%<package>  \PackageError{scrletter%
  }{%
    KOMA-Script variable not defined%
  }{%
    You've tried to #1 the not defined KOMA-Script variable
    `#2'.\MessageBreak
    You have to define the variable using \string\newkomavar\space
    before\MessageBreak
    you do this.%
  }%
}
%    \end{macrocode}
% \end{macro}
%
% \begin{command}{\usekomavar*,\usekomavar}
% \changes{v2.8q}{2002/01/14}{added}
% \changes{v2.9i}{2002/09/04}{robust}
% \changes{v3.20}{2016/04/12}{\cs{@ifstar} replaced by \cs{kernel@ifstar}}
% User command to get either the \meta{output name} (star variant) or the
% \meta{value} (non-star variant) of a variable. Once again the first
% mandatory argument (\texttt{\#2}) is the \meta{name} of the value. Both also
% have an optional first argument (\texttt{\#1}) \meta{code}. The \meta{code}
% is added before the \meta{output name} or \meta{value} in the output
% stream and these are used as argument. You use it, e.g., to define a command
% with the content of \meta{output name} or \meta{value} or change the formatting.
% \begin{macro}{\@usekomavar,\@usekomaname}
% \changes{v2.8q}{2002/01/14}{added}
% \changes{v2.9i}{2002/09/04}{optional argument}
% \changes{v2.9j}{2002/09/18}{\cs{@firstofone} as default for the optional argument}
%    \begin{macrocode}
\DeclareRobustCommand*{\usekomavar}{%
  \kernel@ifstar {\@usekomaname}{\@usekomavar}}
\newcommand*{\@usekomavar}[2][\@firstofone]{%
  \@ifundefined{scr@#2@var}{%
    \@komavar@err{use}{#2}%
  }{%
    #1{\@nameuse{scr@#2@var}}%
  }%
}
\newcommand*{\@usekomaname}[2][\@firstofone]{%
  \@ifundefined{scr@#2@var}{%
    \@komavar@err{use}{#2}%
  }{%
    #1{\@nameuse{scr@#2@name}}%
  }%
}
%    \end{macrocode}
% \end{macro}^^A \@usekomavar,\@usekomaname
% \end{command}
%
% \begin{command}{\ifkomavar,\Ifkomavar}
% \changes{v3.03}{2009/03/04}{added}
% \changes{v3.28}{2019/11/15}{\cs{ifkomavar} renamed to {Ifkomavar}}
% \cs{Ifkomavar} tests if the variable with the \meta{name} of \texttt{\#1} is
% defined and execute the \meta{code} of \texttt{\#2} and otherwise the
% \meta{code} of \texttt{\#3}. The old command \cs{ifkomavar} is deprecated
% and should not be used any longer. This incompatible change has been done,
% because of a new recommendation of The \LaTeX{} Team (who does break its own
% recommendation, e.g., with \cs{ifToplevel}, \cs{ifthenelse} \dots).
%    \begin{macrocode}
\newcommand*{\ifkomavar}{%
%<package>  \PackageWarning{scrletter}{%
%<class>  \ClassWarning{\KOMAClassName}{%
    Usage of deprecated command `\string\ifkomavar'.\MessageBreak
    The command has been renamed to because of a\MessageBreak
    recommendation of LaTeX3 team members.\MessageBreak
    Please replace `\string\ifkomavar' by `\string\Ifkomavar'%
  }%
  \Ifkomavar
}
\DeclareRobustCommand*{\Ifkomavar}[1]{%
  \scr@ifundefinedorrelax{scr@#1@var}{%
    \expandafter\@secondoftwo
  }{%
    \expandafter\@firstoftwo
  }%
}
%    \end{macrocode}
% \end{command}
%
%
% \begin{command}{\ifkomavarempty*,\ifkomavarempty,\Ifkomavarempty*,\Ifkomavarempty}
% \changes{v2.8q}{2002/01/14}{added}
% \changes{v2.9i}{2002/09/04}{spurious spaced removed}
% \changes{v2.9i}{2002/09/04}{robust}
% \changes{v3.20}{2016/04/12}{\cs{@ifstar} replaced by \cs{kernel@ifstar}}
% \changes{v3.28}{2019/11/15}{\cs{ifkomavarempty} renamed to \cs{Ifkomavar}}
% \cs{Ifkomavarempty} tests if the variable with the \meta{name} of
% \texttt{\#1} is defined and the \meta{value} (non-star variant)
% resp. \meta{output name} (star variant) is empty. If it is undefined or
% empty the \meta{code} of \texttt{\#2} is executed. If is defined and not
% empty the \meta{code} of \texttt{\#3} is executed.
% The old command \cs{ifkomavarempty} resp. \cs{ifkomavarempty*} is deprecated
% and should not be used any longer. This incompatible change has been done,
% because of a new recommendation of The \LaTeX{} Team (who does break its own
% recommendation, e.g., with \cs{ifToplevel}, \cs{ifthenelse} \dots).
% \begin{macro}{\if@komavarempty}
% \changes{v2.8q}{2002/01/14}{added}
% \changes{v3.03}{2009/03/04}{using \cs{ifkomavar}}
% \changes{v3.03}{2009/03/04}{don't read the arguments but delegate them}
% \changes{v3.28}{2019/11/15}{\cs{ifkomavar} replaced by \cs{Ifkomavar}}
% Because of the star variant (and because we are currently not using
% \pkg{xparse}) we need an internal helper macro.
%    \begin{macrocode}
\newcommand*{\ifkomavarempty}{%
%<package>  \PackageWarning{scrletter}{%
%<class>  \ClassWarning{\KOMAClassName}{%
    Usage of deprecated command `\string\ifkomavarempty'.\MessageBreak
    The command has been renamed to because of a\MessageBreak
    recommendation of LaTeX3 team members.\MessageBreak
    Please replace `\string\ifkomavarempty' by `\string\Ifkomavarempty'%
  }%
  \Ifkomavarempty
}
\DeclareRobustCommand*{\Ifkomavarempty}{%
  \begingroup
  \kernel@ifstar {\@tempswatrue\@ifkomavarempty}%
                 {\@tempswafalse\@ifkomavarempty}%
}
\newcommand{\@ifkomavarempty}[1]{%
    \Ifkomavar{#1}{%
      \if@tempswa
        \@ifundefined{scr@#1@name}{%
          \aftergroup\@firstoftwo
        }{%
          \expandafter\ifx\csname scr@#1@name\endcsname\@empty
            \aftergroup\@firstoftwo
          \else
            \aftergroup\@secondoftwo
          \fi
        }%
      \else
        \expandafter\ifx\csname scr@#1@var\endcsname\@empty
          \aftergroup\@firstoftwo
        \else
          \aftergroup\@secondoftwo
        \fi
      \fi
    }{%
      \@komavar@err{use}{#1}%
      \aftergroup\@gobbletwo
    }%
  \endgroup
}
%    \end{macrocode}
% \end{macro}%^^A \@ifkomavarempty
% \end{command}%^^A \ifkomavarempty* … \Ifkomavarempty
%
%
% \begin{command}{\foreachkomavar}
% \changes{v3.27}{2019/03/13}{added}
% Executes \texttt{\#2} for each variable in the comma separated list
% \texttt{\#1}. An argument is appended to \texttt{\#2} which is the current
% variable to be processed. This command is short. So if you want so execute
% long code you should define a command for it. Sideeffect: \cs{reserved@a} is
% changed.
%    \begin{macrocode}
\newcommand*{\foreachkomavar}[2]{%
  \@for \reserved@a:=#1\do{%
    \scr@trim@spaces\reserved@a
    \ifx\reserved@a\@empty\else
      \edef\reserved@a{\unexpanded{#2}{\reserved@a}}\expandafter\reserved@a
    \fi
  }%
}
%    \end{macrocode}
% \end{command}
%
%
% \begin{command}{\foreachkomavarifempty}
% \changes{v3.27}{2019/03/13}{added}
% \changes{v3.28}{2019/11/15}{\cs{ifkomavarempty} replaced by \cs{Ifkomavarempty}}
% Same like \cs{foreachkomavar} but \texttt{\#2} is called only for empty
% variables and \texttt{\#3} is called for non empty.
%    \begin{macrocode}
\newcommand*{\foreachkomavarifempty}[3]{%
  \@for \reserved@a:=#1\do{%
    \scr@trim@spaces\reserved@a
    \ifx\reserved@a\@empty\else
      \edef\reserved@a{%
        \noexpand\Ifkomavarempty{\reserved@a}%
        {\unexpanded{#2}{\reserved@a}}%
        {\unexpanded{#3}{\reserved@a}}%
      }\expandafter\reserved@a
    \fi
  }%
}
%    \end{macrocode}
% \end{command}
%
% \begin{command}{\foreachemptykomavar}
% \changes{v3.27}{2019/03/13}{added}
% \changes{v3.28}{2019/11/15}{\cs{ifkomavarempty} replaced by \cs{Ifkomavarempty}}
% Similar to \cs{foreachemptykomavar} with \texttt{\#3}=\cs{@gobble}.
%    \begin{macrocode}
\newcommand*{\foreachemptykomavar}[2]{%
  \@for \reserved@a:=#1\do{%
    \scr@trim@spaces\reserved@a
    \ifx\reserved@a\@empty\else
      \edef\reserved@a{%
        \noexpand\Ifkomavarempty{\reserved@a}{\unexpanded{#2}{\reserved@a}}{}%
      }\expandafter\reserved@a
    \fi
  }%
}
%    \end{macrocode}
% \end{command}
%
% \begin{command}{\foreachnonemptykomavar}
% \changes{v3.27}{2019/03/13}{added}
% \changes{v3.28}{2019/11/15}{\cs{ifkomavarempty} replaced by \cs{Ifkomavarempty}}
% Similar to \cs{foreachemptykomavar} with \texttt{\#2}=\cs{@gobble}.
%    \begin{macrocode}
\newcommand*{\foreachnonemptykomavar}[2]{%
  \@for \reserved@a:=#1\do{%
    \scr@trim@spaces\reserved@a
    \ifx\reserved@a\@empty\else
      \edef\reserved@a{%
        \noexpand\Ifkomavarempty{\reserved@a}{}{\unexpanded{#2}{\reserved@a}}%
      }\expandafter\reserved@a
    \fi
  }%
}
%</body>
%    \end{macrocode}
% \end{command}
%
%
%    \begin{macrocode}
%</letter>
%    \end{macrocode}
% 
% \Finale
% \PrintChanges
% 
\endinput
% Local Variables:
% mode: doctex
% ispell-local-dictionary: "en_US"
% eval: (flyspell-mode 1)
% TeX-master: t
% TeX-engine: luatex-dev
% eval: (setcar (or (cl-member "Index" (setq-local TeX-command-list (copy-alist TeX-command-list)) :key #'car :test #'string-equal) (setq-local TeX-command-list (cons nil TeX-command-list))) '("Index" "mkindex %s" TeX-run-index nil t :help "makeindex for dtx"))
% End:
