% \iffalse meta-comment
%
% makeshape.dtx
% 25 January 2013
%
% This is version 1.0 of makeshape.dtx
%    It generates a style file (makeshape.sty), a sample shape (sampleshape.tex),
%       and a user guide and documentation (makeshape.pdf).
%
% Generation of the makeshape.sty and sampleshape.tex is with 
%    pdflatex makeshape.ins
% which contains   
%     \generate{\file{makeshape.sty}
%              {\from{makeshape.dtx}{package}}}
%     \generate{\file{sampleshape.tex}
%              {\from{makeshape.dtx}{sample}}}
%
% Generation of makeshape.pdf is with
%    pdflatex makeshape.dtx
%
% The \OnlyDescription command is used to suppress implementation 
% documentation.
%
% This is part of Silhouette 2.1
%
% -----------------------------------------------------------
% \fi
%
%^^A-------------------------------------------------------------------
%^^A    .sty file header
%^^A-------------------------------------------------------------------
% \iffalse
%<package>\NeedsTeXFormat{LaTeX2e}[2011/06/27]
%^^A - ** Use package version number (not file version) **
%<package>\ProvidesPackage{makeshape}
%<package>   [2013/01/25 2.1 Making custom shapes for PGF/TikZ]
%<package>\RequirePackage{tikz}
%<package>\usepgflibrary{intersections}
%
%^^A-------------------------------------------------------------------
%^^A    Preamble for documentation
%^^A-------------------------------------------------------------------
%<*driver>
\documentclass[10pt,a4paper]{ltxdoc}

\usepackage[T1]{fontenc}
\usepackage{lmodern}

\usepackage{fancyvrb}
\usepackage{enumitem}
\usepackage{url}
\usepackage{bigstrut}
\usepackage[draft]{hyperref}
%\usepackage{hyperref}

\usepackage{tikz}
\usetikzlibrary{patterns}
\usetikzlibrary{arrows}
\usetikzlibrary{positioning} 
\usepgflibrary{intersections}

%%
%% This is file `sampleshape.tex',
%% generated with the docstrip utility.
%%
%% The original source files were:
%%
%% makeshape.dtx  (with options: `sample')
%% 
%% Copyright (C) 2013 by Adrian P Robson
%%    adrian.robson@nepsweb.co.uk
%% 
%% This work may be distributed and/or modified under the
%% conditions of the LaTeX Project Public License, either
%% version 1.3c of this license or (at your option) any later
%% version. The latest version of this license is in
%%    http://www.latex-project.org/lppl.txt
%% 
%% This work has the LPPL maintenance status `maintained'.
%% The Current Maintainer of this work is Adrian Robson.
%% 
%% This work consists of the files makeshape.ins
%%                                 makeshape.dtx and
%%                                 ontesting.dtx
%% and the derived files           makeshape.sty
%%                                 makeshape.pdf
%%                                 sampleshape.tex
%%                                 ontesting.pdf
%%                                 testsample.tex and
%%                                 testsample.pdf
%% 
%%--------------------------------------------------
%%
%% This is a sample shape for the makeshape package
%%
%% It can be adapted for other shapes. See
%% makeshape.pdf for instructions.
%%
%% 25 January 2013

\usepackage{makeshape}
\makeatletter

%% Constant for sample shape
\def\gap{4pt}

%% Anchor path <- change and rename as required
\def\sampleanchor{
   % get corrected text box
   \pgf@xa=\ctbnex
   \pgf@ya=\ctbney
   % make room for shape
   \advance\pgf@xa by \gap
   \advance\pgf@ya by \gap
   % correct for minheight and minwidth and
   % for outerxsep or outerysep
   \mincorrect{\pgf@xa}{\pgfshapeminwidth}
   \advance\pgf@xa\pgfshapeouterxsep
   \mincorrect{\pgf@ya}{\pgfshapeminheight}
   \advance\pgf@ya\pgfshapeouterysep
   % draw the path
   \pgfpathmoveto{\pgfpoint{\pgf@xa}{\pgf@ya}}
   \pgfpathlineto{\pgfpoint{\pgf@xa}{-\pgf@ya}}
   \pgfpathlineto{\pgfpoint{-\pgf@xa}{-\pgf@ya}}
   \pgfpathlineto{\pgfpoint{-\pgf@xa}{\pgf@ya}}
   \pgfpathclose
}

%% Background path <- change and rename as required
\def\sampleborder{
   % get corrected text box
   \pgf@xa=\ctbnex
   \pgf@ya=\ctbney
   % make room for shape
   \advance\pgf@xa by \gap
   \advance\pgf@ya by \gap
   % correct for minheight and minwidth but
   %    not for outerxsep or outerysep
   \mincorrect{\pgf@xa}{\pgfshapeminwidth}
   \mincorrect{\pgf@ya}{\pgfshapeminheight}
   % draw outer shape
   \pgfpathmoveto{\pgfpoint{\pgf@xa}{\pgf@ya}}
   \pgfpathlineto{\pgfpoint{\pgf@xa}{-\pgf@ya}}
   \pgfpathlineto{\pgfpoint{-\pgf@xa}{-\pgf@ya}}
   \pgfpathlineto{\pgfpoint{-\pgf@xa}{\pgf@ya}}
   \pgfpathclose
   % draw inner shape
   \advance\pgf@xa by -\gap
   \advance\pgf@ya by -\gap
   \pgfpathmoveto{\pgfpoint{\pgf@xa}{\pgf@ya}}
   \pgfpathlineto{\pgfpoint{\pgf@xa}{-\pgf@ya}}
   \pgfpathlineto{\pgfpoint{-\pgf@xa}{-\pgf@ya}}
   \pgfpathlineto{\pgfpoint{-\pgf@xa}{\pgf@ya}}
   \pgfpathclose
}

%%------------------------------------------------
%% Shape declaration <- Change name as required
\pgfdeclareshape{sample}{

   % Set paths <- change path macros as required
   \setpaths{\sampleanchor}{\sampleborder}

   % Saved anchors <- change as required
   \savedanchor{\northeast}{
      \pgf@x = \ctbnex
      \pgf@y = \ctbney
      \advance\pgf@x by \gap
      \advance\pgf@y by \gap
      \mincorrect{\pgf@x}{\pgfshapeminwidth}
      \mincorrect{\pgf@y}{\pgfshapeminheight}
      \advance\pgf@x\pgfshapeouterxsep
      \advance\pgf@y\pgfshapeouterysep
   }

   % Anchors <- change as required
   \anchor{north}{\northeast \pgf@x=0pt}
   \anchor{north east}{\northeast}
   \anchor{east}{\northeast \pgf@y=0pt}
   \anchor{south east}{\northeast \pgf@y=-\pgf@y}
   \anchor{south}{\northeast \pgf@x=0pt \pgf@y=-\pgf@y}
   \anchor{south west}{\northeast \pgf@x=-\pgf@x \pgf@y=-\pgf@y}
   \anchor{west}{\northeast \pgf@x=-\pgf@x \pgf@y=0pt}
   \anchor{north west}{\northeast \pgf@x=-\pgf@x}

}

\makeatother

\endinput
%%
%% End of file `sampleshape.tex'.
 % includes makeshape.sty

\newcommand{\mnote}[1]
{\marginpar{\scriptsize \raggedright #1 }}

\newcommand{\pfgManCiteA}{\cite{pgfMan},  \emph{Constructing Paths}, \S71, pp 579-589}
\newcommand{\pfgManCiteB}{\cite{pgfMan},  \emph{Declaring New Shapes}, \S75.5, pp 625-631}

\EnableCrossrefs
\CodelineIndex
\RecordChanges
\OnlyDescription       % comment out for code implementation

\begin{document}
\DocInput{makeshape.dtx}
\end{document}
%</driver>
% \fi
%
%^^A-------------------------------------------------------------------
%^^A    Change log...
%^^A-------------------------------------------------------------------
%^^A
%^^A 0.0 - 5 January 2013
%    \changes{0.0}{2013/01/05}{
%       Experimental -
%       Develop the style file and a sample shape.
%       Write the user guide.
%       Incorporate makeshape.sty generation and documentation. 
%       Generate the samplshape.tex file.
%    }
%^^A
%^^A 0.1 - 11 January 2013 
%    \changes{0.1}{2013/01/11}{
%       Experimental -
%       Fix the sample shape's cardinal anchor points problem by 
%       correcting a bug in the north-east saved anchor. 
%       Correct errors in the user guide and fill in omissions.
%       \emph{Experimental phase complete.}}
%^^A
%^^A 25 January 2013 
%    \changes{1.0}{2013/01/25}{
%       Prepare for publication -
%       Very minor changes made to the style file code.
%       Standard path commands used in sample shape code. 
%       A lot of the report is rewritten. 
%       An abstract is added, and its contents removed. }
%^^A
%^^A-------------------------------------------------------------------
%^^A    Check sums... 
%^^A-------------------------------------------------------------------
% 
%^^A \CheckSum{0} ^^A no checksum for development
% \CheckSum{272}
% \CharacterTable
%  {Upper-case    \A\B\C\D\E\F\G\H\I\J\K\L\M\N\O\P\Q\R\S\T\U\V\W\X\Y\Z
%   Lower-case    \a\b\c\d\e\f\g\h\i\j\k\l\m\n\o\p\q\r\s\t\u\v\w\x\y\z
%   Digits        \0\1\2\3\4\5\6\7\8\9
%   Exclamation   \!     Double quote  \"     Hash (number) \#
%   Dollar        \$     Percent       \%     Ampersand     \&
%   Acute accent  \'     Left paren    \(     Right paren   \)
%   Asterisk      \*     Plus          \+     Comma         \,
%   Minus         \-     Point         \.     Solidus       \/
%   Colon         \:     Semicolon     \;     Less than     \<
%   Equals        \=     Greater than  \>     Question mark \?
%   Commercial at \@     Left bracket  \[     Backslash     \\
%   Right bracket \]     Circumflex    \^     Underscore    \_
%   Grave accent  \`     Left brace    \{     Vertical bar  \|
%   Right brace   \}     Tilde         \~}
%
%^^A===========================================================================
%^^A    Generate figure files for input into this report
%^^A===========================================================================
%
%^^A---------------------------------------------------------------------------
%^^A    Background path diagram file
%^^A---------------------------------------------------------------------------
% \begin{VerbatimOut}[gobble=5]{msbackground.tmp}
%    %^^A - The background path diagram, part of makeshape,dtx
%    %^^A - This is a temporary file and should be deleted
%    %^^A
%    \begin{tikzpicture}[>=latex',font={\sf \small}]
%
%    %^^A-- diagram 1 --------------------------------------------
%    \begin{scope}
%
%    \def\ctbw{88pt}
%    \def\ctbh{14pt}
%
%    \def\ctbnex{\ctbw/2}
%    \def\ctbney{\ctbh/2}
%
%    \node (outer) at (0,0) [draw, rectangle, pattern=crosshatch dots,
%       minimum width=\ctbw+20pt,
%       minimum height=\ctbh+20pt,
%    ] {} ;
%
%    \node at (\ctbnex,\ctbney) [circle, fill=white] {};
%
%    \node (ctb) at (0,0) [draw, rectangle, fill=white,
%       minimum width=\ctbw,
%       minimum height=\ctbh,
%    ] {};
%
%    \node (ctbne) at (\ctbnex,\ctbney) [draw, circle] {};
%
%    %^^A - Labels
%    \node (ctbneLabel) at (4cm,1cm) 
%       [inner sep = 0] {(\verb|\ctbnex|,\verb|\ctbney|)};
%    \node (ctbLabel) at (2cm,-1.2cm) 
%       [inner sep = 0] {corrected text box};
%    \node (innerLabel) at (-1.8cm,1.2cm) 
%       [] {inner surface};
%    \node (outerLabel) at (-2cm,-1.2cm) 
%       [inner sep = 0] {outer surface};
% 
%    \draw [->] (ctbneLabel.south west) -- (ctbne);
%    \draw [->] (ctbLabel.north west) -- (ctb.320);
%    \draw [->] (innerLabel.320) -- (ctb.160);
%    \draw [->] (outerLabel.north east) -- (outer);
%
%    %^^A - xy axis
%    \draw [dotted] (0,0) -- ++(2.5cm,0);  % origin
%    \draw [dotted] (0,0) -- ++(-2.5cm,0); % origin
%    \draw [dotted] (0,0) -- ++(0,1cm);    % origin
%    \draw [dotted] (0,0) -- ++(0,-1cm);   % origin
% 
%    \end{scope}
% 
%    %^^A-- diagram 2 --------------------------------------------
%    \begin{scope}[yshift=-4cm]
% 
%    \def\ctbw{88pt}
%    \def\ctbh{14pt}
% 
%    \def\ctbnex{\ctbw/2}
%    \def\ctbney{\ctbh/2}
% 
%    \node (outer) at (0,0) [draw, rectangle, pattern=crosshatch dots,
%       minimum width=\ctbw+40pt,
%       minimum height=\ctbh+40pt,
%    ] {} ;
% 
%    \node at (\ctbnex,\ctbney) [circle, fill=white] {};
% 
%    \node (inner) at (0,0) [draw, rectangle, fill=white,
%       minimum width=\ctbw+20pt,
%       minimum height=\ctbh+20pt,
%    ] {};
% 
%    \node (ctb) at (0,0) [draw, rectangle, fill=white,
%       minimum width=\ctbw,
%       minimum height=\ctbh,
%    ] {};
% 
%    \node at (\ctbnex,\ctbney) [draw, circle] (ctbne) {};
% 
%    %^^A - Labels
%    \node (ctbneLabel) at (4cm,1cm) 
%       [inner sep = 0] {(\verb|\ctbnex|,\verb|\ctbney|)};
%    \node (ctbLabel) at (2cm,-1.5cm) 
%       [inner sep = 0] {corrected text box};
%    \node (innerLabel) at (-2cm,-2cm) 
%       [inner sep = 0] {inner surface};
%    \node (outerLabel) at (-2.5cm,-1.5cm) 
%       [inner sep = 0] {outer surface};
% 
%    \draw [->] (ctbneLabel.south west) -- (ctbne);
%    \draw [->] (ctbLabel.north west) -- (ctb.320);
%    \draw [->] (innerLabel.north east) -- (inner.250);
%    \draw [->] (outerLabel.north east) -- (outer);
% 
%    %^^A - minimum dimensions
%    \node (a) [coordinate, left=15pt of outer.south west] {};
%    \node (b) [coordinate, left=15pt of outer.north west] {};
%    \draw [|<->|] (a) -- (b) node [midway,left] {\tt minimum height};
% 
%    \node (c) [coordinate, above=15pt of outer.north west] {};
%    \node (d) [coordinate, above=15pt of outer.north east] {};
%    \draw [|<->|] (c) -- (d) node [midway,above] {\tt minimum width};
% 
%    %^^A - xy axis
%    \draw [dotted] (0,0) -- ++(2.8cm,0);    % origin
%    \draw [dotted] (0,0) -- ++(-2.5cm,0);   % origin
%    \draw [dotted] (0,0) -- ++(0,1.2cm);    % origin
%    \draw [dotted] (0,0) -- ++(0,-1.2cm);   % origin
% 
%    \end{scope}
% 
%    \end{tikzpicture}
% \end{VerbatimOut}
%
%^^A---------------------------------------------------------------------------
%^^A    Anchor path diagram file
%^^A---------------------------------------------------------------------------
% \begin{VerbatimOut}[gobble=5]{msanchor.tmp}
%    %^^A - The anchor path diagram, part of makeshape,dtx
%    %^^A - This is a temporary file and should be deleted
%    %^^A
%    \begin{tikzpicture}[>=latex',font={\sf \small}]
%
%    \def\ctbw{88pt}
%    \def\ctbh{14pt}
%
%    \def\ctbnex{\ctbw/2}
%    \def\ctbney{\ctbh/2}
%
%    %^^A-- diagram 1 --------------------------------------------
%    \begin{scope}
%
%    \node (outer) at (0,0) [draw, rectangle, pattern=crosshatch dots,
%       minimum width=\ctbw+20pt,
%       minimum height=\ctbh+20pt,
%    ] {};
%
%    \node at (\ctbnex,\ctbney) [circle, fill=white] {};
%
%    \node (ctb) at (0,0) [draw, rectangle, fill=white,
%       minimum width=\ctbw,
%       minimum height=\ctbh,
%    ]  {};
%
%    \node (ctbne) at (\ctbnex,\ctbney) [draw, circle] {};
%
%    %^^A - Labels
%    \node (ctbneLabel) at (4cm,1cm) 
%       [inner sep = 0] {(\verb|\ctbnex|,\verb|\ctbney|)};
%    \node (ctbLabel) at (2cm,-1.2cm) 
%       [inner sep = 0] {corrected text box};
%    \node (spaceLabel) at (-1.8cm,1.2cm) 
%       [] {background path space};
%    \node (outerLabel) at (-2.3cm,-1.2cm) 
%       [inner sep = 0] {outer shape surface};
%    \node (anchorLabel) at (-1.5cm,-1.7cm) 
%       [inner sep = 0] {anchor surface};
%
%    \draw [->] (ctbneLabel.south west) -- (ctbne);
%    \draw [->] (ctbLabel.north west) -- (ctb.320);
%    \draw [->, shorten >=5pt] (spaceLabel.320) -- (ctb.160);
%    \draw [->] (outerLabel.north east) -- (outer);
%    \draw [->] (anchorLabel.north east) -- (outer);
%
%    %^^A - xy axis
%    \draw [dotted] (0,0) -- ++(2.5cm,0);  % origin
%    \draw [dotted] (0,0) -- ++(-2.5cm,0); % origin
%    \draw [dotted] (0,0) -- ++(0,1cm);    % origin
%    \draw [dotted] (0,0) -- ++(0,-1cm);   % origin
%
%    \end{scope}
%
%    %^^A-- diagram 2 --------------------------------------------
%    \begin{scope}[yshift=-5cm]
%
%    \node (anchor) at (0,0) [draw, rectangle,
%       minimum width=\ctbw+70pt,
%       minimum height=\ctbh+70pt,
%    ] {} ;
%
%    \node (outer) at (0,0) [draw, rectangle,  pattern=crosshatch dots,
%       minimum width=\ctbw+40pt,
%       minimum height=\ctbh+40pt,
%    ] {} ;
%
%    \node at (\ctbnex,\ctbney) [circle, fill=white] {};
%
%    \node (inner) at (0,0) [draw, rectangle, fill=white,
%       minimum width=\ctbw+20pt,
%       minimum height=\ctbh+20pt,
%    ] {};
%
%    \node (ctb) at (0,0) [draw, rectangle, fill=white,
%       minimum width=\ctbw,
%       minimum height=\ctbh,
%    ] {};
%
%    \node at (\ctbnex,\ctbney) [draw, circle] (ctbne) {};
%
%    %^^A - Labels
%    \node (ctbneLabel) at (4.3cm,1cm) 
%       [inner sep = 0] {(\verb|\ctbnex|,\verb|\ctbney|)};
%    \node (ctbLabel) at (2cm,-1.8cm) 
%       [inner sep = 0] {corrected text box};
%    \node (spaceLabel) at (-3.3cm,-1.8cm) 
%       [inner sep = 0] {background path space};
%    \node (outerLabel) at (-2.8cm,-2.3cm) 
%       [inner sep = 0] {outer path surface};
%    \node (anchorLabel) at (-2.7cm,-2.8cm) 
%       [inner sep = 0] {anchor path surface};
%
%    \draw [->] (ctbneLabel.south west) -- (ctbne);
%    \draw [->] (ctbLabel.north west) -- (ctb.320);
%    \draw [->,  shorten >=4pt] (spaceLabel.north east) -- (inner.210);
%    \draw [->] (outerLabel.north east) -- (outer.227);
%    \draw [->] (anchorLabel.north east) -- (anchor);
%
%    %^^A - minimum dimensions
%    \node (a) [coordinate, left=30pt of outer.south west] {};
%    \node (b) [coordinate, left=30pt of outer.north west] {};
%    \draw [|<->|] (a) -- (b) node [midway,left] {\tt minimum height};
%
%    \node (c) [coordinate, above=30pt of outer.north west] {};
%    \node (d) [coordinate, above=30pt of outer.north east] {};
%    \draw [|<->|] (c) -- (d) node [midway,above] {\tt minimum width};
%
%    \node (e) [coordinate, left=15pt of anchor.north west] {};
%    \draw [<->|] (b) -- (e) node [midway,left] {\tt outer ysep};
%
%    \node (f) [coordinate, above=15pt of anchor.north east] {};
%    \draw [<->|] (d) -- (f) node [right,above] {\hspace{30pt}\tt outer xsep};
%
%    \draw [dotted] (a) -- ++(30pt,0);
%    \draw [dotted] (b) -- ++(30pt,0);
%    \draw [dotted] (c) -- ++(0,-30pt);
%    \draw [dotted] (d) -- ++(0,-30pt);
%
%    %^^A - xy axis
%    \draw [dotted] (0,0) -- ++(3cm,0);      % origin
%    \draw [dotted] (0,0) -- ++(-3cm,0);     % origin
%    \draw [dotted] (0,0) -- ++(0,1.2cm);    % origin
%    \draw [dotted] (0,0) -- ++(0,-1.2cm);   % origin
%
%    \end{scope}
%
%    \end{tikzpicture}
% \end{VerbatimOut}
%
%^^A===========================================================================
%^^A      Start report body
%^^A===========================================================================
%
% \GetFileInfo{makeshape.sty}
%
% \title{The {\sf makeshape} package\thanks{
%        This document corresponds to \textsf{makeshape}~\fileversion, 
%        dated~\filedate.}\\
% and\\ 
% A Method for Creating Custom Shapes in PGF}
%
% \author{Adrian P. Robson\thanks{\texttt{adrian.robson@nepsweb.co.uk}}}
% \date{25 January 2013}
%
% \maketitle
%
%^^A - Abstract
% \vspace{-3ex}
% \begin{quote}
%   The {\sf makeshape} package simplifies writing PGF shapes. 
%   Declaring a custom shape with a correct anchor border can be difficult. 
%   Complex shapes often need complicated calculations to find the touching point 
%   of a connecting line. 
%   This package only requires that a developer write a PGF path describing the 
%   anchor border. 
%   It also provides macros that help with the management of shape parameters 
%   and the definition of anchor points.   
% \end{quote}
%
%
%^^A \tableofcontents
%
%\section{Introduction}
%
% The |\pgfdeclareshape| command can be used to create new shapes with the PGF package
% (\pfgManCiteB).  
%  The following are typically needed:
% \begin{itemize} [itemsep=-0.5ex]
%    \item A shape name.
%    \item Code for computing saved anchors and saved dimensions.
%    \item Code for computing anchor positions in terms of the saved anchors.
%    \item Code for computing border anchors (|\anchorborder|).
%    \item Code for drawing a background path.
% \end{itemize}
% Writing these can be hard for completely new shapes, and in particular the declaration of a 
% suitable |\anchorborder| command can be very difficult for complex shapes. 
%
% \emph{This paper presents a method that makes the process of writing new shapes from scratch 
% a little easier.}
% Its key features are: 
% a mechanism for specifying the |\anchorborder| behaviour as a path;
% a package that provides a set of useful macros for writing and allocating anchor 
% and background paths;
% and an example |\pgfdeclareshape| command that can be used as a template for new shapes. 
% 
% The method involves writing some original code and modifying the template.
% In brief, the process involves: 
% \begin{enumerate} [itemsep=-0.5ex]
%    \item Writing \emph{background path} and \emph{anchor path} macros.
%    \item Writing one or more \emph{saved anchors}.
%    \item Writing \emph{anchors} using the above saved anchors.
% \end{enumerate}
% 
%
% \section{Preliminaries}
%
% \subsection{Background Path Macro}\label{sec:backgroundpath}
%
% \begin{figure}
% \begin{center}
% \input{msbackground.tmp}
% \end{center}
% \caption{The Background Path}\label{fig:backgroundpath}
% \end{figure}
%
% The \emph{background path}, as illustrated in figure~\ref{fig:backgroundpath}, 
% describes the curves that form the actual shape.
% It can be just a single line and does not have to be closed.
% However, there are no limits on is complexity.
% If it is more than a simple outline, then its \emph{inner} and \emph{outer} surfaces 
% need to be considered separately. 
%
% The inner surface should contain the \emph{corrected text box}, 
% which is the shape's text box surrounded by the space specified by its 
% {\tt inner xsep} and {\tt inner ysep} keys. 
% The coordinates of this box are given by the package's |\ctbnex| and |\ctbnex| macros, 
% which are described in \S\ref{sec:corrctdTextBox}.     
%
% The outer surface should have height and width dimensions that are the same as 
% the shape's {\tt minimum} {\tt width} and {\tt minimum} {\tt height} 
% keys if these are larger than the basic dimensions. 
% These keys can be obtained with the |\pgfshapeminwidth| and |\pgfshapeminheight| macros,
% as described in \S\ref{sec:keycommands}. 
% The |\mincorrect| macro described in \S\ref{sec:mincorrmacro} can be used make the necessary 
% comparison and assignment.
%
% \subsubsection{Background Path Algorithm}\label{sec:backgroundpathmacro}
%
% The background path should be implemented as a macro with the following structure:
% \begin{enumerate}  
%
%    \item \label{item:ctb} Use |\ctbnex| and |\ctbnex|, 
%        which give the north east corner of the corrected text box,
%        to calculate significant reference coordinates.
%
%    \item \label{item:mindim} Use |\mincorrect| to modify the reference coordinates to 
%       take account of the shape's minimum dimension keys, 
%       which are given by |\pgfshapeminwidth| and |\pgfshapeminheight|.
%       
%
%    \item Use the reference coordinates from step \ref{item:mindim} to draw the actual path
%        using  PGF path commands (\pfgManCiteA). Typical commands are 
%        |\pgfpathmoveto|,
%        |\pgfpathlineto| and
%        |\pgfpathclose|.
%        The |\pgfpoint| command will also be needed.
%        Do this for
%        \begin{enumerate}
%           \item the outer surface curve, and \label{item:outer} 
%           \item curves in the shape's inner path space 
%        \end{enumerate}
%        Usually when a path is defined, it is displayed with |\pgfusepath{stroke}|.
%        However, it is not needed here.
%
% \end{enumerate}
%
% \subsection{Anchor Path Macro}\label{sec:anchorpath}
%
% The \emph{anchor path}, as illustrated in figure~\ref{fig:anchorpath}, 
% describes the curve that forms the surface where connecting lines meet the shape.
% It it must be a single closed path, and it is normally strongly related to 
% background path's outer surface (see \S\ref{sec:backgroundpath}).
%
% Like the background path, it should use the fundamental reference point of the  
% \emph{corrected text box}, which automatically includes the shape's inner 
% separation.
% It should also take regard of the shape's {\tt minimum} {\tt width} 
% and {\tt minimum} {\tt height} keys if these are larger than the basic dimensions. 
% Furthermore, it should compensate for the {\tt outer} {\tt xsep} and {\tt outer} {\tt ysep} 
% keys as shown in figure~\ref{fig:anchorpath}.
%
% The corrected text box macros are decribed in \S\ref{sec:corrctdTextBox}.
% The minimum dimension keys are given by |\pgfshapeminwidth| and |\pgfshapeminheight|,
% and the outer separation keys are given by |\pgfshapeouterxsep| and 
% |\pgfshapeouterysep|, as described in \S\ref{sec:keycommands}.
% The minimum dimension comparision and assignment is provided by |\mincorect| macro, 
% which is discussed in \S\ref{sec:mincorrmacro}.
%
% \subsubsection{Anchor Path Algorithm}\label{sec:anchorpathmacro}
%
%
% The anchor path should be implemented as a macro with the following structure:
% \begin{enumerate}
%
%    \item Use |\ctbnex| and |\ctbnex|, 
%       which give the north east corner of the corrected text box,
%       to calculate significant reference coordinates. (Normally the same as 
%       \S\ref{sec:backgroundpathmacro} step \ref{item:ctb}). 
%       \label{item:apctb}
%
%    \item Use |\mincorrect| to modify the reference coordinates to 
%       take account of the shape's minimum dimension keys, 
%       which given by |\pgfshapeminwidth| and |\pgfshapeminheight|.
%       (Normally the same as \S\ref{sec:backgroundpathmacro} step \ref{item:mindim}).
%       \label{item:apmindim}
%
%    \item Modify the reference coordinates to take account of outer separation, 
%       which is given by |\pgfshapeouterxsep| and |\pgfshapeouterysep|. 
%       \label{item:apouter}
%
%    \item Use the corrected reference coordinates from step \ref{item:apouter} to specify 
%       the anchor path using PGF path commands  (\pfgManCiteA).
%       The path should be closed so a |\pgfpathclose| command is required.
%       The path will not actually be drawn, and the |\pgfusepath{stroke}| macro 
%       should not be used.
%       (Normally the same as \S\ref{sec:backgroundpathmacro} step \ref{item:outer}).
%       \label{item:appath}
%
% \end{enumerate}
%
% \subsection{Anchors}
%
% A PGF shape requires the definition of a set of anchor point commands, 
% and these need \emph{saved anchors} to provide necessary coordinates. 
% All anchor macros should return a point coordinate by assigning values to the 
% registers |\pgf@x| and |\pgf@y|.
%
% Like the anchor path macro, saved anchors should use the fundamental reference point of the  
% \emph{corrected text box}, which automatically includes the shape's inner 
% separation.
% They should also take regard of the shape's {\tt minimum} {\tt width} 
% and {\tt minimum} {\tt height} keys if these are larger than the basic dimensions. 
% Furthermore, they should compensate for the {\tt outer} {\tt xsep} and {\tt outer} {\tt ysep} 
% keys as shown in figure~\ref{fig:anchorpath}.
%
% The corrected text box macros are decribed in \S\ref{sec:corrctdTextBox}.
% The minimum dimension keys are given by |\pgfshapeminwidth| and |\pgfshapeminheight|,
% and the outer separation keys are given by |\pgfshapeouterxsep| and 
% |\pgfshapeouterysep|, as described in \S\ref{sec:keycommands}.
% The minimum dimension comparision and assignment is provided by |\mincorect| macro, 
% which is discussed in \S\ref{sec:mincorrmacro}.
%
% \begin{figure}
% \begin{center}
% \input{msanchor.tmp}
% \end{center}
% \caption{The Anchor Path}\label{fig:anchorpath}
% \end{figure}
%
% \subsubsection{Saved Anchor Algorithm}
%
% A saved anchor must set the PGF registers |\pgf@x| and |\pgf@y|
% with the required coordinates.
% It has a structure similar to the anchor path.
%
% \begin{enumerate}
%
%    \item Use |\ctbnex| and |\ctbnex|, 
%       which give the north east corner of the corrected text box,
%       to calculate significant reference coordinates. 
%       (Normally similar to \S\ref{sec:anchorpathmacro} step \ref{item:apctb}).
%
%    \item Use |\mincorrect| to modify the reference coordinates to 
%       take account of the shape's minimum dimension keys, 
%       which given by |\pgfshapeminwidth| and |\pgfshapeminheight|.
%       (Normally the same as \S\ref{sec:anchorpathmacro} step \ref{item:apmindim}).
%
%    \item Modify the reference coordinates to take account of outer separation, 
%       which is given by |\pgfshapeouterxsep| and |\pgfshapeouterxsep|.
%       (Normally the same as \S\ref{sec:anchorpathmacro} step \ref{item:apouter}).
%       \label{item:sanchosep}
%
%    \item Use the reference coordinates from step \ref{item:sanchosep} to calculate the 
%       location of the anchor point, and assign values to the |\pgf@x| and |\pgf@y|
%       registers.
%
% \end{enumerate}
%
% \section{Making a Shape}
%
% \subsection{Helper Macros}\label{sec:macros}
%
% The following macros are available for use in the anchor path, 
% background path and saved anchors.
%
% \subsubsection{Corrected text box}\label{sec:corrctdTextBox}
%
% The \emph{corrected text box} is a bounding box for the 
% shape's text that includes the shape's {\tt inner} {\tt xsep} and {\tt inner} {\tt ysep} keys.
% The box is symmetric about the origin.
% Its x and y coordinates are given by the following commands:
%
% \begin{description} [itemsep=-0.5ex, labelindent=-0.5em]
% 
%    \item[] |\ctbnex|: 
%       The x-coordinate of the north east corner of the shape's corrected  text box.
%
%    \item[] |\ctbney|: 
%       The y-coordinate of the north east corner of the shape's 
%       corrected text box.
%
% \end{description}
%
% \subsubsection{Correcting for minimum dimensions}\label{sec:mincorrmacro}
% 
% A shape's size should be corrected for its 
% {\tt minimum} {\tt width} and {\tt minimum} {\tt height} keys. 
% The |\mincorrect| macro is provided to help with this.
%
% \begin{description} [itemsep=-0.5ex, labelindent=-0.5em]
% 
%    \item[] |\mincorrect{|{\it dimreg\/}|}{|{\it minkey\/}|}|:
%       Correct an x or y outer bound dimension for {\tt minimum} {\tt width} or 
%       {\tt minimum} {\tt height} keys.
%
%       \vspace{-1.5ex}
%       \begin{description}
%       
%          \item[{\it dimreg} \rm --] A dimension register.
%             On entry, it should hold the normal dimension of the shape. 
%             On exit, $dimreg$ is assigned the larger of its original value or $minkey/2$.
%
%          \item[{\it minkey} \rm--] A minimum dimension key value.
%             In practice, one of the minimum height or width commands given in
%             \S\ref{sec:keycommands} below.
%
%       \end{description}
%
% \end{description}
%
% \subsubsection{Getting key values}\label{sec:keycommands}
%
% The shape's key values can be accesses with the following commands:
%
% \begin{center}
% \begin{tabular}{ll}
% \hline
% key                        & command            \bigstrut\\\hline
% {\tt minimum} {\tt width}  & |\pgfshapeminwidth|  \bigstrut[t]\\  
% {\tt minimum} {\tt height} & |\pgfshapeminheight| \\
% {\tt outer} {\tt xsep}     & |\pgfshapeouterxsep| \\
% {\tt outer} {\tt ysep}     & |\pgfshapeouterysep| \bigstrut[b]\\\hline
% \end{tabular}
% \end{center}
%
% \noindent
% The outer separation keys are not relevant to the background path,
% but should be used by the anchor path and saved anchors.
%
% The shape's inner separation keys are automatically included in the \emph{corrected text box}
% coordinates |\ctbnex| and |\ctbney|.
%
% \subsubsection{Testing support}
%
% There are a couple of macros can be used in the background path to delineate a 
% shape's text box during development:
%\begin{description}[nosep]
% \item[{\tt \textbackslash path@textbox}] draws a line around the the shape's text 
%                                          that is the uncorrected text box.
% \item[{\tt \textbackslash path@ctextbox}] shows the shape's corrected text box, 
%                                           which includes inner separation.
% \end{description}
% These macros should not be used in `production' shapes.
%
% \subsection{Procedure}
%
% The procedure in outline  for making a new shape is as follows:
% \medskip
%
% \begin{enumerate}[nosep]
% \item Create the border path
% \item Create the anchor path
%\item Declare the shape
% \begin{enumerate}[nosep]
% \item Use |\setpaths| to employ the paths
% \item Create the anchors
% \end{enumerate}
% \end{enumerate}
% \medskip
%
% The sample shape given in \S\ref{sec:samplecode} can be used as a template to 
% create a new shape.
% The code can be copied from there, or it can be acquired as the file |sampleshape.tex|, 
% which is part of the package's distribution.
%
% The procedure for adapting the template code is as follows:
% \begin{enumerate}
%
%    \item Change the name of the shape in the |\pgfdeclareshape| command
%       from `{\tt sample}' to the name of the new shape.
%
%    \item \label{item:steppaths} Write \emph{anchor path} and \emph{background path} macros.
%
%       These and any sub macros are placed at the start of the file and replace
%       the sample |\sampleanchor| and |\sampleborder| macros.
%
%       As discussed in \S\ref{sec:backgroundpath} and \S\ref{sec:anchorpath}, 
%       the background path should take account of the minimum dimension keys, 
%       and the anchor path should handle the minimum keys, and outer 
%       separation. 
%       The macros given in \S\ref{sec:macros} are provided to help with this, 
%       and the sample code gives an indication of how they can be used.
%
%    \item Change the arguments of |\setpaths| in the template to the macros written 
%       in step \ref{item:steppaths}.
%       The first is the anchor path and the second is the background path.
%
%    \item \label{item:stepsavedanchors} Write the saved anchors that will needed by the anchors 
%        written in step~\ref{item:stepanchors}.
%
%       Saved anchors, like the anchor path written in step \ref{item:steppaths}, 
%       should correct for outer separation, and minimum dimensions
%       using the macros in \S\ref{sec:macros}.
%       They should set the |\pgf@x| and |\pgf@y| registers.
% 
%    \item \label{item:stepanchors} 
%        Write anchors that set |\pgf@x| and |\pgf@y| registers using
%              the saved anchors from step \ref{item:stepsavedanchors}. 
%
%       \begin{enumerate}
%      
%          \item Write anchors for the cardinal and sub cardinal points:
%
% \begin{VerbatimOut}[gobble=5]{\jobname.tmp}
%    \anchor{north}{ ... }
%    \anchor{north east}{ ... }
%    \anchor{east}{ ... }
%    \anchor{south east}{ ... }
%    \anchor{south}{ ... }
%    \anchor{south west}{ ... }
%    \anchor{west}{ ... }
%    \anchor{north west}{ ... }
% \end{VerbatimOut}
% \VerbatimInput{\jobname.tmp}
% 
%          \item Write any additional anchors that the shape needs.
%
%    \end{enumerate}
% 
% \end{enumerate}
%
% The PGF registers used throughout this method of declaring new shapes have the 
% |@| character in their names.  
% Unfortunately, this character has a special role in \LaTeX{}.
% So either the |\makeatletter| and |\makeatother| commands must be used as in 
% |sampleshape.tex|, or the shape must be defined in a |sty| file.
%
% \section{Path and Anchor Examples}
%
% Here an example of an actual shape that is implemented using the {\sf makeshape} method:
%
% \bigskip
%
% \begin{tikzpicture}%[>=latex',font={\sf \small}]
% \node at (0,0) [draw, sample,
%    minimum width=2cm,
%    minimum height=1cm] {};
% \end{tikzpicture}
% \bigskip
%
% Its background path,  anchor path, and anchor points use a common macro that gives 
% the space between the outer and inner boundaries of the shape.
%
% \begin{VerbatimOut}[gobble=5]{\jobname.tmp}
%    \def\gap{4pt}
% \end{VerbatimOut}
% \VerbatimInput{\jobname.tmp}
%
% \subsection{Background Path}\label{sec:backgroundpathexmpl}
%
% In this example, the background path macro is called {\tt sampleshape}.
% \begin{VerbatimOut}[gobble=5]{\jobname.tmp}
%    \def\sampleshape{
% \end{VerbatimOut}
% \VerbatimInput[numbers=left]{\jobname.tmp}
% First we get the corrected text box coordinates, 
% and add space for the shape using the |\gap| macro.
% \begin{VerbatimOut}[gobble=5]{\jobname.tmp}
%       \pgf@xa=\ctbnex
%       \pgf@ya=\ctbney
%       \advance\pgf@xa by \gap
%       \advance\pgf@ya by \gap
% \end{VerbatimOut}
% \VerbatimInput[numbers=left, firstnumber=2]{\jobname.tmp}
% Then the standard correction for minimum size is applied using the |\mincorrect| macro.
% \begin{VerbatimOut}[gobble=5]{\jobname.tmp}
%       \mincorrect{\pgf@xa}{\pgfshapeminwidth}
%       \mincorrect{\pgf@ya}{\pgfshapeminheight}
% \end{VerbatimOut}
% \VerbatimInput[numbers=left, firstnumber=6]{\jobname.tmp}
% Using these coordinates the outer boundary of the shape can be drawn
% \begin{VerbatimOut}[gobble=5]{\jobname.tmp}
%       \pgfpathmoveto{\pgfpoint{\pgf@xa}{\pgf@ya}}
%       \pgfpathlineto{\pgfpoint{\pgf@xa}{-\pgf@ya}}
%       \pgfpathlineto{\pgfpoint{-\pgf@xa}{-\pgf@ya}}
%       \pgfpathlineto{\pgfpoint{-\pgf@xa}{\pgf@ya}}
%       \pgfpathclose
% \end{VerbatimOut}
% \VerbatimInput[numbers=left, firstnumber=8]{\jobname.tmp}
% Finally, the reference coordinates are moved to the inner boundary and the shape is completed
% \begin{VerbatimOut}[gobble=5]{\jobname.tmp}
%       \advance\pgf@xa by -\gap
%       \advance\pgf@ya by -\gap
%       \pgfpathmoveto{\pgfpoint{\pgf@xa}{\pgf@ya}}
%       \pgfpathlineto{\pgfpoint{\pgf@xa}{-\pgf@ya}}
%       \pgfpathlineto{\pgfpoint{-\pgf@xa}{-\pgf@ya}}
%       \pgfpathlineto{\pgfpoint{-\pgf@xa}{\pgf@ya}}
%       \pgfpathclose
%    }
% \end{VerbatimOut}
% \VerbatimInput[numbers=left, firstnumber=13]{\jobname.tmp}
%
% \subsection{Anchor Path}\label{sec:anchorpathexmpl}
%
% The shape's anchor path macro called {\tt sampleanchor}, 
% and is similar to the path macro in \S\ref{sec:backgroundpathexmpl}, 
% but it also handles outer separation.
% \begin{VerbatimOut}[gobble=5]{\jobname.tmp}
%    \def\sampleanchor{
% \end{VerbatimOut}
% \VerbatimInput[numbers=left]{\jobname.tmp}
% First we get the corrected text box coordinates, and add space for the shape using 
% the |\gap| macro.
% \begin{VerbatimOut}[gobble=5]{\jobname.tmp}
%       \pgf@xa=\ctbnex
%       \pgf@ya=\ctbney
%       \advance\pgf@xa by \gap
%       \advance\pgf@ya by \gap
% \end{VerbatimOut}
% \VerbatimInput[numbers=left, firstnumber=2]{\jobname.tmp}
% Then corrections for minimum size and outer separation are applied:
% \begin{VerbatimOut}[gobble=5]{\jobname.tmp}
%       \mincorrect{\pgf@xa}{\pgfshapeminwidth}
%       \advance\pgf@xa\pgfshapeouterxsep
%       \mincorrect{\pgf@ya}{\pgfshapeminheight}
%       \advance\pgf@ya\pgfshapeouterysep
% \end{VerbatimOut}
% \VerbatimInput[numbers=left, firstnumber=6]{\jobname.tmp}
% Using these coordinates the anchor boundary path is finally drawn
% \begin{VerbatimOut}[gobble=5]{\jobname.tmp}
%       \pgfpathmoveto{\pgfpoint{\pgf@xa}{\pgf@ya}}
%       \pgfpathlineto{\pgfpoint{\pgf@xa}{-\pgf@ya}}
%       \pgfpathlineto{\pgfpoint{-\pgf@xa}{-\pgf@ya}}
%       \pgfpathlineto{\pgfpoint{-\pgf@xa}{\pgf@ya}}
%       \pgfpathclose
%    }
% \end{VerbatimOut}
% \VerbatimInput[numbers=left, firstnumber=10]{\jobname.tmp}
%
% \subsection{Shape Declaration and Anchor Points}
%
% The {\tt sample} shape declaration is started with
% \begin{VerbatimOut}[gobble=5]{\jobname.tmp}
%    \pgfdeclareshape{sample}{
% \end{VerbatimOut}
% \VerbatimInput[numbers=left]{\jobname.tmp}
% Then the shape and anchor paths, which were defined in \S\ref{sec:backgroundpathexmpl}
% and \S\ref{sec:anchorpathexmpl}, are employed with 
% \begin{VerbatimOut}[gobble=5]{\jobname.tmp}
%       \setpaths{\sampleanchor}{\sampleshape}
% \end{VerbatimOut}
% \VerbatimInput[numbers=left, firstnumber=2]{\jobname.tmp}
%
% \subsubsection{Anchors}
%
% Next in the shape's declaration are its saved anchors. 
% In this case, we establish the north east anchor location with code that is equivalent 
% to the first part of the anchor path macro (\S\ref{sec:anchorpathexmpl}). 
% Note how the anchor point's coordinate is returned in |\pgf@x| and |\pgf@y|.
% \begin{VerbatimOut}[gobble=5]{\jobname.tmp}
%       \savedanchor{\northeast}{
%          \pgf@x = \ctbnex
%          \pgf@y = \ctbney
%          \advance\pgf@xa by \gap
%          \advance\pgf@ya by \gap
%          \mincorrect{\pgf@x}{\pgfshapeminwidth}
%          \mincorrect{\pgf@y}{\pgfshapeminheight}
%          \advance\pgf@x\pgfshapeouterxsep
%          \advance\pgf@y\pgfshapeouterysep
%       }
% \end{VerbatimOut}
% \VerbatimInput[numbers=left, firstnumber=3]{\jobname.tmp}
%
% \noindent
% The saved anchor is then used to construct the shape's anchors, 
% and the declaration is ended as follows:
%
% \begin{VerbatimOut}[gobble=5]{\jobname.tmp}
%       \anchor{north}{\northeast \pgf@x=0pt}
%       \anchor{north east}{\northeast}
%       \anchor{east}{\northeast \pgf@y=0pt}
%       \anchor{south east}{\northeast \pgf@y=-\pgf@y}
%       \anchor{south}{\northeast \pgf@x=0pt \pgf@y=-\pgf@y}
%       \anchor{south west}{\northeast \pgf@x=-\pgf@x \pgf@y=-\pgf@y}
%       \anchor{west}{\northeast \pgf@x=-\pgf@x \pgf@y=0pt}
%       \anchor{north west}{\northeast \pgf@x=-\pgf@x}
%    }
% \end{VerbatimOut}
% \VerbatimInput[numbers=left, firstnumber=13]{\jobname.tmp}
%
%
% \subsection{Sample Shape Code}\label{sec:samplecode}
%
% The following is a full listing of the code needed to make the sample shape 
% described above.
% It can be used as a templete for new shapes.
% A copy of this code can be found in the the file {\tt sampleshape.tex}, 
% which is included in the package's distribution.
% Note the use of the |\makeatletter| and |\makeatother| commands, which are required 
% because the sample shape is in a |tex| file.
%
%^^A - Code must be < 67 characters to fit on page at 10pt ...
% \VerbatimInput[ ^^A numbers=left,
%                firstline=40, lastline=127]{sampleshape}
%
%
%^^A===========================================================================
%^^A                            References
%^^A===========================================================================
%
%^^A - Used in \StopEventually ...
%
% \def\printBib{
%   \begin{thebibliography}{9}
%   \raggedright
%
%   \bibitem{pgfMan}
%   Till Tantau, 
%   \emph{The TikZ and PGF Packages, Manual for version 2.10}, 
%   2010.
%   Available as 
%   \href{http://mirrors.ctan.org/graphics/pgf/base/doc/generic/pgf/pgfmanual.pdf}
%        {\tt pgfmanual.pdf}
%   from the \href{http://ctan.org}{Comprehensive TeX Archive Network}.
%
%^^A  in pgfMan \emph{Constructing Paths}, section 71, pp 579-589.
%^^A  in pgfMan \emph{Declaring New Shapes},  section 75.5, pp 625-631.
%
% \end{thebibliography}
% }
%
%
%^^A===========================================================================
%^^A                Start code implementation
%^^A===========================================================================
%
% \StopEventually{ \printBib }
%^^A \StopEventually{ \printBib \PrintChanges }
%^^A    -> Run pdfLaTeX makeshape.dtx
%^^A           makeindex -s gglo.ist -o makeshape.gls makeshape.glo
%^^A           pdfLaTeX makeshape.dtx
%
% \section{Implementation}
%
% \subsection{Text Box Calculations}\label{sec:textboxcalc}
% 
% The shape's text is at its centre for shape's made with this package.
% The text box position is defined by the shape's |text| anchor, which must give the coordinate 
% of the west end of the text's base line.
% PGF provides three useful dimensions:
%
% \smallskip
% \begin{tabular}{lll}
% |\wd\pgfnodeparttextbox| & width & $wd$\\
% |\ht\pgfnodeparttextbox| & height & $ht$\\
% |\dp\pgfnodeparttextbox| & depth & $dp$\\
% \end{tabular}
%
% \smallskip\noindent
% These are related as follows:
% \medskip
%
%^^A----------------------------------
%^^A- Text box diagram - start
%^^A----------------------------------
% \begin{tikzpicture}[>=latex',font={\sf \small}]
%
% \def\tbht{30pt}         ^^A - text box height
% \def\tbdp{15pt}         ^^A - text box depth
% \def\tbwd{150pt}        ^^A - text box width
% 
% \def\blwx{-\tbwd/2}           ^^A - baseline west x
% \def\blwy{-\tbht/2+\tbdp/2}   ^^A - baseline west y
% 
% \def\tbnex{\tbwd/2}            ^^A - text box north east x
% \def\tbney{\tbht/2+\tbdp/2}    ^^A - text box north east y
%
% \node (textbox) at (0,0) [draw, rectangle, fill=white,
%    minimum width=\tbwd,
%    minimum height=\tbht+\tbdp,
% ] {};
% \draw [] (\blwx,\blwy) -- ++(\tbwd,0);
%
% \node (textanchor) [coordinate] at (\blwx,\blwy){};
%
% \node at (textanchor) [draw, circle] (blinewest) {};
% 
%^^A - Labels
% \node (blinelable) at (3.5cm,\blwy) [inner sep = 0] {baseline};
% \node (textanchlabel) at (\blwx+28pt,\blwy+8pt) [inner sep = 0] {text anchor};
% 
%^^A - Dimensions
% \node (hta) [coordinate, left=15pt of textbox.south west] {};
% \node (htc) [coordinate, left=15pt of textbox.north west] {};
% \node (htb) [coordinate, left=15pt of  textanchor] {};
% \draw [|<->|] (hta) -- (htb) node [midway,left] {depth};
% \draw [<->|] (htb) -- (htc) node [midway,left] {height};
% \node (wda) [coordinate, above=10pt of textbox.north west] {};
% \node (wdb) [coordinate, above=10pt of textbox.north east] {};
% \draw [|<->|] (wda) -- (wdb) node [midway,above] {width};
% 
%^^A - xy axis
% \def\xaxisdim{0.5cm}
% \def\yaxisdim{0.5cm}
% \draw [dotted] (0,0) -- ++(\xaxisdim,0);    ^^A - origin
% \draw [dotted] (0,0) -- ++(-\xaxisdim,0);   ^^A - origin
% \draw [dotted] (0,0) -- ++(0,\yaxisdim);    ^^A - origin
% \draw [dotted] (0,0) -- ++(0,-\yaxisdim);   ^^A - origin
% 
% \end{tikzpicture}
%^^A----------------------------------
%^^A-   Text box diagram - end
%^^A----------------------------------
%\medskip
% 
% For this box to be centred on the origin, we need to calculate suitable coordinates for 
% the |text| anchor.
% So, if the box's south west corner is at $(x_{sw},y_{sw})$ and the box is centred, then
% $x_{sw} = -wd/2$ and $y_{sw}=-(ht+dp)/2$.
% If the |text| anchor is at $(x_{text},y_{text})$, then
% \vspace{-2ex}
%\begin{eqnarray}
% x_{text} & = & x_{sw} = -\frac{wd}{2}\label{eqn:xtext}\\
% y_{text} & = & y_{sw} + dp = -\frac{ht}{2} - \frac{dp}{2} + dp  
%            = \frac{dp}{2} - \frac{ht}{2} \label{eqn:ytext}
%\end{eqnarray}
%
% Furthermore, the shape's north east corner $(x_{ne},y_{ne})$ is at
%\begin{eqnarray}
% x_{ne} & = & \frac{wd}{2} \label{eqn:xne}\\[1ex]
% y_{ne} & = & \frac{ht+dp}{2} \label{eqn:yne}
%\end{eqnarray}
%
%\iffalse 
%<*package> 
%\fi
%
%\subsection{Symetric bounding box}\label{sec:symbbox}
%
% We require a rectangle that contains the anchor path, and has its centre at the origin.
% Such a rectangle, defined by the coordinate of its north east corner, is needed 
% in the anchor border algorithm described in \S\ref{sec:anchoralgorithm}.
%
% \vspace{2ex}
%
%^^A----------------------------------
%^^A- Bounding box diagram - start
%^^A----------------------------------
% \begin{tikzpicture}[>=latex',font={\sf \small}]
%
% \def\bbnex{60pt}         ^^A - box ne x 
% \def\bbney{30pt}         ^^A - box ne y
% 
% \node (boundingbox) at (0,0) [draw, rectangle, dashed,
%    minimum width=2*\bbnex,
%    minimum height=2*\bbney,
% ] {};
%
% \node at (\bbnex,\bbney) [draw, circle] (boxne) {};
%
% \def\nx{0pt}   \def\ny{20pt}
% \def\ex{30pt}  \def\ey{-10pt}
% \def\sx{-20pt}   \def\sy{-\bbney}
% \def\wx{-\bbnex} \def\wy{10pt}
%
%^^A - Anchor path
% \draw [] (\nx,\ny) -- (\ex,\ey);
% \draw [] (\ex,\ey) -- (\sx,\sy);
% \draw [] (\sx,\sy) -- (\wx,\wy);
% \draw [] (\wx,\wy) -- (\nx,\ny);
%
%^^A - mins and maxs with labels
% \draw [dotted] (\nx,\ny) -- (\bbnex+10pt,\ny);
% \draw [dotted] (\bbnex,\sy) -- (\bbnex+10pt,\sy);
% \draw [dotted] (\ex,\ey) -- (\ex,-\bbney-10pt);
% \draw [dotted] (\wx,-\bbney) -- (\wx,-\bbney-10pt);
% \node (maxy) at (\bbnex+24pt,\ny) [inner sep = 0] {$y_{max}$};
% \node (miny) at (\bbnex+24pt,\sy) [inner sep = 0] {$y_{min}$};
% \node (maxx) at (\wx,-\bbney-16pt) [inner sep = 0] {$x_{min}$};
% \node (maxx) at (\ex,-\bbney-16pt) [inner sep = 0] {$x_{max}$};%
% 
%^^A - Labels
% \node (path) at (\wx-40pt,\wy-32pt) [inner sep = 0] {anchor path};
% \draw [->] (path.east) -- (-40pt,-10pt);
% \node (box) at (\wx-40pt,\wy+13pt) [inner sep = 0] {bounding box};
% \draw [->] (box.east) -- (\wx,23pt);
% \node (bbne) at (\bbnex+25pt,\bbney+10pt) [inner sep = 0] {$(x_{ne},y_{ne})$};
%
%^^A - xy axis
% \def\xaxisdim{0.2cm}
% \def\yaxisdim{0.2cm}
% \draw [] (0,0) -- ++(\xaxisdim,0);    ^^A - origin
% \draw [] (0,0) -- ++(-\xaxisdim,0);   ^^A - origin
% \draw [] (0,0) -- ++(0,\yaxisdim);    ^^A - origin
% \draw [] (0,0) -- ++(0,-\yaxisdim);   ^^A - origin
% 
% \end{tikzpicture}
%^^A----------------------------------
%^^A- Bounding box diagram - end
%^^A----------------------------------
%
% \medskip
% The \emph{symmetric bounding box} is defined by the coordinate of its north east corner:
% \vspace{-3ex}
% \begin{eqnarray}
%    x_{ne} &=& \max{(\vert x_{min} \vert \, ,\vert x_{max} \vert )} \label{eqn:bbx}\\ 
%    y_{ne} &=& \max{(\vert y_{min} \vert \, ,\vert y_{max} \vert )} \label{eqn:bby}
% \end{eqnarray}
% 
% \subsection{The Anchor Border Algorithm}\label{sec:anchoralgorithm}
% 
% A shape's |\anchorborder| command must calculate the intersection  of a line drawn 
% between the origin and a target point, and the shape's border.
% The aim here is to do this calculation using an \emph{anchor path} that describes 
% the shapes outer border.
% 
% The  |\pgfintersectionofpaths| command can be used to do this, but it will only work 
% for points that are outside the anchor path.
% The solution is to \emph{externalise the point} using the |\pgfpointborderrectangle| command.
%
% \bigskip
%^^A------------------------------------------------
%^^A- Anchor border point algorithm diagram - start
%^^A------------------------------------------------
% \begin{tikzpicture}[>=latex',font={\sf \small}]
%
%^^A - xy axis
% \def\xaxisdim{0.2cm}
% \def\yaxisdim{0.2cm}
% \draw [] (0,0) -- ++(-\xaxisdim,0);   ^^A - origin
% \draw [] (0,0) -- ++(0,\yaxisdim);    ^^A - origin
% \draw [] (0,0) -- ++(0,-\yaxisdim);   ^^A - origin
% \filldraw[white] (0,0) circle (2pt);
% 
%^^A - Bounding box
% \def\bbnex{60pt}         ^^A - box ne x 
% \def\bbney{30pt}         ^^A - box ne y
% 
% \node (boundingbox) at (0,0) [draw, rectangle, dashed,
%    minimum width=2*\bbnex,
%    minimum height=2*\bbney,
% ] {};
%
% \draw ((\bbnex,\bbney) circle (2pt);
%
%^^A - Anchor path
% \def\nx{30pt}     \def\ny{20pt}
% \def\ex{50pt}    \def\ey{-20pt}
% \def\sx{-20pt}   \def\sy{-\bbney}
% \def\wx{-\bbnex} \def\wy{10pt}
%
% \def\apath{
%    \pgfpathmoveto{\pgfpoint{\nx}{\ny}}
%    \pgfpathlineto{\pgfpoint{\ex}{\ey}}
%    \pgfpathlineto{\pgfpoint{\sx}{\sy}}
%    \pgfpathlineto{\pgfpoint{\wx}{\wy}}
%    \pgfpathclose }
%
%^^A - Draw the anchor path
% \apath
% \pgfusepath{stroke}
%
%^^A - Target point and vectors
% \def\tx{20pt}
% \def\ty{5pt}
% \def\endx{3.5*\tx}
% \def\endy{3.5*\ty}
%
% \draw [] (0,0) -- (\tx,\ty);
% \draw [dotted] (\tx,\ty) -- (\endx,\endy);
% \filldraw (\tx,\ty) circle (2pt);
%
%^^A - Intersection with bounding box
% \pgfpathcircle{\pgfpointborderrectangle{\pgfpoint{\tx}{\ty}}
%                              {\pgfpoint{\bbnex}{\bbney}}}{2pt}
% \pgfusepath{stroke}
%
%^^A - intersection with anchor path
% \pgfintersectionofpaths{\apath}
%       { \pgfpathmoveto{\pgfpoint{0}{0}}
%         \pgfpathlineto{\pgfpoint{\endx}{\endy}} }
% \pgfpathcircle{\pgfpointintersectionsolution{1}}{2pt}
% \pgfusepath{stroke}
% 
%^^A - Labels
% \node (path) at (\wx-40pt,\wy-32pt) [inner sep = 0] {anchor path};
% \draw [->] (path.east) -- (-40pt,-10pt);
% \node (box) at (\wx-40pt,\wy+13pt) [inner sep = 0] {bounding box};
% \draw [->] (box.east) -- (\wx,23pt);
% \node (p1) at (\tx+5pt,\ty-8pt) [inner sep = 0] {p1};
% \node (p2) at (\bbnex+10pt,\bbney+5pt) [inner sep = 0] {p2};
% \node (p3) at (\bbnex+10pt,8pt) [inner sep = 0] {p3};
% \node (p4) at (40pt,18pt) [inner sep = 0] {p4};
%
%^^A - point key
%^^A \def\pkeyx{90pt}
%^^A \def\pkeyy{10pt}
%^^A \def\pkeystep{-10pt}
%^^A \node (key1) at (\pkeyx,\pkeyy) [right, inner sep = 0] {p1 = target point};
%^^A \node (key2) at (\pkeyx,\pkeyy+\pkeystep) [right, inner sep = 0] {p2 = bounding box corner};
%^^A \node (key3) at (\pkeyx,\pkeyy+2*\pkeystep) [right, inner sep = 0] 
%^^A     {p3 = bounding box intersection};
%^^A \node (key4) at (\pkeyx,\pkeyy+3*\pkeystep) [right, inner sep = 0] 
%^^A     {p4 = anchor path intersection};
%
% \end{tikzpicture}
%^^A------------------------------------------------
%^^A- Anchor border point algorithm diagram - end
%^^A------------------------------------------------
%
%\bigskip
%
%\begin{description}
%
% \item[Step 1 - Externalise the target point]
%
% The vector to the target point {\sf p1} is extended to the bounding box to find 
% the intersection {\sf p3}.
% The bounding box is by definition outside the anchor path.  
% Thus point {\sf p3} must also be outside the anchor path. 
%
% To achieve this, the |\pgfpointborderrectangle| command is applied to the target point {\sf p1} 
% and the centred rectangle defined by {\sf p2}, which is the symmetric bounding box defined 
% in \S\ref{sec:symbbox}.
%
% \item[Step 2 - Find the border intersection]
%
% The line from the origin to the point {\sf p3} passes through {\sf p1}, 
% and its intersection with the anchor path is the required point {\sf p4}.
%
% The intersection is found with the |\pgfintersectionofpaths| command, 
% which takes the anchor path and the path of the line from the origin to {\sf p3} as input.
% Since {\sf p3} is always outside the anchor path, such an intersection will always be found.
%
% \end{description}
%
% \subsection{Declare Shape Commands}
%
% \begin{macro}{\setpaths}
%    The |\setpaths| macro is used in |\pgfdeclareshape| definitions to specify the shape's paths.
%    It has two parameters:
%    \begin{description}[nosep]
%       \item[\tt ~\#1] is a macro defining the anchor path.
%       \item[\tt ~\#2] is a macro defining the background path.
%    \end{description}
%
%    \begin{macrocode}
\def\setpaths#1#2{
%    \end{macrocode}
%
% \noindent
% It defines saved macros for use in the shape's path macros and saved anchors.
% There are also macros defined that are only used within the packages other macros.
%
% \subsubsection{Corrected text box saved macros}
%
% These two saved macros give the x and y coordinates of the corrected text box's north east corner.
% They are invoked by users of the package to define a shape's path macros.
%
% \begin{macro}{\ctbnex}
%    The |\ctbnex| saved macro gives the x-coordinate,  
%    which is calculated as described by equation \ref{eqn:xne} in \S\ref{sec:textboxcalc}. 
%    \begin{macrocode}
   \savedmacro\ctbnex{
      \pgf@x=.5\wd\pgfnodeparttextbox
      \pgfmathsetlength\pgf@xa{\pgfshapeinnerxsep}
      \advance\pgf@x\pgf@xa
      \edef\temp@x{\the\pgf@x}
      \let\ctbnex\temp@x
   }
%    \end{macrocode}
% \end{macro}
%
% \begin{macro}{\ctbney}
%    The y-coordinate is given by the |\ctbney| saved macro, 
%    which implements equation \ref{eqn:yne}  in \S\ref{sec:textboxcalc}.  
%    \begin{macrocode}
   \savedmacro\ctbney{
      \pgf@y=.5\ht\pgfnodeparttextbox
      \advance\pgf@y.5\dp\pgfnodeparttextbox
      \pgfmathsetlength\pgf@ya{\pgfshapeinnerysep}
      \advance\pgf@y\pgf@ya
      \edef\temp@y{\the\pgf@y}
      \let\ctbney\temp@y
   }
%    \end{macrocode}
% \end{macro}
%
% \subsubsection{Key value saved macros}
%
% These saved macros give the values of significant shape keys. 
% They have the same names as the standard PGF macros,
% and allow the user's anchor and background path macros to use the standard names.
%
% \begin{macro}{\pgfshapeouterxsep}
%    The value of the x outer separation key is given by |\pgfshapeouterxsep|.
%    \begin{macrocode}
   \savedmacro\pgfshapeouterxsep{
      \pgfkeysgetvalue{/pgf/outer xsep}{\temp@sep}
      \let\pgfshapeouterxsep\temp@sep
   }
%    \end{macrocode}
% \end{macro}
%
% \begin{macro}{\pgfshapeouterysep}
%    The value of the y outer separation key is given by |\pgfshapeouterysep|.
%    \begin{macrocode}
   \savedmacro\pgfshapeouterysep{
      \pgfkeysgetvalue{/pgf/outer ysep}{\temp@sep}
      \let\pgfshapeouterysep\temp@sep
   }
%    \end{macrocode}
% \end{macro}
%
% \begin{macro}{\pgfshapeminwidth}
%    The value of the minimum width key is given by |\pgfshapeminwidth|.
%    \begin{macrocode}
   \savedmacro{\pgfshapeminwidth}{
      \pgfkeysgetvalue{/pgf/minimum width}{\temp@wd}
      \let\pgfshapeminwidth\temp@wd
   }
%    \end{macrocode}
% \end{macro}
%
% \begin{macro}{\pgfshapeminheight}
%    The value of the minimum height key is given by |\pgfshapeminheight|.
%    \begin{macrocode}
   \savedmacro{\pgfshapeminheight}{
      \pgfkeysgetvalue{/pgf/minimum height}{\temp@ht}
      \let\pgfshapeminheight\temp@ht
   }
%    \end{macrocode}
% \end{macro}
%
% \subsubsection{Saved anchors}\label{sec:savedanchors}
% 
% The saved anchors are used to implement the anchors provided by the package,
% but can be used by the user to define custom anchors if required.
% They are required to return the coordinates of a point in |\pgf@x| and |\pgf@y|.
%
% \begin{macro}{\centre}
%    The |\centre| saved anchor is used to implement the package |center| anchor. 
%    The |\pgfpointorigin| command sets |\pgf@x| and |\pgf@y|.
%    \begin{macrocode}
   \savedanchor{\centre}{
      \pgfpointorigin
   }
%    \end{macrocode}
% \end{macro}
%
% \begin{macro}{\text}
%    The |\text| saved anchor is used to implement the package |text| anchor. 
%    It gives the coordinate of the west end of the text base line.
%    The calculation it uses is explained in \S\ref{sec:textboxcalc}.
%    Equation \ref{eqn:xtext} gives the x-coordinate 
%    and equation \ref{eqn:ytext} gives the y-coordinate.
%    \begin{macrocode}
   \savedanchor{\text}{
      \pgf@y=-0.5\ht\pgfnodeparttextbox
      \advance\pgf@y by 0.5\dp\pgfnodeparttextbox
      \pgf@x=-0.5\wd\pgfnodeparttextbox
   }
%    \end{macrocode}
% \end{macro}
%
% \begin{macro}{\boundbox@ne}
%    The |\boundbox@ne| saved anchor gives the north east corner of the 
%    anchor path's symmetric bounding rectangle (see \S\ref{sec:symbbox}).
%    \begin{macrocode}
   \savedanchor{\boundbox@ne}{
%    \end{macrocode}
%
%    First clear any existing path, and then execute the anchor path macro given as 
%    the first parameter of |\setpaths|.
%    \begin{macrocode} 
      \pgfusepath{}
      #1
%    \end{macrocode}
%    Store the path's minimum and maximum coordinates, and then discard the path. 
%    \begin{macrocode}
      \pgf@xb=\pgf@pathmaxx
      \pgf@yb=\pgf@pathmaxy
      \pgf@xc=\pgf@pathminx
      \pgf@yc=\pgf@pathminy
      \pgfusepath{}
%    \end{macrocode}
%
% The north east corner x-coordinate of the symmetric bounding box is calculated using 
% equation \ref{eqn:bbx} from \S\ref{sec:symbbox}.
%
%    \begin{macrocode}
      \ifdim\pgf@xb<0pt \pgf@xb=-\pgf@xb \fi
      \ifdim\pgf@xc<0pt \pgf@xc=-\pgf@xc \fi
      \ifdim\pgf@xb<\pgf@xc
         \pgf@x=\pgf@xc
      \else
         \pgf@x=\pgf@xb
      \fi
%    \end{macrocode}
% The y-coordinate is calculated with equation \ref{eqn:bby}:
%    \begin{macrocode}
      \ifdim\pgf@yb<0pt \pgf@yb=-\pgf@yb \fi
      \ifdim\pgf@yc<0pt \pgf@yc=-\pgf@yc \fi
      \ifdim\pgf@yb<\pgf@yc
         \pgf@y=\pgf@yc
      \else
         \pgf@y=\pgf@yb
      \fi
   }
%    \end{macrocode}
% \end{macro}
%
% \subsubsection{Anchors} 
%
% \begin{macro}{center}
% \begin{macro}{text}
%    PGF shapes require |center| and |text| anchors. 
%    They are implemented by the saved anchors |\center| and |\text| (see \ref{sec:savedanchors}), 
%    and in effect centre the shape's text on the origin. 
%    \begin{macrocode}
   \anchor{center}{\centre}
   \anchor{text}{\text}
%    \end{macrocode}
% \end{macro}
% \end{macro}
%
% \begin{macro}{\anchorborder}
%    The |\anchorborder| anchor is critically responsible for calculating points on 
%    the shape's boundary
%    Its algoritm is explained in \S\ref{sec:anchoralgorithm}.
%    \begin{macrocode}
   \anchorborder{
%    \end{macrocode}
%    On entry, the coordinates of the target point are supplied in |\pgf@x| and |\pgf@y|, 
%    and these are immediately saved:
%    \begin{macrocode}
      \edef\targpointx{\the\pgf@x}
      \edef\targpointy{\the\pgf@y}
%    \end{macrocode}
%
%    A couple of settings are needed to support shapes that are not centred on the origin:
%    \begin{macrocode}
      \pgf@relevantforpicturesizefalse
      \pgftransformreset
%    \end{macrocode}
%    The first suppresses updating of the picture bounding box and ensures correct positioning 
%    of the shape.
%    The second stops coordinate transformations being applied, 
%    which is needed for the correct behaviour of points relying on the |\anchorborder| command.
%
%    Start the computation by finding the north east corner of the anchor path's bounding box:
%    \begin{macrocode}
      \boundbox@ne
      \pgf@xa=\pgf@x
      \pgf@ya=\pgf@y
%    \end{macrocode}
%    Perform step 1 of the algorithm described in \S\ref{sec:anchoralgorithm} to externalise
%    the target to a point on the bounding box:
%    \begin{macrocode}
      \pgfpointborderrectangle{\pgfpoint{\targpointx}{\targpointy}}
                              {\pgfpoint{\pgf@xa}{\pgf@ya}}
      \edef\corrx{\the\pgf@x}
      \edef\corry{\the\pgf@y}
%    \end{macrocode}
%    Next, use this externalised point to find the required boundary point with step 2 of 
%    the algorithm.
%    The anchor path is |#1|, which is the first argument |setpaths|. 
%    \begin{macrocode}
      \pgfintersectionofpaths {
         #1
      }{
         \pgfpathmoveto{\pgfpoint{0}{0}}
         \pgfpathlineto{\pgfpoint{\corrx}{\corry}}
      }
%    \end{macrocode}
%    Finally, the command |\pgfpointintersectionsolution| puts the point's coordinates in 
%    |\pgf@x| and |\pgf@y|. 
%    \begin{macrocode}
      \pgfpointintersectionsolution{1}
   }
%    \end{macrocode}
% \end{macro}
%
% \subsubsection{Background path}
%
% \begin{macro}{\backgroundpath}
%    The shape is actually drawn by |\backgroundpath|, which executes the second argument of 
%    |\setpaths|, which is the shape's background path.
%    \begin{macrocode}
   \backgroundpath{
      #2
   }
%    \end{macrocode}
% \end{macro}
%    Finally, |setpaths| is completed.
%    \begin{macrocode}
}
%    \end{macrocode}
% \end{macro}
%
% \subsection{Utility macros}
%
% \subsubsection{Dimension correction}
%
% \begin{macro}{\mincorrect}
%     The |\mincorrect| macro finds the larger of two dimensions.
%     It is applied to shape coordinates and keys. 
%     It uses a working dimension register, and has two parameters  
%    \begin{macrocode}
\newdimen\crct@dim
\def\mincorrect#1#2{
%    \end{macrocode}
% The parameters have the following roles:  
%     \begin{description}[nosep]
%        \item[\tt ~\#1] A register holding the normal outer bound dimension of the shape, 
%           and then the largest value, in a PGF register such as |\pgf@xa|. 
%        \item[\tt ~\#2] A key value, normally one of |\pgfshapeminwidth| or |\pgfshapeminheight|.
%     \end{description}
%     The |\crct@dim| register is used to calculate half of |#2|:
%    \begin{macrocode}
   \pgfmathsetlength\crct@dim{#2}
   \pgfmathsetlength\crct@dim{0.5\crct@dim}
%    \end{macrocode}
%
%    Then if |#1| is larger than |\crct@dim|, |#1| is assigned this value 
%    otherwise |#1| remains unchanged.
%    \begin{macrocode}
   \ifdim#1<\crct@dim
      #1=\crct@dim
   \fi
}
%    \end{macrocode}
% \end{macro}
%
% \subsubsection{Macros for testing} 
%
% There are a couple of macros can be used to delineate a shape's text box during development.
% They should not be used in `production' shapes.
% 
% \begin{macro}{\path@textbox}
% The |\path@textbox| macro draws a line around the the shape's text that is the 
% uncorrected text box.
%    \begin{macrocode}
\def\path@textbox{
   \pgf@xa=.5\wd\pgfnodeparttextbox
   \pgf@ya=.5\ht\pgfnodeparttextbox
   \advance\pgf@ya.5\dp\pgfnodeparttextbox
   \typeout{**** TEST text box (x,y) = (\the\pgf@xa,\the\pgf@ya) }
   \pgfpathmoveto{\pgfpoint{\pgf@xa}{\pgf@ya}}
   \pgfpathlineto{\pgfpoint{\pgf@xa}{-\pgf@ya}}
   \pgfpathlineto{\pgfpoint{-\pgf@xa}{-\pgf@ya}}
   \pgfpathlineto{\pgfpoint{-\pgf@xa}{\pgf@ya}}
   \pgfpathclose
}
%    \end{macrocode}
% \end{macro}
%
% \begin{macro}{\path@ctextbox}
% The |\path@ctextbox| macro shows the shape's corrected text box, 
% which includes inner separation.
%    \begin{macrocode}
\def\path@ctextbox{
   \pgf@xa=.5\wd\pgfnodeparttextbox
   \pgfmathsetlength\pgf@xc{\pgfshapeinnerxsep}
   \advance\pgf@xa\pgf@xc
   \pgf@ya=.5\ht\pgfnodeparttextbox
   \advance\pgf@ya.5\dp\pgfnodeparttextbox
   \pgfmathsetlength\pgf@yc{\pgfshapeinnerysep}
   \advance\pgf@ya\pgf@yc
   \typeout{**** TEST corrected text box (x,y) = 
                           (\the\pgf@xa,\the\pgf@ya) }
   \pgfpathmoveto{\pgfpoint{\pgf@xa}{\pgf@ya}}
   \pgfpathlineto{\pgfpoint{\pgf@xa}{-\pgf@ya}}
   \pgfpathlineto{\pgfpoint{-\pgf@xa}{-\pgf@ya}}
   \pgfpathlineto{\pgfpoint{-\pgf@xa}{\pgf@ya}}
   \pgfpathclose
}
%    \end{macrocode}
% \end{macro}
%
%\iffalse 
%</package>
%\fi
%
% \Finale
%
%^^A###########################################################################
%^^A#                      Additional installed files                         #
%^^A###########################################################################
%
%^^A - Option = sample ...
%^^A - This is the package's sample shape.  
%^^A - It is included in the report, and it is given as a file to be used as 
%^^A -    a template for new shapes.
%\iffalse
%<*sample>
%%--------------------------------------------------
%%
%% This is a sample shape for the makeshape package
%%
%% It can be adapted for other shapes. See
%% makeshape.pdf for instructions.
%%
%% 25 January 2013

\usepackage{makeshape}
\makeatletter

%% Constant for sample shape
\def\gap{4pt}

%% Anchor path <- change and rename as required
\def\sampleanchor{
   % get corrected text box
   \pgf@xa=\ctbnex
   \pgf@ya=\ctbney
   % make room for shape
   \advance\pgf@xa by \gap
   \advance\pgf@ya by \gap
   % correct for minheight and minwidth and 
   % for outerxsep or outerysep
   \mincorrect{\pgf@xa}{\pgfshapeminwidth}
   \advance\pgf@xa\pgfshapeouterxsep
   \mincorrect{\pgf@ya}{\pgfshapeminheight}
   \advance\pgf@ya\pgfshapeouterysep
   % draw the path
   \pgfpathmoveto{\pgfpoint{\pgf@xa}{\pgf@ya}}
   \pgfpathlineto{\pgfpoint{\pgf@xa}{-\pgf@ya}}
   \pgfpathlineto{\pgfpoint{-\pgf@xa}{-\pgf@ya}}
   \pgfpathlineto{\pgfpoint{-\pgf@xa}{\pgf@ya}}
   \pgfpathclose
}

%% Background path <- change and rename as required
\def\sampleborder{
   % get corrected text box
   \pgf@xa=\ctbnex
   \pgf@ya=\ctbney
   % make room for shape
   \advance\pgf@xa by \gap
   \advance\pgf@ya by \gap
   % correct for minheight and minwidth but 
   %    not for outerxsep or outerysep
   \mincorrect{\pgf@xa}{\pgfshapeminwidth}
   \mincorrect{\pgf@ya}{\pgfshapeminheight}
   % draw outer shape
   \pgfpathmoveto{\pgfpoint{\pgf@xa}{\pgf@ya}}
   \pgfpathlineto{\pgfpoint{\pgf@xa}{-\pgf@ya}}
   \pgfpathlineto{\pgfpoint{-\pgf@xa}{-\pgf@ya}}
   \pgfpathlineto{\pgfpoint{-\pgf@xa}{\pgf@ya}}
   \pgfpathclose
   % draw inner shape
   \advance\pgf@xa by -\gap
   \advance\pgf@ya by -\gap
   \pgfpathmoveto{\pgfpoint{\pgf@xa}{\pgf@ya}}
   \pgfpathlineto{\pgfpoint{\pgf@xa}{-\pgf@ya}}
   \pgfpathlineto{\pgfpoint{-\pgf@xa}{-\pgf@ya}}
   \pgfpathlineto{\pgfpoint{-\pgf@xa}{\pgf@ya}}
   \pgfpathclose
}

%%------------------------------------------------
%% Shape declaration <- Change name as required
\pgfdeclareshape{sample}{

   % Set paths <- change path macros as required
   \setpaths{\sampleanchor}{\sampleborder} 
                                           
   % Saved anchors <- change as required
   \savedanchor{\northeast}{
      \pgf@x = \ctbnex
      \pgf@y = \ctbney
      \advance\pgf@x by \gap
      \advance\pgf@y by \gap
      \mincorrect{\pgf@x}{\pgfshapeminwidth}
      \mincorrect{\pgf@y}{\pgfshapeminheight}
      \advance\pgf@x\pgfshapeouterxsep
      \advance\pgf@y\pgfshapeouterysep
   }

   % Anchors <- change as required
   \anchor{north}{\northeast \pgf@x=0pt}
   \anchor{north east}{\northeast}
   \anchor{east}{\northeast \pgf@y=0pt}
   \anchor{south east}{\northeast \pgf@y=-\pgf@y}
   \anchor{south}{\northeast \pgf@x=0pt \pgf@y=-\pgf@y}
   \anchor{south west}{\northeast \pgf@x=-\pgf@x \pgf@y=-\pgf@y}
   \anchor{west}{\northeast \pgf@x=-\pgf@x \pgf@y=0pt}
   \anchor{north west}{\northeast \pgf@x=-\pgf@x}

}

\makeatother

%</sample>
%\fi
%
\endinput

