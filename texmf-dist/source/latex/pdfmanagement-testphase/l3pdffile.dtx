% \iffalse meta-comment
%
%% File: l3pdffile.dtx
%
% Copyright (C) 2018-2021 The LaTeX Project
%
% It may be distributed and/or modified under the conditions of the
% LaTeX Project Public License (LPPL), either version 1.3c of this
% license or (at your option) any later version.  The latest version
% of this license is in the file
%
%    http://www.latex-project.org/lppl.txt
%
% This file is part of the "LaTeX PDF management testphase bundle" (The Work in LPPL)
% and all files in that bundle must be distributed together.
%
% -----------------------------------------------------------------------
%
% The development version of the package can be found at
%
%    https://github.com/latex3/pdfresources
%
% for those people who are interested.
%
%<*driver>
\RequirePackage{pdfmanagement-testphase}
\DeclareDocumentMetadata{pdfstandard=A-2b}
\makeatletter
\declare@file@substitution{doc.sty}{doc-v3beta.sty}
\makeatother
\documentclass[full]{l3doc}
\usepackage{array,booktabs,hyperxmp}
\hypersetup{pdfauthor=The LaTeX Project,pdftitle=l3pdffile (LaTeX PDF management testphase bundle)}
\usepackage{caption}
\providecommand\potentialclash{\noindent\llap{\dbend\ }}
\begin{document}
  \DocInput{\jobname.dtx}
\end{document}
%</driver>
% \fi
%
% \title{^^A
%   The \pkg{l3pdffile} module\\ Embedding and referencing files in a PDF ^^A
%   \\ \LaTeX{} PDF management testphase bundle
% }
%
% \author{^^A
%  The \LaTeX{} Project\thanks
%    {^^A
%      E-mail:
%        \href{mailto:latex-team@latex-project.org}
%          {latex-team@latex-project.org}^^A
%    }^^A
% }
%
% \date{Version 0.95i, released 2021-08-28}
%
% \maketitle
% \begin{documentation}
%
% \section{\pkg{l3pdffile} documentation}
% \subsection{Introduction}
% \subsubsection{Background}
% External files can be referenced from a PDF in three ways:
% \begin{enumerate}
% \item through an annotation of type Link,
% \item by referencing a local file in the file system,
% \item by embedding the file directly into the PDF
% \end{enumerate}
% Case 1 (Links) are created with the \cs{pdfannot} commands.
% This module handles the two other cases. Actually from the view
% of the PDF format they are quite similar: Case 2 is case 3 without the stream
% object and without the /EF entry in the /Filespec dictionary (this points to the
% stream object of the file). Not embedding the file makes the PDF smaller. But it is
% also less portable: the files can only be found if they are in the right location
% relative to the PDF. The normal case is to embed the file.
%
% The tasks to embed and reference such a file are
% \begin{enumerate}
% \item Embed the file in a stream.
% \item Create a Filespec dictionary which references the stream object in the /EF
% dictionary:
% \begin{verbatim}
%  <<
%   /Type /Filespec
%   /F  (l3pdffile.dtx)
%   /UF (l3pdffile.dtx)
%   /AFRelationship /Source
%   /EF <</F 21 0 R /UF 21 0 R>>   %case 3, embedded file
%  >>
% \end{verbatim}
% The file names in the /UF and /F value don't need to be identical to the
% name of file on the disc. It is quite possible to embed a \texttt{zzz.tex}
% and name it \texttt{blub.tex}. The second name is then what the user will see
% in the attachment list or in the properties of an annotation.
%
%\item Reference the Filespec dictionary so that the user can access the file.
%   This can be done in various way:
%   \begin{enumerate}
%      \item With an annotation (/Subtype/FileAttachment). This is done by
%       \pkg{attachfile}, \pkg{attachfile2} and \pkg{intopdf}.
%       Typical entries of such an annotation are:
%
% \medskip
%             \begin{tabular}{lll}
%             key      & value type  & notes\\\hline
%             /FS       & object reference &(Filespec dictionary)\\
%             /Name     & name        & /Graph, /PushPin, /Paperclip, /Tag\\
%             /Contents & text string & optional but recommended\\
%             /F        & integer     & Flags\\
%             /AP       & dictionary  & Appearance (required if rectangle >0) \\
%             /AS       & name\\
%             \end{tabular}
%
%             The |/AP| takes precedence over Border and similar keys.
%      \item Through an entry in the |/EmbeddedFiles| name tree.
%       This is what \pkg{embedfiles} does.
%       \begin{verbatim}
%       20 0 obj %Document Name tree
%        <</EmbeddedFiles 21 0 R>>
%           endobj
%         21 0 obj %Embedded Files Name dictionary
%        <</Names [(AcmeCustomCrypto Protected PDF.pdf) 17 0 R]>>
%        endobj
%       \end{verbatim}
%       The strings (keys) in the |/Names| dictionary must be sorted lexically.
%       But they don't have to be the file name or anything related to
%       the file name. The resource management code uses l3ef0001, l3ef0002~\ldots,
%       which allows up to 9999 files.
%       The key can be needed to identify the start file in a collection,
%       so their relation to the files are stored in a property list.
%
%      \item Through the |/AF| key in various objects (pdf 2.0).
%       The value is normally an array of object
%       references, but it can also be a name which is mapped to an array in /Properties:
%        \begin{verbatim}
%        /AF /NamedAF BDC
%        /Properties <</NamedAF [12 0 R]
%        \end{verbatim}
%        The related |/Filespec| dictionary should contain an
%        |/AFRelationship| key in this case (but it doesn't harm to add it by
%        default anyway). The values of this key is describe in table~\ref{tab:AFrel}.
%
% \begin{table}
%       \caption{Values of the \texttt{/AFRelationship} key\label{tab:AFrel}}
%       \begin{tabular}{lp{8cm}}
%       Source  & shall be used if this file specification is the original
%                source material for the associated content.\\
%       Data    & shall be used if this file specification represents
%                 information used to
%                 derive a visual presentation – such as for a table or a graph.\\
%       Alternative & shall be used if this file specification is an
%               alternative representation of content, for example audio.\\
%       Supplement & shall be used if this file specification represents
%       a supplemental representation of the original source or data
%       that may be more easily consumable
%       (e.g., A MathML version of an equation).\\
%       EncryptedPayload & shall be used if this file specification
%       is an encrypted payload document that should be displayed to the user
%       if the PDF processor has the cryptographic filter
%       needed to decrypt the document.\\
%       FormData & shall be used if this file specification
%       is the data associated with the AcroForm
%       (see 12.7.3, \enquote{Interactive form dictionary}) of this PDF.\\
%       Schema & shall be used if this file specification is a schema
%       definition for the associated object
%       (e.g. an XML schema associated with a metadata stream).\\
%       Unspecified &(default value) shall be used when the
%       relationship is not known
%       or cannot be described using one of the other values.\\
%       Other names & Second-class names (see Annex E,
%       \enquote{(normative) PDF Name Registry}) should be used to
%       represent other types of relationships.
%       \end{tabular}
% \end{table}
%      \end{enumerate}
% \end{enumerate}
%
% \subsubsection{Task 1: Embedding a file}
% Embedding an existing file is in most cases quite straightforward. This module
% offers commands, but it can also be done with the basic commands
% from the |l3pdf| module \cs{pdf_object_unnamed_write:nn} or
% \cs{pdf_object_new:nn}/\cs{pdf_object_write:nn} or primitive commands
% to create objects.
% The object number should be stored for the reference
% in the |/Filespec| dictionary.
%
% \begin{verbatim}
% \pdf_object_unnamed_write:nx {fstream}
%   {
%     {
%       /Type /EmbeddedFile
%       /Subtype /application\c_hash_str2Fpostscript
%       /Params
%         <<
%           /ModDate  ~ (\file_timestamp:n{example-image.eps})
%           /Size     ~  \file_size:n {example-image.eps}
%           /CheckSum ~ (\file_mdfive_hash:n {example-image.eps})
%          >>
%     }
%     {example-image.eps}
%   }
% \tl_set:Nx \l_my_fileobj_tl {\pdf_object_ref_last:}
% \end{verbatim}
%
% \begin{itemize}
% \item The |/Params| dictionary is not always required, but the commands of
% these module will prefill them as shown in the examples. A |/CreationDate| entry
% has to be added explicitly as there is no sensible way
% to retrieve this automatically.
% \item The mimetype (in the |/Subtype|) should be properly escaped.
% This module contains a property list with maps a number of file extensions
% to mimetypes and the commands try to detect and fill the mimetype automatically.
% \item The dictionary can contain additional keys (|/Filter|, |/DecodeParms|),
% see the pdf reference.
% \end{itemize}
%
% \subsubsection{Task 2: Creating the \texttt{/Filespec} dictionary}
% The |/Filespec| dictionary is a simple dictionary object, and can also
% be created in various ways. If it refers to an embedded file it should
% reference it in the |/EF| key.
%
% \subsubsection{Task 3: Referencing the \texttt{/Filespec} dictionary}
%
% Using the dictionary reference in annotations and |/AF| keys is unproblematic.
%
% \potentialclash
% But to add it to the |/EmbeddedFiles| name tree so that it appears in the
% attachment panel requires special care:
% This name tree is a global resource and uncoordinated access can lead to
% clashes and files that are not visible or inaccessible.
% The access here is managed by the \pkg{l3pdfmanagement} module:\\[\smallskipamount]%
% |\pdfmanagement_add:nnx{Catalog/Names}{EmbeddedFiles}{|\meta{objref}|}|
%
% \subsection{Commands and tools of these module}
% \begin{function}{file, file/Params, file/streamParams,file/Filespec}
% The module predefines and uses a number of local dictionaries for the
% components of the stream and the |/Filespec| object. These dictionaries are
% then used by the \cs{pdffile_embed_XX}.
% The content of these dictionaries can be changed by users with the commands
% from the \pkg{l3pdfdict} module, but it should be done only locally
% to avoid side effects on uses by other packages/commands.

% The preset values are of these dictionaries are shown in table~\ref{tab:filedict}.
% \end{function}
% \begin{table}
% \caption{Preset values in the file dictionaries\label{tab:filedict}}
% \hspace*{-3cm}\begin{tabular}{lll}
% dictionary & key      & value \\\hline
% l\_pdffile       & Type     & /EmbeddedFile\\
% l\_pdffile/Params& Size     & |\file_size:n{\l_pdffile_source_name_str}|\\
% l\_pdffile/Params& ModDate  & |(\file_timestamp:n {\l_pdffile_source_name_str})|\\
% l\_pdffile/Params& CheckSum & |(\file_mdfive_hash:n{\l_pdffile_source_name_str})|\\
% l\_pdffile/streamParams&    &  a /ModDate entry with year/month/date \\
%                        &    & (used with \cs{pdffile_embed_stream:nnn})\\
% l\_pdffile/Filespec & Type  & /Filespec\\
% l\_pdffile/Filespec & AFRelationship &Unspecified
% \end{tabular}
% \end{table}
% \begin{function}{\pdffile_embed_file:nnn}
%   \begin{syntax}
%     \cs{pdffile_embed_file:nnn} \Arg{source filename} \Arg{target filename} \Arg{object name }
%   \end{syntax}
%   This commands embeds the file \meta{source filename} in the PDF,
%   and creates a |/Filespec| dictionary object named \meta{object name}.
%   The object name must be unique, it should start with the module name, so
%   e.g. \texttt{module/name}.
%   The command uses the content of the local
%   dictionaries |l_pdffile|, |l_pdffile/Params| and |l_pdffile/Filespec|
%   to setup the dictionary entries of the stream object and the
%   |/Filespec| dictionary. The |/F| and |/UF| entry are filled
%   with \meta{target filename}.
%
%   It is an error if both \meta{target filename} and \meta{source filename}
%   are empty.
%
%   If \meta{target filename} is empty \meta{source filename} is used instead.
%
%   If \meta{source filename} is empty, only a |/Filespec| dictionary is
%   created.
%
%   If the |l_pdffile| dictionary doesn't  contain a
%   Subtype entry with the mimetype, the command tries to guess it
%   from the file extension of \meta{source filename}. Unknown file extensions can be
%   added (or known extension be changed) by adding to or changing the value in
%   the property \cs{g_pdffile_mimetypes_prop}, see below.
%
%   When using \texttt{dvips} and \texttt{pstopdf} the actual embedding is
%   done by \texttt{pstopdf}.  \texttt{pstopdf} will embed files
%   only if used with the option \texttt{-dNOSAFER} and will not be able
%   to use files which are found with \texttt{kpathsea}.
%
%   \meta{target filename} doesn't need to be a file name with an extension,
%   but it is recommended as security settings in the pdf
%   viewer can restrict access to known file types.
% \end{function}
%
% \begin{NOTE}{UF}
%  we should perhaps also consider the chunk method see pdfbase and
%  https://chat.stackexchange.com/transcript/message/54181193#54181193
% \end{NOTE}
%
% \begin{function}{\pdffile_embed_stream:nnn}
%   \begin{syntax}
%     \cs{pdffile_embed_stream:nnn} \Arg{content} \Arg{target filename} \Arg{object name }
%   \end{syntax}
%   This commands embeds the \meta{content} in the PDF in a stream objects and
%   creates a |/Filespec| dictionary object named \meta{object name}.
%   \meta{content} is wrapped in a \cs{exp_not:n}.
%   The object name must be unique. The command uses the content of the local
%   dictionaries |l_pdffile|, |l_pdffile/streamParams| and |l_pdffile/Filespec|
%   to setup the dictionary entries of the stream object and the /Filespec dictionary.
%   The /F and /UF entry are filled with \meta{target filename}.
%   If \meta{target filename} is empty the fix name \texttt{stream.txt}
%   is used instead.
%
%   If the |l_pdffile| dictionary doesn't  contain a
%   Subtype entry with the mimetype, the command tries to guess it
%   from the file extension of \meta{target filename}.
%
%   \meta{target filename} doesn't need to be a file name with an extension,
%   but it is recommended as security settings in the pdf
%   viewer can restrict access to known file types.
%
%   The stream should not be too long, at least PS imposes a size limit for strings.
% \end{function}
%
% \begin{function}{\pdffile_filespec:nnn,\pdffile_filespec:nnx}
%   \begin{syntax}
%     \cs{ pdffile_filespec:nnn }\Arg{object name}\Arg{file name}{stream object reference}
%   \end{syntax}
%  The previous commands are fine if stream and filespec dictionary can be created together.
%  But there are cases where the |filespec| dictionary should be referenced when the
%  stream object doesn't exist yet. For example in the |AF| key of a structure at
%  the begin of an environment where the stream is created from the body.
%
%  This command allows to write a filespec dictionary alone and reference a previously
%  created stream.
%
%  \begin{verbatim}
%  \pdf_object_new:nn{module/filespec/A}{dict} % a new filespec object
%  \pdf_object_ref:n {module/filespec/A}       % a reference
%  \pdf_object_unnamed_write:nn { stream }{ {...}{content} } %writing the stream
%  % filling and writing the filespec dictionary:
%  \pdffile_filespec:nnn {module/filespec/A}{A.xml}{\pdf_object_ref_last:}
%  \end{verbatim}
%
% \end{function}
%
%  \begin{variable}{\g_pdffile_mimetypes_prop}
%   This property contains a list of extensions and their mimetypes.
%   Values can be added or changed with the standard commands:
%
%   |\prop_gput:Nnn \g_pdffile_mimetypes_prop {.abc}{text/plain}|
%
%   The extension should start with a period, the mimetype should be given
%   as plain text (it will be escaped internally). Extensions with two periods
%   are not supported.
% \end{variable}
%
% \begin{variable}{\l_pdffile_source_name_str}
% This variable is set at the begin of \cs{pdffile_embed_file:nnn}. It can be
% (and is) used in the file dictionaries, see table~\ref{tab:filedict} for examples.
% \end{variable}
%
% \begin{variable} {\g_pdffile_embed_prop}
% This property holds a list of embedded files. It is used by the following
% show command. The keys are the object names, the argument holds a key word,
% the source file name and the target file name.
% \end{variable}
%
% \begin{function}{\pdffile_embed_show:}
% This shows the embedded files with their source and target name.
% \end{function}
%
% \subsection{Example}
% \begin{verbatim}
% \group_begin:
% %set the relationship:
% \pdfdict_put:nnn  {l_pdffile/Filespec} {AFRelationship}{/Source}
% %set the description key. The text must first be converted:
% \pdf_string_from_unicode:nnN {utf16/string}
%    {this~is~an~odd~description~with~öäü}
%    \l_tmpa_str
% \pdfdict_put:nnx  {l_pdffile/Filespec} {Desc}{\l_tmpa_str}
% %embeds testinput.txt and calls it grüße.txt
% \pdffile_embed_file:nnn {testinput.txt}{grüße.txt}{mymodule/example1}
% %reference it in the panel
% \pdfmanagement_add:nnx
%  {Catalog/Names}
%  {EmbeddedFiles}
%   {\pdf_object_ref:n{mymodule/example1}}
% \group_end:
% \end{verbatim}
% \end{documentation}
%
% \begin{implementation}
%
% \section{\pkg{l3pdffile} implementation}
%
%    \begin{macrocode}
%<*header>
\ProvidesExplPackage{l3pdffile}{2021-08-28}{0.95i}
  {embedding and referencing files in PDF---LaTeX PDF management testphase bundle}
\RequirePackage{l3pdftools}  %temporarly!!
%</header>
%    \end{macrocode}
%
%    \begin{macrocode}
%<*package>
%<@@=pdffile>
\cs_new_protected:Npn \@@_filename_convert_to_print:nN #1 #2
  {\pdf_string_from_unicode:nnN {utf16/hex}{#1}{#2}}
%    \end{macrocode}
% \subsection{Messages}
%    \begin{macrocode}
\msg_new:nnn { pdffile } { file-not-found }
  {
    File~'\tl_to_str:n{#1}'~not~found
  }

\msg_new:nnn { pdffile } { mimetype-missing }
  {
    Mime~type~not~set~for~file~'\tl_to_str:n{#1}'
  }

\msg_new:nnn { pdffile } { target-name-missing }
  {
    a~target~name~for~the~/Filespec~dictionary~is~missing.
  }

\msg_new:nnn { pdffile } { object-exists }
  {
    object~name~'#1'~is~already~used.
  }

\msg_new:nnn  { pdffile } { show-files }
  {
    The~following~files~have~been~embedded\\
     #1
  }
%    \end{macrocode}
% \begin{variable}
%   {
%     \l_@@_tmpa_tl,
%     \l_@@_tmpb_tl,
%     \l_@@_tmpa_str,
%     \l_@@_tmpb_str,
%     \l_@@_ext_str,
%     \l_@@_automimetype_tl,
%     \l_@@_embed_ref_tl
%   }
% temporary variables: generic, for extension, subtype, to store the ref.
% \end{variable}
%    \begin{macrocode}
\tl_new:N  \l_@@_tmpa_tl
\tl_new:N  \l_@@_tmpb_tl
\str_new:N \l_@@_tmpa_str
\str_new:N \l_@@_tmpb_str
\str_new:N \l_@@_ext_str
\tl_new:N  \l_@@_automimetype_tl
\tl_new:N  \l_@@_embed_ref_tl
%    \end{macrocode}
% \begin{variable} {\g_pdffile_mimetypes_prop}
% This variable holds common mimetypes. The key is an extension with (one) period,
% the value the description, e.g. \texttt{text/csv}.
% \end{variable}
%    \begin{macrocode}
\prop_new:N \g_pdffile_mimetypes_prop
\prop_set_from_keyval:Nn \g_pdffile_mimetypes_prop
  {
    ,.csv = text/csv
    ,.html= text/html
    ,.dtx = text/plain %or application/x-tex, not in iana.org list
    ,.eps = application/postscript
    ,.jpg = image/jpeg
    ,.mp4 = video/mp4
    ,.pdf = application/pdf
    ,.png = image/png
    ,.tex = text/plain %or application/x-tex, not in iana.org list
    ,.txt = text/plain
    ,.sty = text/plain
    ,.xml = text/xml
  }
%    \end{macrocode}
% \begin{variable}
%  {
%   \l_pdffile_source_name_str
%  }
% \cs{l_pdffile_source_name_str} will be set at the begin of the command and
% contains the full file name and can be used e.g. with \cs{file_timestamp:n}.
% \end{variable}
%    \begin{macrocode}
\str_new:N  \l_pdffile_source_name_str
%    \end{macrocode}
% Here we define and setup the local dictionaries.
% We add a ModDate to ensure that there is an entry if
% associated files are used.
%    \begin{macrocode}
\pdfdict_new:n { l_pdffile }
\pdfdict_put:nnn { l_pdffile }{Type}{/EmbeddedFile}
\pdfdict_new:n { l_pdffile/Params }
\pdfdict_put:nnn { l_pdffile/Params }
  {ModDate}  { (\file_timestamp:n { \l_pdffile_source_name_str }) }
\pdfdict_put:nnn { l_pdffile/Params }
  {Size}     { \file_size:n { \l_pdffile_source_name_str } }
\pdfdict_put:nnn { l_pdffile/Params }
  {CheckSum} { (\file_mdfive_hash:n { \l_pdffile_source_name_str }) }
\pdfdict_new:n { l_pdffile/streamParams }
\pdfdict_put:nnn { l_pdffile/streamParams }
  {ModDate}  {
               (
                 D:\int_use:N\c_sys_year_int
                  \int_compare:nNnT{\c_sys_month_int}<{10}{0}
                  \int_use:N\c_sys_month_int
                  \int_compare:nNnT{\c_sys_day_int}<{10}{0}
                  \int_use:N\c_sys_day_int
               )
             }
\pdfdict_new:n { l_pdffile/Filespec }
\pdfdict_put:nnn { l_pdffile/Filespec }
  {Type} { /Filespec }
\pdfdict_put:nnn { l_pdffile/Filespec }
  {AFRelationship} { /Unspecified }

%    \end{macrocode}
% \begin{variable}{\g_pdffile_embed_prop}
% we record here the relation\\%
% \meta{object name} $\Rightarrow$
%  \Arg{file/stream or empty}\Arg{sourcename}\Arg{targetname}
%
%    \begin{macrocode}
\prop_new:N \g_pdffile_embed_prop
%    \end{macrocode}
% \end{variable}
% \begin{macro}{\pdffile_embed_show:}
%    \begin{macrocode}
\cs_new_protected:Npn \pdffile_embed_show:
  {
    \msg_show:nnx
     {pdffile}{show-files}
     {
       \prop_map_function:NN {\g_pdffile_embed_prop} \msg_show_item:nn
     }
  }
%    \end{macrocode}
% \end{macro}
%
% \begin{macro}{\pdffile_embed_file:nnn, \pdffile_embed_stream:nnn}
% At first a command to set the mimetype. It either uses the current value
% in the file dictionary, or tries to guess it from the extension.
% \begin{macro}{\@@_mimetype_set:nN,\@@_mimetype_set:VN}
% \begin{macro}{\@@_fstream_write:nN, \@@_fstream_write:VN}
% \begin{macro}{\@@_stream_write:nN, \@@_stream_write:VN}
%    \begin{macrocode}
%#1 file name,
%#2 tl to return the (printed) value for the guessed mimetype
\cs_new_protected:Npn \@@_mimetype_set:nN #1 #2
  {
    \file_parse_full_name:nNNN
          {#1}
          \l_@@_tmpa_str %unused
          \l_@@_tmpb_str %unused
          \l_@@_ext_str
        %check if Subtype has been set
        \pdfdict_get:nnN { l_pdffile}{Subtype}\l_@@_tmpa_tl
        %if not look up in the prop:
        \quark_if_no_value:NT \l_@@_tmpa_tl
          {
            \prop_get:NVNTF
              \g_pdffile_mimetypes_prop
              \l_@@_ext_str
              \l_@@_tmpb_tl
              {
                \tl_set:Nx #2 {/Subtype~\pdf_name_from_unicode_e:V \l_@@_tmpb_tl}
              }
              {
                \msg_warning:nnx { pdffile }{ mimetype-missing} {#1}
                \tl_clear:N #2
              }
          }
    }

\cs_generate_variant:Nn \@@_mimetype_set:nN {VN}

%#1 file name,
%#2 tl, should be empty or contain /Subtype /mimetype
%   e.g. result from \@@_mimetype_set:NN
\cs_new_protected:Npn \@@_fstream_write:nN #1 #2
  {
    \pdf_object_unnamed_write:nx { fstream }
      {
        {
          #2
          \pdfdict_use:n { l_pdffile}
          \pdfdict_if_empty:nF { l_pdffile/Params}
            {
              /Params
                <<
                  \pdfdict_use:n { l_pdffile/Params}
                >>
            }
        }
        { #1 }
      }
     \tl_clear:N \l_@@_automimetype_tl
   }

\cs_generate_variant:Nn \@@_fstream_write:nN {VN}

%#1 file content
%#2 tl, should be empty or contain /Subtype /mimtype
%   e.g. result from \@@_mimetype_set:NN
\cs_new_protected:Npn \@@_stream_write:nN #1 #2
  {
    \pdf_object_unnamed_write:nx { stream }
      {
        {
          #2
          \pdfdict_use:n { l_pdffile}
          \pdfdict_if_empty:nF { l_pdffile/streamParams}
            {
              /Params
                <<
                  \pdfdict_use:n { l_pdffile/streamParams}
                >>
            }
        }
        { \exp_not:n { #1 } }
      }
     \tl_clear:N \l_@@_automimetype_tl
   }

\cs_generate_variant:Nn \@@_stream_write:nN {VN}

%#1 symbolic name of dict object
%#2 target file name,
%#3 object ref of the file stream.
\cs_new_protected:Npn \@@_filespec_write:nnn #1 #2 #3
  {
    \tl_if_blank:nT { #2 }
      {
        \msg_error:nn {pdffile}{target-name-missing}
      }
      {
        \group_begin:
          \@@_filename_convert_to_print:nN  { #2 } \l_@@_tmpa_str
          \pdfdict_put:nnx {l_pdffile/Filespec}{F} { \l_@@_tmpa_str }
          \pdfdict_put:nnx {l_pdffile/Filespec}{UF}{ \l_@@_tmpa_str }
          \pdf_object_write:nx { #1 }
            {
              \pdfdict_use:n { l_pdffile/Filespec}
              \tl_if_empty:nF { #3 }
                {
                  /EF <</F~#3 /UF~#3>>
                }
            }
        \group_end:
      }
  }

\cs_set_eq:NN \pdffile_filespec:nnn \@@_filespec_write:nnn
\cs_generate_variant:Nn \pdffile_filespec:nnn {nnx}
%#1 {source filename}
%#2 {target filename}
%#3 { filespec object name } (will internally get a prefix! ??)
\cs_new_protected:Npn \pdffile_embed_file:nnn #1 #2 #3
  { %               if #1 empty => only filespec
    %               if #2 empty => = #1
    \pdf_object_if_exist:nTF { #3 }
      {
        \msg_error:nnn { pdffile }{ object-exists } { #3 }
      }
      {
        \tl_if_blank:nTF { #1 }
          {
            \tl_set:Nn \l_@@_embed_ref_tl {}
          }
          {
            \file_get_full_name:nNTF {#1} \l_pdffile_source_name_str
              {
                \@@_mimetype_set:VN
                  \l_pdffile_source_name_str
                  \l_@@_automimetype_tl
                \@@_fstream_write:VN
                  \l_pdffile_source_name_str
                  \l_@@_automimetype_tl
                \tl_set:Nx \l_@@_embed_ref_tl { \pdf_object_ref_last: }
              }
              {
                \msg_error:nnn { pdffile }{ file-not-found }{ #1 }
              }

           }
        \prop_gput:Nnx
           \g_pdffile_embed_prop
           { #3 }
           {
             { \tl_if_blank:nTF { #1 } {filespec}{file} }
             {\l_pdffile_source_name_str}
             {
               \tl_if_blank:nTF { #2 }
                 { \l_pdffile_source_name_str }
                 { \tl_to_str:n{#2}}
             }
           }
        \tl_if_blank:nTF { #2 }
          {
            \pdf_object_new:nn   { #3 } {dict}
            \exp_args:Nnnx
              \@@_filespec_write:nnn
                %#1 dict, #2 target file name, #3 object ref
                { #3 }
                { #1 }
                {\l_@@_embed_ref_tl}
          }
          {
            \pdf_object_new:nn   { #3 } {dict}
            \exp_args:Nnnx
              \@@_filespec_write:nnn
                %#1 dict, #2 target file name, #3 object ref
                { #3 }
                { #2 }
                {\l_@@_embed_ref_tl}
          }
      }
  }


%#1{stream content}
%#2{target filename}
%#3{file object name }
\cs_new_protected:Npn \pdffile_embed_stream:nnn #1 #2 #3
  {
    %               if #2 empty => error
    \pdf_object_if_exist:nTF { #3 }
      {
         \msg_error:nnn { pdffile }{ object-exists } { #3 }
      }
      {
         \prop_gput:Nnx
            \g_pdffile_embed_prop
            { #3 }
            {{stream}{}{\tl_if_blank:nTF {#2}{stream.txt}{\exp_not:n{#2}}}}
         \tl_if_blank:nTF {#2}
          { \@@_mimetype_set:nN {stream.txt}\l_@@_automimetype_tl}
          { \@@_mimetype_set:nN { #2 } \l_@@_automimetype_tl }
         \@@_stream_write:nN
           { #1 }
           \l_@@_automimetype_tl
         \tl_set:Nx \l_@@_embed_ref_tl { \pdf_object_ref_last: }
         \pdf_object_new:nn   { #3 } {dict}
         \exp_args:Nnxx
           \@@_filespec_write:nnn
             %#1 dict, #2 target file name, #3 object ref
             { #3 }
             { \tl_if_blank:nTF {#2}{stream.txt}{\exp_not:n{#2}} }
             {\l_@@_embed_ref_tl}
     }
  }


%    \end{macrocode}
% \end{macro}
% \end{macro}
% \end{macro}
% \end{macro}
%    \begin{macrocode}
%</package>
%    \end{macrocode}
%
% \end{implementation}
%
% \PrintIndex
