% \iffalse meta-comment % -*- latex -*-
%
%<*ignore>
\begingroup
  \catcode123=1 %
  \catcode125=2 %
  \def\x{LaTeX2e}%
\expandafter\endgroup
\ifcase 0\ifx\install y1\fi\expandafter
         \ifx\csname processbatchFile\endcsname\relax\else1\fi
         \ifx\fmtname\x\else 1\fi\relax
\else\csname fi\endcsname
%</ignore>
%<*install>
\input docstrip.tex
\keepsilent

\usedir{tex/latex/tikz-page}

\let\MetaPrefix\relax
\preamble
This is a generated file.

Copyright © 2016 by Sébastien Gross <seb•ɑƬ•chezwam•ɖɵʈ•org>

This program is free software. It comes without any warranty, to the extent
permitted by applicable law. You can redistribute it and/or modify it under
the terms of the Do What The Fuck You Want To Public License, Version 2, as
published by Sam Hocevar. See http://sam.zoy.org/wtfpl/COPYING for more
details.

This file may be distributed and/or modified under the conditions of
the LaTeX Project Public License, either version 1.3c of this license
or (at your option) any later version. The latest version of this
license is in:

http://www.latex-project.org/lppl.txt

and version 1.3c or later is part of all distributions of LaTeX
version 2006/05/20 or later.

\endpreamble%
\let\MetaPrefix\DoubleperCent


\generate{\file{tikz-page.sty}{\from{tikz-page.dtx}{tikz-page}}}

\obeyspaces
\Msg{*************************************************************}
\Msg{*                                                           *}
\Msg{* To finish the installation you have to move the following *}
\Msg{* file into a directory searched by TeX:                    *}
\Msg{*                                                           *}
\Msg{* tikz-page.sty                                             *}
\Msg{*                                                           *}
\Msg{* To produce the documentation run the file mypackage.dtx   *}
\Msg{* through LaTeX.                                            *}
\Msg{*                                                           *}
\Msg{* Happy TeXing!                                             *}
\Msg{*                                                           *}
\Msg{*************************************************************}
\endbatchfile
%</install>
%<*ignore>
\fi
%</ignore>
%<*driver>
\ProvidesFile{tikz-page.drv}
\documentclass{ltxdoc}
\EnableCrossrefs
\CodelineIndex
\RecordChanges
\usepackage[english]{babel}
\usepackage[T1]{fontenc} 
\usepackage[utf8]{inputenc}
\usepackage{tcolorbox}
\tcbuselibrary{listings,documentation}
\tcbuselibrary{documentation,listings,minted}
\tcbuselibrary{breakable}
\tcbuselibrary{skins,raster}
\tcbset{listing engine=minted}
\usepackage{url}
\usepackage{minted}
\usepackage{calc}
\usepackage{makeidx}
\usepackage[textpos]{tikz-page}
\renewcommand{\headrulewidth}{0pt}
\makeindex
\definecolor{halfgray}{gray}{0.55}
\newcommand\anglei{-45}
\newcommand\angleii{45}
\newcommand\angleiii{225}
\newcommand\angleiv{135}
\newcommand{\pagedecoration}{
  \tpflip{
    \coordinate (aux1) at ([yshift=-15pt]page.northeast);
    \coordinate (aux2) at ([yshift=-410pt]page.northeast);
    \coordinate (aux3) at ([xshift=-4.5cm]page.northeast);
    \coordinate (aux4) at ([yshift=-150pt]page.northeast);
  }{
    \coordinate (aux1) at ([yshift=-15pt]page.northwest);
    \coordinate (aux2) at ([yshift=-410pt]page.northwest);
    \coordinate (aux3) at ([xshift=4.5cm]page.northwest);
    \coordinate (aux4) at ([yshift=-150pt]page.northwest);
    \renewcommand\anglei{-135}
    \renewcommand\angleii{135}
    \renewcommand\angleiii{-45}
    \renewcommand\angleiv{45}
  }
  \begin{scope}[halfgray!40,line width=12pt,rounded corners=12pt]
    \draw (aux1) -- coordinate (a) ++(\angleiii:5) -- ++(\anglei:5.1) coordinate (b);
    \draw[shorten <= -10pt] (aux3) -- (a) -- (aux1);
    \draw[opacity=0.6,halfgray,shorten <= -10pt] (b) -- ++(\angleiii:2.2) -- ++(\anglei:2.2);
  \end{scope}
  \draw[halfgray,line width=8pt,rounded corners=8pt,shorten <= -10pt]
  (aux4) -- ++(\angleiii:0.8) -- ++(\anglei:0.8);
  \begin{scope}[halfgray!70,line width=6pt,rounded corners=8pt]
    \draw[shorten <= -10pt] (aux2) -- ++(\angleiii:3) coordinate[pos=0.45] (c) -- ++(\anglei:3.1);
    \draw (aux2) -- (c) -- ++(\angleiv:2.5) -- ++(\angleii:2.5) -- ++(\anglei:2.5) coordinate[pos=0.3] (d);   
    \draw (d) -- +(\angleii:1);
  \end{scope}
}

\newcommand{\tikzpagelayout}{
  \pagedecoration
  \node [,outer sep=0,inner sep=0,anchor=base] at (page.footer center) {\thepage};
  \node [,outer sep=0,inner sep=0,anchor=base, rotate=90, color=black!10] at (page.left south) {\footnotesize{tikz-page \pkgfileversion}};
}
\fancypagestyle{plain}{
  \fancyhf{}
  \chead{\tikzpage}
}
\fancypagestyle{titlepage}{
  \fancyhf{}
  \chead{\tikzpage}
}
\pagestyle{plain}
\begin{document}
\DocInput{tikz-page.dtx}
\printindex

\end{document}
%</driver>
%<*README>

%</README>

% \fi

% \iffalse
%% DeclareTCBListing{macrocode} leave a blank line after code.
% \fi

% \expandafter \let \csname macrocode\endcsname \relax \expandafter \let \csname endmacrocode\endcsname \relax
% \expandafter \let \csname macrocode*\endcsname \relax \expandafter \let \csname endmacrocode*\endcsname \relax
% \newminted[macrocode]{latex}{breaklines,breakafter=()/+-\\,baselinestretch=0.5,fontsize=\footnotesize,linenos,numbersep=3mm,firstnumber=last}
%
% \BeforeBeginEnvironment{macrocode}{%
% \begin{tcolorbox}[boxrule=0.1pt,colframe=red!75!black,left=8mm,width=\linewidth,breakable=true,colback=white,enhanced]}
% \AfterEndEnvironment{macrocode}{\end{tcolorbox}}%


% \newminted[example]{latex}{breaklines,breakafter=()/+-\\,baselinestretch=0.5,fontsize=\footnotesize}
% \BeforeBeginEnvironment{example}{%
% \begin{tcolorbox}[boxrule=0.1pt,colframe=red!75!black,left=8mm,width=\linewidth,breakable=true,colback=white,enhanced]}
% \AfterEndEnvironment{example}{\end{tcolorbox}}%

% \tcbset{
%   tpexample/.style = {%
%     enhanced jigsaw,
%     lower separated=false,
%     leftlower=0pt,
%     rightlower=0pt,
%     boxrule=0.1pt,
%     breakable=true,
%     colframe=red!75!black,
%     colback=white,
%     listing engine=minted,
%     minted options={breaklines,breakafter=()/+-\\,baselinestretch=0.5,fontsize=\footnotesize},
%     listing and comment,
%     pdf comment,
%     compilable listing,
%     run pdflatex}
% }


% \iffalse
%<*tikz-page>
% \fi
\def\pkgfileversion{1.0}
\def\pkgfiledate{2016/08/22}
% \iffalse
%</tikz-page>
% \fi

% \title{the \textsc{tikz-page} package }
% \author {Sébastien Gross <seb chezwam org>}
% \date{This file describes version \pkgfileversion\  (\pkgfiledate)}
% \maketitle
% \tableofcontents


% \section{Introduction}

% There are many ways to embelish a page with \LaTeX. One of the most
% easiest way is to use |fancyhdr| which allows to redefine both headers and
% footer. The |geometry| package is also useful to setup correct margins. If
% you need to put some background materials this might become painful,
% especially if you need your background to reach the page borders.

% There are some trick that help you in this task. |tikz-page| helps you in
% this way by the use of semeral mechanisms. Either you can use plain |tikz|
% picture on the background of your page, or use the \meta{textpos} option
% which enables absolute |textpos| positionning. Each method has its
% benefits and nuisances. With |tikz| you have to compile your document
% twice (which can be painful while you are designing your page layout) and
% with |textpos| you can get some incompatibility issues (please refer to
% |textpos| documentation).

% |tikz-page| is trying to give you best of both world by creating a new
% page object in a |tikzpicture| with many anchors. So you can easily place
% your page material at its correct position.


% \section{Usage}

% Basically you only need to add \mintinline{latex}{\usepackage{tikz-page}}
% at the begining of your document. Then you have to declare a
% \docAuxCommand{tikzpagelayout} command which is executed inside the
% background |tikzpicture|. Thus you can access the |page| shape and all its
% anchors. For example the following simple example add the page number to
% the footer center:

% \iffalse
%<*example>
% \fi
% \begin{example}
\newcommand{tikzpagelayout}{%
  \node [outer sep=0,inner sep=0,anchor=base] at (page.footer center) {\thepage};
}
% \end{example}
% \iffalse
%</example>
% \fi


% If you want to simulate the default |fancyhdr| behaviour you can define the
% \docAuxCommand{tikzpagelayout} as following:


% \iffalse
%<*example>
% \fi
% \begin{example}
\newcommand{tikzpagelayout}{%
  \node [outer sep=0,inner sep=0, anchor=mid east] at (page.header east) {\tpflip{\sl\leftmark}{\sl\rightmark}};
  \node [outer sep=0,inner sep=0, anchor=mid west] at (page.header west) {\tpflip{\sl\rightmark}{\sl\leftmark}};
  \node [,outer sep=0,inner sep=0,anchor=base] at (page.footer center) {\thepage};
% \end{example}
% \iffalse
%</example>
% \fi

% A minimum working example:
% \iffalse
%<*example>
% \fi
%\begin{tcblisting}{tpexample}
\documentclass{article}
\usepackage{tikz-page}
\usepackage{lipsum}
\newcommand{\tikzpagelayout}{
  \tpshowframes
  \tikzpageputanchors
}
\pagestyle{plain}
\begin{document}
\lipsum
\end{document}
% \end{tcblisting}
% \iffalse
%</example>
% \fi


% A more complex example\cite{Graphical Decoration}:
% \iffalse
%<*example>
% \fi
%\begin{tcblisting}{tpexample}
\documentclass{article}
\usepackage{tikz-page}
\usepackage{lipsum}
\definecolor{halfgray}{gray}{0.55}
\newcommand\anglei{-45}
\newcommand\angleii{45}
\newcommand\angleiii{225}
\newcommand\angleiv{135}
\newcommand{\tikzpagelayout}{
  \tpflip{
    \coordinate (aux1) at ([yshift=-15pt]page.northeast);
    \coordinate (aux2) at ([yshift=-410pt]page.northeast);
    \coordinate (aux3) at ([xshift=-4.5cm]page.northeast);
    \coordinate (aux4) at ([yshift=-150pt]page.northeast);
  }{
    \coordinate (aux1) at ([yshift=-15pt]page.northwest);
    \coordinate (aux2) at ([yshift=-410pt]page.northwest);
    \coordinate (aux3) at ([xshift=4.5cm]page.northwest);
    \coordinate (aux4) at ([yshift=-150pt]page.northwest);
    \renewcommand\anglei{-135}
    \renewcommand\angleii{135}
    \renewcommand\angleiii{-45}
    \renewcommand\angleiv{45}
  }
  \begin{scope}[halfgray!40,line width=12pt,rounded corners=12pt]
    \draw (aux1) -- coordinate (a) ++(\angleiii:5) -- ++(\anglei:5.1) coordinate (b);
    \draw[shorten <= -10pt] (aux3) -- (a) -- (aux1);
    \draw[opacity=0.6,halfgray,shorten <= -10pt] (b) -- ++(\angleiii:2.2) -- ++(\anglei:2.2);
  \end{scope}
  \draw[halfgray,line width=8pt,rounded corners=8pt,shorten <= -10pt]
  (aux4) -- ++(\angleiii:0.8) -- ++(\anglei:0.8);
  \begin{scope}[halfgray!70,line width=6pt,rounded corners=8pt]
    \draw[shorten <= -10pt] (aux2) -- ++(\angleiii:3) coordinate[pos=0.45] (c) -- ++(\anglei:3.1);
    \draw (aux2) -- (c) -- ++(\angleiv:2.5) -- ++(\angleii:2.5) -- ++(\anglei:2.5) coordinate[pos=0.3] (d);   
    \draw (d) -- +(\angleii:1);
  \end{scope}
}
\pagestyle{plain}
\begin{document}
\lipsum
\end{document}
% \end{tcblisting}
% \iffalse
%</example>
% \fi



% \iffalse
%<*tikz-page>
% \fi

% \iffalse
%% Reset the FancyVerbLine counter to have the correct lines number in front
%% of implementation.
% \fi
% \setcounter{FancyVerbLine}{0}

% \section{Implementation}

%    \begin{macrocode}
\NeedsTeXFormat{LaTeX2e}
\ProvidesPackage{tikz-page}[\pkgfiledate\space (v\pkgfileversion)]
%    \end{macrocode}

% The \meta{textpos} option can be used if you want to use |textpos|
% \meta{overlay} option instead of |current page| to position the page
% layout. Beware that |textpos| with \meta{overlay} option maybe incompatible
% with some other packages. On the other hand |tikz| |current page| requires
% at least 2 compilation to work correctly. Thus you might want to use
% \meta{textpos} at conception time and remove this option for your final
% build or if you have incompatibility issues.

%    \begin{macrocode}
\newif\if@tp@use@textpos\@tp@use@textposfalse
\DeclareOption{textpos}{\@tp@use@textpostrue}
\ProcessOptions

\if@tp@use@textpos
\RequirePackage[absolute]{textpos}
\fi
%    \end{macrocode}

%    \begin{macrocode}
\RequirePackage{fancyhdr}
\RequirePackage{tikz}
\usetikzlibrary{plotmarks,calc,shapes,positioning,decorations.text}
\RequirePackage{graphicx}
\RequirePackage{calc}
%    \end{macrocode}


\makeatletter

% All margin sizes are defined in\docLength{@tp@left@margin},
% \docLength{@tp@right@margin}, \docLength{@tp@top@margin},
% \docLength{@tp@bottom@margin} their values are computed by the
% \refCom{tp@compute@margins}.

%    \begin{macrocode}
\newlength{\@tp@left@margin}
\newlength{\@tp@right@margin}
\newlength{\@tp@top@margin}
\newlength{\@tp@bottom@margin}
%    \end{macrocode}


% \begin{docCommand}{@tp@create@length}{\marg{block name}\marg{length name}}

% Generate a \cs{tp@\meta{block name}@\meta{length name}} length. This
% command is intended to be only used to create block length defined below.

%    \begin{macrocode}
\newcommand\@tp@create@length[2]{%
  \expandafter\newskip\csname tp@#1@#2\endcsname%
}%
%    \end{macrocode}
% \end{docCommand}
% 

% For each standard blocks in the page (|page|, |body|, |marginpar|,
% |header|, |footer|) and additionnal blocks (|top|, |right|, |bottom|,
% |left|), 6 lenths are computed in order to define their anchors. Each
% length is defined using the \refCom{@tp@create@length} macro.

%    \begin{macrocode}
\foreach\@@tp@element in {page,body,marginpar,header,footer,top,right,bottom,left}{%
  \foreach\@@tp@len in {xmin,xmax,xmid,ymin,ymax,ymid}{%
    \@tp@create@length{\@@tp@element}{\@@tp@len}%
}}%
%    \end{macrocode}

% \begin{docCommand}{tcflip}{\marg{odd page code}\marg{even page code}}
% Execute \meta{odd page even code} on odd pages and \meta{even page code}
% on even ones.
% \end{docCommand}
%    \begin{macrocode}
\newcommand{\tpflip}[2]{\ifodd\thepage#1\else#2\fi}
%    \end{macrocode}

% \begin{docCommand}{tp@compute@margins}{}
%
% This is where the magic happens. This command sets all \cs{tp@\meta{block
% name}@\meta{length name}} lengths.
%
% \end{docCommand}

%    \begin{macrocode}
\def\tp@compute@margins{%
  \setlength{\tp@page@xmin}{0pt}%
  \setlength{\tp@page@ymin}{0pt}%
  \setlength{\tp@page@xmax}{\paperwidth}%
  \setlength{\tp@page@ymax}{\paperheight}%
  \setlength{\tp@page@xmid}{\dimexpr(\tp@page@xmin+\tp@page@xmax)/2\relax}%
  \setlength{\tp@page@ymid}{\dimexpr(\tp@page@ymin+\tp@page@ymax)/2\relax}%
  %
  \setlength\@tp@left@margin{\dimexpr(1in+\hoffset+\tpflip{\oddsidemargin}{\evensidemargin})\relax}%
  \setlength\@tp@right@margin{\dimexpr(\paperwidth-\@tp@left@margin-\textwidth)\relax}%
  \setlength\@tp@top@margin{\dimexpr(1in+\voffset+\topmargin+\headheight+\headsep)\relax}%
  \setlength\@tp@bottom@margin{\dimexpr(\paperheight-(\textheight+\@tp@top@margin))\relax}%
  %% Body computation
  \setlength\tp@body@xmin{\dimexpr\tp@page@xmin+\@tp@left@margin\relax}%
  \setlength\tp@body@xmax{\dimexpr\tp@page@xmax-\@tp@right@margin\relax}%
  \setlength\tp@body@xmid{\dimexpr((\tp@body@xmax+\tp@body@xmin)/2)\relax}%
  \setlength\tp@body@ymax{\dimexpr(\tp@page@ymax-\@tp@top@margin)\relax}%
  \setlength\tp@body@ymin{\dimexpr\tp@body@ymin+\@tp@bottom@margin\relax}%
  \setlength\tp@body@ymid{\dimexpr(\tp@body@ymin+(\tp@body@ymax-\tp@body@ymin)/2)\relax}%
  %
  %% Margin computation
  %
  \tpflip{%
    \setlength\tp@marginpar@xmin{\dimexpr\tp@body@xmax+\marginparsep\relax}
    \setlength\tp@marginpar@xmax{\dimexpr\tp@marginpar@xmin+\marginparwidth\relax}%
  }{%
    \setlength\tp@marginpar@xmax{\dimexpr\tp@body@xmin-\marginparsep\relax}%
    \setlength\tp@marginpar@xmin{\dimexpr\tp@marginpar@xmax-\marginparwidth\relax}%
  }%
  \setlength\tp@marginpar@xmid{\dimexpr((\tp@marginpar@xmax+\tp@marginpar@xmin)/2)\relax}%
  \setlength\tp@marginpar@ymax{\tp@body@ymax}%
  \setlength\tp@marginpar@ymin{\tp@body@ymin}%
  \setlength\tp@marginpar@ymid{\tp@body@ymid}%
  %
  %% header
  %
  \setlength\tp@header@xmax{\tp@body@xmax}%
  \setlength\tp@header@xmin{\tp@body@xmin}%
  \setlength\tp@header@xmid{\tp@body@xmid}%
  \setlength\tp@header@ymin{\dimexpr\tp@body@ymax+\headsep\relax}%
  \setlength\tp@header@ymax{\dimexpr\tp@header@ymin+\headheight\relax}%
  \setlength\tp@header@ymid{\dimexpr((\tp@header@ymax+\tp@header@ymin)/2)\relax}%
  %
  %% footer
  %
  \setlength\tp@footer@xmax{\tp@body@xmax}%
  \setlength\tp@footer@xmin{\tp@body@xmin}%
  \setlength\tp@footer@xmid{\tp@body@xmid}%
  \setlength\tp@footer@ymin{\dimexpr\tp@body@ymin-\footskip\relax}%
  \setlength\tp@footer@ymax{\tp@footer@ymin}%
  \setlength\tp@footer@ymid{\dimexpr((\tp@footer@ymax+\tp@footer@ymin)/2)\relax}%
  %%
  %% blocks%
  %%
  \setlength\tp@top@xmin{\tp@page@xmin}%
  \setlength\tp@top@xmax{\tp@page@xmax}%
  \setlength\tp@top@xmid{\dimexpr((\tp@top@xmax+\tp@top@xmin)/2)\relax}%
  \setlength\tp@top@ymin{\tp@body@ymax}%
  \setlength\tp@top@ymax{\tp@page@ymax}%
  \setlength\tp@top@ymid{\dimexpr((\tp@top@ymax+\tp@top@ymin)/2)\relax}%
  %%
  \setlength\tp@bottom@xmin{\tp@page@xmin}%
  \setlength\tp@bottom@xmax{\tp@page@xmax}%
  \setlength\tp@bottom@xmid{\dimexpr((\tp@bottom@xmax+\tp@bottom@xmin)/2)\relax}%
  \setlength\tp@bottom@ymin{\tp@page@ymin}%
  \setlength\tp@bottom@ymax{\tp@body@ymin}%
  \setlength\tp@bottom@ymid{\dimexpr((\tp@bottom@ymax+\tp@bottom@ymin)/2)\relax}%
  %%
  \setlength\tp@left@xmin{\tp@page@xmin}%
  \setlength\tp@left@xmax{\tp@body@xmin}%
  \setlength\tp@left@xmid{\dimexpr((\tp@left@xmax+\tp@left@xmin)/2)\relax}%
  \setlength\tp@left@ymin{\tp@body@ymin}%
  \setlength\tp@left@ymax{\tp@body@ymax}%
  \setlength\tp@left@ymid{\dimexpr((\tp@left@ymax+\tp@left@ymin)/2)\relax}%
  %%
  \setlength\tp@right@xmin{\tp@body@xmax}%
  \setlength\tp@right@xmax{\tp@page@xmax}%
  \setlength\tp@right@xmid{\dimexpr((\tp@right@xmax+\tp@right@xmin)/2)\relax}%
  \setlength\tp@right@ymin{\tp@body@ymin}%
  \setlength\tp@right@ymax{\tp@body@ymax}%
  \setlength\tp@right@ymid{\dimexpr((\tp@right@ymax+\tp@right@ymin)/2)\relax}%
}%%    \end{macrocode}

% \begin{docCommand}{@tp@genanchors}{\marg{block name}}
% Generate all 9 anchors (|northwest|, |north|, |northest|, |west|,
% |center|, |east|, |southwest|, |south|, |southest|) for \meta{block name}.
% \end{docCommand}

%    \begin{macrocode}
\def\@tp@genanchors#1{%
  \anchor{#1 north}{\pgf@x=\csname tp@#1@xmid\endcsname \pgf@y=\csname tp@#1@ymax\endcsname}%
  \anchor{#1 south}{\pgf@x=\csname tp@#1@xmid\endcsname \pgf@y=\csname tp@#1@ymin\endcsname}%
  \anchor{#1 west}{\pgf@x=\csname tp@#1@xmin\endcsname \pgf@y=\csname tp@#1@ymid\endcsname}%
  \anchor{#1 northwest}{\pgf@x=\csname tp@#1@xmin\endcsname \pgf@y=\csname tp@#1@ymax\endcsname}%
  \anchor{#1 southwest}{\pgf@x=\csname tp@#1@xmin\endcsname \pgf@y=\csname tp@#1@ymin\endcsname}%
  \anchor{#1 east}{\pgf@x=\csname tp@#1@xmax\endcsname \pgf@y=\csname tp@#1@ymid\endcsname}%
  \anchor{#1 northeast}{\pgf@x=\csname tp@#1@xmax\endcsname \pgf@y=\csname tp@#1@ymax\endcsname}%
  \anchor{#1 southeast}{\pgf@x=\csname tp@#1@xmax\endcsname \pgf@y=\csname tp@#1@ymin\endcsname}%
  \anchor{#1 center}{\pgf@x=\csname tp@#1@xmid\endcsname \pgf@y=\csname tp@#1@ymid\endcsname}%
}% 
%    \end{macrocode}

%    \begin{macrocode}
\newcommand\tp@pgfdeclareanchoralias[3]{%
  \expandafter\def\csname pgf@anchor@#1@#3\expandafter\endcsname
    \expandafter{\csname pgf@anchor@#1@#2\endcsname}}
%    \end{macrocode}

%    \begin{macrocode}
\pgfdeclareshape{page}{
  \backgroundpath{
    \pgfpathmoveto{\pgfpoint{\tp@page@xmin}{\tp@page@ymin}}
    \pgfpathlineto{\pgfpoint{\tp@page@xmin}{\tp@page@ymax}}
    \pgfpathlineto{\pgfpoint{\tp@page@xmax}{\tp@page@ymax}}
    \pgfpathlineto{\pgfpoint{\tp@page@xmax}{\tp@page@xmin}}
    \pgfpathclose
  }
  %% basic anchors
  \anchor{north}{\pgf@x=\tp@page@xmid \pgf@y=\tp@page@ymax}%
  \anchor{south}{\pgf@x=\tp@page@xmid \pgf@y=\tp@page@ymin}%
  \anchor{west}{\pgf@x=\tp@page@xmin \pgf@y=\tp@page@ymid}%
  \anchor{northwest}{\pgf@x=\tp@page@xmin \pgf@y=\tp@page@ymax}%
  \anchor{southwest}{\pgf@x=\tp@page@xmin \pgf@y=\tp@page@ymin}%
  \anchor{east}{\pgf@x=\tp@page@xmax \pgf@y=\tp@page@ymid}%
  \anchor{northeast}{\pgf@x=\tp@page@xmax \pgf@y=\tp@page@ymax}%
  \anchor{southeast}{\pgf@x=\tp@page@xmax \pgf@y=\tp@page@ymin}%
  %\anchor{center}{\pgfpointorigin}
  \anchor{center}{\pgf@x=\tp@page@xmid \pgf@y=\tp@page@ymid}
  \anchor{origin}{\pgfpointorigin}%\pgf@x=0pt \pgf@y=0pt}
  \@tp@genanchors{page}
  %% Body anchors
  \@tp@genanchors{body}
  \@tp@genanchors{marginpar}
  \@tp@genanchors{header}
  \@tp@genanchors{footer}
  \@tp@genanchors{top}
  \@tp@genanchors{bottom}
  \@tp@genanchors{left}
  \@tp@genanchors{right}
}
%    \end{macrocode}


%Create a new |tpx| mark to show anchor location when using
%\refCom{tikzpageputanchors} to display anchors on the page.
%
%    \begin{macrocode}
\newdimen\tp@linewidth
\newdimen\tp@marksize
\setlength\tp@marksize{3pt}
\pgfdeclareplotmark{tpx}{
  \setlength{\tp@linewidth}{\pgflinewidth}
  \pgfsetlinewidth{0.1pt}
  \pgfpathmoveto{\pgfpoint{-\tp@marksize}{-\tp@marksize}}
  \pgfpathlineto{\pgfpoint{\tp@marksize}{\tp@marksize}}
  \pgfpathmoveto{\pgfpoint{-\tp@marksize}{\tp@marksize}}
  \pgfpathlineto{\pgfpoint{\tp@marksize}{-\tp@marksize}}
  \pgfusepathqstroke
  \setlength{\pgflinewidth}{\tp@linewidth}
}
%    \end{macrocode}


% Anchors can be displayed block by block (using
% \cs{tikzpageputanchorsdefaults}, \cs{tikzpageputanchors}
% \cs{tikzpageputanchorsmarginpar}, \cs{tikzpageputanchorsheader},
% \cs{tikzpageputanchorsfooter}, \cs{tikzpageputanchorstop},
% \cs{tikzpageputanchorsright}, \cs{tikzpageputanchorsbottom},
% \cs{tikzpageputanchorsleft}) or globally (using \refCom{tikzpageputanchors}).

%    \begin{macrocode}
\def\tikzpageputanchorsdefaults{
  \foreach \anchor/\placement in {%
    northwest/below right%
    ,north/below%
    ,northeast/below left%
    ,west/right%
    ,center/below%
    ,east/left%
    ,southwest/above right%
    ,south/above%
    ,southeast/above left%
  } \draw[red,shift=(page.\anchor)] plot[mark=tpx%% my plot mark
  ] coordinates{(0,0)}
  node[blue,\placement] {\scriptsize\texttt{(page.\anchor)}};
}

\def\tikzpageputanchorsbody{
  \foreach \anchor/\placement in {%
    body northwest/below right%
    ,body north/below%
    ,body northeast/below left%
    ,body west/right%
    ,body center/below%
    ,body east/left%
    ,body southwest/above right%
    ,body south/above%
    ,body southeast/above left%
  } \draw[red,shift=(page.\anchor)] plot[mark=tpx%% my plot mark
  ] coordinates{(0,0)}
  node[blue,\placement] {\scriptsize\texttt{(page.\anchor)}};
}


\def\tikzpageputanchorsmarginpar{
  \foreach \anchor/\placement in {%
    marginpar northwest/below left%
    ,marginpar north/left%
    ,marginpar northeast/above left%
    ,marginpar west/below%
    ,marginpar center/below%
    ,marginpar east/above%
    ,marginpar southwest/below right%
    ,marginpar south/right%
    ,marginpar southeast/above right%
  } \draw[red,shift=(page.\anchor)] plot[mark=tpx%% my plot mark
  ] coordinates{(0,0)}
  node[blue,\placement, rotate=90] {\scriptsize\texttt{(page.\anchor)}};
}


\def\tikzpageputanchorsheader{
  \foreach \anchor/\placement in {%
    header northwest/above right%
    ,header north/above%
    ,header northeast/above left%
    ,header west/right%
    ,header center/right%
    ,header east/left%
    ,header southwest/below right%
    ,header south/below%
    ,header southeast/below left%
  } \draw[red,shift=(page.\anchor)] plot[mark=tpx%% my plot mark
  ] coordinates{(0,0)}
  node[blue,\placement] {\scriptsize\texttt{(page.\anchor)}};
}


\def\tikzpageputanchorsfooter{
  \foreach \anchor/\placement in {%
    footer northwest/above right%
    ,footer north/above%
    ,footer northeast/above left%
    ,footer west/right%
    ,footer center/right%
    ,footer east/left%
    ,footer southwest/below right%
    ,footer south/below%
    ,footer southeast/below left%
  } \draw[red,shift=(page.\anchor)] plot[mark=tpx%% my plot mark
  ] coordinates{(0,0)}
  node[blue,\placement] {\scriptsize\texttt{(page.\anchor)}};
}

\def\tikzpageputanchorstop{
  \foreach \anchor/\placement in {%
    top northwest/below right%
    ,top north/below%
    ,top northeast/below left%
    ,top west/right%
    ,top center/below%
    ,top east/left%
    ,top southwest/above right%
    ,top south/above%
    ,top southeast/above left%
  } \draw[red,shift=(page.\anchor)] plot[mark=tpx%% my plot mark
  ] coordinates{(0,0)}
  node[blue,\placement] {\scriptsize\texttt{(page.\anchor)}};
}


\def\tikzpageputanchorsbottom{
  \foreach \anchor/\placement in {%
    bottom northwest/below right%
    ,bottom north/below%
    ,bottom northeast/below left%
    ,bottom west/right%
    ,bottom center/below%
    ,bottom east/left%
    ,bottom southwest/above right%
    ,bottom south/above%
    ,bottom southeast/above left%
  } \draw[red,shift=(page.\anchor)] plot[mark=tpx%% my plot mark
  ] coordinates{(0,0)}
  node[blue,\placement] {\scriptsize\texttt{(page.\anchor)}};
}


\def\tikzpageputanchorsleft{
  \foreach \anchor/\placement in {%
    left northwest/below left%
    ,left north/left%
    ,left northeast/above left%
    ,left west/below%
    ,left center/below%
    ,left east/above%
    ,left southwest/below right%
    ,left south/right%
    ,left southeast/above right%
  } \draw[red,shift=(page.\anchor)] plot[mark=tpx%% my plot mark
  ] coordinates{(0,0)}
  node[blue,\placement, rotate=90] {\scriptsize\texttt{(page.\anchor)}};
}

\def\tikzpageputanchorsright{
  \foreach \anchor/\placement in {%
    right northwest/below left%
    ,right north/left%
    ,right northeast/above left%
    ,right west/below%
    ,right center/below%
    ,right east/above%
    ,right southwest/below right%
    ,right south/right%
    ,right southeast/above right%
  } \draw[red,shift=(page.\anchor)] plot[mark=tpx%% my plot mark
  ] coordinates{(0,0)}
  node[blue,\placement, rotate=90] {\scriptsize\texttt{(page.\anchor)}};
}
%    \end{macrocode}

% \begin{docCommand}{tikzpageputanchors}{}
% A simple short hand to display all anchors at once.
% \end{docCommand}
%    \begin{macrocode}
\def\tikzpageputanchors{
  \tikzpageputanchorsdefaults
  \tikzpageputanchorsbody
  \tikzpageputanchorsmarginpar
  \tikzpageputanchorsheader
  \tikzpageputanchorsfooter
  \tikzpageputanchorstop
  \tikzpageputanchorsbottom
  \tikzpageputanchorsleft
  \tikzpageputanchorsright
}
%    \end{macrocode}



% \begin{docCommand}{tpshowframes}{}
% 
% Display |top|, |right|, |bottom| and |left| block using a specific
% background. This can be used in conjunction with \refCom{tikzpageputanchors} for
% debuging purposes.
% 
% \end{docCommand}
%    \begin{macrocode}
\def\tpshowframes{
  \draw[fill=blue!50, opacity=.3, draw] (page.bottom northwest) rectangle (page.bottom southeast);
  \draw[fill=yellow!50, opacity=.3, draw] (page.top northwest) rectangle (page.top southeast);
  \draw[fill=red!50, opacity=.3, draw] (page.left northwest) rectangle (page.left southeast);
  \draw[fill=green!50, opacity=.3, draw] (page.right northwest) rectangle (page.right southeast);
}
%    \end{macrocode}


% \begin{docCommand}{tpfancyhdrdefault}{}
% An example to display headers and footer as |fancyhdr| does.
% \end{docCommand}
%    \begin{macrocode}
\def\tpfancyhdrdefault{
  \node [outer sep=0,inner sep=0, anchor=mid] at (page.header center) {};
  \node [outer sep=0,inner sep=0, anchor=mid east] at (page.header east) {\tpflip{\sl\leftmark}{\sl\rightmark}};
  \node [outer sep=0,inner sep=0, anchor=mid west] at (page.header west) {\tpflip{\sl\rightmark}{\sl\leftmark}};
  \node [outer sep=0,inner sep=0, anchor=base east] at (page.footer east) {};
  \node [,outer sep=0,inner sep=0,anchor=base] at (page.footer center) {\thepage};
  \node [outer sep=0,inner sep=0, anchor=base west] at (page.footer west) {};
}
%    \end{macrocode}

% \begin{docCommand}{tikzpage}{}
% Generate a |tikzpicture| for the whole page. if a \cs{tikzpagelayout}
% command exists, it will be executed.
% \end{docCommand}
%    \begin{macrocode}
\newcommand{\tikzpage}{
  \if@tp@use@textpos
  \begin{textblock*}{\textwidth}[0,0](0pt,0pt)%
    \fi
    \tp@compute@margins%
    \if@tp@use@textpos
    \begin{tikzpicture}[]%
      \clip (0,0) rectangle (\paperwidth, \paperheight);
      \else
      \begin{tikzpicture}[remember picture, overlay]%
      \fi
      \if@tp@use@textpos
      \node[anchor=origin,shape=page] (page) {};
      \else
      \node[anchor=origin,shape=page] (page) at (current page.south west) {};
      \fi
      \@ifundefined{tikzpagelayout}{}{\tikzpagelayout}
    \end{tikzpicture}%
  \if@tp@use@textpos
  \end{textblock*}%
  \fi
}
%    \end{macrocode}


\fancypagestyle{plain}{
  \fancyhf{}
  \chead{\tikzpage}
}
\fancypagestyle{empty}{
  \fancyhf{}
  \chead{\tikzpage}
}


%
\makeatother
% \iffalse
%</tikz-page>
% \fi
%
% \Finale
%
% 
% \iffalse
% <*example>
% \fi
% \begin{thebibliography}{9}

% \bibitem{Graphical Decoration} Trying to do graphical decorations in
% “ClassicThesis style” \url{http://tex.stackexchange.com/questions/86294}
% \end{thebibliography}
% \iffalse
% </example>
% \fi

\endinput
