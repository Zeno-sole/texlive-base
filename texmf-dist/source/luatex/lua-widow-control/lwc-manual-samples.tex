% lua-widow-control
% https://github.com/gucci-on-fleek/lua-widow-control
% SPDX-License-Identifier: MPL-2.0+ OR CC-BY-SA-4.0+
% SPDX-FileCopyrightText: 2022 Max Chernoff

\startbuffer[demo-text]
    \definepapersize[smallpaper][
        width=6cm,
        height=8.3cm
    ]\setuppapersize[smallpaper]

    \setuplayout[
        topspace=0.1cm,
        backspace=0.1cm,
        width=middle,
        height=middle,
        header=0pt,
        footer=0pt,
    ]

    \def\lwc/{\sans{lua-\allowbreak widow-\allowbreak control}}
    \def\Lwc/{\sans{Lua-\allowbreak widow-\allowbreak control}}

    \setupbodyfont[9pt]
    \setupindenting[yes, 2em]

    \definepalet[layout][grid=middlegray]
    \showgrid[nonumber, none, lines]

    \definefontfeature[default][default][expansion=quality,protrusion=quality]

    \usetypescript[modern-base]
    \setupbodyfont[reset,modern]

    \setupalign[hz,hanging,tolerant]

    \setuplanguage[en][spacing=packed]

    \starttext
        \Lwc/ can remove most widows and orphans from a document, \emph{without} stretching any glue or shortening any pages.

        It does so by automatically lengthening a paragraph on a page where a widow or orphan would otherwise occur. While \TeX{} breaks paragraphs into their natural length, \lwc/ is breaking the paragraph 1~line longer than its natural length. \TeX{}'s paragraph is output to the page, but \lwc/'s paragraph is just stored for later. When a widow or orphan occurs, \lwc/ can take over. It selects the previously-saved paragraph with the least badness; then, it replaces \TeX{}'s paragraph with its saved paragraph. This increases the text block height of the page by 1~line.

        Now, the last line of the current page can be pushed to the top of the next page. This removes the widow or the orphan without creating any additional work.
    \stoptext
\stopbuffer
\savebuffer[list=demo-text]

\startbuffer[shorten]
    \parskip=0pt
    \input lwc-manual-demo-text.tmp
\stopbuffer

\startbuffer[shorten-code]
    \parskip=0pt

    \clubpenalty=10000
    \widowpenalty=10000
\stopbuffer

\startbuffer[stretch]
    \parskip=0pt plus 1fill
    \input lwc-manual-demo-text.tmp
\stopbuffer

\startbuffer[stretch-code]
    \parskip=0pt plus 1fill

    \clubpenalty=10000
    \widowpenalty=10000
\stopbuffer

\startbuffer[ignore]
    \startsetups[*default]
        \clubpenalty=0
        \widowpenalty=0
        \displaywidowpenalty=0
        \interlinepenalty=0
        \brokenpenalty=0
    \stopsetups

    \setups[*default]

    \input lwc-manual-demo-text.tmp
\stopbuffer

\startbuffer[ignore-code]
    \parskip=0pt

    \clubpenalty=0
    \widowpenalty=0
\stopbuffer

\startbuffer[lwc]
    \usemodule[lua-widow-control]
    \input lwc-manual-demo-text.tmp
\stopbuffer

\startbuffer[lwc-code]
    \usepackage{lua-widow-control}
\stopbuffer

\startbuffer[widow-orphan]
    % TODO This is all really quite hacky
    \define[1]\rulewords{\dorecurse{#1}{\blackrule[height=1.5ex, width=1em] \blackrule[height=1.5ex, width=2em] \blackrule[height=1.5ex, width=1.5em] \blackrule[height=1.5ex, width=3em] }}

    \define[2]\fakestart{\framed[width=broad, align=normal, frame=off]{\parfillskip=0pt\spaceskip=0.2em plus 1fill\hskip 5em\rulewords{#1}\blackrule[height=1.5ex, width=#2]}}

    \define[2]\fakeend{\framed[width=broad, align=normal, frame=off]{\parfillskip=5em\spaceskip=0.2em plus 1fill\rulewords{#1}\blackrule[height=1.5ex, width=#2]}}

    \setupTABLE[width=broad, frame=off]
    \setupTABLE[row][1][align=middle, style=\ssbf]
    \setupTABLE[row][2][align=low, frame=on]
    \setupTABLE[row][3][align=high, frame=on]
    \setupTABLE[column][2][frame=off, width=1em]
    \startTABLE
        \NC Widow \NC\NC Orphan \NC\NR
        \NC\fakestart{5}{1.5em}\NC\NC\fakestart{1}{2em}\NC\NR
        \NC\fakeend{1}{2em}\NC\NC\fakeend{5}{1.5em}\NC\NR
    \stopTABLE
\stopbuffer

\startbuffer[nobreak]
    % This is also really hacky and terrible
    \parfillskip=0pt
    \define\lineone{%
        \hbox to 0.3\textwidth{\blackrule[height=1.5ex, width=1em]\hfill%
        \blackrule[height=1.5ex, width=3em]\hfill%
        \blackrule[height=1.5ex, width=1em]\hfill%
        \blackrule[height=1.5ex, width=3em]\hfill%
        \blackrule[height=1.5ex, width=1em]}%
    }

    \define\linetwo{%
        \hbox to 0.3\textwidth{\blackrule[height=1.5ex, width=2em]\hfill%
        \blackrule[height=1.5ex, width=1em]\hfill%
        \blackrule[height=1.5ex, width=2.5em]\hfill%
        \blackrule[height=1.5ex, width=1em]\hfill%
        \blackrule[height=1.5ex, width=2.5em]}%
    }

    \define\linethree{%
        \hbox to 0.3\textwidth{\blackrule[height=1.5ex, width=1em]\hfill%
        \blackrule[height=1.5ex, width=3em]\hfill%
        \blackrule[height=1.5ex, width=1em]\hfill%
        \blackrule[height=1.5ex, width=3em]\hfill%
        \hskip1em\relax}%
    }

    \define\heading{\bold{Heading}}

    \setupTABLE[
        width=broad,
        topframe=off,
        bottomframe=off,
        leftframe=on,
        rightframe=on,
        height=1.2\baselineskip
    ]
    \setupTABLE[row][1][
        align=middle,
        style=\ttbf,
        bottomframe=on,
        leftframe=off,
        rightframe=off
    ]
    \setupTABLE[row][3][bottomframe=on]
    \setupTABLE[row][5][bottomframe=on]
    \setupTABLE[column][2][bottomframe=off, width=1em]
    \setupTABLE[column][4][bottomframe=off, width=1em]
    \startTABLE
        \NC keep     \NC\NC split    \NC\NC warn     \NC\NR
        \NC          \NC\NC          \NC\NC \heading \NC\NR
        \NC          \NC\NC \heading \NC\NC \lineone    \NC\NR
        \NC \heading \NC\NC \lineone \NC\NC \linetwo    \NC\NR
        \NC \lineone \NC\NC \linetwo \NC\NC \linethree  \NC\NR
    \stopTABLE
\stopbuffer
