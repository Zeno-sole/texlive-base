% \iffalse meta-comment
%
% Copyright (C) 2009-2013 by Élie Roux <elie.roux@telecom-bretagne.eu>
% Copyright (C) 2010, 2011 by Manuel Pégourié-Gonnard <mpg@elzevir.fr>
% Copyright (C) 2015 David Carlisle and Joseph Wright
%
% It may be distributed and/or modified under the conditions of
% the LaTeX Project Public License (LPPL), either version 1.3c of
% this license or (at your option) any later version.  The latest
% version of this license is in the file:
%
%   http://www.latex-project.org/lppl.txt
%
%<emu>\ifx\BeginCatcodeRegime\undefined\else\expandafter\endinput\fi
%<tex,sty>
%<tex,sty>\ifx
%<sty>  \ProvidesPackage\undefined\begingroup\def\ProvidesPackage
%<tex>  \ProvidesFile\undefined\begingroup\def\ProvidesFile
%<tex,sty>  #1#2[#3]{\endgroup\immediate\write-1{Package: #1 #3}}
%<tex,sty>\fi
%<emu>\ProvidesPackage{luatexbase}
%<emu-cmp>\ProvidesPackage{luatexbase-compat}
%<emu-mod>\ProvidesPackage{luatexbase-modutils}
%<emu-loa>\ProvidesPackage{luatexbase-loader}
%<emu-reg>\ProvidesPackage{luatexbase-regs}
%<emu-att>\ProvidesPackage{luatexbase-attr}
%<emu-cct>\ProvidesPackage{luatexbase-cctb}
%<emu-mcb>\ProvidesPackage{luatexbase-mcb}
%<*driver>
\ProvidesFile{luatexbase.dtx}
%</driver>
%<*tex,sty>
[2015/10/04 v1.3
%</tex,sty>
%<emu>  luatexbase interface to LuaTeX
%<emu-cmp>  luatexbase interface to LuaTeX (legacy subpackage)
%<emu-mod>  luatexbase interface to LuaTeX (legacy subpackage)
%<emu-loa>  luatexbase interface to LuaTeX (legacy subpackage)
%<emu-reg>  luatexbase interface to LuaTeX (legacy subpackage)
%<emu-att>  luatexbase interface to LuaTeX (legacy subpackage)
%<emu-cct>  luatexbase interface to LuaTeX (legacy subpackage)
%<emu-mcb>  luatexbase interface to LuaTeX (legacy subpackage)
%<*tex,sty>
]
%</tex,sty>
%<*driver>
\documentclass{ltxdoc}
\GetFileInfo{luatexbase.dtx}
\begin{document}
\title{\filename\\(Lua\TeX{}-specific support, luatexbase interface)}
\author{David Carlisle and Joseph Wright\footnote{Significant portions
  of the code here are adapted/simplified from the packages \textsf{luatex} and
  \textsf{luatexbase} written by Heiko Oberdiek, \'{E}lie Roux,
  Manuel P\'{e}gouri\'{e}-Gonnar and Philipp Gesang.}}
\date{\filedate}
\maketitle
\setcounter{tocdepth}{2}
\tableofcontents
\DocInput{\filename}
\end{document}
%</driver>
% \fi
%
% \section{Overview}
%
% Lua\TeX{} adds a number of engine-specific functions to \TeX{}. Support
% for those is now available for this area in the \LaTeX{} kernel and as
% an equivalent stand-alone file |ltluatex.tex| for plain users. The
% functionality there is derived from the earlier \textsf{luatex}
% and \textsf{luatexbase} packages by Heiko Oberdiek, \'{E}lie Roux,
% Manuel P\'{e}gouri\'{e}-Gonnar and Philipp Gesang. However, the
% interfaces are not all identical.
%  
% The interfaces defined in this package are closely modelled on the original
% \textsf{luatexbase} package, and provide a compatibility layer between
% the new kernel-level support and existing code using \textsf{luatexbase}.
%
% \section{The \textsf{luatexbase} package interface}
%
% \subsection
% [Catcode tables]
% {Catcode tables\footnote{This
% interface was previously defined in the \textsf{luatexbase-cctbl}
% sub-package.}}
%
% \subsubsection{\TeX}
% 
%
% \noindent
% \DescribeMacro{\CatcodeTableIniTeX}
% \DescribeMacro{\CatcodeTableString}
% \DescribeMacro{\CatcodeTableLaTeX}
% \DescribeMacro{\CatcodeTableLaTeXAtLetter}
% \DescribeMacro{\CatcodeTableOther}
% \DescribeMacro{\CatcodeTableExpl}
% \TeX\ access to predefined catcode tables.
%
% The first four tables are aliases giving alternative names for some
% catcodetables that are defined in the \textsf{ltluatex} core.
%
% |\CatcodeTableOther| is like |\CatcodeTableString| except that
% the catcode of space is $12$ (other).
%
% |\CatcodeTableExpl| is similar to the environment set by the
% \textsf{expl3} command |\ExplSyntaxOn| note that this only affects
% catcode settings, not for example |\endlinechar|.
%
% One difference between this implementation and the tables defined
% in the earlier \textsf{luatexbase} package is that these tables are
% defined to match the settings used by \LaTeX\ over the full Unicode
% range (as set in the file \textsf{unicode-letters.def}).
%
% \noindent
% \DescribeMacro{\SetCatcodeRange}
% An alias for |\@setrangecatcode| which is defined in the
% \textsf{ctablestack} package imported into this version of
% \textsf{luatexbase}. (The order of arguments is the
% same despite the variation in the naming). This is useful for
% setting up a new catcode table and assigns a given catcode to a
% range of characters.
% 
% \noindent
% \DescribeMacro{\BeginCatcodeRegime}
% \DescribeMacro{\EndCatcodeRegime}
% A simple wrapper around |\@pushcatcodetable| providing a slightly
% different interface. The usage is:\\%
% \verb|\BeginCatcodeRegime|\meta{catcode table}\\
% \verb|   |\meta{code}\\
% \verb|\EndCatcodeRegime|
%
% \noindent
% \DescribeMacro{\PushCatcodeTableNumStack}
% \DescribeMacro{\PopCatcodeTableNumStack}
% These are defined to be aliases for |\@pushcatcodetable| and
% |\@popcatcodetable| although the actual implementation is quite different
% to the older packages, the use of the commands should match.
%
% \noindent
% \DescribeMacro{\newluatexcatcodetable}
% \DescribeMacro{\setluatexcatcodetable}
% Aliases for the \textsf{ltluatex} functions
% dropping |luatex|  from the name to match the convention of not
% using  |luatex|-prefixed names for the Lua\TeX\ primitives.
%
% \subsubsection{Lua}
%
% The standard way to access catcode table numbers from Lua in
% \textsf{ltluatex} is the |registernumber| function.  This
% package provides a |catcodetables| table with a metatable
% that accesses this function and is extended with aliases for the
% predefined tables so you can use |catcodetables.expl| as an
% alternative to |catcodetables.CatcodeTableExpl|, both being
% equivalent to\\
% |registernumber('CatcodeTableExpl')|.
%
%
% \subsection
% [Lua Callbacks]
% {Lua Callbacks\footnote{This
% interface was previously defined in the \textsf{luatexbase-mcb}
% sub-package.}}
%
% The |luatexbase| table is extended with some additional Lua
% functions to provide the interfaces provided by the previous
% implementation.
%
%
% \noindent
% \DescribeMacro{priority\_in\_callback}\meta{name}\meta{description}\\%
% As in the earlier interfaces the function
% is provided to return a number indicating the position of a
% specified function in a callback list. However it is usually used
% just as a boolean test that the function is registered with the
% callback. Kernel-level support does not directly expose the priority numbers,
% however the function here is defined to return the number of the specified
% function in the list returned by |luatexbase.callback_descriptions|.
%
%
% \noindent
% \DescribeMacro{is\_active\_callback}\meta{name}\meta{description}\\%
% This boolean function was defined in the development sources
% of the previous  implementation. Here it is defined as an alias for
% the function |in_callback| provided by \textsf{ltluatex}.
% Given a callback and a description string, it returns true if a
% callback function with that description is currently registered.
%
% \noindent
% \DescribeMacro{reset\_callback}\meta{name}\meta{make\_false}\\
% This function unregisters all functions registered for the callback
% \meta{name}. If \meta{make\_false} is true, the callback is then set
% to false (rather than nil). Unlike the earlier implementation
% This version does call |remove_from_callback| on each function in
% the callback list for \meta{name}, and each removal will be recorded
% in the log.
%
% \noindent
% \DescribeMacro{remove\_from\_callback}\meta{name}\meta{description}\\%
% This function is unchanged from the kernel-level implementation.
% It is backward compatible with the previous \textsf{luatexbase}
% package but enhanced as it returns the removed callback and its
% description.
% Together with the |callback_descriptions| function this allows much
% finer control over the order of functions in a callback list as the
% functions can be removed then re-added to the list in any desired order.
%
% \noindent
% \DescribeMacro{add\_to\_callback}\meta{name}\meta{function}\meta{description}\meta{priority}\\%
% This function is defined as a wrapper around the kernel-level
% implementation, which does not have the fourth \meta{priority}
% argument. 
%
% If multiple callbacks are registered to a callback of type
% \textsf{exclusive} then \textsf{ltluatex} raises an error, but
% here it is allowed if \texttt{priority} is $1$, in which case the
% \verb|reset_callback| is first called to remove the existing
% callback.
%
% In general the \texttt{priority} argument is implemented by 
% temporarily removing some callbacks from the list and replacing
% them after having added the new callback.
%
% \noindent
% \DescribeMacro{create\_callback}\meta{name}\meta{type}\meta{default}\\%
% This function is unchanged from kernel-level
% implementation, the only change is a change of terminology for the
% types of callback, the type |first| is now classified as |exclusive|
% and the kernel code raises an error if multiple callback functions
% are registered. The previous \textsf{luatexbase} implementation
% allowed multiple functions to be registered, but only activated the
% first in the list.
%
% \subsection
% [Module declaration]
% {Module declaration\footnote{This
% interface was previously defined in the \textsf{luatexbase-modutils}
% sub-package.}}
%
% \subsubsection{\TeX}
%
% \noindent
% \DescribeMacro{\RequireLuaModule}\meta{file}\oarg{info}\\
% This command is provided as a wrapper around
% |\directlua{require(|\meta{file}|}|, and executes the Lua code in
% the specified file.
% The optional argument is accepted but ignored.
% 
% Current versions of Lua\TeX\ all use the |kpse| \TeX\ path searching
% library with the |require| function, so the more complicated
% definition used in earlier implementations is no
% longer needed.
%
% \subsubsection{Lua}
% 
% \noindent
% \DescribeMacro{provides\_module}\meta{info}\\
% The \textsf{luatexbase} version of |provides_module| returns a list of log
% and error functions so that it is usually called as:\\
% |local err, warning, info, log = luatexbase.provides_module({name=..|
%
% The returned functions are all instances of the functions provided by
% the kernel: |module_error|,
% |module_warning| and |module_info|, They all use their first argument
% as a format string fo rany later arguments.
%
% \DescribeMacro{errwarinf}\meta{name}\\
% Returns four error and warning functions associated with \meta{name}
% mostly a helper function for \verb|provides\_module|, but can be called
% separately.  
%
% \subsection
% [Lua Attributes]
% {Lua Attributes and Whatsits\footnote{This
% interface was previously defined in the \textsf{luatexbase-attr}
% sub-package.}}
%
%
% \subsubsection{\TeX}
%
% \noindent
% \DescribeMacro{\newluatexattribute}
% \DescribeMacro{\setluatexattribute}
% \DescribeMacro{\unsetluatexattribute}
% As for catcode tables, aliases for the attribute allocation functions are
% provided with |luatex| in the names.
%
%
% \subsubsection{Lua}
% The lua code in this section is concerned with an experimental
% whatsit handling suite of functions in the original package.
% This is not fully documented here and is guraded by the
% \textsf{docstrip} guard \verb|whatsit| so it may optionally be
% included or excluded from the sources when the package is built.
%
% \subsection{Prefixed names for lua\TeX\ primitives}
% \noindent
% \DescribeMacro{\luatexattributedef}
% \DescribeMacro{\luatexcatcodetable}
% \DescribeMacro{\luatexluaescapestring}
% \DescribeMacro{\luatexlatelua}
% \DescribeMacro{\luatexoutputbox}
% \DescribeMacro{\luatexscantextokens}
% Aliases for commonly ued lua\TeX\ primitives that existing packages
% using \textsf{luatexbase} use with prefixed names.
%
% If additional primtives are required it is recommended that the
% code is updated to use unprefixed names. To ensure that the code
% works with the original \textsf{luatexbase} package on older formats
% you may use the lua function \texttt{tex.enableprimitives} to enable
% some or all primitives to be available with unprefixed names.
%
% \StopEventually{}
%
% \section{Implementation}
%
% \subsection{\textsf{luatexbase} interface}
%
%    \begin{macrocode}
%<*emu>
\edef\emuatcatcode{\the\catcode`\@}
\catcode`\@=11
%    \end{macrocode}
%
% Load |ctablestack|.
%    \begin{macrocode}
\ifx\@setrangecatcode\@undefined
  \ifx\RequirePackage\@undefined
    \ifx
  \ProvidesPackage\undefined\begingroup\def\ProvidesPackage
    #1#2[#3]{\endgroup\immediate\write-1{Package: #1 #3}}
\fi
\ProvidesPackage{ctablestack}
  [2015/10/01 v1.0 Catcode table stable support]
\edef\ctstackatcatcode{\the\catcode`\@}
\catcode`\@=11
\ifx\newluafunction\@undefined
  %%
%% This is file `ltluatex.tex',
%% generated with the docstrip utility.
%%
%% The original source files were:
%%
%% ltluatex.dtx  (with options: `tex,plain')
%% 
%% This is a generated file.
%% 
%% The source is maintained by the LaTeX Project team and bug
%% reports for it can be opened at https://latex-project.org/bugs.html
%% (but please observe conditions on bug reports sent to that address!)
%% 
%% 
%% Copyright (C) 1993-2021
%% The LaTeX Project and any individual authors listed elsewhere
%% in this file.
%% 
%% This file was generated from file(s) of the LaTeX base system.
%% --------------------------------------------------------------
%% 
%% It may be distributed and/or modified under the
%% conditions of the LaTeX Project Public License, either version 1.3c
%% of this license or (at your option) any later version.
%% The latest version of this license is in
%%    https://www.latex-project.org/lppl.txt
%% and version 1.3c or later is part of all distributions of LaTeX
%% version 2008 or later.
%% 
%% This file has the LPPL maintenance status "maintained".
%% 
%% This file may only be distributed together with a copy of the LaTeX
%% base system. You may however distribute the LaTeX base system without
%% such generated files.
%% 
%% The list of all files belonging to the LaTeX base distribution is
%% given in the file `manifest.txt'. See also `legal.txt' for additional
%% information.
%% 
%% The list of derived (unpacked) files belonging to the distribution
%% and covered by LPPL is defined by the unpacking scripts (with
%% extension .ins) which are part of the distribution.
\ifx\newluafunction\undefined\else\expandafter\endinput\fi
\ifx
  \ProvidesFile\undefined\begingroup\def\ProvidesFile
  #1#2[#3]{\endgroup\immediate\write-1{File: #1 #3}}
\fi
\ProvidesFile{ltluatex.tex}%
[2021/04/18 v1.1t
  LuaTeX support for plain TeX (core)
]
\edef\etatcatcode{\the\catcode`\@}
\catcode`\@=11
\ifnum\luatexversion<60 %
  \wlog{***************************************************}
  \wlog{* LuaTeX version too old for ltluatex support *}
  \wlog{***************************************************}
  \expandafter\endinput
\fi
\long\def\@gobble#1{}
\long\def\@firstofone#1{#1}
\directlua{tex.enableprimitives("",tex.extraprimitives("luatex"))}
\ifx\e@alloc\@undefined
  \ifx\documentclass\@undefined
    \ifx\loccount\@undefined
      % \iffalse meta-comment
%
% Copyright 1997, 1998, 2008 2015 2016 LaTeX Project and Peter Breitenlohner.
% 
% This file (etex.sty) may be distributed and/or modified under the
% conditions of the LaTeX Project Public License, either version 1.3 of
% this license or (at your option) any later version.  The latest
% version of this license is in
%   http://www.latex-project.org/lppl.txt
% and version 1.3 or later is part of all distributions of LaTeX
% version 2003/12/01 or later.
% 
% This work has the LPPL maintenance status "maintained".
% 
% The Current Maintainer of this work is David Carlisle.
% https://github.com/davidcarlisle/dpctex/issues
% \fi

\NeedsTeXFormat{LaTeX2e}
\ProvidesPackage{etex}
%        [1997/08/12 v0.1 eTeX basic definition package (DPC)]
%        [1998/03/26 v2.0 eTeX basic definition package (PEB)]
%        [2015/03/02 v2.1 eTeX basic definition package (PEB,DPC)]
%        [2015/07/06 v2.2 eTeX basic definition package (PEB,DPC)]
%        [2015/07/08 v2.3 eTeX basic definition package (PEB,DPC)]
%        [2015/09/02 v2.4 eTeX basic definition package (PEB,DPC)]
%        [2016/01/07 v2.5 eTeX basic definition package (PEB,DPC)]
%        [2016/01/11 v2.6 eTeX basic definition package (PEB,DPC)]
         [2016/08/01 v2.7 eTeX basic definition package (PEB,DPC)]

%%%%%%%%%%%%%%%%%%%%%%%%%%%%%%%%%%%%%%%%%%%%%%%%%%%%%%%%%%%%%%%%%%%%%%%%

%% A basic interface to some etex primitives, closely modeled on
%% etex.src and etexdefs.lib provided by the core etex team.

%% The etex.src `module' system is not copied here, the standard
%% LaTeX package option mechanism is used instead,
%% however the package options match the module names.
%% (Currently grouptypes, interactionmodes, nodetypes, iftypes.)
%% The individual type names are different too: We use, e.g.,
%%
%% `\bottomleveltype' and `\simplegrouptype' instead of
%% `\grouptypes{bottomlevel}' and `\grouptypes{simple}'.

%%%%%%%%%%%%%%%%%%%%%%%%%%%%%%%%%%%%%%%%%%%%%%%%%%%%%%%%%%%%%%%%%%%%%%%%

%% Other Comments...

%% The names of the `interactionmodes' are not too good.
%% In particular \scroll and \batch are likely to clash with existing
%% uses. These names have been changed into \batchinteractionmode,
%% \scrollinteractionmode etc.
%% Similarly, the names of the `groupetypes' have been changed, in
%% particular \mathgroup would conflict with the LaTeX kernel.

%% \etex logo could have the same trick as \LaTeXe to pick up a bold
%% epsilon when needed. (Not done here, I hate wasting tokens on logos.)
%% This version does have a \m@th not in the original.

%% The \globcountvector, \loccountvector, etc. allocation macros are
%% not (yet) implemented.

%% Currently if run on a standard TeX, the package generates an error.
%% Perhaps it should instead load some code to try to fake
%% the new etex primitives in that case???
%% Likewise, the package generates an error when used with e-TeX V 1

%% The etex.src language mechanism is not copied here. That facility
%% does not use any of the etex features. LaTeX should be customised
%% using the same hyphen.cfg mechanism as for a format built with a
%% standard TeX.

%% David Carlisle

%% Upgraded for e-TeX V 2.0
%% Peter Breitenlohner

%%%%%%%%%%%%%%%%%%%%%%%%%%%%%%%%%%%%%%%%%%%%%%%%%%%%%%%%%%%%%%%%%%%%%%%%



\ifx\eTeXversion\@undefined
  \PackageError{etex}
    {This package may only be run using an\MessageBreak
     etex in extended mode}
    {Perhaps you forgot the `*' when making the format with (e)initex.%
    }
\fi

\ifnum\eTeXversion<2
  \PackageError{etex}
    {This package requires e-TeX V 2}
    {You are probably using the obsolete e-TeX V 1.%
    }
\fi

% 2.2
% Check if the new latex 2015/01/01 allocation is already using
% extended reisters. If so it is too late to change allocation scheme.
% Older versions of LaTeX would have given an error when the classic
% TeX registers were all allocated, but newer formats allocate from
% the extended range, so usually this package is not needed.
\@tempswafalse
\ifnum\count10>\@cclv\@tempswatrue\else
\ifnum\count11>\@cclv\@tempswatrue\else
\ifnum\count12>\@cclv\@tempswatrue\else
\ifnum\count13>\@cclv\@tempswatrue\else
\ifnum\count14>\@cclv\@tempswatrue\else
\ifnum\count15>\@cclv\@tempswatrue
\fi\fi\fi\fi\fi\fi

\if@tempswa
\PackageWarningNoLine{etex}{%
Extended allocation already in use.\MessageBreak
etex.sty code will not be used.\MessageBreak
To force etex package to load, add\MessageBreak
\string\RequirePackage{etex}\MessageBreak
at the start of the document}

% 2.5 define the global allocation to be the standard ones
% as extended allocation is already in use. Helps with
% compatibility with some packages that use these commands 
% after loading etex.
% 2.6 avoid error from outer if used with (e)plain
\expandafter\let\csname globcount\expandafter\endcsname
                \csname newcount\endcsname
\expandafter\let\csname globdimen\expandafter\endcsname
                \csname newdimen\endcsname
\expandafter\let\csname globskip\expandafter\endcsname
                \csname newskip\endcsname
\expandafter\let\csname globmuskip\expandafter\endcsname
                \csname newmuskip\endcsname
\expandafter\let\csname globtoks\expandafter\endcsname
                \csname newtoks\endcsname
\expandafter\let\csname globmarks\expandafter\endcsname
                \csname newmarks\endcsname
% end of 2.5/2.6 change

\expandafter\endinput\fi

% End of 2.2 addition.

% 2.3 move option handling after the above error checks.
\DeclareOption{grouptypes}{\catcode`\G=9}
\DeclareOption{interactionmodes}{\catcode`\I=9}
\DeclareOption{nodetypes}{\catcode`\N=9}
\DeclareOption{iftypes}{\catcode`\C=9}
\DeclareOption{localalloclog}{\let\et@xwlog\wlog} % the default
\DeclareOption{localallocnolog}{\let\et@xwlog\@gobble} % be quiet
\DeclareOption{localallocshow}{\let\et@xwlog\typeout} % debugging
% End of 2.3 addition.

% v2.7
% \extrafloats does not work with this package
% but make it give a sensible error, not mis-parse \ifnum.
%
% Note that using \extrafloats earlier might not be safe as
% it could in principle clash with registers used for local allocation.
% However it probably works (as local allocation is used locally...).
% A better fix would be not to load this package with current LaTeX.
% This current etex package is just provided to force the old behaviour
% and such documents should not be using new features such as \extrafloats).
\ifdefined\extrafloats
\def\extrafloats#1{%
  \PackageError{etex}{%
    \noexpand\extrafloats is incompatible with etex.sty allocation.\MessageBreak
    Try using \noexpand\extrafloats before loading etex}%
    \@ehc}
\fi
% end of v2.7 change

\def\eTeX{%
  $\m@th\varepsilon$-\TeX}

\def\tracingall{%
  \tracingcommands\thr@@        % etex
  \tracingstats\tw@
  \tracingpages\@ne
  \tracinglostchars\tw@         % etex
  \tracingmacros\tw@
  \tracingparagraphs\@ne
  \tracingrestores\@ne
  \tracinggroups\@ne            % etex
  \tracingifs\@ne               % etex
  \tracingscantokens\@ne        % etex
  \tracingnesting\@ne           % etex
  \tracingassigns\@ne           % etex
  \errorcontextlines\maxdimen
  \showoutput}

\def\loggingall{%
  \tracingall
  \tracingonline\z@}

\def\tracingnone{%
  \tracingonline\z@
  \showboxdepth\m@ne
  \showboxbreadth\m@ne
  \tracingoutput\z@
  \errorcontextlines\m@ne
  \tracingassigns\z@
  \tracingnesting\z@
  \tracingscantokens\z@
  \tracingifs\z@
  \tracinggroups\z@
  \tracingrestores\z@
  \tracingparagraphs\z@
  \tracingmacros\z@
  \tracinglostchars\@ne
  \tracingpages\z@
  \tracingstats\z@
  \tracingcommands\z@}

%% Register allocation
%% We have to adjust the Plain TeX / LaTeX register allocation counts
%% for our slightly modified book-keeping, but first we allocate our
%% insertion counter \et@xins, because \insc@ount of Plain TeX / LaTeX
%% will be used differently.

\newcount\et@xins

\advance\count10 by 1 % \count10=23 % allocates \count registers 23, 24, ...
\advance\count11 by 1 % \count11=10 % allocates \dimen registers 10, 11, ...
\advance\count12 by 1 % \count12=10 % allocates \skip registers 10, 11, ...
\advance\count13 by 1 % \count13=10 % allocates \muskip registers 10, 11, ...
\advance\count14 by 1 % \count14=10 % allocates \box registers 10, 11, ...
\advance\count15 by 1 % \count15=10 % allocates \toks registers 10, 11, ...
\advance\count16 by 1 % \count16=0 % allocates input streams 0, 1, ...
\advance\count17 by 1 % \count17=0 % allocates output streams 0, 1, ...
\advance\count18 by 1 % \count18=4 % allocates math families 4, 5, ...
\advance\count19 by 1 % \count19=0 % allocates \language codes 0, 1, ...

\et@xins=\insc@unt % \et@xins=255 % allocates insertions 254, 253, ...


%% To ensure working in LaTeX 2015 release do define \newcount etc
%% with their pre 2015 LaTeX definitions
\def\newcount{\alloc@0\count\countdef\insc@unt}
\def\newdimen{\alloc@1\dimen\dimendef\insc@unt}
\def\newskip{\alloc@2\skip\skipdef\insc@unt}
\def\newmuskip{\alloc@3\muskip\muskipdef\@cclvi}
\def\newbox{\alloc@4\box\chardef\insc@unt}
\def\newtoks{\alloc@5\toks\toksdef\@cclvi}
\def\newread{\alloc@6\read\chardef\sixt@@n}
\def\newwrite{\alloc@7\write\chardef\sixt@@n}
\def\new@mathgroup{\alloc@8\mathgroup\chardef\sixt@@n}
\let\newfam\new@mathgroup
\def\newlanguage{\alloc@9\language\chardef\@cclvi}

%% When the normal register pool for \count, \dimen, \skip, \muskip,
%% \box, or \toks registers is exhausted, we switch to the extended pool.

\def\alloc@#1#2#3#4#5%
 {\ifnum\count1#1<#4% make sure there's still room
    \allocationnumber\count1#1
    \global\advance\count1#1\@ne
    \global#3#5\allocationnumber
    \wlog{\string#5=\string#2\the\allocationnumber}%
  \else\ifnum#1<6
    \begingroup \escapechar\m@ne
    \expandafter\alloc@@\expandafter{\string#2}#5%
  \else\errmessage{No room for a new #2}\fi\fi
 }

%% The \expandafter construction used here allows the generation of
%% \newcount and \globcount from #1=count.

\def\alloc@@#1#2%
 {\endgroup % restore \escapechar
  \wlog{Normal \csname#1\endcsname register pool exhausted,
    switching to extended pool.}%
  \global\expandafter\let
    \csname new#1\expandafter\endcsname
    \csname glob#1\endcsname
  \csname new#1\endcsname#2%
 }

%% We do change the LaTeX definition of \newinsert

\def\newinsert#1{% make sure there's still room for ...
  \ch@ck0\et@xins\count{% ... a \count, ...
    \ch@ck1\et@xins\dimen{% ... \dimen, ...
      \ch@ck2\et@xins\skip{% ... \skip, ...
        \ch@ck4\et@xins\box{% ... and \box register
  \global\advance\et@xins\m@ne
  \unless\ifnum\insc@unt<\et@xins \global\insc@unt\et@xins \fi
  \allocationnumber\et@xins
  \global\chardef#1\allocationnumber
  \wlog{\string#1=\string\insert\the\allocationnumber}}}}}}

\def\ch@ck#1#2#3#4%
 {\ifnum\count1#1<#2#4\else\errmessage{No room for a new #3}\fi}

%% And we define \reserveinserts, so that you can say \reserveinserts{17}
%% in order to reserve room for up to 17 additional insertion classes, that
%% will not be taken away by \newcount, \newdimen, \newskip, or \newbox.

% 2.4 Remove \outer to match LaTeX allocations
% which are never \outer unlike plain TeX.

%\outer
\def\reserveinserts#1%
 {\global\insc@unt\numexpr \et@xins \ifnum#1>\z@ -#1\fi \relax}

% Now, we define \globcount, \globbox, etc., so that you can say
% \globcount\foo and \foo will be defined (with \countdef) to be the
% next count register from the vastly larger but somewhat less efficient
% extended register pool. We also define \loccount, etc., but these
% register definitions are local to the current group.

\count260=277 % globally allocates \count registers 277, 278, ...
\count261=256 % globally allocates \dimen registers 256, 257, ...
\count262=256 % globally allocates \skip registers 256, 257, ...
\count263=256 % globally allocates \muskip registers 256, 257, ...
\count264=256 % globally allocates \box registers 256, 257, ...
\count265=256 % globally allocates \toks registers 256, 257, ...
\count266=1 % globally allocates \marks classes 1, 2, ...

\count270=32768 % locally allocates \count registers 32767, 32766, ...
\count271=32768 % ditto for \dimen registers
\count272=32768 % ditto for \skip registers
\count273=32768 % ditto for \muskip registers
\count274=32768 % ditto for \box registers
\count275=32768 % ditto for \toks registers
\count276=32768 % ditto for \marks classes

% \count registers 256-259 and 267-269 are not (yet) used

% \def \et@xglob #1#2#3#4% <offset>, <type>, <method>, <register>
% \def \et@xloc #1#2#3#4% <offset>, <type>, <method>, <register>

\def \globcount  {\et@xglob 0\count  \countdef}
\def \loccount   {\et@xloc  0\count  \countdef}
\def \globdimen  {\et@xglob 1\dimen  \dimendef}
\def \locdimen   {\et@xloc  1\dimen  \dimendef}
\def \globskip   {\et@xglob 2\skip   \skipdef}
\def \locskip    {\et@xloc  2\skip   \skipdef}
\def \globmuskip {\et@xglob 3\muskip \muskipdef}
\def \locmuskip  {\et@xloc  3\muskip \muskipdef}
\def \globbox    {\et@xglob 4\box    \mathchardef}
\def \locbox     {\et@xloc  4\box    \mathchardef}
\def \globtoks   {\et@xglob 5\toks   \toksdef}
\def \loctoks    {\et@xloc  5\toks   \toksdef}
\def \globmarks  {\et@xglob 6\marks  \mathchardef}
\def \locmarks   {\et@xloc  6\marks  \mathchardef}

\let\newmarks=\globmarks %% this used to be \newmark for e-TeX V 1.1

\def\et@xglob#1#2#3#4%
 {\et@xchk#1#2{% make sure there's still room
  \allocationnumber=\count26#1%
  \global\advance\count26#1\@ne
  \global#3#4\allocationnumber
  \wlog{\string#4=\string#2\the\allocationnumber}}%
 }

\def\et@xloc#1#2#3#4%
 {\et@xchk#1#2{% make sure there's still room
  \advance\count27#1by\m@ne
  \allocationnumber=\count27#1%
  #3#4=\allocationnumber
  \et@xwlog{\string#4=\string#2\the\allocationnumber\space(local)}}%
 }

%% The allocation messages for local allocations use \et@xwlog, such that
%% these messages can easily be switched on/off

\let\et@xwlog=\wlog

\def\et@xchk#1#2#3%
 {\ifnum\count26#1<\count27#1 #3\else\errmessage{No room for a new #2}\fi}

% Next we define \globcountblk, \loccountblk, etc., so that you can
% say \globcountblk\foo{17} and \foo will be defined (with \mathchardef)
% as the first (the zeroth?) of a block of 17 consecutive registers.
% Thus the user is intended to reference elements <\foo+0> to <\foo+n-1>,
% where n is the length of the block allocated.

% \def \et@xgblk #1#2#3#4% <offset>, <type>, <register>, <size>
% \def \et@xlblk #1#2#3#4% <offset>, <type>, <register>, <size>

\def\globcountblk  {\et@xgblk 0\count  }
\def\loccountblk   {\et@xlblk 0\count  }
\def\globdimenblk  {\et@xgblk 1\dimen  }
\def\locdimenblk   {\et@xlblk 1\dimen  }
\def\globskipblk   {\et@xgblk 2\skip   }
\def\locskipblk    {\et@xlblk 2\skip   }
\def\globmuskipblk {\et@xgblk 3\muskip }
\def\locmuskipblk  {\et@xlblk 3\muskip }
\def\globboxblk    {\et@xgblk 4\box    }
\def\locboxblk     {\et@xlblk 4\box    }
\def\globtoksblk   {\et@xgblk 5\toks   }
\def\loctoksblk    {\et@xlblk 5\toks   }
\def\globmarksblk  {\et@xgblk 6\marks  }
\def\locmarksblk   {\et@xlblk 6\marks  }

% \def\et@xchkblk#1#1#3#4% <offset>, <type>, <size>, <action>

\def\et@xgblk#1#2#3#4%
 {\et@xchkblk#1#2{#4}% make sure there's still room
   {\allocationnumber\count26#1%
    \global\advance\count26#1by#4%
    \global\mathchardef#3\allocationnumber
    \wlog{\string#3=\string#2blk{\number#4} at
      \the\allocationnumber}%
   }%
 }

\def\et@xlblk#1#2#3#4%
 {\et@xchkblk#1#2{#4}% make sure there's still room
   {\advance\count27#1-#4%
    \allocationnumber\count27#1%
    \mathchardef#3\allocationnumber
    \et@xwlog{\string#3=\string#2blk{\number#4} at
      \the\allocationnumber\space(local)}%
   }%
 }

\def\et@xchkblk#1#2#3#4%
 {\ifnum#3<\z@
    \errmessage{Negative register block size \number#3}%
  \else\ifnum\numexpr\count26#1+#3>\count27#1%
    \errmessage{No room for new #2block of size \number#3}%
  \else #4\fi \fi
 }

\catcode`\G=14
\catcode`\I=14
\catcode`\N=14
\catcode`\C=14

\ProcessOptions

%% Declare names for `grouptypes'

G \chardef \bottomleveltype       =  0 % for the outside world
G \chardef \simplegrouptype       =  1 % for local structure only
G \chardef \hboxgrouptype         =  2 % for `\hbox{}'
G \chardef \adjustedhboxgrouptype =  3 % for `\hbox{}' in vertical mode
G \chardef \vboxgrouptype         =  4 % for `\vbox{}'
G \chardef \vtopgrouptype         =  5 % for `\vtop{}'
G \chardef \aligngrouptype        =  6 % for `\halign{}', `\valign{}'
G \chardef \noaligngrouptype      =  7 % for `\noalign{}'
G \chardef \outputgrouptype       =  8 % for output routine
G \chardef \mathgrouptype         =  9 % for, e.g, `^{}'
G \chardef \discgrouptype         = 10 % for `\discretionary{}{}{}'
G \chardef \insertgrouptype       = 11 % for `\insert{}', `\vadjust{}'
G \chardef \vcentergrouptype      = 12 % for `\vcenter{}'
G \chardef \mathchoicegrouptype   = 13 % for `\mathchoice{}{}{}{}'
G \chardef \semisimplegrouptype   = 14 % for `\begingroup...\endgroup'
G \chardef \mathshiftgrouptype    = 15 % for `$...$'
G \chardef \mathleftgrouptype     = 16 % for `\left...\right'

%% Declare names for `interactionmodes'

I \chardef \batchinteractionmode     = 0 % omits all stops and omits terminal output
I \chardef \nonstopinteractionmode   = 1 % omits all stops
I \chardef \scrollinteractionmode    = 2 % omits error stops
I \chardef \errorstopinteractionmode = 3 % stops at every opportunity to interact

%% Declare names for `nodetypes'

N \chardef \charnode     =  0 % character nodes
N \chardef \hlistnode    =  1 % hlist nodes
N \chardef \vlistnode    =  2 % vlist nodes
N \chardef \rulenode     =  3 % rule nodes
N \chardef \insnode      =  4 % insertion nodes
N \chardef \marknode     =  5 % a mark node
N \chardef \adjustnode   =  6 % an adjust node
N \chardef \ligaturenode =  7 % a ligature node
N \chardef \discnode     =  8 % a discretionary node
N \chardef \whatsitnode  =  9 % special extension nodes
N \chardef \mathnode     = 10 % a math node
N \chardef \gluenode     = 11 % node that points to a glue specification
N \chardef \kernnode     = 12 % a kern node
N \chardef \penaltynode  = 13 % a penalty node
N \chardef \unsetnode    = 14 % an unset node
N \chardef \mathsnodes   = 15 % nodes that occur only in maths mode

%% Declare names for `iftypes'

C \chardef \charif     =  1 % \if
C \chardef \catif      =  2 % \ifcat
C \chardef \numif      =  3 % \ifnum
C \chardef \dimif      =  4 % \ifdim
C \chardef \oddif      =  5 % \ifodd
C \chardef \vmodeif    =  6 % \ifvmode
C \chardef \hmodeif    =  7 % \ifhmode
C \chardef \mmodeif    =  8 % \ifmmode
C \chardef \innerif    =  9 % \ifinner
C \chardef \voidif     = 10 % \ifvoid
C \chardef \hboxif     = 11 % \ifhbox
C \chardef \vboxif     = 12 % \ifvbox
C \chardef \xif        = 13 % \ifx
C \chardef \eofif      = 14 % \ifeof
C \chardef \trueif     = 15 % \iftrue
C \chardef \falseif    = 16 % \iffalse
C \chardef \caseif     = 17 % \ifcase
C \chardef \definedif  = 18 % \ifdefined
C \chardef \csnameif   = 19 % \ifcsname
C \chardef \fontcharif = 20 % \iffontchar

\catcode`\G=11
\catcode`\I=11
\catcode`\N=11
\catcode`\C=11

%
    \fi
    \catcode`\@=11 %
    \outer\expandafter\def\csname newfam\endcsname
                          {\alloc@8\fam\chardef\et@xmaxfam}
  \else
    \RequirePackage{etex}
    \expandafter\def\csname newfam\endcsname
                    {\alloc@8\fam\chardef\et@xmaxfam}
    \expandafter\let\expandafter\new@mathgroup\csname newfam\endcsname
  \fi
\edef \et@xmaxregs {\ifx\directlua\@undefined 32768\else 65536\fi}
\edef \et@xmaxfam {\ifx\Umathcode\@undefined\sixt@@n\else\@cclvi\fi}
\count 270=\et@xmaxregs % locally allocates \count registers
\count 271=\et@xmaxregs % ditto for \dimen registers
\count 272=\et@xmaxregs % ditto for \skip registers
\count 273=\et@xmaxregs % ditto for \muskip registers
\count 274=\et@xmaxregs % ditto for \box registers
\count 275=\et@xmaxregs % ditto for \toks registers
\count 276=\et@xmaxregs % ditto for \marks classes
\expandafter\let\csname newcount\expandafter\expandafter\endcsname
                \csname globcount\endcsname
\expandafter\let\csname newdimen\expandafter\expandafter\endcsname
                \csname globdimen\endcsname
\expandafter\let\csname newskip\expandafter\expandafter\endcsname
                \csname globskip\endcsname
\expandafter\let\csname newbox\expandafter\expandafter\endcsname
                \csname globbox\endcsname
\chardef\e@alloc@top=65535
\let\e@alloc@chardef\chardef
\def\e@alloc#1#2#3#4#5#6{%
  \global\advance#3\@ne
  \e@ch@ck{#3}{#4}{#5}#1%
  \allocationnumber#3\relax
  \global#2#6\allocationnumber
  \wlog{\string#6=\string#1\the\allocationnumber}}%
\gdef\e@ch@ck#1#2#3#4{%
  \ifnum#1<#2\else
    \ifnum#1=#2\relax
      #1\@cclvi
      \ifx\count#4\advance#1 10 \fi
    \fi
    \ifnum#1<#3\relax
    \else
      \errmessage{No room for a new \string#4}%
    \fi
  \fi}%
\expandafter\csname newcount\endcsname\e@alloc@attribute@count
\expandafter\csname newcount\endcsname\e@alloc@ccodetable@count
\expandafter\csname newcount\endcsname\e@alloc@luafunction@count
\expandafter\csname newcount\endcsname\e@alloc@whatsit@count
\expandafter\csname newcount\endcsname\e@alloc@bytecode@count
\expandafter\csname newcount\endcsname\e@alloc@luachunk@count
\fi
\ifx\e@alloc@attribute@count\@undefined
  \countdef\e@alloc@attribute@count=258
  \e@alloc@attribute@count=\z@
\fi
\def\newattribute#1{%
  \e@alloc\attribute\attributedef
    \e@alloc@attribute@count\m@ne\e@alloc@top#1%
}
\def\setattribute#1#2{#1=\numexpr#2\relax}
\def\unsetattribute#1{#1=-"7FFFFFFF\relax}
\ifx\e@alloc@ccodetable@count\@undefined
  \countdef\e@alloc@ccodetable@count=259
  \e@alloc@ccodetable@count=\z@
\fi
\def\newcatcodetable#1{%
  \e@alloc\catcodetable\chardef
    \e@alloc@ccodetable@count\m@ne{"8000}#1%
  \initcatcodetable\allocationnumber
}
\newcatcodetable\catcodetable@initex
\newcatcodetable\catcodetable@string
\begingroup
  \def\setrangecatcode#1#2#3{%
    \ifnum#1>#2 %
      \expandafter\@gobble
    \else
      \expandafter\@firstofone
    \fi
      {%
        \catcode#1=#3 %
        \expandafter\setrangecatcode\expandafter
          {\number\numexpr#1 + 1\relax}{#2}{#3}
      }%
  }
  \@firstofone{%
    \catcodetable\catcodetable@initex
      \catcode0=12 %
      \catcode13=12 %
      \catcode37=12 %
      \setrangecatcode{65}{90}{12}%
      \setrangecatcode{97}{122}{12}%
      \catcode92=12 %
      \catcode127=12 %
      \savecatcodetable\catcodetable@string
    \endgroup
  }%
\newcatcodetable\catcodetable@latex
\newcatcodetable\catcodetable@atletter
\begingroup
  \def\parseunicodedataI#1;#2;#3;#4\relax{%
    \parseunicodedataII#1;#3;#2 First>\relax
  }%
  \def\parseunicodedataII#1;#2;#3 First>#4\relax{%
    \ifx\relax#4\relax
      \expandafter\parseunicodedataIII
    \else
      \expandafter\parseunicodedataIV
    \fi
      {#1}#2\relax%
  }%
  \def\parseunicodedataIII#1#2#3\relax{%
    \ifnum 0%
      \if L#21\fi
      \if M#21\fi
      >0 %
      \catcode"#1=11 %
    \fi
  }%
  \def\parseunicodedataIV#1#2#3\relax{%
    \read\unicoderead to \unicodedataline
    \if L#2%
      \count0="#1 %
      \expandafter\parseunicodedataV\unicodedataline\relax
    \fi
  }%
  \def\parseunicodedataV#1;#2\relax{%
    \loop
      \unless\ifnum\count0>"#1 %
        \catcode\count0=11 %
        \advance\count0 by 1 %
    \repeat
  }%
  \def\storedpar{\par}%
  \chardef\unicoderead=\numexpr\count16 + 1\relax
  \openin\unicoderead=UnicodeData.txt %
  \loop\unless\ifeof\unicoderead %
    \read\unicoderead to \unicodedataline
    \unless\ifx\unicodedataline\storedpar
      \expandafter\parseunicodedataI\unicodedataline\relax
    \fi
  \repeat
  \closein\unicoderead
  \@firstofone{%
    \catcode64=12 %
    \savecatcodetable\catcodetable@latex
    \catcode64=11 %
    \savecatcodetable\catcodetable@atletter
   }
\endgroup
\ifx\e@alloc@luafunction@count\@undefined
  \countdef\e@alloc@luafunction@count=260
  \e@alloc@luafunction@count=\z@
\fi
\def\newluafunction{%
  \e@alloc\luafunction\e@alloc@chardef
    \e@alloc@luafunction@count\m@ne\e@alloc@top
}
\ifx\e@alloc@whatsit@count\@undefined
  \countdef\e@alloc@whatsit@count=261
  \e@alloc@whatsit@count=\z@
\fi
\def\newwhatsit#1{%
  \e@alloc\whatsit\e@alloc@chardef
    \e@alloc@whatsit@count\m@ne\e@alloc@top#1%
}
\ifx\e@alloc@bytecode@count\@undefined
  \countdef\e@alloc@bytecode@count=262
  \e@alloc@bytecode@count=\z@
\fi
\def\newluabytecode#1{%
  \e@alloc\luabytecode\e@alloc@chardef
    \e@alloc@bytecode@count\m@ne\e@alloc@top#1%
}

\ifx\e@alloc@luachunk@count\@undefined
  \countdef\e@alloc@luachunk@count=263
  \e@alloc@luachunk@count=\z@
\fi
\def\newluachunkname#1{%
  \e@alloc\luachunk\e@alloc@chardef
    \e@alloc@luachunk@count\m@ne\e@alloc@top#1%
    {\escapechar\m@ne
    \directlua{lua.name[\the\allocationnumber]="\string#1"}}%
}
\def\now@and@everyjob#1{%
  \everyjob\expandafter{\the\everyjob
    #1%
  }%
  #1%
}
  \begingroup
    \attributedef\attributezero=0 %
    \chardef     \charzero     =0 %
    \countdef    \CountZero    =0 %
    \dimendef    \dimenzero    =0 %
    \mathchardef \mathcharzero =0 %
    \muskipdef   \muskipzero   =0 %
    \skipdef     \skipzero     =0 %
    \toksdef     \tokszero     =0 %
    \directlua{require("ltluatex")}
  \endgroup
\catcode`\@=\etatcatcode\relax
\endinput
%%
%% End of file `ltluatex.tex'.
%
\fi
\def\@setcatcodetable#1#2{%
  \begingroup
    #2%
    \savecatcodetable#1%
  \endgroup
}
\def\@setrangecatcode#1#2#3{%
  \ifnum#1>#2 %
    \expandafter\@gobble
  \else
    \expandafter\@firstofone
  \fi
    {%
      \catcode#1=#3 %
      \expandafter\@setrangecatcode\expandafter
        {\number\numexpr#1+1\relax}{#2}{#3}%
    }%
}
\def\@catcodetablelist{}
\def\@catcodetablestack{}
\newcount\@catcodetablestackcnt
\def\@pushcatcodetable{%
  \ifx\@catcodetablelist\empty
    \global\advance\@catcodetablestackcnt by\@ne
    \edef\@tempa{\csname ct@\the\@catcodetablestackcnt\endcsname}%
    \expandafter\newcatcodetable\@tempa
    \xdef\@catcodetablelist{\@tempa}%
  \fi
  \expandafter\@pushctbl\@catcodetablelist\@nil
}
\def\@pushctbl#1#2\@nil{%
  \gdef\@catcodetablelist{#2}%
  \xdef\@catcodetablestack{#1\@catcodetablestack}%
  \savecatcodetable#1%
}
\def\@popcatcodetable{%
  \if!\@catcodetablestack!%
    \errmessage{Attempt to pop empty catcodetable stack}%
  \else
    \expandafter\@popctbl\@catcodetablestack\@nil
  \fi
}
\def\@popctbl#1#2\@nil{%
  \gdef\@catcodetablestack{#2}%
  \xdef\@catcodetablelist{\@catcodetablelist#1}%
  \catcodetable#1%
}
\catcode`\@\ctstackatcatcode\relax
%
  \else
    \RequirePackage{ctablestack}
  \fi
\fi
%    \end{macrocode}
%
% Simple require wrapper as we now assume |require| implicitly uses the
% |kpathsea| search library.
%    \begin{macrocode}
\def\RequireLuaModule#1{\directlua{require("#1")}\@gobbleoptarg}
%    \end{macrocode}
%
% In \LaTeX\ (or plain macro package that has defined |\@ifnextchar|)
% use  |\@ifnextchar| otherwise use a simple alternative, in practice this
% will never be followed by a brace group, so full version of |\@ifnextchar|
% not needed.
%    \begin{macrocode}
\ifdefined\@ifnextchar
\def\@gobbleoptarg{\@ifnextchar[\@gobble@optarg{}}%
\else
\long\def\@gobbleoptarg#1{\ifx[#1\expandafter\@gobble@optarg\fi#1}%
\fi
%    \end{macrocode}
%
%    \begin{macrocode}
\def\@gobble@optarg[#1]{}
%    \end{macrocode}
%
% Extended catcode table support.  Use the names from the previous
% \textsf{luatexbase} and \textsf{luatex} packages.
%    \begin{macrocode}
\let\CatcodeTableIniTeX\catcodetable@initex
\let\CatcodeTableString\catcodetable@string
\let\CatcodeTableLaTeX\catcodetable@latex
\let\CatcodeTableLaTeXAtLetter\catcodetable@atletter
%    \end{macrocode}
%
% Additional tables declared in the previous interface.
%    \begin{macrocode}
\newcatcodetable\CatcodeTableOther
\@setcatcodetable\CatcodeTableOther{%
  \catcodetable\CatcodeTableString
  \catcode32 12 }
%    \end{macrocode}
%
%    \begin{macrocode}
\newcatcodetable\CatcodeTableExpl
\@setcatcodetable\CatcodeTableExpl{%
  \catcodetable\CatcodeTableLaTeX
  \catcode126 10 % tilde is a space char
  \catcode32  9  % space is ignored
  \catcode9   9  % tab also ignored
  \catcode95  11 % underscore letter
  \catcode58  11 % colon letter
}
%    \end{macrocode}
%
% Top level access to catcodetable stack.
%    \begin{macrocode}
\def\BeginCatcodeRegime#1{%
  \@pushcatcodetable
  \catcodetable#1\relax}
%    \end{macrocode}
%
%    \begin{macrocode}
\def\EndCatcodeRegime{%
  \@popcatcodetable}
%    \end{macrocode}
%
% The implementation of the stack is completely
% different, but usage should match.
%    \begin{macrocode}
\let\PushCatcodeTableNumStack\@pushcatcodetable
\let\PopCatcodeTableNumStack\@popcatcodetable
%    \end{macrocode}
%
% A simple copy.
%    \begin{macrocode}
\let\SetCatcodeRange\@setrangecatcode
%    \end{macrocode}
%
% Another copy.
%    \begin{macrocode}
\let\setcatcodetable\@setcatcodetable
%    \end{macrocode}
%
% \subsubsection{Additional lua code}
%    \begin{macrocode}
\directlua{
%    \end{macrocode}
%
% Remove all registered callbacks, then disable.
% Set to false if optional second argument is |true|.
%    \begin{macrocode}
function luatexbase.reset_callback(name,make_false)
  for _,v in pairs(luatexbase.callback_descriptions(name))
  do
    luatexbase.remove_from_callback(name,v)
  end
  if make_false == true then
    luatexbase.disable_callback(name)
  end
end
%    \end{macrocode}
%
% Allow exclusive callbacks to be over-written if priority argument is
% 1 to match the ``first'' semantics of the original package.
% \changes{v1.1}{2015/10/02}{Fully handle priority argument}
%
% First save the kernel function.
%    \begin{macrocode}
luatexbase.base_add_to_callback=luatexbase.add_to_callback
%    \end{macrocode}
%
% Implement the priority argument by taking off existing callbacks
% that have higher priority than the new one, adding the new one, 
% Then putting the saved callbacks back.
%    \begin{macrocode}
function luatexbase.add_to_callback(name,fun,description,priority)
%    \end{macrocode}
%^^A  texio.write_nl('\string\n HERE: adding ' .. 
%^^A                  description .. 
%^^A                  ' to ' ..
%^^A                  name .. 
%^^A                  ' with priority ' ..
%^^A                  (priority or '@@@'))
%^^A    texio.write_nl('Original list')
%^^A    for k,v in pairs(luatexbase.callback_descriptions(name)) do
%^^A      texio.write_nl('    ' .. k .. ': ' .. v)
%^^A    end
%    \begin{macrocode}
  local priority= priority
  if priority==nil then
   priority=\string#luatexbase.callback_descriptions(name)+1
  end
  if(luatexbase.callbacktypes[name] == 3 and
     priority == 1 and
     \string#luatexbase.callback_descriptions(name)==1) then
    luatexbase.module_warning("luatexbase",
                              "resetting exclusive callback: " .. name)
    luatexbase.reset_callback(name)
  end
  local saved_callback={},ff,dd
  for k,v in pairs(luatexbase.callback_descriptions(name)) do
    if k >= priority then
      ff,dd= luatexbase.remove_from_callback(name, v)
      saved_callback[k]={ff,dd}
    end
  end
  luatexbase.base_add_to_callback(name,fun,description)
  for k,v in pairs(saved_callback) do
    luatexbase.base_add_to_callback(name,v[1],v[2])
  end
%    \end{macrocode}
%^^A  texio.write_nl('New list')
%^^A  for k,v in pairs(luatexbase.callback_descriptions(name)) do
%^^A    texio.write_nl('    ' .. k .. ': ' .. v)
%^^A  end
%    \begin{macrocode}
  return
end
%    \end{macrocode}
%
% Emulate the catcodetables table.
% Explicitly fill the table rather than rely on the metatable call to
% |registernumber| as that is unreliable on old Lua\TeX{}.
%    \begin{macrocode}
luatexbase.catcodetables=setmetatable(
 {['latex-package'] = \number\CatcodeTableLaTeXAtLetter,
  ini    = \number\CatcodeTableIniTeX,
  string = \number\CatcodeTableString,
  other  = \number\CatcodeTableOther,
  latex  = \number\CatcodeTableLaTeX,
  expl   = \number\CatcodeTableExpl,
  expl3  = \number\CatcodeTableExpl},
 { __index = function(t,key)
    return luatexbase.registernumber(key) or nil
  end}
)}
%    \end{macrocode}
%
% On old Lua\TeX{} workaround hashtable issues.
% Allocate in \TeX{}, and also directly add to |luatexbase.catcodetables|.
%    \begin{macrocode}
\ifnum\luatexversion<80 %
\def\newcatcodetable#1{%
  \e@alloc\catcodetable\chardef
    \e@alloc@ccodetable@count\m@ne{"8000}#1%
  \initcatcodetable\allocationnumber
  {\escapechar=\m@ne
  \directlua{luatexbase.catcodetables['\string#1']=%
    \the\allocationnumber}}%
}
\fi
%    \end{macrocode}
%
%    \begin{macrocode}
\directlua{
%    \end{macrocode}
%
% |priority_in_callback| returns position in the callback list.
% Not provided by default by the kernel as usually it is just used 
% as a boolean test, for which |in_callback| is provided.
%    \begin{macrocode}
function luatexbase.priority_in_callback (name,description)
  for i,v in ipairs(luatexbase.callback_descriptions(name))
  do
    if v == description then
      return i
    end
  end
  return false
end
%    \end{macrocode}
% 
% The (unreleased) version~0.7 of \textsf{luatexbase} provided this
% boolean test under a different name, so we provide an alias here.
%    \begin{macrocode}
luatexbase.is_active_callback = luatexbase.in_callback
%    \end{macrocode}
%
% \textsf{ltluatex} implementation of |provides_module| does not return
% print functions so define modified version here.
% \changes{v1.3}{2015/10/03}{Use the first argument as a format string for
%   later arguments}
%    \begin{macrocode}
luatexbase.base_provides_module=luatexbase.provides_module
function luatexbase.errwarinf(name)
    return
    function(s,...) return luatexbase.module_error(name, s:format(...)) end,
    function(s,...) return luatexbase.module_warning(name, s:format(...)) end,
    function(s,...) return luatexbase.module_info(name, s:format(...)) end,
    function(s,...) return luatexbase.module_info(name, s:format(...)) end
end
function luatexbase.provides_module(info)
  luatexbase.base_provides_module(info)
  return luatexbase.errwarinf(info.name)
end
}
%    \end{macrocode}
%
% Same for attribute table as catcode tables. In old Lua\TeX{}, add to the
% |luatexbase.attributes| table directly.
%    \begin{macrocode}
\ifnum\luatexversion<80 %
\def\newattribute#1{%
  \e@alloc\attribute\attributedef
    \e@alloc@attribute@count\m@ne\e@alloc@top#1%
  {\escapechar=\m@ne
  \directlua{luatexbase.attributes['\string#1']=%
    \the\allocationnumber}}%
}
\fi
%    \end{macrocode}
%
% Define a safe percent command for plain \TeX.
%    \begin{macrocode}
\ifx\@percentchar\@undefined
  {\catcode`\%=12 \gdef\@percentchar{%}}
\fi
%    \end{macrocode}
% \changes{v1.2a}{2015/10/03}{Add missing local definitions for whatsit code}
%
%    \begin{macrocode}
%<*whatsit>
\directlua{%
%    \end{macrocode}
%
%    \begin{macrocode}
local copynode          = node.copy
local newnode           = node.new
local nodesubtype       = node.subtype
local nodetype          = node.id
local stringformat      = string.format
local tableunpack       = unpack or table.unpack
local texiowrite_nl     = texio.write_nl
local texiowrite        = texio.write
local whatsit_t         = nodetype"whatsit"
local user_defined_t    = nodesubtype"user_defined"
local unassociated      = "__unassociated"
local user_whatsits       = {  __unassociated = { } }
local whatsit_ids         = { }
local anonymous_whatsits  = 0
local anonymous_prefix    = "anon"
%    \end{macrocode}
%
%    User whatsit allocation is split into two functions:
%    \verb|new_user_whatsit_id| registers a new id (an integer)
%    and returns it. This is a wrapper around \verb|new_whatsit|
%    but with the extra \texttt{package} argument, and recording
%    the mapping in lua tables
%
%    If no name given, generate a name from a counter.
%
%    \begin{macrocode}
local new_user_whatsit_id = function (name, package)
    if name then
        if not package then
            package = unassociated
        end
    else % anonymous
        anonymous_whatsits = anonymous_whatsits + 1
        warning("defining anonymous user whatsit no. \@percentchar 
                  d", anonymous_whatsits)
        package = unassociated
        name    = anonymous_prefix .. tostring(anonymous_whatsits)
    end

    local whatsitdata = user_whatsits[package]
    if not whatsitdata then
        whatsitdata             = { }
        user_whatsits[package]  = whatsitdata
    end

    local id = whatsitdata[name]
    if id then %- warning
        warning("replacing whatsit \@percentchar s:\@percentchar 
                  s (\@percentchar d)", package, name, id)
    else %- new id
        id=luatexbase.new_whatsit(name)
        whatsitdata[name]   = id
        whatsit_ids[id]     = { name, package }
    end
    return id
end
luatexbase.new_user_whatsit_id = new_user_whatsit_id
%    \end{macrocode}
%
%    \verb|new_user_whatsit| first registers a new id and then also
%    creates the corresponding whatsit node of subtype “user-defined”.
%    Return a nullary function that delivers copies of the whatsit.
%
%    Alternatively, the first argument can be a whatsit node that
%    will then be used as prototype.
%
%    \begin{macrocode}
local new_user_whatsit = function (req, package)
    local id, whatsit
    if type(req) == "string" then
        id              = new_user_whatsit_id(req, package)
        whatsit         = newnode(whatsit_t, user_defined_t)
        whatsit.user_id = id
    elseif req.id == whatsit_t and req.subtype == user_defined_t then
        id      = req.user_id
        whatsit = copynode(req)
        if not whatsit_ids[id] then
            warning("whatsit id \@percentchar d unregistered; "
                    .. "inconsistencies may arise", id)
        end
    end
    return function () return copynode(whatsit) end, id
end
luatexbase.new_user_whatsit         = new_user_whatsit
%    \end{macrocode}
%
%    If one knows the name of a user whatsit, its corresponding id
%    can be retrieved by means of \verb|get_user_whatsit_id|.
%
%    \begin{macrocode}
local get_user_whatsit_id = function (name, package)
    if not package then
        package = unassociated
    end
    return user_whatsits[package][name]
end
luatexbase.get_user_whatsit_id = get_user_whatsit_id
%    \end{macrocode}
%
%    The inverse lookup is also possible via \verb|get_user_whatsit_name|.
%    \begin{macrocode}
local get_user_whatsit_name = function (asked)
    local id
    if type(asked) == "number" then
        id = asked
    elseif type(asked) == "function" then
        %- node generator
        local n = asked()
        id = n.user_id
    else %- node
        id = asked.user_id
    end
    local metadata = whatsit_ids[id]
    if not metadata then % unknown
        warning("whatsit id \@percentchar d unregistered;
                   inconsistencies may arise", id)
        return "", ""
    end
    return tableunpack(metadata)
end
luatexbase.get_user_whatsit_name = get_user_whatsit_name
%    \end{macrocode}
%  A function that outputs the
%    current allocation status to the terminal.
%
%    \begin{macrocode}
local dump_registered_whatsits = function (asked_package)
    local whatsit_list = { }
    if asked_package then
        local whatsitdata = user_whatsits[asked_package]
        if not whatsitdata then
            error("(no user whatsits registered for package
                      \@percentchar s)", asked_package)
            return
        end
        texiowrite_nl("(user whatsit allocation stats for " ..
                          asked_package)
        for name, id in next, whatsitdata do
            whatsit_list[\string#whatsit_list+1] =
                stringformat("(\@percentchar s:\@percentchar 
                     s \@percentchar d)", asked_package, name, id)
        end
    else
        texiowrite_nl("(user whatsit allocation stats")
        texiowrite_nl(stringformat(" ((total \@percentchar d)\string\n
                         (anonymous \@percentchar d))",
            current_whatsit, anonymous_whatsits))
        for package, whatsitdata in next, user_whatsits do
            for name, id in next, whatsitdata do
                whatsit_list[\string#whatsit_list+1] =
                    stringformat("(\@percentchar s:\@percentchar
                        s \@percentchar d)", package, name, id)
            end
        end
    end
    texiowrite_nl" ("
    local first = true
    for i=1, \string#whatsit_list do
        if first then
            first = false
        else % indent
            texiowrite_nl"  "
        end
        texiowrite(whatsit_list[i])
    end
    texiowrite"))\string\n"
end
luatexbase.dump_registered_whatsits = dump_registered_whatsits
%    \end{macrocode}
%    Lastly, we define a couple synonyms for convenience.
%    \begin{macrocode}
luatexbase.newattribute            = new_attribute
luatexbase.newuserwhatsit          = new_user_whatsit
luatexbase.newuserwhatsitid        = new_user_whatsit_id
luatexbase.getuserwhatsitid        = get_user_whatsit_id
luatexbase.getuserwhatsitname      = get_user_whatsit_name
luatexbase.dumpregisteredwhatsits  = dump_registered_whatsits
%    \end{macrocode}
%
%    \begin{macrocode}
}
%</whatsit>
%    \end{macrocode}
%
% Resolve name clashes and prefixed name issues.
%
% Top level \textsf{luatexbase} macros
%    \begin{macrocode}
\let\newluatexattribute\newattribute
\let\setluatexattribute\setattribute
\let\unsetluatexattribute\unsetattribute
\let\newluatexcatcodetable\newcatcodetable
\let\setluatexcatcodetable\setcatcodetable
%    \end{macrocode}
%
% Internal \textsf{luatexbase} macros
%    \begin{macrocode}
\let\luatexbase@directlua\directlua
\let\luatexbase@ensure@primitive\@gobble
%    \end{macrocode}
%
% Lua\TeX\ primitives
%    \begin{macrocode}
\let\luatexattribute\attribute
\let\luatexattributedef\attributedef
\let\luatexcatcodetable\catcodetable
\let\luatexluaescapestring\luaescapestring
\let\luatexlatelua\latelua
\let\luatexoutputbox\outputbox
\let\luatexscantextokens\scantextokens
%    \end{macrocode}
%
% Reset catcode of |@|.
%    \begin{macrocode}
\catcode`\@=\emuatcatcode\relax
%    \end{macrocode}
%
%    \begin{macrocode}
%</emu>
%    \end{macrocode}
%
% \subsection{Legacy \textsf{luatexbase} sub-packages}
%
% The original \textsf{luatexbase} was comprised of seven sub packages
% that could in principle be loaded separately. Here we define them all
% with the same code that just loads the main package, they are
% distinguished just by the |\ProvidesPackage| specified above at the start
% of the file.
%    \begin{macrocode}
%<*emu-cmp,emu-mod,emu-loa,emu-reg,emu-att,emu-cct,emu-mcb>
%    \end{macrocode}
%
%    \begin{macrocode}
\ifx\RequirePackage\undefined
  \ifx\BeginCatcodeRegime\undefined\else\expandafter\endinput\fi

\ifx
  \ProvidesPackage\undefined\begingroup\def\ProvidesPackage
  #1#2[#3]{\endgroup\immediate\write-1{Package: #1 #3}}
\fi
\ProvidesPackage{luatexbase}
[2015/10/04 v1.3
  luatexbase interface to LuaTeX
]
\edef\emuatcatcode{\the\catcode`\@}
\catcode`\@=11
\ifx\@setrangecatcode\@undefined
  \ifx\RequirePackage\@undefined
    \ifx
  \ProvidesPackage\undefined\begingroup\def\ProvidesPackage
    #1#2[#3]{\endgroup\immediate\write-1{Package: #1 #3}}
\fi
\ProvidesPackage{ctablestack}
  [2015/10/01 v1.0 Catcode table stable support]
\edef\ctstackatcatcode{\the\catcode`\@}
\catcode`\@=11
\ifx\newluafunction\@undefined
  %%
%% This is file `ltluatex.tex',
%% generated with the docstrip utility.
%%
%% The original source files were:
%%
%% ltluatex.dtx  (with options: `tex,plain')
%% 
%% This is a generated file.
%% 
%% The source is maintained by the LaTeX Project team and bug
%% reports for it can be opened at https://latex-project.org/bugs.html
%% (but please observe conditions on bug reports sent to that address!)
%% 
%% 
%% Copyright (C) 1993-2021
%% The LaTeX Project and any individual authors listed elsewhere
%% in this file.
%% 
%% This file was generated from file(s) of the LaTeX base system.
%% --------------------------------------------------------------
%% 
%% It may be distributed and/or modified under the
%% conditions of the LaTeX Project Public License, either version 1.3c
%% of this license or (at your option) any later version.
%% The latest version of this license is in
%%    https://www.latex-project.org/lppl.txt
%% and version 1.3c or later is part of all distributions of LaTeX
%% version 2008 or later.
%% 
%% This file has the LPPL maintenance status "maintained".
%% 
%% This file may only be distributed together with a copy of the LaTeX
%% base system. You may however distribute the LaTeX base system without
%% such generated files.
%% 
%% The list of all files belonging to the LaTeX base distribution is
%% given in the file `manifest.txt'. See also `legal.txt' for additional
%% information.
%% 
%% The list of derived (unpacked) files belonging to the distribution
%% and covered by LPPL is defined by the unpacking scripts (with
%% extension .ins) which are part of the distribution.
\ifx\newluafunction\undefined\else\expandafter\endinput\fi
\ifx
  \ProvidesFile\undefined\begingroup\def\ProvidesFile
  #1#2[#3]{\endgroup\immediate\write-1{File: #1 #3}}
\fi
\ProvidesFile{ltluatex.tex}%
[2021/04/18 v1.1t
  LuaTeX support for plain TeX (core)
]
\edef\etatcatcode{\the\catcode`\@}
\catcode`\@=11
\ifnum\luatexversion<60 %
  \wlog{***************************************************}
  \wlog{* LuaTeX version too old for ltluatex support *}
  \wlog{***************************************************}
  \expandafter\endinput
\fi
\long\def\@gobble#1{}
\long\def\@firstofone#1{#1}
\directlua{tex.enableprimitives("",tex.extraprimitives("luatex"))}
\ifx\e@alloc\@undefined
  \ifx\documentclass\@undefined
    \ifx\loccount\@undefined
      \input{etex.src}%
    \fi
    \catcode`\@=11 %
    \outer\expandafter\def\csname newfam\endcsname
                          {\alloc@8\fam\chardef\et@xmaxfam}
  \else
    \RequirePackage{etex}
    \expandafter\def\csname newfam\endcsname
                    {\alloc@8\fam\chardef\et@xmaxfam}
    \expandafter\let\expandafter\new@mathgroup\csname newfam\endcsname
  \fi
\edef \et@xmaxregs {\ifx\directlua\@undefined 32768\else 65536\fi}
\edef \et@xmaxfam {\ifx\Umathcode\@undefined\sixt@@n\else\@cclvi\fi}
\count 270=\et@xmaxregs % locally allocates \count registers
\count 271=\et@xmaxregs % ditto for \dimen registers
\count 272=\et@xmaxregs % ditto for \skip registers
\count 273=\et@xmaxregs % ditto for \muskip registers
\count 274=\et@xmaxregs % ditto for \box registers
\count 275=\et@xmaxregs % ditto for \toks registers
\count 276=\et@xmaxregs % ditto for \marks classes
\expandafter\let\csname newcount\expandafter\expandafter\endcsname
                \csname globcount\endcsname
\expandafter\let\csname newdimen\expandafter\expandafter\endcsname
                \csname globdimen\endcsname
\expandafter\let\csname newskip\expandafter\expandafter\endcsname
                \csname globskip\endcsname
\expandafter\let\csname newbox\expandafter\expandafter\endcsname
                \csname globbox\endcsname
\chardef\e@alloc@top=65535
\let\e@alloc@chardef\chardef
\def\e@alloc#1#2#3#4#5#6{%
  \global\advance#3\@ne
  \e@ch@ck{#3}{#4}{#5}#1%
  \allocationnumber#3\relax
  \global#2#6\allocationnumber
  \wlog{\string#6=\string#1\the\allocationnumber}}%
\gdef\e@ch@ck#1#2#3#4{%
  \ifnum#1<#2\else
    \ifnum#1=#2\relax
      #1\@cclvi
      \ifx\count#4\advance#1 10 \fi
    \fi
    \ifnum#1<#3\relax
    \else
      \errmessage{No room for a new \string#4}%
    \fi
  \fi}%
\expandafter\csname newcount\endcsname\e@alloc@attribute@count
\expandafter\csname newcount\endcsname\e@alloc@ccodetable@count
\expandafter\csname newcount\endcsname\e@alloc@luafunction@count
\expandafter\csname newcount\endcsname\e@alloc@whatsit@count
\expandafter\csname newcount\endcsname\e@alloc@bytecode@count
\expandafter\csname newcount\endcsname\e@alloc@luachunk@count
\fi
\ifx\e@alloc@attribute@count\@undefined
  \countdef\e@alloc@attribute@count=258
  \e@alloc@attribute@count=\z@
\fi
\def\newattribute#1{%
  \e@alloc\attribute\attributedef
    \e@alloc@attribute@count\m@ne\e@alloc@top#1%
}
\def\setattribute#1#2{#1=\numexpr#2\relax}
\def\unsetattribute#1{#1=-"7FFFFFFF\relax}
\ifx\e@alloc@ccodetable@count\@undefined
  \countdef\e@alloc@ccodetable@count=259
  \e@alloc@ccodetable@count=\z@
\fi
\def\newcatcodetable#1{%
  \e@alloc\catcodetable\chardef
    \e@alloc@ccodetable@count\m@ne{"8000}#1%
  \initcatcodetable\allocationnumber
}
\newcatcodetable\catcodetable@initex
\newcatcodetable\catcodetable@string
\begingroup
  \def\setrangecatcode#1#2#3{%
    \ifnum#1>#2 %
      \expandafter\@gobble
    \else
      \expandafter\@firstofone
    \fi
      {%
        \catcode#1=#3 %
        \expandafter\setrangecatcode\expandafter
          {\number\numexpr#1 + 1\relax}{#2}{#3}
      }%
  }
  \@firstofone{%
    \catcodetable\catcodetable@initex
      \catcode0=12 %
      \catcode13=12 %
      \catcode37=12 %
      \setrangecatcode{65}{90}{12}%
      \setrangecatcode{97}{122}{12}%
      \catcode92=12 %
      \catcode127=12 %
      \savecatcodetable\catcodetable@string
    \endgroup
  }%
\newcatcodetable\catcodetable@latex
\newcatcodetable\catcodetable@atletter
\begingroup
  \def\parseunicodedataI#1;#2;#3;#4\relax{%
    \parseunicodedataII#1;#3;#2 First>\relax
  }%
  \def\parseunicodedataII#1;#2;#3 First>#4\relax{%
    \ifx\relax#4\relax
      \expandafter\parseunicodedataIII
    \else
      \expandafter\parseunicodedataIV
    \fi
      {#1}#2\relax%
  }%
  \def\parseunicodedataIII#1#2#3\relax{%
    \ifnum 0%
      \if L#21\fi
      \if M#21\fi
      >0 %
      \catcode"#1=11 %
    \fi
  }%
  \def\parseunicodedataIV#1#2#3\relax{%
    \read\unicoderead to \unicodedataline
    \if L#2%
      \count0="#1 %
      \expandafter\parseunicodedataV\unicodedataline\relax
    \fi
  }%
  \def\parseunicodedataV#1;#2\relax{%
    \loop
      \unless\ifnum\count0>"#1 %
        \catcode\count0=11 %
        \advance\count0 by 1 %
    \repeat
  }%
  \def\storedpar{\par}%
  \chardef\unicoderead=\numexpr\count16 + 1\relax
  \openin\unicoderead=UnicodeData.txt %
  \loop\unless\ifeof\unicoderead %
    \read\unicoderead to \unicodedataline
    \unless\ifx\unicodedataline\storedpar
      \expandafter\parseunicodedataI\unicodedataline\relax
    \fi
  \repeat
  \closein\unicoderead
  \@firstofone{%
    \catcode64=12 %
    \savecatcodetable\catcodetable@latex
    \catcode64=11 %
    \savecatcodetable\catcodetable@atletter
   }
\endgroup
\ifx\e@alloc@luafunction@count\@undefined
  \countdef\e@alloc@luafunction@count=260
  \e@alloc@luafunction@count=\z@
\fi
\def\newluafunction{%
  \e@alloc\luafunction\e@alloc@chardef
    \e@alloc@luafunction@count\m@ne\e@alloc@top
}
\ifx\e@alloc@whatsit@count\@undefined
  \countdef\e@alloc@whatsit@count=261
  \e@alloc@whatsit@count=\z@
\fi
\def\newwhatsit#1{%
  \e@alloc\whatsit\e@alloc@chardef
    \e@alloc@whatsit@count\m@ne\e@alloc@top#1%
}
\ifx\e@alloc@bytecode@count\@undefined
  \countdef\e@alloc@bytecode@count=262
  \e@alloc@bytecode@count=\z@
\fi
\def\newluabytecode#1{%
  \e@alloc\luabytecode\e@alloc@chardef
    \e@alloc@bytecode@count\m@ne\e@alloc@top#1%
}

\ifx\e@alloc@luachunk@count\@undefined
  \countdef\e@alloc@luachunk@count=263
  \e@alloc@luachunk@count=\z@
\fi
\def\newluachunkname#1{%
  \e@alloc\luachunk\e@alloc@chardef
    \e@alloc@luachunk@count\m@ne\e@alloc@top#1%
    {\escapechar\m@ne
    \directlua{lua.name[\the\allocationnumber]="\string#1"}}%
}
\def\now@and@everyjob#1{%
  \everyjob\expandafter{\the\everyjob
    #1%
  }%
  #1%
}
  \begingroup
    \attributedef\attributezero=0 %
    \chardef     \charzero     =0 %
    \countdef    \CountZero    =0 %
    \dimendef    \dimenzero    =0 %
    \mathchardef \mathcharzero =0 %
    \muskipdef   \muskipzero   =0 %
    \skipdef     \skipzero     =0 %
    \toksdef     \tokszero     =0 %
    \directlua{require("ltluatex")}
  \endgroup
\catcode`\@=\etatcatcode\relax
\endinput
%%
%% End of file `ltluatex.tex'.
%
\fi
\def\@setcatcodetable#1#2{%
  \begingroup
    #2%
    \savecatcodetable#1%
  \endgroup
}
\def\@setrangecatcode#1#2#3{%
  \ifnum#1>#2 %
    \expandafter\@gobble
  \else
    \expandafter\@firstofone
  \fi
    {%
      \catcode#1=#3 %
      \expandafter\@setrangecatcode\expandafter
        {\number\numexpr#1+1\relax}{#2}{#3}%
    }%
}
\def\@catcodetablelist{}
\def\@catcodetablestack{}
\newcount\@catcodetablestackcnt
\def\@pushcatcodetable{%
  \ifx\@catcodetablelist\empty
    \global\advance\@catcodetablestackcnt by\@ne
    \edef\@tempa{\csname ct@\the\@catcodetablestackcnt\endcsname}%
    \expandafter\newcatcodetable\@tempa
    \xdef\@catcodetablelist{\@tempa}%
  \fi
  \expandafter\@pushctbl\@catcodetablelist\@nil
}
\def\@pushctbl#1#2\@nil{%
  \gdef\@catcodetablelist{#2}%
  \xdef\@catcodetablestack{#1\@catcodetablestack}%
  \savecatcodetable#1%
}
\def\@popcatcodetable{%
  \if!\@catcodetablestack!%
    \errmessage{Attempt to pop empty catcodetable stack}%
  \else
    \expandafter\@popctbl\@catcodetablestack\@nil
  \fi
}
\def\@popctbl#1#2\@nil{%
  \gdef\@catcodetablestack{#2}%
  \xdef\@catcodetablelist{\@catcodetablelist#1}%
  \catcodetable#1%
}
\catcode`\@\ctstackatcatcode\relax
%
  \else
    \RequirePackage{ctablestack}
  \fi
\fi
\def\RequireLuaModule#1{\directlua{require("#1")}\@gobbleoptarg}
\ifdefined\@ifnextchar
\def\@gobbleoptarg{\@ifnextchar[\@gobble@optarg{}}%
\else
\long\def\@gobbleoptarg#1{\ifx[#1\expandafter\@gobble@optarg\fi#1}%
\fi
\def\@gobble@optarg[#1]{}
\let\CatcodeTableIniTeX\catcodetable@initex
\let\CatcodeTableString\catcodetable@string
\let\CatcodeTableLaTeX\catcodetable@latex
\let\CatcodeTableLaTeXAtLetter\catcodetable@atletter
\newcatcodetable\CatcodeTableOther
\@setcatcodetable\CatcodeTableOther{%
  \catcodetable\CatcodeTableString
  \catcode32 12 }
\newcatcodetable\CatcodeTableExpl
\@setcatcodetable\CatcodeTableExpl{%
  \catcodetable\CatcodeTableLaTeX
  \catcode126 10 % tilde is a space char
  \catcode32  9  % space is ignored
  \catcode9   9  % tab also ignored
  \catcode95  11 % underscore letter
  \catcode58  11 % colon letter
}
\def\BeginCatcodeRegime#1{%
  \@pushcatcodetable
  \catcodetable#1\relax}
\def\EndCatcodeRegime{%
  \@popcatcodetable}
\let\PushCatcodeTableNumStack\@pushcatcodetable
\let\PopCatcodeTableNumStack\@popcatcodetable
\let\SetCatcodeRange\@setrangecatcode
\let\setcatcodetable\@setcatcodetable
\directlua{
function luatexbase.reset_callback(name,make_false)
  for _,v in pairs(luatexbase.callback_descriptions(name))
  do
    luatexbase.remove_from_callback(name,v)
  end
  if make_false == true then
    luatexbase.disable_callback(name)
  end
end
luatexbase.base_add_to_callback=luatexbase.add_to_callback
function luatexbase.add_to_callback(name,fun,description,priority)
  local priority= priority
  if priority==nil then
   priority=\string#luatexbase.callback_descriptions(name)+1
  end
  if(luatexbase.callbacktypes[name] == 3 and
     priority == 1 and
     \string#luatexbase.callback_descriptions(name)==1) then
    luatexbase.module_warning("luatexbase",
                              "resetting exclusive callback: " .. name)
    luatexbase.reset_callback(name)
  end
  local saved_callback={},ff,dd
  for k,v in pairs(luatexbase.callback_descriptions(name)) do
    if k >= priority then
      ff,dd= luatexbase.remove_from_callback(name, v)
      saved_callback[k]={ff,dd}
    end
  end
  luatexbase.base_add_to_callback(name,fun,description)
  for k,v in pairs(saved_callback) do
    luatexbase.base_add_to_callback(name,v[1],v[2])
  end
  return
end
luatexbase.catcodetables=setmetatable(
 {['latex-package'] = \number\CatcodeTableLaTeXAtLetter,
  ini    = \number\CatcodeTableIniTeX,
  string = \number\CatcodeTableString,
  other  = \number\CatcodeTableOther,
  latex  = \number\CatcodeTableLaTeX,
  expl   = \number\CatcodeTableExpl,
  expl3  = \number\CatcodeTableExpl},
 { __index = function(t,key)
    return luatexbase.registernumber(key) or nil
  end}
)}
\ifnum\luatexversion<80 %
\def\newcatcodetable#1{%
  \e@alloc\catcodetable\chardef
    \e@alloc@ccodetable@count\m@ne{"8000}#1%
  \initcatcodetable\allocationnumber
  {\escapechar=\m@ne
  \directlua{luatexbase.catcodetables['\string#1']=%
    \the\allocationnumber}}%
}
\fi
\directlua{
function luatexbase.priority_in_callback (name,description)
  for i,v in ipairs(luatexbase.callback_descriptions(name))
  do
    if v == description then
      return i
    end
  end
  return false
end
luatexbase.is_active_callback = luatexbase.in_callback
luatexbase.base_provides_module=luatexbase.provides_module
function luatexbase.errwarinf(name)
    return
    function(s,...) return luatexbase.module_error(name, s:format(...)) end,
    function(s,...) return luatexbase.module_warning(name, s:format(...)) end,
    function(s,...) return luatexbase.module_info(name, s:format(...)) end,
    function(s,...) return luatexbase.module_info(name, s:format(...)) end
end
function luatexbase.provides_module(info)
  luatexbase.base_provides_module(info)
  return luatexbase.errwarinf(info.name)
end
}
\ifnum\luatexversion<80 %
\def\newattribute#1{%
  \e@alloc\attribute\attributedef
    \e@alloc@attribute@count\m@ne\e@alloc@top#1%
  {\escapechar=\m@ne
  \directlua{luatexbase.attributes['\string#1']=%
    \the\allocationnumber}}%
}
\fi
\ifx\@percentchar\@undefined
  {\catcode`\%=12 \gdef\@percentchar{%}}
\fi
\directlua{%
local copynode          = node.copy
local newnode           = node.new
local nodesubtype       = node.subtype
local nodetype          = node.id
local stringformat      = string.format
local tableunpack       = unpack or table.unpack
local texiowrite_nl     = texio.write_nl
local texiowrite        = texio.write
local whatsit_t         = nodetype"whatsit"
local user_defined_t    = nodesubtype"user_defined"
local unassociated      = "__unassociated"
local user_whatsits       = {  __unassociated = { } }
local whatsit_ids         = { }
local anonymous_whatsits  = 0
local anonymous_prefix    = "anon"
local new_user_whatsit_id = function (name, package)
    if name then
        if not package then
            package = unassociated
        end
    else % anonymous
        anonymous_whatsits = anonymous_whatsits + 1
        warning("defining anonymous user whatsit no. \@percentchar
                  d", anonymous_whatsits)
        package = unassociated
        name    = anonymous_prefix .. tostring(anonymous_whatsits)
    end

    local whatsitdata = user_whatsits[package]
    if not whatsitdata then
        whatsitdata             = { }
        user_whatsits[package]  = whatsitdata
    end

    local id = whatsitdata[name]
    if id then %- warning
        warning("replacing whatsit \@percentchar s:\@percentchar
                  s (\@percentchar d)", package, name, id)
    else %- new id
        id=luatexbase.new_whatsit(name)
        whatsitdata[name]   = id
        whatsit_ids[id]     = { name, package }
    end
    return id
end
luatexbase.new_user_whatsit_id = new_user_whatsit_id
local new_user_whatsit = function (req, package)
    local id, whatsit
    if type(req) == "string" then
        id              = new_user_whatsit_id(req, package)
        whatsit         = newnode(whatsit_t, user_defined_t)
        whatsit.user_id = id
    elseif req.id == whatsit_t and req.subtype == user_defined_t then
        id      = req.user_id
        whatsit = copynode(req)
        if not whatsit_ids[id] then
            warning("whatsit id \@percentchar d unregistered; "
                    .. "inconsistencies may arise", id)
        end
    end
    return function () return copynode(whatsit) end, id
end
luatexbase.new_user_whatsit         = new_user_whatsit
local get_user_whatsit_id = function (name, package)
    if not package then
        package = unassociated
    end
    return user_whatsits[package][name]
end
luatexbase.get_user_whatsit_id = get_user_whatsit_id
local get_user_whatsit_name = function (asked)
    local id
    if type(asked) == "number" then
        id = asked
    elseif type(asked) == "function" then
        %- node generator
        local n = asked()
        id = n.user_id
    else %- node
        id = asked.user_id
    end
    local metadata = whatsit_ids[id]
    if not metadata then % unknown
        warning("whatsit id \@percentchar d unregistered;
                   inconsistencies may arise", id)
        return "", ""
    end
    return tableunpack(metadata)
end
luatexbase.get_user_whatsit_name = get_user_whatsit_name
local dump_registered_whatsits = function (asked_package)
    local whatsit_list = { }
    if asked_package then
        local whatsitdata = user_whatsits[asked_package]
        if not whatsitdata then
            error("(no user whatsits registered for package
                      \@percentchar s)", asked_package)
            return
        end
        texiowrite_nl("(user whatsit allocation stats for " ..
                          asked_package)
        for name, id in next, whatsitdata do
            whatsit_list[\string#whatsit_list+1] =
                stringformat("(\@percentchar s:\@percentchar
                     s \@percentchar d)", asked_package, name, id)
        end
    else
        texiowrite_nl("(user whatsit allocation stats")
        texiowrite_nl(stringformat(" ((total \@percentchar d)\string\n
                         (anonymous \@percentchar d))",
            current_whatsit, anonymous_whatsits))
        for package, whatsitdata in next, user_whatsits do
            for name, id in next, whatsitdata do
                whatsit_list[\string#whatsit_list+1] =
                    stringformat("(\@percentchar s:\@percentchar
                        s \@percentchar d)", package, name, id)
            end
        end
    end
    texiowrite_nl" ("
    local first = true
    for i=1, \string#whatsit_list do
        if first then
            first = false
        else % indent
            texiowrite_nl"  "
        end
        texiowrite(whatsit_list[i])
    end
    texiowrite"))\string\n"
end
luatexbase.dump_registered_whatsits = dump_registered_whatsits
luatexbase.newattribute            = new_attribute
luatexbase.newuserwhatsit          = new_user_whatsit
luatexbase.newuserwhatsitid        = new_user_whatsit_id
luatexbase.getuserwhatsitid        = get_user_whatsit_id
luatexbase.getuserwhatsitname      = get_user_whatsit_name
luatexbase.dumpregisteredwhatsits  = dump_registered_whatsits
}
\let\newluatexattribute\newattribute
\let\setluatexattribute\setattribute
\let\unsetluatexattribute\unsetattribute
\let\newluatexcatcodetable\newcatcodetable
\let\setluatexcatcodetable\setcatcodetable
\let\luatexbase@directlua\directlua
\let\luatexbase@ensure@primitive\@gobble
\let\luatexattribute\attribute
\let\luatexattributedef\attributedef
\let\luatexcatcodetable\catcodetable
\let\luatexluaescapestring\luaescapestring
\let\luatexlatelua\latelua
\let\luatexoutputbox\outputbox
\let\luatexscantextokens\scantextokens
\catcode`\@=\emuatcatcode\relax
%
\else
  \RequirePackage{luatexbase}
\fi
%    \end{macrocode}
%
%    \begin{macrocode}
%</emu-cmp,emu-mod,emu-loa,emu-reg,emu-att,emu-cct,emu-mcb>
%    \end{macrocode}
%
% \subsection{Legacy Lua code}
%
% \changes{v1.2}{2015/10/03}{Provide \texttt{luatexbase.loader.lua}}
%
% The original \textsf{luatexbase} included a file |luatexbase.loader.lua|
% that could be loaded independently of the rest of the package. This really
% doesn't need to do anything!
%    \begin{macrocode}
%<*emu-lua>
%    \end{macrocode}
%
%    \begin{macrocode}
luatexbase = luatexbase or { }
%    \end{macrocode}
%
%    \begin{macrocode}
%</emu-lua>
%    \end{macrocode}
%
% \Finale
