% !TeX encoding = UTF-8
% Ce fichier contient le code de l'extension "chemfig"
%
%%%%%%%%%%%%%%%%%%%%%%%%%%%%%%%%%%%%%%%%%%%%%%%%%%%%%%%%%%%%%%%%%%%%%%
%                                                                    %
\def\CFname                    {chemfig}                             %
\def\CFver                       {1.6c}                              %
%                                                                    %
\def\CFdate                   {2022/09/27}                           %
%                                                                    %
%%%%%%%%%%%%%%%%%%%%%%%%%%%%%%%%%%%%%%%%%%%%%%%%%%%%%%%%%%%%%%%%%%%%%%
%
%____________________________________________________________________
% Author     : Christian Tellechea                                   |
% Status     : Maintained                                            |
% Email      : unbonpetit@netc.fr                                    |
% Package URL: https://www.ctan.org/pkg/chemfig                      |
% Copyright  : Christian Tellechea 2010-2022                         |
% Licence    : Released under the LaTeX Project Public License v1.3c |
%              or later, see http://www.latex-project.org/lppl.txt   |
% Files      : 1) chemfig.tex                                        |
%              2) chemfig.sty                                        |
%              3) README                                             |
%              4) chemfig_doc_fr.tex                                 |
%              5) chemfig_doc_fr.pdf                                 |
%              6) chemfig_doc_en.tex                                 |
%              7) chemfig_doc_fr.pdf                                 |
%              8) chemfig-lewis.tex                                  |
%--------------------------------------------------------------------

%%%%%%%%%%%%%%%%%%%%%%%%%%%%%%%%%%%%%%%%%%%%%%%%%%%%%%%%%%%%%%%%%%%%%%
%%%                       P R É A L A B L E                        %%%
%%%%%%%%%%%%%%%%%%%%%%%%%%%%%%%%%%%%%%%%%%%%%%%%%%%%%%%%%%%%%%%%%%%%%%
\csname CFloadonce\endcsname
\let\CFloadonce\endinput
%-------------------------- Annonce package --------------------------
\ifdefined\CFfromsty\else
	\immediate\write -1 {%
	Package: \CFname\space\CFdate\space\space v\CFver\space\space
	Draw molecule with an easy syntax (CT)}%
\fi

%-------------------------- Régime catcodes --------------------------
\begingroup
	\def\X#1{\catcode\number`#1=\number\catcode`#1\relax}
	\xdef\CFrestorecatcode{\X\[\X\]\X\:\X\(\X\)\X\,\X\-\X\=\X\~\X\!\X\?\X\<\X\>\X\;\X\*\X\|\X\@\X\ \X\_}%
\endgroup
\catcode`\[12 \catcode`\]12 \catcode`\:12 \catcode`\(12
\catcode`\)12 \catcode`\,12 \catcode`\-12 \catcode`\=12
\catcode`\~12 \catcode`\!12 \catcode`\?12 \catcode`\<12
\catcode`\>12 \catcode`\;12 \catcode`\*12 \catcode`\|12
\catcode`\@12 \catcode`\#6  \catcode`\ 10 \catcode`\_11

%-------------------- Vérification des prérequis ---------------------
\def\CF_error#1{\errmessage{Package \CFname\space Error: #1.}}
\def\CF_warning#1{\immediate\write-1{Package \CFname\space Warning: #1^^J}}
\def\CF_checkprimitive#1#2{% Vérifie que #1 est une primitive et sinon, émet le message #2 et exécute \endinput
	\begingroup
		\edef\__tempa{\meaning#1}\edef\__tempb{\string#1}\expandafter
	\endgroup
	\ifx\__tempa\__tempb\else
		\CF_error{#2}%
		\def\CF_temp{\CFrestorecatcode\endinput}%
		\expandafter\CF_temp
	\fi
}
\CF_checkprimitive\eTeXversion{You are not using an eTeX engine, \CFname\space cannot work.}
\CF_checkprimitive\expanded{the \string\expanded\space primitive is not provided by your TeX engine, \CFname\space cannot work.}

%------------------------ Chargement simplekv ------------------------
\input simplekv.tex

%-------------------------- Chargement tikz --------------------------
\unless\ifdefined\tikzpicture
	\begingroup\def\CFtemp{\endgroup\input tikz.tex\relax}%
	\expandafter\CFtemp
\fi
\usetikzlibrary{arrows.meta}

%--------------------------- Allocations -----------------------------
\newcount\CF_cntatomgroup
\newcount\CF_cntgroup
\newcount\CF_cntatom
\newcount\CF_cntcycle
\newcount\CF_cntcyclebonds
\newcount\CF_cntcompound

\newif\ifCF_incycle
\newif\ifCF_cyclearc
\newif\ifCF_definesubmol
\newif\ifCF_adjustnamedp
\newif\ifCF_macrofixedbondlength
\newif\ifCF_compound_is_chemfig

\newdimen\CF_dim
\newdimen\CF_arrowsize
\newdimen\CF_zero \CF_zero=0pt

\newbox\CF_box
\newbox\CF_boxstuff
\newbox\CF_testbox
\newbox\CF_chargebox

\newtoks\CF_substtoks

%-------------------------- Petites macros ---------------------------
\let\CF_begintikzpicture\tikzpicture
\let\CF_endtikzpicture  \endtikzpicture
\def\CF_quark{\CF_quark}
\def\CF_execfirst#1#2{#1}
\def\CF_execsecond#1#2{#2}
\def\CF_id#1{#1}
\def\CF_gobarg#1{}
\def\CF_gobtwoargs#1#2{}
\def\CF_firsttonil#1#2\_nil{#1}
\def\CF_sanitizelastitem#1,\empty#2\_nil{#1}
\def\CF_gobtikzinstruction#1;{}
\def\CF_makeother#1{\catcode`#1=12\relax}
\def\CF_lettoken#1#2{\let#1= #2}\CF_lettoken\CF_sptoken{ }
\def\CF_ifx#1#2{\ifx#1#2\expandafter\CF_execfirst\else\expandafter\CF_execsecond\fi}
\def\CF_ifempty#1{\ifx\empty#1\empty\expandafter\CF_execfirst\else\expandafter\CF_execsecond\fi}
\def\CF_ifnum#1{\ifnum#1\expandafter\CF_execfirst\else\expandafter\CF_execsecond\fi}
\def\CF_ifinsidetikz{\ifdefined\pgfpictureid\expandafter\CF_execfirst\else\expandafter\CF_execsecond\fi}
\def\CF_ifzerodim#1{%
	\setbox\CF_testbox\hbox{\pgfinterruptpicture\printatom{#1}\endpgfinterruptpicture}% bugfix 1.53
	\CF_ifnum{1\ifdim\wd\CF_testbox=\CF_zero0\fi\ifdim\ht\CF_testbox=\CF_zero0\fi\ifdim\dp\CF_testbox=\CF_zero0\fi=1000 }
}
\def\CF_doifempty#1{\ifx\empty#1\empty\expandafter\CF_id\else\expandafter\CF_gobarg\fi}
\def\CF_doifnotempty#1{\ifx\empty#1\empty\expandafter\CF_gobarg\else\expandafter\CF_id\fi}
\def\CF_gobtonil#1\_nil{}
\def\CF_striplastsp#1{%
	\long\def\CF_stripsp##1##2{\expanded{\CF_stripsp_i\_marksp##1\__nil\_marksp#1\_marksp\_nil{##2}}}%
	\long\def\CF_stripsp_i##1\_marksp#1##2\_marksp##3\_nil{\CF_stripsp_ii##3##1##2\__nil#1\__nil\_nil}%
	\long\def\CF_stripsp_ii##1#1\__nil##2\_nil{\CF_stripsp_iii##1##2\_nil}%
	\long\def\CF_stripsp_iii##1##2\__nil##3\_nil##4{\unexpanded{##4{##2}}}%
	\long\def\CF_striplastsp##1##2{\expanded{\CF_striplastsp_i\_marksp##1\__nil#1\__nil\_nil{##2}}}%
	\long\def\CF_striplastsp_i##1#1\__nil##2\_nil{\CF_stripsp_iii##1\__nil\_nil}%
}\CF_striplastsp{ }
\edef\CFhash{\string#}
\begingroup
	\catcode`\_8
	\expandafter\gdef\csname CF\string_underscore\endcsname{_}
\endgroup
\def\CF_threeea{\expandafter\expandafter\expandafter}
\def\CF_exptwomacroargs#1#2#3{\CF_expsecond{\CF_expsecond#1{#2}}{#3}}
\def\CF_expthreemacroargs#1#2#3#4{\CF_expsecond{\CF_exptwomacroargs#1{#2}{#3}}{#4}}
\def\CF_swaparg#1#2{#2{#1}}
\def\CF_expsecond#1#2{\expandafter\CF_swaparg\expandafter{#2}{#1}}% \CF_expsecond<{arg1>}{<arg2>} donne "<arg1>{*<arg2>}"
\def\CF_eexpsecond#1#2{\expandafter\expandafter\expandafter\CF_swaparg\expandafter\expandafter\expandafter{#2}{#1}}% \CF_eexpsecond{<arg1>}{<arg2>} donne "<arg1>{**<arg2>}"
\def\CF_swapunbrace#1#2{#2#1}
\def\CF_expafter#1#2{\expandafter\CF_swapunbrace\expandafter{#2}{#1}}% \CF_expafter{<arg1>}{<arg2>} donne "<arg1>*<arg2>"
\def\CF_eexpafter#1#2{\expandafter\expandafter\expandafter\CF_swapunbrace\expandafter\expandafter\expandafter{#2}{#1}}% \CF_eexpafter{<arg1>}{<arg2>} donne "<arg1>**<arg2>"
\def\CF_addtomacro#1#2{\CF_expsecond{\def#1}{#1#2}}
\def\CF_eaddtomacro#1#2{\CF_expsecond{\CF_addtomacro#1}{#2}}
\def\CF_preaddtomacro#1#2{\CF_expsecond{\CF_preaddtomacroa#1{#2}}#1}
\def\CF_preaddtomacroa#1#2#3{\def#1{#2#3}}
\def\CF_addtotoks#1#2{#1\expandafter{\the#1#2}}
\def\CF_eaddtotoks#1#2{\expandafter\CF_addtotoks\expandafter#1\expandafter{#2}}
\def\CF_assigntonil#1#2\_nil{\def#1{#2}}
\def\CF_edefaddtomacro#1#2{\CF_expsecond{\CF_addtomacro#1}{\expanded{#2}}}
\def\CF_ifnextchar#1#2#3{%
	\let\CF_ifnextchartok=#1% <- espace indésirable, bugfix v1.31
	\def\CF_ifnextcharcodetrue{#2}%
	\def\CF_ifnextcharcodefalse{#3}%
	\futurelet\CF_temptok\CF_ifnextchara
}
\def\CF_ifnextchara{%
	\CF_ifx\CF_temptok\CF_sptoken
		{\CF_ifnextcharb
		}
		{\CF_ifx\CF_temptok\CF_ifnextchartok
			\CF_ifnextcharcodetrue
			\CF_ifnextcharcodefalse
		}%
}
\expandafter\def\expandafter\CF_ifnextcharb\space{\futurelet\CF_temptok\CF_ifnextchara}
\def\CF_ifstar#1{\CF_ifnextchar*{\CF_execfirst{#1}}}
\def\CF_testopt#1#2{\CF_ifnextchar[{#1}{#1[{#2}]}}
\def\CF_ifinteger#1{%
	\begingroup
		\afterassignment\CF_afterinteger
		\CF_cntcyclebonds0#1\relax
}
\def\CF_afterinteger#1\relax{%
	\endgroup
	\CF_ifempty{#1}%
}
\def\CF_iffirsttokmatch#1#2{% est ce que #1 et #2 commencent par les mêmes tokens ?
	\futurelet\CF_toksa\CF_gobtonil#1\relax\_nil
	\futurelet\CF_toksb\CF_gobtonil#2\relax\_nil
	\CF_ifx\CF_toksa\CF_toksb
}
\def\CF_iffirsttokin#1{% teste si le token qui commence #1 appartient aux tokens mis dans #2
	\futurelet\CF_toksa\CF_gobtonil#1\relax\_nil
	\CF_iffirsttokina
}
\def\CF_iffirsttokina#1{%
	\CF_ifempty{#1}
		{\CF_execsecond
		}
		{\futurelet\CF_toksb\CF_gobtonil#1\relax\_nil
		\CF_ifx\CF_toksa\CF_toksb
			{\CF_execfirst}
			{\CF_expsecond\CF_iffirsttokina{\CF_gobarg#1}}%
		}%
}
\def\CF_ifinstr#1#2{%
	\def\CF_ifinstra##1#2##2\_nil{%
		\ifx\empty##2\empty
			\expandafter\CF_execsecond
		\else
			\expandafter\CF_execfirst
		\fi}%
	\CF_ifinstra#1\__nil#2\_nil
}
\def\CF_afterspace#1 #2\_nil{#2}

%--------------------------- Substitution ----------------------------
\def\CF_ifstartwith#1#2{% #1=<texte>  #2=<motif>
	\CF_ifempty{#1}%
		{\CF_execsecond
		}
		{\def\CF_startwithcode{#1}%
		\def\CF_startwithpattern{#2}%
		\CF_ifstartwitha
		}%
}
\def\CF_ifstartwitha{%
	\CF_grabfirstarg\CF_startwithcode\CF_firstargcode
	\CF_grabfirstarg\CF_startwithpattern\CF_firstargpattern
	\CF_ifx\CF_firstargcode\CF_firstargpattern
		{\CF_expsecond\CF_ifempty\CF_startwithpattern
			{\CF_execfirst
			}
			{\CF_expsecond\CF_ifempty\CF_startwithcode
				\CF_execsecond
				\CF_ifstartwitha
			}%
		}
		{\CF_execsecond
		}%
}
\def\CF_grabfirstarg#1#2{%
	\CF_ifx#1\empty
		{\let#2\empty
		}
		{\def\CF_grabmacro{#2}%
		\CF_expsecond\CF_ifbracefirst#1%
			{\expandafter\CF_grabbracearg#1\_nil#1}
			{\CF_expafter{\futurelet\CF_nexttok\CF_grabfirstarga}#1\_nil#1}%
		}%
}
\def\CF_grabfirstarga{%
	\CF_ifx\CF_nexttok\CF_sptoken
		\CF_grabspacearg
		\CF_grabnormalarg
}
\def\CF_grabbracearg#1{%
	\expandafter\def\CF_grabmacro{{#1}}%
	\CF_grabargassigntonil\relax
}
\expandafter\def\expandafter\CF_grabspacearg\space{%
	\expandafter\def\CF_grabmacro{ }%
	\CF_grabargassigntonil\relax
}
\def\CF_grabnormalarg#1{%
	\expandafter\def\CF_grabmacro{#1}%
	\CF_grabargassigntonil\relax
}
\def\CF_grabargassigntonil#1\_nil#2{\CF_expsecond{\def#2}{\CF_gobarg#1}}
\def\CF_ifbracefirst#1{\CF_ifnum{\catcode\CF_threeea`\expandafter\CF_firsttonil\detokenize{#1.}\_nil=1 }}
\def\CF_substonly#1#2{% #1=entier maxi>0  #2=macro : dans la sc#1, remplace tous les <motif> par <pattern> sauf lorsque le motif est suivi d'un caractère >#1
	\def\CF_atendsubstitute{\edef#2{\the\CF_substtoks}}% macro exécutée à la fin
	\let\CF_substnogroups\CF_substnogrouponly
	\CF_ifnum{#1>0 }
		{\let\CF_testifx\empty
		\foreach\CF_x in {1,...,#1}
			{\xdef\CF_testifx{\unexpanded\expandafter{\CF_testifx}\unexpanded\expandafter{\expandafter\ifx\CF_x\CF_nexttok1\fi}}}%
		\let\CF_testif\empty
		\foreach\CF_x in {1,...,#1}
			{\xdef\CF_testif{\unexpanded\expandafter{\CF_testif}\unexpanded\expandafter{\expandafter\if\CF_x\CF_nexttok1\fi}}}%
		\CF_expsecond\CF_substi#2%
		}
		{\CF_substall#2%
		}%
}
\def\CF_substnogrouponly{%
	\CF_exptwomacroargs\CF_ifstartwith\CF_substcode\CF_substsubst
		{\CF_grabfirstarg\CF_substcode\CF_temp
		\CF_expafter{\futurelet\CF_nexttok\CF_gobtonil}\CF_substcode\relax\_nil
		\CF_ifnum{0\CF_testifx=1 }% si le prochain token est \let-égal à 1...#1
			{\edef\CF_nexttok{\CF_threeea\CF_firsttonil\expandafter\string\CF_substcode\_nil}% le détokéniser
			\CF_ifnum{0\CF_testif=1 }
				{\CF_eaddtotoks\CF_substtoks\CF_temp
				\CF_grabfirstarg\CF_substcode\CF_temp
				\CF_eaddtotoks\CF_substtoks\CF_temp
				}
				{\CF_eaddtotoks\CF_substtoks\CF_substpattern
				}%
			}
			{\CF_eaddtotoks\CF_substtoks\CF_substpattern
			}%
		\CF_substgroups
		}
		{\CF_expsecond\CF_ifempty\CF_substcode
			{\CF_atendsubstitute
			}
			{\CF_grabfirstarg\CF_substcode\CF_substauxarg
			\CF_eaddtotoks\CF_substtoks\CF_substauxarg
			\CF_substgroups
			}%
		}%
}
\def\CF_substall#1{% #1=macro
	\def\CF_atendsubstitute{\edef#1{\the\CF_substtoks}}% macro exécutée à la fin
	\let\CF_substnogroups\CF_substnogroupall
	\CF_expsecond\CF_substi#1%
}
\def\CF_substnogroupall{%
	\CF_exptwomacroargs\CF_ifstartwith\CF_substcode\CF_substsubst
		{\CF_eaddtotoks\CF_substtoks\CF_substpattern
		\CF_grabfirstarg\CF_substcode\CF_temp
		\CF_substgroups
		}
		{\CF_expsecond\CF_ifempty\CF_substcode
			{\CF_atendsubstitute
			}
			{\CF_grabfirstarg\CF_substcode\CF_substauxarg
			\CF_eaddtotoks\CF_substtoks\CF_substauxarg
			\CF_substgroups
			}%
		}%
}
\def\CF_substi#1#2#3{% #1=<texte> #2=<motif> #3=<motif de substi>
	\def\CF_substcode{#1}\def\CF_substsubst{#2}\def\CF_substpattern{#3}%
	\CF_substtoks={}%
	\CF_substgroups
}

\def\CF_substgroups{%
	\CF_expsecond\CF_ifbracefirst\CF_substcode
		{\CF_grabfirstarg\CF_substcode\CF_substauxarg
		\begingroup
			\def\CF_atendsubstitute{%
				\expandafter\endgroup\expandafter\CF_addtotoks\expandafter\CF_substtoks\expandafter{\expandafter{\the\CF_substtoks}}%
				\CF_substgroups
			}%
			\CF_substtoks{}% initialiser à vide
			\expandafter\def\expandafter\CF_substcode\CF_substauxarg
			\CF_substgroups
		}%
		{\CF_substnogroups
		}%
}

%---------------------------- Paramètres -----------------------------
\def\CF_defifempty#1#2#3{\CF_ifempty{#2}{\def#1{#3}}{\def#1{#2}}}
\defKV[chemfig]{%
	atom style          = \def\CF_atomstyle                    {#1},
	chemfig style       = \def\CF_chemfigstyle                 {#1},
	cram width          = \CF_defifempty\CF_crambasewidth      {#1}{1.5ex},
	cram dash width     = \CF_defifempty\CF_cramdashlength     {#1}{1pt},
	cram dash sep       = \CF_defifempty\CF_cramdashsep        {#1}{2pt},
	atom sep            = \CF_defifempty\CF_atomsep            {#1}{3em},
	bond offset         = \CF_defifempty\CF_bondoffset         {#1}{2pt},
	double bond sep     = \CF_defifempty\CF_doublesep          {#1}{2pt},
	angle increment     = \CF_defifempty\CF_angleincrement     {#1}{45},
	node style          = \def\CF_nodestyle                    {#1},
	bond style          = \def\CF_bondstyle                    {#1},
	cycle radius coeff  = \CF_defifempty\CF_cycleradiuscoeff   {#1}{0.75},
	stack sep           = \CF_defifempty\CF_stacksep           {#1}{1.5pt},
	compound style      = \def\CF_defaultcompoundstyle         {#1},
	compound sep        = \CF_defifempty\CF_compoundsep        {#1}{5em},
	arrow offset        = \CF_defifempty\CF_arrowoffset        {#1}{1em},
	arrow angle         = \CF_defifempty\CF_arrowangle         {#1}{0},
	arrow coeff         = \CF_defifempty\CF_arrowlength        {#1}{1},
	arrow style         = \def\CF_defaultarrowstyle            {#1},
	arrow double sep    = \CF_defifempty\CF_arrowdoublesep     {#1}{2pt},
	arrow double coeff  = \CF_defifempty\CF_arrowdoubleposstart{#1}{0.6},
	arrow label sep     = \CF_defifempty\CF_arrowlabelsep      {#1}{3pt},
	arrow head          = \CF_defifempty\CF_arrowhead          {#1}{-CF},
	+ sep left          = \CF_defifempty\CF_signspaceante      {#1}{0.5em},
	+ sep right         = \CF_defifempty\CF_signspacepost      {#1}{0.5em},
	+ vshift            = \CF_defifempty\CF_signvshift         {#1}{0pt}
}
\def\setchemfig{\setKV[chemfig]}
\def\resetchemfig{\restoreKV[chemfig]}
\setKVdefault[chemfig]{%
	atom style          = {},% code tikz mis à la fin de every node/.style
	chemfig style       = {},% code tikz mis à la fin de l'arugment optionnel de \tikzpicture
	bond join           = false,
	fixed length        = false,
	cram rectangle      = false,
	cram width          = 1.5ex,
	cram dash width     = 1pt,
	cram dash sep       = 2pt,
	atom sep            = 3em,
	bond offset         = 2pt,
	double bond sep     = 2pt,
	angle increment     = 45,
	node style          = {},
	bond style          = {},
	cycle radius coeff  = 0.75,
	stack sep           = 1.5pt,
	autoreset cntcycle  = true,
	show cntcycle       = false,
	debug               = false,
	scheme debug        = false,
	compound style      = {},
	compound sep        = 5em,
	arrow offset        = 1em,
	arrow angle         = 0,
	arrow coeff         = 1,
	arrow style         = {},
	arrow double sep    = 2pt,
	arrow double coeff  = 0.6,
	arrow double harpoon= true,
	arrow label sep     = 3pt,
	arrow head          = -CF,
	+ sep left          = 0.5em,
	+ sep right         = 0.5em,
	+ vshift            = 0pt,
	gchemname           = true,% ajout de cette clé v1.6c
	schemestart code    = {},% ajout de cette clé v1.6c
	schemestop code     = {}% ajout de cette clé v1.6c
}%

%%%%%%%%%%%%%%%%%%%%%%%%%%%%%%%%%%%%%%%%%%%%%%%%%%%%%%%%%%%%%%%%%%%%%%
%%%               D E S S I N     M O L É C U L E S                %%%
%%%%%%%%%%%%%%%%%%%%%%%%%%%%%%%%%%%%%%%%%%%%%%%%%%%%%%%%%%%%%%%%%%%%%%
\def\CF_sanitizecatcode{%
	\CF_makeother\[\CF_makeother\]\CF_makeother\:\CF_makeother\(\CF_makeother\)%
	\CF_makeother\,\CF_makeother\-\CF_makeother\=\CF_makeother\~\CF_makeother\!%
	\CF_makeother\?\CF_makeother\<\CF_makeother\>\CF_makeother\;\CF_makeother\*%
	\CF_makeother\|\CF_makeother\#\CF_makeother\@%
}

\def\printatom#1{\ifmmode\rm#1\else$\rm#1$\fi}

\def\CF_nodecontent{\CF_expsecond\printatom{\csname atom_\number\CF_cntatom\endcsname\CF_nodestrut}}

\def\chemskipalign{%
	\CF_doifempty\CF_bondoutcontentsaved% sauf si un \chemskipalign a été fait à l'atome précédent
		{\global\let\CF_bondoutcontentsaved\CF_bondoutcontent}% sauvegarder l'atome d'où vient la liaison
	\let\CF_nodestrut\empty
}

\def\definesubmol{\CF_definesubmoltrue\def_submol}
\def\redefinesubmol{\CF_definesubmolfalse\def_submol}

\def\def_submol#1{%
	\CF_cntatomgroup=0 % nombre d'arguments supposé
	\def\CF_temp{#1}% nom
	\futurelet\CF_toksa\CF_submoltestnxttok
}

\def\CF_submoltestnxttok{%
	\if[\noexpand\CF_toksa\expandafter\CF_execfirst\else\expandafter\CF_execsecond\fi
		{\begingroup\CF_sanitizecatcode\CF_expsecond\CF_submolgrabopt{\CF_temp}%
		}
		{\afterassignment\CF_submoltestnxttoka% pas d'argument entre crochet
		\CF_cntatomgroup=0% cherche le nombre d'arguments éventuels
		}%
}

\def\CF_submoltestnxttoka{\futurelet\CF_toksa\CF_submoltestnxttokb}

\def\CF_submoltestnxttokb{%
	\if[\noexpand\CF_toksa\expandafter\CF_execfirst\else\expandafter\CF_execsecond\fi
		{\begingroup\CF_sanitizecatcode\CF_expsecond\CF_submolgrabopt{\CF_temp}}
		{\CF_expsecond{\def_submola}\CF_temp{}}%
}

\def\CF_submolgrabopt#1[#2]#{\endgroup\def_submola{#1}{#2}}

\def\def_submola#1{% #1 nom
	\CF_ifnum{0\CF_ifnum{\CF_cntatomgroup<0 }1{\CF_ifnum{\CF_cntatomgroup>9 }10}>0 }
		{\CF_error{Invalid number of arguments in submol \detokenize\expandafter{\string#1}. Defining it with 0 argument}%
		\CF_cntatomgroup=0
		}
		{}%
	\ifcat\relax\expandafter\noexpand\CF_firsttonil#1\_nil\expandafter\CF_execfirst\else\expandafter\CF_execsecond\fi% si #1 est une séquence de contrôle
		{\expandafter\ifdefined\CF_firsttonil#1\_nil
			\ifCF_definesubmol
				\CF_warning{the submol \expandafter\string\CF_firsttonil#1\_nil\space is already defined, the previous definition is lost}%
			\fi
		\fi
		\begingroup
			\CF_sanitizecatcode
			\def_submolb{#1}%
		}
		{\ifcsname CF__#1\endcsname
			\ifCF_definesubmol
				\CF_warning{the submol "#1" is already defined, the previous definition is lost}%
			\fi
		\fi
		\begingroup
			\CF_sanitizecatcode
			\expandafter\def_submolb\csname CF__#1\endcsname
		}%
}

\def\def_submolb#1#2#3{% #1 nom sous forme de macro, #2 = code si liaison arrive de droite , #3 = code si liaison arrive de gauche, \CF_cntatomgroup = nombre d'arguments
	\def\CF_tempa{#2}\CF_doifnotempty{#2}{\CF_expsecond{\CF_expsecond\CF_substonly{\number\CF_cntatomgroup}\CF_tempa}{\CFhash}{\CFhash}}%
	\def\CF_tempb{#3}\CF_expsecond{\CF_expsecond\CF_substonly{\number\CF_cntatomgroup}\CF_tempb}{\CFhash}{\CFhash}%
	\CF_expsecond{\CF_expsecond{\def_submolc{#1}}\CF_tempa}\CF_tempb
}

\def\def_submolc#1#2#3{% #1 nom sous forme de macro, #2 = code si liaison arrive de droite , #3 = code si liaison arrive de gauche, \CF_cntatomgroup = nombre d'arguments
	\endgroup
	\begingroup
		\global\toks0{\gdef\CFthesubmol}%
		\CF_ifnum{\CF_cntatomgroup>0 }
			{\foreach\CF_x in {1,...,\CF_cntatomgroup}{%
				\global\toks0\expandafter{\expanded{\the\toks0\CFhash\CF_x}}%
				}%
			}
			{}%
		\CF_sanitizecatcode \catcode`\#6 \endlinechar-1 \everyeof{\noexpand}%
		\CF_ifempty{#2}%
			{\scantokens\expandafter{\the\toks0{\empty#3}}%
			}%
			{\scantokens\expandafter{%
				\the\toks0{%
					\expanded{% bugfix 1.52
						\csname CF_exec%
							\ifdim\csname CF_currentangle\endcsname pt>90pt
								\ifdim\csname CF_currentangle\endcsname pt<270pt
									first%
								\else
									second%
								\fi
							\else
								second%
							\fi
						\endcsname
							{\unexpanded{#2}}{\unexpanded{#3}}%
						}%
					}%
				}%
			}%
	\endgroup
	\let#1\CFthesubmol
}

\def\CF_seeknode#1#2#3{% cherche un noeud au début de #1 l'assigne dans la sc #2 et met le reste dans #3
	\let#2\empty
	\def#3{#1}%
	\CF_iffirsttokmatch{#1}{ }%
		{\CF_expsecond\CF_seeknodea{\romannumeral-`\.\noexpand#1}#2#3}% ignore les espaces au début du groupe d'atome
		{\CF_seeknodea{#1}#2#3}%
}

\def\CF_seeknodea#1#2#3{%
	\CF_ifempty{#1}%
		{\let#3\empty
		}
		{\futurelet\CF_toksa\CF_gobtonil#1\relax\_nil
		\CF_ifx\CF_toksa\CF_sptoken
			{\CF_addtomacro#2{ }%
			\CF_expsecond\CF_seeknodea{\CF_afterspace#1\_nil}#2#3%
			}%
			{\CF_ifx\CF_toksa\bgroup
				{\CF_eaddtomacro#2{\expandafter{\CF_firsttonil#1\_nil}}%
				\CF_expsecond\CF_seeknodea{\CF_gobarg#1}#2#3%
				}%
				{\CF_ifx!\CF_toksa% Bugfix v1.5
					{\def\CF_seeksubmloltemp{#1}%
					\CF_seeksubmola
					\CF_expsecond\CF_seeknodea\CF_seeksubmloltemp#2#3%
					}
					{\CF_iffirsttokina{-=(*<>~}%
						{\def#3{#1}%
						}%
						{\CF_eaddtomacro#2{\CF_firsttonil#1\_nil}%
						\CF_expsecond\CF_seeknodea{\CF_gobarg#1}#2#3%
						}%
					}
				}%
			}%
		}%
}

% on sait que #1 commence par -,=,~,<,>. On analyse cette liaison
% #2 reçoit le type de liaisons (1 pour -, 2 pour =, 3 pour ~)
\def\CF_assignbondcode#1#2{%
	\futurelet\CF_toksa\CF_gobtonil#1\_nil
	\edef#2{%
		\ifx-\CF_toksa1\else
		\ifx=\CF_toksa2\else
		\ifx~\CF_toksa3\else
		\ifx>\CF_toksa4\else
		\ifx<\CF_toksa5\else0% si 0 --> il y a une erreur non due à l'utilisateur
		\fi\fi\fi\fi\fi}%
	\ifnum#2>3 % si c'est une liaison de Cram
		\CF_expafter{\futurelet\CF_toksa\CF_gobtonil}{\CF_gobarg#1\_nil}% chope le caractère suivant
		\CF_ifx:\CF_toksa
			{\edef#2{\number\numexpr#2+2}% si c 'est un ":", signe du pointillé, ajoute 2
			}%
			{\CF_ifx|\CF_toksa% si c 'est un "|", signe du triangle évidé, ajouter 4
				{\edef#2{\number\numexpr#2+4}}
				{}%
			}%
	\fi
}

\def\CF_grabbondoffseta#1,#2\_nil{%
	\def\CF_startoffset{#1}\def\CF_endoffset{#2}%
}

\def\CF_grabbondoffset#1(#2)#3\_nil{%
	\CF_doifnotempty{#2}%
		{\CF_ifinstr{#2},%
			{\CF_grabbondoffseta#2\_nil}%
			{\def\CF_startoffset{#2}}%
		}%
	\def\CF_remainafterbond{#3}%
}

\def\CF_analysebond#1#2{%
	\CF_assignbondcode{#1}#2%
	\CF_expsecond{\def\CF_remainafterbond}{\CF_gobarg#1}%mange le premier signe de la liaison
	\let\CF_doublebondtype\CF_zero
	\ifnum#2=2 % si c'est une double liaison, regarde s'il y a un + ou - derrière
		\CF_expafter{\futurelet\CF_toksa\CF_gobtonil}{\CF_gobarg#1\_nil}%
		\CF_ifx^\CF_toksa
			{\def\CF_doublebondtype{1}%
			\CF_expsecond{\def\CF_remainafterbond}{\CF_gobtwoargs#1}% mange le "^"
			}
			{\expandafter\ifx\CF_underscore\CF_toksa
				\def\CF_doublebondtype{2}%
				\CF_expsecond{\def\CF_remainafterbond}{\CF_gobtwoargs#1}% mange le "_"
			\fi
			}%
	\else
		\ifnum#2>5 % si c'est une laision de Cram pointillée ou triangle évidé
			\CF_expsecond{\def\CF_remainafterbond}{\CF_gobtwoargs#1}% mange un caractère de plus
		\fi
	\fi
	\CF_expsecond\CF_iffirsttokmatch\CF_remainafterbond\CFhash
		{\CF_eexpsecond\CF_iffirsttokmatch{\expandafter\CF_gobarg\CF_remainafterbond.}(%si parenthèse juste après
			{\expandafter\CF_grabbondoffset\CF_remainafterbond\_nil}
			{}%
		}%
		{}%
	\CF_expsecond\CF_iffirsttokmatch\CF_remainafterbond @%
		{\expandafter\CF_grabmovearg\CF_remainafterbond\_nil}%
		{}%
	\CF_expsecond\CF_iffirsttokmatch{\CF_remainafterbond}[%
		{\expandafter\CF_analyseoptarg\CF_remainafterbond\_nil\CF_remainafterbond
		}%
		{\let\CF_currentstringangle\CF_defaultstringangle
		\let\CF_currentlength\CF_defaultlength
		\let\CF_currentfromatom\CF_defaultfromatom
		\let\CF_currenttoatom\CF_defaulttoatom
		\let\CF_currenttikz\CF_defaulttikz
		\let\CF_movebondname\empty
		}%
	\ifCF_incycle
		\pgfmathsetmacro\CF_cycleincrementangle{360/\CF_cyclenum+\CF_initcycleangle}%
		\edef\CF_currentstringangle{::+\CF_cycleincrementangle}%
		\def\CF_initcycleangle{0}%
		\let\CF_currentlength\CF_defaultlength% et on ignore la longueur de liaison spécifiée
	\fi
	\CF_expsecond\CF_setbondangle{\CF_currentstringangle}\CF_currentangle
}

\def\CF_setbondangle#1#2{% le code de la direction est contenu dans #1, en sortie, #2 contient l'angle
	\CF_ifempty{#1}%
		{\let#2\CF_defaultangle
		}
		{\if:\expandafter\noexpand\CF_firsttonil#1\_nil
			\if:\CF_threeea\noexpand\expandafter\CF_firsttonil\CF_gobarg#1\_nil
				\pgfmathsetmacro#2{\CF_previousangle+\expandafter\CF_gobarg\CF_gobarg#1}%
			\else
				\pgfmathsetmacro#2{\CF_gobarg#1}%
			\fi
		\else
			\pgfmathsetmacro#2{#1*\CF_angleincrement}%
		\fi% puis normalise l'angle entre 0 et 360
		\ifdim\ifdim#2pt<0pt -\fi#2pt>360pt
			\pgfmathsetmacro#2{#2-360*floor(#2/360)}%
		\fi% si |#2|>360
		\ifdim#2pt<0pt
			\pgfmathsetmacro#2{#2+360}%
		\fi
		}%
}

\def\CF_analysemovearg#1,#2\_nil#3{%
	\def#3{#1}\def\CF_movebondcoeff{#2}%
}

% Argument limités légitimes ici car #2 (qui est ce qui suit "@{<nom>}" dans l'argument optionnel) ne DOIT PAS
% commencer par une accolade.
\def\CF_grabmovearg @#1#2\_nil{%
	\CF_ifinstr{#1},%
		{\CF_analysemovearg#1\_nil\CF_movebondname}%
		{\def\CF_movebondname{#1}\def\CF_movebondcoeff{0.5}}%
	\def\CF_remainoptarg{#2}%
}

\def\CF_testemptyandassign#1#2#3{%
	\CF_ifempty{#2}
		{\let#1#3}
		{\def#1{#2}}%
}

\def\CF_parseoptlist#1,#2,#3,#4,#5\_nil{%
	\CF_testemptyandassign\CF_currentstringangle{#1}\CF_defaultstringangle
	\CF_testemptyandassign\CF_currentlength     {#2}\CF_defaultlength
	\CF_testemptyandassign\CF_currentfromatom   {#3}\CF_defaultfromatom
	\CF_testemptyandassign\CF_currenttoatom     {#4}\CF_defaulttoatom
	\CF_expsecond{\CF_testemptyandassign\CF_currenttikz}{\CF_sanitizelastitem#5,\empty\_nil}\CF_defaulttikz
}%

\def\CF_analyseoptarg[#1]{%
	\CF_doifnotempty{#1}%
		{\CF_iffirsttokmatch{#1}{@}%
			{\CF_grabmovearg#1\_nil
			}
			{\let\CF_movebondname\empty
			\def\CF_remainoptarg{#1}%
			}%
		\expandafter\CF_parseoptlist\CF_remainoptarg,\empty,\empty,\empty,\empty\_nil
		}%
	\CF_analyseoptarga\relax
}

\def\CF_analyseoptarga#1\_nil#2{%
	\CF_expsecond{\def#2}{\CF_gobarg#1}%
}

\def\CF_seeksubmol#1#2{% cherche et remplace ! au début de #1. #1=code #2=macro recevant le résultat
	\def\CF_seeksubmloltemp{#1}%
	\CF_seeksubmola
	\let#2\CF_seeksubmloltemp
}

\def\CF_seeksubmola{%
	\CF_expsecond{\def\CF_seeksubmloltemp}{\romannumeral-`\.\expandafter\noexpand\CF_seeksubmloltemp}%
	\CF_expsecond\CF_iffirsttokmatch{\CF_seeksubmloltemp}!%
		{\CF_eexpsecond{\def\CF_seeksubmloltemp}{\expandafter\CF_gobarg\CF_seeksubmloltemp}% enlève le "!"
		\CF_ifx\empty\CF_seeksubmloltemp
			{\CF_error{no submol name found after "!"}}
			{}%
		\ifcat\relax\CF_threeea\noexpand\expandafter\CF_firsttonil\CF_seeksubmloltemp*\_nil
			\expandafter\CF_seeksubmolb\CF_seeksubmloltemp\_nil
		\else
			\expandafter\CF_seeksubmolc\CF_seeksubmloltemp\_nil
		\fi
		\CF_seeksubmola
		}%
		{}%
}

\def\CF_seeksubmolb#1{\CF_seeksubmold#1\relax}

\def\CF_seeksubmolc#1{\expandafter\CF_seeksubmold\csname CF__#1\endcsname\relax}% nom de la sous molécule

\def\CF_seeksubmold#1#2\_nil{%#1=macro de la sous molécule #2=reste du code commençant par \relax
	\CF_expsecond{\CF_eexpsecond{\def\CF_seeksubmloltemp}}{\expandafter#1\CF_gobarg#2}% supprime le \relax puis ajoute la macro au début et la 2-développe
}

\def\CF_insertemptygroup#1{% insère {} au début de la sc #1
	\CF_expsecond{\def#1}{\expandafter{\expandafter}#1}%
}

\def\chemfig{\CF_testopt\CF_chemfiga{}}

\def\CF_chemfiga[#1]{%
	\begingroup
		\CF_sanitizecatcode
		\CF_exptwomacroargs\CF_chemfigb\CF_begintikzpicture\CF_endtikzpicture[#1]%
}

\def\CF_chemfigb#1#2[#3]#4{%
	\endgroup
	\begingroup
		\setchemfig{#3}%
		\CF_ifinsidetikz
			{\pgfinterruptpicture
			\let\CF_atendofchemfig\endpgfinterruptpicture
			}
			{\let\CF_atendofchemfig\relax
			}%
		\expanded{% début du tikzpicture
			\unexpanded{#1}[%
				remember picture,%
				every node/.style={%
					anchor=base,%
					inner sep=0pt,%
					outer sep=0pt,%
					minimum size=0pt,%
					\unexpanded\expandafter{\CF_atomstyle}%
					},%
				baseline,%
				\unexpanded\expandafter{\CF_chemfigstyle}%
				]%
			}%
			\begingroup% \endgroup rajouté en sortie de tracé par \CF_chemfigd
				\let\CF_hooklist\empty
				\ifboolKV[chemfig]{fixed length}
						{\CF_macrofixedbondlengthtrue}
						{\CF_macrofixedbondlengthfalse}%
				\ifboolKV[chemfig]{bond join}
					{\let\CF_drawaxisbond\CF_drawaxisbondjoin}
					{\let\CF_drawaxisbond\CF_drawaxisbondnojoin}%
				\ifboolKV[chemfig]{cram rectangle}
					{\let\CF_clipcramornot\CF_gobtikzinstruction}
					{\let\CF_clipcramornot\clip}%
				\CF_incyclefalse
				\CF_cntgroup0
				\ifboolKV[chemfig]{autoreset cntcycle}
					{\global\CF_cntcycle0 }
					{}%
				\let\CF_lastaction\CF_zero% 0=début du dessin 1=tracé d'un noeud 2=tracé d'une liaison
				\let\CF_startoffset\empty
				\let\CF_endoffset\empty
				\let\CF_bondoutcontentsaved\empty
				\def\CF_cycleanglecorrection{180/\CF_cyclenum}%
				\def\CF_defaultangle{0}%
				\def\CF_defaultstringangle{:0}% angle pris par défaut si le champ est vide
				\def\CF_defaultlength{1}%
				\let\CF_defaultfromatom\empty% numero de l'atome d'où partent les liaisons par défaut
				\let\CF_defaulttoatom\empty% numéro de l'atome où arrivent les laisons par défaut
				\let\CF_defaulttikz\empty
				\let\CF_previousbondangle\empty
				\let\CF_joinbond\CF_zero
				\let\CF_previoustikz\empty
				\everyeof{\_nil}\endlinechar-1
				\CF_sanitizecatcode
				\expandafter\CF_assigntonil\expandafter\CF_molecule\scantokens{#4}%
				\CF_expsecond{\CF_substall\CF_molecule}\CFhash\CFhash
				\CF_expsecond\CF_chemfigc{\CF_molecule}%
			%\endgroup <-- rajouté par \CF_chemfigd
		#2% fin du tikzpicture
		\CF_atendofchemfig
	\endgroup
	\let\CF_flipstate\CF_zero
}

\def\CF_chemfigc#1{% #1 est le code de la molécule
	\ifnum\CF_lastaction=3
		\ifCF_incycle
			\def\CF_defaultangle{0}%
		\else
			\ifnum\CF_cntcyclebonds=0 % si c'est le début d'un cycle
				\pgfmathsetmacro\CF_defaultangle{\CF_previousangle+180+\CF_cycleanglecorrection}% on met la liaison à +180° + correction
			\else
				\pgfmathsetmacro\CF_defaultangle{\CF_previousangle-90+180/\CF_cyclenum}% sinon à la bissectrice du sommet du cycle
			\fi
		\fi
		\let\CF_defaultstringangle\empty
	\fi
	\let\CF_currentangle\CF_defaultangle
	\def\CF_molecule{#1}%
	\CF_expsecond\CF_seeksubmol\CF_molecule\CF_molecule% alias en premier ?
	\if[\CF_threeea\noexpand\expandafter\CF_firsttonil\CF_molecule\_nil
		\expandafter\CF_analyseoptarg\CF_molecule\_nil\CF_molecule
		\CF_expsecond\CF_setbondangle{\CF_currentstringangle}\CF_currentangle
		\let\CF_defaultangle\CF_currentangle
		\let\CF_previousangle\CF_currentangle
		\CF_doifnotempty\CF_currentstringangle{\let\CF_defaultangle\CF_currentangle}%
		\CF_doifnotempty\CF_currentlength     {\let\CF_defaultlength\CF_currentlength}%
		\CF_doifnotempty\CF_currentfromatom   {\let\CF_defaultfromatom\CF_currentfromatom}%
		\CF_doifnotempty\CF_currenttoatom     {\let\CF_defaulttoatom\CF_currenttoatom}%
		\CF_doifnotempty\CF_currenttikz       {\let\CF_defaulttikz\CF_currenttikz}%
		\CF_expsecond\CF_seeksubmol\CF_molecule\CF_molecule
	\fi
	\edef\CF_defaultstringangle{:\CF_defaultangle}%
	\let\CF_currentlength\CF_defaultlength
	\let\CF_currentfromatom\CF_defaultfromatom
	\let\CF_currenttoatom\CF_defaulttoatom
	\let\CF_currenttikz\CF_defaulttikz
	\ifCF_incycle% si on commence un cycle
		\let\CF_currentangle\CF_previousangle
		\pgfmathsetmacro\CF_cyclearcinitangle{\CF_currentangle+\CF_initcycleangle+180/\CF_cyclenum+90}%
		\pgfmathsetmacro\CF_centeroffset{\CF_currentlength*\CF_atomsep/(2*sin(180/\CF_cyclenum))}%
		\node[at=(\CF_bondoutnode),shift=(\CF_cyclearcinitangle:\CF_centeroffset pt),anchor=center](cyclecenter\number\CF_cntcycle){};% le centre du cycle
		\ifboolKV[chemfig]{show cntcycle}
			{\node[at=(cyclecenter\number\CF_cntcycle),anchor=center,overlay]{\tiny\number\CF_cntcycle};}
			{}%
		\ifCF_cyclearc% on doit tracer l'arc de cercle dans le cycle ?
			\pgfmathsetmacro\CF_cyclearcradius{\CF_cycleradiuscoeff*\CF_currentlength*\CF_atomsep/(2*tan(180/\CF_cyclenum))}%
			\node[at=(cyclecenter\number\CF_cntcycle),shift=(\CF_cyclearcstartangle:\CF_cyclearcradius pt)](initarc){};% le début de l'arc
			\CF_expafter{\draw[}\CF_cyclearcdirecttikz](initarc) arc (\CF_cyclearcstartangle:\CF_cyclearcendangle:\CF_cyclearcradius pt);%
		\fi
	\else
		\let\CF_currentangle\CF_defaultangle
	\fi
	\ifnum\CF_lastaction=0
		\let\CF_previousangle\CF_defaultangle
		\node(CF_node){};
		\CF_expsecond\CF_iffirsttokin{\CF_molecule}{-=(*~?<>}%
			{\CF_insertemptygroup\CF_molecule}%
			{}%
	\fi
	\CF_chemfigd
}

\def\CF_chemfigd{%
	\let\CF_nextaction\CF_chemfigd% à priori, on reboucle
	\CF_ifx\CF_molecule\empty
		{\let\CF_nextaction\endgroup
		}
		{\CF_expsecond\CF_seeknode{\CF_molecule}\CF_currentatomgroup\CF_molecule
		\CF_ifx\empty\CF_currentatomgroup% pas de noeud pour commencer ?
			{\def\CF_bondoutnode{%
				n\CF_lastgroupnumber-%
				\ifx\CF_currentfromatom\empty
					\ifdim\CF_currentangle pt<90pt
						\number\CF_cntatomgroup
					\else
						\ifdim\CF_currentangle pt>270pt
							\number\CF_cntatomgroup
						\else
							1%
						\fi
					\fi
				\else
					\CF_currentfromatom
				\fi}%
			\CF_expafter{\futurelet\CF_toksa\CF_gobtonil}{\CF_molecule\relax\_nil}%
			\CF_iffirsttokina{-=<>~}% la suite est une liaison
				{\ifnum\CF_lastaction=2 % c'est la deuxième liaison consécutive ?
					\CF_insertemptygroup\CF_molecule% insère un groupe vide
					\edef\CF_bondoutnode{\CF_bondoutnode}%
				\else
					\ifCF_incycle
						\advance\CF_cntcyclebonds1
					\fi
					\CF_expsecond\CF_analysebond{\CF_molecule}\CF_bondtype
					\edef\CF_bondoutnode{\CF_bondoutnode}%
					\let\CF_molecule\CF_remainafterbond
					\ifCF_incycle
						\ifnum\CF_cntcyclebonds=\CF_cyclenum\relax
							\CF_threeea\CF_execfirst
						\else
							\ifnum\CF_cntcyclebonds=1
								\let\CF_cyclefirsttikz\CF_currenttikz
								\CF_doifnotempty\CF_startoffset{\let\CF_cyclejoinlast\CF_zero}%
							\fi
							\CF_threeea\CF_execsecond
						\fi
					\else
						\expandafter\CF_execsecond
					\fi
						{\let\CF_nextaction\endgroup
						\CF_drawbond\CF_bondtype{\CF_bondoutnode}{\CF_hookcycle}\CF_previousatomgroup\CF_hookatomgroup
						}%
						{\node[at=(\CF_bondoutnode\ifCF_incycle\else\ifCF_macrofixedbondlength.\CF_currentangle\fi\fi),shift=(\ifcase\CF_flipstate\or180-\or-\fi\CF_currentangle:\CF_currentlength*\CF_atomsep)](CF_node){};
						\let\CF_previousangle\CF_currentangle
						\def\CF_lastaction{2}%
						}%
				\fi
				\ifcat\relax\detokenize\expandafter{\romannumeral-`\.\expandafter\noexpand\CF_molecule}\relax
				% s'il ne reste plus rien après la liaison (sans tenir compte de l'espace devant)-> insère un groupe vide
					\CF_insertemptygroup\CF_molecule
				\fi
				}%
				{\edef\CF_bondoutnode{\CF_bondoutnode}% évalue le l'atome de départ de liaison
				\CF_ifx(\CF_toksa% une parenthèse pour commencer ?
					{\ifnum\CF_lastaction=2 % il y avait une liaison juste avant ?
						\CF_insertemptygroup\CF_molecule
					\else
						\CF_expsecond\CF_grabsubmol{\CF_molecule}%
						\begingroup
							\ifCF_incycle\def\CF_lastaction{3}\fi% on était dans un cycle
							\CF_incyclefalse
							\aftergroup\CF_chemfigd
							\def\CF_nextaction{\CF_expsecond\CF_chemfigc{\CF_molinparen}}%
					\fi
					}%
					{\CF_ifx\CF_molecule\empty
						{\let\CF_nextaction\endgroup
						}
						{% ce qui reste après le noeud courant n'est pas vide, ne commence pas par "-=~", ni par une parenthèse
						\CF_ifx*\CF_toksa% un cycle ?
							{\ifnum\CF_lastaction=2
								\CF_insertemptygroup\CF_molecule% insère un groupe vide
							\else
								\ifCF_incycle
									\def\CF_lastaction{3}%
								\fi% on était dans un cycle
								\ifnum\CF_lastaction=3
									\let\CF_lastcyclenum\CF_cyclenum
								\fi
								\CF_eexpsecond\CF_iffirsttokmatch{\expandafter\CF_gobarg\CF_molecule}*%
									{\CF_eexpsecond{\def\CF_molecule}{\expandafter\CF_gobarg\CF_molecule}% enlève la 1er étoile
									\CF_eexpsecond\CF_iffirsttokmatch{\expandafter\CF_gobarg\CF_molecule}[% un crochet ensuite ?
										{\expandafter\CF_cycleparsepreamblewithoptarg\CF_molecule\_nil% \begingroup inclus
										}%
										{\def\CF_cyclearcstartangle{0}\def\CF_cyclearcendangle{360}%
										\let\CF_cyclearcdirecttikz\empty
										\expandafter\CF_cycleparsepreamble\CF_molecule\_nil% \begingroup inclus
										}%
									\CF_cyclearctrue
									}%
									{\expandafter\CF_cycleparsepreamble\CF_molecule\_nil% \begingroup inclus
									\CF_cyclearcfalse
									}%
								\CF_cntcyclebonds0
								\edef\CF_hookcycle{\CF_bondoutnode}%
								\let\CF_hookatomgroup\CF_previousatomgroup
								\CF_ifzerodim\CF_hookatomgroup
									{\def\CF_cyclejoinlast{1}}% joindre le dernier
									{\def\CF_cyclejoinlast{0}}%
								\CF_incycletrue
								\global\advance\CF_cntcycle1
								\ifnum\CF_lastaction=3
									\pgfmathsetmacro\CF_initcycleangle{360/\CF_lastcyclenum-180}% c'est un cycle dans un cycle
								\else
									\pgfmathsetmacro\CF_initcycleangle{-180/\CF_cyclenum-90+\CF_cycleanglecorrection}%
								\fi
								\aftergroup\CF_chemfigd
								\def\CF_nextaction{\CF_expsecond\CF_chemfigc{\CF_molinparen}}%
							\fi
							}%
							{\CF_error{something went wrong here: \detokenize\expandafter{\CF_molecule}^^JIf you think it's a bug, please, send a Minimal Example to the author}%
							}%
						}%
					}%
				}%
			}%
			{\CF_expthreemacroargs\CF_drawatomgroup\CF_currentangle\CF_currenttoatom\CF_currentatomgroup
			}%
		}%
	\CF_nextaction
}

\def\CF_cycleparsepreamble*#1#2\_nil{%
	\ifnum#1<3
		\CF_error{a cycle must be at least a triangle.^^JThe number following "*" must be 3 or more}%
	\fi
	\def\CF_molecule{#2}%
	\CF_expsecond\CF_grabsubmol{\CF_molecule}%
	\begingroup
	\def\CF_cyclenum{#1}%
}

\def\CF_cycleparsepreamblewithoptarg*[#1]#2#3\_nil{%
	\CF_cycleparseoptarg#1,\empty,\empty,\empty\_nil
	\CF_cycleparsepreamble*#2#3\_nil
}

\def\CF_cycleparseoptarg#1,#2,#3\_nil{%
	\CF_ifempty{#1}
		{\def\CF_cyclearcstartangle{0}}
		{\def\CF_cyclearcstartangle{#1}}%
	\CF_ifempty{#2}
		{\def\CF_cyclearcendangle{360}}
		{\def\CF_cyclearcendangle{#2}}%
	\CF_expsecond{\def\CF_cyclearcdirecttikz}{\CF_sanitizelastitem#3,\empty\_nil}%
}

\def\CF_grabsubmol#1{%
	\begingroup
		\catcode`(1 \catcode`)2
		\CF_threeea
	\endgroup
	\expandafter\CF_grabsubmola\scantokens{\relax#1}%
}

\def\CF_grabsubmola#1\_nil{%
	\expandafter\CF_assigntonil\expandafter\CF_molecule\scantokens\CF_threeea{\expandafter\CF_gobarg    \CF_gobarg#1}%
	\expandafter\CF_assigntonil\expandafter\CF_molinparen\scantokens\CF_threeea{\expandafter\CF_firsttonil\CF_gobarg#1\_nil}%
}

\def\CF_ifcarisupperletter#1{%
	\ifcat\relax\noexpand#1%
		\let\CF_next\CF_execsecond% faux si c'est une sc
	\else
		\ifnum`#1<`A
			\let\CF_next\CF_execsecond
		\else
			\ifnum`#1>`Z
				\let\CF_next\CF_execsecond
			\else
				\let\CF_next\CF_execfirst
			\fi
		\fi
	\fi
	\CF_next
}

% Créé 4 noeuds au dessus et au dessous des noeuds #1 et #2
% à une distance de #3 du noeud #1 et #4 du noeud #2
\def\CF_createnormnodes#1#2#3#4{%
	\CF_doifnotempty{#3}
		{\node[shape=coordinate,at=(#1),xshift=#3*\CF_normx,yshift=#3*\CF_normy](#11){};
		\node[shape=coordinate,at=(#1),xshift=-#3*\CF_normx,yshift=-#3*\CF_normy](#12){};
		}%
	\CF_doifnotempty{#4}
		{\node[shape=coordinate,at=(#2),xshift=#4*\CF_normx,yshift=#4*\CF_normy](#21){};
		\node[shape=coordinate,at=(#2),xshift=-#4*\CF_normx,yshift=-#4*\CF_normy](#22){};
		}%
}

\def\CF_distancebetweenpoints#1#2#3#4#5{%
	\pgfextractx\CF_dim{\pgfpointanchor{#1}{#2}}\edef\CF_dimax{\the\CF_dim}%
	\pgfextracty\CF_dim{\pgfpointanchor{#1}{#2}}\edef\CF_dimay{\the\CF_dim}%
	\pgfextractx\CF_dim{\pgfpointanchor{#3}{#4}}\edef\CF_dimbx{\the\CF_dim}%
	\pgfextracty\CF_dim{\pgfpointanchor{#3}{#4}}\edef\CF_dimby{\the\CF_dim}%
	\pgfmathsetmacro#5{veclen(\CF_dimbx-\CF_dimax,\CF_dimby-\CF_dimay)}%
}

\def\CF_computenodevect#1#2{%
	\CF_distancebetweenpoints{#1}{center}{#2}{center}\CF_vectorlen
	\pgfmathsetmacro\CF_normx{(\CF_dimay-\CF_dimby)/\CF_vectorlen}%
	\pgfmathsetmacro\CF_normy{(\CF_dimbx-\CF_dimax)/\CF_vectorlen}%
}

\def\CF_setoffset#1#2{%
	\CF_doifempty#1{%
		\CF_ifzerodim{#2}
			{\def#1{0pt}}
			{\edef#1{\CF_bondoffset}}%
	}%
}

\def\CF_drawbond#1#2#3#4#5{% #1=type de liaison #2 et #3:nom de noeuds de début et fin #4 et #5: contenu des atomes de début et fin
	\CF_setoffset\CF_startoffset#4%
	\CF_setoffset\CF_endoffset#5%
	\let\CF_currentbondstyle\CF_bondstyle
	\CF_doifnotempty\CF_currenttikz{\CF_eaddtomacro\CF_currentbondstyle{\expandafter,\CF_currenttikz}}%
	\path(#2)--(#3)coordinate[pos=0](#2@)coordinate[pos=1](#3@);%
	\CF_computenodevect{#2@}{#3@}%
	\pgfmathsetmacro\CF_startcoeff{\CF_startoffset/\CF_vectorlen}%
	\pgfmathsetmacro\CF_endcoeff{1-\CF_endoffset/\CF_vectorlen}%
	\path(#2@)--(#3@)coordinate[pos=\CF_startcoeff](#2@@)coordinate[pos=\CF_endcoeff](#3@@);%
	\CF_doifnotempty\CF_movebondname% on doit poser un noeud sur la liaison
		{\path(#2@@)--(#3@@)coordinate[overlay,pos=\CF_movebondcoeff](\CF_movebondname);
		\let\CF_movebondname\empty
		}%
	\ifcase#1\relax
		\CF_error{unknown bond type, this error should not occur^^JIf you think it's a bug, send a Minimal Example to the author}%
	\or% 1 = liaison simple
		\CF_drawaxisbond{#2}{#3}% trace la liaison simple dans l'axe
	\or% 2 = liaison double
		\ifCF_incycle
			\ifnum\CF_doublebondtype=0
				\def\CF_doublebondtype{1}%
			\fi
			\ifnum\CF_flipstate>0
				\def\CF_doublebondtype{2}%
			\fi
			\pgfmathsetmacro\CF_doublebondlengthcorrection{\CF_doublesep*tan(180/\CF_cyclenum)}%
		\fi
		\ifcase\CF_doublebondtype
			\CF_createnormnodes{#2@@}{#3@@}{\CF_doublesep/2}{\CF_doublesep/2}%
			\CF_drawbonda(#2@@1)--(#3@@1);
			\CF_drawbonda(#2@@2)--(#3@@2);
			\let\CF_joinbond\CF_zero
		\or
			\CF_createnormnodes{#2@@}{#3@@}\CF_doublesep\CF_doublesep
			\CF_drawaxisbond{#2}{#3}% trace la liaison simple dans l'axe\CF_drawbonda(#2@@)--(#3@@);
			\begingroup% ajuste éventuellement les longueurs des liaisons doubles
				\ifCF_incycle
					\ifdim\CF_startoffset=0pt
						\CF_edefaddtomacro\CF_currentbondstyle{,shorten <=\CF_doublebondlengthcorrection pt}%
					\fi
					\ifdim\CF_endoffset=0pt
						\CF_edefaddtomacro\CF_currentbondstyle{,shorten >=\CF_doublebondlengthcorrection pt}%
					\fi
				\fi
				\CF_drawbonda(#2@@1)--(#3@@1);
			\endgroup
		\or
			\CF_createnormnodes{#2@@}{#3@@}\CF_doublesep\CF_doublesep
			\CF_drawaxisbond{#2}{#3}% trace la liaison simple dans l'axe\CF_drawbonda(#2@@)--(#3@@);
			\begingroup% ajuste éventuellement les longueurs des liaisons doubles
				\ifCF_incycle
					\ifdim\CF_startoffset=0pt
						\CF_edefaddtomacro\CF_currentbondstyle{,shorten \ifnum\CF_flipstate=0 <=-\else>=\fi\CF_doublebondlengthcorrection pt}%
					\fi
					\ifdim\CF_endoffset=0pt
						\CF_edefaddtomacro\CF_currentbondstyle{,shorten \ifnum\CF_flipstate=0 >=-\else<=\fi\CF_doublebondlengthcorrection pt}%
					\fi
				\fi
				\CF_drawbonda(#2@@2)--(#3@@2);
			\endgroup
		\fi
	\or% 3 = liaison triple
		\CF_createnormnodes{#2@@}{#3@@}\CF_doublesep\CF_doublesep
		\CF_drawaxisbond{#2}{#3}% trace la liaison simple dans l'axe\CF_drawbonda(#2@@)--(#3@@);
		\CF_drawbonda(#2@@1)--(#3@@1);
		\CF_drawbonda(#2@@2)--(#3@@2);
	\or% 4 = liaison Cram pleine de #2 vers #3
		\CF_createnormnodes{#2@@}{#3@@}{\CF_crambasewidth/2}{}%
		\CF_expafter{\filldraw[}\CF_currentbondstyle,line join=bevel](#2@@1)--(#2@@2)--(#3@@)--cycle;
		\let\CF_joinbond\CF_zero
	\or% 5 = liaison Cram creuse de #3 vers #2
		\CF_createnormnodes{#3@@}{#2@@}{\CF_crambasewidth/2}{}%
		\CF_expafter{\filldraw[}\CF_currentbondstyle,line join=bevel](#3@@1)--(#3@@2)--(#2@@)--cycle;
		\let\CF_joinbond\CF_zero
	\or% 6 = liaison Cram pointillée de #2 vers #3
		\scope
			\CF_createnormnodes{#2@@}{#3@@}{\CF_crambasewidth/2}{}%
			\CF_clipcramornot(#2@@1)--(#2@@2)--(#3@@)--(#2@@1);
			\CF_expafter{\draw[}\CF_currentbondstyle,dash pattern=on \CF_cramdashlength off \CF_cramdashsep,line width=\CF_crambasewidth](#2@@)--(#3@@);
		\endscope
		\let\CF_joinbond\CF_zero
	\or% 7 = liaison Cram pointillée de #3 vers #2
		\scope
			\CF_createnormnodes{#3@@}{#2@@}{\CF_crambasewidth/2}{}%
			\CF_clipcramornot(#3@@1)--(#3@@2)--(#2@@)--(#3@@1);
			\CF_expafter{\draw[}\CF_currentbondstyle,dash pattern=on \CF_cramdashlength off \CF_cramdashsep,line width=\CF_crambasewidth](#3@@)--(#2@@);
		\endscope
		\let\CF_joinbond\CF_zero
	\or% 8 = liaison cram rectangle évidé de #2 vers #3
		\CF_createnormnodes{#2@@}{#3@@}{\CF_crambasewidth/2}{}%
		\CF_expafter{\draw[}\CF_currentbondstyle,line join=bevel](#2@@1)--(#2@@2)--(#3@@)--cycle;
		\let\CF_joinbond\CF_zero
	\or% 9 = liaison cram rectangle évidé de #3 vers #1
		\CF_createnormnodes{#3@@}{#2@@}{\CF_crambasewidth/2}{}%
		\CF_expafter{\draw[}\CF_currentbondstyle,line join=bevel](#3@@1)--(#3@@2)--(#2@@)--cycle;
		\let\CF_joinbond\CF_zero
	\else
		\CF_error{unknown bond type, this error should not occur^^JIf you think it's a bug, send a Minimal Example to the author}%
	\fi
	\let\CF_startoffset\empty
	\let\CF_endoffset\empty
	\let\CF_previoustikz\CF_currenttikz
	\let\CF_previousbondangle\CF_previousangle
	\def\CF_previousbond{#1}%
}

\def\CF_drawaxisbondnojoin#1#2{\CF_drawbonda(#1@@)--(#2@@);}

\def\CF_drawaxisbondjoin#1#2{% dessine une liaison simple dans l'axe avec raccord rétrograde
	\ifCF_incycle\ifnum\CF_cntcyclebonds=\CF_cyclenum\relax
		\let\CF_nexttikz\CF_cyclefirsttikz
	\fi\fi
	\ifnum\CF_joinbond=0
		\ifCF_incycle
			\ifnum\CF_cntcyclebonds=\CF_cyclenum\relax
				\ifnum\CF_cyclejoinlast=1
					\CF_drawbonda(#1@@)--(#2@@)--%
						([shift=(\CF_previousbondangle+2*\CF_cycleincrementangle:.5\pgflinewidth)]#2@@);
				\else
					\CF_drawbonda(#1@@)--(#2@@);
				\fi
			\else
				\CF_drawbonda(#1@@)--(#2@@);
			\fi
		\else
			\CF_drawbonda(#1@@)--(#2@@);
		\fi
		\def\CF_joinbond{1}%
	\else
		\CF_ifx\CF_previoustikz\CF_currenttikz
			{\def\CF_joinbond{1}%
			\CF_ifzerodim\CF_previousatomgroup
				{\CF_ifx\CF_previousbondangle\empty% si début molécule
					{\CF_drawbonda(#1@@)--(#2@@);
					}
					{\ifdim\CF_startoffset=0pt
						\ifCF_incycle
							\ifnum\CF_cntcyclebonds=\CF_cyclenum\relax
								\ifnum\CF_cyclejoinlast=1
									\CF_ifx\CF_cyclefirsttikz\CF_currenttikz
										{\CF_drawbonda([shift=(\CF_previousbondangle:-.5\pgflinewidth)]#1@@)--(#1@@)--(#2@@)--%
										([shift=(\CF_previousbondangle+2*\CF_cycleincrementangle:.5\pgflinewidth)]#2@@);
										}
										{\CF_drawbonda([shift=(\CF_previousbondangle:-.5\pgflinewidth)]#1@@)--(#1@@)--(#2@@);
										}%
								\else
									\CF_drawbonda([shift=(\CF_previousbondangle:-.5\pgflinewidth)]#1@@)--(#1@@)--(#2@@);
								\fi
							\else
								\CF_drawbonda([shift=(\CF_previousbondangle:-.5\pgflinewidth)]#1@@)--(#1@@)--(#2@@);
							\fi
						\else
							\CF_drawbonda([shift=(\CF_previousbondangle:-.5\pgflinewidth)]#1@@)--(#1@@)--(#2@@);
						\fi
					\else
						\CF_drawbonda(#1@@)--(#2@@);
					\fi
					}%
				}
				{\CF_drawbonda(#1@@)--(#2@@);}%
			}
			{\ifCF_incycle
				\ifnum\CF_cntcyclebonds=\CF_cyclenum\relax
					\ifnum\CF_cyclejoinlast=1
						\CF_ifx\CF_nexttikz\CF_currenttikz
							{\CF_drawbonda(#1@@)--(#2@@)--([shift=(\CF_previousbondangle+2*\CF_cycleincrementangle:.5\pgflinewidth)]#2@@);}%
							{\CF_drawbonda(#1@@)--(#2@@);}%
					\else
						\CF_drawbonda(#1@@)--(#2@@);
					\fi
				\else
					\CF_drawbonda(#1@@)--(#2@@);
				\fi
			\else
				\CF_drawbonda(#1@@)--(#2@@);
			\fi
			}%
	\fi
	\ifdim\CF_endoffset=0pt \else
		\let\CF_joinbond\CF_zero
	\fi
}

\def\CF_drawbonda{\CF_expafter{\draw[}\CF_currentbondstyle]}

\def\CF_hookdrawall{% dessine tous les crochets contenus dans la sc \CF_hookdraw
	\CF_doifnotempty\CF_hookdrawlist
		{\expandafter\CF_hookdrawfirst\CF_hookdrawlist\_nil% trace un lien de crochet à crochet
		\CF_hookdrawall
		}%
}

\def\CF_hookdrawfirst[#1,#2,#3]#4#5#6#7#8\_nil{%
	\def\CF_hookdrawlist{#8}%
	\begingroup
		\let\CF_joinbond\CF_zero
		\def\CF_currenttikz{#3}%
		\def\CF_hookstartcontent{#6}\def\CF_hookendcontent{#7}%
		\CF_ifinteger{#2}%
			{\CF_drawbond{#2}{#4}{#5}\CF_hookstartcontent\CF_hookendcontent
			}%
			{\CF_assignbondcode{#2}\CF_bondcurrentnum
			\CF_drawbond\CF_bondcurrentnum{#4}{#5}\CF_hookstartcontent\CF_hookendcontent
			}%
	\endgroup
}

\def\CF_extractatom#1-#2\_nil{#2}% transforme le bound@outnode en n° de l'atome

\def\CF_gobblemovearg @#1#2\_nil#3{%
	\expandafter\def\csname atom_\number\CF_cntatom\endcsname{#2}%
	\CF_ifinstr{#1},%
		{\CF_analysemovearg#1\_nil#3\let\CF_movebondcoeff\empty}%
		{\def#3{#1}}%
	\CF_doifempty{#2}{\let\CF_nodestrut\empty}%
}%

\def\hflipnext{\def\CF_flipstate{1}}

\def\vflipnext{\def\CF_flipstate{2}}
\let\CF_flipstate\CF_zero

\def\CF_drawatomgroup#1#2#3{% #1=angle d'arrivée de la liaison #2=numero atome sur lequel arrive la liaison #3=groupe d'atomes
	\expandafter\let\expandafter\CF_bondoutcontent% assigne le contenu de l'atome d'où part la liaison
		\csname
			\ifdefined\CF_bondoutnode
				atom_\expandafter\CF_extractatom\CF_bondoutnode\_nil
			\else
				empty%
			\fi
		\endcsname
	\global\advance\CF_cntgroup1
	\let\CF_currentatom\empty
	\global\let\CF_hookdrawlist\empty
	\CF_cntatomgroup0 % est le nombre d'atome dans le groupe que va calculer \CF_drawatomgroupa
	\CF_iffirsttokmatch{#3}?
		{\CF_drawatomgroupa{{}#3}}
		{\CF_drawatomgroupa{#3}}%
	\def\CF_currentatomgroup{#3}%
	\CF_removemovearg\CF_currentatomgroup% enlève les "@{<nom>}"
	\CF_ifinstr{#3}?%
		{\CF_removehook\CF_currentatomgroup
		\ifcat\relax\detokenize\expandafter{\romannumeral-`\.\expandafter\noexpand\CF_currentatomgroup}\relax
			\let\CF_currentatomgroup\empty
		\fi
		}%
		{}%
	\CF_doifnotempty{#2}
		{\ifnum#2<1
			\CF_warning{no atom found at position #2, pershaps you mispelled the optional argument of the bond.}%
		\else
			\ifnum#2>\CF_cntatomgroup
				\CF_error{no atom found at position #2, pershaps you mispelled the optional argument of the bond.}%
			\fi
		\fi
		}%
	\edef\CF_hookatomnumber{%
		\CF_ifempty{#2}
			{\ifdim#1pt>90pt
				\ifdim#1pt<270pt
					\number\CF_cntatomgroup
				\else
					1%
				\fi
			\else
				1%
			\fi
			}
			{#2%
			}%
	}%
	\CF_cntatom\CF_hookatomnumber
	\CF_ifzerodim\CF_currentatomgroup
		{\let\CF_nodestrut\empty
		}
		{\CF_ifx\empty\CF_bondoutcontentsaved
			{\def\CF_nodestrut{\vphantom\CF_bondoutcontent}}%
			{\def\CF_nodestrut{\vphantom\CF_bondoutcontentsaved}}%
		}%
	\edef\CF_optstring{anchor=\ifnum\CF_lastaction=0 base\else\ifCF_incycle center\else\ifCF_macrofixedbondlength 180+#1\else center\fi\fi\fi,at=(CF_node),\CF_nodestyle}% premier atome de la molécule affiché
	\loop
		\unless\ifnum\CF_cntatom>\CF_cntatomgroup
		\CF_eexpafter
			{\futurelet\CF_toksa}
			{\expandafter\expandafter\expandafter\CF_gobtonil\csname atom_\number\CF_cntatom\endcsname\_nil}%
		\CF_ifx @\CF_toksa% l'atome courant commence par un "@"
			{\CF_threeea\CF_gobblemovearg\csname atom_\number\CF_cntatom\endcsname\_nil\CF_moveatomname
			\CF_expafter{\node[}\CF_optstring,overlay](\CF_moveatomname){\phantom{\CF_nodecontent}};%
			\let\CF_moveatomname\empty
			}
			{}%
		\ifboolKV[chemfig]{debug}
			{\CF_expafter{\node[}\CF_optstring,draw=gray](n\number\CF_cntgroup-\number\CF_cntatom){\CF_nodecontent};%
			\CF_show_debug_atom
			}
			{\CF_expafter{\node[}\CF_optstring](n\number\CF_cntgroup-\number\CF_cntatom){\CF_nodecontent};%
			}%
		\let\CF_nodestrut\empty
		\advance\CF_cntatom1
		\edef\CF_optstring{anchor=base \ifnum\CF_flipstate=1 east\else west\fi,at=(n\number\CF_cntgroup-\number\numexpr\CF_cntatom-1.base \ifnum\CF_flipstate=1 west\else east\fi),\CF_nodestyle}%
	\repeat
	\CF_cntatom\CF_hookatomnumber
	\ifnum\CF_lastaction=2 % s'il faut tracer une liaison
		\gdef\CF_cycleanglecorrection{0}% alors c'est qu'un cycle ne peut pas commencer la molécule : annulation de la correction d'angle
		\CF_drawbond\CF_bondtype{\CF_bondoutnode}{n\number\CF_cntgroup-\number\CF_cntatom}\CF_previousatomgroup\CF_currentatomgroup
	\fi
	\def\CF_lastaction{1}% met la dernière action à 1 : affichage d'un noeud
	\loop
		\ifnum\CF_cntatom>1
		\advance\CF_cntatom-1
		\edef\CF_optstring{anchor=base \ifnum\CF_flipstate=1 west\else east\fi,at=(n\number\CF_cntgroup-\number\numexpr\CF_cntatom+1.base \ifnum\CF_flipstate=1 east\else west\fi),\CF_nodestyle}%
		\CF_eexpafter
			{\futurelet\CF_toksa}
			{\expandafter\expandafter\expandafter\CF_gobtonil\csname atom_\number\CF_cntatom\endcsname\_nil}%
		\CF_ifx @\CF_toksa% l'atome courant commence par un "@"
			{\CF_threeea\CF_gobblemovearg\csname atom_\number\CF_cntatom\endcsname\_nil\CF_moveatomname
			\CF_expafter{\node[}\CF_optstring,overlay](\CF_moveatomname){\phantom{\CF_nodecontent}};%
			\let\CF_moveatomname\empty
			}
			{}%
		\ifboolKV[chemfig]{debug}
			{\CF_expafter{\node[}\CF_optstring,draw=gray](n\number\CF_cntgroup-\number\CF_cntatom){\CF_nodecontent};%
			\CF_show_debug_atom
			}
			{\CF_expafter{\node[}\CF_optstring](n\number\CF_cntgroup-\number\CF_cntatom){\CF_nodecontent};%
			}%
	\repeat
	\ifboolKV[chemfig]{debug}
		\CF_show_debug_atomgroup
		{}%
	\CF_hookdrawall
	\edef\CF_lastgroupnumber{\number\CF_cntgroup}%
	\let\CF_previousatomgroup\CF_currentatomgroup
}

\def\CF_show_debug_atom{%
	\node[at=(n\number\CF_cntgroup-\number\CF_cntatom.south),anchor=north,outer sep=1pt,overlay]{$\scriptscriptstyle\color{gray}\number\CF_cntatom$};%
}

\def\CF_show_debug_atomgroup{%
	\draw[red,overlay] ([xshift=-.5pt,yshift=.5pt]n\number\CF_cntgroup-1.north west) rectangle ([xshift=.5pt,yshift=-.5pt]n\number\CF_cntgroup-\number\CF_cntatomgroup.south east);%
	\path (n\number\CF_cntgroup-1.north west) -- (n\number\CF_cntgroup-\number\CF_cntatomgroup.north east)
		node [midway,yshift=1pt,overlay] {$\scriptscriptstyle\color{red}\number\CF_cntgroup$};
}

\def\CF_keepmovearg @#1#2\_nil{\def\CF_currentatom{@{#1}}}

\def\CF_drawatomgroupa#1{% transforme #1 en un groupe d'atomes
	\CF_ifempty{#1}
		{\expandafter\let\csname atom_\number\CF_cntatomgroup\endcsname\CF_currentatom
		}
		{\advance\CF_cntatomgroup1
		\futurelet\CF_toksa\CF_gobtonil#1\_nil
		\CF_ifx @\CF_toksa
			{\CF_keepmovearg#1\_nil
			\CF_removemovearga#1\_nil\CF_aftermovearg
			\CF_expsecond\CF_drawatomgroupb{\CF_aftermovearg}%
			}%
			{\let\CF_currentatom\empty
			\CF_drawatomgroupb{#1}%
			}%
		}%
}

\def\CF_drawatomgroupb#1{%
	\CF_ifempty{#1}
		{\expandafter\let\csname atom_\number\CF_cntatomgroup\endcsname\CF_currentatom
		}
		{\futurelet\CF_toksa\CF_gobtonil#1\_nil
		\CF_ifx\bgroup\CF_toksa
			{\CF_eaddtomacro\CF_currentatom{\expandafter{\CF_firsttonil#1\_nil}}%
			\CF_expsecond\CF_drawatomgroupba{\CF_gobarg#1}%
			}%
			{\CF_ifx\CF_sptoken\CF_toksa
				{\CF_addtomacro\CF_currentatom{ }%
				\CF_expsecond\CF_drawatomgroupba{\CF_afterspace#1\_nil}%
				}%
				{\CF_eaddtomacro\CF_currentatom{\CF_firsttonil#1\_nil}%
				\CF_expsecond\CF_drawatomgroupba{\CF_gobarg#1}%
				}%
			}%
		}%
}

% enlève tous les "@{nom}" de la sc #1
\def\CF_removemovearg#1{%
	\CF_expsecond\CF_ifinstr{#1}@%
		{\expandafter\CF_removemovearga#1\_nil#1%
		\CF_removemovearg#1%
		}%
		{}%
}

% enlève le premier "@{<nom>}" de l'argument et l'assigne à #2
\def\CF_removemovearga#1\_nil#2{%
	\def\CF_removemoveargb##1@{%
		\CF_expsecond{\def#2}{\CF_gobarg##1}% mange le \relax
		\CF_removemoveargc\relax
	}%
	\def\CF_removemoveargc##1\_nil{\CF_eaddtomacro#2{\CF_gobtwoargs##1}}% mange le \relax et le <nom>
	\CF_removemoveargb\relax#1\_nil
}

\def\CF_drawatomgroupba#1{% transforme #1 en un groupe d'atomes
	\CF_ifempty{#1}
		{\expandafter\let\csname atom_\number\CF_cntatomgroup\endcsname\CF_currentatom
		\let\CF_currentatom\empty
		}
		{\futurelet\CF_toksa\CF_gobtonil#1\_nil
		\CF_ifx @\CF_toksa
			{\expandafter\let\csname atom_\number\CF_cntatomgroup\endcsname\CF_currentatom
			\let\CF_currentatom\empty
			\CF_drawatomgroupa{#1}%
			}%
			{\CF_ifx|\CF_toksa
				{\expandafter\let\csname atom_\number\CF_cntatomgroup\endcsname\CF_currentatom
				\let\CF_currentatom\empty
				\CF_expsecond\CF_drawatomgroupa{\CF_gobarg#1}%
				}%
				{\CF_ifx\CF_sptoken\CF_toksa
					{\CF_addtomacro\CF_currentatom{ }%
					\CF_expsecond\CF_drawatomgroupba{\CF_afterspace#1\_nil}%
					}%
					{\CF_ifx\bgroup\CF_toksa
						{\CF_eaddtomacro\CF_currentatom{\expandafter{\CF_firsttonil#1\_nil}}%
						\CF_expsecond\CF_drawatomgroupba{\CF_gobarg#1}%
						}%
						{\CF_expsecond\CF_ifcarisupperletter{\CF_firsttonil#1\_nil}%
							{\expandafter\let\csname atom_\number\CF_cntatomgroup\endcsname\CF_currentatom
							\let\CF_currentatom\empty
							\CF_drawatomgroupa{#1}%
							}%
							{\CF_ifx?\CF_toksa
								{\CF_expsecond\CF_iffirsttokmatch{\CF_gobarg#1}[% un crochet après le "?"
									{\expandafter\CF_graboptarg\CF_gobarg#1\_nil\CF_afterhook}%
									{\CF_expafter{\CF_graboptarg[]}{\CF_gobarg#1}\_nil\CF_afterhook}%
								\CF_exptwomacroargs\CF_ifinstr{\CF_hooklist}{\expandafter(\CF_hookcurrentname)}% crochet déjà défini ?
									{\CF_expsecond\CF_hookparselist{\CF_hookcurrentname}% chercher les caractéristiques du crochet sauvegardé
									\CF_edefaddtomacro\CF_hookdrawlist{%
										[\CF_hookcurrentname,\CF_hookcurrentlink,\CF_hookcurrenttikz]{\CF_hooksavedcoord}{n\number\CF_cntgroup-\number\CF_cntatomgroup}}%
									\CF_eaddtomacro\CF_hookdrawlist{\expandafter{\CF_hooksavedcontent}}%
									\CF_eaddtomacro\CF_hookdrawlist{\expandafter{\CF_currentatom}}% ajoute les 4 arguments à la liste des crochets à tracer
									\global\let\CF_hookdrawlist\CF_hookdrawlist
									}%
									{\CF_edefaddtomacro\CF_hooklist{(\CF_hookcurrentname)|n\number\CF_cntgroup-\number\CF_cntatomgroup|}%
									\CF_eaddtomacro\CF_hooklist{\CF_currentatom|}%
									\global\let\CF_hooklist\CF_hooklist
									}%
								\CF_expsecond\CF_drawatomgroupba{\CF_afterhook}%
								}%
								{\CF_eaddtomacro\CF_currentatom{\CF_firsttonil#1\_nil}%
								\CF_expsecond\CF_drawatomgroupba{\CF_gobarg#1}%
								}%
							}%
						}%
					}%
				}%
			}%
		}%
}

\def\CF_kookdefaultname{a}
\def\CF_hookdefaultlink{-}
\def\CF_hookdefaulttikz{}

\def\CF_hookparseoptarg#1,#2,#3\_nil{%
	\CF_testemptyandassign\CF_hookcurrentname{#1}\CF_kookdefaultname
	\CF_testemptyandassign\CF_hookcurrentlink{#2}\CF_hookdefaultlink
	\CF_testemptyandassign\CF_hookcurrenttikz{#3}\CF_hookdefaulttikz
}

\def\CF_graboptarg[#1]#2\_nil#3{%
	\CF_hookparseoptarg#1,,\_nil
	\def#3{#2}%
}

\def\CF_hookparselist#1{% #1 est le nom du noeud à retrouver
	\def\CF_hookparselista##1(#1)|##2|##3|##4\_nil{\def\CF_hooksavedcoord{##2}\def\CF_hooksavedcontent{##3}}%
	\expandafter\CF_hookparselista\CF_hooklist\_nil
}

\def\CF_removehook#1{%
	\CF_expsecond\CF_ifinstr{#1}?%
		{\CF_expafter{\CF_removehooka\relax}#1\_nil#1%
		\CF_removehook#1%
		}
		{}%
}

\def\CF_removehooka#1?#2\_nil#3{%
	\CF_iffirsttokmatch{#2}[%
		{\CF_removehookb#1?#2\_nil#3}
		{\CF_expsecond{\def#3}{\CF_gobarg#1#2}}%
}

\def\CF_removehookb#1?[#2]#3\_nil#4{\CF_expsecond{\def#4}{\CF_gobarg#1#3}}

\defKV[charge]{%
	.radius      = \CF_defifempty\CF_dotradius {#1}{0.15ex},
	:sep         = \CF_defifempty\CF_dotsep    {#1}{0.3em},
	.style       = \CF_defifempty\CF_dotstyle  {#1}{fill=black},
	"length      = \CF_defifempty\CF_rectlength{#1}{1.5ex},
	"width       = \CF_defifempty\CF_rectwidth {#1}{.3ex}
}
\def\setcharge#{\setKV[charge]}
\def\resetcharge{\restoreKV[charge]}
\setKVdefault[charge]{%
	debug        = false,% trace les contours des noeuds
	macro atom   = \printatom,%macro qui prendra comme argument l'atome recevant la charge
	circle       = false,% false => noeud atome = rectangle
	macro charge = ,% macro attendue (\printatom ou  \ensuremath, par exemple) qui prendra comme argument la charge
	extra sep    = 1.5pt,% séparation additionnelle entre le noeud (cercle ou rectangle) et la position des charges
	overlay      = true,% charges en "surimpression"
	shortcuts    = true,% raccourcis \. \: \| et \" actifs pour Lewis
	lewisautorot = true,% rotation auto charge Lewis
	.radius      = 0.15ex,% rayon du point
	:sep         = 0.3em,% séparation des deux points
	.style       = {fill=black},% style des points
	"length      = 1.5ex,% longueur rectangle
	"width       = .3ex,% largeur rectangle
	"style       = {black,line width=0.4pt},% style rectangle
	|style       = {black,line width=0.4pt},% style ligne
}%
\def\chargedot{\CF_testopt\chargedot_a{}}
\def\chargedot_a[#1]{%
	\begingroup
		\setKV[charge]{#1}%
		\CF_expafter{\tikz\draw[}{\CF_dotstyle}](0,0)circle(\CF_dotradius);%
	\endgroup
}
\def\chargeddot{\CF_testopt\chargeddot_a{}}
\def\chargeddot_a[#1]{%
	\begingroup
		\setKV[charge]{#1}%
		\ifboolKV[charge]{lewisautorot}
			{\pgfmathsetmacro\CF_lewisrot{90+\chargeangle}}
			{\def\CF_lewisrot{0}}%
		\pgfmathsetmacro\CF_halfsep{\CF_dotsep/2}%
		\tikzpicture[anchor=center,rotate=\CF_lewisrot]%
			\CF_expafter{\draw[}{\CF_dotstyle}]%
				(-\CF_halfsep pt,0)circle(\CF_dotradius)%
				(\CF_halfsep pt,0)circle(\CF_dotradius);%
		\endtikzpicture
	\endgroup
}
\def\chargerect{\CF_testopt\chargerect_a{}}
\def\chargerect_a[#1]{%
	\begingroup
		\setKV[charge]{#1}%
		\ifboolKV[charge]{lewisautorot}
			{\pgfmathsetmacro\CF_lewisrot{90+\chargeangle}}
			{\def\CF_lewisrot{0}}%
		\pgfmathsetmacro\CF_halfwidth{\CF_rectwidth/2}%
		\pgfmathsetmacro\CF_halflength{\CF_rectlength/2}%
		\tikzpicture[anchor=center,rotate=\CF_lewisrot]%
			\CF_eexpafter{\draw[}{\useKV[charge]{"style}}](-\CF_halflength pt,-\CF_halfwidth pt)rectangle(\CF_halflength pt,\CF_halfwidth pt);% bugfix 1.51
		\endtikzpicture
	\endgroup
}
\def\chargeline{\CF_testopt\chargeline_a{}}
\def\chargeline_a[#1]{%
	\begingroup
		\setKV[charge]{#1}%
		\ifboolKV[charge]{lewisautorot}
			{\pgfmathsetmacro\CF_lewisrot{90+\chargeangle}}
			{\def\CF_lewisrot{0}}%
		\pgfmathsetmacro\CF_halflength{\CF_rectlength/2}%
		\tikzpicture[anchor=center,rotate=\CF_lewisrot]%
			\CF_eexpafter{\draw[}{\useKV[charge]{|style}}](-\CF_halflength pt,0)--(\CF_halflength pt,0);% bugfix 1.51
		\endtikzpicture
	\endgroup
}
\def\CF_enableshortcuts{%
	\let\CF_saveddot \.\let\.\chargedot
	\let\CF_savedddot\:\let\:\chargeddot
	\let\CF_savedrect\"\let\"\chargerect
	\let\CF_savedline\|\let\|\chargeline
	\let\enableshortcuts\relax
	\let\disableshortcuts\CF_disableshortcuts
}
\def\CF_disableshortcuts{%
	\let\.\CF_saveddot
	\let\:\CF_savedddot
	\let\"\CF_savedrect
	\let\|\CF_savedline
	\let\enableshortcuts\CF_enableshortcuts
	\let\disableshortcuts\relax
}
\def\charge{%
	\begingroup
		\catcode`\: 12
		\charge_a{true}%
}
\def\Charge{%
	\begingroup
		\catcode`\: 12
		\charge_a{false}%
}
\def\charge_a#1#2{% #1=TF #2=liste emplacements
	\CF_testopt{\charge_b{#1}}{}#2\_nil
}
\def\charge_b#1[#2]#3\_nil{%
	\charge_c{#1}[#2]{#3}%
}
\def\charge_c#1[#2]#3#4{% #1=TF pour overlay, #2= réglages, #3=liste d'emplacements, #4=atome
		\setcharge{overlay=#1,#2}%
		\setbox\CF_chargebox\hbox{\useKV[charge]{macro atom}{#4}}%
		\CF_ifinsidetikz
			{\pgfinterruptpicture
			\let\CF_atendofcharge\endpgfinterruptpicture
			}
			{\let\CF_atendofcharge\relax
			}%
		\expanded{\noexpand
		\tikzpicture[every node/.style={%
					\ifboolKV[charge]{debug}{draw=red,}{}%
					anchor=base,%
					inner sep=0pt,%
					outer sep=0pt,%
					minimum size=0pt},%
				baseline]}%
			\expanded{\noexpand
			\node[%
				\ifboolKV[charge]{circle}{circle,}{}%
				\ifboolKV[charge]{debug}{draw=green,}{}%
				anchor=base%
				]}%
				(atombox)at(0,0)%
				{\copy\CF_chargebox};% noeud contenant l'atome
			\expanded{\noexpand
			\node[%
				\ifboolKV[charge]{circle}{circle,}{}%
				\ifboolKV[charge]{debug}{draw=blue,}{}%
				anchor=base,%
				inner sep=\useKV[charge]{extra sep},%
				overlay%
				]}%
				(atom)at(0,0){%
						\vrule width0pt height\ht\CF_chargebox  depth\dp\CF_chargebox
						\vrule width\wd\CF_chargebox height\CF_zero depth\CF_zero};% noeud pour placer les charges
			\let\enableshortcuts\relax
			\let\disableshortcuts\relax
			\ifboolKV[charge]{shortcuts}\CF_enableshortcuts{}% l'atome n'est _PAS_ concerné par les racourcis
			\charge_d#3,\CF_quark=%
		\endtikzpicture
		\CF_atendofcharge
	\endgroup
}
\def\charge_d#1={%
	\CF_ifx\CF_quark{#1}%
		{}
		{\CF_striplastsp{#1}\charge_e=}% bugfix 1.54
}
\def\charge_e#1={%
	\CF_ifinstr{#1}[
		{\charge_f#1=}
		{\charge_f#1[]=}%
}
\def\charge_f#1[#2]={%
	\CF_ifinstr{#1}:
		{\charge_g#1[#2]=}
		{\charge_g#1:0pt[#2]=}%
}
\def\charge_g#1:#2[#3]=#4,{% #1=angle, #2=offset, #3=code tikz charge, #4=charge
	\CF_stripsp{#1}\CF_ifinteger
		{\pgfmathsetmacro\chargeangle{mod(#1,360)}%
		}
		{\pgfmathanglebetweenpoints{\pgfpointanchor{atom}{center}}{\CF_stripsp{#1}{\pgfpointanchor{atom}}}%
		\let\chargeangle\pgfmathresult% incorrect si (atom.center==atom.#1) && (extra sep==0) TODO: mettre un warning ?
		}%
	\edef\CF_offset{\the\dimexpr#2+0pt}%
	\CF_stripsp{#1}{\CF_distancebetweenpoints{atom}{center}{atom}}\CF_chargedistance
	\CF_eexpafter{\node[anchor=center,}{\ifboolKV[charge]{overlay}{overlay,}{}}#3]%
		at([shift=(\chargeangle:\CF_chargedistance pt+\CF_offset)]atom.center){\useKV[charge]{macro charge}{#4}};%
	\charge_d
}

\def\Chembelow{\begingroup\let\CF_temp\CF_gobarg\CF_chembelowa}

\def\chembelow{\begingroup\let\CF_temp\CF_id\CF_chembelowa}

\def\CF_chembelowa{\CF_testopt\CF_chembelowb\CF_stacksep}

\def\CF_chembelowb[#1]#2#3{%
	\setbox\CF_box\hbox{\printatom{#2}}%
	\expandafter\vtop\CF_temp{to\ht\CF_box}{%
		\offinterlineskip
		\hbox{\printatom{#2}}%
		\kern#1\relax
		\hbox to\wd\CF_box{\hss\printatom{#3}\hss}%
		\CF_temp\vss
		}%
	\endgroup
}

\def\Chemabove{\begingroup\let\CF_temp\CF_gobarg\CF_chemabovea}

\def\chemabove{\begingroup\let\CF_temp\CF_id\CF_chemabovea}

\def\CF_chemabovea{\CF_testopt\CF_chemaboveb\CF_stacksep}

\def\CF_chemaboveb[#1]#2#3{%
	\setbox\CF_box\hbox{\printatom{#2}}%
	\expandafter\vbox\CF_temp{to\ht\CF_box}{%
		\offinterlineskip
		\CF_temp\vss
		\hbox to\wd\CF_box{\hss\printatom{#3}\hss}%
		\kern#1\relax
		\hbox{\printatom{#2}}%
	}%
	\endgroup
}

\def\chemmove{\CF_testopt\CF_chemmove{}}

\def\CF_chemmove[#1]#2{%
	\CF_doifnotempty{#2}%
		{\expandafter\tikzpicture\expanded{[overlay,remember picture,-CF\CF_ifempty{#1}{}{,\unexpanded{#1}}]}%
			#2%
		\endtikzpicture
		}%
}

\def\chemnameinit#1{%
	\setbox\CF_boxstuff\hbox{#1}%
	\ifboolKV[chemfig]{gchemname}\xdef\edef\CF_dpmax{\the\dp\CF_boxstuff}%
}
\chemnameinit{}

\def\CF_parsemolname#1\\#2\_nil{%
	\hbox to\CF_wdstuffbox{\hss#1\hss}%
	\CF_doifnotempty{#2}{\CF_parsemolname#2\_nil}%
}

\def\chemname{%
	\CF_ifstar
		{\CF_adjustnamedpfalse\CF_chemnamea}
		{\CF_adjustnamedptrue \CF_chemnamea}%
}

\def\CF_chemnamea{\CF_testopt\CF_chemnameb{1.5ex}}

\def\CF_chemnameb[#1]#2#3{%
	\setbox\CF_boxstuff\hbox{#2}%
	\edef\CF_wdstuffbox{\the\wd\CF_boxstuff}\edef\CF_dpstuffbox{\the\dp\CF_boxstuff}%
	\leavevmode
	\ifdim\CF_dpmax<\CF_dpstuffbox
		\ifboolKV[chemfig]{gchemname}\global{}\let\CF_dpmax\CF_dpstuffbox
	\fi
	\vtop{%
		\box\CF_boxstuff
		\nointerlineskip
		\kern\dimexpr#1\ifCF_adjustnamedp+\CF_dpmax-\CF_dpstuffbox\fi\relax
		\CF_parsemolname#3\\\_nil
	}%
}

%%%%%%%%%%%%%%%%%%%%%%%%%%%%%%%%%%%%%%%%%%%%%%%%%%%%%%%%%%%%%%%%%%%%%%
%%%           S C H É M A S      R É A C T I O N N E L S           %%%
%%%%%%%%%%%%%%%%%%%%%%%%%%%%%%%%%%%%%%%%%%%%%%%%%%%%%%%%%%%%%%%%%%%%%%
\let\CF_schemenest\CF_zero

\def\CF_subscheme{\CF_testopt\CF_subschemea{}}
\def\CF_subschemea[#1]{\CF_testopt{\CF_subschemeb[#1]}{text}}
\def\CF_subschemeb[#1][#2]#3{\schemestart[#1][#2]#3\schemestop}

\def\chemleft#1#2\chemright#3{%
	\leavevmode
	\begingroup
		\setbox0\hbox{$\vcenter{\hbox{}}$}\edef\CF_delimmathht{\the\ht0}%
		\setbox0\hbox{#2}\edef\CF_delimdim{\the\dimexpr(\ht0+\dp0)/2}%
		\edef\CF_delimshift{\the\dimexpr(\ht0-\dp0)/2-\CF_delimmathht}%
		\raise\CF_delimshift\hbox{$\left#1\vrule height\CF_delimdim depth\CF_delimdim width0pt\right.$}\box0
		\raise\CF_delimshift\hbox{$\left.\vrule height\CF_delimdim depth\CF_delimdim width0pt\right#3$}%
	\endgroup
}

\def\chemright#1{%
	\CF_warning{"\string\chemright\string#1" ignored! No \string\chemleft\space previously found.}%
}

\def\chemup#1#2\chemdown#3{%
	\begingroup
		\setbox0\hbox{\printatom{#2}}\edef\CF_delimdim{\the\dimexpr\wd0/2}%
		\tikzpicture[every node/.style={inner sep=0pt,outer sep=0pt,minimum size=0pt},baseline]%
			\node[anchor=base west](chem@stuff){\box0};%
			\node[at=(chem@stuff.north),anchor=east,rotate=-90]{$\left#1\vrule height\CF_delimdim depth\CF_delimdim width0pt\right.$};%
			\node[at=(chem@stuff.south),anchor=west,rotate=-90]{$\left.\vrule height\CF_delimdim depth\CF_delimdim width0pt\right#3$};%
		\endtikzpicture
	\endgroup
}

\def\chemdown#1{%
	\CF_warning{"\string\chemdown\string#1" ignored! No \string\chemup\space previously found.}%
}

\def\CF_setstyle#1,#2,#3\_nil#4#5#6{%
	\def#4{#1}\let#5\empty\let#6\empty
	\CF_iffirsttokmatch\CF_quark{#2\relax}
		{}%
		{\def#5{#2}%
		\CF_iffirsttokmatch\CF_quark{#3\relax}
			{}%
			{\CF_setstylea#3\_nil#6}%
		}%
}
\def\CF_setstylea#1,\CF_quark#2\_nil#3{\def#3{#1}}

\def\CF_and{\futurelet\CF_toksa\CF_anda}

\def\CF_anda{%
	\CF_ifx\CF_toksa\bgroup
		{\CF_andb}
		{\CF_andb{}}%
}

\def\CF_andb#1{%
	\CF_setstyle#1,\CF_quark,\CF_quark\_nil\CF_signspaceante_\CF_signspacepost_\CF_signvshift_
	\CF_doifnotempty\CF_signspaceante_{\let\CF_signspaceante\CF_signspaceante_}%
	\CF_doifnotempty\CF_signspacepost_{\let\CF_signspacepost\CF_signspacepost_}%
	\CF_doifnotempty\CF_signvshift_{\let\CF_signvshift\CF_signvshift_}%
	\raise\CF_signvshift\hbox{\kern\CF_signspaceante$+$\kern\CF_signspacepost}%
}

\def\schemestart{%
	\begingroup
	\ifnum\CF_schemenest=0 \expandafter\CF_id\else\expandafter\CF_gobarg\fi{\useKV[chemfig]{schemestart code}}%
	\xdef\CF_schemenest{\number\numexpr\CF_schemenest+1}%
	\CF_testopt\CF_schemestarta{}%
}

\def\CF_schemestarta[#1]{%
	\CF_setstyle#1,\CF_quark,\CF_quark\_nil\CF_arrowangle_\CF_arrowlength_\CF_arrowstyle_
	\CF_doifnotempty\CF_arrowangle_{\let\CF_arrowangle\CF_arrowangle_}%
	\CF_doifnotempty\CF_arrowlength_{\let\CF_arrowlength\CF_arrowlength_}%
	\CF_eexpsecond{\def\CF_arrowtip}{\expandafter\CF_gobarg\CF_arrowhead}%
	\CF_expsecond{\CF_preaddtomacro\CF_defaultarrowstyle}{\CF_arrowhead,}%
	\let\CF_arrowstyle\CF_defaultarrowstyle
	\CF_doifnotempty\CF_arrowstyle_{\CF_eaddtomacro\CF_arrowstyle{\expandafter,\CF_arrowstyle_}}%
	\pgfmathsetmacro\CF_arrowdoublesep{\CF_arrowdoublesep/2}%
	\pgfmathsetmacro\CF_arrowdoubleposstart{(1-\CF_arrowdoubleposstart)/2}%
	\pgfmathsetmacro\CF_arrowdoubleposend{1-\CF_arrowdoubleposstart}%
	\ifboolKV[chemfig]{scheme debug}
		{\tikzpicture[every node/.style={draw,anchor=base,inner sep=0pt,outer sep=0pt,minimum size=1.5pt},baseline,remember picture]}
		{\tikzpicture[every node/.style={anchor=base,inner sep=0pt,outer sep=0pt,minimum size=0pt},baseline,remember picture]}%
	\let\merge\CF_merge
	\expandafter\let\csname+\endcsname\CF_and
	\let\arrow\CF_arrow
	\let\schemestop\CF_schemestop
	\let\subscheme\CF_subscheme
	\CF_testopt{\CF_schemestartb}{text}%
}

\def\CF_schemestartb[#1]{%
	\ifnum\CF_schemenest=1 % la commande n'est pas imbriquée ?
		\CF_cntcompound0
	\fi
	\edef\CF_currentnodename{c\number\CF_cntcompound}%
	\let\CF_nextnodename\empty
	\let\CF_nextnodestyle\empty
	\let\CF_directarrowlist\empty
	\ifboolKV[chemfig]{scheme debug}
		{\node[fill,green](\CF_currentnodename){};}
		{\node(\CF_currentnodename){};}%
	\def\CF_nextnodeanchor{#1}%
	\CF_doifempty\CF_nextnodeanchor{\def\CF_nextnodeanchor{text}}%
	\let\CF_compound\empty
	\CF_schemestartc
}

\def\CF_schemestartc{%
	\futurelet\CF_toksa\CF_schemestarte
}

\expandafter\def\expandafter\CF_schemestartd\space{\futurelet\CF_toksa\CF_schemestarte}

\def\CF_schemestarte{% ... et l'examine :
	\CF_iffirsttokina{\arrow\schemestop\merge}%
		{}
		{\CF_ifx\CF_toksa\bgroup
			{\ifCF_compound_is_chemfig% bugfix 1.6
				\edef\CF_restore_hashcatcode{\catcode\number`\#=\number\catcode`\# \relax}%
				\catcode`\#12 % TODO ou carrément mettre # à 12 dans tout l'environnement ?
			\fi
			\CF_addnextarg
			}
			{\CF_ifx\CF_toksa\CF_sptoken
				{\CF_addtomacro\CF_compound{ }%
				\CF_schemestartd
				}
				{\CF_ifx\CF_toksa\chemfig
					\CF_compound_is_chemfigtrue% mettre le flag à vrai
					{}%
				\afterassignment\CF_schemestartc
				\CF_addtomacro\CF_compound
				}%
			}%
		}%
}

\def\CF_addnextarg#1{%
	\CF_addtomacro\CF_compound{{#1}}%
	\ifCF_compound_is_chemfig% bugfix 1.6
		\CF_restore_hashcatcode
		\CF_compound_is_chemfigfalse% mettre le flag à faux
	\fi
	\CF_schemestartc
}

\def\CF_displaycompound#1#2{% #1 = nom et #2 = style
	\CF_doifnotempty\CF_compound
		{\global\advance\CF_cntcompound1
		\CF_ifx\CF_defaultcompoundstyle\empty
			{\let\CF_currentnodestyle\empty}
			{\CF_expsecond{\def\CF_currentnodestyle}{\CF_defaultcompoundstyle,}}%
		\CF_addtomacro\CF_currentnodestyle{anchor=\CF_nextnodeanchor,at=(\CF_currentnodename)}%
		\CF_ifempty{#2}%
			{\CF_doifnotempty\CF_nextnodestyle
				{\CF_eaddtomacro\CF_currentnodestyle{\expandafter,\CF_nextnodestyle}}%
			}
			{\CF_doifnotempty\CF_nextnodestyle
				{\CF_warning{two styles for the same node, first style "\CF_nextnodestyle" ignored}%
				}%
			\CF_addtomacro\CF_currentnodestyle{,#2}%
			}%
		\CF_ifempty{#1}
			{\edef\CF_temp{%
				\CF_ifempty\CF_nextnodename
					{c\number\CF_cntcompound}
					{\CF_nextnodename}%
				}%
			}
			{\CF_doifnotempty\CF_nextnodename
				{\CF_warning{two names for the same node, first name "\CF_nextnodename" ignored}%
				}%
			\edef\CF_temp{#1}%
			}%
		\CF_expafter{\node[}\CF_currentnodestyle](\CF_temp){\CF_compound};%
		\ifboolKV[chemfig]{scheme debug}%
			{\node[draw=none,anchor=270,at=(\CF_temp.90),fill=green!60,overlay,opacity=0.5]{\scriptsize\bfseries\CF_temp};%
			}
			{}%
		\let\CF_currentnodename\CF_temp
		}%
}

\def\CF_schemestop{%
	\CF_displaycompound{}{}%
	\CF_directarrowlist
	\endtikzpicture
	\xdef\CF_schemenest{\number\numexpr\CF_schemenest-1}%
	\ifnum\CF_schemenest=0 \expandafter\CF_id\else\expandafter\CF_gobarg\fi{\useKV[chemfig]{schemestop code}}%
	\endgroup
}

\def\CF_analysearrowarg#1{\CF_analysearrowarga#1[]\_nil}

\def\CF_analysearrowarga#1[#2]#3\_nil{%
	\CF_ifinstr{#1}.
		{\CF_addtomacro\CF_temp{#1[#2]}}
		{\CF_addtomacro\CF_temp{#1.[#2]}}%
}

\def\CF_arrow{%
	\CF_ifnextchar(%
		{\CF_arrowa
		}
		{\CF_ifnextchar\bgroup
			{\CF_arrowb(.[]--.[])}
			{\CF_arrowb(.[]--.[]){}}%
		}%
}

\def\CF_arrowa(#1--#2){%
	\def\CF_temp{(}%
	\CF_analysearrowarg{#1}%
	\CF_addtomacro\CF_temp{--}%
	\CF_analysearrowarg{#2}%
	\CF_addtomacro\CF_temp)%
	\CF_ifnextchar\bgroup
		{\expandafter\CF_arrowb\CF_temp}
		{\expandafter\CF_arrowb\CF_temp{}}%
}

\def\CF_arrowb(#1.#2[#3]--#4.#5[#6])#7{%
	\def\CF_currentarrowtype{#7}% nom de la flèche
	\CF_doifempty\CF_currentarrowtype{\def\CF_currentarrowtype{->}}%
	\CF_testopt{\CF_arrowc(#1.#2[#3]--#4.#5[#6])}{}%
}

\def\CF_arrowc(#1.#2[#3]--#4.#5[#6])[#7]{%
	\def\CF_temp{\CF_arrowe(#1.#2[#3]--#4.#5[#6])}%
	\CF_arrowd#7,\empty,\empty\_nil
}

\def\CF_arrowd#1,#2,#3\_nil{%
	\CF_addtomacro\CF_temp{{#1}}%
	\CF_eaddtomacro\CF_temp{\expandafter{#2}}%
	\expandafter\CF_eaddtomacro\expandafter\CF_temp\expandafter{\expandafter\expandafter\expandafter{\expandafter\CF_sanitizelastitem#3,\empty\_nil}}%
	\CF_temp
}

% #1, #4 : nom des nodes    #2, #5 : ancres des nodes    #3, #6 : styles des nodes
% #7 : angle flèche    #8 : longueur flèche    #9 : style tikz de la flèche
\def\CF_arrowe(#1.#2[#3]--#4.#5[#6])#7#8#9{%
	\let\CF_arrowcurrentstyle\CF_arrowstyle
	\if @\expandafter\CF_firsttonil\detokenize{#1.}\_nil% si #1 commence par @
		\if @\expandafter\CF_firsttonil\detokenize{#4.}\_nil
			\CF_eaddtomacro\CF_directarrowlist{\expandafter\CF_directarrow\expandafter{\CF_currentarrowtype}{#1}{#2}{#4}{#5}{#9}}%
			\let\CF_nextaction\CF_schemestartc
		\else
			\CF_doifnotempty\CF_arrowcurrentstyle{\CF_addtomacro\CF_arrowcurrentstyle,}%
			\CF_doifnotempty{#9}{\CF_addtomacro\CF_arrowcurrentstyle{#9,}}%
			\CF_displaycompound{}{#3}%
			\def\CF_nextnodename{#4}%
			\CF_expsecond{\def\CF_currentnodename}{\CF_gobarg#1}%
			\let\CF_arrowstartname\CF_currentnodename
			\let\CF_arrowendname\CF_nextnodename
			\CF_arrowf{#7}{#8}{#2}{#5}%
			\def\CF_nextnodestyle{#6}%
		\fi
	\else
		\CF_doifnotempty\CF_arrowcurrentstyle{\CF_addtomacro\CF_arrowcurrentstyle,}%
		\CF_doifnotempty{#9}{\CF_addtomacro\CF_arrowcurrentstyle{#9,}}%
		\if @\expandafter\CF_firsttonil\detokenize{#2.}\_nil
			\CF_error{syntax "(<name>--@<name>)" is not allowed}%
		\else
			\CF_displaycompound{#1}{#3}%
			\edef\CF_arrowstartname{%
				\CF_ifempty{#1}
					\CF_currentnodename
					{#1}%
				\CF_doifnotempty{#2}{.#2}%
			}%
			\CF_arrowf{#7}{#8}{#2}{#5}%
			\def\CF_nextnodename{#4}%
			\def\CF_nextnodestyle{#6}%
		\fi
	\fi
	\CF_arrowgobspaces% mange les espaces puis exécute \CF_nextaction
}

\def\CF_arrowgobspaces{\futurelet\CF_toksa\CF_arrowgobspacesa}

\def\CF_arrowgobspacesa{%
	\CF_ifx\CF_sptoken\CF_toksa
		\CF_arrowgobspacesb
		\CF_nextaction
}

\expandafter\def\expandafter\CF_arrowgobspacesb\space{\futurelet\CF_toksa\CF_arrowgobspacesa}

\def\CF_arrowf#1#2#3#4{% #1=angle   #2=longueur    #3=ancre départ   #4=ancre arrivée
	\def\CF_nextaction{\let\CF_compound\empty\CF_schemestartc}%
	\def\CF_arrowcurrentangle{#1}\CF_doifempty\CF_arrowcurrentangle{\let\CF_arrowcurrentangle\CF_arrowangle}%
	\def\CF_currentarrowlength{#2}\CF_doifempty\CF_currentarrowlength{\let\CF_currentarrowlength\CF_arrowlength}%
	\node[at=(\CF_currentnodename.\CF_ifempty{#3}\CF_arrowcurrentangle{#3}),shift=(\CF_arrowcurrentangle:\CF_currentarrowlength*\CF_compoundsep),cyan,fill](end@arrow@i@\number\CF_schemenest){};%
	\edef\CF_arrowendname{end@arrow@i@\number\CF_schemenest\CF_doifnotempty{#4}{.#4}}%
	\ifboolKV[chemfig]{scheme debug}
		{\node[at=(\CF_currentnodename.\CF_ifempty{#3}\CF_arrowcurrentangle{#3}),shift=(\CF_arrowcurrentangle:\CF_arrowoffset),red,fill](start@arrow){};%
		\node[at=(\CF_currentnodename.\CF_ifempty{#3}\CF_arrowcurrentangle{#3}),shift=(\CF_arrowcurrentangle:\CF_currentarrowlength*\CF_compoundsep-\CF_arrowoffset),red,fill](end@arrow){};%
		}
		{\node[at=(\CF_currentnodename.\CF_ifempty{#3}\CF_arrowcurrentangle{#3}),shift=(\CF_arrowcurrentangle:\CF_arrowoffset)](start@arrow){};%
		\node[at=(\CF_currentnodename.\CF_ifempty{#3}\CF_arrowcurrentangle{#3}),shift=(\CF_arrowcurrentangle:\CF_currentarrowlength*\CF_compoundsep-\CF_arrowoffset)](end@arrow){};%
		}%
	\def\CF_arrowstartnode{start@arrow}\def\CF_arrowendnode{end@arrow}%
	\csname\expandafter\CF_grabarrowname\CF_currentarrowtype[\_nil\CF_threeea\endcsname
		\expandafter\CF_grabarrowargs\CF_currentarrowtype[]\_nil[][][][][][][][]\_nil
	\def\CF_currentnodename{end@arrow@i@\number\CF_schemenest}%
	\edef\CF_nextnodeanchor{\CF_ifempty{#4}{180+\CF_arrowcurrentangle}{#4}}%
}

% trace un flèche initiée par (@nom--@nom)
% #1=type de flèche   #2=nom depart   #3=ancre départ    #4=nom arrivée    #5=ancre arrivée    #6=style flèche
\def\CF_directarrow#1#2#3#4#5#6{%
	\CF_expsecond{\def\CF_arrowstartname}{\CF_gobarg#2}%
	\CF_expsecond{\def\CF_arrowendname}{\CF_gobarg#4}%
	\path[sloped,allow upside down](\CF_gobarg#2\ifx\empty#3\empty\else.#3\fi)--(\CF_gobarg#4\ifx\empty#5\empty\else.#5\fi)%
		coordinate[pos=0,xshift=\CF_arrowoffset](start@direct@arrow)%
		coordinate[pos=1,xshift=-\CF_arrowoffset](end@direct@arrow);%
	\def\CF_arrowstartnode{start@direct@arrow}%
	\def\CF_arrowendnode{end@direct@arrow}%
	\pgfmathanglebetweenpoints
		{\pgfpointanchor{\CF_gobarg#2}{\ifx\empty#3\empty center\else#3\fi}}% Ne pas utiliser \CF_ifempty ici !!!
		{\pgfpointanchor{\CF_gobarg#4}{\ifx\empty#5\empty center\else#5\fi}}%
	\let\CF_arrowcurrentangle\pgfmathresult
	\CF_doifnotempty{#6}{\CF_addtomacro\CF_arrowcurrentstyle{#6,}}%
	\csname\CF_grabarrowname#1[]\_nil\expandafter\endcsname\CF_grabarrowargs#1[]\_nil[][][][][][][][]\_nil
}

\def\CF_mergegrabchardir#1[#2][#3]#4\_nil{%
	\CF_expafter{\futurelet\CF_toksa\CF_gobtonil}{\CF_firsttonil#1>\_nil}\_nil
	\ifx>\CF_toksa
		\def\CF_mergeangle{0}\def\CF_mergeextreme{xmax}\def\CF_mergesign{+}%
	\else
	\ifx<\CF_toksa
		\def\CF_mergeangle{180}\def\CF_mergeextreme{xmin}\def\CF_mergesign{-}%
	\else
	\ifx^\CF_toksa
		\def\CF_mergeangle{90}\def\CF_mergeextreme{ymax}\def\CF_mergesign{+}%
	\else
	\ifx v\CF_toksa
		\def\CF_mergeangle{-90}\def\CF_mergeextreme{ymin}\def\CF_mergesign{-}%
	\fi\fi\fi\fi
	\def\CF_mergelabelup{#2}\def\CF_mergelabeldo{#3}%
}

\def\CF_merge#1({%
	\CF_mergegrabchardir#1[][]\_nil
	\CF_mergea(%
}

\def\CF_mergea#1--(#2){\CF_testopt{\CF_mergeb#1--(#2)}{}}

\def\CF_mergeb#1--(#2)[#3]{%
	\CF_displaycompound{}{}%
	\CF_parsemergeopt#3,\CF_quark,\CF_quark,\CF_quark\_nil
	\def\CF_mergexmax{-16383.99999pt}\let\CF_mergeymax\CF_mergexmax
	\def\CF_mergexmin{16383.99999pt}\let\CF_mergeymin\CF_mergexmin
	\CF_mergeparsenodelist#1(\relax)% calcule les maxi des positions
	\pgfmathsetmacro\CF_mergeextremeresult{\csname CF_merge\CF_mergeextreme\endcsname\CF_mergesign\CF_mergefromcoeff*\CF_compoundsep}%
	\CF_mergec#1(\relax)% trace les lignes entre les noeuds précédents et la ligne de jonction
	\CF_expsecond{\def\CF_temp}{\expandafter[\CF_mergestyle,shorten <=0,shorten >=0,-]}%
	\if x\expandafter\CF_firsttonil\CF_mergeextreme\_nil
		\CF_addtomacro\CF_temp{(\CF_mergeextremeresult pt,\CF_mergeymax)--(\CF_mergeextremeresult pt,\CF_mergeymin)}%
	\else
		\CF_addtomacro\CF_temp{(\CF_mergexmin,\CF_mergeextremeresult pt)--(\CF_mergexmax,\CF_mergeextremeresult pt)}%
	\fi
	\expandafter\draw\CF_temp node[pos=\CF_mergesplitcoeff](merge@point){}% trace la ligne de jonction
		node[at=(merge@point),shift=(\CF_mergeangle:\CF_compoundsep*\CF_mergetocoeff-\CF_arrowoffset)](end@merge){}%
		node[at=(merge@point),shift=(\CF_mergeangle:\CF_compoundsep*\CF_mergetocoeff)](end@merge@i){};%
	\let\CF_arrowcurrentangle\CF_mergeangle
	\CF_expafter{\draw[}\CF_mergestyle,shorten <=0](merge@point)--(end@merge)%
		\expandafter\CF_arrowdisplaylabela\expandafter{\CF_mergelabelup}{.5}+\expandafter\CF_arrowdisplaylabela\expandafter{\CF_mergelabeldo}{.5}-;%
	\def\CF_currentnodename{end@merge@i}%
	\let\CF_temp\empty
	\CF_analysearrowarg{#2}%
	\expandafter\CF_merged\CF_temp\_nil
}

\def\CF_mergec(#1){%
	\if\relax\expandafter\noexpand\CF_firsttonil#1\_nil
	\else
		\CF_ifdot{#1}%
			{\edef\merge_currentnodename{\CF_beforedot#1\_nil}%
			\edef\merge_currentanchor{\CF_afterdot#1\_nil}%
			}%
			{\def\merge_currentnodename{#1}%
			\let\merge_currentanchor\CF_mergeangle
			}%
		\if x\expandafter\CF_firsttonil\CF_mergeextreme\_nil
			\pgfextracty\CF_dim{\pgfpointanchor\merge_currentnodename\merge_currentanchor}%
			\CF_expafter{\draw[}\CF_mergestyle,shorten >=0,-]([shift=(\CF_mergeangle:\CF_arrowoffset)]\merge_currentnodename.\merge_currentanchor)--(\CF_mergeextremeresult pt,\CF_dim);%
		\else
			\pgfextractx\CF_dim{\pgfpointanchor\merge_currentnodename\merge_currentanchor}%
			\CF_expafter{\draw[}\CF_mergestyle,shorten >=0,-]([shift=(\CF_mergeangle:\CF_arrowoffset)]\merge_currentnodename.\merge_currentanchor)--(\CF_dim,\CF_mergeextremeresult pt);%
		\fi
		\expandafter\CF_mergec
	\fi
}

\def\CF_merged#1.#2[#3]\_nil{%
	\def\CF_nextnodename{#1}%
	\edef\CF_nextnodeanchor{%
		\CF_ifempty{#2}
			{180+\CF_mergeangle}
			{#2}%
		}%
	\def\CF_nextnodestyle{#3}%
	\let\CF_compound\empty
	\CF_schemestartc
}

\def\CF_parsemergeopt#1,#2,#3,#4\_nil{%
	\CF_ifempty{#1}
		{\def\CF_mergefromcoeff{0.5}}
		{\def\CF_mergefromcoeff{#1}}%
	\def\CF_mergetocoeff{0.5}%
	\def\CF_mergesplitcoeff{0.5}%
	\CF_expsecond{\def\CF_mergestyle}{\CF_arrowhead}%
	\CF_iffirsttokmatch\CF_quark{#2\relax}
		{}
		{\CF_ifempty{#2}
			{\def\CF_mergetocoeff{0.5}}
			{\def\CF_mergetocoeff{#2}}%
		\CF_iffirsttokmatch\CF_quark{#3\relax}
			{}
			{\CF_ifempty{#3}
				{\def\CF_mergesplitcoeff{0.5}}
				{\def\CF_mergesplitcoeff{#3}}%
			\CF_iffirsttokmatch\CF_quark{#4\relax}
				{}
				{\CF_parsemergeopta#4\_nil}%
			}%
		}%
}

\def\CF_parsemergeopta#1,\CF_quark#2\_nil{%
	\CF_ifempty{#1}
		{}
		{\CF_addtomacro\CF_mergestyle{,#1}}%
}

\def\CF_mergeparsenodelist(#1){%
	\if\relax\expandafter\noexpand\CF_firsttonil#1\_nil
	\else
		\CF_ifdot{#1}%
			{\edef\merge_currentnodename{\CF_beforedot#1\_nil}\edef\merge_currentanchor{\CF_afterdot#1\_nil}}%
			{\def\merge_currentnodename{#1}\let\merge_currentanchor\CF_mergeangle}%
		\pgfextractx\CF_dim{\pgfpointanchor\merge_currentnodename\merge_currentanchor}%
		\ifdim\CF_dim>\CF_mergexmax
			\edef\CF_mergexmax{\the\CF_dim}%
		\fi
		\ifdim\CF_dim<\CF_mergexmin
			\edef\CF_mergexmin{\the\CF_dim}%
		\fi
		\pgfextracty\CF_dim{\pgfpointanchor\merge_currentnodename\merge_currentanchor}%
		\ifdim\CF_dim>\CF_mergeymax
			\edef\CF_mergeymax{\the\CF_dim}%
		\fi
		\ifdim\CF_dim<\CF_mergeymin
			\edef\CF_mergeymin{\the\CF_dim}%
		\fi
		\expandafter\CF_mergeparsenodelist
	\fi
}

\def\CF_grabarrowname#1[#2\_nil{\detokenize{CF_arrow(#1)}}
\def\CF_grabarrowargs#1[#2\_nil{[#2}

\def\CF_makeparametertext#1{%
	\toks0{}%
	\CF_cntgroup#1\relax
	\CF_makeparametertexta1%
}

\def\CF_makeparametertexta#1{%
	\unless\ifnum#1>\CF_cntgroup
		\toks0\expandafter{\the\toks0[###1]}%
		\expandafter\CF_makeparametertexta\expandafter{\number\numexpr#1+1\expandafter}%
	\fi
}

% #1 est le nombre d'arguments optionnels, #2 est le nom et #3 le code
\def\definearrow#1#2#3{%
	\begingroup
		\CF_makeparametertext{#1}%
	\expandafter\endgroup
	\expandafter\def\csname\detokenize{CF_arrow(#2)}\expandafter\endcsname\the\toks0{#3\CF_gobtonil}%
}

\def\CF_ifdot#1{\CF_ifdota#1.\_nil}
\def\CF_ifdota#1.#2\_nil{\ifx\empty#2\empty\expandafter\CF_execsecond\else\expandafter\CF_execfirst\fi}
\def\CF_beforedot#1.#2\_nil{#1}
\def\CF_afterdot#1.#2\_nil{#2}

\def\CF_rotatenode*#1#2\_nil{%
	\CF_ifdot{#1}
		{\CF_beforedot#1\_nil}
		{#1}%
}
\def\CF_anchornode*#1#2\_nil#3{%
	\CF_ifdot{#1}
		{\CF_afterdot#1\_nil}
		{\CF_arrowcurrentangle-#390-#1}%
}

% #1 = label  #2 = position  #3 = + ou - (au dessus ou au dessous)  #4 : nom du noeud de départ
% #5 = label  #6 = position  #7 = + ou - (au dessus ou au dessous)  #8 : nom du noeud de fin
\def\CF_arrowdisplaylabel#1#2#3#4#5#6#7#8{%
	\CF_doifnotempty{#1#5}
		{\path(#4)--(#8)\CF_arrowdisplaylabela{#1}{#2}{#3}\CF_arrowdisplaylabela{#5}{#6}{#7};}%
}

\def\CF_arrowdisplaylabela#1#2#3{%
	\CF_doifnotempty{#1}
		{\if*\expandafter\CF_firsttonil\detokenize{#1}\_nil
			\ifboolKV[chemfig]{scheme debug}
				{node[pos=#2,sloped,yshift=#3\CF_arrowlabelsep,draw,fill,cyan](shifted@node){}%
				node[draw,rotate=\CF_rotatenode#1\_nil,anchor=\CF_anchornode#1\_nil#3,at=(shifted@node)]{\expandafter\CF_gobarg\CF_gobarg#1}%
				}
				{node[pos=#2,sloped,yshift=#3\CF_arrowlabelsep](shifted@node){}%
				node[rotate=\CF_rotatenode#1\_nil,anchor=\CF_anchornode#1\_nil#3,at=(shifted@node)]{\expandafter\CF_gobarg\CF_gobarg#1}%
				}%
		\else
			\ifboolKV[chemfig]{scheme debug}
				{node[pos=#2,sloped,yshift=#3\CF_arrowlabelsep,draw,fill,cyan](shifted@node){}%
				node[draw,pos=#2,anchor=-#390,sloped,yshift=#3\CF_arrowlabelsep]{#1}%
				}
				{node[pos=#2,anchor=-#390,sloped,yshift=#3\CF_arrowlabelsep]{#1}%
				}
		\fi
		}%
}

% pose des noeuds décalés de la dimension #1 à (\CF_arrowstartnode) et (\CF_arrowendnode)
\def\CF_arrowshiftnodes#1{%
	\unless\ifdim\CF_ifempty{#1}\CF_zero{#1}=0pt
		\expanded{%
			\noexpand\path(\CF_arrowstartnode)--(\CF_arrowendnode)%
			node[pos=0,sloped,yshift=#1](\CF_arrowstartnode1){}node[pos=1,sloped,yshift=#1](\CF_arrowendnode1){};}%
		\edef\CF_arrowstartnode{\CF_arrowstartnode1}\edef\CF_arrowendnode{\CF_arrowendnode1}%
	\fi
}

\definearrow3{->}{%
	\CF_arrowshiftnodes{#3}%
	\CF_expafter{\draw[}\CF_arrowcurrentstyle](\CF_arrowstartnode)--(\CF_arrowendnode);%
	\CF_arrowdisplaylabel{#1}{0.5}+\CF_arrowstartnode{#2}{0.5}-\CF_arrowendnode
}

\definearrow3{<-}{%
	\CF_arrowshiftnodes{#3}%
	\CF_expafter{\draw[}\CF_arrowcurrentstyle](\CF_arrowendnode)--(\CF_arrowstartnode);%
	\CF_arrowdisplaylabel{#1}{0.5}+\CF_arrowstartnode{#2}{0.5}-\CF_arrowendnode
}

\definearrow5{-/>}{%
	\CF_arrowshiftnodes{#3}%
	\CF_expafter{\draw[}\CF_arrowcurrentstyle](\CF_arrowstartnode)--(\CF_arrowendnode)%
		coordinate[midway,shift=(\CF_arrowcurrentangle:-1pt)](midway@i)%
		coordinate[midway,shift=(\CF_arrowcurrentangle:1pt)](midway@ii)%
		coordinate[at=(midway@i),shift=(\CF_ifempty{#4}{225}{#4+180}+\CF_arrowcurrentangle:\CF_ifempty{#5}{5pt}{#5})](line@start)%
		coordinate[at=(midway@i),shift=(\CF_ifempty{#4}{45}{#4}+\CF_arrowcurrentangle:\CF_ifempty{#5}{5pt}{#5})](line@end)%
		coordinate[at=(midway@ii),shift=(\CF_ifempty{#4}{225}{#4+180}+\CF_arrowcurrentangle:\CF_ifempty{#5}{5pt}{#5})](line@start@i)%
		coordinate[at=(midway@ii),shift=(\CF_ifempty{#4}{45}{#4}+\CF_arrowcurrentangle:\CF_ifempty{#5}{5pt}{#5})](line@end@i);
	\draw(line@start)--(line@end);%
	\draw(line@start@i)--(line@end@i);%
	\CF_arrowdisplaylabel{#1}{0.5}+\CF_arrowstartnode{#2}{0.5}-\CF_arrowendnode
}

\definearrow3{<->}{%
	\CF_arrowshiftnodes{#3}%
	\CF_expafter{\draw[}\CF_arrowcurrentstyle,\CF_arrowtip-\CF_arrowtip](\CF_arrowstartnode)--(\CF_arrowendnode);%
	\CF_arrowdisplaylabel{#1}{0.5}+\CF_arrowstartnode{#2}{0.5}-\CF_arrowendnode
}

\definearrow3{<=>}{%
	\CF_arrowshiftnodes{#3}%
	\path[allow upside down](\CF_arrowstartnode)--(\CF_arrowendnode)%
			node[pos=0,sloped,yshift=\CF_arrowdoublesep](\CF_arrowstartnode @u0){}%
			node[pos=0,sloped,yshift=-\CF_arrowdoublesep](\CF_arrowstartnode @d0){}%
			node[pos=1,sloped,yshift=\CF_arrowdoublesep](\CF_arrowstartnode @u1){}%
			node[pos=1,sloped,yshift=-\CF_arrowdoublesep](\CF_arrowstartnode @d1){};%
	\begingroup
		\ifboolKV[chemfig]{arrow double harpoon}
			{\pgfarrowharpoontrue}
			{}%
		\CF_expafter{\draw[}\CF_arrowcurrentstyle](\CF_arrowstartnode @u0)--(\CF_arrowstartnode @u1);%
		\CF_expafter{\draw[}\CF_arrowcurrentstyle](\CF_arrowstartnode @d1)--(\CF_arrowstartnode @d0);%
	\endgroup
	\CF_arrowdisplaylabel{#1}{0.5}+\CF_arrowstartnode{#2}{0.5}-\CF_arrowendnode%
}

\definearrow3{<->>}{%
	\CF_arrowshiftnodes{#3}%
	\path[allow upside down](\CF_arrowstartnode)--(\CF_arrowendnode)%
			node[pos=0,sloped,yshift=1pt](\CF_arrowstartnode @u0){}%
			node[pos=\CF_arrowdoubleposstart,sloped,yshift=-1pt](\CF_arrowstartnode @d0){}%
			node[pos=1,sloped,yshift=1pt](\CF_arrowstartnode @u1){}%
			node[pos=\CF_arrowdoubleposend,sloped,yshift=-1pt](\CF_arrowstartnode @d1){};%
	\begingroup
		\ifboolKV[chemfig]{arrow double harpoon}
			{\pgfarrowharpoontrue}
			{}%
		\CF_expafter{\draw[}\CF_arrowcurrentstyle](\CF_arrowstartnode @u0)--(\CF_arrowstartnode @u1);%
		\CF_expafter{\draw[}\CF_arrowcurrentstyle](\CF_arrowstartnode @d1)--(\CF_arrowstartnode @d0);%
	\endgroup
	\CF_arrowdisplaylabel{#1}{0.5}+\CF_arrowstartnode{#2}{0.5}-\CF_arrowendnode%
}

\definearrow3{<<->}{%
	\path[allow upside down](\CF_arrowstartnode)--(\CF_arrowendnode)%
			node[pos=\CF_arrowdoubleposstart,sloped,yshift=1pt](\CF_arrowstartnode @u0){}%
			node[pos=0,sloped,yshift=-1pt](\CF_arrowstartnode @d0){}%
			node[pos=\CF_arrowdoubleposend,sloped,yshift=1pt](\CF_arrowstartnode @u1){}%
			node[pos=1,sloped,yshift=-1pt](\CF_arrowstartnode @d1){};%
	\begingroup
		\ifboolKV[chemfig]{arrow double harpoon}
			{\pgfarrowharpoontrue}
			{}%
		\CF_expafter{\draw[}\CF_arrowcurrentstyle](\CF_arrowstartnode @u0)--(\CF_arrowstartnode @u1);%
		\CF_expafter{\draw[}\CF_arrowcurrentstyle](\CF_arrowstartnode @d1)--(\CF_arrowstartnode @d0);%
	\endgroup
	\CF_arrowdisplaylabel{#1}{0.5}+\CF_arrowstartnode{#2}{0.5}-\CF_arrowendnode
}

\definearrow30{%
	\CF_arrowshiftnodes{#3}%
	\CF_arrowdisplaylabel{#1}{0.5}+\CF_arrowstartnode{#2}{0.5}-\CF_arrowendnode
}

\definearrow5{-U>}{%
	\CF_arrowshiftnodes{#3}%
	\CF_expafter{\draw[}\CF_arrowcurrentstyle](\CF_arrowstartnode)--(\CF_arrowendnode)node[midway](Uarrow@arctangent){};%
	\CF_ifempty{#4}
		{\def\CF_Uarrowradius{0.333}}
		{\def\CF_Uarrowradius{#4}}%
	\CF_ifempty{#5}%
		{\def\CF_Uarrowabsangle{60}}
		{\pgfmathsetmacro\CF_Uarrowabsangle{abs(#5)}}% ne prendre en compte que la valeur absolue de l'angle
	\expandafter\draw\expanded{[\CF_ifempty{#1}{draw=none}{\unexpanded\expandafter{\CF_arrowcurrentstyle}},-]}(Uarrow@arctangent)%
		arc[radius=\CF_compoundsep*\CF_currentarrowlength*\CF_Uarrowradius,start angle=\CF_arrowcurrentangle-90,delta angle=-\CF_Uarrowabsangle]node(Uarrow@start){};
	\expandafter\draw\expanded{[\CF_ifempty{#2}{draw=none}{\unexpanded\expandafter{\CF_arrowcurrentstyle}}]}(Uarrow@arctangent)%
		arc[radius=\CF_compoundsep*\CF_currentarrowlength*\CF_Uarrowradius,start angle=\CF_arrowcurrentangle-90,delta angle=\CF_Uarrowabsangle]node(Uarrow@end){};
	\pgfmathsetmacro\CF_temp{\CF_Uarrowradius*cos(\CF_arrowcurrentangle)<0?"-":"+"}%
	\ifdim\CF_Uarrowradius pt>0pt
		\CF_arrowdisplaylabel{#1}{0}\CF_temp{Uarrow@start}{#2}{1}\CF_temp{Uarrow@end}%
	\else
		\CF_arrowdisplaylabel{#2}{0}\CF_temp{Uarrow@start}{#1}{1}\CF_temp{Uarrow@end}%
	\fi
}

\def\CF_grabdelim#1#2#3\_CFnil{\def\CF_leftdelim{#1}\def\CF_rightdelim{#2}}
\defKV[CFdelimiters]{%
	delimiters   = \CF_grabdelim#1()\_CFnil,
	height       = \def\CF_delimheight{#1},
	depth        = \CF_expsecond{\CF_defifempty\CF_delimdepth{#1}}{\CF_delimheight},
	open xshift  = \edef\CF_leftdelimxshift{\the\dimexpr#1},
	close xshift = \edef\CF_rightdelimxshift{\CF_ifempty{#1}{-\CF_leftdelimxshift}{-\the\dimexpr#1}}
}
\setKVdefault[CFdelimiters]{%
	delimiters   = (),
	height       = 10pt,
	depth        = ,
	open xshift  = 0pt,
	close xshift = ,
	h align      = true,
	auto rotate  = false,
	rotate       = 0,
	indice       = n
}%
\def\polymerdelim{\CF_ifnextchar[{\CF_polymerdelima}{\CF_polymerdelima[]}}
\def\CF_polymerdelima[#1]#2#3{%
	\restoreKV[CFdelimiters]%
	\CF_doifnotempty{#1}{\setKV[CFdelimiters]{#1}}%
	\edef\CF_delimhalfdim{\the\dimexpr(\CF_delimheight+\CF_delimdepth)/2}%
	\edef\CF_delimvshift {\the\dimexpr(\CF_delimheight-\CF_delimdepth)/2}%
	\chemmove{%
		\nulldelimiterspace0pt
		\pgfextractx\CF_dim{\pgfpointanchor{#2}{center}}\edef\CF_leftdelimx{\the\CF_dim}%
		\pgfextracty\CF_dim{\pgfpointanchor{#2}{center}}\edef\CF_leftdelimy{\the\CF_dim}%
		\pgfextractx\CF_dim{\pgfpointanchor{#3}{center}}\edef\CF_rightdelimx{\the\CF_dim}%
		\pgfextracty\CF_dim{\pgfpointanchor{#3}{center}}\edef\CF_rightdelimy{\the\CF_dim}%
		\def\CF_autorotate{0}%
		\ifboolKV[CFdelimiters]{h align}
			{\let\CF_rightdelimy\CF_leftdelimy
			}
			{%
			\ifboolKV[CFdelimiters]{auto rotate}
				{\pgfmathatantwo{\CF_rightdelimy-\CF_leftdelimy}{\CF_rightdelimx-\CF_leftdelimx}%
				\let\CF_autorotate\pgfmathresult
				}
				{\CF_eexpsecond\CF_ifempty{\useKV[CFdelimiters]{rotate}}
					{}
					{\edef\CF_autorotate{\useKV[CFdelimiters]{rotate}}}%
				}%
			}%
		\node[at={(\CF_leftdelimx+\CF_leftdelimxshift,\CF_leftdelimy+\CF_delimvshift)},rotate=\CF_autorotate]
			{$\left\CF_leftdelim\vrule height\CF_delimhalfdim depth\CF_delimhalfdim width0pt\right.$};%
		\node[at={(\CF_rightdelimx+\CF_rightdelimxshift,\CF_rightdelimy+\CF_delimvshift)},rotate=\CF_autorotate]
			{$\left.\vrule height\CF_delimhalfdim depth\CF_delimhalfdim width0pt\right\CF_rightdelim
				\CF_eexpsecond\CF_doifnotempty{\useKV[CFdelimiters]{indice}}
					{\CF_underscore{\rlap{$\scriptstyle\useKV[CFdelimiters]{indice}$}}}
			$};
	}%
}

\catcode`\@11
\pgfdeclarearrow{%
	name = CF,%
	defaults = {%
		length  = 3pt 5 1,%
		width'  = 0pt .8,%
		inset'  = 0pt .5,%
		line width = 0pt 1 1,%
		round%
	},%
	setup code = {%
		% Cap the line width at 1/4th distance from inset to tip
		\pgf@x\pgfarrowlength
		\advance\pgf@x by-\pgfarrowinset
		\pgf@x.25\pgf@x
		\ifdim\pgf@x<\pgfarrowlinewidth\pgfarrowlinewidth\pgf@x\fi
		% Compute front miter length:
		\pgfmathdivide@{\pgf@sys@tonumber\pgfarrowlength}{\pgf@sys@tonumber\pgfarrowwidth}%
		\let\pgf@temp@quot\pgfmathresult%
		\pgf@x\pgfmathresult pt%
		\pgf@x\pgfmathresult\pgf@x%
		\pgf@x4\pgf@x%
		\advance\pgf@x by1pt%
		\pgfmathsqrt@{\pgf@sys@tonumber\pgf@x}%
		\pgf@xc\pgfmathresult\pgfarrowlinewidth% xc is front miter
		\pgf@xc.5\pgf@xc
		\pgf@xa\pgf@temp@quot\pgfarrowlinewidth% xa is extra harpoon miter
		% Compute back miter length:
		\pgf@ya.5\pgfarrowwidth%
		\csname pgfmathatan2@\endcsname{\pgfmath@tonumber\pgfarrowlength}{\pgfmath@tonumber\pgf@ya}%
		\pgf@yb\pgfmathresult pt%
		\csname pgfmathatan2@\endcsname{\pgfmath@tonumber\pgfarrowinset}{\pgfmath@tonumber\pgf@ya}%
		\pgf@ya\pgfmathresult pt%
		\advance\pgf@yb by-\pgf@ya%
		\pgf@yb.5\pgf@yb% half angle in yb
		\pgfmathtan@{\pgf@sys@tonumber\pgf@yb}%
		\pgfmathreciprocal@{\pgfmathresult}%
		\pgf@yc\pgfmathresult\pgfarrowlinewidth%
		\pgf@yc.5\pgf@yc%
		\advance\pgf@ya by\pgf@yb%
		\pgfmathsincos@{\pgf@sys@tonumber\pgf@ya}%
		\pgf@ya\pgfmathresulty\pgf@yc% ya is the back miter
		\pgf@yb\pgfmathresultx\pgf@yc% yb is the top miter
		\ifdim\pgfarrowinset=0pt
			\pgf@ya.5\pgfarrowlinewidth% easy: back miter is half linewidth
		\fi
		% Compute inset miter length:
		\pgfmathdivide@{\pgf@sys@tonumber\pgfarrowinset}{\pgf@sys@tonumber\pgfarrowwidth}%
		\let\pgf@temp@quot\pgfmathresult%
		\pgf@x\pgfmathresult pt%
		\pgf@x\pgfmathresult\pgf@x%
		\pgf@x4\pgf@x%\pgf@ya
		\advance\pgf@x by1pt%
		\pgfmathsqrt@{\pgf@sys@tonumber\pgf@x}%
		\pgf@yc\pgfmathresult\pgfarrowlinewidth% yc is inset miter
		\pgf@yc.5\pgf@yc% 
		% Inner length (pgfutil@tempdima) is now arrowlength - front miter - back miter
		\pgfutil@tempdima\pgfarrowlength%
		\advance\pgfutil@tempdima by-\pgf@xc%
		\advance\pgfutil@tempdima by-\pgf@ya%
		\pgfutil@tempdimb.5\pgfarrowwidth%
		\advance\pgfutil@tempdimb by-\pgf@yb%
		% harpoon miter correction
		\ifpgfarrowroundjoin
			\pgfarrowssetbackend{\pgf@ya\advance\pgf@x by-.5\pgfarrowlinewidth}%
		\else
			\pgfarrowssetbackend{0pt}
		\fi
		\ifpgfarrowharpoon
			\pgfarrowssetlineend{\pgfarrowinset\advance\pgf@x by\pgf@yc\advance\pgf@x by.5\pgfarrowlinewidth}%
		\else
			\pgfarrowssetlineend{\pgfarrowinset\advance\pgf@x by\pgf@yc\advance\pgf@x by-.25\pgfarrowlinewidth}%
			\ifpgfarrowreversed
				\ifdim\pgfinnerlinewidth>0pt
					\pgfarrowssetlineend{\pgfarrowinset}%
				\else
					\pgfarrowssetlineend{\pgfutil@tempdima\advance\pgf@x by\pgf@ya\advance\pgf@x by-.25\pgfarrowlinewidth}%
				\fi
			\fi
		\fi
		\ifpgfarrowroundjoin
			\pgfarrowssettipend{\pgfutil@tempdima\advance\pgf@x by\pgf@ya\advance\pgf@x by.5\pgfarrowlinewidth}%
		\else
			\pgfarrowssettipend{\pgfarrowlength\ifpgfarrowharpoon\advance\pgf@x by\pgf@xa\fi}%
		\fi
		% The hull:
		\pgfarrowshullpoint{\pgfarrowlength\ifpgfarrowroundjoin\else\ifpgfarrowharpoon\advance\pgf@x by\pgf@xa\fi\fi}{\ifpgfarrowharpoon-.5\pgfarrowlinewidth\else0pt\fi}%
		\pgfarrowsupperhullpoint{0pt}{.5\pgfarrowwidth}%
		\pgfarrowshullpoint{\pgfarrowinset}{\ifpgfarrowharpoon-.5\pgfarrowlinewidth\else 0pt\fi}%
		% Adjust inset
		\pgfarrowssetvisualbackend{\pgfarrowinset}%
		\advance\pgfarrowinset by\pgf@yc%
		% The following are needed in the code:
		\pgfarrowssavethe\pgfutil@tempdima
		\pgfarrowssavethe\pgfutil@tempdimb
		\pgfarrowssavethe\pgfarrowlinewidth
		\pgfarrowssavethe\pgf@ya
		\pgfarrowssavethe\pgfarrowinset
	},%
	drawing code = {%
		\pgfsetdash{}{0pt}%
		\ifpgfarrowroundjoin\pgfsetroundjoin\else\pgfsetmiterjoin\fi
		\ifdim\pgfarrowlinewidth=\pgflinewidth\else\pgfsetlinewidth{\pgfarrowlinewidth}\fi
		\pgfpathmoveto{\pgfqpoint{\pgfutil@tempdima\advance\pgf@x by\pgf@ya}{0pt}}%
		\pgfpathlineto{\pgfqpoint{\pgf@ya}{\pgfutil@tempdimb}}%
		\pgfpathlineto{\pgfqpoint{\pgfarrowinset}{0pt}}%
		\ifpgfarrowharpoon \else
			\pgfpathlineto{\pgfqpoint{\pgf@ya}{-\pgfutil@tempdimb}}%
		\fi
		\pgfpathclose
		\ifpgfarrowopen\pgfusepathqstroke\else\ifdim\pgfarrowlinewidth>0pt \pgfusepathqfillstroke\else\pgfusepathqfill\fi\fi
	},%
	parameters = {%
		\the\pgfarrowlinewidth,%
		\the\pgfarrowlength,%
		\the\pgfarrowwidth,%
		\the\pgfarrowinset,%
		\ifpgfarrowharpoon h\fi%
		\ifpgfarrowopen o\fi%
		\ifpgfarrowroundjoin j\fi%
	}%
}
\CFrestorecatcode
\endinput

%%%%%%%%%%%%%%%%%%%%%%%%%%%%%%%%%%%%%%%%%%%%%%%%%%%%%%%%%%%%%%%%%%%%%%
%%%                       H I S T O R I Q U E                      %%%
%%%%%%%%%%%%%%%%%%%%%%%%%%%%%%%%%%%%%%%%%%%%%%%%%%%%%%%%%%%%%%%%%%%%%%

v0.1        2010/06/23
    - Première version publique sur le CTAN
----------------------------------------------------------------------
v0.2        2010/08/31
    - Ajout de la documentation en anglais.
    - Correction de bugs.
    - \printatom est désormais une macro publique.
    - Les espaces sont permis dans les molécules. Ils seront
      ignorés par défaut puisque les atomes sont composés en
      mode math par \printatom
    - Une paire de Lewis peut être représentée ":".
    - Dans les cycles, une correction de la longueur du trait
      déporté des liaisons doubles est fait de telle sorte que si
      l'on écrit \chemfig{*5(=====)}, on obtient deux polygones
      réguliers concentriques.
    - La séquence de contrôle \setnodestyle permet de spécifier
      le style des n?uds dessinés par tikz.
----------------------------------------------------------------------
v0.3        2010/11/21
    - Amélioration de \definesubmol qui accepte les séquences de
      contrôle. On peut aussi choisir un alias dont la substitution
      est différente selon l'orientation de la liaison qui lui
      arrive dessus.
    - Le caractère "|" force la fin d'un atome. Si on écrit
      "D|ef" alors, chemfig verra deux atomes "D" et "ef".
    - Le caractère "#" est reconnu lorsqu'il suit un caractère de
      liaison. Il doit être suivi d'un argument entre parenthèses
      qui contient l'offset de début et de fin qui s'appliqueront
      à cette liaison.
    - La macro \chemfig admet un argument optionnel qui sera passé
      à l'environnement tikzpicture dans lequel elle est dessinée
    - Mise en place de la représentation des mécanismes
      réactionnels avec la syntaxe "@{<nom>}" devant un atome où
      "@{<nom>,<coeff>}" au tout début de l'argument d'une liaison.
      Cette syntaxe permet de placer un n?ud (au sens de tikz) qui
      deviendra l'extrémité des flèches des mécanismes.
      Le tracé des flèches est faite par la macro \chemmove dont
      l'argument optionnel devient celui de l'environnement
      tikzpicture dans lequel sont faites les flèches.
    - Pour le mécanisme d'alignement vertical via le \vphantom, la
      commande \chemskipalign permet d'ignorer le groupe d'atomes
      dans lequel elle est écrite.
    - La commande \chemname permet d'afficher un nom sous une
      molécule. la commande \chemnameinit initialise la plus grande
      profondeur rencontrée.
    - La commande \lewis a été modifiée de telle sorte que les
      dessins des décorations soient proportionnels à la taille
      de la police.
----------------------------------------------------------------------
v0.3a       2011/01/08
    - Correction d'un bug dans l'argument optionnel de \definesubmol
      lorsque celui-ci comporte des crochets.
    - Mise à jour du manuel en anglais.
    - Ajout de \vflipnext et \hflipnext pour retourner
      horizontalement ou verticalement la prochaine molécule.
----------------------------------------------------------------------
v0.4        2011/03/07
    - chemfig est désormais écrit en plain-etex et donc
      utilisable par d'autres formats que LaTeX.
    - Un peu plus de rigueur avec les catcodes des caractères
      spéciaux, notamment lorsque la commande \chemfig se trouve
      dans l'argument de \chemmove, \chemabove, \chembelow, \chemrel.
      TODO : faut-il \scantoker l'argument de \chemfig pour être
      définitivement débarrassé de ces histoires de catcode ???
    - Correction d'un bug dans le calcul de l'angle des liaisons
----------------------------------------------------------------------
v0.4a       2011/04/10
    - Correction d'un bug concernant l'argument optionnel en début
      de molécule.
----------------------------------------------------------------------
v0.4b       2011/04/24
    - l'argument de \chemfig est tokénisé avec \scantokens ce qui
      rend caduc tout souci de code de catégorie, à part #.
    - la commande \setbondstyle permet de définir le style des
      liaisons.
    - correction de l'affichage incorrect des doubles liaisons dans
      les cycles après les commandes \hflipnext et \vflipnext
    - correction d'un bug lorsqu'un alias commence une molécule
----------------------------------------------------------------------
v1.0       2011/06/15
    - les schémas réactionnels sont désormais disponibles.
    - \Chemabove et \Chembelow modifient la boite englobante.
    - \Lewis modifie la boite englobante
    - les macros \chemleft, \chemright, \chemup et \chemdown
      affichent des délimiteurs extensibles à gauche, à droite,
      au dessus et au dessous d'un matériel.
----------------------------------------------------------------------
v1.0a      2011/09/18
    - les macros \Lewis et \lewis admettent un argument optionnel
    - la macro \setlewisdist règle la distance entre les 2
      électrons
----------------------------------------------------------------------
v1.0b      2011/11/29
    - la commande \merge est désormais protégée entre
      \schemestart et \schemestop contre des définitions par d'autres
      packages.
    - \box0 est utilisé au lieu du maladroit \unhbox0
----------------------------------------------------------------------
v1.0c      2011/11/30
    - la macro \+ n'est plus explicitement écrite
    - vérifie que eTeX est le moteur utilisé
----------------------------------------------------------------------
v1.0d      2011/12/19
    - les cercles des cycles étaient tracés au mauvais moment. La
      longueur de la liaison qui les précédait influait sur le
      rayon du cercle : \chemfig{-[,0.5]**6(------)} donnait un bug
      à l'affichage.
----------------------------------------------------------------------
v1.0e      2012/01/13
    - la gestion des espaces dans les groupes d'atomes est
      désormais plus rigoureuse. Plusieurs bugs ont été
      corrigé
----------------------------------------------------------------------
v1.0f      2012/02/24
    - correction d'un bug avec \definesubmol, les catcodes n'étaient
      pas correctement gérés.
----------------------------------------------------------------------
v1.0g      2012/11/16
    - correction d'un bug dans \CF_directarrow pour faire prendre en
      compte le style des flèche par défaut
    - correction d'un bug dans \CF_lewisc : la boite *doit* être
      composée en dehors de l'environnement tikzpicture pour
      éviter nullfont si jamais \printatom ne passe pas en mode
      math.
    - correction d'un bug dans \CF_chemfigc : si une longueur par
      défaut est modifiée par [,<l>] au début d'une molécule
      et si des cycles étaient emboités, cette longueur n'était
      pas appliquée aux sous-cycles.
    - ré-écriture des macros \chemabove et \chembemow pour
      prendre en compte le bug (désormais corrigé) dans luatex.
    - nouvelle macro \setstacksep qui définit l'espacement par
      défaut dans les macros \chemabove et \chembelow.
----------------------------------------------------------------------
v1.0h      2013/11/28
    - \chemname admet maintenant une version étoilé qui ne tient
      pas compte des profondeurs précédentes.
    - \CF_dpmax est géré globalement.
    - correction d'un bug dans "-U>" : le style de la flèche
      n'était pris en compte pour l'arc.
    - correction d'un bug dans \CF_directarrow : l'angle de la
      flèche n'était pas calculé
----------------------------------------------------------------------
v1.1       2015/02/13
    - correction d'un bug dans \CF_seeksubmol : la macro
      \CF_molecule est dépouillé de son éventuel espace
      en première position.
    - correction d'un bug dans \CF_arrowf : le nom du prochain
      n?ud courant "end@arrow@i" était erroné dans le cas où une
      flèche contenait un sous schéma. Ce nom doit dépendre de
      \CF_schemenest.
    - la jonction entre deux liaisons consécutives dans l'axe peut
      être activé avec \enablebondjoin et désactivé avec
      \disablebondjoin (préférable, état par défaut).
    - \chemfig suivi d'une "*" demande à ce que les liaisons aient
      une longueur invariable : la distance inter-atome devient donc
      variable. Cette fonctionnalité est désactivé dans les
      cycles afin que les polygones soient réguliers.
      \enablefixedbondlength permet cette fonctionnalité pour
      toutes les macros \chemfig (même non étoilée) tandis que
      \disablefixedbondlength le désactive.
----------------------------------------------------------------------
v1.1a      2015/02/23
    - correction d'un bug dans \CF_grabbondoffset. Si \chemfig est
      dans l'argument d'une macro, les # sont doublés par l'action
      de \scantokens de la macro \CF_chemfigb et il faut un
      argument délimité avant "(" pour absorber tous les #.
----------------------------------------------------------------------
v1.2       2015/10/08
    - correction d'un bug dans le tracé des liaisons de Cram.
    - création de \setangleincrement.
    - chargement de "arrows.meta" et définition de la flèche "CF"
      basée sur "Stealth" et définie avec \pgfdeclarearrow.
      Les anciennes flèches "CF_full" et "CF_half" sont
      abandonnées puisque définies avec \pgfarrowsdeclare.
    - flèche "-U>" corrigée : le placement des labels est
      maintenant correct dans tous les cas. Ainsi :
                         -U>[<a>][<b>][<d>][r][a]
      place le label <a> près du début de la flèche, quels que
      soient les signes du rayon r et de l'angle a.
    - \chemrel, \setchemrel et \chemsign sont supprimées.
    - compatibilité, avec les limitations d'usage, avec la
      librairie "externalize" : le \begin{tikzpicture} voit
      désormais le \end{tikzpicture} correspondant dans la macro
      \CF_chemfigb.
----------------------------------------------------------------------
v1.2a      2015/10/21
    - erreur de copier-coller dans le code: une adresse url était
      malencontreusement présente en plein milieu d'une ligne de
      code
----------------------------------------------------------------------
v1.2b      2015/11/15
    - bug dans \CF_seeksubmol qui laissait "*" dans le flux de
      lecture de TeX. Un message d'erreur est également ajouté
      en cas de "!" en fin de traitement.
    - correction d'un bug dans \CF_setbondangle où l'angle [<:a>]
      n'était pas évalué par \pgfmathsetmacro.
----------------------------------------------------------------------
v1.2c      2015/11/20
    - Correction d'un bug dans \CF_setbondangle : l'angle renvoyé
      pouvait être négatif
    - Correction d'un bug dans \CF_directarrow : la macro \CF_ifempty
      n'est pas correctement développée dans l'argument de
      \pgfpointanchor
----------------------------------------------------------------------
v1.2d      2015/12/01
    - correction d'un bug dans la flèche "-U"
    - la version étoilée de \setcrambond dessine les liaisons de
      Cram en pointillés sous forme de trait large et non pas sous
      forme de triangle.
----------------------------------------------------------------------
v1.2e      2017/05/20
    - la macro contenant la définition d'une flèche est
      désormais "\CF_arrow(<nom>)", ainsi la macro \0 n'est plus
      définie par \definearrow
    - remerciements rajoutés après une suppression indue, pour ne
      froisser aucune susceptibilité
----------------------------------------------------------------------
v1.3       2018/03/08
    - tous les paramètres sont désormais passés via \setchemfig qui
      fait appel à "simplekv". Par conséquent, _toutes_ les macros qui
      réglaient des paramètres deviennent obsolètes, à savoir :
          \setcrambond, \setatomsep, \setbondoffset, \setdoublesep,
          \setangleincrement, \enablefixedbondlength,
          \disablefixedbondlength, \setnodestyle, \setbondstyle,
          \setlewis, \setlewisdist, \setstacksep, \setcompoundstyle,
          \setarrowdefault, \setandsign, \setarrowoffset,
          \setcompoundsep, \setarrowlabelsep, \enablebondjoin,
          \disablebondjoin et \schemedebug.
      et ces macros seront *supprimées* dans une future version.
    - la version étoilée "\chemfig*" et les deux arguments optionnels
      de la macro "\chemfig[][]" sont également optionnels et seront
      *supprimé*s dans une future version afin d'accéder à la syntaxe
      \chemfig[clés=valeurs]{code}
    - 6 nouveaux paramètres : "lewis radius", "arrow double sep",
      "arrow double coeff", "arrow double harpoon", "cycle radius
      coeff", "arrow head".
    - correction d'un bug dans \CF_parsemergeopt qui dans certains
      cas, envoyait vers l'affichage des caractères
    - petit toilettage du code
    - macro \polymerdelim (non documentée) expérimentale et encore
      en phase de tests
    - suppression d'un registre d'écriture de fichier
----------------------------------------------------------------------
v1.31       2018/04/05
    - correction d'un espace indésirable dans \CF_ifnextchar
----------------------------------------------------------------------
v1.32       2018/08/23
    - définition de \printatom, \CF_begintikzpicture et 
      \CF_endtikzpicture dans le fichier t-chemfig.tex
----------------------------------------------------------------------
v1.33       2018/10/31
    - les macros définies par \definesubmol peuvent désormais avoir un ou
      plusieurs arguments
    - macro \polymerdelim documentée
----------------------------------------------------------------------
v1.34       2019/02/23
    - bug dans la flèche "<->" corrigé
----------------------------------------------------------------------
v1.4        2019/04/18
    - corrections de nombreux bugs
    - caractère privé "_" et non plus "@" -> modifications à prévoir
      notamment dans la doc avec les codes spécifiques aux flèches, ça
      risque de couiner sur tex.stackexchange.com
    - anciennes macros abandonnées et désormais indéfinies :
          \setcrambond, \setatomsep, \setbondoffset, \setdoublesep,
          \setangleincrement, \enablefixedbondlength,
          \disablefixedbondlength, \setnodestyle, \setbondstyle,
          \setlewis, \setlewisdist, \setstacksep, \setcompoundstyle,
          \setarrowdefault, \setandsign, \setarrowoffset,
          \setcompoundsep, \setarrowlabelsep, \enablebondjoin,
          \disablebondjoin et \schemedebug
    - l'ancienne syntaxe \chemfig[][]{} est abandonnée et n'est plus
      acceptée, désormais c'est 
      \chemfig[<clés>=<valeurs>]{<code molécule>}
    - l'ancienne syntaxe \lewis[<coeff>] ou \Lewis[<coeff>] n'est
      plus acceptée au profit de \lewis[<clés>=<valeurs>]
----------------------------------------------------------------------
v1.41       2019/05/21
    - utilisation de la nouvelle primitive \expanded
    - nouvelle clé "h align" (true par défaut) pour les délimiteurs
      de \polymerdelim. Lorsque à false, les délimiteurs ne sont
      plus alignés horizontalement mais positionnés aux noeuds demandés
    - nouvelle clé "auto rotate" qui n'a de sens que si h align=false :
      les délimiteurs sont automatiquement inclinés
    - nouvelle clé "rotate" qui n'a de sens que si halign=false ET
      auto rotate=false : l'inclinaison des délimiteurs peut être
      choisie
----------------------------------------------------------------------
v1.5        2020/03/05
    - nouvelles macros \charge et \Charge. Les macros \lewis et \Lewis
      sont obsolètes et amenées à disparaitre à moyen terme (au moins
      9 mois), soit fin 2020
    - prise en compte de la dimension d'un groupe d'atome pour tracer
      des liaisons jointives
    - bug corrigé dans \CF_seeknode
    - ajout d'une section dans le manuel (placement des atomes)
----------------------------------------------------------------------
v1.51       2020/04/06
    - bug corrigé dans \chargerect_a et \chargeline_a
----------------------------------------------------------------------
v1.52       2020/04/14
    - bug : définition corrigée de \CFthesubmol dans \def_submolc pour
            qu'elle se développe en 1 coup seulement
----------------------------------------------------------------------
v1.53       2020/04/27
    - mise à jour en fonction des nouvelles fonctionnalités de
      l'extension simplekv
    - bug : \CF_ifzerodim interrompt maintenant le tracé dans la \hbox
----------------------------------------------------------------------
v1.54       2020/05/21
    - chemfig ne peut plus fonctionner sans \expanded
    - bug : un signe "=" laissé par erreur dans le flux
----------------------------------------------------------------------
v1.55       2020/06/15
    - chemfig est incompatible avec conTeXt, vu que ce moteur redéfinit
      des primitives telles que \expanded, \unexpanded et peut être
      d'autres.
----------------------------------------------------------------------
v1.56       2020/07/13
    - le centre des cycles est désormais accessible via un noeud
      spécifique pour chacun d'eux.
----------------------------------------------------------------------
v1.6        2021/02/26
    - les macros des formules de Lewis sont retirées et placées dans
      le fichier séparé "lewis.tex" que l'utilisateur peut charger
      s'il le souhaite
    - ajout d'une clé <debug> pour le trousseau [chemfig]
    - à l'intérieur d'un schéma, le token '#' est permis dans
      l'argument de \chemfig
----------------------------------------------------------------------
v1.6a       2021/02/28
    - le fichier lewis.tex a été renommé chemfig-lewis.tex
----------------------------------------------------------------------
v1.6b       2021/08/01
    - encodage UTF-8
    - la macro \# n'était pas définie pour remplacer «#(...)» lorsque
      \chemfig se trouve dans l'argument d'une macro.
----------------------------------------------------------------------
v1.6c       2022/09/27
    - nouvelles clés "gchemname", "schemestart code" et
      "schemestop code" (suggestion de Balazs Debreceni)