%%
%% This is file `t-lt3luabridge.tex',
%% generated with the docstrip utility.
%%
%% The original source files were:
%%
%% lt3luabridge.dtx  (with options: `context-package')
%% 
%% Copyright (C) 2016-2022 Vít Novotný
%% 
%% This work may be distributed and/or modified under the
%% conditions of the LaTeX Project Public License, either version 1.3c
%% of this license or (at your option) any later version.
%% The latest version of this license is in
%% 
%%    http://www.latex-project.org/lppl.txt
%% 
%% This work has the LPPL maintenance status `maintained'.
%% The Current Maintainer of this work is Vít Novotný.
%% 
%% Send bug reports, requests for additions and questions
%% either to the GitHub issue tracker at
%% 
%%   https://github.com/Witiko/lt3luabridge/issues
%% 
%% or to the e-mail address <witiko@mail.muni.cz>.
%% 
%% MODIFICATION ADVICE:
%% 
%% If you want to customize this file, it is best to make a copy of
%% the source file(s) from which it was produced.  Use a different
%% name for your copy(ies) and modify the copy(ies); this will ensure
%% that your modifications do not get overwritten when you install a
%% new release of the standard system.  You should also ensure that
%% your modified source file does not generate any modified file with
%% the same name as a standard file.
%% 
%% You will also need to produce your own, suitably named, .ins file to
%% control the generation of files from your source file; this file
%% should contain your own preambles for the files it generates, not
%% those in the standard .ins files.
%% 
%% The names of the source files used are shown above.
%% 
\writestatus{loading}{ConTeXt User Module / lt3luabridge}
\startmodule[lt3luabridge]
\unprotect
\input lt3luabridge\relax
\endinput
%%
%% End of file `t-lt3luabridge.tex'.
