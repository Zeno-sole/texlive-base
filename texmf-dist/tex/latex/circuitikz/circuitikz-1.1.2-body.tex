% Copyright 2018-2020 by Romano Giannetti
% Copyright 2015-2020 by Stefan Lindner
% Copyright 2013-2020 by Stefan Erhardt
% Copyright 2007-2020 by Massimo Redaelli
%
% This file may be distributed and/or modified
%
% 1. under the LaTeX Project Public License and/or
% 2. under the GNU Public License.
%
% See the files gpl-3.0_license.txt and lppl-1-3c_license.txt for more details.

\NeedsTeXFormat{LaTeX2e}

\def\pgfcircversion{1.1.2}
\def\pgfcircversiondate{2020/05/17}

%% Version 3.0 of pgf/TikZ is required
\RequirePackage{tikz}
\usetikzlibrary{calc}
%
% "arrows" library is deprecated, and behave badly with
% arrows on short paths. Change to the new arrows.meta
% In pfgcirc.define, we will add the old definition of
% latex' which we have lost in the transition
%
\usetikzlibrary{arrows.meta, bending}
\usetikzlibrary{fpu} % may be needed for use fpu reciprocal (v1.0.1)

% The options are listed in the manual in this order

\DeclareOption{europeanvoltage}{
    \ctikzset{voltage=european}
}

\DeclareOption{straightvoltages}{
    \pgf@circuit@bipole@voltage@straighttrue
}

\DeclareOption{americanvoltage}{
    \ctikzset{voltage=american}
}

\DeclareOption{europeancurrent}{
    \ctikzset{current = european}
}

\DeclareOption{americancurrent}{
    \ctikzset{current = american}
}



\DeclareOption{americanresistor}{
    \ctikzset{resistor = american}
}

\DeclareOption{europeanresistor}{
    \ctikzset{resistor = european}
}

\DeclareOption{americaninductor}{
    \ctikzset{inductor = american}
}

\DeclareOption{europeaninductor}{
    \ctikzset{inductor = european}
}

\DeclareOption{cuteinductor}{
    \ctikzset{inductor = cute}
}

\DeclareOption{americanport}{
    \ctikzset{logic ports = american}
}

\DeclareOption{europeanport}{
    \ctikzset{logic ports = european}
}

\DeclareOption{european}{
    \ctikzset{voltage=european} \ctikzset{current=european} \ctikzset{inductor=european}
    \ctikzset{resistor=european} \ctikzset{logic ports=european} \ctikzset{gas filled surge arrester choice = european}
}

\DeclareOption{american}{
    \ctikzset{voltage=american} \ctikzset{current=american} \ctikzset{resistor=american} \ctikzset{inductor=american} \ctikzset{gas filled surge arrester choice = american}
    \ctikzset{logic ports = american}
}

\DeclareOption{fulldiodes}{
    \ctikzset{diode = full}
}

\DeclareOption{emptydiodes}{
    \ctikzset{diode = empty}
}

\DeclareOption{europeanvoltages}{
    \ctikzset{voltage=european}
}

\DeclareOption{americanvoltages}{
    \ctikzset{voltage=american}
}

\DeclareOption{europeancurrents}{
    \ctikzset{current = european}
}

\DeclareOption{americancurrents}{
    \ctikzset{current = american}
}

\DeclareOption{americanresistors}{
    \ctikzset{resistor = american}
}

\DeclareOption{europeanresistors}{
    \ctikzset{resistor = european}
}

\DeclareOption{americaninductors}{
    \ctikzset{inductor = american}
}

\DeclareOption{europeaninductors}{
    \ctikzset{inductor = european}
}

\DeclareOption{cuteinductors}{
    \ctikzset{inductor = cute}
}

\DeclareOption{americanports}{
    \ctikzset{logic ports = american}
}

\DeclareOption{europeanports}{
    \ctikzset{logic ports = european}
}

\DeclareOption{americangfsurgearrester}{
    \ctikzset{gas filled surge arrester choice = american}
}

\DeclareOption{europeangfsurgearrester}{
    \ctikzset{gas filled surge arrester choice = european}
}

\DeclareOption{siunitx}{
    \pgf@circ@siunitxtrue
}

\DeclareOption{nosiunitx}{
    \pgf@circ@siunitxfalse
}

\DeclareOption{fulldiode}{
    \ctikzset{diode = full}
}

\DeclareOption{emptydiode}{
    \ctikzset{diode = empty}
}

\DeclareOption{strokediode}{
    \ctikzset{diode = stroke}
}

\DeclareOption{arrowmos}{
    \pgf@circuit@mos@arrowstrue
}

\DeclareOption{noarrowmos}{
    \pgf@circuit@mos@arrowsfalse
}

\DeclareOption{fetbodydiode}{
    \pgf@circuit@fet@bodydiodetrue
}

\DeclareOption{nofetbodydiode}{
    \pgf@circuit@fet@bodydiodefalse
}

\DeclareOption{fetsolderdot}{
    \pgf@circuit@fet@solderdottrue
}

\DeclareOption{nofetsolderdot}{
    \pgf@circuit@fet@solderdotfalse
}

\DeclareOption{emptypmoscircle}{
    \pgf@circuit@pmos@emptycircletrue
}


\DeclareOption{lazymos}{
    \ctikzset{tripoles/nmos/width=.5}
    \ctikzset{tripoles/nmos/gate height=.35}
    \ctikzset{tripoles/nmos/base height=.35}
    \ctikzset{tripoles/nmos/height/.initial=1.2}
    \ctikzset{tripoles/nmos/base width=.5}
    \ctikzset{tripoles/nmos/gate width=.65}

    \ctikzset{tripoles/pmos/width=.5}
    \ctikzset{tripoles/pmos/gate height=.35}
    \ctikzset{tripoles/pmos/base height=.35}
    \ctikzset{tripoles/pmos/height/.initial=1.2}
    \ctikzset{tripoles/pmos/base width=.5}
    \ctikzset{tripoles/pmos/gate width=.65}

    \pgf@circuit@pmos@emptycircletrue
}

\DeclareOption{straightlabels}{
    \ctikzset{label/align = straight}
}

\DeclareOption{rotatelabels}{
    \ctikzset{label/align = rotate}
}

\DeclareOption{smartlabels}{
    \ctikzset{label/align = smart}
}

\DeclareOption{compatibility}{
    \pgf@circuit@compattrue
}

\DeclareOption{oldvoltagedirection}{
    \pgf@circ@oldvoltagedirectiontrue
    \pgf@circ@explicitvdirtrue
    \pgf@circ@fixbatteriesfalse
}
\DeclareOption{nooldvoltagedirection}{
    \pgf@circ@oldvoltagedirectionfalse
    \pgf@circ@explicitvdirtrue
    \pgf@circ@fixbatteriesfalse
}

\DeclareOption{RPvoltages}{
    \pgf@circ@oldvoltagedirectiontrue
    \pgf@circ@explicitvdirtrue
    \pgf@circ@fixbatteriestrue
}
\DeclareOption{EFvoltages}{
    \pgf@circ@oldvoltagedirectionfalse
    \pgf@circ@explicitvdirtrue
    \pgf@circ@fixbatteriestrue
}

\DeclareOption{legacytransistorstext}{
    \pgf@circuit@transisors@fixlabelsfalse
}

\DeclareOption{nolegacytransistorstext}{
    \pgf@circuit@transisors@fixlabelstrue
}

\DeclareOption{centertransistorstext}{
    \pgf@circuit@transisors@fixlabelstrue
}

\DeclareOption{betterproportions}{
    \ctikzset{monopoles/ground/width/.initial=.15}

    \ctikzset{bipoles/resistor/height/.initial=.23}
    \ctikzset{bipoles/resistor/width/.initial=.6}
    \ctikzset{bipoles/capacitor/height/.initial=.4}
    \ctikzset{bipoles/capacitor/width/.initial=.1}
    \ctikzset{bipoles/potentiometer/height/.initial=.6}
    \ctikzset{bipoles/potentiometer/height 2/.initial=.23}
    \ctikzset{bipoles/potentiometer/width/.initial=.6}
    \ctikzset{bipoles/photoresistor/height/.initial=.6}
    \ctikzset{bipoles/photoresistor/height 2/.initial=.23}
    \ctikzset{bipoles/photoresistor/width/.initial=.6}
    \ctikzset{bipoles/thermistor/main/.initial=.7}
    \ctikzset{bipoles/thermistor/height/.initial=.328}%.23/.7
    \ctikzset{bipoles/thermistor/width/.initial=.6}
    \ctikzset{bipoles/thermistorntc/width/.initial=.6}
    \ctikzset{bipoles/thermistorntc/main/.initial=.7}
    \ctikzset{bipoles/thermistorntc/height/.initial=.328}%.23/.7
    \ctikzset{bipoles/thermistorntc/height 2/.initial=.75}%.23/.7
    \ctikzset{bipoles/thermistorptc/width/.initial=.6}
    \ctikzset{bipoles/thermistorptc/main/.initial=.7}
    \ctikzset{bipoles/thermistorptc/height/.initial=.328}%.23/.7
    \ctikzset{bipoles/varistor/main/.initial=.7}
    \ctikzset{bipoles/varistor/height/.initial=.328}%.23/.7
    \ctikzset{bipoles/varistor/width/.initial=.6}
    \ctikzset{bipoles/vresistor/height/.initial=.45}
    \ctikzset{bipoles/vresistor/width/.initial=.6}
    \ctikzset{bipoles/generic/height/.initial=.23}
    \ctikzset{bipoles/generic/width/.initial=.6}
    \ctikzset{bipoles/generic potentiometer/height/.initial=.6}
    \ctikzset{bipoles/generic potentiometer/height 2/.initial=.23}
    \ctikzset{bipoles/generic potentiometer/width/.initial=.6}
    \ctikzset{bipoles/ageneric/height/.initial=.23}
    \ctikzset{bipoles/ageneric/width/.initial=.6}
    \ctikzset{bipoles/memristor/height/.initial=.23}
    \ctikzset{bipoles/memristor/wave height/.initial=.375}
    \ctikzset{bipoles/memristor/width/.initial=.60}
    \ctikzset{bipoles/tgeneric/height/.initial=.525}
    \ctikzset{bipoles/tgeneric/width/.initial=.6}
    \ctikzset{bipoles/tfullgeneric/height/.initial=.525}
    \ctikzset{bipoles/tfullgeneric/width/.initial=.60}
    \ctikzset{bipoles/fullgeneric/height/.initial=.23}
    \ctikzset{bipoles/fullgeneric/width/.initial=.6}
    \ctikzset{bipoles/diode/height/.initial=.3}
    \ctikzset{bipoles/diode/width/.initial=.25}
    \ctikzset{bipoles/bidirectionaldiode/height/.initial=.66}
    \ctikzset{bipoles/bidirectionaldiode/width/.initial=.6}
    \ctikzset{bipoles/bidirectionaldiode/diode width left/.initial=.3}
    \ctikzset{bipoles/bidirectionaldiode/diode width right/.initial=.3}

    \ctikzset{tripoles/thyristor/height/.initial=.66}
    \ctikzset{tripoles/thyristor/height 2/.initial=.3}
    \ctikzset{tripoles/thyristor/width/.initial=.6}
    \ctikzset{tripoles/thyristor/diode height/.initial=.3}
    \ctikzset{tripoles/thyristor/diode width left/.initial=.4}
    \ctikzset{tripoles/thyristor/diode width right/.initial=.3}

    \ctikzset{tripoles/triac/height/.initial=.66}
    \ctikzset{tripoles/triac/width/.initial=.6}
    \ctikzset{tripoles/triac/diode width left/.initial=.3}
    \ctikzset{tripoles/triac/diode width right/.initial=.3}
}

% This is a nice hack that prints all the shapes declared
% by the package. Very useful for coverage testing and debugging.
%
% \let\origpgfdeclareshape=\pgfdeclareshape
% \def\pgfdeclareshape#1{%
%     \typeout{SHAPE:\space"#1"}%
%     \origpgfdeclareshape{#1}
% }


%%%%%%%%%
%%%%%%%%%%% Springe nach tex/pgfcirc.defines
%%%---------- open: tex/pgfcirc.defines.tex
% Copyright 2018-2020 by Romano Giannetti
% Copyright 2015-2020 by Stefan Lindner
% Copyright 2013-2020 by Stefan Erhardt
% Copyright 2007-2020 by Massimo Redaelli
%
% This file may be distributed and/or modified
%
% 1. under the LaTeX Project Public License and/or
% 2. under the GNU Public License.
%
% See the files gpl-3.0_license.txt and lppl-1-3c_license.txt for more details.

%
%% Key managements

\long\def\pgf@circ@comment#1{}

\def\circuitikzbasekey{/tikz/circuitikz}

\pgfkeys{\circuitikzbasekey/.is family}

\def\circuitikzset#1{\pgfkeys{\circuitikzbasekey,#1}}
\let\ctikzset\circuitikzset
\def\ctikzvalof#1{\pgfkeysvalueof{\circuitikzbasekey/#1}}
\def\ctikzsetvalof#1#2{\pgfkeyssetvalue{\circuitikzbasekey/#1}{#2}}

%% Temporary Counters

\newcount\pgf@circ@count@a
\newcount\pgf@circ@count@b
\newcount\pgf@circ@count@c
%%%%%%%%%%%%
%% Dimensions

% coordinates
\newdimen\pgf@circ@res@up
\newdimen\pgf@circ@res@down
\newdimen\pgf@circ@res@zero
\newdimen\pgf@circ@res@left
\newdimen\pgf@circ@res@right
\newdimen\pgf@circ@res@other
\newdimen\pgf@circ@res@step
\newdimen\pgf@circ@res@temp
% inital thickness
\newdimen \pgfstartlinewidth

% arrow tips, ported over old arrows library (deprecated)
% see https://tex.stackexchange.com/questions/234084/latex-arrow-tip-with-arrows-meta-library
% this was the original definition of latex' tips, renamed to avoid clashes
%
\pgfarrowsdeclare{latexslim}{latexslim}
{
  \pgfutil@tempdima=0.28pt%
  \advance\pgfutil@tempdima by.3\pgflinewidth%
  \pgfarrowsleftextend{+-4\pgfutil@tempdima}
  \pgfarrowsrightextend{+6\pgfutil@tempdima}
}
{
  \pgfutil@tempdima=0.28pt%
  \advance\pgfutil@tempdima by.3\pgflinewidth%
  \pgfpathmoveto{\pgfqpoint{6\pgfutil@tempdima}{0\pgfutil@tempdima}}
  \pgfpathcurveto
  {\pgfqpoint{3.5\pgfutil@tempdima}{.5\pgfutil@tempdima}}
  {\pgfqpoint{-1\pgfutil@tempdima}{1.5\pgfutil@tempdima}}
  {\pgfqpoint{-4\pgfutil@tempdima}{3.75\pgfutil@tempdima}}
  \pgfpathcurveto
  {\pgfqpoint{-1.5\pgfutil@tempdima}{1\pgfutil@tempdima}}
  {\pgfqpoint{-1.5\pgfutil@tempdima}{-1\pgfutil@tempdima}}
  {\pgfqpoint{-4\pgfutil@tempdima}{-3.75\pgfutil@tempdima}}
  \pgfpathcurveto
  {\pgfqpoint{-1\pgfutil@tempdima}{-1.5\pgfutil@tempdima}}
  {\pgfqpoint{3.5\pgfutil@tempdima}{-.5\pgfutil@tempdima}}
  {\pgfqpoint{6\pgfutil@tempdima}{0\pgfutil@tempdima}}
  \pgfpathclose
  \pgfusepathqfill
}

\pgfarrowsdeclarereversed{latexslim reversed}{latexslim reversed}{latexslim}{latexslim}

\def\pgf@circ@setifdefinedfill#1#2{%
    % if \ctikzclass is defined and \ctikzclass/fill is defined and is not none:
    % set the fill color and execute \pgfusepath{#1}, else execute \pgfusepath{#2}
    \ifdefined\ctikzclass
        \pgfkeysifdefined{\circuitikzbasekey/\ctikzclass/fill}%
        {% yes, it's defined
            \edef\@@tmp{\ctikzvalof{\ctikzclass/fill}}\edef\@@none{none}%
            \ifx\@@tmp\@@none % but it's none
                \pgfusepath{#2}%
            \else
                \pgfsetfillcolor{\@@tmp}%
                \pgfusepath{#1}%
            \fi
        }{% the class is defined but the fill key not; use  #2
            \pgfusepath{#2}%
        }
    \else
    \pgfusepath{#2}%
    \fi
}

%% Macro to fill or draw

\def\pgf@circ@draworfill{%
    \ifx\tikz@fillcolor\pgfutil@empty
        \pgf@circ@setifdefinedfill{draw,fill}{draw}
    \else
        \pgfsetfillcolor{\tikz@fillcolor}
        \pgfusepath{draw, fill}
    \fi
}

\def\pgf@circ@draworfillandclip{%
    \ifx\tikz@fillcolor\pgfutil@empty
        \pgf@circ@setifdefinedfill{draw, clip, fill}{draw, clip}
        \pgfusepath{draw, clip}
    \else
        \pgfsetfillcolor{\tikz@fillcolor}
        \pgfusepath{draw, clip, fill}
    \fi
}

\def\pgf@circ@maybefill{%
    \ifx\tikz@fillcolor\pgfutil@empty
        \pgf@circ@setifdefinedfill{fill}{discard}
    \else
        \pgfsetfillcolor{\tikz@fillcolor}
        \pgfusepath{fill}
    \fi
}

%% Macros for setting linewidth
% #1 is the legacy class (bipoles, tripoles) etc
% #2 is the reference linewidth
\def\pgf@circ@setlinewidth#1#2{%
    \ifdefined\ctikzclass
        \pgfkeysifdefined{\circuitikzbasekey/\ctikzclass/thickness}%
        {% yes, it's defined
        \edef\@@tmp{\ctikzvalof{\ctikzclass/thickness}}\edef\@@none{none}%
        \ifx\@@tmp\@@none % but it's none
            \pgfsetlinewidth{\ctikzvalof{#1/thickness}#2}% passthrough legacy class
        \else
            \pgfsetlinewidth{\@@tmp #2}%
        \fi
        }{ % key not defined, do the legacy thing
            \pgfsetlinewidth{\ctikzvalof{#1/thickness}#2}%
        }% Ok, do nothing
    \else % no class
        \pgfsetlinewidth{\ctikzvalof{#1/thickness}#2}%
    \fi
}
% use \pgf@circ@setlinewidth{none}{\pgflinewidth} if there is no legacy case
\ctikzset{none/thickness/.initial=1.0} % do not touch

%% font changes compatible with plain/LaTeX/ConTeXt
%% thanks to Henri Menke https://github.com/circuitikz/circuitikz/issues/285#issuecomment-537224605

\ifpgfutil@format@is@latex
    \long\def\pgf@circ@font@tiny{\tiny}
    \long\def\pgf@circ@font@small{\small}
    \long\def\pgf@circ@font@bold{\textbf}
    \long\def\pgf@circ@font@boldmath{\boldmath}
    \long\def\pgf@circ@font@sixbm{\fontsize{6}{7}\selectfont\boldmath}
    \long\def\pgf@circ@font@tenbm{\fontsize{10}{12}\selectfont\boldmath}
    \long\def\pgf@circ@font@twelve{\fontsize{12}{14}\selectfont}
\else\ifpgfutil@format@is@plain
    \long\def\pgf@circ@font@tiny{\fiverm}
    \long\def\pgf@circ@font@small{\sevenrm}
    \long\def\pgf@circ@font@bold#1{{\bf#1}}
    \long\def\pgf@circ@font@boldmath{\bf}       % to be tested
    \long\def\pgf@circ@font@sixbm{\sevenrm\bf}  %
    \long\def\pgf@circ@font@tenbm{\tenrm\bf}    %
    \long\def\pgf@circ@font@twelve{\twelverm}   %
\else\ifpgfutil@format@is@context
    \long\def\pgf@circ@font@tiny{\tfxx}
    \long\def\pgf@circ@font@small{\tfx}
    \long\def\pgf@circ@font@bold{\bold}
    \long\def\pgf@circ@font@boldmath{\bold}    % to be tested
    \long\def\pgf@circ@font@sixbm{\tfx\bold}   %
    \long\def\pgf@circ@font@tenbm{\normal\bold}%
    \long\def\pgf@circ@font@twelve{\tfa}       %
\fi\fi\fi


%
% Thanks to Phelype Oleinik https://tex.stackexchange.com/a/520806/38080
%
% this is needed to avoid problems with \ConTeXt
\ifcsname normalunexpanded\endcsname
  \let\pgfcircutil@unexpanded\normalunexpanded
\else
  \let\pgfcircutil@unexpanded\unexpanded
\fi
% minimally expand a pgfkey to check if it's {}/undefined or filled.
\def\unexpandedvalueof#1{%
  \pgfcircutil@unexpanded\expandafter\expandafter
    \expandafter\pgf@circ@valueof@chk\pgfkeysvalueof{#1}}
\def\pgf@circ@valueof@chk#1{%
  \ifx\relax#1%
    \expandafter\pgfutil@firstoftwo
  \else
    \expandafter\pgfutil@secondoftwo
  \fi
    {{}}% #1 is \relax, so consider empty
    {\expandafter{#1}}% otherwise, leave the key after one more expansion
}%
%
% text position (added with ieeestd logic ports. Maybe to be extended to other components,
% like amplifiers)
%
\newif\ifpgf@circ@center@text\pgf@circ@center@texttrue
\tikzset{component text/.is choice}%
\tikzset{component text/center/.code={\pgf@circ@center@texttrue}}%
\tikzset{component text/left/.code={\pgf@circ@center@textfalse}}%
\ctikzset{component text/.is choice}%
\ctikzset{component text/center/.code={\pgf@circ@center@texttrue}}%
\ctikzset{component text/left/.code={\pgf@circ@center@textfalse}}%
\ctikzset{left text distance/.initial=0.3em}%

% voltage options

\newif\ifpgf@circ@siunitx
\newif\ifpgf@circuit@compat
\newif\ifpgf@circ@oldvoltagedirection % default false
\newif\ifpgf@circ@explicitvdir
\newif\ifpgf@circ@fixbatteries

\ctikzset{voltage dir/.is choice}
\ctikzset{voltage dir/old/.code={\pgf@circ@oldvoltagedirectiontrue\pgf@circ@fixbatteriesfalse}}
\ctikzset{voltage dir/noold/.code={\pgf@circ@oldvoltagedirectionfalse\pgf@circ@fixbatteriesfalse}}
\ctikzset{voltage dir/RP/.code={\pgf@circ@oldvoltagedirectiontrue\pgf@circ@fixbatteriestrue}}
\ctikzset{voltage dir/EF/.code={\pgf@circ@oldvoltagedirectionfalse\pgf@circ@fixbatteriestrue}}
\tikzset{voltage dir/.style={circuitikz/voltage dir=#1}}

% Option ">" for twoports
\newif\ifpgf@circuit@inputarrow
\ctikzset{>/.add code={}{\pgf@circuit@inputarrowtrue}}
\ctikzset{inputarrow/.is choice}
\ctikzset{inputarrow/true/.code={\pgf@circuit@inputarrowtrue}}
\ctikzset{inputarrow/false/.code={\pgf@circuit@inputarrowfalse}}

% Option "boxed" for nodes and twoports
\newif\ifpgf@circuit@boxed
\pgfkeys{/tikz/boxed/.add code={}{\pgf@circuit@boxedtrue}}
\ctikzset{boxed/.add code={}{\pgf@circuit@boxedtrue}}
\pgfkeys{/tikz/box/.add code={}{\pgf@circuit@boxedtrue}}
\ctikzset{box/.add code={}{\pgf@circuit@boxedtrue}}

% Option "dashed" for nodes and twoports
\newif\ifpgf@circuit@dashed
\pgfkeys{/tikz/dashed/.add code={}{\pgf@circuit@dashedtrue}}
\ctikzset{dashed/.add code={}{\pgf@circuit@dashedtrue}}

% Option "t=*" for nodes
\pgfkeys{/tikz/t/.add code={}{\ctikzset{text=#1}}}

\ctikzset{current arrow scale/.initial=16}

\ctikzset{bipoles/.is family}
\ctikzset{bipoles/border margin/.initial=1.1}
\ctikzset{bipoles/thickness/.initial=2}
\ctikzset{tripoles/thickness/.initial=2}
\ctikzset{quadpoles/thickness/.initial=2}
\ctikzset{nodes width/.initial=.04}

% Base len for all circuitikz
\newdimen\pgf@circ@Rlen
% scaled length for internal use in scalable shapes
\newdimen\pgf@circ@scaled@Rlen
\ctikzset{bipoles/length/.code={\pgf@circ@Rlen = #1\pgf@circ@scaled@Rlen=\pgf@circ@Rlen}}
\pgf@circ@Rlen = 1.4cm
% by default scale is 1.0
\pgf@circ@scaled@Rlen=\pgf@circ@Rlen

%%%%%%%%%%%%%%%%%%%%%%%%%%%%
%% main style definitions
%%%%%%%%%%%%%%%%%%%%%%%%%%%%



% load a style file: search ctikzstyle-NAME.tex in path
\def\ctikzloadstyle#1{%
    \pgfutil@InputIfFileExists{ctikzstyle-#1}{}{%
        \tikzerror{I did not find the circuitikz style #1}}%
}

% load and enact a style
\def\ctikzsetstyle#1{%
    \ctikzloadstyle{#1}%
    \tikzset{#1 circuit style}%
}

% global style parameters
\ctikzset{default/scale/.initial=1.0}   % do not touch
\ctikzset{default/fill/.initial=none}   % do not touch
\ctikzset{default/thickness/.initial=none}   % do not touch
% default is use as the generic default style for bipoles

% mostly bipoles:

\ctikzset{resistors/scale/.initial=1.0}
\ctikzset{resistors/fill/.initial=none}
\ctikzset{resistors/thickness/.initial=none}

\ctikzset{capacitors/scale/.initial=1.0}
\ctikzset{capacitors/fill/.initial=none}
\ctikzset{capacitors/thickness/.initial=none}

\ctikzset{inductors/scale/.initial=1.0}
\ctikzset{inductors/fill/.initial=none}
\ctikzset{inductors/thickness/.initial=none}

\ctikzset{diodes/scale/.initial=1.0}
\ctikzset{diodes/fill/.initial=none}
\ctikzset{diodes/thickness/.initial=none}

\ctikzset{batteries/scale/.initial=1.0}
\ctikzset{batteries/fill/.initial=none}
\ctikzset{batteries/thickness/.initial=none}

\ctikzset{sources/scale/.initial=1.0}
\ctikzset{sources/fill/.initial=none}
\ctikzset{sources/thickness/.initial=none}

\ctikzset{csources/scale/.initial=1.0}
\ctikzset{csources/fill/.initial=none}
\ctikzset{csources/thickness/.initial=none}

\ctikzset{instruments/scale/.initial=1.0}
\ctikzset{instruments/fill/.initial=none}
\ctikzset{instruments/thickness/.initial=none}

\ctikzset{mechanicals/scale/.initial=1.0}
\ctikzset{mechanicals/fill/.initial=none}
\ctikzset{mechanicals/thickness/.initial=none}

\ctikzset{misc/scale/.initial=1.0}
\ctikzset{misc/fill/.initial=none}
\ctikzset{misc/thickness/.initial=none}

\ctikzset{blocks/scale/.initial=1.0}
\ctikzset{blocks/fill/.initial=none}
\ctikzset{blocks/thickness/.initial=none}

% mostly nodes

\ctikzset{grounds/scale/.initial=1.0}
\ctikzset{grounds/fill/.initial=none}
\ctikzset{grounds/thickness/.initial=none}

\ctikzset{power supplies/scale/.initial=1.0}
\ctikzset{power supplies/fill/.initial=none}
\ctikzset{power supplies/thickness/.initial=none}

\ctikzset{transistors/scale/.initial=1.0}
\ctikzset{transistors/fill/.initial=none}
\ctikzset{transistors/thickness/.initial=none}

\ctikzset{tubes/scale/.initial=1.0}
\ctikzset{tubes/fill/.initial=none}
\ctikzset{tubes/thickness/.initial=none}

\ctikzset{RF/scale/.initial=1.0}
\ctikzset{RF/fill/.initial=none}
\ctikzset{RF/thickness/.initial=none}

\ctikzset{electromechanicals/scale/.initial=1.0}
\ctikzset{electromechanicals/fill/.initial=none}
\ctikzset{electromechanicals/thickness/.initial=none}

% transformers go with inductors
\ctikzset{amplifiers/scale/.initial=1.0}
\ctikzset{amplifiers/fill/.initial=none}
\ctikzset{amplifiers/thickness/.initial=none}

\ctikzset{switches/scale/.initial=1.0}
\ctikzset{switches/fill/.initial=none}
\ctikzset{switches/thickness/.initial=none}

\ctikzset{logic ports/scale/.initial=1.0}
\ctikzset{logic ports/fill/.initial=none}
\ctikzset{logic ports/thickness/.initial=none}

\ctikzset{flipflops/scale/.initial=1.0}
\ctikzset{flipflops/fill/.initial=none}
\ctikzset{flipflops/thickness/.initial=none}

\ctikzset{muxdemuxes/scale/.initial=1.0}
\ctikzset{muxdemuxes/fill/.initial=none}
\ctikzset{muxdemuxes/thickness/.initial=none}

\ctikzset{chips/scale/.initial=1.0}
\ctikzset{chips/fill/.initial=none}
\ctikzset{chips/thickness/.initial=none}

\ctikzset{displays/scale/.initial=1.0}
\ctikzset{displays/fill/.initial=none}
\ctikzset{displays/thickness/.initial=none}

% grounds and power supplies

\ctikzset{monopoles/.is family}
\ctikzset{monopoles/ground/width/.initial=.25}
\ctikzset{monopoles/ground/connectionthickness/.initial=1}
\ctikzset{monopoles/ground/thickness/.initial=2}
\ctikzset{monopoles/rground/thickness/.initial=2}
\ctikzset{monopoles/tground/thickness/.initial=3}
\ctikzset{monopoles/vcc/width/.initial=.2}
\ctikzset{monopoles/vcc/arrow/.initial=legacy}
\ctikzset{monopoles/vee/arrow/.initial=legacy}
\ctikzset{monopoles/match/width/.initial=.4}
\ctikzset{monopoles/chassis/width/.initial=.25}
\ctikzset{monopoles/alternative chassis/width/.initial=.25}
\ctikzset{monopoles/equipotentiality/width/.initial=.25}
\ctikzset{monopoles/antenna/width/.initial=.25}
\ctikzset{monopoles/antenna/label/xanchor/.initial=.4}
\ctikzset{monopoles/antenna/label/yanchor/.initial=.75}
\ctikzset{monopoles/txantenna/label/xanchor/.initial=.4}
\ctikzset{monopoles/txantenna/label/yanchor/.initial=.75}
\ctikzset{monopoles/txantenna/width/.initial=.25}
\ctikzset{monopoles/rxantenna/label/xanchor/.initial=.4}
\ctikzset{monopoles/rxantenna/label/yanchor/.initial=.75}
\ctikzset{monopoles/rxantenna/width/.initial=.25}
\ctikzset{monopoles/bareantenna/width/.initial=.25}
\ctikzset{monopoles/bareantenna/label/xanchor/.initial=1}
\ctikzset{monopoles/bareantenna/label/yanchor/.initial=0.5}
\ctikzset{monopoles/waves/width/.initial=0.5}

% microstrips
\ctikzset{bipoles/mstline/height/.initial=0.3}
\ctikzset{bipoles/mstline/width/.initial=1.2}
\pgfkeys{/tikz/mstlinelen/.add code={}{\ctikzset{bipoles/mstline/width=#1}}}
\ctikzset{monopoles/msport/width/.initial=.5}
\ctikzset{monopoles/msrstub/height/.initial=1.0}
\ctikzset{monopoles/msrstub/width/.initial=0.6}

% microphone and loudspeaker
\ctikzset{bipoles/loudspeaker/height/.initial=.8}
\ctikzset{bipoles/loudspeaker/depth/.initial=.3}
\ctikzset{bipoles/loudspeaker/width/.initial=.8}
\ctikzset{bipoles/mic/height/.initial=1.2}
\ctikzset{bipoles/mic/depth/.initial=.0}
\ctikzset{bipoles/mic/width/.initial=.8}

% Zig Zag resistors
\ctikzset{resistors/zigs/.initial=3}
\ctikzset{resistors/width/.code={%
    \ctikzset{bipoles/resistor/width=#1}%
    \ctikzset{bipoles/vresistor/width=#1}%
    \ctikzset{bipoles/potentiometer/width=#1}%
    \ctikzset{bipoles/resistivesens/width=#1}%
    \ctikzset{bipoles/photoresistor/width=#1}%
    \ctikzset{bipoles/thermistor/width=#1}%
    \ctikzset{bipoles/thermistorntc/width=#1}%
    \ctikzset{bipoles/thermistorptc/width=#1}%
    \ctikzset{bipoles/varistor/width=#1}%
    \ctikzset{bipoles/generic/width=#1}%
    \ctikzset{bipoles/generic potentiometer/width=#1}%
    \ctikzset{bipoles/ageneric/width=#1}%
    \ctikzset{bipoles/tgeneric/width=#1}%
}}
\ctikzset{wiper pos/.code={%
    \ctikzset{bipoles/potentiometer/wiper pos=#1}%
    \ctikzset{bipoles/generic potentiometer/wiper pos=#1}%
}}
% zigzag resistor
\ctikzset{bipoles/resistor/height/.initial=.3}
\ctikzset{bipoles/resistor/width/.initial=.8}
\ctikzset{bipoles/potentiometer/height/.initial=.8}
\ctikzset{bipoles/potentiometer/height 2/.initial=.3}
\ctikzset{bipoles/potentiometer/width/.initial=.8}
\ctikzset{bipoles/potentiometer/wiper pos/.initial=.5}
\ctikzset{bipoles/vresistor/height/.initial=.6}
\ctikzset{bipoles/vresistor/width/.initial=.8}
\ctikzset{bipoles/resistivesens/height/.initial=.6}
\ctikzset{bipoles/resistivesens/width/.initial=.8}
% square resistors
\ctikzset{bipoles/photoresistor/height/.initial=.6}
\ctikzset{bipoles/photoresistor/height 2/.initial=.3}
\ctikzset{bipoles/photoresistor/width/.initial=.8}
\ctikzset{bipoles/thermistor/main/.initial=.7}
\ctikzset{bipoles/thermistor/height/.initial=.428}%.3/.7
\ctikzset{bipoles/thermistorntc/width/.initial=.8}
\ctikzset{bipoles/thermistorntc/main/.initial=.7}
\ctikzset{bipoles/thermistorntc/height/.initial=.428}%.3/.7
\ctikzset{bipoles/thermistorntc/height 2/.initial=.75}%.3/.7
\ctikzset{bipoles/thermistorptc/width/.initial=.8}
\ctikzset{bipoles/thermistorptc/main/.initial=.7}
\ctikzset{bipoles/thermistorptc/height/.initial=.428}%.3/.7
\ctikzset{bipoles/thermistorptc/height 2/.initial=.75}%.3/.7
\ctikzset{bipoles/thermistor/width/.initial=.8}
\ctikzset{bipoles/varistor/main/.initial=.7}
\ctikzset{bipoles/varistor/height/.initial=.428}%.3/.7
\ctikzset{bipoles/varistor/width/.initial=.8}
\ctikzset{bipoles/generic/height/.initial=.30}
\ctikzset{bipoles/generic/width/.initial=.80}
\ctikzset{bipoles/generic potentiometer/height/.initial=.80}
\ctikzset{bipoles/generic potentiometer/height 2/.initial=.30}
\ctikzset{bipoles/generic potentiometer/width/.initial=.80}
\ctikzset{bipoles/generic potentiometer/wiper pos/.initial=.5}
\ctikzset{bipoles/ageneric/height/.initial=.30}
\ctikzset{bipoles/tgeneric/height/.initial=.70}
\ctikzset{bipoles/tgeneric/width/.initial=.80}
\ctikzset{bipoles/ageneric/width/.initial=.80}

%% Capacitors
\ctikzset{bipoles/capacitor/height/.initial=.6}
\ctikzset{bipoles/capacitor/width/.initial=.2}
\ctikzset{bipoles/ecapacitor/height/.initial=.5}
\ctikzset{bipoles/ecapacitor/width/.initial=.2}
\ctikzset{bipoles/ecapacitor/font/.initial=\pgf@circ@font@sixbm}
%%% pcapacitor is deprecated
\ctikzset{bipoles/pcapacitor/height/.initial=.6}
\ctikzset{bipoles/pcapacitor/width/.initial=.2}
\ctikzset{bipoles/pcapacitor/bend width/.initial=1.1}
\ctikzset{bipoles/ccapacitor/height/.initial=.6}
\ctikzset{bipoles/ccapacitor/width/.initial=.12}
\ctikzset{bipoles/ccapacitor/bend width/.initial=1.1}
\ctikzset{bipoles/vcapacitor/height/.initial=.6}
\ctikzset{bipoles/vcapacitor/width/.initial=.5}
\ctikzset{bipoles/vcapacitor/capacitor width/.initial=.4}
\ctikzset{bipoles/piezoelectric/height/.initial=.7}
\ctikzset{bipoles/piezoelectric/width/.initial=.5}

% Inductors
\ctikzset{inductors/coils/.code={%
    \ctikzset{bipoles/cuteinductor/coils=#1}%
    \ctikzset{bipoles/cutechoke/coils=#1}%
    \ctikzset{bipoles/americaninductor/coils=#1}%
    \ctikzset{bipoles/vcuteinductor/coils=#1}%
    \ctikzset{bipoles/vamericaninductor/coils=#1}%
}}
\ctikzset{inductors/width/.code={%
    \ctikzset{bipoles/cuteinductor/width=#1}%
    \ctikzset{bipoles/cutechoke/width=#1}%
    \ctikzset{bipoles/americaninductor/width=#1}%
    \ctikzset{bipoles/vcuteinductor/width=#1}%
    \ctikzset{bipoles/vamericaninductor/width=#1}%
    \ctikzset{bipoles/fullgeneric/width=#1}%
    \ctikzset{bipoles/tfullgeneric/width=#1}%
}}
\ctikzset{bipoles/cuteinductor/height/.initial=.3}
\ctikzset{bipoles/cuteinductor/lower coil height/.initial=.15}
\ctikzset{bipoles/cuteinductor/width/.initial=.6}
\ctikzset{bipoles/cuteinductor/coils/.initial=5}
\ctikzset{bipoles/cuteinductor/coil aspect/.initial=.5}%percentage of inductor width, which is covered by lower coil
%% Cute choke settings
\ctikzset{bipoles/cutechoke/height/.initial=.3}
\ctikzset{bipoles/cutechoke/lower coil height/.initial=.15}
\ctikzset{bipoles/cutechoke/width/.initial=.6}
\ctikzset{bipoles/cutechoke/coils/.initial=5}
\ctikzset{bipoles/cutechoke/coil aspect/.initial=.5}%percentage of choke width, which is covered by lower coil
\ctikzset{bipoles/cutechoke/cstep/.initial=.3}
\ctikzset{bipoles/cutechoke/cdist/.initial=1.3}
\ctikzset{bipoles/cutechoke/cthick/.initial=1}
\newif\ifpgf@circuit@bipole@twolineschoke
\pgf@circuit@bipole@twolineschokefalse
\pgfkeys{/tikz/onelinechoke/.add code={}{\pgf@circuit@bipole@twolineschokefalse}}
\ctikzset{onelinechoke/.add code={}{\pgf@circuit@bipole@twolineschokefalse}}
\pgfkeys{/tikz/twolineschoke/.add code={}{\pgf@circuit@bipole@twolineschoketrue}}
\ctikzset{twolineschoke/.add code={}{\pgf@circuit@bipole@twolineschoketrue}}
%
\ctikzset{bipoles/americaninductor/height/.initial=.3}
\ctikzset{bipoles/americaninductor/height 2/.initial=.1}
\ctikzset{bipoles/americaninductor/width/.initial=.8}
\ctikzset{bipoles/americaninductor/coils/.initial=4}
\ctikzset{bipoles/americaninductor/coil height/.initial=.15}
\ctikzset{bipoles/vcuteinductor/height/.initial=.6}
\ctikzset{bipoles/vcuteinductor/lower coil height/.initial=.3}
\ctikzset{bipoles/vcuteinductor/width/.initial=.6}
\ctikzset{bipoles/vcuteinductor/coils/.initial=5}
\ctikzset{bipoles/vcuteinductor/coil aspect/.initial=.5}%percentage of inductor width, which is covered by lower coil
\ctikzset{bipoles/vamericaninductor/height/.initial=.6}
\ctikzset{bipoles/vamericaninductor/height 2/.initial=.2}
\ctikzset{bipoles/vamericaninductor/width/.initial=.8}
\ctikzset{bipoles/vamericaninductor/coils/.initial=4}
\ctikzset{bipoles/vamericaninductor/coil height/.initial=.15}
\ctikzset{bipoles/tfullgeneric/height/.initial=.70}
\ctikzset{bipoles/tfullgeneric/width/.initial=.80}
\ctikzset{bipoles/fullgeneric/height/.initial=.30}
\ctikzset{bipoles/fullgeneric/width/.initial=.80}

% Sources

\ctikzset{bipoles/esource/height/.initial=.60}
\ctikzset{bipoles/esource/width/.initial=.60}
\ctikzset{bipoles/pvsource/height/.initial=.60}
\ctikzset{bipoles/pvsource/width/.initial=.60}
\ctikzset{bipoles/isource/height/.initial=.60}
\ctikzset{bipoles/isource/width/.initial=.60}
\ctikzset{bipoles/oosource/height/.initial=.60}
\ctikzset{bipoles/oosource/width/.initial=.60}
\ctikzset{bipoles/oosource/circlesize/.initial=.65}%circlesize+circleoffset should be =1
\ctikzset{bipoles/oosource/circleoffset/.initial=.35}%circlesize+circleoffset should be =1
\ctikzset{bipoles/dcisource/angle/.initial=80}
\ctikzset{bipoles/dcisource/height/.initial=.60}
\ctikzset{bipoles/dcisource/width/.initial=.60}
\ctikzset{bipoles/dcvsource/height/.initial=.60}
\ctikzset{bipoles/dcvsource/width/.initial=.60}
\ctikzset{bipoles/vsourcetri/height/.initial=.60}
\ctikzset{bipoles/vsourcetri/width/.initial=.60}
\ctikzset{bipoles/isourceam/height/.initial=.60}
\ctikzset{bipoles/isourceam/width/.initial=.60}
\ctikzset{bipoles/vsource/height/.initial=.60}
\ctikzset{bipoles/vsource/width/.initial=.60}
\ctikzset{bipoles/vsourceam/height/.initial=.60}
\ctikzset{bipoles/vsourceam/width/.initial=.60}
\ctikzset{bipoles/vsourceam/margin/.initial=.7}
\ctikzset{bipoles/isourcesin/height/.initial=.60}
\ctikzset{bipoles/isourcesin/width/.initial=.60}
\ctikzset{bipoles/vsourcesin/height/.initial=.60}
\ctikzset{bipoles/vsourcesin/width/.initial=.60}
\ctikzset{bipoles/vsourcesquare/height/.initial=.60}
\ctikzset{bipoles/vsourcesquare/width/.initial=.60}
\ctikzset{bipoles/cisource/height/.initial=.7}
\ctikzset{bipoles/cisource/width/.initial=.7}
\ctikzset{bipoles/cisourceam/height/.initial=.7}
\ctikzset{bipoles/cisourceam/width/.initial=.7}
\ctikzset{bipoles/ecsource/height/.initial=.7}
\ctikzset{bipoles/ecsource/width/.initial=.7}
\ctikzset{bipoles/cvsource/height/.initial=.7}
\ctikzset{bipoles/cvsource/width/.initial=.7}
\ctikzset{bipoles/cvsourceam/height/.initial=.7}
\ctikzset{bipoles/cvsourceam/width/.initial=.7}
\ctikzset{bipoles/cvsourceam/margin/.initial=.7}
\ctikzset{bipoles/cvsourceam/text scale/.initial=1}
\ctikzset{bipoles/cisourcesin/width/.initial=.7}
\ctikzset{bipoles/cvsourcesin/height/.initial=.7}
\ctikzset{bipoles/cvsourcesin/width/.initial=.7}
\ctikzset{bipoles/battery/height/.initial=.6}
\ctikzset{bipoles/battery/width/.initial=.3}
\ctikzset{bipoles/battery1/height/.initial=.6}
\ctikzset{bipoles/battery1/width/.initial=.3}
\ctikzset{bipoles/battery2/height/.initial=.6}
\ctikzset{bipoles/battery2/width/.initial=.3}

% % % ootransformer
\ctikzset{bipoles/oosourcetrans/height/.initial=.6}
\ctikzset{bipoles/oosourcetrans/width/.initial=.6}
\ctikzset{bipoles/oosourcetrans/circlesize/.initial=.6}%circlesize+circleoffset should be =1
\ctikzset{bipoles/oosourcetrans/circleoffset/.initial=.4}%circlesize+circleoffset should be =1
\ctikzset{bipoles/oosourcetrans/vectorgroupscale/.initial=1}

% % % oootransformer
\ctikzset{bipoles/ooosource/height/.initial=.6}
\ctikzset{bipoles/ooosource/circlesize/.initial=.55}%circlesize+circleoffset should be =1
\ctikzset{bipoles/ooosource/circleoffset/.initial=.45}%circlesize+circleoffset should be =1
\ctikzset{bipoles/ooosource/vectorgroupscale/.initial=1}

% % % primary windings
\newif\ifpgf@circ@prim@delta
\newif\ifpgf@circ@prim@wye
\newif\ifpgf@circ@prim@zig
\pgfkeys{tikz/prim/.is choice}
\pgfkeys{tikz/prim/delta/.add code={}{\pgf@circ@prim@deltatrue}}
\pgfkeys{tikz/prim/wye/.add code={}{\pgf@circ@prim@wyetrue}}
\pgfkeys{tikz/prim/zig/.add code={}{\pgf@circ@prim@zigtrue}}

% % % secondary windings
\newif\ifpgf@circ@sec@delta
\newif\ifpgf@circ@sec@wye
\newif\ifpgf@circ@sec@zig
\pgfkeys{tikz/sec/.is choice}
\pgfkeys{tikz/sec/delta/.add code={}{\pgf@circ@sec@deltatrue}}
\pgfkeys{tikz/sec/wye/.add code={}{\pgf@circ@sec@wyetrue}}
\pgfkeys{tikz/sec/zig/.add code={}{\pgf@circ@sec@zigtrue}}

% % % tertiary windings (ooosource)
\newif\ifpgf@circ@tert@delta
\newif\ifpgf@circ@tert@wye
\newif\ifpgf@circ@tert@zig
\pgfkeys{tikz/tert/.is choice}
\pgfkeys{tikz/tert/delta/.add code={}{\pgf@circ@tert@deltatrue}}
\pgfkeys{tikz/tert/wye/.add code={}{\pgf@circ@tert@wyetrue}}
\pgfkeys{tikz/tert/zig/.add code={}{\pgf@circ@tert@zigtrue}}

% diodes

\ctikzset{bipoles/diode/height/.initial=.50}
\ctikzset{bipoles/diode/width/.initial=.40}
\ctikzset{bipoles/bidirectionaldiode/height/.initial=1.1}
\ctikzset{bipoles/bidirectionaldiode/width/.initial=1}
\ctikzset{bipoles/bidirectionaldiode/diode width left/.initial=.3}
\ctikzset{bipoles/bidirectionaldiode/diode width right/.initial=.3}
\ctikzset{bipoles/varcap/height/.initial=.50}
\ctikzset{bipoles/varcap/width/.initial=.45}

% switches

\ctikzset{bipoles/spst/height/.initial=.35}
\ctikzset{bipoles/spst/width/.initial=.35}
\ctikzset{bipoles/spst/depth/.initial=.2}
\ctikzset{bipoles/nos/height/.initial=.3}
\ctikzset{bipoles/nos/width/.initial=.35}
\ctikzset{bipoles/nos/depth/.initial=.2}
\ctikzset{bipoles/ncs/height/.initial=.35}
\ctikzset{bipoles/ncs/width/.initial=.35}
\ctikzset{bipoles/ncs/depth/.initial=.2}
%%% reed switch
\ctikzset{bipoles/reed/height/.initial=.4}
\ctikzset{bipoles/reed/width/.initial=.8}% 0.35 in nos
\ctikzset{bipoles/reed/depth/.initial=.4}

% misc

\ctikzset{bipoles/european gas filled surge arrester/height/.initial=.30}
\ctikzset{bipoles/european gas filled surge arrester/width/.initial=.80}
\ctikzset{bipoles/european gas filled surge arrester/inside/.initial=.30}
\ctikzset{bipoles/american gas filled surge arrester/height/.initial=.60}
\ctikzset{bipoles/american gas filled surge arrester/width/.initial=.60}
\ctikzset{bipoles/american gas filled surge arrester/inside/.initial=.15}
\ctikzset{bipoles/american gas filled surge arrester/dot x/.initial=.25}
\ctikzset{bipoles/american gas filled surge arrester/dot y/.initial=.45}
\ctikzset{bipoles/american gas filled surge arrester/size/.initial=.1}
\ctikzset{bipoles/fuse/height/.initial=.20}
\ctikzset{bipoles/fuse/width/.initial=.50}
\ctikzset{bipoles/afuse/height/.initial=.20}
\ctikzset{bipoles/afuse/width/.initial=.50}
\ctikzset{bipoles/memristor/height/.initial=.30}
\ctikzset{bipoles/memristor/wave height/.initial=.5}
\ctikzset{bipoles/memristor/width/.initial=.80}
\ctikzset{bipoles/open/height/.initial=.3} %necessary for curly voltages
\ctikzset{bipoles/open/width/.initial=.3} %necessary for curly voltages
\ctikzset{bipoles/open/voltage/straight label distance/.initial=0}
\ctikzset{bipoles/open/voltage/distance from node/.initial=.2}
\ctikzset{bipoles/short/height/.initial=0} %dummy height for voltage positioning
\ctikzset{bipoles/short/width/.initial=0} %dummy width for voltage positioning
%\ctikzset{bipoles/short/voltage/straight label distance/.initial=.2}
%\ctikzset{bipoles/short/voltage/distance from node/.initial=.5}
% multiwire
\ctikzset{bipoles/multiwire/height/.initial=0.4}
\ctikzset{bipoles/multiwire/width/.initial=0.2}
\ctikzset{bipoles/multiwire/spacing/.initial=0.05}

\ctikzset{bipoles/ammeter/height/.initial=.60}
\ctikzset{bipoles/ammeter/width/.initial=.60}
\ctikzset{bipoles/ohmmeter/height/.initial=.60}
\ctikzset{bipoles/ohmmeter/width/.initial=.60}
\ctikzset{bipoles/voltmeter/height/.initial=.60}
\ctikzset{bipoles/voltmeter/width/.initial=.60}
\ctikzset{bipoles/smeter/height/.initial=.60}
\ctikzset{bipoles/smeter/width/.initial=.60}
\ctikzset{bipoles/smeter/voltage/additional label shift/.initial=1}
\ctikzset{bipoles/qmeter/depth/.initial=.40}
\ctikzset{bipoles/qmeter/height/.initial=.80}
\ctikzset{bipoles/qmeter/width/.initial=.60}
% this must be specified for each one
\ctikzset{bipoles/qvprobe/voltage/additional label shift/.initial=.5}
\ctikzset{bipoles/qiprobe/voltage/additional label shift/.initial=.5}
\ctikzset{bipoles/qpprobe/voltage/additional label shift/.initial=.5}
\ctikzset{bipoles/iloop/width/.initial=.40}
\ctikzset{bipoles/iloop/height/.initial=.60}

\ctikzset{bipoles/oscope/height/.initial=.60}
\ctikzset{bipoles/oscope/width/.initial=.60}
\ctikzset{bipoles/oscope/voltage/additional label shift/.initial=1}


% option to not rotate the new (Romano's) instruments
\newif\ifpgf@circuit@straightinstruments\pgf@circuit@straightinstrumentstrue
\pgfkeys{/tikz/straight instruments/.add code={}{\pgf@circuit@straightinstrumentstrue}}
\ctikzset{straight instruments/.add code={}{\pgf@circuit@straightinstrumentstrue}}
\pgfkeys{/tikz/rotated instruments/.add code={}{\pgf@circuit@straightinstrumentsfalse}}
\ctikzset{rotated instruments/.add code={}{\pgf@circuit@straightinstrumentsfalse}}


\ctikzset{bipoles/buffer/height/.initial=1}
\ctikzset{bipoles/buffer/width/.initial=1}
\ctikzset{bipoles/not port/width/.initial=1}
\ctikzset{bipoles/not port/height/.initial=.8}
\ctikzset{bipoles/not port/circle width/.initial=.15}
\ctikzset{bipoles/lamp/height/.initial=.60}
\ctikzset{bipoles/lamp/width/.initial=.60}
\ctikzset{bipoles/bulb/height/.initial=.8}
\ctikzset{bipoles/bulb/width/.initial=.8}
\ctikzset{bipoles/tline/height/.initial=.3}
\ctikzset{bipoles/tline/width/.initial=.8}
\ctikzset{bipoles/squid/height/.initial=.60}
\ctikzset{bipoles/squid/width/.initial=.60}
\ctikzset{bipoles/barrier/height/.initial=.60}
\ctikzset{bipoles/barrier/width/.initial=.60}
\ctikzset{bipoles/openbarrier/gap/.initial=0.5}
\ctikzset{bipoles/thermocouple/height/.initial=.250}
\ctikzset{bipoles/thermocouple/height 2/.initial=.60}
\ctikzset{bipoles/thermocouple/width/.initial=.140}
\ctikzset{bipoles/pushbutton/height/.initial=.5}
\ctikzset{bipoles/pushbutton/height 2/.initial=.2}
\ctikzset{bipoles/pushbutton/width/.initial=.50}

\ctikzset{bipoles/twoport/width/.initial=.7}
\ctikzset{bipoles/twoport/height/.initial=.7}
\ctikzset{bipoles/twoport/text/.initial=}
\ctikzset{bipoles/vco/width/.initial=.7}
\ctikzset{bipoles/bandpass/width/.initial=.7}
\ctikzset{bipoles/bandstop/width/.initial=.7}
\ctikzset{bipoles/highpass/width/.initial=.7}
\ctikzset{bipoles/lowpass/width/.initial=.7}
\ctikzset{bipoles/adc/width/.initial=.7}
\ctikzset{bipoles/dac/width/.initial=.7}
\ctikzset{bipoles/dsp/width/.initial=.7}
\ctikzset{bipoles/fft/width/.initial=.7}
\ctikzset{bipoles/amp/width/.initial=.7}
\ctikzset{bipoles/vamp/width/.initial=.7}
\ctikzset{bipoles/piattenuator/width/.initial=.7}
\ctikzset{bipoles/vpiattenuator/width/.initial=.7}
\ctikzset{bipoles/tattenuator/width/.initial=.7}
\ctikzset{bipoles/vtattenuator/width/.initial=.7}
\ctikzset{bipoles/phaseshifter/width/.initial=.7}
\ctikzset{bipoles/vphaseshifter/width/.initial=.7}
\ctikzset{bipoles/detector/width/.initial=.7}

% crossing wires
\ctikzset{bipoles/crossing/size/.initial=.2}

% Transistors starts here

\newif\ifpgf@circuit@trans@depletiontype
\pgf@circuit@trans@depletiontypefalse

\newif\ifpgf@circuit@mos@arrows
\ctikzset{tripoles/mos style/.is choice}
\ctikzset{tripoles/mos style/no arrows/.code={\pgf@circuit@mos@arrowsfalse}}
\ctikzset{tripoles/mos style/arrows/.code={\pgf@circuit@mos@arrowstrue}}
\pgfkeys{/tikz/arrowmos/.add code={}{\pgf@circuit@mos@arrowstrue}}
\pgfkeys{/tikz/noarrowmos/.add code={}{\pgf@circuit@mos@arrowsfalse}}

% Fixed label positions
\newif\ifpgf@circuit@transisors@fixlabels
\pgf@circuit@transisors@fixlabelstrue
\pgfkeys{/tikz/center transistors text/.add code={}{\pgf@circuit@transisors@fixlabelstrue}}
\ctikzset{fix transistors text/.add code={}{\pgf@circuit@transisors@fixlabelstrue}}
\pgfkeys{/tikz/legacy transistors text/.add code={}{\pgf@circuit@transisors@fixlabelsfalse}}
\ctikzset{legacy transistors text/.add code={}{\pgf@circuit@transisors@fixlabelsfalse}}

% Option solderdot for fet
\newif\ifpgf@circuit@fet@solderdot
\pgfkeys{/tikz/solderdot/.add code={}{\pgf@circuit@fet@solderdottrue}}
\ctikzset{solderdot/.add code={}{\pgf@circuit@fet@solderdottrue}}
\pgfkeys{/tikz/nosolderdot/.add code={}{\pgf@circuit@fet@solderdotfalse}}
\ctikzset{nosolderdot/.add code={}{\pgf@circuit@fet@solderdotfalse}}

% Option bodydiode for fet
\newif\ifpgf@circuit@fet@bodydiode
\pgfkeys{/tikz/bodydiode/.add code={}{\pgf@circuit@fet@bodydiodetrue}}
\ctikzset{bodydiode/.add code={}{\pgf@circuit@fet@bodydiodetrue}}
\pgfkeys{/tikz/nobodydiode/.add code={}{\pgf@circuit@fet@bodydiodefalse}}
\ctikzset{nobodydiode/.add code={}{\pgf@circuit@fet@bodydiodefalse}}

% Option draw fet without gate connection
\newif\ifpgf@circuit@bpt@drawgate
\pgf@circuit@bpt@drawgatetrue
\pgfkeys{/tikz/nogate/.add code={}{\pgf@circuit@bpt@drawgatefalse}}
\ctikzset{nogate/.add code={}{\pgf@circuit@bpt@drawgatefalse}}
\pgfkeys{/tikz/nobase/.add code={}{\pgf@circuit@bpt@drawgatefalse}}
\ctikzset{nobase/.add code={}{\pgf@circuit@bpt@drawgatefalse}}

% Option draw bpt with schottky base
\newif\ifpgf@circuit@bpt@schottky
\pgf@circuit@bpt@schottkyfalse
\pgfkeys{/tikz/schottky base/.add code={}{\pgf@circuit@bpt@schottkytrue}}
\ctikzset{schottky base/.add code={}{\pgf@circuit@bpt@schottkytrue}}
\pgfkeys{/tikz/no schottky base/.add code={}{\pgf@circuit@bpt@schottkyfalse}}
\ctikzset{no schottky base/.add code={}{\pgf@circuit@bpt@schottkyfalse}}
\ctikzset{tripoles/schottky base size/.initial=0.07}

% Option draw bpt without base connection
\newif\ifpgf@circuit@bpt@drawbase
\pgf@circuit@bpt@drawbasetrue
\pgfkeys{/tikz/nobase/.add code={}{\pgf@circuit@bpt@drawbasefalse}}
\ctikzset{nobase/.add code={}{\pgf@circuit@bpt@drawbasefalse}}
\pgfkeys{/tikz/nogate/.add code={}{\pgf@circuit@bpt@drawbasefalse}}
\ctikzset{nogate/.add code={}{\pgf@circuit@bpt@drawbasefalse}}

% Option draw bpt with optical input
\newif\ifpgf@circuit@bpt@drawphoto
\pgf@circuit@bpt@drawphotofalse
\pgfkeys{/tikz/photo/.add code={}{\pgf@circuit@bpt@drawphototrue}}
\ctikzset{photo/.add code={}{\pgf@circuit@bpt@drawphototrue}}
\pgfkeys{/tikz/photo/.add code={}{\pgf@circuit@bpt@drawphototrue}}
\ctikzset{photo/.add code={}{\pgf@circuit@bpt@drawphototrue}}

% Option draw fet without bulk connection -- Added by Burak Kelleci
\newif\ifpgf@circuit@bpt@drawbulk
\pgfkeys{/tikz/bulk/.add code={}{\pgf@circuit@bpt@drawbulktrue}}
\ctikzset{bulk/.add code={}{\pgf@circuit@bpt@drawbulktrue}}
\pgfkeys{/tikz/nobulk/.add code={}{\pgf@circuit@bpt@drawbulkfalse}}
\ctikzset{nobulk/.add code={}{\pgf@circuit@bpt@drawbulkfalse}}

% Option draw pmos with empty circle
\newif\ifpgf@circuit@pmos@emptycircle
\pgf@circuit@pmos@emptycirclefalse
\ctikzset{tripoles/pmos style/.is choice}
\pgfkeys{/tikz/emptycircle/.add code={}{\pgf@circuit@pmos@emptycircletrue}}
\ctikzset{tripoles/pmos style/emptycircle/.add code={}{\pgf@circuit@pmos@emptycircletrue}}
% Option draw pmos with no circle
\newif\ifpgf@circuit@pmos@nocircle
\pgf@circuit@pmos@nocirclefalse
\pgfkeys{/tikz/nocircle/.add code={}{\pgf@circuit@pmos@nocircletrue}}
\ctikzset{tripoles/pmos style/nocircle/.add code={}{\pgf@circuit@pmos@nocircletrue}}
% back to normal
\pgfkeys{/tikz/fullcircle/.add code={}{\pgf@circuit@pmos@emptycirclefalse\pgf@circuit@pmos@nocirclefalse}}
\ctikzset{tripoles/pmos style/fullcircle/.add code={}{\pgf@circuit@pmos@emptycirclefalse\pgf@circuit@pmos@nocirclefalse}}
% arrows at the end, the correct way
\newif\ifpgf@circuit@trans@arrowatend
\pgf@circuit@trans@arrowatendfalse
\ctikzset{transistors/arrow pos/.is choice}
\ctikzset{transistors/arrow pos/legacy/.code={\pgf@circuit@trans@arrowatendfalse}}
\ctikzset{transistors/arrow pos/end/.code={\pgf@circuit@trans@arrowatendtrue}}

\newif\ifpgf@circuit@trans@ntype
\pgf@circuit@trans@ntypetrue %default true
%\newif\ifpgf@circuit@trans@ptype
%\ctikzset{tripoles/nmos/.add code={\pgf@circuit@trans@ntypetrue}}
\ctikzset{tripoles/nmos/width/.initial=.7}
\ctikzset{tripoles/nmos/gate height/.initial=.35}
\ctikzset{tripoles/nmos/base height/.initial=.5}
\ctikzset{tripoles/nmos/conn height/.initial=0}
\ctikzset{tripoles/nmos/height/.initial=1.1}
\ctikzset{tripoles/nmos/base width/.initial=.5}
\ctikzset{tripoles/nmos/gate width/.initial=.62}
\ctikzset{tripoles/nmos/arrow pos/.initial=.6}
\ctikzset{tripoles/nmos/bodydiode scale/.initial=.3}
\ctikzset{tripoles/nmos/bodydiode distance/.initial=.3}
\ctikzset{tripoles/nmos/bodydiode conn/.initial=.6}
\ctikzset{tripoles/nmos/curr direction/.initial=1}

\ctikzset{tripoles/pmos/width/.initial=.7}
\ctikzset{tripoles/pmos/gate height/.initial=.35}
\ctikzset{tripoles/pmos/base height/.initial=.5}
\ctikzset{tripoles/pmos/conn height/.initial=0}
\ctikzset{tripoles/pmos/height/.initial=1.1}
\ctikzset{tripoles/pmos/base width/.initial=.5}
\ctikzset{tripoles/pmos/gate width/.initial=.62}
\ctikzset{tripoles/pmos/arrow pos/.initial=.4}
\ctikzset{tripoles/pmos/bodydiode scale/.initial=.3}
\ctikzset{tripoles/pmos/bodydiode distance/.initial=.3}
\ctikzset{tripoles/pmos/bodydiode conn/.initial=.6}
\ctikzset{tripoles/pmos/curr direction/.initial=-1}

\ctikzset{tripoles/hemt/width/.initial=.7}
\ctikzset{tripoles/hemt/gate height/.initial=.35}
\ctikzset{tripoles/hemt/base height/.initial=.5}
\ctikzset{tripoles/hemt/conn height/.initial=0}
\ctikzset{tripoles/hemt/height/.initial=1.1}
\ctikzset{tripoles/hemt/base width/.initial=.5}
\ctikzset{tripoles/hemt/gate width/.initial=.62}
\ctikzset{tripoles/hemt/bodydiode scale/.initial=.3}
\ctikzset{tripoles/hemt/bodydiode distance/.initial=.3}
\ctikzset{tripoles/hemt/bodydiode conn/.initial=.6}
\ctikzset{tripoles/hemt/curr direction/.initial=1}

\ctikzset{tripoles/nfet/width/.initial=.7}
\ctikzset{tripoles/nfet/gate height/.initial=.35}
\ctikzset{tripoles/nfet/base height/.initial=.5}
\ctikzset{tripoles/nfet/conn height/.initial=0}
\ctikzset{tripoles/nfet/height/.initial=1.1}
\ctikzset{tripoles/nfet/base width/.initial=.5}
\ctikzset{tripoles/nfet/gate width/.initial=.62}
\ctikzset{tripoles/nfet/bodydiode scale/.initial=.3}
\ctikzset{tripoles/nfet/bodydiode distance/.initial=.3}
\ctikzset{tripoles/nfet/bodydiode conn/.initial=.6}
\ctikzset{tripoles/nfet/curr direction/.initial=1}

\ctikzset{tripoles/pfet/width/.initial=.7}
\ctikzset{tripoles/pfet/gate height/.initial=.35}
\ctikzset{tripoles/pfet/base height/.initial=.5}
\ctikzset{tripoles/pfet/conn height/.initial=0}
\ctikzset{tripoles/pfet/height/.initial=1.1}
\ctikzset{tripoles/pfet/base width/.initial=.5}
\ctikzset{tripoles/pfet/gate width/.initial=.62}
\ctikzset{tripoles/pfet/bodydiode scale/.initial=.3}
\ctikzset{tripoles/pfet/bodydiode distance/.initial=.3}
\ctikzset{tripoles/pfet/bodydiode conn/.initial=.6}
\ctikzset{tripoles/pfet/curr direction/.initial=-1}

\ctikzset{tripoles/nigfete/width/.initial=.7}
\ctikzset{tripoles/nigfete/gate height/.initial=.35}
\ctikzset{tripoles/nigfete/base height/.initial=.5}
\ctikzset{tripoles/nigfete/conn height/.initial=.35}
\ctikzset{tripoles/nigfete/height/.initial=1.1}
\ctikzset{tripoles/nigfete/base width/.initial=.5}
\ctikzset{tripoles/nigfete/gate width/.initial=.62}
\ctikzset{tripoles/nigfete/bodydiode scale/.initial=.3}
\ctikzset{tripoles/nigfete/bodydiode distance/.initial=.3}
\ctikzset{tripoles/nigfete/bodydiode conn/.initial=.6}
\ctikzset{tripoles/nigfete/curr direction/.initial=1}

\ctikzset{tripoles/nigfetd/width/.initial=.7}
\ctikzset{tripoles/nigfetd/gate height/.initial=.35}
\ctikzset{tripoles/nigfetd/base height/.initial=.5}
\ctikzset{tripoles/nigfetd/conn height/.initial=.35}
\ctikzset{tripoles/nigfetd/height/.initial=1.1}
\ctikzset{tripoles/nigfetd/base width/.initial=.5}
\ctikzset{tripoles/nigfetd/gate width/.initial=.62}
\ctikzset{tripoles/nigfetd/bodydiode scale/.initial=.3}
\ctikzset{tripoles/nigfetd/bodydiode distance/.initial=.3}
\ctikzset{tripoles/nigfetd/bodydiode conn/.initial=.6}
\ctikzset{tripoles/nigfetd/curr direction/.initial=1}

\ctikzset{tripoles/nigfetebulk/width/.initial=.7}
\ctikzset{tripoles/nigfetebulk/gate height/.initial=.35}
\ctikzset{tripoles/nigfetebulk/base height/.initial=.5}
\ctikzset{tripoles/nigfetebulk/conn height/.initial=.35}
\ctikzset{tripoles/nigfetebulk/height/.initial=1.1}
\ctikzset{tripoles/nigfetebulk/base width/.initial=.5}
\ctikzset{tripoles/nigfetebulk/gate width/.initial=.62}
\ctikzset{tripoles/nigfetebulk/bodydiode scale/.initial=.3}
\ctikzset{tripoles/nigfetebulk/bodydiode distance/.initial=.3}
\ctikzset{tripoles/nigfetebulk/bodydiode conn/.initial=.6}
\ctikzset{tripoles/nigfetebulk/curr direction/.initial=1}

\ctikzset{tripoles/pigfete/width/.initial=.7}
\ctikzset{tripoles/pigfete/gate height/.initial=.35}
\ctikzset{tripoles/pigfete/base height/.initial=.5}
\ctikzset{tripoles/pigfete/conn height/.initial=.35}
\ctikzset{tripoles/pigfete/height/.initial=1.1}
\ctikzset{tripoles/pigfete/base width/.initial=.5}
\ctikzset{tripoles/pigfete/gate width/.initial=.62}
\ctikzset{tripoles/pigfete/bodydiode scale/.initial=.3}
\ctikzset{tripoles/pigfete/bodydiode distance/.initial=.3}
\ctikzset{tripoles/pigfete/bodydiode conn/.initial=.6}
\ctikzset{tripoles/pigfete/curr direction/.initial=-1}

\ctikzset{tripoles/pigfetd/width/.initial=.7}
\ctikzset{tripoles/pigfetd/gate height/.initial=.35}
\ctikzset{tripoles/pigfetd/base height/.initial=.5}
\ctikzset{tripoles/pigfetd/conn height/.initial=.35}
\ctikzset{tripoles/pigfetd/height/.initial=1.1}
\ctikzset{tripoles/pigfetd/base width/.initial=.5}
\ctikzset{tripoles/pigfetd/gate width/.initial=.62}
\ctikzset{tripoles/pigfetd/bodydiode scale/.initial=.3}
\ctikzset{tripoles/pigfetd/bodydiode distance/.initial=.3}
\ctikzset{tripoles/pigfetd/bodydiode conn/.initial=.6}
\ctikzset{tripoles/pigfetd/curr direction/.initial=-1}

\ctikzset{tripoles/pigfetebulk/width/.initial=.7}
\ctikzset{tripoles/pigfetebulk/gate height/.initial=.35}
\ctikzset{tripoles/pigfetebulk/conn height/.initial=.35}
\ctikzset{tripoles/pigfetebulk/base height/.initial=.5}
\ctikzset{tripoles/pigfetebulk/height/.initial=1.1}
\ctikzset{tripoles/pigfetebulk/base width/.initial=.5}
\ctikzset{tripoles/pigfetebulk/gate width/.initial=.62}
\ctikzset{tripoles/pigfetebulk/bodydiode scale/.initial=.3}
\ctikzset{tripoles/pigfetebulk/bodydiode distance/.initial=.3}
\ctikzset{tripoles/pigfetebulk/bodydiode conn/.initial=.6}
\ctikzset{tripoles/pigfetebulk/curr direction/.initial=-1}

\ctikzset{tripoles/npn/width/.initial=.6}
\ctikzset{tripoles/npn/base height/.initial=.45}
\ctikzset{tripoles/npn/base height 2/.initial=.15}
\ctikzset{tripoles/npn/base height/.initial=.4}
\ctikzset{tripoles/npn/conn height/.initial=0}
\ctikzset{tripoles/npn/height/.initial=1.1}
\ctikzset{tripoles/npn/base width/.initial=.5}
\ctikzset{tripoles/npn/arrow pos/.initial=.5}
\ctikzset{tripoles/npn/bodydiode scale/.initial=.3}
\ctikzset{tripoles/npn/bodydiode distance/.initial=.3}
\ctikzset{tripoles/npn/bodydiode conn/.initial=.6}
\ctikzset{tripoles/npn/curr direction/.initial=1}

\ctikzset{tripoles/pnp/width/.initial=.6}
\ctikzset{tripoles/pnp/base height/.initial=.45}
\ctikzset{tripoles/pnp/base height 2/.initial=.15}
\ctikzset{tripoles/pnp/base height/.initial=.4}
\ctikzset{tripoles/pnp/conn height/.initial=0}
\ctikzset{tripoles/pnp/height/.initial=1.1}
\ctikzset{tripoles/pnp/base width/.initial=.5}
\ctikzset{tripoles/pnp/arrow pos/.initial=.5}
\ctikzset{tripoles/pnp/bodydiode scale/.initial=.3}
\ctikzset{tripoles/pnp/bodydiode distance/.initial=.3}
\ctikzset{tripoles/pnp/bodydiode conn/.initial=.6}
\ctikzset{tripoles/pnp/curr direction/.initial=-1}

\ctikzset{tripoles/pigbt/width/.initial=.6}
\ctikzset{tripoles/pigbt/gate height/.initial=.45}
\ctikzset{tripoles/pigbt/gate height 2/.initial=.15}
\ctikzset{tripoles/pigbt/base height/.initial=.4}
\ctikzset{tripoles/pigbt/outer base height/.initial=.4}
\ctikzset{tripoles/pigbt/outer base thickness/.initial=1}
\ctikzset{tripoles/pigbt/conn height/.initial=0}
\ctikzset{tripoles/pigbt/height/.initial=1.1}
\ctikzset{tripoles/pigbt/gate width/.initial=.62}
\ctikzset{tripoles/pigbt/base width/.initial=.5}
\ctikzset{tripoles/pigbt/bodydiode scale/.initial=.3}
\ctikzset{tripoles/pigbt/bodydiode distance/.initial=.3}
\ctikzset{tripoles/pigbt/bodydiode conn/.initial=.6}
\ctikzset{tripoles/pigbt/curr direction/.initial=-1}

\ctikzset{tripoles/nigbt/width/.initial=.6}
\ctikzset{tripoles/nigbt/gate height/.initial=.45}
\ctikzset{tripoles/nigbt/gate height 2/.initial=.15}
\ctikzset{tripoles/nigbt/base height/.initial=.4}
\ctikzset{tripoles/nigbt/outer base height/.initial=.4}
\ctikzset{tripoles/nigbt/outer base thickness/.initial=1}
\ctikzset{tripoles/nigbt/conn height/.initial=0}
\ctikzset{tripoles/nigbt/height/.initial=1.1}
\ctikzset{tripoles/nigbt/gate width/.initial=.62}
\ctikzset{tripoles/nigbt/base width/.initial=.5}
\ctikzset{tripoles/nigbt/bodydiode scale/.initial=.3}
\ctikzset{tripoles/nigbt/bodydiode distance/.initial=.3}
\ctikzset{tripoles/nigbt/bodydiode conn/.initial=.6}
\ctikzset{tripoles/nigbt/curr direction/.initial=1}

\ctikzset{tripoles/Lpigbt/width/.initial=.6}
\ctikzset{tripoles/Lpigbt/gate height/.initial=.45}
\ctikzset{tripoles/Lpigbt/gate height 2/.initial=.15}
\ctikzset{tripoles/Lpigbt/base height/.initial=.4}
\ctikzset{tripoles/Lpigbt/outer base height/.initial=.4}
\ctikzset{tripoles/Lpigbt/outer base thickness/.initial=1}
\ctikzset{tripoles/Lpigbt/conn height/.initial=.4}
\ctikzset{tripoles/Lpigbt/height/.initial=1.1}
\ctikzset{tripoles/Lpigbt/gate width/.initial=.62}
\ctikzset{tripoles/Lpigbt/base width/.initial=.5}
\ctikzset{tripoles/Lpigbt/bodydiode scale/.initial=.3}
\ctikzset{tripoles/Lpigbt/bodydiode distance/.initial=.3}
\ctikzset{tripoles/Lpigbt/bodydiode conn/.initial=.6}
\ctikzset{tripoles/Lpigbt/curr direction/.initial=-1}

\ctikzset{tripoles/Lnigbt/width/.initial=.6}
\ctikzset{tripoles/Lnigbt/gate height/.initial=.45}
\ctikzset{tripoles/Lnigbt/gate height 2/.initial=.15}
\ctikzset{tripoles/Lnigbt/base height/.initial=.4}
\ctikzset{tripoles/Lnigbt/outer base height/.initial=.4}
\ctikzset{tripoles/Lnigbt/outer base thickness/.initial=1}
\ctikzset{tripoles/Lnigbt/conn height/.initial=.4}
\ctikzset{tripoles/Lnigbt/height/.initial=1.1}
\ctikzset{tripoles/Lnigbt/gate width/.initial=.62}
\ctikzset{tripoles/Lnigbt/base width/.initial=.5}
\ctikzset{tripoles/Lnigbt/bodydiode scale/.initial=.3}
\ctikzset{tripoles/Lnigbt/bodydiode distance/.initial=.3}
\ctikzset{tripoles/Lnigbt/bodydiode conn/.initial=.6}
\ctikzset{tripoles/Lnigbt/curr direction/.initial=1}

\ctikzset{tripoles/igbt/outer base height/.code={
    \ctikzset{tripoles/nigbt/outer base height/.initial=#1}
    \ctikzset{tripoles/pigbt/outer base height/.initial=#1}
    \ctikzset{tripoles/Lnigbt/outer base height/.initial=#1}
    \ctikzset{tripoles/Lpigbt/outer base height/.initial=#1}
    \ctikzset{tripoles/Lnigbt/conn height/.initial=#1}
    \ctikzset{tripoles/Lpigbt/conn height/.initial=#1}
}}
\ctikzset{tripoles/igbt/outer base thickness/.code={
    \ctikzset{tripoles/nigbt/outer base thickness=#1}
    \ctikzset{tripoles/pigbt/outer base thickness=#1}
    \ctikzset{tripoles/Lnigbt/outer base thickness=#1}
    \ctikzset{tripoles/Lpigbt/outer base thickness=#1}
}}

\ctikzset{tripoles/njfet/width/.initial=.7}
\ctikzset{tripoles/njfet/gate height/.initial=.5}
\ctikzset{tripoles/njfet/gate height 2/.initial=.35}
\ctikzset{tripoles/njfet/gate width/.initial=.5}
\ctikzset{tripoles/njfet/conn height/.initial=.35}
\ctikzset{tripoles/njfet/height/.initial=1.1}
\ctikzset{tripoles/njfet/bodydiode scale/.initial=.3}
\ctikzset{tripoles/njfet/bodydiode distance/.initial=.3}
\ctikzset{tripoles/njfet/bodydiode conn/.initial=.6}
\ctikzset{tripoles/njfet/curr direction/.initial=1}

\ctikzset{tripoles/pjfet/width/.initial=.7}
\ctikzset{tripoles/pjfet/gate height/.initial=.5}
\ctikzset{tripoles/pjfet/gate height 2/.initial=.35}
\ctikzset{tripoles/pjfet/gate width/.initial=.5}
\ctikzset{tripoles/pjfet/conn height/.initial=.35}
\ctikzset{tripoles/pjfet/height/.initial=1.1}
\ctikzset{tripoles/pjfet/bodydiode scale/.initial=.3}
\ctikzset{tripoles/pjfet/bodydiode distance/.initial=.3}
\ctikzset{tripoles/pjfet/bodydiode conn/.initial=.6}
\ctikzset{tripoles/pjfet/curr direction/.initial=-1}

\ctikzset{tripoles/isfet/width/.initial=1}
\ctikzset{tripoles/isfet/gate height/.initial=.35}
\ctikzset{tripoles/isfet/base height/.initial=.5}
\ctikzset{tripoles/isfet/height/.initial=1.1}
\ctikzset{tripoles/isfet/base width/.initial=.3}
\ctikzset{tripoles/isfet/gate width/.initial=.5}
\ctikzset{tripoles/isfet/conn height/.initial=0}
\ctikzset{tripoles/isfet/wave width/.initial=.16}
\ctikzset{tripoles/isfet/wave amp/.initial=.06}
\ctikzset{tripoles/isfet/waves y sep/.initial=.22}
\ctikzset{tripoles/isfet/waves x sep/.initial=.8}
\ctikzset{tripoles/isfet/bodydiode scale/.initial=.3}
\ctikzset{tripoles/isfet/bodydiode distance/.initial=.3}
\ctikzset{tripoles/isfet/bodydiode conn/.initial=.6}
\ctikzset{tripoles/isfet/curr direction/.initial=1}
%
% multi-emitter and multi-collector BJTs by Romano Giannetti
%
\ctikzset{tripoles/bjt/emitters/.initial=1}
\ctikzset{tripoles/bjt/collectors/.initial=1}
\pgfkeys{/tikz/emitters/.add code={}{\ctikzset{tripoles/bjt/emitters=#1}}}
\pgfkeys{/tikz/collectors/.add code={}{\ctikzset{tripoles/bjt/collectors=#1}}}
\ctikzset{tripoles/bjt/pins width/.initial=0.3}
\pgfkeys{/tikz/bjt pins width/.add code={}{\ctikzset{tripoles/bjt/pins width=#1}}}
\ctikzset{tripoles/bjt/multi height/.initial=.5}
\pgfkeys{/tikz/bjt multi height/.add code={}{\ctikzset{tripoles/bjt/multi height/.initial=#1}}}
%
\ctikzset{tripoles/bjt/width/.initial=.3}
\ctikzset{tripoles/bjt/base height 2/.initial=.4}
\ctikzset{tripoles/bjt/base height/.initial=1.1}
\ctikzset{tripoles/bjt/height/.initial=.4}
\ctikzset{tripoles/bjt/base width/.initial=1}
\ctikzset{tripoles/bjt/arrow pos/.initial=.5}
% do NOT touch these two!
\ctikzset{tripoles/bjt/npn/curr direction/.initial=1}
\ctikzset{tripoles/bjt/pnp/curr direction/.initial=-1}

%% transistors stop here

%% Logic ports

\newif\ifpgf@circuit@europeanlogicport
\ctikzset{logic ports/.is choice}
\ctikzset{logic ports/european/.code= {\pgf@circuit@europeanlogicporttrue
    \tikzset{and port/.style={shape=european and port}}%
    \tikzset{or port/.style={shape=european or port}}%
    \tikzset{xor port/.style={shape=european xor port}}%
    \tikzset{buffer port/.style={shape=european buffer port}}%
    \tikzset{not port/.style={shape=european not port}}%
    \tikzset{nand port/.style={shape=european nand port}}%
    \tikzset{nor port/.style={shape=european nor port}}%
    \tikzset{xnor port/.style={shape=european xnor port}}%
}}
\ctikzset{logic ports/american/.code= {\pgf@circuit@europeanlogicportfalse
    \tikzset{and port/.style={shape=american and port}}%
    \tikzset{or port/.style={shape=american or port}}%
    \tikzset{xor port/.style={shape=american xor port}}%
    \tikzset{buffer port/.style={shape=american buffer port}}%
    \tikzset{not port/.style={shape=american not port}}%
    \tikzset{nand port/.style={shape=american nand port}}%
    \tikzset{nor port/.style={shape=american nor port}}%
    \tikzset{xnor port/.style={shape=american xnor port}}%
}}

\ctikzset{logic ports origin/.is choice}
\ctikzset{logic ports origin/legacy/.code={
    \ctikzset{tripoles/american and port/origin/.initial=0.8}%
    \ctikzset{tripoles/american nand port/origin/.initial=0.8}%
    \ctikzset{tripoles/american nor port/origin/.initial=0.8}%
    \ctikzset{tripoles/american or port/origin/.initial=0.8}%
    \ctikzset{tripoles/american xor port/origin/.initial=0.8}%
    \ctikzset{tripoles/american xnor port/origin/.initial=0.8}%
    \ctikzset{tripoles/european and port/origin/.initial=0.8}%
    \ctikzset{tripoles/european nand port/origin/.initial=0.8}%
    \ctikzset{tripoles/european or port/origin/.initial=0.8}%
    \ctikzset{tripoles/european nor port/origin/.initial=0.8}%
    \ctikzset{tripoles/european xor port/origin/.initial=0.8}%
    \ctikzset{tripoles/european xnor port/origin/.initial=0.8}%
    \ctikzset{tripoles/european buffer port/origin/.initial=0.8}%
    \ctikzset{tripoles/european not port/origin/.initial=0.8}%
    }%
}
\ctikzset{logic ports origin/center/.code={%
    \ctikzset{tripoles/american and port/origin/.initial=0}%
    \ctikzset{tripoles/american nand port/origin/.initial=0}%
    \ctikzset{tripoles/american nor port/origin/.initial=0}%
    \ctikzset{tripoles/american or port/origin/.initial=0}%
    \ctikzset{tripoles/american xor port/origin/.initial=0}%
    \ctikzset{tripoles/american xnor port/origin/.initial=0}%
    \ctikzset{tripoles/european and port/origin/.initial=0}%
    \ctikzset{tripoles/european nand port/origin/.initial=0}%
    \ctikzset{tripoles/european or port/origin/.initial=0}%
    \ctikzset{tripoles/european nor port/origin/.initial=0}%
    \ctikzset{tripoles/european xor port/origin/.initial=0}%
    \ctikzset{tripoles/european xnor port/origin/.initial=0}%
    \ctikzset{tripoles/european buffer port/origin/.initial=0}%
    \ctikzset{tripoles/european not port/origin/.initial=0}%
    }%
}

\newif\ifpgfcirc@roundy@or@shapes\pgfcirc@roundy@or@shapesfalse
\ctikzset{american or shape/.is choice}
\ctikzset{american or shape/roundy/.code={\pgfcirc@roundy@or@shapestrue}}
\ctikzset{american or shape/pointy/.code={\pgfcirc@roundy@or@shapesfalse}}

\newif\ifpgfcirc@draw@leads\pgfcirc@draw@leadstrue
\ctikzset{logic ports draw input leads/.is choice}
\ctikzset{logic ports draw input leads/true/.code={\pgfcirc@draw@leadstrue}}
\ctikzset{logic ports draw input leads/false/.code={\pgfcirc@draw@leadsfalse}}
\tikzset{input leads/.code={\pgfcirc@draw@leadstrue}}
\tikzset{no input leads/.code={\pgfcirc@draw@leadsfalse}}

\ctikzset{tripoles/american and port/width/.initial=1.1}
\ctikzset{tripoles/american and port/height/.initial=.8}
\ctikzset{tripoles/american and port/port width/.initial=.7}
\ctikzset{tripoles/american and port/input height/.initial=.5}
\ctikzset{tripoles/american nand port/width/.initial=1.1}
\ctikzset{tripoles/american nand port/height/.initial=.8}
\ctikzset{tripoles/american nand port/port width/.initial=.7}
\ctikzset{tripoles/american nand port/circle width/.initial=.15}
\ctikzset{tripoles/american nand port/input height/.initial=.5}
\ctikzset{tripoles/american or port/width/.initial=1.1}
\ctikzset{tripoles/american or port/height/.initial=.8}
\ctikzset{tripoles/american or port/port width/.initial=.7}
\ctikzset{tripoles/american or port/input height/.initial=.5}
\ctikzset{tripoles/american or port/input skip/.initial=.25}
\ctikzset{tripoles/american or port/aaa/.initial=.6}
\ctikzset{tripoles/american or port/bbb/.initial=.4}
\ctikzset{tripoles/american or port/ccc/.initial=.5}
\ctikzset{tripoles/american or port/ddd/.initial=.0}
\ctikzset{tripoles/american nor port/width/.initial=1.1}
\ctikzset{tripoles/american nor port/height/.initial=.8}
\ctikzset{tripoles/american nor port/port width/.initial=.7}
\ctikzset{tripoles/american nor port/input height/.initial=.5}
\ctikzset{tripoles/american nor port/input skip/.initial=.25}
\ctikzset{tripoles/american nor port/circle width/.initial=.15}
\ctikzset{tripoles/american nor port/aaa/.initial=.6}
\ctikzset{tripoles/american nor port/bbb/.initial=.4}
\ctikzset{tripoles/american nor port/ccc/.initial=.5}
\ctikzset{tripoles/american nor port/ddd/.initial=.0}
\ctikzset{tripoles/american xor port/width/.initial=1.1}
\ctikzset{tripoles/american xor port/height/.initial=.8}
\ctikzset{tripoles/american xor port/port width/.initial=.7}
\ctikzset{tripoles/american xor port/input height/.initial=.5}
\ctikzset{tripoles/american xor port/input skip/.initial=.15}
\ctikzset{tripoles/american xor port/distance/.initial=.1}
\ctikzset{tripoles/american xnor port/width/.initial=1.1}
\ctikzset{tripoles/american xnor port/height/.initial=.8}
\ctikzset{tripoles/american xnor port/port width/.initial=.7}
\ctikzset{tripoles/american xnor port/input height/.initial=.5}
\ctikzset{tripoles/american xnor port/input skip/.initial=.15}
\ctikzset{tripoles/american xnor port/distance/.initial=.1}
\ctikzset{tripoles/american xnor port/circle width/.initial=.15}
\ctikzset{tripoles/american and port/origin/.initial=0.8}
\ctikzset{tripoles/american and port/inputs/.initial=2}
% variable number of inputs
\ctikzset{tripoles/american nand port/origin/.initial=0.8}
\ctikzset{tripoles/american nand port/inputs/.initial=2}
\ctikzset{tripoles/american nor port/origin/.initial=0.8}
\ctikzset{tripoles/american nor port/inputs/.initial=2}
\ctikzset{tripoles/american nor port/angle/.initial=70}
\ctikzset{tripoles/american nor port/inner/.initial=0.3}
\ctikzset{tripoles/american or port/origin/.initial=0.8}
\ctikzset{tripoles/american or port/inputs/.initial=2}
\ctikzset{tripoles/american or port/angle/.initial=70}
\ctikzset{tripoles/american or port/inner/.initial=0.3}
\ctikzset{tripoles/american xor port/origin/.initial=0.8}
\ctikzset{tripoles/american xor port/inputs/.initial=2}
\ctikzset{tripoles/american xor port/angle/.initial=70}
\ctikzset{tripoles/american xor port/inner/.initial=0.3}
\ctikzset{tripoles/american xnor port/origin/.initial=0.8}
\ctikzset{tripoles/american xnor port/inputs/.initial=2}
\ctikzset{tripoles/american xnor port/angle/.initial=70}
\ctikzset{tripoles/american xnor port/inner/.initial=0.3}

\ctikzset{tripoles/european and port/width/.initial=1.4}
\ctikzset{tripoles/european and port/height/.initial=.65}
\ctikzset{tripoles/european and port/reserved/.initial=.6}
\ctikzset{tripoles/european and port/input height/.initial=.6}
\ctikzset{tripoles/european or port/width/.initial=1.4}
\ctikzset{tripoles/european or port/height/.initial=.65}
\ctikzset{tripoles/european or port/reserved/.initial=.6}
\ctikzset{tripoles/european or port/input height/.initial=.6}
\ctikzset{tripoles/european xor port/width/.initial=1.4}
\ctikzset{tripoles/european xor port/height/.initial=.65}
\ctikzset{tripoles/european xor port/reserved/.initial=.6}
\ctikzset{tripoles/european xor port/input height/.initial=.6}
\ctikzset{tripoles/european nand port/width/.initial=1.4}
\ctikzset{tripoles/european nand port/not height/.initial=.3}
\ctikzset{tripoles/european nand port/not width/.initial=.8}
\ctikzset{tripoles/european nand port/height/.initial=.65}
\ctikzset{tripoles/european nand port/reserved/.initial=.6}
\ctikzset{tripoles/european nand port/input height/.initial=.6}
\ctikzset{tripoles/european buffer port/width/.initial=1.4}
\ctikzset{tripoles/european buffer port/not height/.initial=.3}
\ctikzset{tripoles/european buffer port/not width/.initial=.8}
\ctikzset{tripoles/european buffer port/height/.initial=.65}
\ctikzset{tripoles/european buffer port/reserved/.initial=.6}
\ctikzset{tripoles/european buffer port/input height/.initial=0}
\ctikzset{tripoles/european not port/width/.initial=1.4}
\ctikzset{tripoles/european not port/not height/.initial=.3}
\ctikzset{tripoles/european not port/not width/.initial=.8}
\ctikzset{tripoles/european not port/height/.initial=.65}
\ctikzset{tripoles/european not port/reserved/.initial=.6}
\ctikzset{tripoles/european not port/input height/.initial=0}
\ctikzset{tripoles/european xnor port/width/.initial=1.4}
\ctikzset{tripoles/european xnor port/not height/.initial=.3}
\ctikzset{tripoles/european xnor port/not width/.initial=.8}
\ctikzset{tripoles/european xnor port/height/.initial=.65}
\ctikzset{tripoles/european xnor port/reserved/.initial=.6}
\ctikzset{tripoles/european xnor port/input height/.initial=.6}
\ctikzset{tripoles/european nor port/width/.initial=1.4}
\ctikzset{tripoles/european nor port/not height/.initial=.3}
\ctikzset{tripoles/european nor port/not width/.initial=.8}
\ctikzset{tripoles/european nor port/height/.initial=.65}
\ctikzset{tripoles/european nor port/reserved/.initial=.6}
\ctikzset{tripoles/european nor port/input height/.initial=.6}
% variable number of inputs
\ctikzset{tripoles/european and port/origin/.initial=0.8}
\ctikzset{tripoles/european and port/inputs/.initial=2}
\ctikzset{tripoles/european nand port/origin/.initial=0.8}
\ctikzset{tripoles/european nand port/inputs/.initial=2}
\ctikzset{tripoles/european or port/origin/.initial=0.8}
\ctikzset{tripoles/european or port/inputs/.initial=2}
\ctikzset{tripoles/european nor port/origin/.initial=0.8}
\ctikzset{tripoles/european nor port/inputs/.initial=2}
\ctikzset{tripoles/european xor port/origin/.initial=0.8}
\ctikzset{tripoles/european xor port/inputs/.initial=2}
\ctikzset{tripoles/european xnor port/origin/.initial=0.8}
\ctikzset{tripoles/european xnor port/inputs/.initial=2}
\ctikzset{tripoles/european buffer port/origin/.initial=0.8}
\ctikzset{tripoles/european buffer port/inputs/.initial=1}%
\ctikzset{tripoles/european not port/origin/.initial=0.8}
\ctikzset{tripoles/european not port/inputs/.initial=1}%
%%% parameters that are not used anymore after multi-input
%%% gates --- left for compatibility of source code.
\ctikzset{tripoles/american xor port/aaa/.initial=.6}
\ctikzset{tripoles/american xor port/bbb/.initial=.4}
\ctikzset{tripoles/american xor port/ccc/.initial=.5}
\ctikzset{tripoles/american xor port/ddd/.initial=.0}
\ctikzset{tripoles/american xnor port/aaa/.initial=.6}
\ctikzset{tripoles/american xnor port/bbb/.initial=.4}
\ctikzset{tripoles/american xnor port/ccc/.initial=.5}
\ctikzset{tripoles/american xnor port/ddd/.initial=.0}
% switches for logic gates
%
\pgfkeys{/tikz/number inputs/.initial=0}
\pgfkeys{/tikz/number inputs/.default=0}

%% flip-flop specific keys (most others are the same as chips)

\ctikzset{multipoles/flipflop/font/.initial=\pgf@circ@font@small}
\ctikzset{multipoles/flipflop/fontud/.initial=\pgf@circ@font@tiny}
\ctikzset{multipoles/flipflop/width/.initial=1.2}
\ctikzset{multipoles/flipflop/pin spacing/.initial=0.6}
\ctikzset{multipoles/flipflop/clock wedge size/.initial=0.2}

%% muxdemuxes keys

\ctikzset{multipoles/muxdemux/base len/.initial=0.4}
\ctikzset{multipoles/muxdemux/Lh/.initial=8.0}
\ctikzset{multipoles/muxdemux/Rh/.initial=6.0}
\ctikzset{multipoles/muxdemux/w/.initial=3.0}
\ctikzset{multipoles/muxdemux/inset w/.initial=0.0}
\ctikzset{multipoles/muxdemux/inset Lh/.initial=0.0}
\ctikzset{multipoles/muxdemux/inset Rh/.initial=0.0}
\ctikzset{multipoles/muxdemux/NL/.initial=8}
\ctikzset{multipoles/muxdemux/NR/.initial=1}
\ctikzset{multipoles/muxdemux/NB/.initial=3}
\ctikzset{multipoles/muxdemux/NT/.initial=0}
\ctikzset{multipoles/muxdemux/square pins/.initial=0}
%
% switches for op amps
% changing input polarity
%
\newif\ifpgf@circuit@oa@iplusup\pgf@circuit@oa@iplusupfalse
\pgfkeys{/tikz/noinv input up/.add code={}{\pgf@circuit@oa@iplusuptrue}}
\ctikzset{noinv input up/.add code={}{\pgf@circuit@oa@iplusuptrue}}
\pgfkeys{/tikz/noinv input down/.add code={}{\pgf@circuit@oa@iplusupfalse}}
\ctikzset{noinv input down/.add code={}{\pgf@circuit@oa@iplusupfalse}}
%
% changing output polarity (for fully diff objects)
%
\newif\ifpgf@circuit@oa@oplusup\pgf@circuit@oa@oplusuptrue
\pgfkeys{/tikz/noinv output up/.add code={}{\pgf@circuit@oa@oplusuptrue}}
\ctikzset{noinv output up/.add code={}{\pgf@circuit@oa@oplusuptrue}}
\pgfkeys{/tikz/noinv output down/.add code={}{\pgf@circuit@oa@oplusupfalse}}
\ctikzset{noinv output down/.add code={}{\pgf@circuit@oa@oplusupfalse}}

% Operational amplifier
\ctikzset{tripoles/op amp/width/.initial=1.7}              % Total width
\ctikzset{tripoles/op amp/port width/.initial=.7}          % Terminals length
\ctikzset{tripoles/op amp/height/.initial=1.4}             % Total height
\ctikzset{tripoles/op amp/input height/.initial=.5}        % Input port vertical separation
\ctikzset{tripoles/op amp/up pos/.initial=.45}             % Top and bottom anchor position
\ctikzset{tripoles/op amp/font/.initial=\pgf@circ@font@tenbm}  % Absolute font size needed!

% Fully differential operational amplifier
\ctikzset{tripoles/fd op amp/width/.initial=1.7}           % Total width
\ctikzset{tripoles/fd op amp/port width/.initial=.7}       % Terminals length
\ctikzset{tripoles/fd op amp/height/.initial=1.4}          % Total height
\ctikzset{tripoles/fd op amp/input height/.initial=.5}     % Input port vertical separation
\ctikzset{tripoles/fd op amp/output height/.initial=.5}    % Output port vertical separation
\ctikzset{tripoles/fd op amp/up pos/.initial=.45}          % Top and bottom anchor position
\ctikzset{tripoles/fd op amp/font/.initial=\pgf@circ@font@tenbm}  % Absolute font size needed!

\ctikzset{tripoles/en amp/width/.initial=1.7}
\ctikzset{tripoles/en amp/port width/.initial=.7}
\ctikzset{tripoles/en amp/height/.initial=1.6}
\ctikzset{tripoles/en amp/input height/.initial=.3}
\ctikzset{tripoles/en amp/up pos/.initial=.45}
\ctikzset{tripoles/en amp/font/.initial=\pgf@circ@font@tenbm}   % Absolute font size needed!
\ctikzset{tripoles/en amp/font2/.initial=\pgf@circ@font@twelve}  % Absolute font size needed!
\ctikzset{tripoles/en amp/text/.initial={$\mathstrut{\triangleright}\,{\infty}$}}
\tikzset{
    en amp text/.code = {%
        \ctikzsetvalof{tripoles/en amp/text}{#1}%
    },
    en amp text A/.code = {%
        \ctikzsetvalof{tripoles/en amp/text}{$\mathstrut{\triangleright}\,\mathrm{A}$}%
    },
}

% Transconductance amplifier
\ctikzset{tripoles/gm amp/width/.initial=1.7}              % Total width
\ctikzset{tripoles/gm amp/port width/.initial=.7}          % Terminals length
\ctikzset{tripoles/gm amp/height/.initial=1.4}             % Left side of the trapezoid
\ctikzset{tripoles/gm amp/height 2/.initial=0.5}           % Right side of the trapezoid
\ctikzset{tripoles/gm amp/input height/.initial=.5}        % Input port vertical separation
\ctikzset{tripoles/gm amp/up pos/.initial=.45}             % Top and bottom anchor position
\ctikzset{tripoles/gm amp/font/.initial=\pgf@circ@font@tenbm}  % Absolute font size needed!

% Instrumentation amplifier
\ctikzset{tripoles/inst amp/width/.initial=1.7}            % Total width
\ctikzset{tripoles/inst amp/port width/.initial=.7}        % Terminals length
\ctikzset{tripoles/inst amp/height/.initial=1.4}           % Left side of the trapezoid
\ctikzset{tripoles/inst amp/height 2/.initial=0.6}         % Right side of the trapezoid
\ctikzset{tripoles/inst amp/input height/.initial=.5}      % Input ports vertical separation
\ctikzset{tripoles/inst amp/up pos/.initial=.4}            % Top and bottom anchor position
\ctikzset{tripoles/inst amp/refv pos/.initial=.7}          % Top and bottom voltage reference position
\ctikzset{tripoles/inst amp/font/.initial=\pgf@circ@font@tenbm}  % Absolute font size needed!

% Instrumentation amplifier with differential output
\ctikzset{tripoles/fd inst amp/width/.initial=1.7}         % Total width
\ctikzset{tripoles/fd inst amp/port width/.initial=.7}     % Terminals length
\ctikzset{tripoles/fd inst amp/height/.initial=1.4}        % Left side of the trapezoid
\ctikzset{tripoles/fd inst amp/height 2/.initial=0.6}      % Right side of the trapezoid
\ctikzset{tripoles/fd inst amp/input height/.initial=.5}   % Input ports vertical separation
\ctikzset{tripoles/fd inst amp/output height/.initial=.5}  % Output ports vertical separation
\ctikzset{tripoles/fd inst amp/up pos/.initial=.4}         % Top and bottom anchor position
\ctikzset{tripoles/fd inst amp/refv pos/.initial=.7}       % Top and bottom voltage reference position
\ctikzset{tripoles/fd inst amp/font/.initial=\pgf@circ@font@tenbm}  % Absolute font size needed!

% Instrumentation amplifier with gain resistor terminals
\ctikzset{tripoles/inst amp ra/width/.initial=2.4}         % Total width
\ctikzset{tripoles/inst amp ra/port width/.initial=.7}     % Terminals length
\ctikzset{tripoles/inst amp ra/height/.initial=2.9}        % Left side of the trapezoid
\ctikzset{tripoles/inst amp ra/height 2/.initial=0.4}      % Right side of the trapezoid
\ctikzset{tripoles/inst amp ra/input height/.initial=.7}   % Input ports vertical separation
\ctikzset{tripoles/inst amp ra/up pos/.initial=.4}         % Top and bottom anchor position
\ctikzset{tripoles/inst amp ra/refv pos/.initial=.7}       % Top and bottom voltage reference position
\ctikzset{tripoles/inst amp ra/ra pos/.initial=.6}         % Gain resistor terminals vertical separation
\ctikzset{tripoles/inst amp ra/font/.initial=\pgf@circ@font@tenbm}  % Absolute font size needed!

% Plain amplifier
\ctikzset{tripoles/plain amp/width/.initial=1.7}           % Total width
\ctikzset{tripoles/plain amp/port width/.initial=.7}       % Terminals length
\ctikzset{tripoles/plain amp/height/.initial=1.4}          % Total height
\ctikzset{tripoles/plain amp/input height/.initial=.5}     % Input ports vertical separation
\ctikzset{tripoles/plain amp/up pos/.initial=.45}          % Top and bottom anchor position

% changing fonts and symbols of amplifiers
\ctikzset{amplifiers/symbol font/.code={%
        \ctikzset{tripoles/inst amp ra/font=#1}%
        \ctikzset{tripoles/fd inst amp/font=#1}%
        \ctikzset{tripoles/inst amp/font=#1}%
        \ctikzset{tripoles/gm amp/font=#1}%
        \ctikzset{tripoles/en amp/font=#1}%
        \ctikzset{tripoles/fd op amp/font=#1}%
        \ctikzset{tripoles/op amp/font=#1}%
}}
\ctikzset{amplifiers/plus/.initial={$+$}}
\ctikzset{amplifiers/minus/.initial={$-$}}
\tikzset{amp symbol font/.code={%
        \ctikzset{amplifiers/symbol font={#1}}%
    }
}
\tikzset{amp plus/.code={\ctikzsetvalof{amplifiers/plus}{#1}}}
\tikzset{amp minus/.code={\ctikzsetvalof{amplifiers/minus}{#1}}}
\def\pgf@circ@ampli@plus{\ctikzvalof{amplifiers/plus}}
\def\pgf@circ@ampli@minus{\ctikzvalof{amplifiers/minus}}

\ctikzset{tripoles/thyristor/height/.initial=1.10}
\ctikzset{tripoles/thyristor/height 2/.initial=.5}
\ctikzset{tripoles/thyristor/width/.initial=1.0}
\ctikzset{tripoles/thyristor/diode height/.initial=.5}
\ctikzset{tripoles/thyristor/diode width left/.initial=.4}
\ctikzset{tripoles/thyristor/diode width right/.initial=.3}

\ctikzset{tripoles/triac/height/.initial=1.1}
\ctikzset{tripoles/triac/width/.initial=1}
\ctikzset{tripoles/triac/diode width left/.initial=.3}
\ctikzset{tripoles/triac/diode width right/.initial=.3}

%% Cute switches
\ctikzset{bipoles/cuteswitch/shape/.initial={ocirc}}
\ctikzset{bipoles/cuteswitch/height/.initial=.6}
\ctikzset{bipoles/cuteswitch/height 2/.initial=.2}
\ctikzset{bipoles/cuteswitch/width/.initial=.50}
\ctikzset{bipoles/cuteswitch/thickness/.initial=1}

\ctikzset{tripoles/spdt/width/.initial=.85}
\ctikzset{tripoles/spdt/height/.initial=.45}
\ctikzset{tripoles/spdt/margin/.initial=.45}

\ctikzset{tripoles/toggleswitch/height/.initial=.8}
\ctikzset{tripoles/toggleswitch/height 2/.initial=.0}
\ctikzset{tripoles/toggleswitch/width/.initial=.80}

\ctikzset{tripoles/mixer/width/.initial=0.7}
\ctikzset{tripoles/adder/width/.initial=0.7}
\ctikzset{tripoles/circulator/width/.initial=.7}
\ctikzset{tripoles/oscillator/width/.initial=.7}


\ctikzset{tripoles/wilkinson/height/.initial=1.3}
\ctikzset{tripoles/wilkinson/width/.initial=1.3}

\ctikzset{quadpoles/transformer/inner/.initial=0.4}
\ctikzset{quadpoles/transformer/width/.initial=1.5}
\ctikzset{quadpoles/transformer/width1/.initial=.4}
\ctikzset{quadpoles/transformer/height/.initial=1.5}
\ctikzset{quadpoles/transformer/height1/.initial=.8}
\ctikzset{quadpoles/transformer core/inner/.initial=0.4}
\ctikzset{quadpoles/transformer core/width/.initial=1.5}
\ctikzset{quadpoles/transformer core/height/.initial=1.5}
\ctikzset{quadpoles/transformer core/core height/.initial=.5}
\ctikzset{quadpoles/transformer core/core width/.initial=.05}
\ctikzset{quadpoles/gyrator/inner/.initial=0.4}
\ctikzset{quadpoles/gyrator/width/.initial=1.5}
\ctikzset{quadpoles/gyrator/height/.initial=1.5}
\ctikzset{quadpoles/fourport/width/.initial=1.3}
\ctikzset{quadpoles/fourport/height/.initial=1.3}
\ctikzset{quadpoles/coupler/width/.initial=1.3}
\ctikzset{quadpoles/coupler/height/.initial=1.3}
\ctikzset{quadpoles/coupler2/width/.initial=1.3}
\ctikzset{quadpoles/coupler2/height/.initial=1.3}

\ctikzset{/tikz/circuitikz/tripoles/magnetron/width/.initial=1}

\ctikzset{quadpoles style/.is choice}
\ctikzset{quadpoles style/inward/.code={% default value
        \ctikzset{quadpoles/transformer/inner=0.4}%
        \ctikzset{quadpoles/transformer/width=1.5}%
        \ctikzset{quadpoles/transformer core/inner=0.4}%
        \ctikzset{quadpoles/transformer core/width=1.5}%
        \ctikzset{quadpoles/gyrator/inner=0.4}%
        \ctikzset{quadpoles/gyrator/width=1.5}%
    }%
}
\ctikzset{quadpoles style/inline/.code={% now horizontal baffle
        \ctikzset{quadpoles/transformer/inner=1}%
        \ctikzset{quadpoles/transformer/width=0.6}%
        \ctikzset{quadpoles/transformer core/inner=1}%
        \ctikzset{quadpoles/transformer core/width=0.6}%
        \ctikzset{quadpoles/gyrator/inner=1} % FIXME
        \ctikzset{quadpoles/gyrator/width=0.6}%
    }%
}

%
% multipoles
%
\ctikzset{multipoles/thickness/.initial=2}
\ctikzset{multipoles/font/.initial=\pgf@circ@font@tiny}
% DIP (dual in line package) chips
\ctikzset{multipoles/dipchip/width/.initial=1.2}
\ctikzset{multipoles/dipchip/num pins/.initial=8}
\ctikzset{multipoles/dipchip/pin spacing/.initial=0.4}
\pgfkeys{/tikz/num pins/.add code={}{\ctikzset{multipoles/dipchip/num pins=#1}}}
% QFP (quad flat package) chips
\ctikzset{multipoles/qfpchip/num pins/.initial=8}
\ctikzset{multipoles/qfpchip/pin spacing/.initial=0.4}
\pgfkeys{/tikz/num pins/.add code={}{\ctikzset{multipoles/qfpchip/num pins=#1}}}
% chip numbers
\newif\ifpgf@circuit@chip@shownumbers\pgf@circuit@chip@shownumberstrue
\pgfkeys{/tikz/show numbers/.add code={}{\pgf@circuit@chip@shownumberstrue}}
\ctikzset{show numbers/.add code={}{\pgf@circuit@chip@shownumberstrue}}
\pgfkeys{/tikz/hide numbers/.add code={}{\pgf@circuit@chip@shownumbersfalse}}
\ctikzset{hide numbers/.add code={}{\pgf@circuit@chip@shownumbersfalse}}
\newif\ifpgf@circuit@chip@straightnumbers\pgf@circuit@chip@straightnumberstrue
\pgfkeys{/tikz/straight numbers/.add code={}{\pgf@circuit@chip@straightnumberstrue}}
\ctikzset{straight numbers/.add code={}{\pgf@circuit@chip@straightnumberstrue}}
\pgfkeys{/tikz/rotated numbers/.add code={}{\pgf@circuit@chip@straightnumbersfalse}}
\ctikzset{rotated numbers/.add code={}{\pgf@circuit@chip@straightnumbersfalse}}
% external chip pins
\ctikzset{multipoles/external pins thickness/.initial=1}
\ctikzset{multipoles/external pins width/.initial=0.2}
\ctikzset{multipoles/external pad fraction/.initial=0}
\pgfkeys{/tikz/external pins width/.add code={}{\ctikzset{multipoles/external pins width=#1}}}
\pgfkeys{/tikz/external pad fraction/.add code={}{\ctikzset{multipoles/external pad fraction=#1}}}
% topmarks
\newif\ifpgf@circuit@chip@topmark\pgf@circuit@chip@topmarktrue
\pgfkeys{/tikz/topmark/.add code={}{\pgf@circuit@chip@topmarktrue}}
\ctikzset{topmark/.add code={}{\pgf@circuit@chip@topmarktrue}}
\pgfkeys{/tikz/no topmark/.add code={}{\pgf@circuit@chip@topmarkfalse}}
\ctikzset{no topmark/.add code={}{\pgf@circuit@chip@topmarkfalse}}

% rotary switch by Romano
\ctikzset{multipoles/rotary/thickness/.initial=1}
\ctikzset{multipoles/rotary/shape/.initial=ocirc}
\ctikzset{multipoles/rotary/channels/.initial=5}
\ctikzset{multipoles/rotary/angle/.initial=60}
\ctikzset{multipoles/rotary/wiper/.initial=20}
\ctikzset{multipoles/rotary/arrow/.is choice}
\newif\ifpgf@circ@rotaryarrow\pgf@circ@rotaryarrowfalse
\newif\ifpgf@circ@rotaryarrow@cw\pgf@circ@rotaryarrow@cwfalse
\newif\ifpgf@circ@rotaryarrow@ccw\pgf@circ@rotaryarrow@ccwfalse
\ctikzset{multipoles/rotary/arrow/none/.code={\pgf@circ@rotaryarrowfalse\pgf@circ@rotaryarrow@cwfalse\pgf@circ@rotaryarrow@ccwfalse}}
\ctikzset{multipoles/rotary/arrow/both/.code={\pgf@circ@rotaryarrowtrue\pgf@circ@rotaryarrow@cwtrue\pgf@circ@rotaryarrow@ccwtrue}}
\ctikzset{multipoles/rotary/arrow/cw/.code={\pgf@circ@rotaryarrowtrue\pgf@circ@rotaryarrow@cwtrue\pgf@circ@rotaryarrow@ccwfalse}}
\ctikzset{multipoles/rotary/arrow/ccw/.code={\pgf@circ@rotaryarrowtrue\pgf@circ@rotaryarrow@cwfalse\pgf@circ@rotaryarrow@ccwtrue}}

\tikzset{%
    rotary switch/.style args={#1in#2wiper#3}{%
        shape=rotaryswitch,
        /tikz/circuitikz/multipoles/rotary/channels=#1,
        /tikz/circuitikz/multipoles/rotary/angle=#2,
        /tikz/circuitikz/multipoles/rotary/wiper=#3,
    },
    rotary switch -/.style args={#1in#2wiper#3}{
        rotary switch=#1 in #2 wiper #3,
        /tikz/circuitikz/multipoles/rotary/arrow=none,
    },
    rotary switch <-/.style args={#1in#2wiper#3}{
        rotary switch=#1 in #2 wiper #3,
        /tikz/circuitikz/multipoles/rotary/arrow=ccw,
    },
    rotary switch ->/.style args={#1in#2wiper#3}{
        rotary switch=#1 in #2 wiper #3,
        /tikz/circuitikz/multipoles/rotary/arrow=cw,
    },
    rotary switch <->/.style args={#1in#2wiper#3}{
        rotary switch=#1 in #2 wiper #3,
        /tikz/circuitikz/multipoles/rotary/arrow=both,
    },
    % Notice that these should be the same as the initial values of the keys
    rotary switch/.default={5 in 60 wiper 20},
    rotary switch -/.default={5 in 60 wiper 20},
    rotary switch <-/.default={5 in 60 wiper 20},
    rotary switch ->/.default={5 in 60 wiper 20},
    rotary switch <->/.default={5 in 60 wiper 20},
}

% %Mechanical section
\ctikzset{/tikz/circuitikz/tripoles/elmech/height/.initial=.8}
\ctikzset{/tikz/circuitikz/tripoles/elmech/width/.initial=.6}
\ctikzset{bipoles/spring/height/.initial=.5}
\ctikzset{bipoles/spring/width/.initial=.5}
\ctikzset{bipoles/inerter/height/.initial=.7}
\ctikzset{bipoles/inerter/width/.initial=.175}
\ctikzset{bipoles/mass/height/.initial=.55}
\ctikzset{bipoles/mass/box height/.initial=.4}
\ctikzset{bipoles/mass/width/.initial=.5}

\ctikzset{bipoles/damper/height/.initial=.35}
\ctikzset{bipoles/damper/length/.initial=.3}
\ctikzset{bipoles/damper/width/.initial=.4}


\ctikzset{bipole/.is family}
\ctikzset{bipole/kind/.initial=}
\ctikzset{bipole/name/.initial=}
\newif\ifpgf@circuit@bipole@isvoltage
\ctikzset{bipole/is voltage/.is if=pgf@circuit@bipole@isvoltage}
\newif\ifpgf@circuit@bipole@voltageoutsideofsymbol
\ctikzset{bipole/is voltageoutsideofsymbol/.is if=pgf@circuit@bipole@voltageoutsideofsymbol}
\newif\ifpgf@circuit@bipole@strokedsymbol
\ctikzset{bipole/is strokedsymbol/.is if=pgf@circuit@bipole@strokedsymbol}
\newif\ifpgf@circuit@bipole@iscurrent
\ctikzset{bipole/is current/.is if=pgf@circuit@bipole@iscurrent}

\ctikzset{bipole/voltage/.is family}
\newif\ifpgf@circuit@bipole@voltage@backward
\ctikzset{bipole/voltage/direction/.is choice}
\ctikzset{bipole/voltage/direction/forward/.code={\pgf@circuit@bipole@voltage@backwardfalse}}
\ctikzset{bipole/voltage/direction/backward/.code={\pgf@circuit@bipole@voltage@backwardtrue}}
\newif\ifpgf@circuit@bipole@voltage@below
\ctikzset{bipole/voltage/position/.is choice}
\ctikzset{bipole/voltage/position/above/.code={\pgf@circuit@bipole@voltage@belowfalse}}
\ctikzset{bipole/voltage/position/below/.code={\pgf@circuit@bipole@voltage@belowtrue}}

\ctikzset{bipole/voltage/label/unit/.initial=}
\ctikzset{bipole/voltage/label/name/.initial=}

\ctikzset{bipole/current/.is family}
\newif\ifpgf@circuit@bipole@current@backward
\ctikzset{bipole/current/direction/.is choice}
\ctikzset{bipole/current/direction/forward/.code={\pgf@circuit@bipole@current@backwardfalse}}
\ctikzset{bipole/current/direction/backward/.code={\pgf@circuit@bipole@current@backwardtrue}}
\newif\ifpgf@circuit@bipole@current@before
\ctikzset{bipole/current/x position/.is choice}
\ctikzset{bipole/current/x position/after/.code={\pgf@circuit@bipole@current@beforefalse}}
\ctikzset{bipole/current/x position/before/.code={\pgf@circuit@bipole@current@beforetrue}}
\newif\ifpgf@circuit@bipole@current@below
\ctikzset{bipole/current/y position/.is choice}
\ctikzset{bipole/current/y position/above/.code={\pgf@circuit@bipole@current@belowfalse}}
\ctikzset{bipole/current/y position/below/.code={\pgf@circuit@bipole@current@belowtrue}}
\ctikzset{bipole/current/label/unit/.initial=}
\ctikzset{bipole/current/label/name/.initial=}

\ctikzset{bipole/flow/.is family}
\newif\ifpgf@circuit@bipole@flow@backward
\ctikzset{bipole/flow/direction/.is choice}
\ctikzset{bipole/flow/direction/forward/.code={\pgf@circuit@bipole@flow@backwardfalse}}
\ctikzset{bipole/flow/direction/backward/.code={\pgf@circuit@bipole@flow@backwardtrue}}
\newif\ifpgf@circuit@bipole@flow@before
\ctikzset{bipole/flow/x position/.is choice}
\ctikzset{bipole/flow/x position/after/.code={\pgf@circuit@bipole@flow@beforefalse}}
\ctikzset{bipole/flow/x position/before/.code={\pgf@circuit@bipole@flow@beforetrue}}
\newif\ifpgf@circuit@bipole@flow@below
\ctikzset{bipole/flow/y position/.is choice}
\ctikzset{bipole/flow/y position/above/.code={\pgf@circuit@bipole@flow@belowfalse}}
\ctikzset{bipole/flow/y position/below/.code={\pgf@circuit@bipole@flow@belowtrue}}
\ctikzset{bipole/flow/label/unit/.initial=}
\ctikzset{bipole/flow/label/name/.initial=}
\ctikzset{flow/distance/.initial = .5}
\ctikzset{flow/offset/.initial = .2}%distance between flow-arrow and conductor

\ctikzset{bipole/label/.is family}
\ctikzset{bipole/label/position/.initial=90}
\ctikzset{bipole/label/unit/.initial=}
\ctikzset{bipole/label/name/.initial=}
\ctikzset{bipole/annotation/.is family}
\ctikzset{bipole/annotation/position/.initial=-90}
\ctikzset{bipole/annotation/unit/.initial=}
\ctikzset{bipole/annotation/name/.initial=}

\newif\ifpgf@circuit@europeanresistor
\ctikzset{resistor/.is choice}
\ctikzset{resistor/american/.code = \pgf@circuit@europeanresistorfalse }
\ctikzset{resistor/european/.code = \pgf@circuit@europeanresistortrue }

\ctikzset{inductor/.is choice}
\ctikzset{inductor/cute/.code={\ctikzsetvalof{inductor}{cute}}}
\ctikzset{inductor/european/.code={\ctikzsetvalof{inductor}{european}}}
\ctikzset{inductor/american/.code={\ctikzsetvalof{inductor}{american}}}

%
% Flipping arrows in LED and photodiodes
%
\newif\ifpgf@led@fliparrows
\newif\ifpgf@pd@fliparrows
\pgf@led@fliparrowsfalse
\pgf@pd@fliparrowsfalse
%
% by default the arrows start (LED) and go (PD) to the anode.
%
\ctikzset{led arrows from anode/.code=\pgf@led@fliparrowsfalse}
\ctikzset{pd arrows to anode/.code=\pgf@pd@fliparrowsfalse}
%
% but they can start form the cathode (LED) or go to it (PD)
%
\ctikzset{led arrows from cathode/.code=\pgf@led@fliparrowstrue}
\ctikzset{pd arrows to cathode/.code=\pgf@pd@fliparrowstrue}

\newif\ifpgf@circuit@strokediode
\newif\ifpgf@circuit@fulldiode
\pgf@circuit@strokediodefalse
\pgf@circuit@fulldiodefalse

\ctikzset{diode/.is choice}
\ctikzset{diode/empty/.code = \pgf@circuit@fulldiodefalse\pgf@circuit@strokediodefalse}%default
\ctikzset{diode/full/.code = \pgf@circuit@fulldiodetrue }
\ctikzset{diode/stroke/.code = \pgf@circuit@strokediodetrue}

\tikzset{full diodes/.style = { \circuitikzbasekey/diode = full}}
\tikzset{empty diodes/.style = { \circuitikzbasekey/diode = empty}}
\tikzset{stroke diodes/.style = { \circuitikzbasekey/diode = stroke}}

\ctikzset{thickness/.initial=2}
\ctikzset{color/.initial=black}
\pgfkeys{/tikz/color/.add code={}{\ctikzset{color={#1}}}}

\newif\ifpgf@circ@siunitx
\newif\ifpgf@circ@siunitx@res

%
% voltage and current options
%

\newif\ifpgf@circuit@europeancurrent
\newif\ifpgf@circuit@europeanvoltage
\newif\ifpgf@circuit@bipole@voltage@straight

\ctikzset{voltage/.is choice}
%
% straight is expected to be a subset of european, so disable it in american style
%
\ctikzset{voltage/american/.code = {\pgf@circuit@europeanvoltagefalse\pgf@circuit@bipole@voltage@straightfalse}}
\ctikzset{voltage/european/.code = {\pgf@circuit@europeanvoltagetrue\pgf@circuit@bipole@voltage@straightfalse}}
\ctikzset{voltage/straight/.code = {\pgf@circuit@europeanvoltagetrue\pgf@circuit@bipole@voltage@straighttrue}}
\ctikzset{voltage/curved/.code = {\pgf@circuit@europeanvoltagetrue\pgf@circuit@bipole@voltage@straighttrue}}

\ctikzset{current/.is choice}
\ctikzset{current/american/.code = \pgf@circuit@europeancurrentfalse}
\ctikzset{current/european/.code = \pgf@circuit@europeancurrenttrue}

% this is left for backward compatibility...
\ctikzset{straight/.is choice}
\ctikzset{straight/true/.code = {\pgf@circuit@bipole@voltage@straighttrue}}
\ctikzset{straight/false/.code = {\pgf@circuit@bipole@voltage@straightfalse}}
\ctikzset{bipole/straight/.is if=pgf@circuit@bipole@voltage@straight}
% never used, removed (RG 2020-05-17)
% \ctikzset{straightvoltage value/.initial=true}
% \ctikzset{straightvoltage/.style = {/tikz/circuitikz/straight=true}}

%
% voltage is used also to set parameters, apart for the /.is choice
% above. I hope it is ok --- would be a mess otherwise
%

\ctikzset{voltage/distance from node/.initial=.5} %\pgf@circ@Rlen units
\ctikzset{voltage/distance from line/.initial=.08} % pos, tra 0 e 1
\ctikzset{voltage/bump a/.initial=1.2}
\ctikzset{voltage/bump b/.initial=1.5}
\ctikzset{voltage/shift/.initial=0.0} % shift form the cable of voltage symbols
\ctikzset{voltage shift/.style={voltage/shift=#1}}
\tikzset{voltage shift/.style={\circuitikzbasekey/voltage/shift=#1}}
\ctikzset{voltage/european label distance/.initial=1.4}
\ctikzset{voltage/american label distance/.initial=1.1}
% shaping the +/- sign, see pgfcircvoltage.tex
\ctikzset{voltage/american font/.initial={}}
\ctikzset{voltage/american plus/.initial={$+$}}
\ctikzset{voltage/american minus/.initial={$-$}}

% special cases for voltage positions
\ctikzset{bipoles/generic/voltage/distance from node/.initial=.4}
\ctikzset{bipoles/generic/voltage/bump b/.initial=2}
\ctikzset{bipoles/generic/voltage/european label distance/.initial=1.8}
%\ctikzset{bipoles/thermocouple/voltage/distance from node/.initial=.3}
\ctikzset{bipoles/thermocouple/voltage/bump b/.initial=2.2}
\ctikzset{bipoles/thermocouple/voltage/european label distance/.initial=1.5}
\ctikzset{bipoles/ageneric/voltage/distance from node/.initial=.4}
\ctikzset{bipoles/ageneric/voltage/bump b/.initial=2}
\ctikzset{bipoles/ageneric/voltage/european label distance/.initial=1.8}
\ctikzset{bipoles/fullgeneric/voltage/distance from node/.initial=.4}
\ctikzset{bipoles/fullgeneric/voltage/bump b/.initial=2}
\ctikzset{bipoles/fullgeneric/voltage/european label distance/.initial=1.8}
\ctikzset{bipoles/memristor/voltage/distance from node/.initial=.4}
\ctikzset{bipoles/memristor/voltage/bump b/.initial=2}
\ctikzset{bipoles/memristor/voltage/european label distance/.initial=1.8}
%\ctikzset{bipoles/tline/voltage/distance from node/.initial=.2}
\ctikzset{bipoles/tline/voltage/bump b/.initial=2.4}
\ctikzset{bipoles/tline/voltage/european label distance/.initial=2.1}
%\ctikzset{bipoles/varistor/voltage/distance from node/.initial=.2}
\ctikzset{bipoles/varistor/voltage/bump b/.initial=2.4}
\ctikzset{bipoles/varistor/voltage/european label distance/.initial=2}
%\ctikzset{bipoles/photoresistor/voltage/distance from node/.initial=.2}
\ctikzset{bipoles/photoresistor/voltage/bump b/.initial=2}
\ctikzset{bipoles/photoresistor/voltage/european label distance/.initial=1.8}
%\ctikzset{bipoles/thermistor/voltage/distance from node/.initial=.2}
\ctikzset{bipoles/thermistor/voltage/bump b/.initial=2.4}
\ctikzset{bipoles/thermistor/voltage/european label distance/.initial=2}
%\ctikzset{bipoles/thermistorntc/voltage/distance from node/.initial=.2}
\ctikzset{bipoles/thermistorntc/voltage/bump b/.initial=1.6}
%\ctikzset{bipoles/thermistorptc/voltage/distance from node/.initial=.2}
\ctikzset{bipoles/thermistorptc/voltage/bump b/.initial=1.6}
%
% american open voltage adjusting
%
\newif\ifpgf@adjust@open@voltage\pgf@adjust@open@voltagetrue
\ctikzset{american open voltage/.is choice}
\ctikzset{american open voltage/center/.code={\pgf@adjust@open@voltagetrue}}
\ctikzset{american open voltage/legacy/.code={\pgf@adjust@open@voltagefalse}}

%
% currents
%
\ctikzset{current/distance/.initial = .5}

%
% voltage and current styles
%

\tikzset{american currents/.style = {\circuitikzbasekey/current = american}}
\tikzset{european currents/.style = {\circuitikzbasekey/current = european}}
\tikzset{american voltages/.style = {\circuitikzbasekey/voltage = american}}
\tikzset{european voltages/.style = {\circuitikzbasekey/voltage = european}}
\tikzset{straight voltages/.style = {\circuitikzbasekey/voltage = straight}}
\tikzset{american resistors/.style = {\circuitikzbasekey/resistor = american}}
\tikzset{european resistors/.style = {\circuitikzbasekey/resistor = european}}

\newif\ifpgf@circuit@europeangfsurgearrester
\ctikzset{gas filled surge arrester choice/.is choice}
\ctikzset{gas filled surge arrester choice/european/.code= {\pgf@circuit@europeangfsurgearrestertrue}}
\ctikzset{gas filled surge arrester choice/american/.code= {\pgf@circuit@europeangfsurgearresterfalse}}

\tikzset{american gas filled surge arrester set/.style = {\circuitikzbasekey/gas filled surge arrester choice=american}}
\tikzset{european gas filled surge arrester set/.style = {\circuitikzbasekey/gas filled surge arrester choice=european}}
\tikzset{american inductors/.style = {\circuitikzbasekey/inductor = american}}
\tikzset{european inductors/.style = {\circuitikzbasekey/inductor = european}}
\tikzset{cute inductors/.style = {\circuitikzbasekey/inductor = cute}}
\tikzset{american ports/.style = {\circuitikzbasekey/logic ports = american}}
\tikzset{european ports/.style = {\circuitikzbasekey/logic ports = european}}

%
% general styles
%

\tikzset{european/.style = {european currents, european voltages, european resistors, european inductors, european ports, european gas filled surge arrester set}}
\tikzset{american/.style = {american currents, american voltages, american resistors, american inductors, american ports, american gas filled surge arrester set}}
\tikzset{cute/.style = {european currents, european voltages, american resistors, cute inductors, american ports}}


% noise sources
\ctikzset{bipoles/noise sources/fillcolor/.initial=gray!50}

\ctikzset{label/align/.is choice}
\ctikzset{label/align/straight/.code={\ctikzsetvalof{label/align}{straight}}}
\ctikzset{label/align/rotate/.code={\ctikzsetvalof{label/align}{rotate}}}
\ctikzset{label/align/smart/.code={\ctikzsetvalof{label/align}{smart}}}

% seven segment displays by RGtti

\newif\ifpgf@circ@sevenseg@dot
\newif\ifpgf@circ@sevenseg@box
\def\pgf@circ@sevenseg@dotstate{empty}
\ctikzset{seven seg/.is family}
\ctikzset{seven seg/dot/.is choice}
% none means no dot, not space for it. Empty means no dot, but space
\ctikzset{seven seg/dot/none/.code={\pgf@circ@sevenseg@dotfalse}}
\ctikzset{seven seg/dot/empty/.code={\pgf@circ@sevenseg@dottrue\def\pgf@circ@sevenseg@dotstate{empty}}}
\ctikzset{seven seg/dot/off/.code={\pgf@circ@sevenseg@dottrue\def\pgf@circ@sevenseg@dotstate{off}}}
\ctikzset{seven seg/dot/on/.code={\pgf@circ@sevenseg@dottrue\def\pgf@circ@sevenseg@dotstate{on}}}
%
\ctikzset{seven seg/width/.initial=0.4}% relative to \pgf@circ@Rlen
\ctikzset{seven seg/thickness/.initial=4pt}% segment thickness
\ctikzset{seven seg/segment sep/.initial=0.2pt}% gap between segments
\ctikzset{seven seg/box sep/.initial=1pt}% external box gap
\ctikzset{seven seg/color on/.initial=red}% color for segment "on"
\ctikzset{seven seg/color off/.initial=gray!20!white} % ...and "off"
\ctikzset{seven seg/box/.is choice}
\ctikzset{seven seg/box/off/.code={\pgf@circ@sevenseg@boxfalse}}
\ctikzset{seven seg/box/on/.code={\pgf@circ@sevenseg@boxtrue}}

\ctikzset{seven seg/bits/.initial=0000000}

\ctikzset{seven seg/value/.code={%
    \edef\@@tmp{#1}%
    \edef\@@n{0} \ifx\@@tmp\@@n\relax\ctikzset{seven seg/bits=1111110}\fi
    \edef\@@n{1} \ifx\@@tmp\@@n\relax\ctikzset{seven seg/bits=0110000}\fi
    \edef\@@n{2} \ifx\@@tmp\@@n\relax\ctikzset{seven seg/bits=1101101}\fi
    \edef\@@n{3} \ifx\@@tmp\@@n\relax\ctikzset{seven seg/bits=1111001}\fi
    \edef\@@n{4} \ifx\@@tmp\@@n\relax\ctikzset{seven seg/bits=0110011}\fi
    \edef\@@n{5} \ifx\@@tmp\@@n\relax\ctikzset{seven seg/bits=1011011}\fi
    \edef\@@n{6} \ifx\@@tmp\@@n\relax\ctikzset{seven seg/bits=1011111}\fi
    \edef\@@n{7} \ifx\@@tmp\@@n\relax\ctikzset{seven seg/bits=1110000}\fi
    \edef\@@n{8} \ifx\@@tmp\@@n\relax\ctikzset{seven seg/bits=1111111}\fi
    \edef\@@n{9} \ifx\@@tmp\@@n\relax\ctikzset{seven seg/bits=1111011}\fi
    \edef\@@n{10}\ifx\@@tmp\@@n\relax\ctikzset{seven seg/bits=1110111}\fi
    \edef\@@n{11}\ifx\@@tmp\@@n\relax\ctikzset{seven seg/bits=0011111}\fi
    \edef\@@n{12}\ifx\@@tmp\@@n\relax\ctikzset{seven seg/bits=1001110}\fi
    \edef\@@n{13}\ifx\@@tmp\@@n\relax\ctikzset{seven seg/bits=0111101}\fi
    \edef\@@n{14}\ifx\@@tmp\@@n\relax\ctikzset{seven seg/bits=1001111}\fi
    \edef\@@n{15}\ifx\@@tmp\@@n\relax\ctikzset{seven seg/bits=1000111}\fi
    \edef\@@n{A} \ifx\@@tmp\@@n\relax\ctikzset{seven seg/bits=1110111}\fi
    \edef\@@n{B} \ifx\@@tmp\@@n\relax\ctikzset{seven seg/bits=0011111}\fi
    \edef\@@n{C} \ifx\@@tmp\@@n\relax\ctikzset{seven seg/bits=1001110}\fi
    \edef\@@n{D} \ifx\@@tmp\@@n\relax\ctikzset{seven seg/bits=0111101}\fi
    \edef\@@n{E} \ifx\@@tmp\@@n\relax\ctikzset{seven seg/bits=1001111}\fi
    \edef\@@n{F} \ifx\@@tmp\@@n\relax\ctikzset{seven seg/bits=1000111}\fi
    \edef\@@n{a} \ifx\@@tmp\@@n\relax\ctikzset{seven seg/bits=1110111}\fi
    \edef\@@n{b} \ifx\@@tmp\@@n\relax\ctikzset{seven seg/bits=0011111}\fi
    \edef\@@n{c} \ifx\@@tmp\@@n\relax\ctikzset{seven seg/bits=1001110}\fi
    \edef\@@n{d} \ifx\@@tmp\@@n\relax\ctikzset{seven seg/bits=0111101}\fi
    \edef\@@n{e} \ifx\@@tmp\@@n\relax\ctikzset{seven seg/bits=1001111}\fi
    \edef\@@n{f} \ifx\@@tmp\@@n\relax\ctikzset{seven seg/bits=1000111}\fi
    \edef\@@n{-} \ifx\@@tmp\@@n\relax\ctikzset{seven seg/bits=0000001}\fi
}}

\tikzset{%
    seven segment val/.style args={#1dot#2box#3}{%
        shape=bare7seg,
        /tikz/circuitikz/seven seg/value=#1,
        /tikz/circuitikz/seven seg/dot=#2,
        /tikz/circuitikz/seven seg/box=#3,
    },
    seven segment bits/.style args={#1dot#2box#3}{%
        shape=bare7seg,
        /tikz/circuitikz/seven seg/bits=#1,
        /tikz/circuitikz/seven seg/dot=#2,
        /tikz/circuitikz/seven seg/box=#3,
    },
}


%%%%%%%%%% new
\pgfkeys{\circuitikzbasekey/.search also={/tikz}}

% Electronic tubes: diodetube, triode, tetrode and pentode
\ctikzset{tubes/width/.initial=1}                    % relative width
\ctikzset{tubes/height/.initial=1.4}                 % relative height
\ctikzset{tubes/tube radius/.initial=0.40}           % radius of tube circle
\ctikzset{tubes/anode distance/.initial=0.40}        % distance from center
\ctikzset{tubes/anode width/.initial=0.40}           % width of an anode/plate
\ctikzset{tubes/grid protrusion/.initial=0.25}       % distance from center
\ctikzset{tubes/grid dashes/.initial=5}              % number of grid dashes
\ctikzset{tubes/grid separation/.initial=0.2}        % separation between grids
\ctikzset{tubes/grid shift/.initial=0.0}             % y shift grids from center
\ctikzset{tubes/cathode distance/.initial=0.40}      % distance from grid
\ctikzset{tubes/cathode width/.initial=0.40}         % width of an cathode
\ctikzset{tubes/cathode corners/.initial=0.06}       % corners of the cathode wire
\ctikzset{tubes/cathode right extend/.initial=0.075} % extension at the right side
\ctikzset{tubes/filament distance/.initial=0.1}      % distance from cathode
\ctikzset{tubes/filament angle/.initial=15}          % Angle from centerpoint

\newif\ifpgf@circuit@tubes@filament\pgf@circuit@tubes@filamentfalse
\pgfkeys{/tikz/filament/.add code={}{\pgf@circuit@tubes@filamenttrue}}
\ctikzset{tubes/filament/.add code={}{\pgf@circuit@tubes@filamenttrue}}
\newif\ifpgf@circuit@tubes@nocathode\pgf@circuit@tubes@nocathodefalse
\pgfkeys{/tikz/nocathode/.add code={}{\pgf@circuit@tubes@nocathodetrue}}
\ctikzset{tubes/nocathode/.add code={}{\pgf@circuit@tubes@nocathodetrue}}
\newif\ifpgf@circuit@tubes@fullcathode\pgf@circuit@tubes@fullcathodefalse
\pgfkeys{/tikz/fullcathode/.add code={}{\pgf@circuit@tubes@fullcathodetrue}}
\ctikzset{tubes/fullcathode/.add code={}{\pgf@circuit@tubes@fullcathodetrue}}

% powerelectronic blocks
\ctikzset{bipoles/sacdc/width/.initial=.7}
\ctikzset{bipoles/sdcac/width/.initial=.7}
\ctikzset{bipoles/tacdc/width/.initial=.7}
\ctikzset{bipoles/tdcac/width/.initial=.7}
\ctikzset{quadpoles/gridnode/width/.initial=.7} %not sure if quadpole?

%%%---------- close: tex/pgfcirc.defines
%%%%%%%%%%% Springe nach tex/pgfcircutils
%%%---------- open: tex/pgfcircutils.tex
% Copyright 2018-2020 by Romano Giannetti
% Copyright 2015-2020 by Stefan Lindner
% Copyright 2013-2020 by Stefan Erhardt
% Copyright 2007-2020 by Massimo Redaelli
%
% This file may be distributed and/or modified
%
% 1. under the LaTeX Project Public License and/or
% 2. under the GNU Public License.
%
% See the files gpl-3.0_license.txt and lppl-1-3c_license.txt for more details.

\def\pgf@circ@handleSI#1{
    \noexpandarg
    \def\pgf@temp{}
    \StrBetween{#1}{<}{>}[\pgf@circ@handleSI@unit]
    \StrLen{\pgf@circ@handleSI@unit}[\pgf@circ@handleSI@unit@len]

    \ifnum\pgf@circ@handleSI@unit@len=0
    \pgf@circ@siunitx@resfalse
    \else
    \IfEndWith{#1}{>}{
        \pgf@circ@siunitx@restrue
        \noexpandarg
        \StrBefore{#1}{<}[\pgf@circ@handleSI@val]
        %\typeout{si |#1|}
        }{
        \pgf@circ@siunitx@resfalse
        %\typeout{no si |#1|}
    }
\fi
}

\def\pgf@circ@ifkeyempty#1{
    \pgfextra{
        \ctikzset{#1/.get=\pgf@circ@temp}
        \edef\pgf@temp{}
    }
    \ifx\pgf@circ@temp\pgf@temp
    }

%%%%%%%%%%%%%%%%%%%%%%%%%%%%%%%%
%%    Math routines

\def\pgf@circ@stripdecimals#1.#2\pgf@nil{#1}

%%%%%%%%%%%%%%%%%%%%%%%%%%%%%
%% useful commands

\ifpgfutil@format@is@latex
    %% flipping text
    \def\ctikzflipx#1{\scalebox{-1}[1]{#1}}
    \def\ctikzflipy#1{\scalebox{1}[-1]{#1}}
    \def\ctikzflipxy#1{\scalebox{-1}[-1]{#1}}
    % text mode overbar
    % Thanks to @egreg https://tex.stackexchange.com/a/24133/38080
    \def\ctikztextnot#1{$\overline{\hbox{#1}}\m@th$}
\else\ifpgfutil@format@is@plain
    % text mode overbar
    % Thanks to @egreg https://tex.stackexchange.com/a/24133/38080
    \def\ctikztextnot#1{$\overline{\hbox{#1}}$}
\else\ifpgfutil@format@is@context
    % text mode overbar
    % Thanks to @egreg https://tex.stackexchange.com/a/24133/38080
    \def\ctikztextnot#1{$\overline{\hbox{#1}}$}
\fi\fi\fi

%%%%%%%%%%%%%%%%%%%%%%%%%%%
%% switch to use fpu in reciprocal scale transformations
%%
%% this code has been contributed by Schrödinger's cat
%% https://tex.stackexchange.com/a/529159/38080
%%
\tikzset{use fpu reciprocal/.code={%
\def\pgfmathreciprocal@##1{%
    \begingroup
    \pgfkeys{/pgf/fpu=true,/pgf/fpu/output format=fixed}%
    \pgfmathparse{1/##1}%
    \pgfmath@smuggleone\pgfmathresult
    \endgroup
}}}%

%%%---------- close: tex/pgfcircutils

%%%%%%%%%%% Springe nach tex/pgfcircshapes
%%%---------- open: tex/pgfcircshapes.tex
% Copyright 2018-2020 by Romano Giannetti
% Copyright 2015-2020 by Stefan Lindner
% Copyright 2013-2020 by Stefan Erhardt
% Copyright 2007-2020 by Massimo Redaelli
%
% This file may be distributed and/or modified
%
% 1. under the LaTeX Project Public License and/or
% 2. under the GNU Public License.
%
% See the files gpl-3.0_license.txt and lppl-1-3c_license.txt for more details.

%%%%%%%%%%%%%%%%%%%%%%%%%%%%%%%%%%%%%%%%
%%  Other shapes

%% Nothing

\pgfdeclareshape{emptyshape}{
    \savedanchor{\northeast}{%
        \pgf@x=.5\wd\pgfnodeparttextbox%
        \pgf@y=.5\ht\pgfnodeparttextbox%
    }
    \anchor{north}{\northeast\pgf@x=0cm\relax}
    \anchor{east}{\northeast\pgf@y=0cm\relax}
    \anchor{south}{\northeast\pgf@y=-\pgf@y \pgf@x=0cm\relax}
    \anchor{west}{\northeast\pgf@y=0cm\pgf@x=-\pgf@x}
    \anchor{north east}{\northeast}
    \anchor{north west}{\northeast\pgf@x=-\pgf@x}
    \anchor{south east}{\northeast\pgf@y=-\pgf@y}
    \anchor{south west}{\northeast\pgf@y=-\pgf@y\pgf@x=-\pgf@x}
    \anchor{text}{\pgfpoint{-.5\wd\pgfnodeparttextbox}{\dimexpr.5\dp\pgfnodeparttextbox-.5\ht\pgfnodeparttextbox}}
    \anchor{center}{
        \pgfpointorigin
    }
}
%
% Provision for changing opacity. Only expert use, see the manual.
%
\ctikzset{poles/open fill opacity/.initial=1.0}% better not touch it
\tikzset{open poles opacity/.code={%
        \ctikzset{poles/open fill opacity=#1}%
}}
\ctikzset{poles/full fill opacity/.initial=1.0}% better not touch it
\tikzset{full poles opacity/.code={%
        \ctikzset{poles/full fill opacity=#1}%
}}

%% Full terminal

\pgfdeclareshape{circ}{
    \anchor{center}{
        \pgfpointorigin
    }
    \savedanchor\northwest{%
        \pgf@y=\ctikzvalof{nodes width}\pgf@circ@Rlen
        \pgf@x=-\pgf@y
    }
    \anchor{center}{ \pgf@y=0pt \pgf@x=0pt }
    \anchor{east}{ \northwest \pgf@y=0pt \pgf@x=-\pgf@x  }
    \anchor{e}{ \northwest \pgf@y=0pt \pgf@x=-\pgf@x  }
    \anchor{west}{ \northwest \pgf@y=0pt }
    \anchor{w}{ \northwest \pgf@y=0pt }
    \anchor{south}{ \northwest \pgf@x=0pt \pgf@y=-\pgf@y }
    \anchor{s}{ \northwest \pgf@x=0pt \pgf@y=-\pgf@y }
    \anchor{north}{ \northwest \pgf@x=0pt }
    \anchor{n}{ \northwest \pgf@x=0pt }
    \anchor{south west}{ \northwest \pgf@y=-\pgf@y }
    \anchor{north east}{ \northwest \pgf@x=-\pgf@x }
    \anchor{north west}{ \northwest }
    \anchor{south east}{ \northwest \pgf@x=-\pgf@x \pgf@y=-\pgf@y }
    \anchorborder{
        \pgf@circ@res@left=\pgf@x
        \pgf@circ@res@up=\pgf@y
        \pgfpointborderellipse{\pgfpoint{\pgf@circ@res@left}{\pgf@circ@res@up}
        }{\pgfpoint{\ctikzvalof{nodes width}*\pgf@circ@Rlen}{\ctikzvalof{nodes width}*\pgf@circ@Rlen}}
    }
    \behindforegroundpath{

        \pgfscope
            \pgfpathcircle{\pgfpointorigin}{\ctikzvalof{nodes width}*\pgf@circ@Rlen}
            \pgfsetcolor{\ctikzvalof{color}}
            \pgfsetfillopacity{\ctikzvalof{poles/full fill opacity}}% normally 1.0
            \pgfusepath{draw,fill}
        \endpgfscope

    }
}

%% Empty round terminal

\pgfdeclareshape{ocirc}{
    \anchor{center}{
        \pgfpointorigin
    }
    \savedanchor\northwest{%
        \pgf@y=\ctikzvalof{nodes width}\pgf@circ@Rlen
        \pgf@x=-\pgf@y
    }
    \anchor{center}{ \pgf@y=0pt \pgf@x=0pt }
    \anchor{east}{ \northwest \pgf@y=0pt \pgf@x=-\pgf@x  }
    \anchor{e}{ \northwest \pgf@y=0pt \pgf@x=-\pgf@x  }
    \anchor{west}{ \northwest \pgf@y=0pt }
    \anchor{w}{ \northwest \pgf@y=0pt }
    \anchor{south}{ \northwest \pgf@x=0pt \pgf@y=-\pgf@y }
    \anchor{s}{ \northwest \pgf@x=0pt \pgf@y=-\pgf@y }
    \anchor{north}{ \northwest \pgf@x=0pt }
    \anchor{n}{ \northwest \pgf@x=0pt }
    \anchor{south west}{ \northwest \pgf@y=-\pgf@y }
    \anchor{north east}{ \northwest \pgf@x=-\pgf@x }
    \anchor{north west}{ \northwest }
    \anchor{south east}{ \northwest \pgf@x=-\pgf@x \pgf@y=-\pgf@y }
    \anchorborder{
        \pgf@circ@res@left=\pgf@x
        \pgf@circ@res@up=\pgf@y
        \pgfpointborderellipse{\pgfpoint{\pgf@circ@res@left}{\pgf@circ@res@up}
        }{\pgfpoint{\ctikzvalof{nodes width}*\pgf@circ@Rlen}{\ctikzvalof{nodes width}*\pgf@circ@Rlen}}
    }

    \behindforegroundpath{

        \pgfscope
            \pgfpathcircle{\pgfpointorigin}{\ctikzvalof{nodes width}*\pgf@circ@Rlen}
            \pgfsetcolor{\ctikzvalof{color}}
            \ifx\tikz@fillcolor\pgfutil@empty
                % set the default fill color to white
                \pgfsetfillcolor{white}
                % ...but override it if the class is defined!
                % note that this element has no class, but will inherit it when used
                % into another component
                \pgf@circ@setifdefinedfill{draw, fill}{draw, fill}
            \else
                \pgfsetfillcolor{\tikz@fillcolor}
            \fi
            \pgfsetfillopacity{\ctikzvalof{poles/open fill opacity}}% normally 1.0
            \pgfusepath{draw,fill}
        \endpgfscope

    }
}

%% Diamond terminal

\pgfdeclareshape{diamondpole}{
    \anchor{center}{
        \pgfpointorigin
    }
    \savedanchor\northwest{%
        \pgfmathsetlength{\pgf@y}{sqrt(2)*\ctikzvalof{nodes width}*\pgf@circ@Rlen}
        \pgf@x=-\pgf@y
    }
    \anchor{center}{ \pgf@y=0pt \pgf@x=0pt }
    \anchor{east}{ \northwest \pgf@y=0pt \pgf@x=-\pgf@x  }
    \anchor{e}{ \northwest \pgf@y=0pt \pgf@x=-\pgf@x  }
    \anchor{west}{ \northwest \pgf@y=0pt }
    \anchor{w}{ \northwest \pgf@y=0pt }
    \anchor{south}{ \northwest \pgf@x=0pt \pgf@y=-\pgf@y }
    \anchor{s}{ \northwest \pgf@x=0pt \pgf@y=-\pgf@y }
    \anchor{north}{ \northwest \pgf@x=0pt }
    \anchor{n}{ \northwest \pgf@x=0pt }
    \anchor{south west}{ \northwest \pgf@y=-\pgf@y }
    \anchor{north east}{ \northwest \pgf@x=-\pgf@x }
    \anchor{north west}{ \northwest }
    \anchor{south east}{ \northwest \pgf@x=-\pgf@x \pgf@y=-\pgf@y }
    \anchorborder{
        % \typeout{IN\space X:\the\pgf@x\space Y:\the\pgf@y}
        \pgfmathsetmacro{\@@switchx}{ifthenelse(\pgf@x>0,1,-1)}
        \pgfmathsetmacro{\@@switchy}{ifthenelse(\pgf@y>0,1,-1)}
        \pgfmathsetlength{\pgf@xa}{abs(\pgf@x)}
        \pgfmathsetlength{\pgf@ya}{abs(\pgf@y)}
        \pgfextracty{\pgf@circ@res@up}{\northwest}
        % \typeout{MID\space X:\the\pgf@xa\space Y:\the\pgf@ya\space L:\the\pgf@circ@res@up}
        % \typeout{MID\space SX:\@@switchx\space SY:\@@switchy}
        \pgfpointintersectionoflines
            {\pgfpointorigin}{\pgfqpoint{\pgf@xa}{\pgf@ya}}
            {\pgfqpoint{0pt}{\pgf@circ@res@up}}{\pgfqpoint{\pgf@circ@res@up}{0pt}}
        % \typeout{CROSS \space X:\the\pgf@x\space Y:\the\pgf@y}
        \pgf@x=\@@switchx\pgf@x
        \pgf@y=\@@switchy\pgf@y
    }
    \behindforegroundpath{
        \pgfscope
            \pgfmathsetlength{\pgf@circ@res@temp}{\ctikzvalof{nodes width}*\pgf@circ@Rlen}
            \pgftransformrotate{45}
            \pgfpathrectanglecorners
            {\pgfpoint{-\pgf@circ@res@temp}{-\pgf@circ@res@temp}}
            {\pgfpoint{\pgf@circ@res@temp}{\pgf@circ@res@temp}}
            \pgfsetcolor{\ctikzvalof{color}}
            \pgfsetfillopacity{\ctikzvalof{poles/full fill opacity}}% normally 1.0
            \pgfusepath{draw,fill}
        \endpgfscope
    }
}

%% Diamond terminal, unfilled

\pgfdeclareshape{odiamondpole}{
    \anchor{center}{
        \pgfpointorigin
    }
    \savedanchor\northwest{%
        \pgfmathsetlength{\pgf@y}{sqrt(2)*\ctikzvalof{nodes width}*\pgf@circ@Rlen}
        \pgf@x=-\pgf@y
    }
    \anchor{center}{ \pgf@y=0pt \pgf@x=0pt }
    \anchor{east}{ \northwest \pgf@y=0pt \pgf@x=-\pgf@x  }
    \anchor{e}{ \northwest \pgf@y=0pt \pgf@x=-\pgf@x  }
    \anchor{west}{ \northwest \pgf@y=0pt }
    \anchor{w}{ \northwest \pgf@y=0pt }
    \anchor{south}{ \northwest \pgf@x=0pt \pgf@y=-\pgf@y }
    \anchor{s}{ \northwest \pgf@x=0pt \pgf@y=-\pgf@y }
    \anchor{north}{ \northwest \pgf@x=0pt }
    \anchor{n}{ \northwest \pgf@x=0pt }
    \anchor{south west}{ \northwest \pgf@y=-\pgf@y }
    \anchor{north east}{ \northwest \pgf@x=-\pgf@x }
    \anchor{north west}{ \northwest }
    \anchor{south east}{ \northwest \pgf@x=-\pgf@x \pgf@y=-\pgf@y }
    \anchorborder{
        % \typeout{IN\space X:\the\pgf@x\space Y:\the\pgf@y}
        \pgfmathsetmacro{\@@switchx}{ifthenelse(\pgf@x>0,1,-1)}
        \pgfmathsetmacro{\@@switchy}{ifthenelse(\pgf@y>0,1,-1)}
        \pgfmathsetlength{\pgf@xa}{abs(\pgf@x)}
        \pgfmathsetlength{\pgf@ya}{abs(\pgf@y)}
        \pgfextracty{\pgf@circ@res@up}{\northwest}
        % \typeout{MID\space X:\the\pgf@xa\space Y:\the\pgf@ya\space L:\the\pgf@circ@res@up}
        % \typeout{MID\space SX:\@@switchx\space SY:\@@switchy}
        \pgfpointintersectionoflines
            {\pgfpointorigin}{\pgfqpoint{\pgf@xa}{\pgf@ya}}
            {\pgfqpoint{0pt}{\pgf@circ@res@up}}{\pgfqpoint{\pgf@circ@res@up}{0pt}}
        % \typeout{CROSS \space X:\the\pgf@x\space Y:\the\pgf@y}
        \pgf@x=\@@switchx\pgf@x
        \pgf@y=\@@switchy\pgf@y
    }
    \behindforegroundpath{
        \pgfscope
            \pgfmathsetlength{\pgf@circ@res@temp}{\ctikzvalof{nodes width}*\pgf@circ@Rlen}
            \pgftransformrotate{45}
            \pgfpathrectanglecorners
            {\pgfpoint{-\pgf@circ@res@temp}{-\pgf@circ@res@temp}}
            {\pgfpoint{\pgf@circ@res@temp}{\pgf@circ@res@temp}}
            \pgfsetcolor{\ctikzvalof{color}}
            \ifx\tikz@fillcolor\pgfutil@empty
                % set the default fill color to white
                \pgfsetfillcolor{white}
                % ...but override it if the class is defined!
                % note that this element has no class, but will inherit it when used
                % into another component
                \pgf@circ@setifdefinedfill{draw, fill}{draw, fill}
            \else
                \pgfsetfillcolor{\tikz@fillcolor}
            \fi
            \pgfsetfillopacity{\ctikzvalof{poles/open fill opacity}}% normally 1.0
            \pgfusepath{draw,fill}
        \endpgfscope
    }
}

%% square terminal, filled

\pgfdeclareshape{squarepole}{
    \anchor{center}{
        \pgfpointorigin
    }
    \savedanchor\northwest{%
        \pgfmathsetlength{\pgf@y}{\ctikzvalof{nodes width}*\pgf@circ@Rlen}
        \pgf@x=-\pgf@y
    }
    \anchor{center}{ \pgf@y=0pt \pgf@x=0pt }
    \anchor{east}{ \northwest \pgf@y=0pt \pgf@x=-\pgf@x  }
    \anchor{e}{ \northwest \pgf@y=0pt \pgf@x=-\pgf@x  }
    \anchor{west}{ \northwest \pgf@y=0pt }
    \anchor{w}{ \northwest \pgf@y=0pt }
    \anchor{south}{ \northwest \pgf@x=0pt \pgf@y=-\pgf@y }
    \anchor{s}{ \northwest \pgf@x=0pt \pgf@y=-\pgf@y }
    \anchor{north}{ \northwest \pgf@x=0pt }
    \anchor{n}{ \northwest \pgf@x=0pt }
    \anchor{south west}{ \northwest \pgf@y=-\pgf@y }
    \anchor{north east}{ \northwest \pgf@x=-\pgf@x }
    \anchor{north west}{ \northwest }
    \anchor{south east}{ \northwest \pgf@x=-\pgf@x \pgf@y=-\pgf@y }
    \anchorborder{
        \pgf@xa=\pgf@x
        \pgf@ya=\pgf@y
        \pgfextracty{\pgf@circ@res@up}{\northwest}
        \pgfpointborderrectangle
            {\pgfqpoint{\pgf@xa}{\pgf@ya}}
            {\pgfqpoint{\pgf@circ@res@up}{\pgfqpoint{\pgf@circ@res@up}}}
    }
    \behindforegroundpath{
        \pgfscope
            \pgfmathsetlength{\pgf@circ@res@temp}{\ctikzvalof{nodes width}*\pgf@circ@Rlen}
            \pgfpathrectanglecorners
            {\pgfpoint{-\pgf@circ@res@temp}{-\pgf@circ@res@temp}}
            {\pgfpoint{\pgf@circ@res@temp}{\pgf@circ@res@temp}}
            \pgfsetcolor{\ctikzvalof{color}}
            \pgfsetfillopacity{\ctikzvalof{poles/full fill opacity}}% normally 1.0
            \pgfusepath{draw,fill}
        \endpgfscope
    }
}
%% square terminal, unfilled

\pgfdeclareshape{osquarepole}{
    \anchor{center}{
        \pgfpointorigin
    }
    \savedanchor\northwest{%
        \pgfmathsetlength{\pgf@y}{\ctikzvalof{nodes width}*\pgf@circ@Rlen}
        \pgf@x=-\pgf@y
    }
    \anchor{center}{ \pgf@y=0pt \pgf@x=0pt }
    \anchor{east}{ \northwest \pgf@y=0pt \pgf@x=-\pgf@x  }
    \anchor{e}{ \northwest \pgf@y=0pt \pgf@x=-\pgf@x  }
    \anchor{west}{ \northwest \pgf@y=0pt }
    \anchor{w}{ \northwest \pgf@y=0pt }
    \anchor{south}{ \northwest \pgf@x=0pt \pgf@y=-\pgf@y }
    \anchor{s}{ \northwest \pgf@x=0pt \pgf@y=-\pgf@y }
    \anchor{north}{ \northwest \pgf@x=0pt }
    \anchor{n}{ \northwest \pgf@x=0pt }
    \anchor{south west}{ \northwest \pgf@y=-\pgf@y }
    \anchor{north east}{ \northwest \pgf@x=-\pgf@x }
    \anchor{north west}{ \northwest }
    \anchor{south east}{ \northwest \pgf@x=-\pgf@x \pgf@y=-\pgf@y }
    \anchorborder{
        \pgf@xa=\pgf@x
        \pgf@ya=\pgf@y
        \pgfextracty{\pgf@circ@res@up}{\northwest}
        \pgfpointborderrectangle
            {\pgfqpoint{\pgf@xa}{\pgf@ya}}
            {\pgfqpoint{\pgf@circ@res@up}{\pgfqpoint{\pgf@circ@res@up}}}
    }
    \behindforegroundpath{
        \pgfscope
            \pgfmathsetlength{\pgf@circ@res@temp}{\ctikzvalof{nodes width}*\pgf@circ@Rlen}
            \pgfpathrectanglecorners
            {\pgfpoint{-\pgf@circ@res@temp}{-\pgf@circ@res@temp}}
            {\pgfpoint{\pgf@circ@res@temp}{\pgf@circ@res@temp}}
            \pgfsetcolor{\ctikzvalof{color}}
            \ifx\tikz@fillcolor\pgfutil@empty
                % set the default fill color to white
                \pgfsetfillcolor{white}
                % ...but override it if the class is defined!
                % note that this element has no class, but will inherit it when used
                % into another component
                \pgf@circ@setifdefinedfill{draw, fill}{draw, fill}
            \else
                \pgfsetfillcolor{\tikz@fillcolor}
            \fi
            \pgfsetfillopacity{\ctikzvalof{poles/open fill opacity}}% normally 1.0
            \pgfusepath{draw,fill}
        \endpgfscope
    }
}
% BNC connector

\pgfdeclareshape{bnc}{
    \anchor{center}{
        \pgfpointorigin
    }
    % BNC size is 2.5 times the size of the internal "ocirc"
    \savedanchor\northwest{%
        \pgf@y=\ctikzvalof{nodes width}\pgf@circ@Rlen
        \pgf@y=2.5\pgf@y
        \pgf@x=-\pgf@y
    }
    % center is on the opening
    \anchor{center}{ \northwest \pgf@y=0pt \pgf@x=-\pgf@x  }
    \anchor{zero}{ \pgfpointorigin }
    \anchor{hot}{ \northwest \pgf@y=0pt \pgf@x=-\pgf@x  }
    \anchor{shield}{ \northwest \pgf@x=0pt \pgf@y=-\pgf@y }
    % geo-anchors
    \anchor{east}{ \northwest \pgf@y=0pt \pgf@x=-\pgf@x  }
    \anchor{right}{ \northwest \pgf@y=0pt \pgf@x=-\pgf@x  }
    \anchor{west}{ \northwest \pgf@y=0pt }
    \anchor{left}{ \northwest \pgf@y=0pt }
    \anchor{south}{ \northwest \pgf@x=0pt \pgf@y=-\pgf@y }
    \anchor{north}{ \northwest \pgf@x=0pt }
    \anchor{south west}{ \northwest \pgf@y=-\pgf@y }
    \anchor{north east}{ \northwest \pgf@x=-\pgf@x }
    \anchor{north west}{ \northwest }
    \anchor{south east}{ \northwest \pgf@x=-\pgf@x \pgf@y=-\pgf@y }
    % put the node text above and centered
    \anchor{text}{\pgfextracty{\pgf@circ@res@up}{\northwest}
        \pgfpoint{-.5\wd\pgfnodeparttextbox}{
            \dimexpr.5\dp\pgfnodeparttextbox+.5\ht\pgfnodeparttextbox+\pgf@circ@res@up\relax
        }
    }
    \anchorborder{
        \pgf@circ@res@left=\pgf@x
        \pgf@circ@res@up=\pgf@y
        \pgfpointborderellipse{\pgfpoint{\pgf@circ@res@left}{\pgf@circ@res@up}
        }{\pgfpoint{2.5*\ctikzvalof{nodes width}*\pgf@circ@Rlen}{2.5*\ctikzvalof{nodes width}*\pgf@circ@Rlen}}
    }
    \behindforegroundpath{
        \pgfextracty{\pgf@circ@res@other}{\northwest}
        \pgf@circ@res@step=\ctikzvalof{nodes width}\pgf@circ@Rlen
        \pgfscope
            \pgfstartlinewidth=\pgflinewidth
            \pgf@circ@setlinewidth{bipoles}{\pgflinewidth}
            \pgfsetcolor{\ctikzvalof{color}}
            % external circle
            \pgfscope
                % clipping path: first a rectangle bigger then the shape
                % to avoid problems with the line thickness
                \pgfpathrectanglecorners{\pgfpoint{-2\pgf@circ@res@other}{-2\pgf@circ@res@other}}
                    {\pgfpoint{2\pgf@circ@res@other}{2\pgf@circ@res@other}}
                % next the opening to the right
                \pgfpathrectanglecorners{\pgfpoint{-\pgf@circ@res@step}{-\pgf@circ@res@step}}
                    {\pgfpoint{2\pgf@circ@res@other}{\pgf@circ@res@step}}
                % do the difference and clip before drawing
                \pgfseteorule
                \pgfusepath{clip}
                \pgfpathcircle{\pgfpointorigin}{\pgf@circ@res@other}
                \pgfusepath{draw}
            \endpgfscope
            % internal circle
            \pgfpathcircle{\pgfpointorigin}{\pgf@circ@res@step}
            \pgf@circ@draworfill
            % and the contact line to the right
            \pgfsetlinewidth{\pgfstartlinewidth}
            \pgfpathmoveto{\pgfpoint{\pgf@circ@res@step}{0pt}}
            \pgfpathlineto{\pgfpoint{\pgf@circ@res@other}{0pt}}
            \pgfusepath{draw}
        \endpgfscope
    }
}

%% Fill for correct rectangular joins

\pgfdeclareshape{rectjoinfill}{
    \savedanchor{\northeast}{%
        \pgf@x=.5\pgflinewidth%
        \pgf@y=.5\pgflinewidth%
    }
    \anchor{north}{\northeast\pgf@x=0cm\relax}
    \anchor{east}{\northeast\pgf@y=0cm\relax}
    \anchor{south}{\northeast\pgf@y=-\pgf@y \pgf@x=0cm\relax}
    \anchor{west}{\northeast\pgf@y=0cm\pgf@x=-\pgf@x}
    \anchor{north east}{\northeast}
    \anchor{north west}{\northeast\pgf@x=-\pgf@x}
    \anchor{south east}{\northeast\pgf@y=-\pgf@y}
    \anchor{south west}{\northeast\pgf@y=-\pgf@y\pgf@x=-\pgf@x}
    \anchor{center}{
        \pgfpointorigin
    }
    \anchorborder{
        \pgf@circ@res@left=\pgf@x
        \pgf@circ@res@up=\pgf@y
    }
    \behindforegroundpath{
        \pgfscope
            \pgfpathrectanglecorners
            {\pgfpoint{0}{.5\pgflinewidth}}
            {\pgfpoint{0}{-.5\pgflinewidth}}
            \pgfsetcolor{\ctikzvalof{color}}
            \pgfusepath{draw,fill}
        \endpgfscope
    }
}

%% transistor arrow

\def\pgf@circ@find@linescale{
    % find the scale inverse of the scale factor: line width do not scale
    % with scale=..., transform shape so we have to counteract it.
    \iftikz@fullytransformed % this is true if `transform shape` is active
        % from @Circumscribe https://tex.stackexchange.com/a/474035/38080
        % Note that this trick is not working inside a `spy` environment...
        \pgfgettransformentries{\scaleA}{\scaleB}{\scaleC}{\scaleD}{\whatevs}{\whatevs}%
        \pgfmathsetmacro{\@@factor}{1.0/sqrt(abs(\scaleA*\scaleD-\scaleB*\scaleC))}%
    \else
        \pgfmathsetmacro{\@@factor}{1.0}
    \fi
}

\pgfdeclareshape{trarrow}{%
    % this arrow is only filled but grows with the linewidth, more or less
    % like currarrow do
    \savedanchor{\northeast}{%
        \pgf@circ@res@step = \pgf@circ@Rlen
        \pgf@circ@find@linescale
        \divide \pgf@circ@res@step by \ctikzvalof{current arrow scale}
        \pgfpoint{0.7*\pgf@circ@res@step +0.5*\@@factor*\pgflinewidth}
            {0.8*\pgf@circ@res@step+0.7593*\@@factor*\pgflinewidth}
    }
    % The arrow size should be more or less the same of a currarrow, which is
    % both filled and stroke, for backward output compatibility (more or less)
    %
    %      angle \beta       W is \pgf@circ@Rlen/\ctikzvalof{current arrow scale}
    %    |-\__               currarrow as the tip at (W,0)
    %    |    |              and the upper tail at (-0.7*W, 0.8*W)
    %    |    \__            it then "overshoot" do to the linew width L
    %    |       \__ xangle \alpha
    %    ---0------->
    %
    %   \beta = atan(0.7/0.8)  \alpha=atan(0.8/1.7)
    %   tip overshoot is (L/2)/sin(\alpha) = 1.743*L only in x direction
    %   tail overshoot is -(L/2) in x, and (L/2)/sin(\beta) = 0.7539*L in y
    %
    \savedanchor{\northwest}{%
        \pgf@circ@res@step = \pgf@circ@Rlen
        \divide \pgf@circ@res@step by \ctikzvalof{current arrow scale}
        \pgf@circ@find@linescale
        \pgfpoint{-0.7*\pgf@circ@res@step -0.5*\@@factor*\pgflinewidth}
            {0.8*\pgf@circ@res@step+0.7593*\@@factor*\pgflinewidth}
    }
    \savedanchor{\tip}{%
        \pgf@circ@res@step = \pgf@circ@Rlen
        \divide \pgf@circ@res@step by \ctikzvalof{current arrow scale}
        \pgf@circ@find@linescale
        \pgfpoint{\pgf@circ@res@step + 1.743*\@@factor*\pgflinewidth}{0pt}
    }
    \anchor{north}{\northeast\pgf@x=0cm\relax}
    \anchor{east}{\northeast\pgf@y=0cm\relax}
    \anchor{south}{\northeast\pgf@y=-\pgf@y \pgf@x=0cm\relax}
    \anchor{west}{\northeast\pgf@y=0cm\pgf@x=-\pgf@x}
    \anchor{north east}{\northeast}
    \anchor{north west}{\northeast\pgf@x=-\pgf@x}
    \anchor{south east}{\northeast\pgf@y=-\pgf@y}
    \anchor{south west}{\northeast\pgf@y=-\pgf@y\pgf@x=-\pgf@x}
    \anchor{center}{
        \pgfpointorigin
    }
    \anchor{tip}{
        \tip
    }
    \anchor{btip}{% this anchor is behind the tip of half a linewidth
        \tip
        \pgf@circ@find@linescale
        \pgf@circ@res@temp=\@@factor\pgflinewidth
        \advance\pgf@x by -.5\pgf@circ@res@temp
    }
    \behindforegroundpath{
        \pgfscope
            \northwest
            \pgf@circ@res@up=\pgf@y
            \pgf@circ@res@left=\pgf@x
            \tip
            \pgf@circ@res@step = \pgf@x
            %
            \pgfpathmoveto{\pgfpoint{\pgf@circ@res@left}{0pt}}
            \pgfpathlineto{\pgfpoint{\pgf@circ@res@left}{\pgf@circ@res@up}}
            \pgfpathlineto{\pgfpoint{1\pgf@circ@res@step}{0pt}}
            \pgfpathlineto{\pgfpoint{\pgf@circ@res@left}{-\pgf@circ@res@up}}
            \pgfpathclose
            \pgfsetcolor{\ctikzvalof{color}}
            \pgfusepath{fill} % just fill
        \endpgfscope
    }
}

%% Current arrow

\pgfdeclareshape{currarrow}{
    \savedanchor{\northeast}{%
        \pgf@circ@res@step = \pgf@circ@Rlen
        \divide \pgf@circ@res@step by \ctikzvalof{current arrow scale}
        \pgf@x=.5\pgf@circ@res@step
        \pgf@y=\pgf@x%
    }
    \anchor{north}{\northeast\pgf@x=0cm\relax}
    \anchor{east}{\northeast\pgf@y=0cm\relax}
    \anchor{south}{\northeast\pgf@y=-\pgf@y \pgf@x=0cm\relax}
    \anchor{west}{\northeast\pgf@y=0cm\pgf@x=-\pgf@x}
    \anchor{north east}{\northeast}
    \anchor{north west}{\northeast\pgf@x=-\pgf@x}
    \anchor{south east}{\northeast\pgf@y=-\pgf@y}
    \anchor{south west}{\northeast\pgf@y=-\pgf@y\pgf@x=-\pgf@x}
    \anchor{center}{
        \pgfpointorigin
    }
    \anchor{tip}{
        \pgfpointorigin
        \pgf@circ@res@step = \pgf@circ@Rlen
        \divide \pgf@circ@res@step by \ctikzvalof{current arrow scale}
        \pgf@x	=\pgf@circ@res@step
    }
    \behindforegroundpath{
        \pgfscope
            \pgf@circ@res@step = \pgf@circ@Rlen
            \divide \pgf@circ@res@step by \ctikzvalof{current arrow scale}

            \pgfpathmoveto{\pgfpoint{-.7\pgf@circ@res@step}{0pt}}
            \pgfpathlineto{\pgfpoint{-.7\pgf@circ@res@step}{-.8\pgf@circ@res@step}}
            \pgfpathlineto{\pgfpoint{1\pgf@circ@res@step}{0pt}}
            \pgfpathlineto{\pgfpoint{-.7\pgf@circ@res@step}{.8\pgf@circ@res@step}}
            \pgfpathlineto{\pgfpoint{-.7\pgf@circ@res@step}{0pt}}
            \pgfsetcolor{\ctikzvalof{color}}
            \pgfusepath{draw,fill}

        \endpgfscope
    }
}

%% Flow arrow

\pgfdeclareshape{flowarrow}{
    \savedanchor{\northeast}{%
        \pgf@circ@res@step = \pgf@circ@Rlen
        \divide \pgf@circ@res@step by \ctikzvalof{current arrow scale}
        \pgf@y=.5\pgf@circ@res@step
        \pgf@circ@res@step = \pgf@circ@Rlen
        \divide \pgf@circ@res@step by 4
        \pgf@x=\pgf@circ@res@step%
    }
    \anchor{north}{\northeast\pgf@x=0cm\relax}
    \anchor{east}{\northeast\pgf@y=0cm\relax}
    \anchor{south}{\northeast\pgf@y=-\pgf@y \pgf@x=0cm\relax}
    \anchor{west}{\northeast\pgf@y=0cm\pgf@x=-\pgf@x}
    \anchor{north east}{\northeast}
    \anchor{north west}{\northeast\pgf@x=-\pgf@x}
    \anchor{south east}{\northeast\pgf@y=-\pgf@y}
    \anchor{south west}{\northeast\pgf@y=-\pgf@y\pgf@x=-\pgf@x}
    \anchor{text}{% text centered above
        \pgfpointorigin
        \pgfpoint{-.5\wd\pgfnodeparttextbox}{\dimexpr.5\dp\pgfnodeparttextbox+.5\ht\pgfnodeparttextbox}
    }
    \anchor{center}{
        \pgfpointorigin
    }
    \anchor{tip}{
        \pgfpointorigin
        \pgf@circ@res@step = \pgf@circ@Rlen
        \divide \pgf@circ@res@step by \ctikzvalof{current arrow scale}
        \pgf@x	=\pgf@circ@res@step
    }
    \behindforegroundpath{
        \pgfscope
            \pgf@circ@res@step = \pgf@circ@Rlen
            \divide \pgf@circ@res@step by 4
            \pgfpathmoveto{\pgfpoint{-\pgf@circ@res@step}{0pt}}
            \pgfpathlineto{\pgfpoint{\pgf@circ@res@step}{0pt}}
            \pgfsetcolor{\ctikzvalof{color}}
            \pgfusepath{draw}
            \pgftransformshift{\pgfpoint{\pgf@circ@res@step}{0pt}}
            \pgfnode{currarrow}{tip}{}{}{\pgfusepath{fill}}
        \endpgfscope
    }
}

%% Input arrow

\pgfdeclareshape{inputarrow}{
    \savedanchor{\northeast}{% this is really not northeast, really -northwest
        \pgf@circ@res@step = \pgf@circ@Rlen
        \divide \pgf@circ@res@step by \ctikzvalof{current arrow scale}
        \pgf@y=.5\pgf@circ@res@step
        \pgf@x=1.7\pgf@circ@res@step
    }
    \anchor{north}{\northeast\pgf@x=0cm\relax}
    \anchor{east}{\northeast\pgf@y=0cm\relax\pgf@x=0cm\relax}
    \anchor{south}{\northeast\pgf@y=-\pgf@y \pgf@x=0cm\relax}
    \anchor{west}{\northeast\pgf@y=0cm\pgf@x=-\pgf@x}
    \anchor{north east}{\northeast\pgf@x=0cm\relax}
    \anchor{north west}{\northeast\pgf@x=-\pgf@x}
    \anchor{south east}{\northeast\pgf@y=-\pgf@y\pgf@x=0cm\relax}
    \anchor{south west}{\northeast\pgf@y=-\pgf@y\pgf@x=-\pgf@x}
    \savedanchor{\tip}{
        \pgfpointorigin
    }
    \anchor{center}{
        \tip
    }
    \anchor{tip}{
        \tip
    }
    \behindforegroundpath{

        \pgfscope
            \pgf@circ@res@step = \pgf@circ@Rlen
            \divide \pgf@circ@res@step by 16

            \pgfpathmoveto{\pgfpoint{-1.7\pgf@circ@res@step}{0pt}}
            \pgfpathlineto{\pgfpoint{-1.7\pgf@circ@res@step}{-.8\pgf@circ@res@step}}
            \pgfpathlineto{\pgfpoint{0pt}{0pt}}
            \pgfpathlineto{\pgfpoint{-1.7\pgf@circ@res@step}{.8\pgf@circ@res@step}}
            \pgfpathlineto{\pgfpoint{-1.7\pgf@circ@res@step}{0pt}}
            \pgfsetcolor{\ctikzvalof{color}}
            \pgfusepath{fill}

        \endpgfscope
    }
}


%% box

\pgfdeclareshape{box}{
    \anchor{center}{
        \pgfpointorigin
    }
    \behindforegroundpath{

        \pgfscope
            \pgf@circ@res@step = \ctikzvalof{bipoles/twoport/width}\pgf@circ@Rlen
            \pgf@circ@res@step = 0.5\pgf@circ@res@step
            \pgf@circ@setlinewidth{bipoles}{\pgfstartlinewidth}
            \pgfpathrectanglecorners{\pgfpoint{-\pgf@circ@res@step}{\pgf@circ@res@step}}{\pgfpoint{\pgf@circ@res@step}{-\pgf@circ@res@step}}
            \pgf@circ@draworfill
        \endpgfscope
    }
}

%% box scaled with blocks

\pgfdeclareshape{blockbox}{
    \saveddimen{\scaledRlen}{\pgfmathsetlength{\pgf@x}{\ctikzvalof{blocks/scale}\pgf@circ@Rlen}}
    \anchor{center}{
        \pgfpointorigin
    }
    \behindforegroundpath{
        \pgfmathsetlength{\pgf@circ@scaled@Rlen}{\ctikzvalof{blocks/scale}\pgf@circ@Rlen}
        \pgfscope
            \pgf@circ@res@step = \ctikzvalof{bipoles/twoport/width}\pgf@circ@scaled@Rlen
            \pgf@circ@res@step = 0.5\pgf@circ@res@step
            \pgf@circ@setlinewidth{bipoles}{\pgfstartlinewidth}
            \pgfpathrectanglecorners{\pgfpoint{-\pgf@circ@res@step}{\pgf@circ@res@step}}{\pgfpoint{\pgf@circ@res@step}{-\pgf@circ@res@step}}
            \pgf@circ@draworfill
        \endpgfscope
    }
}

% full nodes for wire crossing

\pgfdeclareshape{jump crossing}
{
    \savedanchor\northwest{%
        \pgf@y=\ctikzvalof{bipoles/crossing/size}\pgf@circ@Rlen
        \pgf@y=.5\pgf@y
        \pgf@x=-\pgf@y
    }
    \anchor{center}{ \pgf@y=0pt \pgf@x=0pt }
    \anchor{east}{ \northwest \pgf@y=0pt \pgf@x=-\pgf@x  }
    \anchor{e}{ \northwest \pgf@y=0pt \pgf@x=-\pgf@x  }
    \anchor{west}{ \northwest \pgf@y=0pt }
    \anchor{w}{ \northwest \pgf@y=0pt }
    \anchor{south}{ \northwest \pgf@x=0pt \pgf@y=-\pgf@y }
    \anchor{s}{ \northwest \pgf@x=0pt \pgf@y=-\pgf@y }
    \anchor{north}{ \northwest \pgf@x=0pt }
    \anchor{n}{ \northwest \pgf@x=0pt }
    \anchor{south west}{ \northwest \pgf@y=-\pgf@y }
    \anchor{north east}{ \northwest \pgf@x=-\pgf@x }
    \anchor{north west}{ \northwest }
    \anchor{south east}{ \northwest \pgf@x=-\pgf@x \pgf@y=-\pgf@y }
    \behindbackgroundpath{
        \northwest
        \pgf@circ@res@up = \pgf@y
        \pgf@circ@res@down = -\pgf@y
        \pgf@circ@res@right = -\pgf@x
        \pgf@circ@res@left = \pgf@x
        % horizontal jumper
        \pgfpathmoveto{\pgfpoint{\pgf@circ@res@left}{0pt}}
        \pgfpathlineto{\pgfpoint{0.4\pgf@circ@res@left}{0pt}}
        \pgfpatharc{0}{-180}{0.4*\pgf@circ@res@left}
        \pgfsetbeveljoin
        \pgfpathlineto{\pgfpoint{\pgf@circ@res@right}{0pt}}
        % vertical, broken path
        \pgfpathmoveto{\pgfpoint{0pt}{\pgf@circ@res@up}}
        \pgfpathlineto{\pgfpoint{0pt}{0.5\pgf@circ@res@up}}
        \pgfpathmoveto{\pgfpoint{0pt}{0.3\pgf@circ@res@up}}
        \pgfpathlineto{\pgfpoint{0pt}{\pgf@circ@res@down}}
        \pgfusepath{draw}

    }
}
\pgfdeclareshape{plain crossing}
{
    \savedanchor\northwest{%
        \pgf@y=\ctikzvalof{bipoles/crossing/size}\pgf@circ@Rlen
        \pgf@y=.5\pgf@y
        \pgf@x=-\pgf@y
    }
    \anchor{center}{ \pgf@y=0pt \pgf@x=0pt }
    \anchor{east}{ \northwest \pgf@y=0pt \pgf@x=-\pgf@x  }
    \anchor{e}{ \northwest \pgf@y=0pt \pgf@x=-\pgf@x  }
    \anchor{west}{ \northwest \pgf@y=0pt }
    \anchor{w}{ \northwest \pgf@y=0pt }
    \anchor{south}{ \northwest \pgf@x=0pt \pgf@y=-\pgf@y }
    \anchor{s}{ \northwest \pgf@x=0pt \pgf@y=-\pgf@y }
    \anchor{north}{ \northwest \pgf@x=0pt }
    \anchor{n}{ \northwest \pgf@x=0pt }
    \anchor{south west}{ \northwest \pgf@y=-\pgf@y }
    \anchor{north east}{ \northwest \pgf@x=-\pgf@x }
    \anchor{north west}{ \northwest }
    \anchor{south east}{ \northwest \pgf@x=-\pgf@x \pgf@y=-\pgf@y }
    \behindbackgroundpath{
        \northwest
        \pgf@circ@res@up = \pgf@y
        \pgf@circ@res@down = -\pgf@y
        \pgf@circ@res@right = -\pgf@x
        \pgf@circ@res@left = \pgf@x
        % horizontal jumper
        \pgfpathmoveto{\pgfpoint{\pgf@circ@res@left}{0pt}}
        \pgfpathlineto{\pgfpoint{\pgf@circ@res@right}{0pt}}
        % vertical, broken path
        \pgfpathmoveto{\pgfpoint{0pt}{\pgf@circ@res@up}}
        \pgfpathlineto{\pgfpoint{0pt}{0.1\pgf@circ@res@up}}
        \pgfpathmoveto{\pgfpoint{0pt}{0.1\pgf@circ@res@down}}
        \pgfpathlineto{\pgfpoint{0pt}{\pgf@circ@res@down}}
        \pgfusepath{draw}

    }
}

%%%---------- close: tex/pgfcircshapes
%%%%%%%%%%% Springe nach tex/pgfcircmonopoles
%%%---------- open: tex/pgfcircmonopoles.tex
% Copyright 2018-2020 by Romano Giannetti
% Copyright 2015-2020 by Stefan Lindner
% Copyright 2013-2020 by Stefan Erhardt
% Copyright 2007-2020 by Massimo Redaelli
%
% This file may be distributed and/or modified
%
% 1. under the LaTeX Project Public License and/or
% 2. under the GNU Public License.
%
% See the files gpl-3.0_license.txt and lppl-1-3c_license.txt for more details.

%%%%%%%%%%%%%%%%%%%%%%%%%%%%%%%%%%%%%%
%% Monopoles
%%%%%%%%%%%%%%%%%%%%%%%%%%%%%%%%%%%%%%

%%%%%%%%%%%%%
%% Grounds
%%%%%%%%%%%%%


%% Ground symbol
% #1 -> name
% #2 -> width
% #3 -> depth
% #4 -> code
\long\def\pgf@circ@declareground#1#2#3#4{
    \pgfdeclareshape{#1}{
        \savedmacro{\ctikzclass}{\edef\ctikzclass{grounds}}  % class of these components
        \saveddimen{\scaledRlen}{\pgfmathsetlength{\pgf@x}{\ctikzvalof{\ctikzclass/scale}\pgf@circ@Rlen}}
        \savedanchor{\southeast}{
            \pgfmathsetlength{\pgf@circ@scaled@Rlen}{\ctikzvalof{\ctikzclass/scale}\pgf@circ@Rlen}
            \pgf@x=\ctikzvalof{monopoles/ground/width}\pgf@circ@scaled@Rlen
            \pgf@x=#2\pgf@x
            \pgf@y=\ctikzvalof{monopoles/ground/width}\pgf@circ@scaled@Rlen
            \pgf@y=-#3\pgf@y
        }
        \anchor{north}{\pgfpointorigin}
        \anchor{north east}{\southeast\pgf@y=0pt\relax}
        \anchor{east}{\southeast\pgf@y=.5\pgf@y}
        \anchor{south east}{\southeast}
        \anchor{south}{\southeast\pgf@x=0pt\relax}
        \anchor{south west}{\southeast\pgf@x=-\pgf@x}
        \anchor{west}{\southeast\pgf@y=.5\pgf@y\pgf@x=-\pgf@x}
        \anchor{north west}{\southeast\pgf@y=0pt\pgf@x=-\pgf@x}
        \anchor{left}{\pgfpointorigin}
        \anchor{right}{\pgfpointorigin}
        \anchor{center}{\pgfpointorigin}
        \behindforegroundpath{
            \pgf@circ@scaled@Rlen=\scaledRlen
            \pgf@circ@res@step=\ctikzvalof{monopoles/ground/width}\pgf@circ@scaled@Rlen
            \pgfscope
                \pgfstartlinewidth=\pgflinewidth
                #4
            \endpgfscope
        }
    }
}


\pgf@circ@declareground{ground}{0.6}{1.6}{
    \pgfsetlinewidth{\ctikzvalof{monopoles/ground/connectionthickness}\pgfstartlinewidth}
    \pgfpathmoveto{\pgfpointorigin}
    \pgfpathlineto{\pgfpoint{0pt}{-1.2\pgf@circ@res@step}}
    \pgfusepath{draw}
    \pgf@circ@setlinewidth{monopoles/ground}{\pgfstartlinewidth}
    \pgfpathmoveto{\pgfpoint{-.6\pgf@circ@res@step}{-1.2\pgf@circ@res@step}}
    \pgfpathlineto{\pgfpoint{.6\pgf@circ@res@step}{-1.2\pgf@circ@res@step}}
    \pgfpathmoveto{\pgfpoint{-.4\pgf@circ@res@step}{-1.4\pgf@circ@res@step}}
    \pgfpathlineto{\pgfpoint{.4\pgf@circ@res@step}{-1.4\pgf@circ@res@step}}
    \pgfpathmoveto{\pgfpoint{-.25\pgf@circ@res@step}{-1.6\pgf@circ@res@step}}
    \pgfpathlineto{\pgfpoint{.25\pgf@circ@res@step}{-1.6\pgf@circ@res@step}}
    \pgfusepath{draw}
}

\pgf@circ@declareground{tlground}{0.6}{0.4}{
    \pgf@circ@setlinewidth{monopoles/ground}{\pgfstartlinewidth}
    \pgfpathmoveto{\pgfpoint{-.6\pgf@circ@res@step}{0pt}}
    \pgfpathlineto{\pgfpoint{.6\pgf@circ@res@step}{0pt}}
    \pgfpathmoveto{\pgfpoint{-.4\pgf@circ@res@step}{-0.2\pgf@circ@res@step}}
    \pgfpathlineto{\pgfpoint{.4\pgf@circ@res@step}{-0.2\pgf@circ@res@step}}
    \pgfpathmoveto{\pgfpoint{-.25\pgf@circ@res@step}{-0.4\pgf@circ@res@step}}
    \pgfpathlineto{\pgfpoint{.25\pgf@circ@res@step}{-0.4\pgf@circ@res@step}}
    \pgfusepath{draw}
}


\pgf@circ@declareground{rground}{0.6}{1}{
    \pgfpathmoveto{\pgfpointorigin}
    \pgfpathlineto{\pgfpoint{0pt}{-\pgf@circ@res@step}}
    \pgfusepath{draw}

    \pgfstartlinewidth=\pgflinewidth
    \pgf@circ@setlinewidth{monopoles/rground}{\pgfstartlinewidth}

    \pgfpathmoveto{\pgfpoint{-.6\pgf@circ@res@step}{-\pgf@circ@res@step}}
    \pgfpathlineto{\pgfpoint{.6\pgf@circ@res@step}{-\pgf@circ@res@step}}
    \pgfsetroundcap\pgfusepath{draw}
}

\pgf@circ@declareground{tground}{0.6}{0}{
    \pgfpathmoveto{\pgfpointorigin}
    \pgfusepath{draw}

    \pgfstartlinewidth=\pgflinewidth
    \pgf@circ@setlinewidth{monopoles/tground}{\pgfstartlinewidth}

    \pgfpathmoveto{\pgfpoint{-.6\pgf@circ@res@step}{0pt}}
    \pgfpathlineto{\pgfpoint{.6\pgf@circ@res@step}{0pt}}
    \pgfusepath{draw}
}

\pgf@circ@declareground{sground}{0.6}{1.8}{
    \pgfpathmoveto{\pgfpointorigin}
    \pgfpathlineto{\pgfpoint{0pt}{-\pgf@circ@res@step}}
    \pgfusepath{draw}

    \pgfstartlinewidth=\pgflinewidth
    \pgf@circ@setlinewidth{bipoles}{\pgfstartlinewidth}

    \pgfpathmoveto{\pgfpoint{-.6\pgf@circ@res@step}{-\pgf@circ@res@step}}
    \pgfpathlineto{\pgfpoint{.6\pgf@circ@res@step}{-\pgf@circ@res@step}}
    \pgfpathlineto{\pgfpoint{0}{-1.8\pgf@circ@res@step}}
    \pgfpathclose
    \pgf@circ@draworfill
}

% noiseless ground
\pgf@circ@declareground{nground}{0.9}{1.6}{
    \pgfsetlinewidth{\ctikzvalof{monopoles/ground/connectionthickness}\pgfstartlinewidth}
    \pgfpathmoveto{\pgfpointorigin}
    \pgfpathlineto{\pgfpoint{0pt}{-1.2\pgf@circ@res@step}}
    \pgfusepath{draw}
    \pgf@circ@setlinewidth{monopoles/ground}{\pgfstartlinewidth}
    \pgfpathmoveto{\pgfpoint{-.6\pgf@circ@res@step}{-1.2\pgf@circ@res@step}}
    \pgfpathlineto{\pgfpoint{.6\pgf@circ@res@step}{-1.2\pgf@circ@res@step}}
    \pgfpathmoveto{\pgfpoint{-.4\pgf@circ@res@step}{-1.4\pgf@circ@res@step}}
    \pgfpathlineto{\pgfpoint{.4\pgf@circ@res@step}{-1.4\pgf@circ@res@step}}
    \pgfpathmoveto{\pgfpoint{-.25\pgf@circ@res@step}{-1.6\pgf@circ@res@step}}
    \pgfpathlineto{\pgfpoint{.25\pgf@circ@res@step}{-1.6\pgf@circ@res@step}}
    \pgfusepath{draw}
    \pgfpathmoveto{\pgfpoint{0.9\pgf@circ@res@step}{-1.6\pgf@circ@res@step}}
    \pgfpatharc{0}{180}{0.9\pgf@circ@res@step}
    \pgfusepath{draw}
}

% protective ground
\pgf@circ@declareground{pground}{0.9}{1.8}{
    \pgf@circ@setlinewidth{monopoles/ground}{\pgfstartlinewidth}
    \pgfpathcircle{\pgfpoint{0pt}{-0.9\pgf@circ@res@step}}{0.9\pgf@circ@res@step}
    \pgf@circ@draworfill
    \pgfsetlinewidth{\ctikzvalof{monopoles/ground/connectionthickness}\pgfstartlinewidth}
    \pgfpathmoveto{\pgfpointorigin}
    \pgfpathlineto{\pgfpoint{0pt}{-1\pgf@circ@res@step}}
    \pgfusepath{draw}
    \pgf@circ@setlinewidth{monopoles/ground}{\pgfstartlinewidth}
    \pgfpathmoveto{\pgfpoint{-.6\pgf@circ@res@step}{-1\pgf@circ@res@step}}
    \pgfpathlineto{\pgfpoint{.6\pgf@circ@res@step}{-1\pgf@circ@res@step}}
    \pgfpathmoveto{\pgfpoint{-.4\pgf@circ@res@step}{-1.2\pgf@circ@res@step}}
    \pgfpathlineto{\pgfpoint{.4\pgf@circ@res@step}{-1.2\pgf@circ@res@step}}
    \pgfpathmoveto{\pgfpoint{-.25\pgf@circ@res@step}{-1.4\pgf@circ@res@step}}
    \pgfpathlineto{\pgfpoint{.25\pgf@circ@res@step}{-1.4\pgf@circ@res@step}}
    \pgfusepath{draw}
}

% chassis ground
\pgf@circ@declareground{cground}{1}{2}{
    \pgfsetlinewidth{\ctikzvalof{monopoles/ground/connectionthickness}\pgfstartlinewidth}
    \pgfpathmoveto{\pgfpointorigin}
    \pgfpathlineto{\pgfpoint{0pt}{-1.5\pgf@circ@res@step}}
    \pgfusepath{draw}
    \pgf@circ@setlinewidth{monopoles/ground}{\pgfstartlinewidth}
    \pgfpathmoveto{\pgfpoint{-1.00\pgf@circ@res@step}{-2.10\pgf@circ@res@step}}
    \pgfpathlineto{\pgfpoint{-0.75\pgf@circ@res@step}{-1.50\pgf@circ@res@step}}
    \pgfpathlineto{\pgfpoint{ 0.75\pgf@circ@res@step}{-1.50\pgf@circ@res@step}}
    \pgfpathlineto{\pgfpoint{ 0.50\pgf@circ@res@step}{-2.10\pgf@circ@res@step}}
    \pgfpathmoveto{\pgfpoint{ 0.00\pgf@circ@res@step}{-1.50\pgf@circ@res@step}}
    \pgfpathlineto{\pgfpoint{-0.25\pgf@circ@res@step}{-2.10\pgf@circ@res@step}}
    \pgfusepath{draw}
}

% Contributed by @fotesan https://github.com/fotesan
% european ground
\pgf@circ@declareground{eground}{1.1}{1.7}{
    \pgfpathmoveto{\pgfpointorigin}
    \pgfpathlineto{\pgfpoint{0pt}{-\pgf@circ@res@step}}
    \pgfusepath{draw}

    \pgfstartlinewidth=\pgflinewidth
    \pgf@circ@setlinewidth{monopoles/tground}{\pgfstartlinewidth}

    \pgfpathmoveto{\pgfpoint{-1\pgf@circ@res@step}{-\pgf@circ@res@step}}
    \pgfpathlineto{\pgfpoint{1\pgf@circ@res@step}{-\pgf@circ@res@step}}
    \pgfusepath{draw}

    \pgfsetlinewidth{\pgfstartlinewidth}
    \pgfpathmoveto{\pgfpoint{-1.1\pgf@circ@res@step}{-1.7\pgf@circ@res@step}}
    \pgfpathlineto{\pgfpoint{-.6\pgf@circ@res@step}{-1\pgf@circ@res@step}}
    \pgfpathmoveto{\pgfpoint{-.6\pgf@circ@res@step}{-1.7\pgf@circ@res@step}}
    \pgfpathlineto{\pgfpoint{-.1\pgf@circ@res@step}{-1\pgf@circ@res@step}}
    \pgfpathmoveto{\pgfpoint{-.1\pgf@circ@res@step}{-1.7\pgf@circ@res@step}}
    \pgfpathlineto{\pgfpoint{.4\pgf@circ@res@step}{-1\pgf@circ@res@step}}
    \pgfpathmoveto{\pgfpoint{.4\pgf@circ@res@step}{-1.7\pgf@circ@res@step}}
    \pgfpathlineto{\pgfpoint{.9\pgf@circ@res@step}{-1\pgf@circ@res@step}}
    \pgfusepath{draw}
}

\pgf@circ@declareground{eground2}{1.1}{1.7}{
    \pgfpathmoveto{\pgfpointorigin}
    \pgfpathlineto{\pgfpoint{0pt}{-\pgf@circ@res@step}}
    \pgfusepath{draw}

    \pgfstartlinewidth=\pgflinewidth
    \pgf@circ@setlinewidth{monopoles/tground}{\pgfstartlinewidth}

    \pgfpathmoveto{\pgfpoint{-1\pgf@circ@res@step}{-\pgf@circ@res@step}}
    \pgfpathlineto{\pgfpoint{1\pgf@circ@res@step}{-\pgf@circ@res@step}}
    \pgfusepath{draw}

    \pgfsetlinewidth{\pgfstartlinewidth}
    \pgfpathmoveto{\pgfpoint{-1.1\pgf@circ@res@step}{-1.7\pgf@circ@res@step}}
    \pgfpathlineto{\pgfpoint{-.4\pgf@circ@res@step}{-1\pgf@circ@res@step}}
    \pgfpathmoveto{\pgfpoint{-.45\pgf@circ@res@step}{-1.7\pgf@circ@res@step}}
    \pgfpathlineto{\pgfpoint{.25\pgf@circ@res@step}{-1\pgf@circ@res@step}}
    \pgfpathmoveto{\pgfpoint{.2\pgf@circ@res@step}{-1.7\pgf@circ@res@step}}
    \pgfpathlineto{\pgfpoint{.9\pgf@circ@res@step}{-1\pgf@circ@res@step}}
    \pgfusepath{draw}
}

%%%%%%%%%%%%%%%%%%
%% Power supplies
%%%%%%%%%%%%%%%%%%

% Vcc
\pgfdeclareshape{vcc}{
    \savedmacro{\ctikzclass}{\edef\ctikzclass{power supplies}}  % class of these components
    \saveddimen{\scaledRlen}{\pgfmathsetlength{\pgf@x}{\ctikzvalof{\ctikzclass/scale}\pgf@circ@Rlen}}
    \saveddimen{\scaledwidth}{% thanks to @Schrödinger's cat on https://tex.stackexchange.com/a/506249/38080
        \pgfgettransformentries{\tmpa}{\tmpb}{\tmpc}{\tmpd}{\tmp}{\tmp}%
        \pgfmathsetmacro{\gscale}{sqrt(abs(\tmpa*\tmpd-\tmpb*\tmpc))}% abs should not be needed
        \pgfmathsetlength{\pgf@x}{(\ctikzvalof{\ctikzclass/scale}*\gscale*\ctikzvalof{monopoles/vcc/width})*\pgf@circ@Rlen}%
    }
    \savedanchor{\northeast}{%
        \pgfmathsetlength{\pgf@circ@scaled@Rlen}{\ctikzvalof{\ctikzclass/scale}\pgf@circ@Rlen}
        \pgf@circ@res@step=\ctikzvalof{monopoles/vcc/width}\pgf@circ@scaled@Rlen
        \pgf@x=.5\pgf@circ@res@step
        \pgf@y=3\pgf@x%
    }
    \anchor{north}{\northeast\pgf@x=0cm\relax}
    \anchor{east}{\northeast\pgf@y=0.5\pgf@y\relax}
    \anchor{south}{\pgfpointorigin}
    \anchor{west}{\northeast\pgf@y=0.5\pgf@y\pgf@x=-\pgf@x}
    \anchor{north east}{\northeast}
    \anchor{north west}{\northeast\pgf@x=-\pgf@x}
    \anchor{south east}{\northeast\pgf@y=0pt\relax}
    \anchor{south west}{\northeast\pgf@y=0pt\pgf@x=-\pgf@x}
    \anchor{center}{\pgfpointorigin}
    \anchor{left}{\pgfpointorigin}
    \anchor{right}{\pgfpointorigin}
    \anchor{text}{
        \pgfmathsetlength{\pgf@circ@scaled@Rlen}{\ctikzvalof{\ctikzclass/scale}\pgf@circ@Rlen}
        \pgf@circ@res@step=\ctikzvalof{monopoles/vcc/width}\pgf@circ@scaled@Rlen
        \pgfpathmoveto{\pgfpoint{-.5\wd\pgfnodeparttextbox}{2\pgf@circ@res@step+2\ht\pgfnodeparttextbox}}
        \pgfpathmoveto{\pgfpoint{.5\wd\pgfnodeparttextbox}{2\pgf@circ@res@step+2\ht\pgfnodeparttextbox}}
        \pgf@x=0pt
        \pgf@y=2\pgf@circ@res@step
        \advance \pgf@y by 0.5\ht\pgfnodeparttextbox
        \pgf@x=-.5\wd\pgfnodeparttextbox
    }
    \behindforegroundpath{
        \pgf@circ@scaled@Rlen=\scaledRlen
        \pgf@circ@res@step=\ctikzvalof{monopoles/vcc/width}\pgf@circ@scaled@Rlen
        \pgfscope
            \edef\pgf@circ@temp{\ctikzvalof{monopoles/vcc/arrow}}\edef\pgf@temp{legacy}
            \ifx\pgf@temp\pgf@circ@temp
                \pgfstartlinewidth=\pgflinewidth
                \pgf@circ@setlinewidth{bipoles}{\pgfstartlinewidth}

                \pgfpathmoveto{\pgfpoint{-0.5\pgf@circ@res@step}{.8\pgf@circ@res@step}}
                \pgfpathlineto{\pgfpoint{0}{1.5\pgf@circ@res@step}}
                \pgfpathlineto{\pgfpoint{0.5\pgf@circ@res@step}{.8\pgf@circ@res@step}}
                \pgfusepath{draw}

                \pgfsetlinewidth{\pgfstartlinewidth}
            \else
            \pgfsetarrowsend{\pgf@circ@temp}
        \fi
        \pgfpathmoveto{\pgfpointorigin}
        \pgfpathlineto{\pgfpoint{0pt}{1.5\pgf@circ@res@step}}
        \pgfusepath{draw}
    \endpgfscope
    }
}

% Vee
\pgfdeclareshape{vee}{
    \savedmacro{\ctikzclass}{\edef\ctikzclass{power supplies}}  % class of these components
    \saveddimen{\scaledRlen}{\pgfmathsetlength{\pgf@x}{\ctikzvalof{\ctikzclass/scale}\pgf@circ@Rlen}}
    \saveddimen{\scaledwidth}{% thanks to @Schrödinger's cat on https://tex.stackexchange.com/a/506249/38080
        \pgfgettransformentries{\tmpa}{\tmpb}{\tmpc}{\tmpd}{\tmp}{\tmp}%
        \pgfmathsetmacro{\gscale}{sqrt(abs(\tmpa*\tmpd-\tmpb*\tmpc))}% abs should not be needed
        \pgfmathsetlength{\pgf@x}{(\ctikzvalof{\ctikzclass/scale}*\gscale*\ctikzvalof{monopoles/vcc/width})*\pgf@circ@Rlen}%
    }
    \savedanchor{\northeast}{%
        \pgfmathsetlength{\pgf@circ@scaled@Rlen}{\ctikzvalof{\ctikzclass/scale}\pgf@circ@Rlen}
        \pgf@circ@res@step=\ctikzvalof{monopoles/vcc/width}\pgf@circ@scaled@Rlen
        \pgf@x=.5\pgf@circ@res@step
        \pgf@y=-3\pgf@x%
    }
    \anchor{south}{\northeast\pgf@x=0cm\relax}
    \anchor{east}{\northeast\pgf@y=0.5\pgf@y\relax}
    \anchor{north}{\pgfpointorigin}
    \anchor{west}{\northeast\pgf@y=0.5\pgf@y\pgf@x=-\pgf@x}
    \anchor{south east}{\northeast}
    \anchor{south west}{\northeast\pgf@x=-\pgf@x}
    \anchor{north east}{\northeast\pgf@y=0pt\relax}
    \anchor{north west}{\northeast\pgf@y=0pt\pgf@x=-\pgf@x}
    \anchor{center}{\pgfpointorigin}
    \anchor{left}{\pgfpointorigin}
    \anchor{right}{\pgfpointorigin}
    \anchor{text}{
        \pgfmathsetlength{\pgf@circ@scaled@Rlen}{\ctikzvalof{\ctikzclass/scale}\pgf@circ@Rlen}
        \pgf@circ@res@step=\ctikzvalof{monopoles/vcc/width}\pgf@circ@scaled@Rlen
        \pgfpathmoveto{\pgfpoint{-.5\wd\pgfnodeparttextbox}{-2\pgf@circ@res@step-2\ht\pgfnodeparttextbox}}
        \pgfpathmoveto{\pgfpoint{.5\wd\pgfnodeparttextbox}{-2\pgf@circ@res@step-2\ht\pgfnodeparttextbox}}
        \pgf@x=0pt
        \pgf@y=-2\pgf@circ@res@step
        \advance \pgf@y by -1.5\ht\pgfnodeparttextbox
        \pgf@x=-.5\wd\pgfnodeparttextbox
    }

    \behindforegroundpath{
        \pgf@circ@scaled@Rlen=\scaledRlen
        \pgf@circ@res@step=\ctikzvalof{monopoles/vcc/width}\pgf@circ@scaled@Rlen
        \pgfscope
            \edef\pgf@circ@temp{\ctikzvalof{monopoles/vee/arrow}}\edef\pgf@temp{legacy}
            \ifx\pgf@temp\pgf@circ@temp

                \pgfstartlinewidth=\pgflinewidth
                \pgf@circ@setlinewidth{bipoles}{\pgfstartlinewidth}

                \pgfpathmoveto{\pgfpoint{-0.5\pgf@circ@res@step}{-.8\pgf@circ@res@step}}
                \pgfpathlineto{\pgfpoint{0}{-1.5\pgf@circ@res@step}}
                \pgfpathlineto{\pgfpoint{0.5\pgf@circ@res@step}{-.8\pgf@circ@res@step}}
                \pgfusepath{draw}
                \pgfsetlinewidth{\pgfstartlinewidth}
            \else
                \pgfsetarrowsend{\pgf@circ@temp}
            \fi
            \pgfpathmoveto{\pgfpointorigin}
            \pgfpathlineto{\pgfpoint{0pt}{-1.5\pgf@circ@res@step}}
            \pgfusepath{draw}
        \endpgfscope
    }
}

%%%%%%%%%%%%%%%%
%% RF elements
%%%%%%%%%%%%%%%%

% Legacy tlinestub
% Contributed by Leonardo Azzinnari
\pgfdeclareshape{tlinestub}{
    \savedmacro{\ctikzclass}{\edef\ctikzclass{RF}}
    \saveddimen{\scaledRlen}{\pgfmathsetlength{\pgf@x}{\ctikzvalof{\ctikzclass/scale}\pgf@circ@Rlen}}
    \savedanchor{\northeast}{%
        \pgfmathsetlength{\pgf@circ@scaled@Rlen}{\ctikzvalof{\ctikzclass/scale}\pgf@circ@Rlen}
        \pgf@circ@res@step = \ctikzvalof{bipoles/tline/width} \pgf@circ@scaled@Rlen
        \pgf@x=1.2\pgf@circ@res@step
        \pgf@circ@res@step = \ctikzvalof{bipoles/tline/width} \pgf@circ@scaled@Rlen
        \pgf@y=.2\pgf@circ@res@step%
    }
    % the center is on the left side of the shape for facility of usage
    \anchor{north}{\northeast\pgf@x=0.5\pgf@x\relax}
    \anchor{east}{\northeast\pgf@y=0cm\relax}
    \anchor{south}{\northeast\pgf@y=-\pgf@y \pgf@x=0.5\pgf@x\relax}
    \anchor{west}{\pgfpointorigin}
    \anchor{north east}{\northeast}
    \anchor{north west}{\northeast\pgf@x=0cm\relax}
    \anchor{south east}{\northeast\pgf@y=-\pgf@y}
    \anchor{south west}{\northeast\pgf@x=0cm\pgf@y=-\pgf@y}
    \anchor{center}{\pgfpointorigin}
    \behindforegroundpath{
        \pgfstartlinewidth=\pgflinewidth

        \pgf@circ@scaled@Rlen=\scaledRlen
        \pgf@circ@res@step=\ctikzvalof{bipoles/tline/width} \pgf@circ@scaled@Rlen
        \pgf@circ@res@step=0.6\pgf@circ@scaled@Rlen

        \pgfscope\begin{pgftransparencygroup}
            \pgfpathellipse{\pgfpoint{0.5\pgf@circ@res@step}{0\pgf@circ@res@step}}{\pgfpoint{0.125\pgf@circ@res@step}{0\pgf@circ@res@step}}{\pgfpoint{0\pgf@circ@res@step}{0.25\pgf@circ@res@step}}
            \pgf@circ@maybefill
            \pgfpathmoveto{\pgfpoint{0.5\pgf@circ@res@step}{0.25\pgf@circ@res@step}}
            \pgfpathlineto{\pgfpoint{1.5\pgf@circ@res@step}{0.25\pgf@circ@res@step}}
            \pgfpatharc{90}{-90}{0.125\pgf@circ@res@step and 0.25\pgf@circ@res@step}
            \pgfpathlineto{\pgfpoint{0.5\pgf@circ@res@step}{-0.25\pgf@circ@res@step}}
            \pgf@circ@setlinewidth{bipoles}{\pgfstartlinewidth}
            \pgf@circ@draworfill
            \pgfpathellipse{\pgfpoint{0.5\pgf@circ@res@step}{0\pgf@circ@res@step}}{\pgfpoint{0.125\pgf@circ@res@step}{0\pgf@circ@res@step}}{\pgfpoint{0\pgf@circ@res@step}{0.25\pgf@circ@res@step}}
            \pgfusepath{draw}
        \end{pgftransparencygroup} \endpgfscope
        \pgfpathmoveto{\pgfpointorigin}
        \pgfpathlineto{\pgfpoint{0.5\pgf@circ@res@step}{0pt}}
        \pgfusepath{draw}
    }
}

%% New antennas without tails

% main body of antennas
\def\pgf@circ@antennabody{%
    \pgfstartlinewidth=\pgflinewidth
    \pgf@circ@scaled@Rlen=\scaledRlen
    \pgfsetcolor{\ctikzvalof{color}}
    \pgf@circ@res@step=\ctikzvalof{monopoles/bareantenna/width}\pgf@circ@scaled@Rlen
    \pgfscope
        \pgf@circ@setlinewidth{bipoles}{\pgfstartlinewidth}
        \pgfpathmoveto{\pgfpointorigin}
        \pgfpathlineto{\pgfpoint{-\pgf@circ@res@step}{2\pgf@circ@res@step}}
        \pgfpathlineto{\pgfpoint{\pgf@circ@res@step}{2\pgf@circ@res@step}}
        \pgfsetbeveljoin
        \pgfpathclose
        \pgf@circ@draworfill
    \endpgfscope
    \pgfsetlinewidth{\pgfstartlinewidth}
    \pgfpathmoveto{\pgfpointorigin}
    \pgfpathlineto{\pgfpoint{0pt}{2\pgf@circ@res@step}}
    \pgfusepath{draw}
}

% Waves for the antennas.
\def\pgf@circ@antennawaves{%
    \pgfscope
    % define a triangle for clipping the waves
    \pgfpathmoveto{\pgfpoint{0pt}{\pgf@circ@res@step}}
    \pgfpathlineto{\pgfpoint{4.2\pgf@circ@res@step}{3\pgf@circ@res@step}}
    \pgfpathlineto{\pgfpoint{4.2\pgf@circ@res@step}{-1\pgf@circ@res@step}}
    \pgfpathclose
    \pgfusepath{clip}
    % ...and build the waves as clipped circles
    \pgf@circ@count@a=8\pgf@circ@res@other=0.5\pgf@circ@res@step
    \pgfmathloop%
    \ifnum\pgf@circ@count@a>2
        \pgfpathcircle{\pgfpoint{0pt}{\pgf@circ@res@step}}{\the\pgf@circ@count@a*\pgf@circ@res@other}
        \advance\pgf@circ@count@a-1\relax%
        \repeatpgfmathloop
        \pgfusepath{draw}
    \endpgfscope
}

% additional shape with the waves
\pgfdeclareshape{waves}{
    \savedmacro{\ctikzclass}{\edef\ctikzclass{RF}}
    \saveddimen{\scaledRlen}{\pgfmathsetlength{\pgf@x}{\ctikzvalof{\ctikzclass/scale}\pgf@circ@Rlen}}
    \anchor{center}{\pgfpointorigin}
    \savedanchor{\northeast}{
        \pgfmathsetlength{\pgf@circ@scaled@Rlen}{\ctikzvalof{\ctikzclass/scale}\pgf@circ@Rlen}
        \pgf@circ@res@step=\ctikzvalof{monopoles/waves/width}\pgf@circ@scaled@Rlen
        \pgf@x=\pgf@circ@res@step
        \pgf@y=\pgf@circ@res@step
    }
    \anchor{text}{
        \northeast
        \pgf@y=\dimexpr\pgf@y+.5\dp\pgfnodeparttextbox-.5\ht\pgfnodeparttextbox\relax
    }
    \anchor{top}{\northeast\pgf@x=0pt}
    \anchor{right}{\northeast\pgf@y=0pt}
    \anchor{left}{\northeast\pgf@x=-\pgf@x\pgf@y=0pt}
    \anchor{bottom}{\northeast\pgf@y=-\pgf@y\pgf@x=0pt}
    \anchor{north}{\northeast\pgf@x=0pt}
    \anchor{north east}{\northeast}
    \anchor{east}{\northeast\pgf@y=0pt}
    \anchor{south east}{\northeast\pgf@y=-\pgf@y}
    \anchor{south}{\northeast\pgf@y=-\pgf@y\pgf@x=0pt}
    \anchor{south west}{\northeast\pgf@y=-\pgf@y\pgf@x=-\pgf@x}
    \anchor{west}{\northeast\pgf@x=-\pgf@x\pgf@y=0pt}
    \anchor{north west}{\northeast\pgf@x=-\pgf@x}
    \behindforegroundpath{
        \pgf@circ@scaled@Rlen=\scaledRlen
        \pgf@circ@res@step=\ctikzvalof{monopoles/waves/width}\pgf@circ@scaled@Rlen
        \pgf@circ@res@step=0.5\pgf@circ@res@step
        \pgfsetcolor{\ctikzvalof{color}}
        \pgfscope
        % define a triangle for clipping the waves
        \pgfpathmoveto{\pgfpoint{-2\pgf@circ@res@step}{0pt}}
        \pgfpathlineto{\pgfpoint{2.1\pgf@circ@res@step}{2\pgf@circ@res@step}}
        \pgfpathlineto{\pgfpoint{2.1\pgf@circ@res@step}{-2\pgf@circ@res@step}}
        \pgfpathclose
        \pgfusepath{clip}
        % ...and build the waves as clipped circles
        \c@pgf@counta=8\pgf@circ@res@other=0.5\pgf@circ@res@step
        \pgfmathloop%
        \ifnum\c@pgf@counta>1
            \pgfpathcircle{\pgfpoint{-2\pgf@circ@res@step}{0pt}}{\the\c@pgf@counta*\pgf@circ@res@other}
            \advance\c@pgf@counta-1\relax%
            \repeatpgfmathloop
            \pgfusepath{draw}
        \endpgfscope
    }
}

% the three types of antennas: simple, TX, RX. Notice that you can flip them...

\pgfdeclareshape{bareantenna}{
    \savedmacro{\ctikzclass}{\edef\ctikzclass{RF}}
    \saveddimen{\scaledRlen}{\pgfmathsetlength{\pgf@x}{\ctikzvalof{\ctikzclass/scale}\pgf@circ@Rlen}}
    \anchor{center}{\pgfpointorigin}
    \savedanchor{\northeast}{
        \pgfmathsetlength{\pgf@circ@scaled@Rlen}{\ctikzvalof{\ctikzclass/scale}\pgf@circ@Rlen}
        \pgf@circ@res@step=\ctikzvalof{monopoles/bareantenna/width}\pgf@circ@scaled@Rlen
        \pgf@x=\pgf@circ@res@step
        \pgf@y=2\pgf@circ@res@step
    }
    \anchor{text}{
        \northeast
        \pgf@x=\ctikzvalof{monopoles/bareantenna/label/xanchor}\pgf@x
        \pgf@y=\ctikzvalof{monopoles/bareantenna/label/yanchor}\pgf@y
        \pgf@y=\dimexpr\pgf@y+.5\dp\pgfnodeparttextbox-.5\ht\pgfnodeparttextbox\relax
    }
    \anchor{top}{\northeast\pgf@x=0pt}
    \anchor{right}{\northeast\pgf@y=0.5\pgf@y}
    \anchor{left}{\northeast\pgf@x=-\pgf@x\pgf@y=0.5\pgf@y}
    \anchor{bottom}{\pgfpointorigin}
    \anchor{north}{\northeast\pgf@x=0pt}
    \anchor{east}{\northeast\pgf@y=0.5\pgf@y}
    \anchor{west}{\northeast\pgf@x=-\pgf@x\pgf@y=0.5\pgf@y}
    \anchor{south}{\pgfpointorigin}
    \anchor{north east}{\northeast}
    \anchor{north west}{\northeast\pgf@x=-\pgf@x}
    \anchor{south east}{\northeast\pgf@y=0pt\relax}
    \anchor{south west}{\northeast\pgf@y=0pt\pgf@x=-\pgf@x}
    \behindforegroundpath{
        \pgf@circ@antennabody
    }
}

\pgfdeclareshape{bareTXantenna}{
    \savedmacro{\ctikzclass}{\edef\ctikzclass{RF}}
    \saveddimen{\scaledRlen}{\pgfmathsetlength{\pgf@x}{\ctikzvalof{\ctikzclass/scale}\pgf@circ@Rlen}}
    \anchor{center}{\pgfpointorigin}
    \savedanchor{\northeast}{
        \pgfmathsetlength{\pgf@circ@scaled@Rlen}{\ctikzvalof{\ctikzclass/scale}\pgf@circ@Rlen}
        \pgf@circ@res@step=\ctikzvalof{monopoles/bareantenna/width}\pgf@circ@scaled@Rlen
        \pgf@x=\pgf@circ@res@step
        \pgf@y=2\pgf@circ@res@step
    }
    \savedanchor{\savedwaves}{
        \pgfmathsetlength{\pgf@circ@scaled@Rlen}{\ctikzvalof{\ctikzclass/scale}\pgf@circ@Rlen}
        \pgf@circ@res@step=\ctikzvalof{monopoles/bareantenna/width}\pgf@circ@scaled@Rlen
        \pgf@x=4.2\pgf@circ@res@step
        \pgf@y=\pgf@circ@res@step
    }
    \anchor{text}{
        \northeast
        \pgf@x=\ctikzvalof{monopoles/bareantenna/label/xanchor}\pgf@x
        \pgf@y=\ctikzvalof{monopoles/bareantenna/label/yanchor}\pgf@y
        \pgf@x=\dimexpr-\pgf@x-\wd\pgfnodeparttextbox\relax
        \pgf@y=\dimexpr\pgf@y+.5\dp\pgfnodeparttextbox-.5\ht\pgfnodeparttextbox\relax
    }
    \anchor{waves}{\savedwaves}
    \anchor{top}{\northeast\pgf@x=0pt}
    \anchor{right}{\northeast\pgf@y=0.5\pgf@y}
    \anchor{left}{\northeast\pgf@x=-\pgf@x\pgf@y=0.5\pgf@y}
    \anchor{bottom}{\pgfpointorigin}
    \anchor{north}{\northeast\pgf@x=0pt}
    \anchor{east}{\northeast\pgf@y=0.5\pgf@y}
    \anchor{west}{\northeast\pgf@x=-\pgf@x\pgf@y=0.5\pgf@y}
    \anchor{south}{\pgfpointorigin}
    \anchor{north east}{\northeast}
    \anchor{north west}{\northeast\pgf@x=-\pgf@x}
    \anchor{south east}{\northeast\pgf@y=0pt\relax}
    \anchor{south west}{\northeast\pgf@y=0pt\pgf@x=-\pgf@x}
    \behindforegroundpath{
        \pgf@circ@antennabody
        \pgf@circ@antennawaves
    }
}

\pgfdeclareshape{bareRXantenna}{
    \savedmacro{\ctikzclass}{\edef\ctikzclass{RF}}
    \saveddimen{\scaledRlen}{\pgfmathsetlength{\pgf@x}{\ctikzvalof{\ctikzclass/scale}\pgf@circ@Rlen}}
    \anchor{center}{\pgfpointorigin}
    \savedanchor{\northeast}{
        \pgfmathsetlength{\pgf@circ@scaled@Rlen}{\ctikzvalof{\ctikzclass/scale}\pgf@circ@Rlen}
        \pgf@circ@res@step=\ctikzvalof{monopoles/bareantenna/width}\pgf@circ@scaled@Rlen
        \pgf@x=\pgf@circ@res@step
        \pgf@y=2\pgf@circ@res@step
    }
    \savedanchor{\savedwaves}{
        \pgfmathsetlength{\pgf@circ@scaled@Rlen}{\ctikzvalof{\ctikzclass/scale}\pgf@circ@Rlen}
        \pgf@circ@res@step=\ctikzvalof{monopoles/bareantenna/width}\pgf@circ@scaled@Rlen
        \pgf@x=-4.2\pgf@circ@res@step
        \pgf@y=\pgf@circ@res@step
    }
    \anchor{text}{
        \northeast
        \pgf@x=\ctikzvalof{monopoles/bareantenna/label/xanchor}\pgf@x
        \pgf@y=\ctikzvalof{monopoles/bareantenna/label/yanchor}\pgf@y
        \pgf@y=\dimexpr\pgf@y+.5\dp\pgfnodeparttextbox-.5\ht\pgfnodeparttextbox\relax
    }
    \anchor{waves}{\savedwaves}
    \anchor{top}{\northeast\pgf@x=0pt}
    \anchor{right}{\northeast\pgf@y=0.5\pgf@y}
    \anchor{left}{\northeast\pgf@x=-\pgf@x\pgf@y=0.5\pgf@y}
    \anchor{bottom}{\pgfpointorigin}
    \anchor{north}{\northeast\pgf@x=0pt}
    \anchor{east}{\northeast\pgf@y=0.5\pgf@y}
    \anchor{west}{\northeast\pgf@x=-\pgf@x\pgf@y=0.5\pgf@y}
    \anchor{south}{\pgfpointorigin}
    \anchor{north east}{\northeast}
    \anchor{north west}{\northeast\pgf@x=-\pgf@x}
    \anchor{south east}{\northeast\pgf@y=0pt\relax}
    \anchor{south west}{\northeast\pgf@y=0pt\pgf@x=-\pgf@x}
    \behindforegroundpath{
        \pgf@circ@antennabody
        \pgftransformxshift{-5.2\pgf@circ@res@step}
        \pgf@circ@antennawaves
    }
}

% Microstrip monopoles

\pgfdeclareshape{mslstub}{
    \savedmacro{\ctikzclass}{\edef\ctikzclass{RF}}
    \saveddimen{\scaledRlen}{\pgfmathsetlength{\pgf@x}{\ctikzvalof{\ctikzclass/scale}\pgf@circ@Rlen}}
    \savedanchor{\southeast}{
        \pgfmathsetlength{\pgf@circ@scaled@Rlen}{\ctikzvalof{\ctikzclass/scale}\pgf@circ@Rlen}
        \pgf@x=\ctikzvalof{bipoles/mstline/width}\pgf@circ@scaled@Rlen
        \pgf@x=.5\pgf@x
        \pgf@y=\ctikzvalof{bipoles/mstline/height}\pgf@circ@scaled@Rlen
        \pgf@y=-.5\pgf@y
    }
    \savedanchor{\northwest}{
        \pgfmathsetlength{\pgf@circ@scaled@Rlen}{\ctikzvalof{\ctikzclass/scale}\pgf@circ@Rlen}
        \pgf@x=\ctikzvalof{bipoles/mstline/width}\pgf@circ@scaled@Rlen
        \pgf@x=-.5\pgf@x
        \pgf@y=\ctikzvalof{bipoles/mstline/height}\pgf@circ@scaled@Rlen
        \pgf@y=.5\pgf@y
    }
    \anchor{north}{\northwest\pgf@x=0pt\relax}
    \anchor{north east}{\northwest\pgf@x=-\pgf@x}
    \anchor{east}{\southeast\pgf@y=0pt\relax}
    \anchor{south east}{\southeast}
    \anchor{south}{\southeast\pgf@x=0pt\relax}
    \anchor{south west}{\southeast\pgf@x=-\pgf@x}
    \anchor{west}{\northwest\pgf@y=0pt\relax}
    \anchor{north west}{\northwest}
    %
    \anchor{center}{\northwest\pgf@y=0pt\relax}
    \anchor{left}{\northwest\pgf@y=0pt\relax}
    \anchor{right}{\southeast\pgf@y=0pt\relax}
    % put the node text above and centered
    \anchor{text}{\pgfextracty{\pgf@circ@res@up}{\northwest}
        \pgfpoint{-.5\wd\pgfnodeparttextbox}{
            \dimexpr.5\dp\pgfnodeparttextbox+.5\ht\pgfnodeparttextbox+\pgf@circ@res@up\relax
        }
    }
    \behindbackgroundpath{
        \pgfextracty{\pgf@circ@res@up}{\northwest}
        \pgfextractx{\pgf@circ@res@right}{\southeast}
        \pgfextractx{\pgf@circ@res@left}{\northwest}
        \pgfextracty{\pgf@circ@res@down}{\southeast}
        \pgfmathsetlength{\pgf@circ@res@step}{0.25*\pgf@circ@res@up}
        \pgfstartlinewidth=\pgflinewidth
        \pgfscope
            \pgf@circ@setlinewidth{bipoles}{\pgfstartlinewidth}
            \pgfpathrectanglecorners{\pgfpoint{\pgf@circ@res@left}{\pgf@circ@res@up}}{\pgfpoint{\pgf@circ@res@right}{\pgf@circ@res@down}}
            \pgf@circ@draworfill
        \endpgfscope
    }
}

\pgfdeclareshape{msrstub}{
    \savedmacro{\ctikzclass}{\edef\ctikzclass{RF}}
    \saveddimen{\scaledRlen}{\pgfmathsetlength{\pgf@x}{\ctikzvalof{\ctikzclass/scale}\pgf@circ@Rlen}}
    \savedanchor{\southeast}{
        \pgfmathsetlength{\pgf@circ@scaled@Rlen}{\ctikzvalof{\ctikzclass/scale}\pgf@circ@Rlen}
        \pgf@x=\ctikzvalof{monopoles/msrstub/width}\pgf@circ@scaled@Rlen
        \pgf@x=.5\pgf@x
        \pgf@y=0pt\relax
    }
    \savedanchor{\northwest}{
        \pgfmathsetlength{\pgf@circ@scaled@Rlen}{\ctikzvalof{\ctikzclass/scale}\pgf@circ@Rlen}
        \pgf@x=\ctikzvalof{monopoles/msrstub/width}\pgf@circ@scaled@Rlen
        \pgf@x=-.5\pgf@x
        \pgf@y=\ctikzvalof{monopoles/msrstub/height}\pgf@circ@scaled@Rlen
        \pgf@y=.5\pgf@y
    }
    \anchor{north}{\northwest\pgf@x=0pt\relax}
    \anchor{north east}{\northwest\pgf@x=-\pgf@x}
    \anchor{east}{\southeast\pgf@y=0pt\relax}
    \anchor{south east}{\southeast}
    \anchor{south}{\southeast\pgf@x=0pt\relax}
    \anchor{south west}{\southeast\pgf@x=-\pgf@x}
    \anchor{west}{\northwest\pgf@y=0pt\relax}
    \anchor{north west}{\northwest}
    %
    \anchor{center}{\pgfpointorigin}
    \anchor{left}{\pgfpointorigin}
    \anchor{right}{\pgfpointorigin}
    % put the node text above and centered
    \anchor{text}{\pgfextracty{\pgf@circ@res@up}{\northwest}
        \pgfpoint{-.5\wd\pgfnodeparttextbox}{
            \dimexpr.5\dp\pgfnodeparttextbox+.5\ht\pgfnodeparttextbox+\pgf@circ@res@up\relax
        }
    }
    \behindbackgroundpath{
        \pgfextracty{\pgf@circ@res@up}{\northwest}
        \pgfextractx{\pgf@circ@res@right}{\southeast}
        \pgfextractx{\pgf@circ@res@left}{\northwest}
        \pgfextracty{\pgf@circ@res@down}{\southeast}
        \pgfmathsetlength{\pgf@circ@res@step}{0.25*\pgf@circ@res@up}
        \pgfstartlinewidth=\pgflinewidth
        \pgfscope
            \pgfpathmoveto{\pgfpointorigin}
            \pgfpathlineto{\pgfpoint{0pt}{\pgf@circ@res@step}}
            \pgfusepath{draw}
            \pgf@circ@setlinewidth{bipoles}{\pgfstartlinewidth}
            \pgfpathmoveto{\pgfpointpolar{135}{\pgf@circ@res@step}}
            \pgfpatharc{135}{45}{\pgf@circ@res@step}
            \pgfpathlineto{\pgfpointpolar{45}{\pgf@circ@res@up}}
            \pgfpatharc{45}{135}{\pgf@circ@res@up}
            \pgfclosepath
            \pgf@circ@draworfill
        \endpgfscope
    }
}

\pgfdeclareshape{msport}{
    \savedmacro{\ctikzclass}{\edef\ctikzclass{RF}}
    \saveddimen{\scaledRlen}{\pgfmathsetlength{\pgf@x}{\ctikzvalof{\ctikzclass/scale}\pgf@circ@Rlen}}
    \savedanchor{\southeast}{
        \pgfmathsetlength{\pgf@circ@scaled@Rlen}{\ctikzvalof{\ctikzclass/scale}\pgf@circ@Rlen}
        \pgf@x=\ctikzvalof{monopoles/msport/width}\pgf@circ@scaled@Rlen
        \pgf@x=.5\pgf@x
        \pgf@y=\ctikzvalof{bipoles/mstline/height}\pgf@circ@scaled@Rlen
        \pgf@y=-.5\pgf@y
    }
    \savedanchor{\northwest}{
        \pgfmathsetlength{\pgf@circ@scaled@Rlen}{\ctikzvalof{\ctikzclass/scale}\pgf@circ@Rlen}
        \pgf@x=\ctikzvalof{monopoles/msport/width}\pgf@circ@scaled@Rlen
        \pgf@x=-.5\pgf@x
        \pgf@y=\ctikzvalof{bipoles/mstline/height}\pgf@circ@scaled@Rlen
        \pgf@y=.5\pgf@y
    }
    \anchor{north}{\northwest\pgf@x=0pt\relax}
    \anchor{north east}{\northwest\pgf@x=-\pgf@x}
    \anchor{east}{\southeast\pgf@y=0pt\relax}
    \anchor{south east}{\southeast}
    \anchor{south}{\southeast\pgf@x=0pt\relax}
    \anchor{south west}{\southeast\pgf@x=-\pgf@x}
    \anchor{west}{\northwest\pgf@y=0pt\relax}
    \anchor{north west}{\northwest}
    %
    \anchor{center}{\northwest\pgf@y=0pt\relax}
    \anchor{left}{\northwest\pgf@y=0pt\relax}
    \anchor{right}{\southeast\pgf@y=0pt\relax}
    % put the node text above and centered
    \anchor{text}{\pgfextracty{\pgf@circ@res@up}{\northwest}
        \pgfpoint{-.5\wd\pgfnodeparttextbox}{
            \dimexpr-.5\dp\pgfnodeparttextbox-.5\ht\pgfnodeparttextbox\relax
        }
    }
    \behindbackgroundpath{
        \pgfextracty{\pgf@circ@res@up}{\northwest}
        \pgfextractx{\pgf@circ@res@right}{\southeast}
        \pgfextractx{\pgf@circ@res@left}{\northwest}
        \pgfextracty{\pgf@circ@res@down}{\southeast}
        \pgfmathsetlength{\pgf@circ@res@step}{0.5*\pgf@circ@res@up}
        \pgfstartlinewidth=\pgflinewidth
        \pgfscope
            \pgf@circ@setlinewidth{bipoles}{\pgfstartlinewidth}
            \pgfpathmoveto{\pgfpoint{\pgf@circ@res@left}{0pt}}
            \pgfpathlineto{\pgfpoint{\pgf@circ@res@left+\pgf@circ@res@step}{\pgf@circ@res@up}}
            \pgfpathlineto{\pgfpoint{\pgf@circ@res@right}{\pgf@circ@res@up}}
            \pgfpathlineto{\pgfpoint{\pgf@circ@res@right}{\pgf@circ@res@down}}
            \pgfpathlineto{\pgfpoint{\pgf@circ@res@left+\pgf@circ@res@step}{\pgf@circ@res@down}}
            \pgfpathclose
            \pgf@circ@draworfill
        \endpgfscope
    }
}

% Legacy antennas (with tails)
\def\pgf@circ@shift@antenna@xy#1#2{%
    \pgf@y=\dimexpr\pgf@y+#2\pgf@circ@res@step
    \pgf@x=\dimexpr\pgf@x+#1\pgf@circ@res@step
\relax}

% Legacy antenna
\pgfdeclareshape{antenna}{
    \savedmacro{\ctikzclass}{\edef\ctikzclass{RF}}
    \saveddimen{\scaledRlen}{\pgfmathsetlength{\pgf@x}{\ctikzvalof{\ctikzclass/scale}\pgf@circ@Rlen}}
    \anchor{center}{\pgfpointorigin}
    \savedanchor{\northeast}{
        \pgfmathsetlength{\pgf@circ@scaled@Rlen}{\ctikzvalof{\ctikzclass/scale}\pgf@circ@Rlen}
        \pgf@circ@res@step=\ctikzvalof{monopoles/antenna/width}\pgf@circ@scaled@Rlen
        \pgf@x=\pgf@circ@res@step%-0.5\pgflinewidth
        \pgf@y=4\pgf@circ@res@step
    }
    \anchor{north}{\northeast\pgf@circ@res@step=\pgf@x\pgf@x=0cm\pgf@circ@shift@antenna@xy{0}{2}}
    \anchor{east}{\northeast\pgf@circ@res@step=\pgf@x\pgf@y=0cm\pgf@circ@shift@antenna@xy{0}{3}\relax}
    \anchor{south}{\northeast\pgf@circ@res@step=\pgf@x\pgf@y=-\pgf@y \pgf@x=0cm\pgf@circ@shift@antenna@xy{0}{4}\relax}
    \anchor{west}{\northeast\pgf@circ@res@step=\pgf@x\pgf@y=0cm\pgf@x=-\pgf@x\pgf@circ@shift@antenna@xy{0}{3}}
    \anchor{north east}{\northeast\pgf@circ@res@step=\pgf@x\pgf@circ@shift@antenna@xy{0}{2}}
    \anchor{north west}{\northeast\pgf@circ@res@step=\pgf@x\pgf@x=-\pgf@x\pgf@circ@shift@antenna@xy{0}{2}}
    \anchor{south east}{\northeast\pgf@circ@res@step=\pgf@x\pgf@y=-\pgf@y\pgf@circ@shift@antenna@xy{0}{4}}
    \anchor{south west}{\northeast\pgf@circ@res@step=\pgf@x\pgf@y=-\pgf@y\pgf@x=-\pgf@x\pgf@circ@shift@antenna@xy{0}{4}}
    \anchor{text}{
        \northeast
        \pgf@x=\ctikzvalof{monopoles/antenna/label/xanchor}\pgf@x
        \pgf@y=\ctikzvalof{monopoles/antenna/label/yanchor}\pgf@y
    }
    \behindforegroundpath{
        \pgfstartlinewidth=\pgflinewidth
        \pgf@circ@scaled@Rlen=\scaledRlen
        \pgf@circ@res@step=\ctikzvalof{monopoles/antenna/width}\pgf@circ@scaled@Rlen

        \pgftransformxshift{ -4\pgf@circ@res@step }

        \pgfpathmoveto{\pgfpoint{4\pgf@circ@res@step-0.5\pgflinewidth}{0pt}}
        \pgfpathlineto{\pgfpoint{4\pgf@circ@res@step-0.5\pgflinewidth}{6\pgf@circ@res@step}}

        \pgfusepath{draw}

        \pgfscope
            \pgfpathmoveto{\pgfpoint{4\pgf@circ@res@step-0.5\pgflinewidth}{6\pgf@circ@res@step}}
            \pgfpathlineto{\pgfpoint{5\pgf@circ@res@step-0.5\pgflinewidth}{6\pgf@circ@res@step}}
            \pgfpathlineto{\pgfpoint{4\pgf@circ@res@step-0.5\pgflinewidth}{4\pgf@circ@res@step}}
            \pgfpathlineto{\pgfpoint{3\pgf@circ@res@step-0.5\pgflinewidth}{6\pgf@circ@res@step}}
            \pgfpathlineto{\pgfpoint{4\pgf@circ@res@step-0.5\pgflinewidth}{6\pgf@circ@res@step}}
            \pgfsetcolor{\ctikzvalof{color}}
            \pgf@circ@setlinewidth{bipoles}{\pgfstartlinewidth}
            \pgfusepath{draw}
        \endpgfscope
        \pgfsetlinewidth{\pgfstartlinewidth}

    }
}

% Legacy TX antenna
\pgfdeclareshape{txantenna}{
    \savedmacro{\ctikzclass}{\edef\ctikzclass{RF}}
    \saveddimen{\scaledRlen}{\pgfmathsetlength{\pgf@x}{\ctikzvalof{\ctikzclass/scale}\pgf@circ@Rlen}}
    \anchor{center}{\pgfpointorigin}
    \savedanchor{\northeast}{
        \pgfmathsetlength{\pgf@circ@scaled@Rlen}{\ctikzvalof{\ctikzclass/scale}\pgf@circ@Rlen}
        \pgf@circ@res@step=\ctikzvalof{monopoles/antenna/width}\pgf@circ@scaled@Rlen
        \pgf@x=\pgf@circ@res@step%-0.5\pgflinewidth
        \pgf@y=4\pgf@circ@res@step
    }
    \anchor{north}{\northeast\pgf@circ@res@step=\pgf@x\pgf@x=0cm\pgf@circ@shift@antenna@xy{2}{2}}
    \anchor{east}{\northeast\pgf@circ@res@step=\pgf@x\pgf@y=0cm\pgf@circ@shift@antenna@xy{4}{3}\relax}
    \anchor{south}{\northeast\pgf@circ@res@step=\pgf@x\pgf@y=-\pgf@y \pgf@x=0cm\pgf@circ@shift@antenna@xy{2}{4}\relax}
    \anchor{west}{\northeast\pgf@circ@res@step=\pgf@x\pgf@y=0cm\pgf@x=-\pgf@x\pgf@circ@shift@antenna@xy{1}{3}}
    \anchor{north east}{\northeast\pgf@circ@res@step=\pgf@x\pgf@circ@shift@antenna@xy{4}{2}}
    \anchor{north west}{\northeast\pgf@circ@res@step=\pgf@x\pgf@x=-\pgf@x\pgf@circ@shift@antenna@xy{1}{2}}
    \anchor{south east}{\northeast\pgf@circ@res@step=\pgf@x\pgf@y=-\pgf@y\pgf@circ@shift@antenna@xy{4}{4}}
    \anchor{south west}{\northeast\pgf@circ@res@step=\pgf@x\pgf@y=-\pgf@y\pgf@x=-\pgf@x\pgf@circ@shift@antenna@xy{1}{4}}
    \anchor{text}{
        \northeast
        \pgf@x=\ctikzvalof{monopoles/txantenna/label/xanchor}\pgf@x
        \pgf@y=\ctikzvalof{monopoles/txantenna/label/yanchor}\pgf@y
    }
    \behindforegroundpath{
        \pgfstartlinewidth=\pgflinewidth
        \pgf@circ@scaled@Rlen=\scaledRlen
        \pgf@circ@res@step=\ctikzvalof{monopoles/txantenna/width}\pgf@circ@scaled@Rlen

        \pgfpathmoveto{\pgfpointorigin}
        \pgfpathlineto{\pgfpoint{4\pgf@circ@res@step}{0pt}}
        \pgfpathmoveto{\pgfpoint{4\pgf@circ@res@step-0.5\pgflinewidth}{0pt}}
        \pgfpathlineto{\pgfpoint{4\pgf@circ@res@step-0.5\pgflinewidth}{6\pgf@circ@res@step}}

        \pgfusepath{draw}

        \pgfscope
            \pgfpathmoveto{\pgfpoint{4\pgf@circ@res@step-0.5\pgflinewidth}{6\pgf@circ@res@step}}
            \pgfpathlineto{\pgfpoint{5\pgf@circ@res@step-0.5\pgflinewidth}{6\pgf@circ@res@step}}
            \pgfpathlineto{\pgfpoint{4\pgf@circ@res@step-0.5\pgflinewidth}{4\pgf@circ@res@step}}
            \pgfpathlineto{\pgfpoint{3\pgf@circ@res@step-0.5\pgflinewidth}{6\pgf@circ@res@step}}
            \pgfpathlineto{\pgfpoint{4\pgf@circ@res@step-0.5\pgflinewidth}{6\pgf@circ@res@step}}
            \pgfsetcolor{\ctikzvalof{color}}
            \pgf@circ@setlinewidth{bipoles}{\pgfstartlinewidth}
            \pgfusepath{draw}
        \endpgfscope
        \pgfpathmoveto{\pgfpoint{5.5\pgf@circ@res@step}{6\pgf@circ@res@step}}
        %        \pgfpatharc{60}{-60}{\pgf@circ@res@step and \pgf@circ@res@step}
        \pgfpatharc{30}{-30}{2\pgf@circ@res@step}         \pgfpathmoveto{\pgfpoint{6\pgf@circ@res@step}{6.25\pgf@circ@res@step}}
        \pgfpatharc{30}{-30}{2.5\pgf@circ@res@step}
        \pgfpathmoveto{\pgfpoint{6.5\pgf@circ@res@step}{6.5\pgf@circ@res@step}}
        \pgfpatharc{30}{-30}{3\pgf@circ@res@step}
        \pgfpathmoveto{\pgfpoint{7\pgf@circ@res@step}{6.75\pgf@circ@res@step}}
        \pgfpatharc{30}{-30}{3.5\pgf@circ@res@step}
        \pgfpathmoveto{\pgfpoint{7.5\pgf@circ@res@step}{7\pgf@circ@res@step}}
        \pgfpatharc{30}{-30}{4\pgf@circ@res@step}
        \pgfusepath{draw}
        \pgfsetlinewidth{\pgfstartlinewidth}

    }
}

% Legacy RX antenna
\pgfdeclareshape{rxantenna}{
    \savedmacro{\ctikzclass}{\edef\ctikzclass{RF}}
    \saveddimen{\scaledRlen}{\pgfmathsetlength{\pgf@x}{\ctikzvalof{\ctikzclass/scale}\pgf@circ@Rlen}}
    \anchor{center}{\pgfpointorigin}
    \savedanchor{\northeast}{
        \pgfmathsetlength{\pgf@circ@scaled@Rlen}{\ctikzvalof{\ctikzclass/scale}\pgf@circ@Rlen}
        \pgf@circ@res@step=\ctikzvalof{monopoles/antenna/width}\pgf@circ@scaled@Rlen
        \pgf@x=\pgf@circ@res@step%-0.5\pgflinewidth
        \pgf@y=4\pgf@circ@res@step
    }
    \anchor{north}{\northeast\pgf@circ@res@step=\pgf@x\pgf@x=0cm\pgf@circ@shift@antenna@xy{2}{2}}
    \anchor{east}{\northeast\pgf@circ@res@step=\pgf@x\pgf@y=0cm\pgf@circ@shift@antenna@xy{4}{3}\relax}
    \anchor{south}{\northeast\pgf@circ@res@step=\pgf@x\pgf@y=-\pgf@y \pgf@x=0cm\pgf@circ@shift@antenna@xy{2}{4}\relax}
    \anchor{west}{\northeast\pgf@circ@res@step=\pgf@x\pgf@y=0cm\pgf@x=-\pgf@x\pgf@circ@shift@antenna@xy{1}{3}}
    \anchor{north east}{\northeast\pgf@circ@res@step=\pgf@x\pgf@circ@shift@antenna@xy{4}{2}}
    \anchor{north west}{\northeast\pgf@circ@res@step=\pgf@x\pgf@x=-\pgf@x\pgf@circ@shift@antenna@xy{1}{2}}
    \anchor{south east}{\northeast\pgf@circ@res@step=\pgf@x\pgf@y=-\pgf@y\pgf@circ@shift@antenna@xy{4}{4}}
    \anchor{south west}{\northeast\pgf@circ@res@step=\pgf@x\pgf@y=-\pgf@y\pgf@x=-\pgf@x\pgf@circ@shift@antenna@xy{1}{4}}
    \anchor{text}{
        \northeast
        \pgf@x=\ctikzvalof{monopoles/rxantenna/label/xanchor}\pgf@x
        \pgf@y=\ctikzvalof{monopoles/rxantenna/label/yanchor}\pgf@y
    }
    \behindforegroundpath{
        \pgfstartlinewidth=\pgflinewidth
        \pgf@circ@scaled@Rlen=\scaledRlen
        \pgf@circ@res@step=\ctikzvalof{monopoles/rxantenna/width}\pgf@circ@scaled@Rlen

        \pgfpathmoveto{\pgfpointorigin}
        \pgfpathlineto{\pgfpoint{4\pgf@circ@res@step}{0pt}}
        \pgfpathmoveto{\pgfpoint{4\pgf@circ@res@step-0.5\pgflinewidth}{0pt}}
        \pgfpathlineto{\pgfpoint{4\pgf@circ@res@step-0.5\pgflinewidth}{6\pgf@circ@res@step}}

        \pgfusepath{draw}

        \pgfscope
            \pgfpathmoveto{\pgfpoint{4\pgf@circ@res@step-0.5\pgflinewidth}{6\pgf@circ@res@step}}
            \pgfpathlineto{\pgfpoint{5\pgf@circ@res@step-0.5\pgflinewidth}{6\pgf@circ@res@step}}
            \pgfpathlineto{\pgfpoint{4\pgf@circ@res@step-0.5\pgflinewidth}{4\pgf@circ@res@step}}
            \pgfpathlineto{\pgfpoint{3\pgf@circ@res@step-0.5\pgflinewidth}{6\pgf@circ@res@step}}
            \pgfpathlineto{\pgfpoint{4\pgf@circ@res@step-0.5\pgflinewidth}{6\pgf@circ@res@step}}
            \pgfsetcolor{\ctikzvalof{color}}
            \pgf@circ@setlinewidth{bipoles}{\pgfstartlinewidth}
            \pgfusepath{draw}
        \endpgfscope

        \pgfpathmoveto{\pgfpoint{6\pgf@circ@res@step}{7\pgf@circ@res@step}}
        %             \pgfpatharc{60}{-60}{\pgf@circ@res@step and \pgf@circ@res@step}
        \pgfpatharc{150}{210}{4\pgf@circ@res@step}              \pgfpathmoveto{\pgfpoint{6.5\pgf@circ@res@step}{6.75\pgf@circ@res@step}}
        \pgfpatharc{150}{210}{3.5\pgf@circ@res@step}
        \pgfpathmoveto{\pgfpoint{7\pgf@circ@res@step}{6.5\pgf@circ@res@step}}
        \pgfpatharc{150}{210}{3\pgf@circ@res@step}
        \pgfpathmoveto{\pgfpoint{7.5\pgf@circ@res@step}{6.25\pgf@circ@res@step}}
        \pgfpatharc{150}{210}{2.5\pgf@circ@res@step}
        \pgfpathmoveto{\pgfpoint{8\pgf@circ@res@step}{6\pgf@circ@res@step}}
        \pgfpatharc{150}{210}{2\pgf@circ@res@step}
        \pgfusepath{draw}
        \pgfsetlinewidth{\pgfstartlinewidth}
    }
}

% Legacy match
\pgfdeclareshape{match}{
    \savedmacro{\ctikzclass}{\edef\ctikzclass{RF}}
    \saveddimen{\scaledRlen}{\pgfmathsetlength{\pgf@x}{\ctikzvalof{\ctikzclass/scale}\pgf@circ@Rlen}}
    \savedanchor{\northeast}{%
        \pgfmathsetlength{\pgf@circ@scaled@Rlen}{\ctikzvalof{\ctikzclass/scale}\pgf@circ@Rlen}
        \pgf@circ@res@step = \ctikzvalof{monopoles/match/width} \pgf@circ@scaled@Rlen
        \pgf@x=2\pgf@circ@res@step
        \pgf@circ@res@step = \ctikzvalof{monopoles/match/width} \pgf@circ@scaled@Rlen
        \pgf@y=.5\pgf@circ@res@step%
    }
    % the center is on the left side of the shape for facility of usage
    \anchor{north}{\northeast\pgf@x=0.5\pgf@x\relax}
    \anchor{east}{\northeast\pgf@y=0cm\relax}
    \anchor{south}{\northeast\pgf@y=0pt\pgf@x=0.5\pgf@x\relax}
    \anchor{west}{\pgfpointorigin}
    \anchor{north east}{\northeast}
    \anchor{north west}{\northeast\pgf@x=0cm\relax}
    \anchor{south east}{\northeast\pgf@y=0pt\relax}
    \anchor{south west}{\pgfpointorigin}
    \anchor{center}{
        \pgfpointorigin
    }
    \anchor{text}{
        \pgfmathsetlength{\pgf@circ@scaled@Rlen}{\ctikzvalof{\ctikzclass/scale}\pgf@circ@Rlen}
        \pgf@x=\ctikzvalof{monopoles/match/width}\pgf@circ@scaled@Rlen
        \pgf@x=1.5\pgf@x
        \advance \pgf@x by -.5\wd\pgfnodeparttextbox
        \pgf@y=-1.5\ht\pgfnodeparttextbox
    }
    \behindforegroundpath{
        \pgf@circ@scaled@Rlen=\scaledRlen
        \pgf@circ@res@step=\ctikzvalof{monopoles/match/width}\pgf@circ@scaled@Rlen

        \pgfscope
            \pgfpathmoveto{\pgfpointorigin}
            \pgfpathlineto{\pgfpoint{2\pgf@circ@res@step}{0pt}}
            \pgfusepath{draw}

            \pgfpathmoveto{\pgfpoint{\pgf@circ@res@step}{0pt}}
            \pgfpathlineto{\pgfpoint{2\pgf@circ@res@step}{0.5\pgf@circ@res@step}}
            \pgfpathlineto{\pgfpoint{2\pgf@circ@res@step}{0}}
            \pgfusepath{fill}

            \pgfsetlinewidth{\pgfstartlinewidth}
        \endpgfscope
    }
}

%%%---------- close: tex/pgfcircmonopoles
%%%%%%%%%%% Springe nach tex/pgfcircbipoles
%%%---------- open: tex/pgfcircbipoles.tex
% Copyright 2018-2020 by Romano Giannetti
% Copyright 2015-2020 by Stefan Lindner
% Copyright 2013-2020 by Stefan Erhardt
% Copyright 2007-2020 by Massimo Redaelli
%
% This file may be distributed and/or modified
%
% 1. under the LaTeX Project Public License and/or
% 2. under the GNU Public License.
%
% See the files gpl-3.0_license.txt and lppl-1-3c_license.txt for more details.

%% Generic macro for defining a bipole shape
% #1 - additional anchors
% #2 - lower y-size of the bipole (from the center).
% #3 - #shape is the name of the shape
% #4 - upper y-size of the bipole (from the center)
% #5 - width of the bipole
% #6 - macros drawing the bipole
%
\long\def\pgfcircdeclarebipole{%
    \pgfcircdeclarebipolescaled{default}}

%% Generic macro for defining a bipole shape
% #1 - scale factor
% #2 - additional anchors
% #3 - lower y-size of the bipole (from the center).
% #4 - #shape is the name of the shape
% #5 - upper y-size of the bipole (from the center)
% #6 - width of the bipole
% #7 - macros drawing the bipole
%
\long\def\pgfcircdeclarebipolescaled#1#2#3#4#5#6#7{
    \pgfdeclareshape{#4shape}{
        \savedmacro{\ctikzclass}{\edef\ctikzclass{#1}}
        \saveddimen{\scaledRlen}{\pgfmathsetlength{\pgf@x}{\ctikzvalof{\ctikzclass/scale}\pgf@circ@Rlen}}
        \savedanchor{\northeast}{
            \pgf@circ@scaled@Rlen=\ctikzvalof{\ctikzclass/scale}\pgf@circ@Rlen
            \pgf@y=#5\pgf@circ@scaled@Rlen
            \pgf@y=.5\pgf@y
            \pgf@x=#6\pgf@circ@scaled@Rlen
            \pgf@x=.5\pgf@x
        }
        \savedanchor{\northeastborder}{
            \pgf@circ@scaled@Rlen=\ctikzvalof{\ctikzclass/scale}\pgf@circ@Rlen
            \pgf@y=#5\pgf@circ@scaled@Rlen
            \pgf@y=.5\pgf@y
            \pgf@y=\ctikzvalof{bipoles/border margin}\pgf@y
            \pgf@x=#6\pgf@circ@scaled@Rlen
            \pgf@x=.5\pgf@x
            \pgf@x=\ctikzvalof{bipoles/border margin}\pgf@x
        }
        \savedanchor{\southwestborder}{
            \pgf@circ@scaled@Rlen=\ctikzvalof{\ctikzclass/scale}\pgf@circ@Rlen
            \pgf@y=-#3\pgf@circ@scaled@Rlen
            \pgf@y=.5\pgf@y
            \pgf@y=\ctikzvalof{bipoles/border margin}\pgf@y
            \pgf@x=-#6\pgf@circ@scaled@Rlen
            \pgf@x=.5\pgf@x
            \pgf@x=\ctikzvalof{bipoles/border margin}\pgf@x
        }
        \savedanchor{\southwest}{
            \pgf@circ@scaled@Rlen=\ctikzvalof{\ctikzclass/scale}\pgf@circ@Rlen
            \pgf@y=-#3\pgf@circ@scaled@Rlen
            \pgf@y=.5\pgf@y
            \pgf@x=-#6\pgf@circ@scaled@Rlen
            \pgf@x=.5\pgf@x
        }
        \savedanchor{\centerpoint}{
            \pgf@circ@scaled@Rlen=\ctikzvalof{\ctikzclass/scale}\pgf@circ@Rlen
            \pgf@circ@res@down=-#3\pgf@circ@scaled@Rlen
            \pgf@circ@res@up=#5\pgf@circ@scaled@Rlen
            \pgfpointorigin
            \pgf@y=\pgf@circ@res@up
            \advance\pgf@y by\pgf@circ@res@down
            \pgf@y=.5\pgf@y
        }
        \anchor{center}{\pgfpointorigin}
        \anchor{n}{
            \northeast
            \pgf@x=0cm
        }
        \anchor{north east}{
            \northeast
        }
        \anchor{north west}{
            \northeast
            \pgf@x=-\pgf@x
        }
        \anchor{ne}{
            \northeast
        }
        \anchor{nw}{
            \northeast
            \pgf@x=-\pgf@x
        }
        \anchor{e}{
            \northeast
            \pgf@y=0cm
        }
        \anchor{s}{
            \southwest
            \pgf@x=0cm
        }
        \anchor{south east}{
            \southwest
            \pgf@x=-\pgf@x
        }
        \anchor{south west}{
            \southwest
        }
        \anchor{se}{
            \southwest
            \pgf@x=-\pgf@x
        }
        \anchor{sw}{
            \southwest
        }
        \anchor{w}{
            \southwest
            \pgf@y=0cm
        }
        \anchor{north}{
            \northeast
            \pgf@x=0cm
        }
        \anchor{east}{
            \northeast
            \pgf@y=0cm
        }
        \anchor{south}{
            \southwest
            \pgf@x=0cm
        }
        \anchor{west}{
            \southwest
            \pgf@y=0cm
        }
        \anchor{right}{
            \northeast
            \pgf@y=0cm
        }
        \anchor{above}{
            \northeast
            \pgf@x=0cm
        }
        \anchor{left}{
            \southwest
            \pgf@y=0cm
        }
        \anchor{below}{
            \southwest
            \pgf@x=0cm
        }
        \anchor{a}{
            \northeast
            \pgf@y=0cm
        }
        \anchor{b}{
            \southwest
            \pgf@y=0cm
        }
        \savedanchor{\textanchor}{%
            \pgf@y=\ht\pgfnodeparttextbox
            \pgf@x=-.5\wd\pgfnodeparttextbox
        }
        \anchor{text}{
            \textanchor
        }
        \anchorborder{%
            \ifpgf@circuit@bipole@inverted
                \pgf@circ@res@left=-\pgf@x
                \pgf@circ@res@right=-\pgf@y
            \else
                \pgf@circ@res@left=\pgf@x
                \pgf@circ@res@right=\pgf@y
            \fi
            \ifdim\pgf@circ@res@right>0cm
                \pgfpointborderrectangle{\pgfpoint{\pgf@circ@res@left}{\pgf@circ@res@right}}{\northeastborder}
            \else
                \southwestborder
                \pgfpointborderrectangle{\pgfpoint{\pgf@circ@res@left}{\pgf@circ@res@right}}{\pgfpoint{-\pgf@x}{-\pgf@y}}
            \fi
        }

        #2

        \backgroundpath{
            \pgfsetcolor{\ctikzvalof{color}}

            \northeast
            \pgf@circ@res@up = \pgf@y
            \pgf@circ@res@zero = 0cm
            \pgf@circ@res@left = -\pgf@x
            \pgf@circ@res@right = \pgf@x
            \southwest
            \pgf@circ@res@down = \pgf@y
            \pgf@circ@scaled@Rlen=\scaledRlen
            \pgfstartlinewidth=\pgflinewidth
            \pgfsetcornersarced{\pgfpointorigin}% do not use rounded corners!
            #7

            \pgfsetlinewidth{\pgfstartlinewidth}
        }
    }
}

%%%%%%%%%%%%%%%%%%%%%%%%%%%%%%%%%%%%%%%%%%%%%%
%% Standard bipole shapes declarations
%%%%%%%%%%%%%%%%%%%%%%%%%%%%%%%%%%%%%%%%%%%%%

%%%%%%%%%%%%%%%%%%%%%%%%
% Resistive components
%%%%%%%%%%%%%%%%%%%%%%%%

%% Short circuit

%%% NOTICE that the short is really NOT drawn; we trust the fact that its
%%% natural length is zero.
\pgfcircdeclarebipole
{}
{0}
{short}
{0}
{0}
{ }

%% Open circuit
\pgfcircdeclarebipole
{}
{\ctikzvalof{bipoles/open/height}}
{open}
{\ctikzvalof{bipoles/open/height}}
{\ctikzvalof{bipoles/open/width}}
{ }

% multiwire(s)
\pgfcircdeclarebipole
{}
{\ctikzvalof{bipoles/multiwire/height}}
{multiwire}
{\ctikzvalof{bipoles/multiwire/height}}
{\ctikzvalof{bipoles/multiwire/width}}
{
    \pgfpathmoveto{\pgfpoint{\pgf@circ@res@left}{\pgf@circ@res@down}}
    \pgfpathlineto{\pgfpoint{0pt}{\pgf@circ@res@up}}
    \pgfpathmoveto{\pgfpoint{\pgf@circ@res@left}{0pt}}
    \pgfpathlineto{\pgfpoint{\pgf@circ@res@right}{0pt}}
    \pgfusepath{draw}
}

\pgfcircdeclarebipole
{}
{\ctikzvalof{bipoles/multiwire/height}}
{bmultiwire}
{\ctikzvalof{bipoles/multiwire/height}}
{\ctikzvalof{bipoles/multiwire/width}}
{
    \pgf@circ@res@other=\ctikzvalof{bipoles/multiwire/spacing}\pgf@circ@Rlen
    \pgfpathmoveto{\pgfpoint{\pgf@circ@res@left}{\pgf@circ@res@down}}
    \pgfpathlineto{\pgfpoint{0pt}{\pgf@circ@res@up}}
    \pgfpathmoveto{\pgfpoint{\pgf@circ@res@left+\pgf@circ@res@other}{\pgf@circ@res@down}}
    \pgfpathlineto{\pgfpoint{\pgf@circ@res@other}{\pgf@circ@res@up}}
    \pgfpathmoveto{\pgfpoint{\pgf@circ@res@left}{0pt}}
    \pgfpathlineto{\pgfpoint{\pgf@circ@res@right}{0pt}}
    \pgfusepath{draw}
}

\pgfcircdeclarebipole
{}
{\ctikzvalof{bipoles/multiwire/height}}
{tmultiwire}
{\ctikzvalof{bipoles/multiwire/height}}
{\ctikzvalof{bipoles/multiwire/width}}
{
    \pgf@circ@res@other=\ctikzvalof{bipoles/multiwire/spacing}\pgf@circ@Rlen
    \pgfpathmoveto{\pgfpoint{\pgf@circ@res@left}{\pgf@circ@res@down}}
    \pgfpathlineto{\pgfpoint{0pt}{\pgf@circ@res@up}}
    \pgfpathmoveto{\pgfpoint{\pgf@circ@res@left+\pgf@circ@res@other}{\pgf@circ@res@down}}
    \pgfpathlineto{\pgfpoint{\pgf@circ@res@other}{\pgf@circ@res@up}}
    \pgfpathmoveto{\pgfpoint{\pgf@circ@res@left+2\pgf@circ@res@other}{\pgf@circ@res@down}}
    \pgfpathlineto{\pgfpoint{2\pgf@circ@res@other}{\pgf@circ@res@up}}
    \pgfpathmoveto{\pgfpoint{\pgf@circ@res@left}{0pt}}
    \pgfpathlineto{\pgfpoint{\pgf@circ@res@right}{0pt}}
    \pgfusepath{draw}
}

%
%
%% Generic bipole - used as resistor by some (bleah)
\pgfcircdeclarebipolescaled{resistors}
{}
{\ctikzvalof{bipoles/generic/height}}
{generic}
{\ctikzvalof{bipoles/generic/height}}
{\ctikzvalof{bipoles/generic/width}}
{
    \pgfpathrectanglecorners{\pgfpoint{\pgf@circ@res@left}{\pgf@circ@res@up}}{\pgfpoint{\pgf@circ@res@right}{\pgf@circ@res@down}}
    \pgf@circ@setlinewidth{bipoles}{\pgfstartlinewidth}
    \pgf@circ@draworfill
}

%
% generic crossed, suggested by Radványi Patrik Tamás <patrikradvanyi@gmail.com>
% inherit "generic" properties
%
\pgfcircdeclarebipolescaled{resistors}
{}
{\ctikzvalof{bipoles/generic/height}}
{xgeneric}
{\ctikzvalof{bipoles/generic/height}}
{\ctikzvalof{bipoles/generic/width}}
{
    \pgfpathrectanglecorners{\pgfpoint{\pgf@circ@res@left}{\pgf@circ@res@up}}{\pgfpoint{\pgf@circ@res@right}{\pgf@circ@res@down}}
    \pgf@circ@setlinewidth{bipoles}{\pgfstartlinewidth}
    \pgf@circ@draworfill
    % cross it
    \pgfpathmoveto{\pgfpoint{\pgf@circ@res@left}{\pgf@circ@res@up}}
    \pgfpathlineto{\pgfpoint{\pgf@circ@res@right}{\pgf@circ@res@down}}
    \pgfpathmoveto{\pgfpoint{\pgf@circ@res@right}{\pgf@circ@res@up}}
    \pgfpathlineto{\pgfpoint{\pgf@circ@res@left}{\pgf@circ@res@down}}
    \pgfusepath{draw}
}
%% Generic empty tunable
\pgfcircdeclarebipolescaled{resistors}
{}
{\ctikzvalof{bipoles/tgeneric/height}}
{tgeneric}
{\ctikzvalof{bipoles/tgeneric/height}}
{\ctikzvalof{bipoles/tgeneric/width}}
{
    \pgfpathrectanglecorners{\pgfpoint{\pgf@circ@res@left}{.4\pgf@circ@res@up}}{\pgfpoint{\pgf@circ@res@right}{.4\pgf@circ@res@down}}
    \pgf@circ@setlinewidth{bipoles}{\pgfstartlinewidth}
    \pgf@circ@draworfill
    \pgfscope
        \pgfsetarrowsend{latexslim}
        \pgfpathmoveto{\pgfpoint{.5\pgf@circ@res@left}{\pgf@circ@res@up}}
        \pgfpathlineto{\pgfpoint{-.5\pgf@circ@res@left}{\pgf@circ@res@down}}
        \pgfusepath{draw}
    \endpgfscope
}

%% Generic asymmetric bipole
\pgfcircdeclarebipolescaled{resistors}
{}
{\ctikzvalof{bipoles/ageneric/height}}
{ageneric}
{\ctikzvalof{bipoles/ageneric/height}}
{\ctikzvalof{bipoles/ageneric/width}}
{
    \pgfscope
        \pgfpathrectanglecorners{\pgfpoint{\pgf@circ@res@left}{\pgf@circ@res@up}}{\pgfpoint{\pgf@circ@res@right}{\pgf@circ@res@down}}
        \pgf@circ@setlinewidth{bipoles}{\pgfstartlinewidth}
        \pgf@circ@draworfill
    \endpgfscope

    \pgfpathrectanglecorners{\pgfpoint{.7\pgf@circ@res@right}{\pgf@circ@res@up}}{\pgfpoint{\pgf@circ@res@right}{\pgf@circ@res@down}}
    \pgfusepath{stroke,fill}
}

%% Memristor
\pgfcircdeclarebipolescaled{resistors}
{}
{\ctikzvalof{bipoles/memristor/height}}
{memristor}
{\ctikzvalof{bipoles/memristor/height}}
{\ctikzvalof{bipoles/memristor/width}}
{
    \pgf@circ@res@step = \ctikzvalof{bipoles/resistor/width}\pgf@circ@scaled@Rlen
    \divide \pgf@circ@res@step by 14

    \pgfscope
        \pgfpathrectanglecorners{\pgfpoint{\pgf@circ@res@left}{\pgf@circ@res@up}}{\pgfpoint{\pgf@circ@res@right}{\pgf@circ@res@down}}
        \pgf@circ@setlinewidth{bipoles}{\pgfstartlinewidth}
        \pgf@circ@draworfill
    \endpgfscope
    \pgfpathmoveto{\pgfpoint{\pgf@circ@res@left}{0pt}}
    \pgfpathlineto{\pgfpoint{.72*\pgf@circ@res@left}{0pt}}
    \pgfpathlineto{\pgfpoint{.72*\pgf@circ@res@left}{\ctikzvalof{bipoles/memristor/wave height}\pgf@circ@res@up}}
    \pgfpathlineto{\pgfpoint{.35*\pgf@circ@res@left}{\ctikzvalof{bipoles/memristor/wave height}\pgf@circ@res@up}}
    \pgfpathlineto{\pgfpoint{.35*\pgf@circ@res@left}{-\ctikzvalof{bipoles/memristor/wave height}\pgf@circ@res@up}}
    \pgfpathlineto{\pgfpoint{-.05*\pgf@circ@res@left}{-\ctikzvalof{bipoles/memristor/wave height}\pgf@circ@res@up}}
    \pgfpathlineto{\pgfpoint{-.05*\pgf@circ@res@left}{\ctikzvalof{bipoles/memristor/wave height}\pgf@circ@res@up}}
    \pgfpathlineto{\pgfpoint{.42*\pgf@circ@res@right}{\ctikzvalof{bipoles/memristor/wave height}\pgf@circ@res@up}}
    \pgfpathlineto{\pgfpoint{.42*\pgf@circ@res@right}{0pt}}
    \pgfpathlineto{\pgfpoint{.8*\pgf@circ@res@right}{0pt}}

    \pgfusepath{draw}

    \pgfpathrectanglecorners{\pgfpoint{.7\pgf@circ@res@right}{\pgf@circ@res@up}}{\pgfpoint{\pgf@circ@res@right}{\pgf@circ@res@down}}
    \pgfusepath{stroke,fill}
}

%% Photoresistor
\pgfcircdeclarebipolescaled{resistors}
{}
{\ctikzvalof{bipoles/photoresistor/height 2}}
{photoresistor}
{\ctikzvalof{bipoles/photoresistor/height}}
{\ctikzvalof{bipoles/photoresistor/width}}
{
    \pgfscope
        \pgfpathrectanglecorners{\pgfpoint{\pgf@circ@res@left}{\pgf@circ@res@down}}{\pgfpoint{\pgf@circ@res@right}{-\pgf@circ@res@down}}
        \pgf@circ@setlinewidth{bipoles}{\pgfstartlinewidth}
        \pgf@circ@draworfill
    \endpgfscope

    \pgfscope
        \pgfsetarrowsend{latexslim}
        \pgfpathmoveto{\pgfpoint{.7\pgf@circ@res@right}{\pgf@circ@res@up}}
        \pgfpathlineto{\pgfpoint{.3\pgf@circ@res@right}{-1.2\pgf@circ@res@down}}
        \pgfusepath{draw}
        \pgfpathmoveto{\pgfpoint{.45\pgf@circ@res@right}{\pgf@circ@res@up}}
        \pgfpathlineto{\pgfpoint{.05\pgf@circ@res@right}{-1.2\pgf@circ@res@down}}
        \pgfusepath{draw}
    \endpgfscope
}

%% Thermistor
\pgfcircdeclarebipolescaled{resistors}
{{% anchor for labelling the type of dependency
    \anchor{label}{%
        \southwest
        \pgf@x=0.4\pgf@x
        \pgf@y=1.2\pgf@y
    }%
}}
{\ctikzvalof{bipoles/thermistor/height}}
{thermistor}
{\ctikzvalof{bipoles/thermistor/height}}
{\ctikzvalof{bipoles/thermistor/width}}
{
    \pgfscope
    \pgfpathrectanglecorners{\pgfpoint{\pgf@circ@res@left}{\ctikzvalof{bipoles/thermistor/main}\pgf@circ@res@up}}{\pgfpoint{\pgf@circ@res@right}{-\ctikzvalof{bipoles/thermistor/main}\pgf@circ@res@up}}
    \pgf@circ@setlinewidth{bipoles}{\pgfstartlinewidth}
    \pgf@circ@draworfill
    \endpgfscope

    %\pgfscope
    \pgfpathmoveto{\pgfpoint{-\pgf@circ@res@left}{-\pgf@circ@res@down}}
    \pgfpathlineto{\pgfpoint{.4\pgf@circ@res@left}{1.2\pgf@circ@res@down}}
    \pgfpathlineto{\pgfpoint{\pgf@circ@res@left}{1.2\pgf@circ@res@down}}
    \pgfusepath{draw}
    %\endpgfscope
}

%% Thermistor PTC
\pgfcircdeclarebipolescaled{resistors}
{}
{\ctikzvalof{bipoles/thermistorptc/height 2}}
{thermistorptc}
{\ctikzvalof{bipoles/thermistorptc/height}}
{\ctikzvalof{bipoles/thermistorptc/width}}
{
    \pgfscope
        \pgfpathrectanglecorners{\pgfpoint{\pgf@circ@res@left}{\ctikzvalof{bipoles/thermistorptc/main}\pgf@circ@res@up}}{\pgfpoint{\pgf@circ@res@right}{-\ctikzvalof{bipoles/thermistorptc/main}\pgf@circ@res@up}}
        \pgf@circ@setlinewidth{bipoles}{\pgfstartlinewidth}
        \pgf@circ@draworfill
    \endpgfscope
    \pgfpathmoveto{\pgfpoint{-\pgf@circ@res@left}{\pgf@circ@res@up}}
    \pgfpathlineto{\pgfpoint{.4\pgf@circ@res@left}{-\pgf@circ@res@up}}
    \pgfpathlineto{\pgfpoint{\pgf@circ@res@left}{-\pgf@circ@res@up}}
    \pgfusepath{draw}

    \pgfsetlinewidth{\pgfstartlinewidth}
    \pgftext[top,x=.85\pgf@circ@res@left,y=.75\pgf@circ@res@down]{\pgf@circ@font@tiny$\vartheta$}
    \pgfsetarrowsend{latexslim}
    \pgfpathmoveto{\pgfpoint{.62\pgf@circ@res@left}{\pgf@circ@res@down}}
    \pgfpathlineto{\pgfpoint{.62\pgf@circ@res@left}{.7\pgf@circ@res@down}}
    \pgfusepath{draw}
    \pgfpathmoveto{\pgfpoint{.45\pgf@circ@res@left}{\pgf@circ@res@down}}
    \pgfpathlineto{\pgfpoint{.45\pgf@circ@res@left}{.7\pgf@circ@res@down}}
    \pgfusepath{draw}
}

%% Thermistor NTC
\pgfcircdeclarebipolescaled{resistors}
{}
{\ctikzvalof{bipoles/thermistorntc/height 2}}
{thermistorntc}
{\ctikzvalof{bipoles/thermistorntc/height}}
{\ctikzvalof{bipoles/thermistorntc/width}}
{
    \pgfscope
        \pgfpathrectanglecorners{\pgfpoint{\pgf@circ@res@left}{\ctikzvalof{bipoles/thermistorntc/main}\pgf@circ@res@up}}{\pgfpoint{\pgf@circ@res@right}{-\ctikzvalof{bipoles/thermistorntc/main}\pgf@circ@res@up}}
        \pgf@circ@setlinewidth{bipoles}{\pgfstartlinewidth}
        \pgf@circ@draworfill
    \endpgfscope

    \pgfpathmoveto{\pgfpoint{-\pgf@circ@res@left}{\pgf@circ@res@up}}
    \pgfpathlineto{\pgfpoint{.4\pgf@circ@res@left}{-\pgf@circ@res@up}}
    \pgfpathlineto{\pgfpoint{\pgf@circ@res@left}{-\pgf@circ@res@up}}
    \pgfusepath{draw}

    \pgfsetlinewidth{\pgfstartlinewidth}
    \pgftext[top,x=.85\pgf@circ@res@left,y=.75\pgf@circ@res@down]{\pgf@circ@font@tiny$\vartheta$}
    \pgfsetarrowsend{latexslim}
    \pgfpathmoveto{\pgfpoint{.62\pgf@circ@res@left}{\pgf@circ@res@down}}
    \pgfpathlineto{\pgfpoint{.62\pgf@circ@res@left}{.7\pgf@circ@res@down}}
    \pgfusepath{draw}
    \pgfpathmoveto{\pgfpoint{.45\pgf@circ@res@left}{.7\pgf@circ@res@down}}
    \pgfpathlineto{\pgfpoint{.45\pgf@circ@res@left}{\pgf@circ@res@down}}
    \pgfusepath{draw}
}

%% Generic tunable
\pgfcircdeclarebipolescaled{resistors}
{
    \savedanchor{\wiper}{
            \pgf@circ@scaled@Rlen=\ctikzvalof{\ctikzclass/scale}\pgf@circ@Rlen
            \pgf@ya=\ctikzvalof{bipoles/generic potentiometer/height}\pgf@circ@scaled@Rlen
            \pgf@y=.5\pgf@ya
            \pgf@xa=\ctikzvalof{bipoles/generic potentiometer/width}\pgf@circ@scaled@Rlen
            \pgfmathsetlength{\pgf@x}{(\ctikzvalof{bipoles/generic potentiometer/wiper pos}-0.5)*\pgf@xa}
        }
    \anchor{wiper}{\wiper}
    \anchor{W}{\wiper}
}
{\ctikzvalof{bipoles/generic potentiometer/height 2}}
{genericpotentiometer}
{\ctikzvalof{bipoles/generic potentiometer/height}}
{\ctikzvalof{bipoles/generic potentiometer/width}}
{

    \pgf@circ@res@step = \ctikzvalof{bipoles/resistor/width}\pgf@circ@scaled@Rlen
    \divide \pgf@circ@res@step by 14

    \pgfscope
        \pgfpathrectanglecorners{\pgfpoint{\pgf@circ@res@left}{-\pgf@circ@res@down}}{\pgfpoint{\pgf@circ@res@right}{\pgf@circ@res@down}}
        \pgf@circ@setlinewidth{bipoles}{\pgfstartlinewidth}
        \pgf@circ@draworfill
    \endpgfscope
    \pgfscope
        %\pgfsetlinewidth{\pgfstartlinewidth}
        \pgfsetarrowsend{latexslim}
        \pgfextractx{\pgf@circ@res@other}{\wiper}
        \pgfpathmoveto{\pgfpoint{\pgf@circ@res@other}{\pgf@circ@res@up}}
        \pgfpathlineto{\pgfpoint{\pgf@circ@res@other}{-\pgf@circ@res@down}}
        \pgfusepath{draw}
    \endpgfscope
}

%% Zig zag resistores
\def\pgf@circ@zigzag#1{%
    \divide \pgf@circ@res@step by \numexpr4*\zigs\relax

    \pgfpathmoveto{\pgfpoint{\pgf@circ@res@left-\pgfhorizontaltransformationadjustment*0.5*\pgflinewidth}{\pgf@circ@res@zero}}
    \pgf@circ@res@other = \pgf@circ@res@left
    \pgf@circ@count@a=\zigs\relax
    % first half zig
    \advance\pgf@circ@res@other by \pgf@circ@res@step
    \pgfpathlineto{\pgfpoint{\pgf@circ@res@other}{-#1\pgf@circ@res@down}}
    \pgfmathloop%
    \advance\pgf@circ@count@a by -1\relax% Loop zigs -1 times
    \ifnum\pgf@circ@count@a>0
        \advance\pgf@circ@res@other by 2\pgf@circ@res@step
        \pgfpathlineto{\pgfpoint{\pgf@circ@res@other}{#1\pgf@circ@res@down}}
        \advance\pgf@circ@res@other by 2\pgf@circ@res@step
        \pgfpathlineto{\pgfpoint{\pgf@circ@res@other}{-#1\pgf@circ@res@down}}
    \repeatpgfmathloop%
    % last zig and a half
    \advance\pgf@circ@res@other by 2\pgf@circ@res@step
    \pgfpathlineto{\pgfpoint{\pgf@circ@res@other}{#1\pgf@circ@res@down}}
    \advance\pgf@circ@res@other by \pgf@circ@res@step
    \pgfpathlineto{\pgfpoint{\pgf@circ@res@other}{\pgf@circ@res@zero}}
    \pgfsetbeveljoin
    \pgfusepath{draw}
}

%% Resistor
\pgfcircdeclarebipolescaled{resistors}
{
\savedmacro{\zigs}{\edef\zigs{\ctikzvalof{resistors/zigs}}}
}
{\ctikzvalof{bipoles/resistor/height}}
{resistor}
{\ctikzvalof{bipoles/resistor/height}}
{\ctikzvalof{bipoles/resistor/width}}
{
    \pgf@circ@setlinewidth{bipoles}{\pgfstartlinewidth}
    \pgftransformationadjustments
    \pgfmathsetlength{\pgf@circ@res@step}{\ctikzvalof{bipoles/resistor/width}*\scaledRlen+\pgfhorizontaltransformationadjustment*0.5*\pgflinewidth}
    \pgf@circ@zigzag{1}
}


%% Variable resistor
\pgfcircdeclarebipolescaled{resistors}
{
\savedmacro{\zigs}{\edef\zigs{\ctikzvalof{resistors/zigs}}}
}
{\ctikzvalof{bipoles/vresistor/height}}
{vresistor}
{\ctikzvalof{bipoles/vresistor/height}}
{\ctikzvalof{bipoles/vresistor/width}}
{
    \pgf@circ@setlinewidth{bipoles}{\pgfstartlinewidth}
    \pgftransformationadjustments
    \pgfmathsetlength{\pgf@circ@res@step}{\ctikzvalof{bipoles/vresistor/width}*\scaledRlen+\pgfhorizontaltransformationadjustment*0.5*\pgflinewidth}
    \pgf@circ@zigzag{.5}

    \pgfscope
        \pgfsetarrowsend{latexslim}
        \pgfpathmoveto{\pgfpoint{.4\pgf@circ@res@other}{\pgf@circ@res@up}}
        \pgfpathlineto{\pgfpoint{-.4\pgf@circ@res@other}{\pgf@circ@res@down}}
        \pgfusepath{draw}
    \endpgfscope
}

%% Potentiometer
\pgfcircdeclarebipolescaled{resistors}
{
    \savedanchor{\wiper}{
            \pgf@circ@scaled@Rlen=\ctikzvalof{\ctikzclass/scale}\pgf@circ@Rlen
            \pgf@ya=\ctikzvalof{bipoles/potentiometer/height}\pgf@circ@scaled@Rlen
            \pgf@y=.5\pgf@ya
            \pgf@xa=\ctikzvalof{bipoles/potentiometer/width}\pgf@circ@scaled@Rlen
            \pgfmathsetlength{\pgf@x}{(\ctikzvalof{bipoles/potentiometer/wiper pos}-0.5)*\pgf@xa}
        }
    \anchor{wiper}{\wiper}
    \anchor{W}{\wiper}
    \savedmacro{\zigs}{\edef\zigs{\ctikzvalof{resistors/zigs}}}
}
{\ctikzvalof{bipoles/potentiometer/height 2}}
{potentiometer}
{\ctikzvalof{bipoles/potentiometer/height}}
{\ctikzvalof{bipoles/potentiometer/width}}
{
    \pgf@circ@setlinewidth{bipoles}{\pgfstartlinewidth}
    \pgftransformationadjustments
    \pgfmathsetlength{\pgf@circ@res@step}{\ctikzvalof{bipoles/potentiometer/width}*\scaledRlen+\pgfhorizontaltransformationadjustment*0.5*\pgflinewidth}
    \pgf@circ@zigzag{1}

    \pgfscope
        %\pgfsetlinewidth{\pgfstartlinewidth}
        \pgfsetarrowsend{latexslim}
        \pgfextractx{\pgf@circ@res@other}{\wiper}
        \pgfpathmoveto{\pgfpoint{\pgf@circ@res@other}{\pgf@circ@res@up}}
        \pgfpathlineto{\pgfpoint{\pgf@circ@res@other}{-\pgf@circ@res@down}}
        \pgfusepath{draw}
    \endpgfscope
}

%% Resistive sensor
\pgfcircdeclarebipolescaled{resistors}
{% anchor for labelling the type of dependency
    \anchor{label}{\southwest\pgf@x=0.4\pgf@x}%
    \savedmacro{\zigs}{\edef\zigs{\ctikzvalof{resistors/zigs}}}
}
{\ctikzvalof{bipoles/resistivesens/height}}
{resistivesens}
{\ctikzvalof{bipoles/resistivesens/height}}
{\ctikzvalof{bipoles/resistivesens/width}}
{%
    \pgf@circ@setlinewidth{bipoles}{\pgfstartlinewidth}
    \pgftransformationadjustments
    \pgfmathsetlength{\pgf@circ@res@step}{\ctikzvalof{bipoles/resistivesens/width}*\scaledRlen+\pgfhorizontaltransformationadjustment*0.5*\pgflinewidth}
    \pgf@circ@zigzag{.5}

    \pgfscope
        \pgfpathmoveto{\pgfpoint{.4\pgf@circ@res@other}{\pgf@circ@res@up}}
        \pgfpathlineto{\pgfpoint{-.4\pgf@circ@res@other}{\pgf@circ@res@down}}
        \pgfpathlineto{\pgfpoint{-.9\pgf@circ@res@other}{\pgf@circ@res@down}}
        \pgfusepath{draw}
    \endpgfscope
}

%%%%%%%%%%%%%%
%% Capacitors
%%%%%%%%%%%%%

%% Plain Capacitor
\pgfcircdeclarebipolescaled{capacitors}
{}
{\ctikzvalof{bipoles/capacitor/height}}
{capacitor}
{\ctikzvalof{bipoles/capacitor/height}}
{\ctikzvalof{bipoles/capacitor/width}}
{
    \pgf@circ@setlinewidth{bipoles}{\pgfstartlinewidth}
    \pgfpathmoveto{\pgfpoint{\pgf@circ@res@left}{\pgf@circ@res@up}}
    \pgfpathlineto{\pgfpoint{\pgf@circ@res@left}{\pgf@circ@res@down}}

    \pgfpathmoveto{\pgfpoint{\pgf@circ@res@right}{\pgf@circ@res@up}}
    \pgfpathlineto{\pgfpoint{\pgf@circ@res@right}{\pgf@circ@res@down}}
    \pgfusepath{draw}

}

%% Capacitive sensor
\pgfcircdeclarebipolescaled{capacitors}
{
    \anchor{label}{\southwest\pgf@x=2.6\pgf@x\pgf@y=1.2\pgf@y}%
}
{\ctikzvalof{bipoles/capacitor/height}}
{capacitivesens}
{\ctikzvalof{bipoles/capacitor/height}}
{\ctikzvalof{bipoles/capacitor/width}}
{
    \pgf@circ@setlinewidth{bipoles}{\pgfstartlinewidth}
    \pgfpathmoveto{\pgfpoint{\pgf@circ@res@left}{\pgf@circ@res@up}}
    \pgfpathlineto{\pgfpoint{\pgf@circ@res@left}{\pgf@circ@res@down}}

    \pgfpathmoveto{\pgfpoint{\pgf@circ@res@right}{\pgf@circ@res@up}}
    \pgfpathlineto{\pgfpoint{\pgf@circ@res@right}{\pgf@circ@res@down}}
    \pgfusepath{draw}
    \pgfscope
        \pgfpathmoveto{\pgfpoint{2.6\pgf@circ@res@right}{\pgf@circ@res@up}}
        \pgfpathlineto{\pgfpoint{-2.6\pgf@circ@res@right}{1.2\pgf@circ@res@down}}
        \pgfpathlineto{\pgfpoint{-4.4\pgf@circ@res@right}{1.2\pgf@circ@res@down}}
        \pgfusepath{draw}
    \endpgfscope
}

%% Polar Capacitor (DEPRECATED)
\pgfcircdeclarebipolescaled{capacitors}
{}
{\ctikzvalof{bipoles/pcapacitor/height}}
{polarcapacitor}
{\ctikzvalof{bipoles/pcapacitor/height}}
{\ctikzvalof{bipoles/pcapacitor/width}}
{
    \pgf@circ@setlinewidth{bipoles}{\pgfstartlinewidth}

    \pgfscope
        \pgfpathmoveto{\pgfpoint{\pgf@circ@res@left+ \ctikzvalof{bipoles/pcapacitor/bend width}\pgf@circ@res@left}{\pgf@circ@res@up}}
        \pgftransformrotate{-90}
        \pgfpathsine{\pgfpoint{\pgf@circ@res@up}{-\ctikzvalof{bipoles/pcapacitor/bend width}\pgf@circ@res@left}}
        \pgfpathcosine{\pgfpoint{\pgf@circ@res@up}{\ctikzvalof{bipoles/pcapacitor/bend width}\pgf@circ@res@left}}
        \pgfusepath{draw}
    \endpgfscope

    \pgfpathmoveto{\pgfpoint{\pgf@circ@res@right}{\pgf@circ@res@up}}
    \pgfpathlineto{\pgfpoint{\pgf@circ@res@right}{\pgf@circ@res@down}}
    \pgfusepath{draw}
}


%% Curved capacitor
% see https://tex.stackexchange.com/questions/509594/polar-capacitor-orientation-in-circuitikz-seems-wrong
% for a rationale
%
\pgfcircdeclarebipolescaled{capacitors}
{}
{\ctikzvalof{bipoles/ccapacitor/height}}
{ccapacitor}
{\ctikzvalof{bipoles/ccapacitor/height}}
{\ctikzvalof{bipoles/ccapacitor/width}}
{
    \pgf@circ@setlinewidth{bipoles}{\pgfstartlinewidth}

    \pgfscope
        \pgfpathmoveto{\pgfpoint{\pgf@circ@res@right+ \ctikzvalof{bipoles/ccapacitor/bend width}\pgf@circ@res@right}{\pgf@circ@res@up}}
        \pgftransformrotate{-90}
        \pgfpathsine{\pgfpoint{\pgf@circ@res@up}{-\ctikzvalof{bipoles/ccapacitor/bend width}\pgf@circ@res@right}}
        \pgfpathcosine{\pgfpoint{\pgf@circ@res@up}{\ctikzvalof{bipoles/ccapacitor/bend width}\pgf@circ@res@right}}
        \pgfusepath{draw}
    \endpgfscope

    \pgfpathmoveto{\pgfpoint{\pgf@circ@res@left}{\pgf@circ@res@up}}
    \pgfpathlineto{\pgfpoint{\pgf@circ@res@left}{\pgf@circ@res@down}}
    \pgfusepath{draw}
}




%% Electrolytic Capacitor
\pgfcircdeclarebipolescaled{capacitors}
{}
{\ctikzvalof{bipoles/ecapacitor/height}}
{ecapacitor}
{\ctikzvalof{bipoles/ecapacitor/height}}
{\ctikzvalof{bipoles/ecapacitor/width}}
{
    \pgfsetrectcap
    % % % Draw plus pole
    \pgf@circ@setlinewidth{bipoles}{\pgfstartlinewidth}
    \pgfpathmoveto{\pgfpoint{\pgf@circ@res@left}{\pgf@circ@res@up}}
    \pgfpathlineto{\pgfpoint{.4\pgf@circ@res@left}{\pgf@circ@res@up}}
    \pgfpathlineto{\pgfpoint{.4\pgf@circ@res@left}{\pgf@circ@res@down}}
    \pgfpathlineto{\pgfpoint{\pgf@circ@res@left}{\pgf@circ@res@down}}
    \pgfpathlineto{\pgfpoint{\pgf@circ@res@left}{\pgf@circ@res@up}}
    \pgf@circ@draworfill
    % % Draw minus pole
    \pgfpathmoveto{\pgfpoint{\pgf@circ@res@right}{\pgf@circ@res@up}}
    \pgfpathlineto{\pgfpoint{.4\pgf@circ@res@right}{\pgf@circ@res@up}}
    \pgfpathlineto{\pgfpoint{.4\pgf@circ@res@right}{\pgf@circ@res@down}}
    \pgfpathlineto{\pgfpoint{\pgf@circ@res@right}{\pgf@circ@res@down}}
    \pgfpathlineto{\pgfpoint{\pgf@circ@res@right}{\pgf@circ@res@up}}
    \pgfsetfillcolor{\ctikzvalof{color}}
    \pgfusepath{draw,fill}
    \pgfsetfillcolor{\ctikzvalof{color}}
    % % plus pole annotation
    \pgftext[right,at=\pgfpoint{1.2\pgf@circ@res@left}{.6\pgf@circ@res@up}]
    {\ctikzvalof{bipoles/ecapacitor/font} $+$}
}

%% Variable Capacitor
\pgfcircdeclarebipolescaled{capacitors}
{}
{\ctikzvalof{bipoles/vcapacitor/height}}
{vcapacitor}
{\ctikzvalof{bipoles/vcapacitor/height}}
{\ctikzvalof{bipoles/vcapacitor/width}}
{
    \pgf@circ@res@step = \ctikzvalof{bipoles/vcapacitor/capacitor width} \pgf@circ@res@right

    \pgf@circ@setlinewidth{bipoles}{\pgfstartlinewidth}
    \pgfpathmoveto{\pgfpoint{-\pgf@circ@res@step}{\pgf@circ@res@up}}
    \pgfpathlineto{\pgfpoint{-\pgf@circ@res@step}{\pgf@circ@res@down}}

    \pgfpathmoveto{\pgfpoint{\pgf@circ@res@step}{\pgf@circ@res@up}}
    \pgfpathlineto{\pgfpoint{\pgf@circ@res@step}{\pgf@circ@res@down}}
    \pgfusepath{draw}

    \pgfscope
        \pgfsetarrowsend{latexslim}
        \pgfpathmoveto{\pgfpoint{\pgf@circ@res@left}{\pgf@circ@res@up}}
        \pgfpathlineto{\pgfpoint{\pgf@circ@res@right}{\pgf@circ@res@down}}
        \pgfusepath{draw}
    \endpgfscope

    \pgfpathmoveto{\pgfpoint{\pgf@circ@res@left}{0pt}}
    \pgfpathlineto{\pgfpoint{-\pgf@circ@res@step}{0pt}}
    \pgfpathmoveto{\pgfpoint{\pgf@circ@res@right}{0pt}}
    \pgfpathlineto{\pgfpoint{\pgf@circ@res@step}{0pt}}
    \pgfsetlinewidth{\pgfstartlinewidth}
    \pgfusepath{draw}
}


%% Piezoelectric Element

\pgfcircdeclarebipolescaled{capacitors}
{}
{\ctikzvalof{bipoles/piezoelectric/height}}
{piezoelectric}
{\ctikzvalof{bipoles/piezoelectric/height}}
{\ctikzvalof{bipoles/piezoelectric/width}}
{
    % \pgf@circ@res@step = \ctikzvalof{bipoles/piezoelectric/width}\pgf@circ@Rlen
    % \divide \pgf@circ@res@step by 5

    %% Outer markings
    \pgf@circ@setlinewidth{bipoles}{\pgfstartlinewidth}
    \pgfpathmoveto{\pgfpoint{\pgf@circ@res@left}{\pgf@circ@res@up}}
    \pgfpathlineto{\pgfpoint{\pgf@circ@res@left}{\pgf@circ@res@down}}

    \pgfpathmoveto{\pgfpoint{\pgf@circ@res@right}{\pgf@circ@res@up}}
    \pgfpathlineto{\pgfpoint{\pgf@circ@res@right}{\pgf@circ@res@down}}
    \pgfusepath{draw}

    %% Inner Box
    \pgf@circ@res@step = \pgf@circ@res@right \divide \pgf@circ@res@step by 10
    \pgfscope
        \pgf@circ@setlinewidth{bipoles}{\pgfstartlinewidth}
        \pgfpathrectanglecorners
            {\pgfpoint{\pgf@circ@res@left+4*\pgf@circ@res@step}{\pgf@circ@res@up-\pgf@circ@res@step}}
            {\pgfpoint{\pgf@circ@res@right-4*\pgf@circ@res@step}{\pgf@circ@res@down+\pgf@circ@res@step}}
        \pgf@circ@draworfill
    \endpgfscope
}

%%%%%%%%%%%%%%%
%% Inductors
%%%%%%%%%%%%%%%

%% cute inductor
\pgfcircdeclarebipolescaled{inductors}
{
    \savedanchor{\midtap}{%
        \pgf@x=0pt\relax
        \pgfmathsetlength{\pgf@y}{%
            (mod(\ctikzvalof{bipoles/cuteinductor/coils},2) ?%
            \ctikzvalof{bipoles/cuteinductor/height} :% even
            -\ctikzvalof{bipoles/cuteinductor/lower coil height}) %odd
            *\pgf@circ@scaled@Rlen/2
        }
    }
    \anchor{midtap}{\midtap}
}
{\ctikzvalof{bipoles/cuteinductor/lower coil height}}
{cuteinductor}
{\ctikzvalof{bipoles/cuteinductor/height}}
{\ctikzvalof{bipoles/cuteinductor/width}}
{
    \pgf@circ@setlinewidth{bipoles}{\pgfstartlinewidth}
    \pgftransformationadjustments

    \pgfmathsetlength{\pgf@circ@res@other}%width of small coil
    {0.5*\ctikzvalof{bipoles/cuteinductor/coil aspect}*\ctikzvalof{bipoles/cuteinductor/width}*\scaledRlen/(\ctikzvalof{bipoles/cuteinductor/coils}-1)}


    \pgfmathsetlength{\pgf@circ@res@step}
    {(\ctikzvalof{bipoles/cuteinductor/width}*\scaledRlen+\pgfhorizontaltransformationadjustment\pgflinewidth+(\ctikzvalof{bipoles/cuteinductor/coils}-1)*2*\pgf@circ@res@other)/\ctikzvalof{bipoles/cuteinductor/coils}/2}

    \pgfpathmoveto{\pgfpoint{\pgf@circ@res@left-\pgfhorizontaltransformationadjustment*0.5*\pgflinewidth}{-\pgfverticaltransformationadjustment*0.4*\pgfstartlinewidth}}%correct value would be 0.5 but arcs are not really flat, therefore 0.4 is better is (almost) all cases
    \foreach \x in {2,...,\ctikzvalof{bipoles/cuteinductor/coils}}
    {
        \pgfpatharc{180}{0}{\pgf@circ@res@step and \pgf@circ@res@up}
        \pgfpatharc{0}{-180}{\pgf@circ@res@other and -\pgf@circ@res@down}
    }
    \pgfpatharc{180}{0}{\pgf@circ@res@step and \pgf@circ@res@up}
    \pgfsetbuttcap
    \pgfsetbeveljoin
    \pgfusepath{stroke}
}

%% cute inductive sensor
\pgfcircdeclarebipolescaled{inductors}
{% anchor for labelling the type of dependency
    \anchor{label}{\southwest\pgf@x=0.8\pgf@x\pgf@y=2.6\pgf@y}%
    \savedanchor{\midtap}{%
        \pgf@x=0pt\relax
        \pgfmathsetlength{\pgf@y}{%
            (mod(\ctikzvalof{bipoles/cuteinductor/coils},2) ?%
            \ctikzvalof{bipoles/cuteinductor/height} :% even
            -\ctikzvalof{bipoles/cuteinductor/lower coil height}) %odd
            *\pgf@circ@scaled@Rlen/2
        }
    }
    \anchor{midtap}{\midtap}
}
{\ctikzvalof{bipoles/cuteinductor/lower coil height}}
{scuteinductor}
{\ctikzvalof{bipoles/cuteinductor/height}}
{\ctikzvalof{bipoles/cuteinductor/width}}
{
    \pgf@circ@setlinewidth{bipoles}{\pgfstartlinewidth}
    \pgftransformationadjustments

    \pgfmathsetlength{\pgf@circ@res@other}%width of small coil
    {0.5*\ctikzvalof{bipoles/cuteinductor/coil aspect}*\ctikzvalof{bipoles/cuteinductor/width}*\scaledRlen/(\ctikzvalof{bipoles/cuteinductor/coils}-1)}


    \pgfmathsetlength{\pgf@circ@res@step}
    {(\ctikzvalof{bipoles/cuteinductor/width}*\scaledRlen+\pgfhorizontaltransformationadjustment\pgflinewidth+(\ctikzvalof{bipoles/cuteinductor/coils}-1)*2*\pgf@circ@res@other)/\ctikzvalof{bipoles/cuteinductor/coils}/2}

    \pgfpathmoveto{\pgfpoint{\pgf@circ@res@left-\pgfhorizontaltransformationadjustment*0.5*\pgflinewidth}{-\pgfverticaltransformationadjustment*0.4*\pgfstartlinewidth}}%correct value would be 0.5 but arcs are not really flat, therefore 0.4 is better is (almost) all cases
    \foreach \x in {2,...,\ctikzvalof{bipoles/cuteinductor/coils}}
    {
        \pgfpatharc{180}{0}{\pgf@circ@res@step and \pgf@circ@res@up}
        \pgfpatharc{0}{-180}{\pgf@circ@res@other and -\pgf@circ@res@down}
    }
    \pgfpatharc{180}{0}{\pgf@circ@res@step and \pgf@circ@res@up}
    \pgfsetbuttcap
    \pgfsetbeveljoin
    \pgfusepath{stroke}
    \pgfscope
        \pgfpathmoveto{\pgfpoint{.8\pgf@circ@res@right}{2\pgf@circ@res@up}}
        \pgfpathlineto{\pgfpoint{-.8\pgf@circ@res@right}{2.6\pgf@circ@res@down}}
        \pgfpathlineto{\pgfpoint{-1.6\pgf@circ@res@right}{2.6\pgf@circ@res@down}}
        \pgfusepath{draw}
    \endpgfscope
}

%% cute choke
\pgfcircdeclarebipolescaled{inductors}
{
    \savedanchor{\midtap}{%
        \pgf@x=0pt\relax
        \pgfmathsetlength{\pgf@y}{%
            (mod(\ctikzvalof{bipoles/cuteinductor/coils},2) ?%
            \ctikzvalof{bipoles/cuteinductor/height} :% even
            -\ctikzvalof{bipoles/cuteinductor/lower coil height}) %odd
            *\pgf@circ@scaled@Rlen/2
        }
    }
    \anchor{midtap}{\midtap}
}
{\ctikzvalof{bipoles/cutechoke/lower coil height}}
{cutechoke}
{\ctikzvalof{bipoles/cutechoke/height}}
{\ctikzvalof{bipoles/cutechoke/width}}
{
    \pgf@circ@setlinewidth{bipoles}{\pgfstartlinewidth}
    \pgftransformationadjustments

    \pgfmathsetlength{\pgf@circ@res@other}%width of small coil
    {0.5*\ctikzvalof{bipoles/cutechoke/coil aspect}*\ctikzvalof{bipoles/cutechoke/width}*\scaledRlen/(\ctikzvalof{bipoles/cutechoke/coils}-1)}


    \pgfmathsetlength{\pgf@circ@res@step}
    {(\ctikzvalof{bipoles/cutechoke/width}*\scaledRlen+\pgfhorizontaltransformationadjustment\pgflinewidth+(\ctikzvalof{bipoles/cutechoke/coils}-1)*2*\pgf@circ@res@other)/\ctikzvalof{bipoles/cutechoke/coils}/2}

    \pgfpathmoveto{\pgfpoint{\pgf@circ@res@left-\pgfhorizontaltransformationadjustment*0.5*\pgflinewidth}{-\pgfverticaltransformationadjustment*0.4*\pgfstartlinewidth}}%correct value would be 0.5 but arcs are not really flat, therefore 0.4 is better is (almost) all cases
    \foreach \x in {2,...,\ctikzvalof{bipoles/cutechoke/coils}}
    {
        \pgfpatharc{180}{0}{\pgf@circ@res@step and \pgf@circ@res@up}
        \pgfpatharc{0}{-180}{\pgf@circ@res@other and -\pgf@circ@res@down}
    }
    \pgfpatharc{180}{0}{\pgf@circ@res@step and \pgf@circ@res@up}
    \pgfsetbuttcap
    \pgfsetbeveljoin
    \pgfusepath{stroke}

    \pgfpathmoveto{\pgfpoint{\pgf@circ@res@left}{-\pgfverticaltransformationadjustment*0.4*\pgfstartlinewidth+\ctikzvalof{bipoles/cutechoke/cdist}\pgf@circ@res@up}}
    \pgfpathlineto{\pgfpoint{\pgf@circ@res@right}{-\pgfverticaltransformationadjustment*0.4*\pgfstartlinewidth+\ctikzvalof{bipoles/cutechoke/cdist}\pgf@circ@res@up}}
    \pgf@circ@setlinewidth{bipoles}{\pgfstartlinewidth}
    \pgfsetlinewidth{\ctikzvalof{bipoles/cutechoke/cthick}\pgflinewidth}
    \pgfusepath{stroke}

    \ifpgf@circuit@bipole@twolineschoke
        \pgfpathmoveto{\pgfpoint{\pgf@circ@res@left}{-\pgfverticaltransformationadjustment*0.4*\pgfstartlinewidth+\ctikzvalof{bipoles/cutechoke/cdist}\pgf@circ@res@up+\ctikzvalof{bipoles/cutechoke/cstep}\pgf@circ@res@up}}
        \pgfpathlineto{\pgfpoint{\pgf@circ@res@right}{-\pgfverticaltransformationadjustment*0.4*\pgfstartlinewidth+\ctikzvalof{bipoles/cutechoke/cdist}\pgf@circ@res@up+\ctikzvalof{bipoles/cutechoke/cstep}\pgf@circ@res@up}}
        \pgf@circ@setlinewidth{bipoles}{\pgfstartlinewidth}
        \pgfsetlinewidth{\ctikzvalof{bipoles/cutechoke/cthick}\pgflinewidth}
        \pgfusepath{stroke}
    \fi
}

%% variable cute inductor
\pgfcircdeclarebipolescaled{inductors}
{
    \savedanchor{\midtap}{%
        \pgf@x=0pt\relax
        \pgfmathsetlength{\pgf@y}{%
            (mod(\ctikzvalof{bipoles/cuteinductor/coils},2) ?%
            \ctikzvalof{bipoles/cuteinductor/height} :% even
            -\ctikzvalof{bipoles/cuteinductor/lower coil height}) %odd
            *\pgf@circ@scaled@Rlen/2
        }
    }
    \anchor{midtap}{\midtap}
}
{\ctikzvalof{bipoles/vcuteinductor/lower coil height}}
{vcuteinductor}
{\ctikzvalof{bipoles/vcuteinductor/height}}
{\ctikzvalof{bipoles/vcuteinductor/width}}
{
    \pgf@circ@setlinewidth{bipoles}{\pgfstartlinewidth}
    \pgftransformationadjustments

    \pgfmathsetlength{\pgf@circ@res@other}%width of small coil
    {0.5*\ctikzvalof{bipoles/vcuteinductor/coil aspect}*\ctikzvalof{bipoles/vcuteinductor/width}*\scaledRlen/(\ctikzvalof{bipoles/vcuteinductor/coils}-1)}


    \pgfmathsetlength{\pgf@circ@res@step}
    {(\ctikzvalof{bipoles/vcuteinductor/width}*\scaledRlen+\pgfhorizontaltransformationadjustment\pgflinewidth+(\ctikzvalof{bipoles/vcuteinductor/coils}-1)*2*\pgf@circ@res@other)/\ctikzvalof{bipoles/vcuteinductor/coils}/2}

    \pgfscope
        \pgfsetarrowsend{latexslim}
        \pgfpathmoveto{\pgfpoint{.4\pgf@circ@res@left}{\pgf@circ@res@down}}
        \pgfpathlineto{\pgfpoint{.4\pgf@circ@res@right}{\pgf@circ@res@up}}
        \pgfusepath{draw}
    \endpgfscope

    \pgfpathmoveto{\pgfpoint{\pgf@circ@res@left-\pgfhorizontaltransformationadjustment*0.5*\pgflinewidth}{-\pgfverticaltransformationadjustment*0.4*\pgfstartlinewidth}}%correct value would be 0.5 but arcs are not really flat, therefore 0.4 is better is (almost) all cases
    \foreach \x in {2,...,\ctikzvalof{bipoles/vcuteinductor/coils}}
    {
        \pgfpatharc{180}{0}{\pgf@circ@res@step and .5\pgf@circ@res@up}
        \pgfpatharc{0}{-180}{\pgf@circ@res@other and -.5\pgf@circ@res@down}
    }
    \pgfpatharc{180}{0}{\pgf@circ@res@step and .5\pgf@circ@res@up}
    \pgfsetbuttcap
    \pgfsetbeveljoin
    \pgfusepath{stroke}
}

%% american inductor
\pgfcircdeclarebipolescaled{inductors}
{
    \savedanchor{\midtap}{%
        \pgf@x=0pt\relax
        \pgfmathsetlength{\pgf@y}{%
            (mod(\ctikzvalof{bipoles/americaninductor/coils},2) ?%
            2*\ctikzvalof{bipoles/americaninductor/coil height} :% even
            0) %odd
            *\pgf@circ@scaled@Rlen/2
        }
    }
    \anchor{midtap}{\midtap}
}
{\ctikzvalof{bipoles/americaninductor/height 2}}
{americaninductor}
{\ctikzvalof{bipoles/americaninductor/height}}
{\ctikzvalof{bipoles/americaninductor/width}}
{
    \pgf@circ@res@step=\ctikzvalof{bipoles/americaninductor/width}\pgf@circ@scaled@Rlen
    \pgf@circ@setlinewidth{bipoles}{\pgfstartlinewidth}
    \pgftransformationadjustments
    \advance \pgf@circ@res@step by \pgfhorizontaltransformationadjustment\pgflinewidth
    \divide \pgf@circ@res@step by \ctikzvalof{bipoles/americaninductor/coils}
    \divide \pgf@circ@res@step by 2
    \pgf@circ@res@other = \ctikzvalof{bipoles/americaninductor/coil height}\pgf@circ@scaled@Rlen

    \pgfpathmoveto{\pgfpoint{\pgf@circ@res@left-\pgfhorizontaltransformationadjustment*0.5*\pgflinewidth}{-\pgfverticaltransformationadjustment*0.4*\pgfstartlinewidth}}%correct value would be 0.5 but arcs are not really flat, therefore 0.4 is better is (almost) all cases
    \foreach \x in {1,...,\ctikzvalof{bipoles/americaninductor/coils}}
    {
        \pgfpatharc{180}{0}{\pgf@circ@res@step and  \pgf@circ@res@other}
    }
    \pgfsetbuttcap
    \pgfsetbeveljoin
    \pgfusepath{stroke}
}


%% american inductive sensor
\pgfcircdeclarebipolescaled{inductors}
{% anchor for labelling the type of dependency
    \anchor{label}{\southwest\pgf@x=0.8\pgf@x\pgf@y=2.6\pgf@y}%
    \savedanchor{\midtap}{%
        \pgf@x=0pt\relax
        \pgfmathsetlength{\pgf@y}{%
            (mod(\ctikzvalof{bipoles/americaninductor/coils},2) ?%
            2*\ctikzvalof{bipoles/americaninductor/coil height} :% even
            0) %odd
            *\pgf@circ@scaled@Rlen/2
        }
    }
    \anchor{midtap}{\midtap}
}
{\ctikzvalof{bipoles/americaninductor/height 2}}
{samericaninductor}
{\ctikzvalof{bipoles/americaninductor/height}}
{\ctikzvalof{bipoles/americaninductor/width}}
{
    \pgf@circ@res@step=\ctikzvalof{bipoles/americaninductor/width}\pgf@circ@scaled@Rlen
    \pgf@circ@setlinewidth{bipoles}{\pgfstartlinewidth}
    \pgftransformationadjustments
    \advance \pgf@circ@res@step by \pgfhorizontaltransformationadjustment\pgflinewidth
    \divide \pgf@circ@res@step by \ctikzvalof{bipoles/americaninductor/coils}
    \divide \pgf@circ@res@step by 2
    \pgf@circ@res@other = \ctikzvalof{bipoles/americaninductor/coil height}\pgf@circ@scaled@Rlen

    \pgfpathmoveto{\pgfpoint{\pgf@circ@res@left-\pgfhorizontaltransformationadjustment*0.5*\pgflinewidth}{-\pgfverticaltransformationadjustment*0.4*\pgfstartlinewidth}}%correct value would be 0.5 but arcs are not really flat, therefore 0.4 is better is (almost) all cases
    \foreach \x in {1,...,\ctikzvalof{bipoles/americaninductor/coils}}
    {
        \pgfpatharc{180}{0}{\pgf@circ@res@step and  \pgf@circ@res@other}
    }
    \pgfsetbuttcap
    \pgfsetbeveljoin
    \pgfusepath{stroke}
    \pgfscope
        \pgfpathmoveto{\pgfpoint{.8\pgf@circ@res@right}{2\pgf@circ@res@up}}
        \pgfpathlineto{\pgfpoint{-.8\pgf@circ@res@right}{2.6\pgf@circ@res@down}}
        \pgfpathlineto{\pgfpoint{-1.6\pgf@circ@res@right}{2.6\pgf@circ@res@down}}
        \pgfusepath{draw}
    \endpgfscope
}

%% variable american inductor
\pgfcircdeclarebipolescaled{inductors}
{
    \savedanchor{\midtap}{%
        \pgf@x=0pt\relax
        \pgfmathsetlength{\pgf@y}{%
            (mod(\ctikzvalof{bipoles/vamericaninductor/coils},2) ?%
            2*\ctikzvalof{bipoles/vamericaninductor/coil height} :% even
            0) %odd
            *\pgf@circ@scaled@Rlen/2
        }
    }
    \anchor{midtap}{\midtap}
}
{\ctikzvalof{bipoles/vamericaninductor/height 2}}
{vamericaninductor}
{\ctikzvalof{bipoles/vamericaninductor/height}}
{\ctikzvalof{bipoles/vamericaninductor/width}}
{
    \pgf@circ@res@step=\ctikzvalof{bipoles/vamericaninductor/width}\pgf@circ@scaled@Rlen
    \pgf@circ@setlinewidth{bipoles}{\pgfstartlinewidth}
    \pgftransformationadjustments
    \advance \pgf@circ@res@step by \pgfhorizontaltransformationadjustment\pgflinewidth
    \divide \pgf@circ@res@step by \ctikzvalof{bipoles/vamericaninductor/coils}
    \divide \pgf@circ@res@step by 2
    \pgf@circ@res@other = \ctikzvalof{bipoles/vamericaninductor/coil height}\pgf@circ@scaled@Rlen

    \pgfpathmoveto{\pgfpoint{\pgf@circ@res@left-\pgfhorizontaltransformationadjustment*0.5*\pgflinewidth}{-\pgfverticaltransformationadjustment*0.4*\pgfstartlinewidth}}%correct value would be 0.5 but arcs are not really flat, therefore 0.4 is better is (almost) all cases
    \foreach \x in {1,...,\ctikzvalof{bipoles/vamericaninductor/coils}}
    {
        \pgfpatharc{180}{0}{\pgf@circ@res@step and  \pgf@circ@res@other}
    }
    \pgfsetbuttcap
    \pgfsetbeveljoin
    \pgfusepath{stroke}

    \pgfscope
        \pgfsetarrowsend{latexslim}
        \pgfpathmoveto{\pgfpoint{.4\pgf@circ@res@left}{\pgf@circ@res@down}}
        \pgfpathlineto{\pgfpoint{-.4\pgf@circ@res@left}{\pgf@circ@res@up}}
        \pgfusepath{draw}
    \endpgfscope
}

%% Generic bipole, filled - used as inductor by some
\pgfcircdeclarebipolescaled{inductors}
{
    \anchor{midtap}{\northeast\pgf@x=0pt\relax}
}
{\ctikzvalof{bipoles/fullgeneric/height}}
{fullgeneric}
{\ctikzvalof{bipoles/fullgeneric/height}}
{\ctikzvalof{bipoles/fullgeneric/width}}
{
    \pgfpathrectanglecorners{\pgfpoint{\pgf@circ@res@left}{\pgf@circ@res@up}}{\pgfpoint{\pgf@circ@res@right}{\pgf@circ@res@down}}
    \pgf@circ@setlinewidth{bipoles}{\pgfstartlinewidth}

    \pgfusepath{draw,fill}
}

%% Generic sensor, filled - used as inductive sensor by some
\pgfcircdeclarebipolescaled{inductors}
{% anchor for labelling the type of dependency
    \anchor{label}{\southwest\pgf@x=0.4\pgf@x\pgf@y=2\pgf@y}%
    \anchor{midtap}{\northeast\pgf@x=0pt\relax}
}
{\ctikzvalof{bipoles/fullgeneric/height}}
{sfullgeneric}
{\ctikzvalof{bipoles/fullgeneric/height}}
{\ctikzvalof{bipoles/fullgeneric/width}}
{

    \pgfpathrectanglecorners{\pgfpoint{\pgf@circ@res@left}{\pgf@circ@res@up}}{\pgfpoint{\pgf@circ@res@right}{\pgf@circ@res@down}}
    \pgf@circ@setlinewidth{bipoles}{\pgfstartlinewidth}
    \pgfusepath{draw,fill}
    %\pgfscope
    \pgfpathmoveto{\pgfpoint{-\pgf@circ@res@left}{-2\pgf@circ@res@down}}
    \pgfpathlineto{\pgfpoint{.4\pgf@circ@res@left}{2\pgf@circ@res@down}}
    \pgfpathlineto{\pgfpoint{\pgf@circ@res@left}{2\pgf@circ@res@down}}
    \pgfusepath{draw}
    %\endpgfscope
}

%% Generic full tunable
\pgfcircdeclarebipolescaled{inductors}
{
    \anchor{midtap}{\northeast\pgf@x=0pt\relax}
}
{\ctikzvalof{bipoles/tfullgeneric/height}}
{tfullgeneric}
{\ctikzvalof{bipoles/tfullgeneric/height}}
{\ctikzvalof{bipoles/tfullgeneric/width}}
{
    \pgf@circ@res@step = \ctikzvalof{bipoles/resistor/width}\pgf@circ@Rlen
    \divide \pgf@circ@res@step by 14

    \pgfpathrectanglecorners{\pgfpoint{\pgf@circ@res@left}{.4\pgf@circ@res@up}}{\pgfpoint{\pgf@circ@res@right}{.4\pgf@circ@res@down}}
    \pgf@circ@setlinewidth{bipoles}{\pgfstartlinewidth}
    \pgfusepath{draw,fill}

    \pgfscope
        \pgfsetarrowsend{latexslim}
        \pgfpathmoveto{\pgfpoint{.5\pgf@circ@res@left}{\pgf@circ@res@up}}
        \pgfpathlineto{\pgfpoint{-.5\pgf@circ@res@left}{\pgf@circ@res@down}}
        \pgfusepath{draw}
    \endpgfscope
}

%%%%%%%%%%%
%% Battery
%%%%%%%%%%%

%% Battery
\pgfcircdeclarebipolescaled{batteries}
{}
{\ctikzvalof{bipoles/battery/height}}
{battery}
{\ctikzvalof{bipoles/battery/height}}
{\ctikzvalof{bipoles/battery/width}}
{
    \pgf@circ@res@step = \ctikzvalof{bipoles/battery/width}\pgf@circ@scaled@Rlen
    \divide \pgf@circ@res@step by 6

    \pgf@circ@setlinewidth{bipoles}{\pgfstartlinewidth}

    \ifpgf@circ@fixbatteries
        \pgfpathmoveto{\pgfpoint{\pgf@circ@res@left}{.5\pgf@circ@res@up}}
        \pgfpathlineto{\pgfpoint{\pgf@circ@res@left}{.5\pgf@circ@res@down}}

        \pgfpathmoveto{\pgfpoint{-\pgf@circ@res@step}{\pgf@circ@res@up}}
        \pgfpathlineto{\pgfpoint{-\pgf@circ@res@step}{\pgf@circ@res@down}}

        \pgfpathmoveto{\pgfpoint{\pgf@circ@res@step}{.5\pgf@circ@res@up}}
        \pgfpathlineto{\pgfpoint{\pgf@circ@res@step}{.5\pgf@circ@res@down}}

        \pgfpathmoveto{\pgfpoint{\pgf@circ@res@right}{\pgf@circ@res@up}}
        \pgfpathlineto{\pgfpoint{\pgf@circ@res@right}{\pgf@circ@res@down}}
    \else
        \pgfpathmoveto{\pgfpoint{\pgf@circ@res@left}{\pgf@circ@res@up}}
        \pgfpathlineto{\pgfpoint{\pgf@circ@res@left}{\pgf@circ@res@down}}

        \pgfpathmoveto{\pgfpoint{-\pgf@circ@res@step}{.5\pgf@circ@res@up}}
        \pgfpathlineto{\pgfpoint{-\pgf@circ@res@step}{.5\pgf@circ@res@down}}

        \pgfpathmoveto{\pgfpoint{\pgf@circ@res@step}{\pgf@circ@res@up}}
        \pgfpathlineto{\pgfpoint{\pgf@circ@res@step}{\pgf@circ@res@down}}

        \pgfpathmoveto{\pgfpoint{\pgf@circ@res@right}{.5\pgf@circ@res@up}}
        \pgfpathlineto{\pgfpoint{\pgf@circ@res@right}{.5\pgf@circ@res@down}}
    \fi
    \pgfusepath{draw}
}


%% Battery 1 % poles with equl thickness

\pgfcircdeclarebipolescaled{batteries}
{}
{\ctikzvalof{bipoles/battery1/height}}
{battery1}
{\ctikzvalof{bipoles/battery1/height}}
{\ctikzvalof{bipoles/battery1/width}}
{
    \pgfpathmoveto{\pgfpoint{.33\pgf@circ@res@right}{0}}
    \pgfpathlineto{\pgfpoint{\pgf@circ@res@right}{0}}
    \pgfpathmoveto{\pgfpoint{.33\pgf@circ@res@left}{0}}
    \pgfpathlineto{\pgfpoint{\pgf@circ@res@left}{0}}
    \pgfusepath{draw}

    \pgf@circ@setlinewidth{bipoles}{\pgfstartlinewidth}
    \ifpgf@circ@fixbatteries
        \pgfpathmoveto{\pgfpoint{.33\pgf@circ@res@right}{\pgf@circ@res@up}}
        \pgfpathlineto{\pgfpoint{.33\pgf@circ@res@right}{\pgf@circ@res@down}}

        \pgfpathmoveto{\pgfpoint{.33\pgf@circ@res@left}{.5\pgf@circ@res@up}}
        \pgfpathlineto{\pgfpoint{.33\pgf@circ@res@left}{.5\pgf@circ@res@down}}
    \else
        \pgfpathmoveto{\pgfpoint{.33\pgf@circ@res@right}{.5\pgf@circ@res@up}}
        \pgfpathlineto{\pgfpoint{.33\pgf@circ@res@right}{.5\pgf@circ@res@down}}

        \pgfpathmoveto{\pgfpoint{.33\pgf@circ@res@left}{\pgf@circ@res@up}}
        \pgfpathlineto{\pgfpoint{.33\pgf@circ@res@left}{\pgf@circ@res@down}}
    \fi
    \pgfusepath{draw}
}

%% Battery 2 % negative pole thicker

\pgfcircdeclarebipolescaled{batteries}
{}
{\ctikzvalof{bipoles/battery2/height}}
{battery2}
{\ctikzvalof{bipoles/battery2/height}}
{\ctikzvalof{bipoles/battery2/width}}
{
    \pgfpathmoveto{\pgfpoint{.33\pgf@circ@res@right}{0}}
    \pgfpathlineto{\pgfpoint{\pgf@circ@res@right}{0}}
    \pgfpathmoveto{\pgfpoint{.33\pgf@circ@res@left}{0}}
    \pgfpathlineto{\pgfpoint{\pgf@circ@res@left}{0}}
    \pgfusepath{draw}

    \pgf@circ@setlinewidth{bipoles}{\pgfstartlinewidth}
    \ifpgf@circ@fixbatteries
        \pgfpathmoveto{\pgfpoint{.33\pgf@circ@res@right}{\pgf@circ@res@up}}
        \pgfpathlineto{\pgfpoint{.33\pgf@circ@res@right}{\pgf@circ@res@down}}
        \pgfusepath{draw}
        \pgfsetlinewidth{3\pgflinewidth}
        \pgfpathmoveto{\pgfpoint{.33\pgf@circ@res@left}{.5\pgf@circ@res@up}}
        \pgfpathlineto{\pgfpoint{.33\pgf@circ@res@left}{.5\pgf@circ@res@down}}
    \else
        \pgfpathmoveto{\pgfpoint{.33\pgf@circ@res@left}{\pgf@circ@res@up}}
        \pgfpathlineto{\pgfpoint{.33\pgf@circ@res@left}{\pgf@circ@res@down}}
        \pgfusepath{draw}
        \pgfsetlinewidth{3\pgflinewidth}
        \pgfpathmoveto{\pgfpoint{.33\pgf@circ@res@right}{.5\pgf@circ@res@up}}
        \pgfpathlineto{\pgfpoint{.33\pgf@circ@res@right}{.5\pgf@circ@res@down}}
    \fi
    \pgfusepath{draw}
}

%%%%%%%%%%%
%% Round and diamond sources
%%%%%%%%%%%

%% Independent voltage source
\pgfcircdeclarebipolescaled{sources}
{}
{\ctikzvalof{bipoles/vsource/height}}
{vsource}
{\ctikzvalof{bipoles/vsource/height}}
{\ctikzvalof{bipoles/vsource/width}}
{

    \pgf@circ@setlinewidth{bipoles}{\pgfstartlinewidth}

    \pgfpathellipse{\pgfpointorigin}{\pgfpoint{0}{\pgf@circ@res@up}}{\pgfpoint{\pgf@circ@res@left}{0}}
    \pgf@circ@draworfill
    \pgfpathmoveto{\pgfpoint{\pgf@circ@res@left}{\pgf@circ@res@zero}}
    \pgfpathlineto{\pgfpoint{\pgf@circ@res@right}{\pgf@circ@res@zero}}
    \pgfusepath{draw}
}

%% To change the internal symbols of the voltage source american style
\ctikzset{bipoles/vsourceam/inner plus/.initial={$+$}}
\ctikzset{bipoles/vsourceam/inner minus/.initial={$-$}}
%% Independent voltage source - American style
\pgfcircdeclarebipolescaled{sources}
{}
{\ctikzvalof{bipoles/vsourceam/height}}
{vsourceAM}
{\ctikzvalof{bipoles/vsourceam/height}}
{\ctikzvalof{bipoles/vsourceam/width}}
{

    \pgf@circ@setlinewidth{bipoles}{\pgfstartlinewidth}
    \pgfpathellipse{\pgfpointorigin}{\pgfpoint{0}{\pgf@circ@res@up}}{\pgfpoint{\pgf@circ@res@left}{0}}
    \pgf@circ@draworfill
    \pgfsetcolor{\ctikzvalof{color}}
    \ifpgf@circ@oldvoltagedirection
    \pgftext[bottom,rotate=90,y=\ctikzvalof{bipoles/vsourceam/margin}\pgf@circ@res@down]{\ctikzvalof{bipoles/vsourceam/inner plus}}
        \pgftext[top,rotate=90,y=\ctikzvalof{bipoles/vsourceam/margin}\pgf@circ@res@up]{\ctikzvalof{bipoles/vsourceam/inner minus}}
    \else
        \pgftext[bottom,rotate=90,y=\ctikzvalof{bipoles/vsourceam/margin}\pgf@circ@res@down]{\ctikzvalof{bipoles/vsourceam/inner minus}}
        \pgftext[top,rotate=90,y=\ctikzvalof{bipoles/vsourceam/margin}\pgf@circ@res@up]{\ctikzvalof{bipoles/vsourceam/inner plus}}
    \fi
}

%% Independent sinusoidal voltage source
\pgfcircdeclarebipolescaled{sources}
{}
{\ctikzvalof{bipoles/vsourcesin/height}}
{vsourcesin}
{\ctikzvalof{bipoles/vsourcesin/height}}
{\ctikzvalof{bipoles/vsourcesin/width}}
{

    \pgf@circ@setlinewidth{bipoles}{\pgfstartlinewidth}
    \pgfpathellipse{\pgfpointorigin}{\pgfpoint{0}{\pgf@circ@res@up}}{\pgfpoint{\pgf@circ@res@left}{0}}
    \pgf@circ@draworfill

    \pgf@circ@res@up = .5\pgf@circ@res@up
    \pgfscope
        \pgftransformrotate{90}
        \pgfpathmoveto{\pgfpoint{-\pgf@circ@res@up}{0cm}}
        \pgfpathsine{\pgfpoint{.5\pgf@circ@res@up}{.5\pgf@circ@res@up}}
        \pgfpathcosine{\pgfpoint{.5\pgf@circ@res@up}{-.5\pgf@circ@res@up}}
        \pgfpathsine{\pgfpoint{.5\pgf@circ@res@up}{-.5\pgf@circ@res@up}}
        \pgfpathcosine{\pgfpoint{.5\pgf@circ@res@up}{.5\pgf@circ@res@up}}
        \pgfusepath{draw}
    \endpgfscope
}

% Square Voltage source -  contributed by Alistair Kwan
\pgfcircdeclarebipolescaled{sources}
{}
{\ctikzvalof{bipoles/vsourcesquare/height}}
{vsourcesquare}
{\ctikzvalof{bipoles/vsourcesquare/height}}
{\ctikzvalof{bipoles/vsourcesquare/width}}
{
    \pgf@circ@setlinewidth{bipoles}{\pgfstartlinewidth}
    \pgfpathellipse{\pgfpointorigin}{\pgfpoint{0}{\pgf@circ@res@up}}{\pgfpoint{\pgf@circ@res@left}{0}}
    \pgf@circ@draworfill
    \pgf@circ@res@up = .5\pgf@circ@res@up
    \pgfscope
        \pgftransformrotate{90}
        \pgfpathmoveto{\pgfpoint{-1\pgf@circ@res@up}{0cm}}
        \pgfpathlineto{\pgfpoint{-1\pgf@circ@res@up}{1\pgf@circ@res@up}}
        \pgfpathlineto{\pgfpoint{0\pgf@circ@res@up}{1\pgf@circ@res@up}}
        \pgfpathlineto{\pgfpoint{0\pgf@circ@res@up}{-1\pgf@circ@res@up}}
        \pgfpathlineto{\pgfpoint{1\pgf@circ@res@up}{-1\pgf@circ@res@up}}
        \pgfpathlineto{\pgfpoint{1\pgf@circ@res@up}{0\pgf@circ@res@up}}
        \pgfusepath{draw}
    \endpgfscope
}

% Triangle Voltage source - contributed by Ralf Farkas
\pgfcircdeclarebipolescaled{sources}
{}
{\ctikzvalof{bipoles/vsourcetri/height}}
{vsourcetri}
{\ctikzvalof{bipoles/vsourcetri/height}}
{\ctikzvalof{bipoles/vsourcetri/width}}
{

    \pgf@circ@setlinewidth{bipoles}{\pgfstartlinewidth}
    \pgfpathellipse{\pgfpointorigin}{\pgfpoint{0}{\pgf@circ@res@up}}{\pgfpoint{\pgf@circ@res@left}{0}}
    \pgf@circ@draworfill

    \pgf@circ@res@up = .5\pgf@circ@res@up
    \pgfscope
        \pgftransformrotate{90}
        \pgfpathmoveto{\pgfpoint{-1\pgf@circ@res@up}{0cm}}
        \pgfpathlineto{\pgfpoint{-0.5\pgf@circ@res@up}{0.75\pgf@circ@res@up}}
        \pgfpathlineto{\pgfpoint{0.5\pgf@circ@res@up}{-0.75\pgf@circ@res@up}}
        \pgfpathlineto{\pgfpoint{1\pgf@circ@res@up}{0\pgf@circ@res@up}}
        \pgfusepath{draw}
    \endpgfscope
}


%% PV Source
\pgfcircdeclarebipolescaled{sources}
{}
{\ctikzvalof{bipoles/pvsource/height}}
{pvsource}
{\ctikzvalof{bipoles/pvsource/height}}
{\ctikzvalof{bipoles/pvsource/width}}
{
    \pgfpointorigin
    \pgf@circ@setlinewidth{bipoles}{\pgfstartlinewidth}
    \pgfpathellipse{\pgfpointorigin}{\pgfpoint{0}{\pgf@circ@res@up}}{\pgfpoint{\pgf@circ@res@left}{0}}
    \pgfpathmoveto{\pgfpoint{\pgf@circ@res@step}{\pgf@circ@res@up}}
    \pgf@circ@draworfill

    \pgfpathmoveto{\pgfpoint{.15\pgf@circ@res@right}{0}}
    \pgfpathlineto{\pgfpoint{\pgf@circ@res@right}{0}}
    \pgfpathmoveto{\pgfpoint{.15\pgf@circ@res@left}{0}}
    \pgfpathlineto{\pgfpoint{\pgf@circ@res@left}{0}}
    \pgfusepath{draw}

    \pgf@circ@setlinewidth{bipoles}{\pgfstartlinewidth}
    \pgfpathmoveto{\pgfpoint{.15\pgf@circ@res@left}{.4\pgf@circ@res@up}}
    \pgfpathlineto{\pgfpoint{.15\pgf@circ@res@left}{.4\pgf@circ@res@down}}

    \pgfpathmoveto{\pgfpoint{.15\pgf@circ@res@right}{.6\pgf@circ@res@up}}
    \pgfpathlineto{\pgfpoint{.15\pgf@circ@res@right}{.6\pgf@circ@res@down}}
    \pgfusepath{draw}

    %Arrow Part
    \pgfscope
        \pgfsetarrowsend{latex}
        \pgfpathmoveto{\pgfpointadd{\pgfpoint{.3\pgf@circ@res@left}{0}}{\pgfpointpolar{-45}{2.2\pgf@circ@res@up}}}
        \pgfpathlineto{\pgfpointadd{\pgfpoint{.3\pgf@circ@res@left}{0}}{\pgfpointpolar{-45}{1.3\pgf@circ@res@up}}}
        \pgfusepath{draw}
        \pgfpathmoveto{\pgfpointadd{\pgfpoint{0}{0.3\pgf@circ@res@up}}{\pgfpointpolar{-45}{2.2\pgf@circ@res@up}}}
        \pgfpathlineto{\pgfpointadd{\pgfpoint{0}{0.3\pgf@circ@res@up}}{\pgfpointpolar{-45}{1.3\pgf@circ@res@up}}}
        \pgfusepath{draw}
    \endpgfscope

}

%% Empty Source
\pgfcircdeclarebipolescaled{sources}
{}
{\ctikzvalof{bipoles/esource/height}}
{esource}
{\ctikzvalof{bipoles/esource/height}}
{\ctikzvalof{bipoles/esource/width}}
{
    \pgfpointorigin
    \pgf@circ@setlinewidth{bipoles}{\pgfstartlinewidth}
    \pgfpathellipse{\pgfpointorigin}{\pgfpoint{0}{\pgf@circ@res@up}}{\pgfpoint{\pgf@circ@res@left}{0}}
    \pgf@circ@draworfill
}

%% DC Current Source with open shape
\pgfcircdeclarebipolescaled{sources}
{}
{\ctikzvalof{bipoles/dcisource/height}}
{dcisource}
{\ctikzvalof{bipoles/dcisource/height}}
{\ctikzvalof{bipoles/dcisource/width}}
{
    \pgf@circ@setlinewidth{bipoles}{\pgfstartlinewidth}
    \pgfpathmoveto{\pgfpointorigin}
    \pgfpathellipse{\pgfpointorigin}{\pgfpoint{0}{\pgf@circ@res@up}}{\pgfpoint{\pgf@circ@res@left}{0}}
    \pgf@circ@maybefill
    \edef\@@angle{\ctikzvalof{bipoles/dcisource/angle}}
    \pgfpathmoveto{\pgfpointpolar{\@@angle}{\pgf@circ@res@up}}
    \pgfpatharc{\@@angle}{-\@@angle}{\pgf@circ@res@up}
    \pgfpathmoveto{\pgfpointpolar{180-\@@angle}{\pgf@circ@res@up}}
    \pgfpatharc{180-\@@angle}{180+\@@angle}{\pgf@circ@res@up}
    \pgfusepath{draw}

    \pgfpathmoveto{\pgfpoint{.7\pgf@circ@res@left}{\pgf@circ@res@zero}}
    \pgfpathlineto{\pgfpoint{.6\pgf@circ@res@right}{\pgf@circ@res@zero}}
    \pgfusepath{draw}
    \pgfscope
        \pgftransformshift{\pgfpoint{.5\pgf@circ@res@right}{\pgf@circ@res@zero}}
        \pgfnode{currarrow}{center}{}{}{\pgfusepath{stroke}}
    \endpgfscope
}

%% DC-Voltage source
\pgfcircdeclarebipolescaled{sources}
{}
{\ctikzvalof{bipoles/dcvsource/height}}
{dcvsource}
{\ctikzvalof{bipoles/dcvsource/height}}
{\ctikzvalof{bipoles/dcvsource/width}}
{
    \pgfpointorigin
    \pgf@circ@setlinewidth{bipoles}{\pgfstartlinewidth}
    \pgfpathellipse{\pgfpointorigin}{\pgfpoint{0}{\pgf@circ@res@up}}{\pgfpoint{\pgf@circ@res@left}{0}}
    \pgf@circ@draworfill
    \pgfpathmoveto{\pgfpoint{.2\pgf@circ@res@up}{.5\pgf@circ@res@left}}
    \pgfpathlineto{\pgfpoint{.2\pgf@circ@res@up}{.5\pgf@circ@res@right}}
    \pgfpathmoveto{\pgfpoint{.2\pgf@circ@res@down}{.5\pgf@circ@res@left}}
    \pgfpathlineto{\pgfpoint{.2\pgf@circ@res@down}{.5\pgf@circ@res@right}}
    \pgfusepath{draw}
}

%% Independent current source
\pgfcircdeclarebipolescaled{sources}
{}
{\ctikzvalof{bipoles/isource/height}}
{isource}
{\ctikzvalof{bipoles/isource/height}}
{\ctikzvalof{bipoles/isource/width}}
{
    \pgfpointorigin
    \pgf@circ@setlinewidth{bipoles}{\pgfstartlinewidth}
    \pgfpathellipse{\pgfpointorigin}{\pgfpoint{0}{\pgf@circ@res@up}}{\pgfpoint{\pgf@circ@res@left}{0}}
    \pgfpathmoveto{\pgfpoint{\pgf@circ@res@step}{\pgf@circ@res@up}}
    \pgfpathlineto{\pgfpoint{\pgf@circ@res@step}{\pgf@circ@res@down}}
    \pgf@circ@draworfill
}

%% Independent double oo source
\pgfcircdeclarebipolescaled{sources}
{}
{\ctikzvalof{bipoles/oosource/height}}
{oosource}
{\ctikzvalof{bipoles/oosource/height}}
{\ctikzvalof{bipoles/oosource/width}}
{
    \pgfpointorigin
    \pgf@circ@setlinewidth{bipoles}{\pgfstartlinewidth}
    \pgfpathcircle{\pgfpoint{\ctikzvalof{bipoles/oosource/circleoffset}\pgf@circ@res@left}{0}}
        {\ctikzvalof{bipoles/oosource/circlesize}\pgf@circ@res@left}
    \pgf@circ@maybefill
    \pgfpathcircle{\pgfpoint{\ctikzvalof{bipoles/oosource/circleoffset}\pgf@circ@res@right}{0}}
        {\ctikzvalof{bipoles/oosource/circlesize}\pgf@circ@res@right}
    \pgf@circ@draworfill
    \pgfpathcircle{\pgfpoint{\ctikzvalof{bipoles/oosource/circleoffset}\pgf@circ@res@left}{0}}
        {\ctikzvalof{bipoles/oosource/circlesize}\pgf@circ@res@left}
    \pgfusepath{draw}
}

% % % winding symbols
% triangle
\def\pgf@circ@delta#1{
    \pgfscope
        \pgftransformscale{-.01\pgf@circ@res@left*#1}
        \def\pgfcircmathresult{\expandafter\pgf@circ@stripdecimals\pgf@circ@direction\pgf@nil}
            \pgftransformrotate{-\pgfcircmathresult}

        \pgf@circ@setlinewidth{bipoles}{\pgfstartlinewidth}
        \pgfpathmoveto{\pgfpoint{\pgf@circ@res@left}{\pgf@circ@res@down}}
        \pgfpathlineto{\pgfpoint{\pgf@circ@res@right}{\pgf@circ@res@down}}
        \pgfpathlineto{\pgfpoint{0}{.866\pgf@circ@res@up}}
        \pgfpathclose
        \pgfusepath{stroke}
    \endpgfscope
}

% star
\def\pgf@circ@wye#1{
    \pgfscope
        \pgftransformscale{-.015\pgf@circ@res@left*#1}
        \def\pgfcircmathresult{\expandafter\pgf@circ@stripdecimals\pgf@circ@direction\pgf@nil}
            \pgftransformrotate{-\pgfcircmathresult}
    
        \pgf@circ@setlinewidth{bipoles}{\pgfstartlinewidth}
        \pgfpathmoveto{\pgfpoint{0}{\pgf@circ@res@down}}
        \pgfpathlineto{\pgfpointorigin}
        \pgfpathlineto{\pgfpointpolar{-30}{\pgf@circ@res@down}}
        \pgfpathmoveto{\pgfpointorigin}
        \pgfpathlineto{\pgfpointpolar{-150}{\pgf@circ@res@down}}
        \pgfusepath{stroke}
    \endpgfscope
}

% zigzag
\def\pgf@circ@zig#1{
    \pgfscope
        \pgftransformscale{-.015\pgf@circ@res@left*#1}
        \def\pgfcircmathresult{\expandafter\pgf@circ@stripdecimals\pgf@circ@direction\pgf@nil}
            \pgftransformrotate{-\pgfcircmathresult}
    
        \pgf@circ@setlinewidth{bipoles}{\pgfstartlinewidth}
        \pgfpathmoveto{\pgfpointorigin}
        \pgfpathlineto{\pgfpointpolar{90}{.5\pgf@circ@res@up}}
        \pgfpathlineto{\pgfpointpolar{60}{\pgf@circ@res@up}}
        
        \pgfpathmoveto{\pgfpointorigin}
        \pgfpathlineto{\pgfpointpolar{210}{.5\pgf@circ@res@up}}
        \pgfpathlineto{\pgfpointpolar{0}{\pgf@circ@res@down}}
        
        \pgfpathmoveto{\pgfpointorigin}
        \pgfpathlineto{\pgfpointpolar{330}{.5\pgf@circ@res@up}}
        \pgfpathlineto{\pgfpointpolar{-60}{\pgf@circ@res@up}}
        \pgfusepath{stroke}
    \endpgfscope
}

% % % % round three-phase transformer
\pgfcircdeclarebipolescaled{sources}
{}
{\ctikzvalof{bipoles/oosourcetrans/height}}
{oosourcetrans}
{\ctikzvalof{bipoles/oosourcetrans/height}}
{\ctikzvalof{bipoles/oosourcetrans/width}}
{
    
    \pgfpointorigin
    \pgf@circ@setlinewidth{bipoles}{\pgfstartlinewidth}
    \pgfpathcircle{\pgfpoint{\ctikzvalof{bipoles/oosourcetrans/circleoffset}\pgf@circ@res@left}{0}}
        {\ctikzvalof{bipoles/oosourcetrans/circlesize}\pgf@circ@res@left}
    \pgf@circ@maybefill
    \pgfpathcircle{\pgfpoint{\ctikzvalof{bipoles/oosourcetrans/circleoffset}\pgf@circ@res@right}{0}}
        {\ctikzvalof{bipoles/oosourcetrans/circlesize}\pgf@circ@res@right}
    \pgf@circ@draworfill
    \pgfpathcircle{\pgfpoint{\ctikzvalof{bipoles/oosourcetrans/circleoffset}\pgf@circ@res@left}{0}}
        {\ctikzvalof{bipoles/oosourcetrans/circlesize}\pgf@circ@res@left}
    \pgfusepath{draw}
    
    
% % %     % draw inner symbols
    
    %%primary winding
    \ifpgf@circ@prim@delta
        \pgfscope
            \pgftransformxshift{\ctikzvalof{bipoles/oosourcetrans/circlesize}\pgf@circ@res@left}
            \pgf@circ@delta{\ctikzvalof{bipoles/oosourcetrans/vectorgroupscale}}
        \endpgfscope
        
    \else\ifpgf@circ@prim@wye
        \pgfscope
            \pgftransformxshift{\ctikzvalof{bipoles/oosourcetrans/circlesize}\pgf@circ@res@left}
            \pgf@circ@wye{\ctikzvalof{bipoles/oosourcetrans/vectorgroupscale}}
        \endpgfscope
    
    \else\ifpgf@circ@prim@zig
        \pgfscope
            \pgftransformxshift{\ctikzvalof{bipoles/oosourcetrans/circlesize}\pgf@circ@res@left}
            \pgf@circ@zig{\ctikzvalof{bipoles/oosourcetrans/vectorgroupscale}}
        \endpgfscope
    \fi\fi\fi
    
    %%secondary winding
    \ifpgf@circ@sec@delta
        \pgfscope
            \pgftransformxshift{\ctikzvalof{bipoles/oosourcetrans/circlesize}\pgf@circ@res@right}
            \pgf@circ@delta{\ctikzvalof{bipoles/oosourcetrans/vectorgroupscale}}
        \endpgfscope
        
    \else\ifpgf@circ@sec@wye
        \pgfscope
            \pgftransformxshift{\ctikzvalof{bipoles/oosourcetrans/circlesize}\pgf@circ@res@right}
            \pgf@circ@wye{\ctikzvalof{bipoles/oosourcetrans/vectorgroupscale}}
        \endpgfscope
    
    \else\ifpgf@circ@sec@zig
        \pgfscope
            \pgftransformxshift{\ctikzvalof{bipoles/oosourcetrans/circlesize}\pgf@circ@res@right}
            \pgf@circ@zig{\ctikzvalof{bipoles/oosourcetrans/vectorgroupscale}}
        \endpgfscope
    \fi\fi\fi
}


% % % % ooosource for three phase transformer
\pgfcircdeclarebipolescaled{sources}
{   
    \anchor{right}{
            \northeast
            \pgf@y=0pt
            \pgfmathparse{
                \ctikzvalof{bipoles/ooosource/circleoffset}* sin(30) +
    %%the sqrt must be > 0, the circles have to intersect
                sqrt(
                    pow(\ctikzvalof{bipoles/ooosource/circlesize},2) -
                    pow(\ctikzvalof{bipoles/ooosource/circleoffset}*cos(30),2)
                )
            }
            \pgf@x=\pgfmathresult\pgf@x
    }
    \anchor{east}{
            \northeast
            \pgf@y=0pt
    }
    \savedanchor{\centerprim}{
            \pgf@circ@scaled@Rlen=\ctikzvalof{bipoles/ooosource/height}\pgf@circ@Rlen
            \pgf@circ@scaled@Rlen=-\ctikzvalof{bipoles/ooosource/circleoffset}\pgf@circ@scaled@Rlen
            \pgf@y=0pt
            \pgf@x=.5\pgf@circ@scaled@Rlen
    }
    \anchor{centerprim}{
            \centerprim
    }
    \savedanchor{\centersec}{
            \pgf@circ@scaled@Rlen=\ctikzvalof{bipoles/ooosource/height}\pgf@circ@Rlen
            \pgf@circ@scaled@Rlen=-\ctikzvalof{bipoles/ooosource/circleoffset}\pgf@circ@scaled@Rlen
            \pgfpointpolar{60}{.5\pgf@circ@scaled@Rlen}
            \pgf@y=-\pgf@y
            \pgf@x=-\pgf@x
    }
    \anchor{centersec}{
            \centersec
    }
    \savedanchor{\centertert}{
            \pgf@circ@scaled@Rlen=\ctikzvalof{bipoles/ooosource/height}\pgf@circ@Rlen
            \pgf@circ@scaled@Rlen=-\ctikzvalof{bipoles/ooosource/circleoffset}\pgf@circ@scaled@Rlen
            \pgfpointpolar{60}{.5\pgf@circ@scaled@Rlen}
            \pgf@y=\pgf@y
            \pgf@x=-\pgf@x
    }
    \anchor{centertert}{
            \centertert
    }
            
    % add some anchors in case the are needed :)
    \anchor{prim1}{
            \pgf@circ@scaled@Rlen=\ctikzvalof{bipoles/ooosource/height}\pgf@circ@Rlen
            \pgf@circ@scaled@Rlen=\ctikzvalof{bipoles/ooosource/circlesize}\pgf@circ@scaled@Rlen
            \pgfpointadd{\centerprim}{\pgfpointpolar{135}{.5\pgf@circ@scaled@Rlen}}
    }
    \anchor{prim2}{
            \pgf@circ@scaled@Rlen=\ctikzvalof{bipoles/ooosource/height}\pgf@circ@Rlen
            \pgf@circ@scaled@Rlen=\ctikzvalof{bipoles/ooosource/circlesize}\pgf@circ@scaled@Rlen
            \pgfpointadd{\centerprim}{\pgfpointpolar{-135}{.5\pgf@circ@scaled@Rlen}}
    }
    \anchor{sec1}{
            \pgf@circ@scaled@Rlen=\ctikzvalof{bipoles/ooosource/height}\pgf@circ@Rlen
            \pgf@circ@scaled@Rlen=\ctikzvalof{bipoles/ooosource/circlesize}\pgf@circ@scaled@Rlen
            \pgfpointadd{\centersec}{\pgfpointpolar{0}{.5\pgf@circ@scaled@Rlen}}
    }
    \anchor{sec2}{
            \pgf@circ@scaled@Rlen=\ctikzvalof{bipoles/ooosource/height}\pgf@circ@Rlen
            \pgf@circ@scaled@Rlen=\ctikzvalof{bipoles/ooosource/circlesize}\pgf@circ@scaled@Rlen
            \pgfpointadd{\centersec}{\pgfpointpolar{45}{.5\pgf@circ@scaled@Rlen}}
    }
    \anchor{sec3}{
            \pgf@circ@scaled@Rlen=\ctikzvalof{bipoles/ooosource/height}\pgf@circ@Rlen
            \pgf@circ@scaled@Rlen=\ctikzvalof{bipoles/ooosource/circlesize}\pgf@circ@scaled@Rlen
            \pgfpointadd{\centersec}{\pgfpointpolar{90}{.5\pgf@circ@scaled@Rlen}}
    }
    \anchor{tert1}{
            \pgf@circ@scaled@Rlen=\ctikzvalof{bipoles/ooosource/height}\pgf@circ@Rlen
            \pgf@circ@scaled@Rlen=\ctikzvalof{bipoles/ooosource/circlesize}\pgf@circ@scaled@Rlen
            \pgfpointadd{\centertert}{\pgfpointpolar{0}{.5\pgf@circ@scaled@Rlen}}
    }
    \anchor{tert2}{
            \pgf@circ@scaled@Rlen=\ctikzvalof{bipoles/ooosource/height}\pgf@circ@Rlen
            \pgf@circ@scaled@Rlen=\ctikzvalof{bipoles/ooosource/circlesize}\pgf@circ@scaled@Rlen
            \pgfpointadd{\centertert}{\pgfpointpolar{-45}{.5\pgf@circ@scaled@Rlen}}
    }
    \anchor{tert3}{
            \pgf@circ@scaled@Rlen=\ctikzvalof{bipoles/ooosource/height}\pgf@circ@Rlen
            \pgf@circ@scaled@Rlen=\ctikzvalof{bipoles/ooosource/circlesize}\pgf@circ@scaled@Rlen
            \pgfpointadd{\centertert}{\pgfpointpolar{-90}{.5\pgf@circ@scaled@Rlen}}
    }
}
{\ctikzvalof{bipoles/ooosource/height}}
{ooosource}
{\ctikzvalof{bipoles/ooosource/height}}
{\ctikzvalof{bipoles/ooosource/height}}
{
%     \pgf@circ@res@other = \ctikzvalof{bipoles/ooosource/vectorgroup} \pgf@circ@scaled@Rlen
    
%     % filling
%     left
    \pgf@circ@setlinewidth{bipoles}{\pgfstartlinewidth}
        \pgfpathcircle{\pgfpoint{\ctikzvalof{bipoles/ooosource/circleoffset}\pgf@circ@res@left}{0}}{\ctikzvalof{bipoles/ooosource/circlesize}\pgf@circ@res@left}
    \pgf@circ@maybefill 
    
    % up
    \pgfscope
        \pgfpointorigin
        \pgfpathcircle{\pgfpointpolar{60}{\ctikzvalof{bipoles/ooosource/circleoffset}\pgf@circ@res@right}}{\ctikzvalof{bipoles/ooosource/circlesize}\pgf@circ@res@right}
        \pgf@circ@maybefill
    \endpgfscope
%     down
    \pgfscope
        \pgfpointorigin
        \pgfpathcircle{\pgfpointpolar{-60}{\ctikzvalof{bipoles/ooosource/circleoffset}\pgf@circ@res@right}}{\ctikzvalof{bipoles/ooosource/circlesize}\pgf@circ@res@right}
        \pgf@circ@draworfill
    \endpgfscope

%     drawing
    % left
    \pgfpathcircle{\pgfpoint{\ctikzvalof{bipoles/ooosource/circleoffset}\pgf@circ@res@left}{0}}{\ctikzvalof{bipoles/ooosource/circlesize}\pgf@circ@res@left}
    \pgfusepath{draw}
    
    % up
    \pgfscope
        \pgfpointorigin
        \pgfpathcircle{\pgfpointpolar{60}{\ctikzvalof{bipoles/ooosource/circleoffset}\pgf@circ@res@right}}{\ctikzvalof{bipoles/ooosource/circlesize}\pgf@circ@res@right}
        \pgfusepath{draw}
    \endpgfscope
    
% % %     draw inner symbols

% %     primary winding
    \ifpgf@circ@prim@delta
        \pgfscope
            \pgftransformxshift{\ctikzvalof{bipoles/ooosource/circlesize}\pgf@circ@res@left}
            \pgf@circ@delta{\ctikzvalof{bipoles/ooosource/vectorgroupscale}}
        \endpgfscope
        
    \else\ifpgf@circ@prim@wye
        \pgfscope
            \pgftransformxshift{.6\pgf@circ@res@left}
            \pgf@circ@wye{\ctikzvalof{bipoles/ooosource/vectorgroupscale}}
        \endpgfscope
    
    \else\ifpgf@circ@prim@zig
        \pgfscope
            \pgftransformxshift{.6\pgf@circ@res@left}
            \pgf@circ@zig{\ctikzvalof{bipoles/ooosource/vectorgroupscale}}
        \endpgfscope
    \fi\fi\fi
    
% %     secondary winding
    \ifpgf@circ@sec@delta
        \pgfscope
            \pgfpointorigin
            \pgftransformshift{\pgfpointpolar{60}{\ctikzvalof{bipoles/ooosource/circlesize}\pgf@circ@res@right}}
            \pgf@circ@delta{\ctikzvalof{bipoles/ooosource/vectorgroupscale}}
        \endpgfscope
        
    \else\ifpgf@circ@sec@wye
        \pgfscope
            \pgftransformshift{\pgfpointpolar{60}{\ctikzvalof{bipoles/ooosource/circlesize}\pgf@circ@res@right}}
            \pgf@circ@wye{\ctikzvalof{bipoles/ooosource/vectorgroupscale}}
        \endpgfscope
    
    \else\ifpgf@circ@sec@zig
        \pgfscope
            \pgftransformshift{\pgfpointpolar{60}{\ctikzvalof{bipoles/ooosource/circlesize}\pgf@circ@res@right}}
            \pgf@circ@zig{\ctikzvalof{bipoles/ooosource/vectorgroupscale}}
        \endpgfscope
    \fi\fi\fi
    
% %     tertiary winding
    \ifpgf@circ@tert@delta
        \pgfscope
            \pgftransformshift{\pgfpointpolar{-60}{\ctikzvalof{bipoles/ooosource/circlesize}\pgf@circ@res@right}}
            \pgf@circ@delta{\ctikzvalof{bipoles/ooosource/vectorgroupscale}}
        \endpgfscope
        
    \else\ifpgf@circ@tert@wye
        \pgfscope
            \pgftransformshift{\pgfpointpolar{-60}{\ctikzvalof{bipoles/ooosource/circlesize}\pgf@circ@res@right}}
            \pgf@circ@wye{\ctikzvalof{bipoles/ooosource/vectorgroupscale}}
        \endpgfscope
    
    \else\ifpgf@circ@tert@zig
        \pgfscope
            \pgftransformshift{\pgfpointpolar{-60}{\ctikzvalof{bipoles/ooosource/circlesize}\pgf@circ@res@right}}
            \pgf@circ@zig{\ctikzvalof{bipoles/ooosource/vectorgroupscale}}
        \endpgfscope
    \fi\fi\fi
}

%% Independent current source - American
\pgfcircdeclarebipolescaled{sources}
{}
{\ctikzvalof{bipoles/isourceam/height}}
{isourceAM}
{\ctikzvalof{bipoles/isourceam/height}}
{\ctikzvalof{bipoles/isourceam/width}}
{
    \pgfpointorigin
    \pgf@circ@setlinewidth{bipoles}{\pgfstartlinewidth}
    \pgfpathellipse{\pgfpointorigin}{\pgfpoint{0}{\pgf@circ@res@up}}{\pgfpoint{\pgf@circ@res@left}{0}}
    \pgf@circ@draworfill

    \pgfpathmoveto{\pgfpoint{.7\pgf@circ@res@left}{\pgf@circ@res@zero}}
    \pgfpathlineto{\pgfpoint{.7\pgf@circ@res@right}{\pgf@circ@res@zero}}
    \pgfusepath{draw}
    \pgfscope
        \pgftransformshift{\pgfpoint{.5\pgf@circ@res@right}{\pgf@circ@res@zero}}
        \pgfnode{currarrow}{center}{}{}{\pgfusepath{stroke}}
    \endpgfscope
}

%% Independent sinusoidal current source
\pgfcircdeclarebipolescaled{sources}
{}
{\ctikzvalof{bipoles/isource/height}}
{isourcesin}
{\ctikzvalof{bipoles/isource/height}}
{\ctikzvalof{bipoles/isource/width}}
{
    \pgfpointorigin
    \pgf@circ@setlinewidth{bipoles}{\pgfstartlinewidth}
    \pgfpathellipse{\pgfpointorigin}{\pgfpoint{0}{\pgf@circ@res@up}}{\pgfpoint{\pgf@circ@res@left}{0}}
    \pgf@circ@draworfill

    \pgf@circ@res@up = .5\pgf@circ@res@up
    \pgfscope
        \pgftransformrotate{90}
        \pgfpathmoveto{\pgfpoint{-\pgf@circ@res@up}{0cm}}
        \pgfpathsine{\pgfpoint{.5\pgf@circ@res@up}{.5\pgf@circ@res@up}}
        \pgfpathcosine{\pgfpoint{.5\pgf@circ@res@up}{-.5\pgf@circ@res@up}}
        \pgfpathsine{\pgfpoint{.5\pgf@circ@res@up}{-.5\pgf@circ@res@up}}
        \pgfpathcosine{\pgfpoint{.5\pgf@circ@res@up}{.5\pgf@circ@res@up}}
        \pgfusepath{draw}
    \endpgfscope
}

%% Empty controlled source
\pgfcircdeclarebipolescaled{csources}
{}
{\ctikzvalof{bipoles/ecsource/height}}
{ecsource}
{\ctikzvalof{bipoles/ecsource/height}}
{\ctikzvalof{bipoles/ecsource/width}}
{

    \pgf@circ@setlinewidth{bipoles}{\pgfstartlinewidth}

    \pgfpathmoveto{\pgfpoint{\pgf@circ@res@left}{\pgf@circ@res@zero}}
    \pgfpathlineto{\pgfpoint{\pgf@circ@res@zero}{\pgf@circ@res@up}}
    \pgfpathlineto{\pgfpoint{\pgf@circ@res@right}{\pgf@circ@res@zero}}
    \pgfpathlineto{\pgfpoint{\pgf@circ@res@zero}{\pgf@circ@res@down}}
    \pgfpathclose
    \pgf@circ@draworfill
}

%% Controlled voltage source
\pgfcircdeclarebipolescaled{csources}
{}
{\ctikzvalof{bipoles/cvsource/height}}
{cvsource}
{\ctikzvalof{bipoles/cvsource/height}}
{\ctikzvalof{bipoles/cvsource/width}}
{

    \pgf@circ@setlinewidth{bipoles}{\pgfstartlinewidth}

    \pgfpathmoveto{\pgfpoint{\pgf@circ@res@left}{\pgf@circ@res@zero}}
    \pgfpathlineto{\pgfpoint{\pgf@circ@res@zero}{\pgf@circ@res@up}}
    \pgfpathlineto{\pgfpoint{\pgf@circ@res@right}{\pgf@circ@res@zero}}
    \pgfpathlineto{\pgfpoint{\pgf@circ@res@zero}{\pgf@circ@res@down}}
    \pgfpathclose
    \pgf@circ@draworfill

    \pgfpathmoveto{\pgfpoint{\pgf@circ@res@left}{\pgf@circ@res@zero}}
    \pgfpathlineto{\pgfpoint{\pgf@circ@res@right}{\pgf@circ@res@zero}}
    \pgfusepath{draw}
}

%% To change the internal symbols of the controlled voltage source american style
\ctikzset{bipoles/cvsourceam/inner plus/.initial={$+$}}
\ctikzset{bipoles/cvsourceam/inner minus/.initial={$-$}}
%% Controlled voltage source - American
\pgfcircdeclarebipolescaled{csources}
{}
{\ctikzvalof{bipoles/cvsourceam/height}}
{cvsourceAM}
{\ctikzvalof{bipoles/cvsourceam/height}}
{\ctikzvalof{bipoles/cvsourceam/width}}
{
    \pgf@circ@setlinewidth{bipoles}{\pgfstartlinewidth}
    \pgfpathmoveto{\pgfpoint{\pgf@circ@res@left}{\pgf@circ@res@zero}}
    \pgfpathlineto{\pgfpoint{\pgf@circ@res@zero}{\pgf@circ@res@up}}
    \pgfpathlineto{\pgfpoint{\pgf@circ@res@right}{\pgf@circ@res@zero}}
    \pgfpathlineto{\pgfpoint{\pgf@circ@res@zero}{\pgf@circ@res@down}}
    \pgfpathclose
    \pgf@circ@draworfill

    \pgfsetcolor{\ctikzvalof{color}}
    \ifpgf@circ@oldvoltagedirection
        \pgftext[bottom,rotate=90,y=\ctikzvalof{bipoles/cvsourceam/margin}\pgf@circ@res@left]{\ctikzvalof{bipoles/cvsourceam/inner plus}}
        \pgftext[top,rotate=90,y=\ctikzvalof{bipoles/cvsourceam/margin}\pgf@circ@res@right]{\ctikzvalof{bipoles/cvsourceam/inner minus}}
    \else
        \pgftext[bottom,rotate=90,y=\ctikzvalof{bipoles/cvsourceam/margin}\pgf@circ@res@left]{\ctikzvalof{bipoles/cvsourceam/inner minus}}
        \pgftext[top,rotate=90,y=\ctikzvalof{bipoles/cvsourceam/margin}\pgf@circ@res@right]{\ctikzvalof{bipoles/cvsourceam/inner plus}}
    \fi
}

%% Controlled sinusoidal voltage source
\pgfcircdeclarebipolescaled{csources}
{}
{\ctikzvalof{bipoles/cvsourcesin/height}}
{cvsourcesin}
{\ctikzvalof{bipoles/cvsourcesin/height}}
{\ctikzvalof{bipoles/cvsourcesin/width}}
{

    \pgf@circ@setlinewidth{bipoles}{\pgfstartlinewidth}

    \pgfscope
        \pgfpathmoveto{\pgfpoint{\pgf@circ@res@left}{\pgf@circ@res@zero}}
        \pgfpathlineto{\pgfpoint{\pgf@circ@res@zero}{\pgf@circ@res@up}}
        \pgfpathlineto{\pgfpoint{\pgf@circ@res@right}{\pgf@circ@res@zero}}
        \pgfpathlineto{\pgfpoint{\pgf@circ@res@zero}{\pgf@circ@res@down}}
        \pgfpathclose
        \pgf@circ@draworfill
    \endpgfscope

    \pgf@circ@res@up = .5\pgf@circ@res@up
    \pgfscope
        \pgftransformrotate{90}
        \pgfpathmoveto{\pgfpoint{-\pgf@circ@res@up}{0cm}}
        \pgfpathsine{\pgfpoint{.5\pgf@circ@res@up}{.5\pgf@circ@res@up}}
        \pgfpathcosine{\pgfpoint{.5\pgf@circ@res@up}{-.5\pgf@circ@res@up}}
        \pgfpathsine{\pgfpoint{.5\pgf@circ@res@up}{-.5\pgf@circ@res@up}}
        \pgfpathcosine{\pgfpoint{.5\pgf@circ@res@up}{.5\pgf@circ@res@up}}
        \pgfusepath{draw}
    \endpgfscope
}

%% Controlled sinusoidal current source
\pgfcircdeclarebipolescaled{csources}
{}
{\ctikzvalof{bipoles/cvsourcesin/height}}
{cisourcesin}
{\ctikzvalof{bipoles/cvsourcesin/height}}
{\ctikzvalof{bipoles/cvsourcesin/width}}
{

    \pgf@circ@setlinewidth{bipoles}{\pgfstartlinewidth}

    \pgfscope
        \pgfpathmoveto{\pgfpoint{\pgf@circ@res@left}{\pgf@circ@res@zero}}
        \pgfpathlineto{\pgfpoint{\pgf@circ@res@zero}{\pgf@circ@res@up}}
        \pgfpathlineto{\pgfpoint{\pgf@circ@res@right}{\pgf@circ@res@zero}}
        \pgfpathlineto{\pgfpoint{\pgf@circ@res@zero}{\pgf@circ@res@down}}
        \pgfpathclose
        \pgf@circ@draworfill
    \endpgfscope

    \pgf@circ@res@up = .5\pgf@circ@res@up
    \pgfscope
        \pgftransformrotate{90}
        \pgfpathmoveto{\pgfpoint{-\pgf@circ@res@up}{0cm}}
        \pgfpathsine{\pgfpoint{.5\pgf@circ@res@up}{.5\pgf@circ@res@up}}
        \pgfpathcosine{\pgfpoint{.5\pgf@circ@res@up}{-.5\pgf@circ@res@up}}
        \pgfpathsine{\pgfpoint{.5\pgf@circ@res@up}{-.5\pgf@circ@res@up}}
        \pgfpathcosine{\pgfpoint{.5\pgf@circ@res@up}{.5\pgf@circ@res@up}}
        \pgfusepath{draw}
    \endpgfscope
}

%% Controlled current source
\pgfcircdeclarebipolescaled{csources}
{}
{\ctikzvalof{bipoles/cisource/height}}
{cisource}
{\ctikzvalof{bipoles/cisource/height}}
{\ctikzvalof{bipoles/cisource/width}}
{

    \pgf@circ@setlinewidth{bipoles}{\pgfstartlinewidth}

    \pgfpathmoveto{\pgfpoint{\pgf@circ@res@left}{\pgf@circ@res@zero}}
    \pgfpathlineto{\pgfpoint{\pgf@circ@res@zero}{\pgf@circ@res@up}}
    \pgfpathlineto{\pgfpoint{\pgf@circ@res@right}{\pgf@circ@res@zero}}
    \pgfpathlineto{\pgfpoint{\pgf@circ@res@zero}{\pgf@circ@res@down}}
    \pgfpathclose
    \pgf@circ@draworfill

    \pgfpathmoveto{\pgfpoint{\pgf@circ@res@zero}{\pgf@circ@res@up}}
    \pgfpathlineto{\pgfpoint{\pgf@circ@res@zero}{\pgf@circ@res@down}}
    \pgfusepath{draw}
}

%% Controlled current source - American
\pgfcircdeclarebipolescaled{csources}
{}
{\ctikzvalof{bipoles/cisourceam/height}}
{cisourceAM}
{\ctikzvalof{bipoles/cisourceam/height}}
{\ctikzvalof{bipoles/cisourceam/width}}
{
    \pgf@circ@setlinewidth{bipoles}{\pgfstartlinewidth}

    \pgfpathmoveto{\pgfpoint{\pgf@circ@res@left}{\pgf@circ@res@zero}}
    \pgfpathlineto{\pgfpoint{\pgf@circ@res@zero}{\pgf@circ@res@up}}
    \pgfpathlineto{\pgfpoint{\pgf@circ@res@right}{\pgf@circ@res@zero}}
    \pgfpathlineto{\pgfpoint{\pgf@circ@res@zero}{\pgf@circ@res@down}}
    \pgfpathclose
    \pgf@circ@draworfill
    \pgfpathmoveto{\pgfpoint{.7\pgf@circ@res@left}{\pgf@circ@res@zero}}
    \pgfpathlineto{\pgfpoint{.7\pgf@circ@res@right}{\pgf@circ@res@zero}}
    \pgfusepath{draw}
    \pgfscope
        \pgftransformshift{\pgfpoint{.5\pgf@circ@res@right}{\pgf@circ@res@zero}}
        \pgfnode{currarrow}{center}{}{}{\pgfusepath{stroke}}
    \endpgfscope
}

%% Cute Independent voltage source
\pgfcircdeclarebipolescaled{sources}
{}
{\ctikzvalof{bipoles/vsource/height}}
{vsourceC}
{\ctikzvalof{bipoles/vsource/height}}
{\ctikzvalof{bipoles/vsource/width}}
{
    \pgf@circ@setlinewidth{bipoles}{\pgfstartlinewidth}
    \pgfpathellipse{\pgfpointorigin}{\pgfpoint{0}{\pgf@circ@res@up}}{\pgfpoint{\pgf@circ@res@left}{0}}
    \pgf@circ@draworfill
    \pgfmathsetmacro{\@@thicker}{3*\ctikzvalof{bipoles/thickness}}
    \pgfsetlinewidth{\@@thicker\pgfstartlinewidth}
    \pgfpathmoveto{\pgfpoint{.6\pgf@circ@res@left}{\pgf@circ@res@zero}}
    \pgfpathlineto{\pgfpoint{.6\pgf@circ@res@right}{\pgf@circ@res@zero}}
    \pgfsetroundcap\pgfusepath{draw}
}

%% Cute Independent current source
\pgfcircdeclarebipolescaled{sources}
{}
{\ctikzvalof{bipoles/isource/height}}
{isourceC}
{\ctikzvalof{bipoles/isource/height}}
{\ctikzvalof{bipoles/isource/width}}
{
    \pgfpointorigin
    \pgf@circ@setlinewidth{bipoles}{\pgfstartlinewidth}
    \pgfpathellipse{\pgfpointorigin}{\pgfpoint{0}{\pgf@circ@res@up}}{\pgfpoint{\pgf@circ@res@left}{0}}
    \pgf@circ@draworfill
    \pgfmathsetmacro{\@@thicker}{3*\ctikzvalof{bipoles/thickness}}
    \pgfsetlinewidth{\@@thicker\pgfstartlinewidth}
    \pgfpathmoveto{\pgfpoint{\pgf@circ@res@step}{.6\pgf@circ@res@up}}
    \pgfpathlineto{\pgfpoint{\pgf@circ@res@step}{.6\pgf@circ@res@down}}
    \pgfsetroundcap\pgfusepath{draw}
}

%% Cute Controlled voltage source
\pgfcircdeclarebipolescaled{csources}
{}
{\ctikzvalof{bipoles/cvsource/height}}
{cvsourceC}
{\ctikzvalof{bipoles/cvsource/height}}
{\ctikzvalof{bipoles/cvsource/width}}
{
    \pgf@circ@setlinewidth{bipoles}{\pgfstartlinewidth}
    \pgfpathmoveto{\pgfpoint{\pgf@circ@res@left}{\pgf@circ@res@zero}}
    \pgfpathlineto{\pgfpoint{\pgf@circ@res@zero}{\pgf@circ@res@up}}
    \pgfpathlineto{\pgfpoint{\pgf@circ@res@right}{\pgf@circ@res@zero}}
    \pgfpathlineto{\pgfpoint{\pgf@circ@res@zero}{\pgf@circ@res@down}}
    \pgfpathlineto{\pgfpoint{\pgf@circ@res@left}{\pgf@circ@res@zero}}
    \pgf@circ@draworfill
    \pgfmathsetmacro{\@@thicker}{3*\ctikzvalof{bipoles/thickness}}
    \pgfsetlinewidth{\@@thicker\pgfstartlinewidth}
    \pgfpathmoveto{\pgfpoint{.6\pgf@circ@res@left}{\pgf@circ@res@zero}}
    \pgfpathlineto{\pgfpoint{.6\pgf@circ@res@right}{\pgf@circ@res@zero}}
    \pgfsetroundcap\pgfusepath{draw}
}

%% Cute Controlled current source
\pgfcircdeclarebipolescaled{csources}
{}
{\ctikzvalof{bipoles/cisource/height}}
{cisourceC}
{\ctikzvalof{bipoles/cisource/height}}
{\ctikzvalof{bipoles/cisource/width}}
{
    \pgf@circ@setlinewidth{bipoles}{\pgfstartlinewidth}
    \pgfpathmoveto{\pgfpoint{\pgf@circ@res@left}{\pgf@circ@res@zero}}
    \pgfpathlineto{\pgfpoint{\pgf@circ@res@zero}{\pgf@circ@res@up}}
    \pgfpathlineto{\pgfpoint{\pgf@circ@res@right}{\pgf@circ@res@zero}}
    \pgfpathlineto{\pgfpoint{\pgf@circ@res@zero}{\pgf@circ@res@down}}
    \pgfpathlineto{\pgfpoint{\pgf@circ@res@left}{\pgf@circ@res@zero}}
    \pgf@circ@draworfill
    \pgfmathsetmacro{\@@thicker}{3*\ctikzvalof{bipoles/thickness}}
    \pgfsetlinewidth{\@@thicker\pgfstartlinewidth}
    \pgfpathmoveto{\pgfpoint{\pgf@circ@res@zero}{.6\pgf@circ@res@up}}
    \pgfpathlineto{\pgfpoint{\pgf@circ@res@zero}{.6\pgf@circ@res@down}}
    \pgfsetroundcap\pgfusepath{draw}
}

%%  Noise voltage source
\pgfcircdeclarebipolescaled{sources}
{}
{\ctikzvalof{bipoles/vsource/height}}
{vsourceN}
{\ctikzvalof{bipoles/vsource/height}}
{\ctikzvalof{bipoles/vsource/width}}
{
    \pgfscope
        \pgf@circ@setlinewidth{bipoles}{\pgfstartlinewidth}
        \pgfpathellipse{\pgfpointorigin}{\pgfpoint{0}{\pgf@circ@res@up}}{\pgfpoint{\pgf@circ@res@left}{0}}
        %
        \pgf@circ@scaled@Rlen=\scaledRlen
        \pgf@circ@res@step=0.125\pgf@circ@scaled@Rlen\relax
        \edef\pgf@noise@temp{dashed}
        \edef\pgf@noise@fill{\ctikzvalof{bipoles/noise sources/fillcolor}}
        \ifx\pgf@noise@temp\pgf@noise@fill
            % fillable in this case
            \pgf@circ@draworfillandclip
            \pgfmathsetmacro{\@@thinner}{.5*\ctikzvalof{bipoles/thickness}}
            \pgfsetlinewidth{\@@thinner\pgfstartlinewidth}
            \pgfpathmoveto{\pgfpoint{\pgf@circ@res@left}{\pgf@circ@res@up}}
            \pgfpathlineto{\pgfpoint{\pgf@circ@res@right}{\pgf@circ@res@down}}
            \advance\pgf@circ@res@up by \pgf@circ@res@step\advance\pgf@circ@res@down by \pgf@circ@res@step
            \pgfpathmoveto{\pgfpoint{\pgf@circ@res@left}{\pgf@circ@res@up}}
            \pgfpathlineto{\pgfpoint{\pgf@circ@res@right}{\pgf@circ@res@down}}
            \advance\pgf@circ@res@up by \pgf@circ@res@step\advance\pgf@circ@res@down by \pgf@circ@res@step
            \pgfpathmoveto{\pgfpoint{\pgf@circ@res@left}{\pgf@circ@res@up}}
            \pgfpathlineto{\pgfpoint{\pgf@circ@res@right}{\pgf@circ@res@down}}
            \advance\pgf@circ@res@up by \pgf@circ@res@step\advance\pgf@circ@res@down by \pgf@circ@res@step
            \pgfpathmoveto{\pgfpoint{\pgf@circ@res@left}{\pgf@circ@res@up}}
            \pgfpathlineto{\pgfpoint{\pgf@circ@res@right}{\pgf@circ@res@down}}
            %
            \advance\pgf@circ@res@up by -4\pgf@circ@res@step \advance\pgf@circ@res@down by -4\pgf@circ@res@step
            \pgfpathmoveto{\pgfpoint{\pgf@circ@res@left}{\pgf@circ@res@up}}
            \pgfpathlineto{\pgfpoint{\pgf@circ@res@right}{\pgf@circ@res@down}}
            \advance\pgf@circ@res@up by -\pgf@circ@res@step\advance\pgf@circ@res@down by -\pgf@circ@res@step
            \pgfpathmoveto{\pgfpoint{\pgf@circ@res@left}{\pgf@circ@res@up}}
            \pgfpathlineto{\pgfpoint{\pgf@circ@res@right}{\pgf@circ@res@down}}
            \advance\pgf@circ@res@up by -\pgf@circ@res@step\advance\pgf@circ@res@down by -\pgf@circ@res@step
            \pgfpathmoveto{\pgfpoint{\pgf@circ@res@left}{\pgf@circ@res@up}}
            \pgfpathlineto{\pgfpoint{\pgf@circ@res@right}{\pgf@circ@res@down}}
            \pgfusepath{draw}
        \else
            \pgfsetfillcolor{\pgf@noise@fill}
            \pgfusepath{draw,fill}
        \fi
    \endpgfscope
    \pgfmathsetmacro{\@@thicker}{3*\ctikzvalof{bipoles/thickness}}
    \pgfsetlinewidth{\@@thicker\pgfstartlinewidth}
    \pgfpathmoveto{\pgfpoint{.6\pgf@circ@res@left}{\pgf@circ@res@zero}}
    \pgfpathlineto{\pgfpoint{.6\pgf@circ@res@right}{\pgf@circ@res@zero}}
    \pgfsetroundcap\pgfusepath{draw}
}
%% Noise current source

\pgfcircdeclarebipolescaled{sources}
{}
{\ctikzvalof{bipoles/isource/height}}
{isourceN}
{\ctikzvalof{bipoles/isource/height}}
{\ctikzvalof{bipoles/isource/width}}
{
    \pgfpointorigin
    \pgfscope
        \pgf@circ@setlinewidth{bipoles}{\pgfstartlinewidth}
        \pgfpathellipse{\pgfpointorigin}{\pgfpoint{0}{\pgf@circ@res@up}}{\pgfpoint{\pgf@circ@res@left}{0}}
        %
        \pgf@circ@scaled@Rlen=\scaledRlen
        \pgf@circ@res@step=0.125\pgf@circ@scaled@Rlen\relax
        \edef\pgf@noise@temp{dashed}
        \edef\pgf@noise@fill{\ctikzvalof{bipoles/noise sources/fillcolor}}
        \ifx\pgf@noise@temp\pgf@noise@fill
            % fillable in this case
            \pgf@circ@draworfillandclip
            \pgfmathsetmacro{\@@thinner}{.5*\ctikzvalof{bipoles/thickness}}
            \pgfsetlinewidth{\@@thinner\pgfstartlinewidth}
            \pgfpathmoveto{\pgfpoint{\pgf@circ@res@left}{\pgf@circ@res@up}}
            \pgfpathlineto{\pgfpoint{\pgf@circ@res@right}{\pgf@circ@res@down}}
            \advance\pgf@circ@res@up by \pgf@circ@res@step\advance\pgf@circ@res@down by \pgf@circ@res@step
            \pgfpathmoveto{\pgfpoint{\pgf@circ@res@left}{\pgf@circ@res@up}}
            \pgfpathlineto{\pgfpoint{\pgf@circ@res@right}{\pgf@circ@res@down}}
            \advance\pgf@circ@res@up by \pgf@circ@res@step\advance\pgf@circ@res@down by \pgf@circ@res@step
            \pgfpathmoveto{\pgfpoint{\pgf@circ@res@left}{\pgf@circ@res@up}}
            \pgfpathlineto{\pgfpoint{\pgf@circ@res@right}{\pgf@circ@res@down}}
            \advance\pgf@circ@res@up by \pgf@circ@res@step\advance\pgf@circ@res@down by \pgf@circ@res@step
            \pgfpathmoveto{\pgfpoint{\pgf@circ@res@left}{\pgf@circ@res@up}}
            \pgfpathlineto{\pgfpoint{\pgf@circ@res@right}{\pgf@circ@res@down}}
            %
            \advance\pgf@circ@res@up by -4\pgf@circ@res@step \advance\pgf@circ@res@down by -4\pgf@circ@res@step
            \pgfpathmoveto{\pgfpoint{\pgf@circ@res@left}{\pgf@circ@res@up}}
            \pgfpathlineto{\pgfpoint{\pgf@circ@res@right}{\pgf@circ@res@down}}
            \advance\pgf@circ@res@up by -\pgf@circ@res@step\advance\pgf@circ@res@down by -\pgf@circ@res@step
            \pgfpathmoveto{\pgfpoint{\pgf@circ@res@left}{\pgf@circ@res@up}}
            \pgfpathlineto{\pgfpoint{\pgf@circ@res@right}{\pgf@circ@res@down}}
            \advance\pgf@circ@res@up by -\pgf@circ@res@step\advance\pgf@circ@res@down by -\pgf@circ@res@step
            \pgfpathmoveto{\pgfpoint{\pgf@circ@res@left}{\pgf@circ@res@up}}
            \pgfpathlineto{\pgfpoint{\pgf@circ@res@right}{\pgf@circ@res@down}}
            \pgfusepath{draw}
        \else
            \pgfsetfillcolor{\pgf@noise@fill}
            \pgfusepath{draw,fill}
        \fi
    \endpgfscope
    \pgfmathsetmacro{\@@thicker}{3*\ctikzvalof{bipoles/thickness}}
    \pgfsetlinewidth{\@@thicker\pgfstartlinewidth}
    \pgfpathmoveto{\pgfpoint{\pgf@circ@res@step}{.6\pgf@circ@res@up}}
    \pgfpathlineto{\pgfpoint{\pgf@circ@res@step}{.6\pgf@circ@res@down}}
    \pgfsetroundcap\pgfusepath{draw}
}

%%%%%%%%%%%%%%
%% Diodes
%%%%%%%%%%%%%%

%% Black generic diode
\pgfcircdeclarebipolescaled{diodes}
{}
{\ctikzvalof{bipoles/diode/height}}
{fulldiode}
{\ctikzvalof{bipoles/diode/height}}
{\ctikzvalof{bipoles/diode/width}}
{

    \pgf@circ@setlinewidth{bipoles}{\pgfstartlinewidth}

    \pgfscope
        \pgftransformxshift{\pgf@circ@res@left}
        \pgfpathmoveto{\pgfpoint{\pgf@circ@res@right-\pgf@circ@res@left}{0pt}}
        \pgfpathlineto{\pgfpoint{0pt}{\pgf@circ@res@up}}
        \pgfpathlineto{\pgfpoint{0pt}{\pgf@circ@res@down}}
        \pgfpathlineto{\pgfpoint{\pgf@circ@res@right-\pgf@circ@res@left}{0pt}}
        \pgfusepath{draw,fill}
    \endpgfscope
    \pgfpathmoveto{\pgfpoint{\pgf@circ@res@right}{\pgf@circ@res@down}}
    \pgfpathlineto{\pgfpoint{\pgf@circ@res@right}{\pgf@circ@res@up}}
    \pgfusepath{draw}
}

%% Black Zener diode
\pgfcircdeclarebipolescaled{diodes}
{}
{\ctikzvalof{bipoles/diode/height}}
{fullzdiode}
{\ctikzvalof{bipoles/diode/height}}
{\ctikzvalof{bipoles/diode/width}}
{

    \pgf@circ@setlinewidth{bipoles}{\pgfstartlinewidth}

    \pgfscope
        \pgftransformxshift{\pgf@circ@res@left}
        \pgfpathmoveto{\pgfpoint{\pgf@circ@res@right-\pgf@circ@res@left}{0pt}}
        \pgfpathlineto{\pgfpoint{0pt}{\pgf@circ@res@up}}
        \pgfpathlineto{\pgfpoint{0pt}{\pgf@circ@res@down}}
        \pgfpathlineto{\pgfpoint{\pgf@circ@res@right-\pgf@circ@res@left}{0pt}}
        \pgfusepath{draw,fill}
        \pgfpathmoveto{\pgfpoint{\pgf@circ@res@right-\pgf@circ@res@left}{\pgf@circ@res@down}}
        \pgfpathlineto{\pgfpoint{\pgf@circ@res@right-\pgf@circ@res@left}{\pgf@circ@res@up}}
        \pgfpathlineto{\pgfpoint{\pgf@circ@res@right-.6\pgf@circ@res@left}{\pgf@circ@res@up}}
        \pgfusepath{draw}
    \endpgfscope
}

%% Black alternative zigzag Zener diode
\pgfcircdeclarebipolescaled{diodes}
{}
{\ctikzvalof{bipoles/diode/height}}
{fullzzdiode}
{\ctikzvalof{bipoles/diode/height}}
{\ctikzvalof{bipoles/diode/width}}
{

    \pgf@circ@setlinewidth{bipoles}{\pgfstartlinewidth}

    \pgfscope
        \pgftransformxshift{\pgf@circ@res@left}
        \pgfpathmoveto{\pgfpoint{\pgf@circ@res@right-\pgf@circ@res@left}{0pt}}
        \pgfpathlineto{\pgfpoint{0pt}{\pgf@circ@res@up}}
        \pgfpathlineto{\pgfpoint{0pt}{\pgf@circ@res@down}}
        \pgfpathlineto{\pgfpoint{\pgf@circ@res@right-\pgf@circ@res@left}{0pt}}
        \pgfusepath{draw,fill}
        \pgfpathmoveto{\pgfpoint{\pgf@circ@res@right-1.8\pgf@circ@res@left}{\pgf@circ@res@down-0.5\pgf@circ@res@up}}
        \pgfpathlineto{\pgfpoint{\pgf@circ@res@right-\pgf@circ@res@left}{\pgf@circ@res@down}}
        \pgfpathlineto{\pgfpoint{\pgf@circ@res@right-\pgf@circ@res@left}{\pgf@circ@res@up}}
        \pgfpathlineto{\pgfpoint{\pgf@circ@res@right-0.2\pgf@circ@res@left}{\pgf@circ@res@up-0.5\pgf@circ@res@down}}
        \pgfusepath{draw}
    \endpgfscope
}

%% Black Schottky diode
\pgfcircdeclarebipolescaled{diodes}
{}
{\ctikzvalof{bipoles/diode/height}}
{fullsdiode}
{\ctikzvalof{bipoles/diode/height}}
{\ctikzvalof{bipoles/diode/width}}
{

    \pgf@circ@setlinewidth{bipoles}{\pgfstartlinewidth}
    \pgfscope
        \pgftransformxshift{\pgf@circ@res@left}
        \pgfpathmoveto{\pgfpoint{\pgf@circ@res@right-\pgf@circ@res@left}{0pt}}
        \pgfpathlineto{\pgfpoint{0pt}{\pgf@circ@res@up}}
        \pgfpathlineto{\pgfpoint{0pt}{\pgf@circ@res@down}}
        \pgfpathlineto{\pgfpoint{\pgf@circ@res@right-\pgf@circ@res@left}{0pt}}
        \pgfusepath{draw,fill}

        \pgfpathmoveto{\pgfpoint{\pgf@circ@res@right-.6\pgf@circ@res@left}{.6\pgf@circ@res@down}}
        \pgfpathlineto{\pgfpoint{\pgf@circ@res@right-.6\pgf@circ@res@left}{\pgf@circ@res@down}}
        \pgfpathlineto{\pgfpoint{\pgf@circ@res@right-\pgf@circ@res@left}{\pgf@circ@res@down}}
        \pgfpathlineto{\pgfpoint{\pgf@circ@res@right-\pgf@circ@res@left}{\pgf@circ@res@down}}
        \pgfpathlineto{\pgfpoint{\pgf@circ@res@right-\pgf@circ@res@left}{\pgf@circ@res@up}}
        \pgfpathlineto{\pgfpoint{\pgf@circ@res@right-1.4\pgf@circ@res@left}{\pgf@circ@res@up}}
        \pgfpathlineto{\pgfpoint{\pgf@circ@res@right-1.4\pgf@circ@res@left}{.6\pgf@circ@res@up}}
        \pgfusepath{draw}
    \endpgfscope

}

%% Black tunnel diode
\pgfcircdeclarebipolescaled{diodes}
{}
{\ctikzvalof{bipoles/diode/height}}
{fulltdiode}
{\ctikzvalof{bipoles/diode/height}}
{\ctikzvalof{bipoles/diode/width}}
{

    \pgf@circ@setlinewidth{bipoles}{\pgfstartlinewidth}
    \pgfscope
        \pgftransformxshift{\pgf@circ@res@left}
        \pgfpathmoveto{\pgfpoint{\pgf@circ@res@right-\pgf@circ@res@left}{0pt}}
        \pgfpathlineto{\pgfpoint{0pt}{\pgf@circ@res@up}}
        \pgfpathlineto{\pgfpoint{0pt}{\pgf@circ@res@down}}
        \pgfpathlineto{\pgfpoint{\pgf@circ@res@right-\pgf@circ@res@left}{0pt}}
        \pgfusepath{draw,fill}

        \pgfpathmoveto{\pgfpoint{\pgf@circ@res@right-.6\pgf@circ@res@left}{\pgf@circ@res@down}}
        \pgfpathlineto{\pgfpoint{\pgf@circ@res@right-\pgf@circ@res@left}{\pgf@circ@res@down}}
        \pgfpathlineto{\pgfpoint{\pgf@circ@res@right-\pgf@circ@res@left}{\pgf@circ@res@up}}
        \pgfpathlineto{\pgfpoint{\pgf@circ@res@right-.6\pgf@circ@res@left}{\pgf@circ@res@up}}
        \pgfusepath{draw}
    \endpgfscope
}

%
% draw LED arrows
%
\def\pgf@circ@draw@ledarrows{%
    \pgfsetlinewidth{\pgfstartlinewidth}
    \pgfsetarrowsend{latexslim}
    \ifpgf@led@fliparrows
        \pgfpathmoveto{\pgfpoint{0pt}{0.8\pgf@circ@res@up}}
        \pgfpathlineto{\pgfpoint{-0.6\pgf@circ@res@right}{1.8\pgf@circ@res@up}}
        \pgfusepath{draw}
        \pgfpathmoveto{\pgfpoint{0.6\pgf@circ@res@right}{0.6\pgf@circ@res@up}}
        \pgfpathlineto{\pgfpoint{0pt}{1.6\pgf@circ@res@up}}
        \pgfusepath{draw}
    \else
        \pgfpathmoveto{\pgfpoint{-0.4\pgf@circ@res@right}{\pgf@circ@res@up}}
        \pgfpathlineto{\pgfpoint{0.6\pgf@circ@res@right}{2\pgf@circ@res@up}}
        \pgfusepath{draw}
        \pgfpathmoveto{\pgfpoint{0.2\pgf@circ@res@right}{0.8\pgf@circ@res@up}}
        \pgfpathlineto{\pgfpoint{1.2\pgf@circ@res@right}{1.8\pgf@circ@res@up}}
        \pgfusepath{draw}
    \fi
}
%
% ---and photodiode arrows
%
\def\pgf@circ@draw@pdarrows{%
    \pgfsetlinewidth{\pgfstartlinewidth}
    \pgfsetarrowsstart{latexslim}
    \ifpgf@pd@fliparrows
        \pgfpathmoveto{\pgfpoint{0pt}{0.8\pgf@circ@res@up}}
        \pgfpathlineto{\pgfpoint{-0.6\pgf@circ@res@right}{1.8\pgf@circ@res@up}}
        \pgfusepath{draw}
        \pgfpathmoveto{\pgfpoint{0.6\pgf@circ@res@right}{0.6\pgf@circ@res@up}}
        \pgfpathlineto{\pgfpoint{0pt}{1.6\pgf@circ@res@up}}
        \pgfusepath{draw}
    \else
        \pgfpathmoveto{\pgfpoint{-0.4\pgf@circ@res@right}{\pgf@circ@res@up}}
        \pgfpathlineto{\pgfpoint{0.6\pgf@circ@res@right}{2\pgf@circ@res@up}}
        \pgfusepath{draw}
        \pgfpathmoveto{\pgfpoint{0.2\pgf@circ@res@right}{0.8\pgf@circ@res@up}}
        \pgfpathlineto{\pgfpoint{1.2\pgf@circ@res@right}{1.8\pgf@circ@res@up}}
        \pgfusepath{draw}
    \fi
}
%% Black light emitting diode
\pgfcircdeclarebipolescaled{diodes}
{}
{\ctikzvalof{bipoles/diode/height}}
{fulllediode}
{\ctikzvalof{bipoles/diode/height}}
{\ctikzvalof{bipoles/diode/width}}
{

    \pgf@circ@setlinewidth{bipoles}{\pgfstartlinewidth}
    \pgfscope
        \pgftransformxshift{\pgf@circ@res@left}
        \pgfpathmoveto{\pgfpoint{\pgf@circ@res@right-\pgf@circ@res@left}{0pt}}
        \pgfpathlineto{\pgfpoint{0pt}{\pgf@circ@res@up}}
        \pgfpathlineto{\pgfpoint{0pt}{\pgf@circ@res@down}}
        \pgfpathlineto{\pgfpoint{\pgf@circ@res@right-\pgf@circ@res@left}{0pt}}
        \pgfusepath{draw,fill}
    \endpgfscope
    \pgfpathmoveto{\pgfpoint{\pgf@circ@res@right}{\pgf@circ@res@down}}
    \pgfpathlineto{\pgfpoint{\pgf@circ@res@right}{\pgf@circ@res@up}}
    \pgfusepath{draw}

    \pgf@circ@draw@ledarrows
}

%% Black photodiode
\pgfcircdeclarebipolescaled{diodes}
{}
{\ctikzvalof{bipoles/diode/height}}
{fullpdiode}
{\ctikzvalof{bipoles/diode/height}}
{\ctikzvalof{bipoles/diode/width}}
{

    \pgf@circ@setlinewidth{bipoles}{\pgfstartlinewidth}
    \pgfscope
        \pgftransformxshift{\pgf@circ@res@left}
        \pgfpathmoveto{\pgfpoint{\pgf@circ@res@right-\pgf@circ@res@left}{0pt}}
        \pgfpathlineto{\pgfpoint{0pt}{\pgf@circ@res@up}}
        \pgfpathlineto{\pgfpoint{0pt}{\pgf@circ@res@down}}
        \pgfpathlineto{\pgfpoint{\pgf@circ@res@right-\pgf@circ@res@left}{0pt}}
        \pgfusepath{draw,fill}
    \endpgfscope
    \pgfpathmoveto{\pgfpoint{\pgf@circ@res@right}{\pgf@circ@res@down}}
    \pgfpathlineto{\pgfpoint{\pgf@circ@res@right}{\pgf@circ@res@up}}
    \pgfusepath{draw}

    \pgf@circ@draw@pdarrows
}

%% Black varcap
\pgfcircdeclarebipolescaled{diodes}
{}
{\ctikzvalof{bipoles/varcap/height}}
{fullvarcap}
{\ctikzvalof{bipoles/varcap/height}}
{\ctikzvalof{bipoles/varcap/width}}
{
    \pgf@circ@res@temp=\ctikzvalof{bipoles/thickness}\pgfstartlinewidth
    \pgfsetlinewidth{\pgf@circ@res@temp}
    \pgfpathmoveto{\pgfpoint{\pgf@circ@res@right-2\pgf@circ@res@temp}{0pt}}
    \pgfpathlineto{\pgfpoint{\pgf@circ@res@left}{\pgf@circ@res@up}}
    \pgfpathlineto{\pgfpoint{\pgf@circ@res@left}{\pgf@circ@res@down}}
    \pgfpathlineto{\pgfpoint{\pgf@circ@res@right-2\pgf@circ@res@temp}{0pt}}
    \pgfusepath{draw,fill}
    %
    \pgfpathmoveto{\pgfpoint{\pgf@circ@res@right-2\pgf@circ@res@temp}{\pgf@circ@res@down}}
    \pgfpathlineto{\pgfpoint{\pgf@circ@res@right-2\pgf@circ@res@temp}{\pgf@circ@res@up}}
    \pgfpathmoveto{\pgfpoint{\pgf@circ@res@right}{\pgf@circ@res@down}}
    \pgfpathlineto{\pgfpoint{\pgf@circ@res@right}{\pgf@circ@res@up}}
    \pgfusepath{draw}
}

%% Code for the diode triangle
\def\pgf@circ@basicdiodeshape{
    % \pgfscope
        \pgftransformxshift{\pgf@circ@res@left}
        \pgfpathmoveto{\pgfpoint{\pgf@circ@res@right-\pgf@circ@res@left}{0pt}}
        \pgfpathlineto{\pgfpoint{0pt}{\pgf@circ@res@up}}
        \pgfpathlineto{\pgfpoint{0pt}{\pgf@circ@res@down}}
        \pgfpathclose
        \pgf@circ@draworfill
        \pgfscope
            % to allow filling, we need to draw explicitly the stroke here.
            \pgfsetlinewidth{\pgfstartlinewidth}
            \ifpgf@circuit@bipole@strokedsymbol
                \pgfpathmoveto{\pgfpoint{\pgf@circ@res@right-\pgf@circ@res@left}{0pt}}
                \pgfpathlineto{\pgfpoint{0pt}{0pt}}
                \pgfusepath{draw}
            \fi
        \endpgfscope
    % \endpgfscope
}

%% Empty generic diode
\pgfcircdeclarebipolescaled{diodes}
{}
{\ctikzvalof{bipoles/diode/height}}
{emptydiode}
{\ctikzvalof{bipoles/diode/height}}
{\ctikzvalof{bipoles/diode/width}}
{
    \pgf@circ@setlinewidth{bipoles}{\pgfstartlinewidth}
    \pgfscope
        \pgf@circ@basicdiodeshape
    \endpgfscope
    \pgfpathmoveto{\pgfpoint{\pgf@circ@res@right}{\pgf@circ@res@down}}
    \pgfpathlineto{\pgfpoint{\pgf@circ@res@right}{\pgf@circ@res@up}}
    \pgfusepath{draw}
}

%% Empty Zener diode
\pgfcircdeclarebipolescaled{diodes}
{}
{\ctikzvalof{bipoles/diode/height}}
{emptyzdiode}
{\ctikzvalof{bipoles/diode/height}}
{\ctikzvalof{bipoles/diode/width}}
{

    \pgf@circ@setlinewidth{bipoles}{\pgfstartlinewidth}

    \pgfscope
        \pgf@circ@basicdiodeshape
        \pgfpathmoveto{\pgfpoint{\pgf@circ@res@right-\pgf@circ@res@left}{\pgf@circ@res@down}}
        \pgfpathlineto{\pgfpoint{\pgf@circ@res@right-\pgf@circ@res@left}{\pgf@circ@res@up}}
        \pgfpathlineto{\pgfpoint{\pgf@circ@res@right-0.6\pgf@circ@res@left}{\pgf@circ@res@up}}
        \pgfusepath{draw}
    \endpgfscope
}

%% Empty alternative zigzag Zener diode
\pgfcircdeclarebipolescaled{diodes}
{}
{\ctikzvalof{bipoles/diode/height}}
{emptyzzdiode}
{\ctikzvalof{bipoles/diode/height}}
{\ctikzvalof{bipoles/diode/width}}
{

    \pgf@circ@setlinewidth{bipoles}{\pgfstartlinewidth}

    \pgfscope
        \pgf@circ@basicdiodeshape
        \pgfpathmoveto{\pgfpoint{\pgf@circ@res@right-1.8\pgf@circ@res@left}{\pgf@circ@res@down-0.5\pgf@circ@res@up}}
        \pgfpathlineto{\pgfpoint{\pgf@circ@res@right-\pgf@circ@res@left}{\pgf@circ@res@down}}
        \pgfpathlineto{\pgfpoint{\pgf@circ@res@right-\pgf@circ@res@left}{\pgf@circ@res@up}}
        \pgfpathlineto{\pgfpoint{\pgf@circ@res@right-0.2\pgf@circ@res@left}{\pgf@circ@res@up-0.5\pgf@circ@res@down}}
        \pgfusepath{draw}
    \endpgfscope
}
%% Empty Schottky diode
\pgfcircdeclarebipolescaled{diodes}
{}
{\ctikzvalof{bipoles/diode/height}}
{emptysdiode}
{\ctikzvalof{bipoles/diode/height}}
{\ctikzvalof{bipoles/diode/width}}
{

    \pgf@circ@setlinewidth{bipoles}{\pgfstartlinewidth}
    \pgfscope
        \pgf@circ@basicdiodeshape
        \pgfpathmoveto{\pgfpoint{\pgf@circ@res@right-.6\pgf@circ@res@left}{.6\pgf@circ@res@down}}
        \pgfpathlineto{\pgfpoint{\pgf@circ@res@right-.6\pgf@circ@res@left}{\pgf@circ@res@down}}
        \pgfpathlineto{\pgfpoint{\pgf@circ@res@right-\pgf@circ@res@left}{\pgf@circ@res@down}}
        \pgfpathlineto{\pgfpoint{\pgf@circ@res@right-\pgf@circ@res@left}{\pgf@circ@res@down}}
        \pgfpathlineto{\pgfpoint{\pgf@circ@res@right-\pgf@circ@res@left}{\pgf@circ@res@up}}
        \pgfpathlineto{\pgfpoint{\pgf@circ@res@right-1.4\pgf@circ@res@left}{\pgf@circ@res@up}}
        \pgfpathlineto{\pgfpoint{\pgf@circ@res@right-1.4\pgf@circ@res@left}{.6\pgf@circ@res@up}}
        \pgfusepath{draw}
    \endpgfscope

}

%% Empty tunnel diode
\pgfcircdeclarebipolescaled{diodes}
{}
{\ctikzvalof{bipoles/diode/height}}
{emptytdiode}
{\ctikzvalof{bipoles/diode/height}}
{\ctikzvalof{bipoles/diode/width}}
{

    \pgf@circ@setlinewidth{bipoles}{\pgfstartlinewidth}
    \pgfscope
        \pgf@circ@basicdiodeshape
        \pgfpathmoveto{\pgfpoint{\pgf@circ@res@right-.6\pgf@circ@res@left}{\pgf@circ@res@down}}
        \pgfpathlineto{\pgfpoint{\pgf@circ@res@right-\pgf@circ@res@left}{\pgf@circ@res@down}}
        \pgfpathlineto{\pgfpoint{\pgf@circ@res@right-\pgf@circ@res@left}{\pgf@circ@res@up}}
        \pgfpathlineto{\pgfpoint{\pgf@circ@res@right-.6\pgf@circ@res@left}{\pgf@circ@res@up}}
        \pgfusepath{draw}
    \endpgfscope
}

%% Empty light emitting diode
\pgfcircdeclarebipolescaled{diodes}
{}
{\ctikzvalof{bipoles/diode/height}}
{emptylediode}
{\ctikzvalof{bipoles/diode/height}}
{\ctikzvalof{bipoles/diode/width}}
{

    \pgf@circ@setlinewidth{bipoles}{\pgfstartlinewidth}
    \pgfscope
        \pgf@circ@basicdiodeshape
    \endpgfscope
    \pgfpathmoveto{\pgfpoint{\pgf@circ@res@right}{\pgf@circ@res@down}}
    \pgfpathlineto{\pgfpoint{\pgf@circ@res@right}{\pgf@circ@res@up}}
    \pgfusepath{draw}

    \pgf@circ@draw@ledarrows
}

%% Empty photodiode
\pgfcircdeclarebipolescaled{diodes}
{}
{\ctikzvalof{bipoles/diode/height}}
{emptypdiode}
{\ctikzvalof{bipoles/diode/height}}
{\ctikzvalof{bipoles/diode/width}}
{

    \pgf@circ@setlinewidth{bipoles}{\pgfstartlinewidth}
    \pgfscope
        \pgf@circ@basicdiodeshape
    \endpgfscope
    \pgfpathmoveto{\pgfpoint{\pgf@circ@res@right}{\pgf@circ@res@down}}
    \pgfpathlineto{\pgfpoint{\pgf@circ@res@right}{\pgf@circ@res@up}}
    \pgfusepath{draw}

    \pgf@circ@draw@pdarrows
}

%% Empty varcap
\pgfcircdeclarebipolescaled{diodes}
{}
{\ctikzvalof{bipoles/varcap/height}}
{emptyvarcap}
{\ctikzvalof{bipoles/varcap/height}}
{\ctikzvalof{bipoles/varcap/width}}
{
    \pgf@circ@res@temp=\ctikzvalof{bipoles/thickness}\pgfstartlinewidth
    \pgfsetlinewidth{\pgf@circ@res@temp}
    \pgfpathmoveto{\pgfpoint{\pgf@circ@res@right-2\pgf@circ@res@temp}{0pt}}
    \pgfpathlineto{\pgfpoint{\pgf@circ@res@left}{\pgf@circ@res@up}}
    \pgfpathlineto{\pgfpoint{\pgf@circ@res@left}{\pgf@circ@res@down}}
    % \pgfpathlineto{\pgfpoint{\pgf@circ@res@right-2\pgf@circ@res@temp}{0pt}}
    \pgfpathclose
    \pgf@circ@draworfill
    \pgfscope
        % to allow filling, we need to draw explicitily the stroke here.
        \pgfsetlinewidth{\pgfstartlinewidth}
        \ifpgf@circuit@bipole@strokedsymbol
            \pgfpathmoveto{\pgfpoint{\pgf@circ@res@right-2\pgf@circ@res@temp}{0pt}}
            \pgfpathlineto{\pgfpoint{\pgf@circ@res@left}{0pt}}
            \pgfusepath{draw}
        \fi
    \endpgfscope
    %
    \pgfpathmoveto{\pgfpoint{\pgf@circ@res@right-2\pgf@circ@res@temp}{\pgf@circ@res@down}}
    \pgfpathlineto{\pgfpoint{\pgf@circ@res@right-2\pgf@circ@res@temp}{\pgf@circ@res@up}}
    \pgfpathmoveto{\pgfpoint{\pgf@circ@res@right}{\pgf@circ@res@down}}
    \pgfpathlineto{\pgfpoint{\pgf@circ@res@right}{\pgf@circ@res@up}}
    \pgfusepath{draw}
}

%% Empty bidirectionaldiode
\pgfcircdeclarebipolescaled{diodes}
{}
{\ctikzvalof{bipoles/bidirectionaldiode/height}}
{emptybidirectionaldiode}
{\ctikzvalof{bipoles/bidirectionaldiode/height}}
{\ctikzvalof{bipoles/bidirectionaldiode/width}}
{

    \pgf@circ@setlinewidth{bipoles}{\pgfstartlinewidth}

    \pgf@circ@res@other = \ctikzvalof{bipoles/bidirectionaldiode/diode width left}\pgf@circ@res@left
    \pgf@circ@res@step = \ctikzvalof{bipoles/bidirectionaldiode/diode width right}\pgf@circ@res@right

    \pgfpathmoveto{\pgfpoint{\pgf@circ@res@other}{0pt}}
    \pgfpathlineto{\pgfpoint{0.95\pgf@circ@res@step}{0.707*(\pgf@circ@res@other-\pgf@circ@res@step)}} % sqrt(1/2)
    \pgfpathlineto{\pgfpoint{\pgf@circ@res@other}{1.414*(\pgf@circ@res@other-\pgf@circ@res@step)}}
    \pgfpathlineto{\pgfpoint{\pgf@circ@res@other}{-1.414*(\pgf@circ@res@other-\pgf@circ@res@step)}}

    \pgfpathmoveto{\pgfpoint{\pgf@circ@res@step}{0pt}}
    \pgfpathlineto{\pgfpoint{0.95\pgf@circ@res@other}{-0.707*(\pgf@circ@res@other-\pgf@circ@res@step)}} % sqrt(1/2)
    \pgfpathlineto{\pgfpoint{\pgf@circ@res@step}{-1.414*(\pgf@circ@res@other-\pgf@circ@res@step)}}
    \pgfpathlineto{\pgfpoint{\pgf@circ@res@step}{1.414*(\pgf@circ@res@other-\pgf@circ@res@step)}}

    \pgf@circ@draworfill

    \pgfsetlinewidth{\pgfstartlinewidth}

    \pgfpathmoveto{\pgfpoint{\pgf@circ@res@other}{0pt}}
    \pgfpathlineto{\pgfpoint{\pgf@circ@res@left}{0pt}}

    \pgfpathmoveto{\pgfpoint{\pgf@circ@res@step}{0pt}}
    \pgfpathlineto{\pgfpoint{\pgf@circ@res@right}{0pt}}

    \pgfusepath{draw}
}

%% Full bidirectionaldiode
\pgfcircdeclarebipolescaled{diodes}
{}
{\ctikzvalof{bipoles/bidirectionaldiode/height}}
{fullbidirectionaldiode}
{\ctikzvalof{bipoles/bidirectionaldiode/height}}
{\ctikzvalof{bipoles/bidirectionaldiode/width}}
{

    \pgf@circ@setlinewidth{bipoles}{\pgfstartlinewidth}

    \pgf@circ@res@other = \ctikzvalof{bipoles/bidirectionaldiode/diode width left}\pgf@circ@res@left
    \pgf@circ@res@step = \ctikzvalof{bipoles/bidirectionaldiode/diode width right}\pgf@circ@res@right

    \pgfpathmoveto{\pgfpoint{\pgf@circ@res@other}{0pt}}
    \pgfpathlineto{\pgfpoint{0.95\pgf@circ@res@step}{0.707*(\pgf@circ@res@other-\pgf@circ@res@step)}} % sqrt(1/2)
    \pgfpathlineto{\pgfpoint{\pgf@circ@res@other}{1.414*(\pgf@circ@res@other-\pgf@circ@res@step)}}
    \pgfpathlineto{\pgfpoint{\pgf@circ@res@other}{-1.414*(\pgf@circ@res@other-\pgf@circ@res@step)}}

    \pgfpathmoveto{\pgfpoint{\pgf@circ@res@step}{0pt}}
    \pgfpathlineto{\pgfpoint{0.95\pgf@circ@res@other}{-0.707*(\pgf@circ@res@other-\pgf@circ@res@step)}} % sqrt(1/2)
    \pgfpathlineto{\pgfpoint{\pgf@circ@res@step}{-1.414*(\pgf@circ@res@other-\pgf@circ@res@step)}}
    \pgfpathlineto{\pgfpoint{\pgf@circ@res@step}{1.414*(\pgf@circ@res@other-\pgf@circ@res@step)}}

    \pgfusepath{draw, fill}

    \pgfsetlinewidth{\pgfstartlinewidth}

    \pgfpathmoveto{\pgfpoint{\pgf@circ@res@other}{0pt}}
    \pgfpathlineto{\pgfpoint{\pgf@circ@res@left}{0pt}}

    \pgfpathmoveto{\pgfpoint{\pgf@circ@res@step}{0pt}}
    \pgfpathlineto{\pgfpoint{\pgf@circ@res@right}{0pt}}

    \pgfusepath{draw}
}

%% Black thyristor
\pgfcircdeclarebipolescaled{diodes}
{
    \anchor{gate}{\northeast}
    \anchor{anode}{\southwest\pgf@y=0cm}
    \anchor{G}{\northeast}
    \anchor{cathode}{\northeast\pgf@y=0cm }
}
{\ctikzvalof{tripoles/thyristor/height 2}}
{fullthyristor}
{\ctikzvalof{tripoles/thyristor/height}}
{\ctikzvalof{tripoles/thyristor/width}}
{

    \pgf@circ@setlinewidth{bipoles}{\pgfstartlinewidth}

    \pgf@circ@res@other = \ctikzvalof{tripoles/thyristor/diode width left}\pgf@circ@res@left
    \pgf@circ@res@step = \ctikzvalof{tripoles/thyristor/diode width right}\pgf@circ@res@right

    \pgfscope
        \pgftransformxshift{\pgf@circ@res@other}
        \pgfpathmoveto{\pgfpoint{\pgf@circ@res@step-\pgf@circ@res@other}{0pt}}

        \pgfpathlineto{\pgfpoint{0pt}{-\pgf@circ@res@down}}
        \pgfpathlineto{\pgfpoint{0pt}{\pgf@circ@res@down}}
        \pgfpathlineto{\pgfpoint{\pgf@circ@res@step-\pgf@circ@res@other}{0pt}}
        \pgfusepath{draw,fill}

        \pgfsetlinewidth{\pgfstartlinewidth}

        \pgfpathmoveto{\pgfpoint{\pgf@circ@res@step-\pgf@circ@res@other}{0pt}}
        \pgfpathlineto{\pgfpoint{2*\pgf@circ@res@step-2*\pgf@circ@res@other}{\ctikzvalof{tripoles/thyristor/diode height}\pgf@circ@res@up}}
        \pgfpathlineto{\pgfpoint{2*\pgf@circ@res@step-2*\pgf@circ@res@other}{\pgf@circ@res@up}}
        \pgfusepath{draw}
    \endpgfscope

    \pgfpathmoveto{\pgfpoint{\pgf@circ@res@step}{\pgf@circ@res@down}}
    \pgfpathlineto{\pgfpoint{\pgf@circ@res@step}{-\pgf@circ@res@down}}

    \pgfusepath{draw}

    \pgfsetlinewidth{\pgfstartlinewidth}

    \pgfpathmoveto{\pgfpoint{\pgf@circ@res@other}{0pt}}
    \pgfpathlineto{\pgfpoint{\pgf@circ@res@left}{0pt}}

    \pgfpathmoveto{\pgfpoint{\pgf@circ@res@step}{0pt}}
    \pgfpathlineto{\pgfpoint{\pgf@circ@res@right}{0pt}}

    \pgfusepath{draw}

}

%% Empty thyristor

\pgfcircdeclarebipolescaled{diodes}
{
    \anchor{gate}{\northeast}
    \anchor{anode}{\southwest\pgf@y=0cm}
    \anchor{G}{\northeast}
    \anchor{cathode}{\northeast\pgf@y=0cm }
}
{\ctikzvalof{tripoles/thyristor/height 2}}
{emptythyristor}
{\ctikzvalof{tripoles/thyristor/height}}
{\ctikzvalof{tripoles/thyristor/width}}
{

    \pgf@circ@setlinewidth{bipoles}{\pgfstartlinewidth}

    \pgf@circ@res@other = \ctikzvalof{tripoles/thyristor/diode width left}\pgf@circ@res@left
    \pgf@circ@res@step = \ctikzvalof{tripoles/thyristor/diode width right}\pgf@circ@res@right

    \pgfscope
        \pgftransformxshift{\pgf@circ@res@other}
        \pgfpathmoveto{\pgfpoint{\pgf@circ@res@step-\pgf@circ@res@other}{0pt}}

        \pgfpathlineto{\pgfpoint{0pt}{-\pgf@circ@res@down}}
        \pgfpathlineto{\pgfpoint{0pt}{\pgf@circ@res@down}}
        \pgfpathclose
        \pgf@circ@draworfill
        \pgfscope
            % to allow filling, we need to draw explicitily the stroke here.
            \pgfsetlinewidth{\pgfstartlinewidth}
            \ifpgf@circuit@bipole@strokedsymbol
                \pgfpathmoveto{\pgfpoint{\pgf@circ@res@step-\pgf@circ@res@other}{0pt}}
                \pgfpathlineto{\pgfpoint{0pt}{0pt}}
                \pgfusepath{draw}
            \fi
        \endpgfscope

        \pgfsetlinewidth{\pgfstartlinewidth}

        \pgfpathmoveto{\pgfpoint{\pgf@circ@res@step-\pgf@circ@res@other}{0pt}}
        \pgfpathlineto{\pgfpoint{2*\pgf@circ@res@step-2*\pgf@circ@res@other}{\ctikzvalof{tripoles/thyristor/diode height}\pgf@circ@res@up}}
        \pgfpathlineto{\pgfpoint{2*\pgf@circ@res@step-2*\pgf@circ@res@other}{\pgf@circ@res@up}}
        \pgfusepath{draw}
    \endpgfscope

    \pgfpathmoveto{\pgfpoint{\pgf@circ@res@step}{\pgf@circ@res@down}}
    \pgfpathlineto{\pgfpoint{\pgf@circ@res@step}{-\pgf@circ@res@down}}

    \pgfusepath{draw}

    \pgfsetlinewidth{\pgfstartlinewidth}

    \pgfpathmoveto{\pgfpoint{\pgf@circ@res@other}{0pt}}
    \pgfpathlineto{\pgfpoint{\pgf@circ@res@left}{0pt}}

    \pgfpathmoveto{\pgfpoint{\pgf@circ@res@step}{0pt}}
    \pgfpathlineto{\pgfpoint{\pgf@circ@res@right}{0pt}}

    \pgfusepath{draw}
}

%% Empty triac
\pgfcircdeclarebipolescaled{diodes}
{
    \anchor{gate}{\northeast}
    \anchor{anode}{\southwest\pgf@y=0cm}
    \anchor{G}{\northeast}
    \anchor{cathode}{\northeast\pgf@y=0cm }
}
{\ctikzvalof{tripoles/triac/height}}
{emptytriac}
{\ctikzvalof{tripoles/triac/height}}
{\ctikzvalof{tripoles/triac/width}}
{

    \pgf@circ@setlinewidth{bipoles}{\pgfstartlinewidth}

    \pgf@circ@res@other = \ctikzvalof{tripoles/triac/diode width left}\pgf@circ@res@left
    \pgf@circ@res@step = \ctikzvalof{tripoles/triac/diode width right}\pgf@circ@res@right

    \pgfpathmoveto{\pgfpoint{\pgf@circ@res@other}{0pt}}
    \pgfpathlineto{\pgfpoint{0.95\pgf@circ@res@step}{0.707*(\pgf@circ@res@other-\pgf@circ@res@step)}} % sqrt(1/2)
    \pgfpathlineto{\pgfpoint{\pgf@circ@res@other}{1.414*(\pgf@circ@res@other-\pgf@circ@res@step)}}
    \pgfpathlineto{\pgfpoint{\pgf@circ@res@other}{-1.414*(\pgf@circ@res@other-\pgf@circ@res@step)}}

    \pgfpathmoveto{\pgfpoint{\pgf@circ@res@step}{0pt}}
    \pgfpathlineto{\pgfpoint{0.95\pgf@circ@res@other}{-0.707*(\pgf@circ@res@other-\pgf@circ@res@step)}} % sqrt(1/2)
    \pgfpathlineto{\pgfpoint{\pgf@circ@res@step}{-1.414*(\pgf@circ@res@other-\pgf@circ@res@step)}}
    \pgfpathlineto{\pgfpoint{\pgf@circ@res@step}{1.414*(\pgf@circ@res@other-\pgf@circ@res@step)}}

    \pgf@circ@draworfill

    \pgfsetlinewidth{\pgfstartlinewidth}

    \pgfpathmoveto{\pgfpoint{\pgf@circ@res@step}{-0.707*(\pgf@circ@res@other-\pgf@circ@res@step)}} % sqrt(1/2)
    \pgfpathlineto{\pgfpoint{\pgf@circ@res@right}{-1.414*(\pgf@circ@res@other-\pgf@circ@res@step)}} % sqrt(1/2)
    \pgfpathlineto{\pgfpoint{\pgf@circ@res@right}{\pgf@circ@res@up}} % sqrt(1/2)

    \pgfpathmoveto{\pgfpoint{\pgf@circ@res@other}{0pt}}
    \pgfpathlineto{\pgfpoint{\pgf@circ@res@left}{0pt}}

    \pgfpathmoveto{\pgfpoint{\pgf@circ@res@step}{0pt}}
    \pgfpathlineto{\pgfpoint{\pgf@circ@res@right}{0pt}}

    \pgfusepath{draw}
}

%% Full triac
\pgfcircdeclarebipolescaled{diodes}
{
    \anchor{gate}{\northeast}
    \anchor{anode}{\southwest\pgf@y=0cm}
    \anchor{G}{\northeast}
    \anchor{cathode}{\northeast\pgf@y=0cm }
}
{\ctikzvalof{tripoles/triac/height}}
{fulltriac}
{\ctikzvalof{tripoles/triac/height}}
{\ctikzvalof{tripoles/triac/width}}
{
    \pgf@circ@setlinewidth{bipoles}{\pgfstartlinewidth}

    \pgf@circ@res@other = \ctikzvalof{tripoles/triac/diode width left}\pgf@circ@res@left
    \pgf@circ@res@step = \ctikzvalof{tripoles/triac/diode width right}\pgf@circ@res@right

    \pgfpathmoveto{\pgfpoint{\pgf@circ@res@other}{0pt}}
    \pgfpathlineto{\pgfpoint{0.95\pgf@circ@res@step}{0.707*(\pgf@circ@res@other-\pgf@circ@res@step)}} % sqrt(1/2)
    \pgfpathlineto{\pgfpoint{\pgf@circ@res@other}{1.414*(\pgf@circ@res@other-\pgf@circ@res@step)}}
    \pgfpathlineto{\pgfpoint{\pgf@circ@res@other}{-1.414*(\pgf@circ@res@other-\pgf@circ@res@step)}}

    \pgfpathmoveto{\pgfpoint{\pgf@circ@res@step}{0pt}}
    \pgfpathlineto{\pgfpoint{0.95\pgf@circ@res@other}{-0.707*(\pgf@circ@res@other-\pgf@circ@res@step)}} % sqrt(1/2)
    \pgfpathlineto{\pgfpoint{\pgf@circ@res@step}{-1.414*(\pgf@circ@res@other-\pgf@circ@res@step)}}
    \pgfpathlineto{\pgfpoint{\pgf@circ@res@step}{1.414*(\pgf@circ@res@other-\pgf@circ@res@step)}}

    \pgfusepath{draw,fill}

    \pgfsetlinewidth{\pgfstartlinewidth}

    \pgfpathmoveto{\pgfpoint{\pgf@circ@res@step}{-0.707*(\pgf@circ@res@other-\pgf@circ@res@step)}} % sqrt(1/2)
    \pgfpathlineto{\pgfpoint{\pgf@circ@res@right}{-1.414*(\pgf@circ@res@other-\pgf@circ@res@step)}} % sqrt(1/2)
    \pgfpathlineto{\pgfpoint{\pgf@circ@res@right}{\pgf@circ@res@up}} % sqrt(1/2)

    \pgfpathmoveto{\pgfpoint{\pgf@circ@res@other}{0pt}}
    \pgfpathlineto{\pgfpoint{\pgf@circ@res@left}{0pt}}

    \pgfpathmoveto{\pgfpoint{\pgf@circ@res@step}{0pt}}
    \pgfpathlineto{\pgfpoint{\pgf@circ@res@right}{0pt}}

    \pgfusepath{draw}
}

%%%%%%%%%%%%%
%% switches
%%%%%%%%%%%%%

%% (Closing) SPST
\pgfcircdeclarebipolescaled{switches}
{}
{\ctikzvalof{bipoles/spst/depth}}
{cspst}
{\ctikzvalof{bipoles/spst/height}}
{\ctikzvalof{bipoles/spst/width}}
{
    \pgf@circ@setlinewidth{bipoles}{\pgfstartlinewidth}
    \pgfpathmoveto{\pgfpoint{\pgf@circ@res@left}{0pt}}
    \pgfpathlineto{\pgfpoint{.6\pgf@circ@res@right}{\pgf@circ@res@up}}
    \pgfusepath{draw}

    \pgfscope
        \pgftransformshift{\pgfpoint{\pgf@circ@res@left}{0pt}}
        \pgfpathmoveto{\pgfpointpolar{90}{1.2\pgf@circ@res@right}}
        \pgfpatharc{90}{-20}{1.2\pgf@circ@res@right}
        \pgfsetarrowsend{latexslim}
        \pgfsetbeveljoin
        \pgfusepath{draw}
    \endpgfscope
}

%% Opening SPST
\pgfcircdeclarebipolescaled{switches}
{}
{\ctikzvalof{bipoles/spst/depth}}
{ospst}
{\ctikzvalof{bipoles/spst/height}}
{\ctikzvalof{bipoles/spst/width}}
{

    \pgf@circ@setlinewidth{bipoles}{\pgfstartlinewidth}
    \pgfpathmoveto{\pgfpoint{\pgf@circ@res@left}{0pt}}
    \pgfpathlineto{\pgfpoint{.6\pgf@circ@res@right}{\pgf@circ@res@up}}
    \pgfusepath{draw}

    \pgfscope
        \pgftransformshift{\pgfpoint{\pgf@circ@res@left}{0pt}}
        \pgfpathmoveto{\pgfpointpolar{-10}{1.2\pgf@circ@res@right}}
        \pgfpatharc{-10}{90}{1.2\pgf@circ@res@right}
        \pgfsetarrowsend{latexslim}
        \pgfsetbeveljoin
        \pgfusepath{draw}
    \endpgfscope
}

%% Normal open Switch
\pgfcircdeclarebipolescaled{switches}
{}
{\ctikzvalof{bipoles/nos/depth}}
{nos}
{\ctikzvalof{bipoles/nos/height}}
{\ctikzvalof{bipoles/nos/width}}
{

    \pgf@circ@setlinewidth{bipoles}{\pgfstartlinewidth}
    \pgfpathmoveto{\pgfpoint{\pgf@circ@res@left}{0pt}}
    \pgfpathlineto{\pgfpoint{.9\pgf@circ@res@right}{\pgf@circ@res@up}}
    \pgfsetbuttcap
    \pgfusepath{draw}
}

%% Normal closed Switch
\pgfcircdeclarebipolescaled{switches}
{}
{\ctikzvalof{bipoles/ncs/depth}}
{ncs}
{\ctikzvalof{bipoles/ncs/height}}
{\ctikzvalof{bipoles/ncs/width}}
{
    \pgf@circ@setlinewidth{bipoles}{\pgfstartlinewidth}
    \pgfpathmoveto{\pgfpoint{\pgf@circ@res@left}{0pt}}
    \pgfpathlineto{\pgfpoint{.9\pgf@circ@res@right}{\pgf@circ@res@up}}
    \pgfusepath{draw}
    \pgfsetlinewidth{\pgfstartlinewidth}
    \pgfpathmoveto{\pgfpoint{\pgf@circ@res@right}{0pt}}
    \pgfpathlineto{\pgfpoint{.2\pgf@circ@res@right}{0pt}}
    \pgfpathlineto{\pgfpoint{.2\pgf@circ@res@right}{\pgf@circ@res@up}}
    \pgfusepath{draw}
}

%% Push Button
\pgfcircdeclarebipolescaled{switches}
{}
{\ctikzvalof{bipoles/pushbutton/height 2}}
{pushbutton}
{\ctikzvalof{bipoles/pushbutton/height}}
{\ctikzvalof{bipoles/pushbutton/width}}
{
    \pgf@circ@setlinewidth{bipoles}{\pgfstartlinewidth}
    \pgfpathmoveto{\pgfpoint{\pgf@circ@res@left}{.5\pgf@circ@res@up}}
    \pgfpathlineto{\pgfpoint{\pgf@circ@res@right}{.5\pgf@circ@res@up}}
    \pgfpathmoveto{\pgfpoint{0}{.5\pgf@circ@res@up}}
    \pgfpathlineto{\pgfpoint{0}{\pgf@circ@res@up}}
    \pgfusepath{draw}

    \pgftransformshift{\pgfpoint{\pgf@circ@res@left}{0pt}}
    \pgfnode{ocirc}{center}{}{}{\pgfusepath{draw}}
    \pgftransformshift{\pgfpoint{2\pgf@circ@res@right}{0pt}}
    \pgfnode{ocirc}{center}{}{}{\pgfusepath{draw}}
}

%% Normally closed Push Button
\pgfcircdeclarebipolescaled{switches}
{}
{\ctikzvalof{bipoles/pushbutton/height 2}}
{ncpushbutton}
{\ctikzvalof{bipoles/pushbutton/height}}
{\ctikzvalof{bipoles/pushbutton/width}}
{
    \pgf@circ@setlinewidth{bipoles}{\pgfstartlinewidth}
    % This is the radius of the "ocirc" shape (see pgfcircshapes.tex)
    % Warning, if the nodes will have a class, we have to touch this.
    \pgf@circ@res@temp=\ctikzvalof{nodes width}\pgf@circ@Rlen
    \pgfpathmoveto{\pgfpoint{\pgf@circ@res@left}{-\pgf@circ@res@temp}}
    \pgfpathlineto{\pgfpoint{\pgf@circ@res@right}{-\pgf@circ@res@temp}}
    \pgfpathmoveto{\pgfpoint{0}{-\pgf@circ@res@temp}}
    \pgfpathlineto{\pgfpoint{0}{\pgf@circ@res@up}}
    \pgfusepath{draw}
    %
    \pgftransformshift{\pgfpoint{\pgf@circ@res@left}{0pt}}
    \pgfnode{ocirc}{center}{}{}{\pgfusepath{draw}}
    \pgftransformshift{\pgfpoint{2\pgf@circ@res@right}{0pt}}
    \pgfnode{ocirc}{center}{}{}{\pgfusepath{draw}}
}
%%% reed switches
\pgfcircdeclarebipolescaled{switches}
{}
{\ctikzvalof{bipoles/reed/depth}}
{reed}
{\ctikzvalof{bipoles/reed/height}}
{\ctikzvalof{bipoles/reed/width}}
{
    % this is designed to be the same as a "nos".
    \pgfmathsetmacro{\@@tmpx}{0.9*\ctikzvalof{bipoles/nos/width}/\ctikzvalof{bipoles/reed/width}}
    \pgfmathsetmacro{\@@tmpy}{\ctikzvalof{bipoles/nos/height}/\ctikzvalof{bipoles/reed/height}}
    \pgfscope
        \pgf@circ@setlinewidth{bipoles}{\pgfstartlinewidth}
        % eclosure
        \pgfpathmoveto{\pgfpoint{\pgf@circ@res@right-\pgf@circ@res@up}{\pgf@circ@res@up}}
        \pgfpatharc{90}{-90}{\pgf@circ@res@up}
        \pgfpathlineto{\pgfpoint{\pgf@circ@res@left+\pgf@circ@res@up}{-\pgf@circ@res@up}}
        \pgfpatharc{270}{90}{\pgf@circ@res@up}
        \pgfpathclose
        \pgf@circ@draworfill
        % switch
        \pgfpathmoveto{\pgfpoint{\@@tmpx\pgf@circ@res@left}{0pt}}
        \pgfpathlineto{\pgfpoint{\@@tmpx\pgf@circ@res@right}{\@@tmpy\pgf@circ@res@up}}
        % connection lines
        \pgfsetbuttcap
        \pgfusepath{draw}
    \endpgfscope
    % connection lines
    \pgfpathmoveto{\pgfpoint{\pgf@circ@res@left}{0pt}}
    \pgfpathlineto{\pgfpoint{\@@tmpx\pgf@circ@res@left}{0pt}}
    \pgfpathmoveto{\pgfpoint{\pgf@circ@res@right}{0pt}}
    \pgfpathlineto{\pgfpoint{\@@tmpx\pgf@circ@res@right}{0pt}}
    \pgfusepath{draw}
}

% cute switch "to" shapes help function
% #1 -> name
% #2 -> barposition
% #3 -> arrowcode
\long\def\pgfcircdeclarecutesw#1#2#3{
    \pgfcircdeclarebipolescaled{switches}
    {
        \savedanchor\midlever{
            % these values are calculated when we create the definition of the shape.
            \pgfmathsetlength{\pgf@circ@scaled@Rlen}{\ctikzvalof{\ctikzclass/scale}\pgf@circ@Rlen}
            \pgf@y=\ctikzvalof{bipoles/cuteswitch/height}\pgf@circ@scaled@Rlen
            \pgf@circ@res@temp=\ctikzvalof{nodes width}\pgf@circ@scaled@Rlen
            \pgf@circ@res@temp=\ctikzvalof{bipoles/cuteswitch/thickness}\pgf@circ@res@temp
            \pgf@circ@res@down = -.5\pgf@y
            \pgf@circ@res@up = .5\pgf@y
            \pgfextracty{\pgf@circ@res@other}{#2}
            \pgf@x=0pt
            \pgf@y=.5\pgf@circ@res@other
        }
        % radius of the connector
        % This is the radius of the "ocirc" shape (see pgfcircshapes.tex)
        % If cnnecting nodes are scaled, we have to modify this
        \saveddimen{\radius}{\pgfmathsetlength\pgf@x{\pgf@circ@Rlen*\ctikzvalof{nodes width}}}
        % shapename
        \savedmacro{\thisshape}{\def\thisshape{\tikz@fig@name}}
        % shape type
        \savedmacro{\cshape}{\def\cshape{\ctikzvalof{bipoles/cuteswitch/shape}}}
        % mid of the lever, to stack switches
        \anchor{mid}{\midlever}
        \anchor{cout}{\northeast \pgf@y=0cm}
        \anchor{cin}{\southwest\pgf@y=0cm}
        \anchor{out}{\northeast \pgf@y=0cm\advance\pgf@x by \radius}
        \anchor{in}{\southwest\pgf@y=0cm\advance\pgf@x by -\radius}
    }
    {\ctikzvalof{bipoles/cuteswitch/height 2}}
    {#1}
    {\ctikzvalof{bipoles/cuteswitch/height}}
    {\ctikzvalof{bipoles/cuteswitch/width}}{
        \pgfscope
        % This is the radius of the "ocirc" shape (see pgfcircshapes.tex)
        % If cnnecting nodes are scaled, we have to modify this
        \pgf@circ@res@temp=\ctikzvalof{nodes width}\pgf@circ@Rlen
        \pgf@circ@res@temp=\ctikzvalof{bipoles/cuteswitch/thickness}\pgf@circ@res@temp
        \pgfsetlinewidth{2\pgf@circ@res@temp}
        \pgfpathmoveto{\pgfpoint{\pgf@circ@res@left}{0pt}}
        \pgfpathlineto{#2}
        \pgfsetroundcap\pgfusepath{draw}
        \endpgfscope
        \pgfscope % arrow
        #3
        \endpgfscope
        \pgftransformshift{\pgfpoint{\pgf@circ@res@left}{0pt}}
        \pgfnode{\cshape}{center}{}{\thisshape-in}{\pgfusepath{draw}}
        \pgftransformshift{\pgfpoint{2\pgf@circ@res@right}{0pt}}
        \pgfnode{\cshape}{center}{}{\thisshape-out}{\pgfusepath{draw}}
    }
}

%% closed cute switch
\pgfcircdeclarecutesw{cuteclosedswitch}
    {\pgfpoint{\pgf@circ@res@right}{1.5\pgf@circ@res@temp}}
    {}

%% open cute switch
\pgfcircdeclarecutesw{cuteopenswitch}
    {\pgfpoint{\pgf@circ@res@right}{\pgf@circ@res@up}}
    {}

%% closing cute switch
\pgfcircdeclarecutesw{cuteclosingswitch}
    {\pgfpoint{\pgf@circ@res@right}{\pgf@circ@res@up}}
    {
    \pgf@circ@setlinewidth{bipoles}{\pgfstartlinewidth}
    \pgftransformshift{\pgfpoint{\pgf@circ@res@left}{0pt}} % in node
    \pgfpathmoveto{\pgfpointpolar{70}{1.2\pgf@circ@res@right}}
    \pgfpatharc{70}{-10}{1.2\pgf@circ@res@right}
    \pgfsetarrowsend{latexslim}
    \pgfusepath{draw}
    }

%% opening cute switch
\pgfcircdeclarecutesw{cuteopeningswitch}
    {\pgfpoint{\pgf@circ@res@right}{1.5\pgf@circ@res@temp}}
    {
    \pgf@circ@setlinewidth{bipoles}{\pgfstartlinewidth}
    \pgftransformshift{\pgfpoint{\pgf@circ@res@left}{0pt}} % in node
    \pgfpathmoveto{\pgfpointpolar{-10}{1.2\pgf@circ@res@right}}
    \pgfpatharc{-10}{60}{1.2\pgf@circ@res@right}
    \pgfsetarrowsend{latexslim}
    \pgfusepath{draw}
    }

%%%%%%%%%%%%%%%%%
%% Instruments
%%%%%%%%%%%%%%%%%

% % METERINGSHAPE
\long\def\drawmeteringcircle{
    \def\pgf@circ@temp{right}
    \ifx\tikz@res@label@pos\pgf@circ@temp
        \pgf@circ@res@step=-1.2\pgf@circ@res@up
    \else
        \def\pgf@circ@temp{below}
        \ifx\tikz@res@label@pos\pgf@circ@temp
            \pgf@circ@res@step=-1.2\pgf@circ@res@up
        \else
            \pgf@circ@res@step=1.2\pgf@circ@res@up
        \fi
    \fi
    %draw connections to circle
    \pgfpathmoveto{\pgfpoint{\pgf@circ@res@left}{\pgf@circ@res@zero}}
    \pgfpathlineto{\pgfpoint{.9\pgf@circ@res@left}{\pgf@circ@res@zero}}
    \pgfpathmoveto{\pgfpoint{.9\pgf@circ@res@right}{\pgf@circ@res@zero}}
    \pgfpathlineto{\pgfpoint{\pgf@circ@res@right}{\pgf@circ@res@zero}}
    \pgfusepath{draw}
    %draw circle
    \pgfscope
        \pgf@circ@setlinewidth{bipoles}{\pgfstartlinewidth}
        \pgfpathcircle{\pgfpointorigin}{.9\pgf@circ@res@up}
        \pgf@circ@draworfill
    \endpgfscope
    %draw arrow
    \pgfscope
        \pgfsetarrowsend{latex}
        \pgfpathmoveto{\pgfpoint{\pgf@circ@res@left}{\pgf@circ@res@down}}
        \pgfpathlineto{\pgfpoint{-\pgf@circ@res@left}{\pgf@circ@res@up}}
        \pgfusepath{draw}
    \endpgfscope
}

%AMPEREMETER
\pgfcircdeclarebipolescaled{instruments}
{}
{\ctikzvalof{bipoles/ammeter/height}}
{ammeter}
{\ctikzvalof{bipoles/ammeter/height}}
{\ctikzvalof{bipoles/ammeter/width}}
{
    \drawmeteringcircle
    \pgfnode{circle}{center}{\pgf@circ@font@bold{A}}{}{}
}
%OHMMETER
\pgfcircdeclarebipolescaled{instruments}
{}
{\ctikzvalof{bipoles/ohmmeter/height}}
{ohmmeter}
{\ctikzvalof{bipoles/ohmmeter/height}}
{\ctikzvalof{bipoles/ohmmeter/width}}
{
    \drawmeteringcircle
    \pgfnode{circle}{center}{\boldmath$\Omega$}{}{}
}
%VOLTMETER
\pgfcircdeclarebipolescaled{instruments}
{}
{\ctikzvalof{bipoles/voltmeter/height}}
{voltmeter}
{\ctikzvalof{bipoles/voltmeter/height}}
{\ctikzvalof{bipoles/voltmeter/width}}
{
    \drawmeteringcircle
    \pgfnode{circle}{center}{\pgf@circ@font@bold{V}}{}{}

}

% oscilloscope, suggested by @nobrl https://github.com/circuitikz/circuitikz/issues/176
\pgfcircdeclarebipolescaled{instruments}
{
    \anchor{in 1}{\southwest\pgf@y=0.75\pgf@y\pgf@x=0.4\pgf@x}
    \anchor{in 2}{\southwest\pgf@y=0.75\pgf@y\pgf@x=-0.4\pgf@x}
    % put the node text above and centered
    \anchor{text}{\pgfextracty{\pgf@circ@res@up}{\northeast}
        \pgfpoint{-.5\wd\pgfnodeparttextbox}{
            \dimexpr.5\dp\pgfnodeparttextbox+.5\ht\pgfnodeparttextbox+\pgf@circ@res@up\relax
        }
    }
}
{\ctikzvalof{bipoles/oscope/height}}
{oscope}
{\ctikzvalof{bipoles/oscope/height}}
{\ctikzvalof{bipoles/oscope/width}}
{
    \pgf@circ@setlinewidth{bipoles}{\pgfstartlinewidth}
    \pgfextracty{\pgf@circ@res@up}{\northeast}
    \pgfextractx{\pgf@circ@res@right}{\northeast}
    \pgfextractx{\pgf@circ@res@left}{\southwest}
    \pgfextracty{\pgf@circ@res@down}{\southwest}
    \pgfmathsetlength{\pgf@circ@res@step}{0.25*\pgf@circ@res@up}
    \pgfscope
        \pgfsetcornersarced{\pgfpoint{\pgf@circ@res@step}{\pgf@circ@res@step}}
        \pgfpathrectanglecorners{\pgfpoint{\pgf@circ@res@left}{\pgf@circ@res@down}}{\pgfpoint{\pgf@circ@res@right}{\pgf@circ@res@up}}
        % this would create a round (analog?) scope...
        % \pgfpathellipse{\pgfpointorigin}{\pgfpoint{0}{\pgf@circ@res@up}}{\pgfpoint{\pgf@circ@res@left}{0}}
        \pgf@circ@draworfill
    \endpgfscope
    % get the rotation
    \ifpgf@circuit@straightinstruments
        \pgfgettransformentries\a\b\temp\temp\temp\temp
        \pgfmathsetmacro{\rot}{-atan2(\b,\a)}
    \else
        \edef\rot{0}
    \fi
    % and unrotate the scope
    \pgfscope
        \pgftransformrotate{\rot}
        % grid
        \pgfscope
            \pgfsetlinewidth{0.5\pgfstartlinewidth}
            \pgfpathmoveto{\pgfpoint{0.75\pgf@circ@res@left}{0.25\pgf@circ@res@down}}
            \pgfpathgrid[stepx=\pgf@circ@res@step, stepy=\pgf@circ@res@step]%
            {\pgfpoint{0.75\pgf@circ@res@left}{0.5\pgf@circ@res@down}}
            {\pgfpoint{0.75\pgf@circ@res@right}{0.5\pgf@circ@res@up}}
            \pgfsetstrokeopacity{0.5}
            \pgfusepath{draw}
        \endpgfscope
        % function displayed, thanks to
        \pgf@circ@setlinewidth{bipoles}{\pgfstartlinewidth}
        \pgfpathmoveto{\pgfpoint{0.75\pgf@circ@res@left}{0.25\pgf@circ@res@down}}
        \pgfpathlineto{\pgfpoint{0.05\pgf@circ@res@left}{0.25\pgf@circ@res@up}}
        \pgfpathlineto{\pgfpoint{0.05\pgf@circ@res@left}{0.25\pgf@circ@res@down}}
        \pgfpathlineto{\pgfpoint{0.65\pgf@circ@res@right}{0.25\pgf@circ@res@up}}
        \pgfpathlineto{\pgfpoint{0.65\pgf@circ@res@right}{0.25\pgf@circ@res@down}}
        \pgfusepath{draw}
    \endpgfscope
}

% generic round meter with always horizontal text, no arrow
\pgfcircdeclarebipolescaled{instruments}
{}
{\ctikzvalof{bipoles/esource/height}}
{rmeter}
{\ctikzvalof{bipoles/esource/height}}
{\ctikzvalof{bipoles/esource/width}}
{
    \pgfpointorigin
    \pgf@circ@setlinewidth{bipoles}{\pgfstartlinewidth}
    \pgfpathellipse{\pgfpointorigin}{\pgfpoint{0}{\pgf@circ@res@up}}{\pgfpoint{\pgf@circ@res@left}{0}}
    \pgf@circ@draworfill
    % draw the text label
    % get the rotation
    \ifpgf@circuit@straightinstruments
        \pgfgettransformentries\a\b\temp\temp\temp\temp
        \pgfmathsetmacro{\rot}{-atan2(\b,\a)}
    \else
        \edef\rot{0}
    \fi
    % and unrotate the scope
    \pgfscope
        \pgfsetcolor{\ctikzvalof{color}}
        \pgftransformrotate{\rot}
        \pgftext[center,x=0,y=0]{\ctikzvalof{bipoles/twoport/text}}
    \endpgfscope
}

% generic round meter with always horizontal text, with arrow
\pgfcircdeclarebipolescaled{instruments}
{}
{\ctikzvalof{bipoles/esource/height}}
{rmeterwa}
{\ctikzvalof{bipoles/esource/height}}
{\ctikzvalof{bipoles/esource/width}}
{
    \pgfpointorigin
    \pgf@circ@setlinewidth{bipoles}{\pgfstartlinewidth}
    \pgfpathellipse{\pgfpointorigin}{\pgfpoint{0}{\pgf@circ@res@up}}{\pgfpoint{\pgf@circ@res@left}{0}}
    \pgf@circ@draworfill
    % draw the text label
    % get the rotation
    \ifpgf@circuit@straightinstruments
        \pgfgettransformentries\a\b\temp\temp\temp\temp
        \pgfmathsetmacro{\rot}{-atan2(\b,\a)}
    \else
        \edef\rot{0}
    \fi
    % and unrotate the scope
    \pgfscope
        \pgftransformrotate{\rot}
        \pgfsetcolor{\ctikzvalof{color}}
        \pgfsetlinewidth{\pgfstartlinewidth}
        % arrow: create  a center hole to have better visual
        \pgfscope
            % clipping path: first a rectangle bigger then the shape
            % to avoid problems with the line thickness
            \pgfpathrectanglecorners{\pgfpoint{2\pgf@circ@res@left}{2\pgf@circ@res@down}}
            {\pgfpoint{2\pgf@circ@res@right}{2\pgf@circ@res@up}}
            % next open a circle into it
            \pgfpathellipse{\pgfpointorigin}{\pgfpoint{0}{.6\pgf@circ@res@up}}{\pgfpoint{.6\pgf@circ@res@left}{0}}
            % do the difference and clip before drawing
            \pgfseteorule
            \pgfusepath{clip}
            \pgfsetarrowsend{latexslim}
            % the arrow is better if it has a bit of breath and it's not 45º
            \pgfpathmoveto{\pgfpoint{.8\pgf@circ@res@left}{1.2\pgf@circ@res@down}}
            \pgfpathlineto{\pgfpoint{.8\pgf@circ@res@right}{1.2\pgf@circ@res@up}}
            \pgfusepath{draw}
        \endpgfscope
        \pgftext[center]{\ctikzvalof{bipoles/twoport/text}}
    \endpgfscope
}

% generic square meter with always horizontal text
\pgfcircdeclarebipolescaled{instruments}
{
    \anchor{in 1}{\southwest\pgf@y=0.75\pgf@y\pgf@x=0.4\pgf@x}
    \anchor{in 2}{\southwest\pgf@y=0.75\pgf@y\pgf@x=-0.4\pgf@x}
    % put the node text above and centered
    \anchor{text}{\pgfextracty{\pgf@circ@res@up}{\northeast}
        \pgfpoint{-.5\wd\pgfnodeparttextbox}{
            \dimexpr.5\dp\pgfnodeparttextbox+.5\ht\pgfnodeparttextbox+\pgf@circ@res@up\relax
        }
    }
}
{\ctikzvalof{bipoles/smeter/height}}
{smeter}
{\ctikzvalof{bipoles/smeter/height}}
{\ctikzvalof{bipoles/smeter/width}}
{
    \pgf@circ@setlinewidth{bipoles}{\pgfstartlinewidth}
    \pgfextracty{\pgf@circ@res@up}{\northeast}
    \pgfextractx{\pgf@circ@res@right}{\northeast}
    \pgfextractx{\pgf@circ@res@left}{\southwest}
    \pgfextracty{\pgf@circ@res@down}{\southwest}
    \pgfmathsetlength{\pgf@circ@res@step}{0.25*\pgf@circ@res@up}
    \pgfscope
        \pgfsetcornersarced{\pgfpoint{\pgf@circ@res@step}{\pgf@circ@res@step}}
        \pgfpathrectanglecorners{\pgfpoint{\pgf@circ@res@left}{\pgf@circ@res@down}}{\pgfpoint{\pgf@circ@res@right}{\pgf@circ@res@up}}
        \pgf@circ@draworfill
    \endpgfscope
    % get the rotation
    \ifpgf@circuit@straightinstruments
        \pgfgettransformentries\a\b\temp\temp\temp\temp
        \pgfmathsetmacro{\rot}{-atan2(\b,\a)}
    \else
        \edef\rot{0}
    \fi
    % and unrotate the scope
    \pgfscope
        \pgftransformrotate{\rot}
        % the metering window
        \pgfscope
            \def\@starta{105}\def\@stopa{75}
            \pgfsetlinewidth{\pgfstartlinewidth}
            \pgftransformshift{\pgfpoint{0pt}{-1.8\pgf@circ@res@up}}
            \pgfpathmoveto{\pgfpointpolar{\@starta}{2\pgf@circ@res@up}}
            \pgfpatharc{\@starta}{\@stopa}{2\pgf@circ@res@up}
            \pgfpathlineto{\pgfpointpolar{\@stopa}{2.5\pgf@circ@res@up}}
            \pgfpatharc{\@stopa}{\@starta}{2.5\pgf@circ@res@up}
            \pgfclosepath
            \pgfpathmoveto{\pgfpointpolar{80}{2\pgf@circ@res@up}}
            \pgfpathlineto{\pgfpointpolar{80}{2.4\pgf@circ@res@up}}
            \pgfusepath{draw}
        \endpgfscope
        \pgftext[center, y=0.5\pgf@circ@res@down]{\ctikzvalof{bipoles/twoport/text}}
    \endpgfscope
}

% probes qucs style:
% #1 : name
% #2 : extra code
\long\def\pgfcirc@qucsprobe#1#2{
    \pgfcircdeclarebipolescaled{instruments}
    {
        \anchor{v+}{\southwest\pgf@x=0.6\pgf@x}
        \anchor{v-}{\southwest\pgf@x=-0.6\pgf@x}
        % put the node text above and centered
        \anchor{text}{\pgfextracty{\pgf@circ@res@up}{\northeast}
            \pgfpoint{-.5\wd\pgfnodeparttextbox}{
                \dimexpr.5\dp\pgfnodeparttextbox+.5\ht\pgfnodeparttextbox+\pgf@circ@res@up\relax
            }
        }
    }
    {\ctikzvalof{bipoles/qmeter/depth}}
    {#1}
    {\ctikzvalof{bipoles/qmeter/height}}
    {\ctikzvalof{bipoles/qmeter/width}}
    {
        \pgf@circ@setlinewidth{bipoles}{\pgfstartlinewidth}
        \pgfextracty{\pgf@circ@res@up}{\northeast}
        \pgfextractx{\pgf@circ@res@right}{\northeast}
        \pgfextractx{\pgf@circ@res@left}{\southwest}
        \pgfextracty{\pgf@circ@res@down}{\southwest}
        \pgfmathsetlength{\pgf@circ@res@step}{0.25*\pgf@circ@res@up}
        \pgfscope
            \pgfscope
                \pgfsetcornersarced{\pgfpoint{\pgf@circ@res@step}{\pgf@circ@res@step}}
                \pgfpathrectanglecorners{\pgfpoint{\pgf@circ@res@left}{\pgf@circ@res@down}}{\pgfpoint{\pgf@circ@res@right}{\pgf@circ@res@up}}
            \pgf@circ@draworfill
            \endpgfscope
            \def\@starta{103}\def\@stopa{77}
            \pgfsetlinewidth{\pgfstartlinewidth}
            \pgfscope
                \pgftransformshift{\pgfpoint{0pt}{-1.7\pgf@circ@res@up}}
                \pgfpathmoveto{\pgfpointpolar{\@starta}{2.1\pgf@circ@res@up}}
                \pgfpatharc{\@starta}{\@stopa}{2.1\pgf@circ@res@up}
                \pgfpathlineto{\pgfpointpolar{\@stopa}{2.5\pgf@circ@res@up}}
                \pgfpatharc{\@stopa}{\@starta}{2.5\pgf@circ@res@up}
                \pgfclosepath
                \pgfpathmoveto{\pgfpointpolar{83}{2.1\pgf@circ@res@up}}
                \pgfpathlineto{\pgfpointpolar{83}{2.4\pgf@circ@res@up}}
                \pgfusepath{draw}
                \pgf@circ@draworfill
            \endpgfscope
            #2
        \endpgfscope
    }
}

\pgfcirc@qucsprobe{qiprobe}{
    \pgfpathmoveto{\pgfpoint{\pgf@circ@res@left}{0pt}}
    \pgfpathlineto{\pgfpoint{\pgf@circ@res@right}{0pt}}
    \pgfusepath{draw}
    \pgfnode{currarrow}{center}{}{}{}
}

\pgfcirc@qucsprobe{qvprobe}{
    \pgfmathsetlength{\pgf@circ@res@other}{\ctikzvalof{nodes width}*\pgf@circ@scaled@Rlen}
    \pgfpathmoveto{\pgfpoint{\pgf@circ@res@left}{0pt}}
    \pgfpathlineto{\pgfpoint{.6\pgf@circ@res@left}{0pt}}
    \pgfpathcircle{\pgfpoint{.6\pgf@circ@res@left+\pgf@circ@res@other}{0pt}}{\pgf@circ@res@other}
    \pgfpathmoveto{\pgfpoint{.6\pgf@circ@res@right}{0pt}}
    \pgfpathlineto{\pgfpoint{\pgf@circ@res@right}{0pt}}
    \pgfpathcircle{\pgfpoint{.6\pgf@circ@res@right-\pgf@circ@res@other}{0pt}}{\pgf@circ@res@other}
    \pgfusepath{draw}
    \pgfscope
        % "+" and "-", drawn so that they scale correctly
        \pgfsetlinewidth{2\pgfstartlinewidth}
        \pgfpathmoveto{\pgfpoint{.6\pgf@circ@res@left+\pgf@circ@res@other}{-1.5\pgf@circ@res@other}}
        \pgfpathlineto{\pgfpoint{.6\pgf@circ@res@left+\pgf@circ@res@other}{-3.5\pgf@circ@res@other}}
        \pgfpathmoveto{\pgfpoint{.6\pgf@circ@res@left+0\pgf@circ@res@other}{-2.5\pgf@circ@res@other}}
        \pgfpathlineto{\pgfpoint{.6\pgf@circ@res@left+2\pgf@circ@res@other}{-2.5\pgf@circ@res@other}}
        \pgfpathmoveto{\pgfpoint{.6\pgf@circ@res@right+0\pgf@circ@res@other}{-2.5\pgf@circ@res@other}}
        \pgfpathlineto{\pgfpoint{.6\pgf@circ@res@right-2\pgf@circ@res@other}{-2.5\pgf@circ@res@other}}
        \pgfusepath{draw}
    \endpgfscope
}

\pgfcirc@qucsprobe{qpprobe}{
    \pgfpathmoveto{\pgfpoint{\pgf@circ@res@left}{0pt}}
    \pgfpathlineto{\pgfpoint{\pgf@circ@res@right}{0pt}}
    \pgfusepath{draw}
    \pgfnode{currarrow}{center}{}{}{}
    \pgfmathsetlength{\pgf@circ@res@other}{\ctikzvalof{nodes width}*\pgf@circ@scaled@Rlen}
    \pgfpathcircle{\pgfpoint{.6\pgf@circ@res@left}{-3\pgf@circ@res@other}}{\pgf@circ@res@other}
    \pgfpathmoveto{\pgfpoint{.6\pgf@circ@res@left}{-4\pgf@circ@res@other}}
    \pgfpathlineto{\pgfpoint{.6\pgf@circ@res@left}{\pgf@circ@res@down}}
    \pgfpathcircle{\pgfpoint{.6\pgf@circ@res@right}{-3\pgf@circ@res@other}}{\pgf@circ@res@other}
    \pgfpathmoveto{\pgfpoint{.6\pgf@circ@res@right}{-4\pgf@circ@res@other}}
    \pgfpathlineto{\pgfpoint{.6\pgf@circ@res@right}{\pgf@circ@res@down}}
    \pgfusepath{draw}
    \pgfscope
        % "+" and "-", drawn so that they scale correctly
        \pgfsetlinewidth{2\pgfstartlinewidth}
        \pgfpathmoveto{\pgfpoint{.6\pgf@circ@res@left+3\pgf@circ@res@other}{-2\pgf@circ@res@other}}
        \pgfpathlineto{\pgfpoint{.6\pgf@circ@res@left+3\pgf@circ@res@other}{-4\pgf@circ@res@other}}
        \pgfpathmoveto{\pgfpoint{.6\pgf@circ@res@left+2\pgf@circ@res@other}{-3\pgf@circ@res@other}}
        \pgfpathlineto{\pgfpoint{.6\pgf@circ@res@left+4\pgf@circ@res@other}{-3\pgf@circ@res@other}}
        \pgfpathmoveto{\pgfpoint{.6\pgf@circ@res@right-4\pgf@circ@res@other}{-3\pgf@circ@res@other}}
        \pgfpathlineto{\pgfpoint{.6\pgf@circ@res@right-2\pgf@circ@res@other}{-3\pgf@circ@res@other}}
        \pgfusepath{draw}
    \endpgfscope
}

% current loop for oscope and similar: stylized
\pgfcircdeclarebipolescaled{instruments}
{
    \anchor{i}{\northeast\pgf@x=0pt\relax}
    \anchor{text}{
        \pgfextracty{\pgf@circ@res@up}{\northeast}
        \pgfextractx{\pgf@circ@res@left}{\northeast}
        \pgfpoint{-.5\wd\pgfnodeparttextbox+\pgf@circ@res@left}{
            \dimexpr.5\dp\pgfnodeparttextbox+.5\ht\pgfnodeparttextbox+\pgf@circ@res@up\relax
        }
    }
}
{\ctikzvalof{bipoles/iloop/height}}
{iloop}
{\ctikzvalof{bipoles/iloop/height}}
{\ctikzvalof{bipoles/iloop/width}}
{
    \pgfextracty{\pgf@circ@res@up}{\northeast}
    \pgfextractx{\pgf@circ@res@right}{\northeast}
    \pgf@circ@res@down=-\pgf@circ@res@up
    \pgf@circ@res@left=-\pgf@circ@res@right
    \pgfscope
        \pgfstartlinewidth=\pgflinewidth
        \pgfsetcolor{\ctikzvalof{color}}
        % external ellipse
        \pgfscope
            \pgf@circ@setlinewidth{bipoles}{\pgflinewidth}
            % clipping path: first a rectangle bigger then the shape
            % to avoid problems with the line thickness
            \pgfpathrectanglecorners{\pgfpoint{2\pgf@circ@res@left}{2\pgf@circ@res@down}}
            {\pgfpoint{2\pgf@circ@res@right}{2\pgf@circ@res@up}}
            % next the opening to the left
            \pgfpathrectanglecorners{\pgfpoint{2\pgf@circ@res@left}{.1\pgf@circ@res@down}}
            {\pgfpoint{0pt}{.1\pgf@circ@res@up}}
            % do the difference and clip before drawing
            \pgfseteorule
            \pgfusepath{clip}
            \pgfpathellipse{\pgfpointorigin}{
                \pgfpoint{0pt}{0.8\pgf@circ@res@up}}{
            \pgfpoint{0.4\pgf@circ@res@right}{0pt}}
            \pgfusepath{draw}
        \endpgfscope
        % internal wire
        \pgfsetlinewidth{\pgfstartlinewidth}
        \pgfpathmoveto{\pgfpoint{\pgf@circ@res@left}{0pt}}
        \pgfpathlineto{\pgfpoint{0.3\pgf@circ@res@right}{0pt}}
        \pgfpathmoveto{\pgfpoint{0.5\pgf@circ@res@right}{0pt}}
        \pgfpathlineto{\pgfpoint{\pgf@circ@res@right}{0pt}}
        % and the contact line up
        \pgfpathmoveto{\pgfpoint{0pt}{0.8\pgf@circ@res@up}}
        \pgfpathlineto{\pgfpoint{0pt}{\pgf@circ@res@up}}
        \pgfusepath{draw}
    \endpgfscope
}

% current loop for oscope and similar: real (double connection)
\pgfcircdeclarebipolescaled{instruments}
{
    \anchor{i+}{
        \pgfextracty{\pgf@circ@res@up}{\northeast}
        \pgfextractx{\pgf@circ@res@right}{\northeast}
        \pgf@circ@res@step=0.4\pgf@circ@res@right
        \pgf@circ@res@other=0.8\pgf@circ@res@up
        \pgfpointpolar{105}{\pgf@circ@res@step and \pgf@circ@res@other}
        \pgf@y=\pgf@circ@res@up
    }
    \anchor{i-}{
        \pgfextracty{\pgf@circ@res@up}{\northeast}
        \pgfextractx{\pgf@circ@res@right}{\northeast}
        \pgf@circ@res@step=0.4\pgf@circ@res@right
        \pgf@circ@res@other=0.8\pgf@circ@res@up
        \pgfpointpolar{75}{\pgf@circ@res@step and \pgf@circ@res@other}
        \pgf@y=\pgf@circ@res@up
    }
    \anchor{text}{
        \pgfextracty{\pgf@circ@res@up}{\northeast}
        \pgfextractx{\pgf@circ@res@left}{\northeast}
        \pgfpoint{-.5\wd\pgfnodeparttextbox+\pgf@circ@res@left}{
            \dimexpr.5\dp\pgfnodeparttextbox+.5\ht\pgfnodeparttextbox+\pgf@circ@res@up\relax
        }
    }
}
{\ctikzvalof{bipoles/iloop/height}}
{iloop2}
{\ctikzvalof{bipoles/iloop/height}}
{\ctikzvalof{bipoles/iloop/width}}
{
    \pgfextracty{\pgf@circ@res@up}{\northeast}
    \pgfextractx{\pgf@circ@res@right}{\northeast}
    \pgf@circ@res@down=-\pgf@circ@res@up
    \pgf@circ@res@left=-\pgf@circ@res@right
    % must be the same than internal i+ and i- anchors definition
    \pgf@circ@res@step=0.4\pgf@circ@res@right
    \pgf@circ@res@other=0.8\pgf@circ@res@up
    \def\@plus{\pgfpointpolar{105}{\pgf@circ@res@step and \pgf@circ@res@other}}
    \def\@minus{\pgfpointpolar{75}{\pgf@circ@res@step and \pgf@circ@res@other}}
    \pgfscope
        \pgfstartlinewidth=\pgflinewidth
        \pgfsetcolor{\ctikzvalof{color}}
        % external ellipse
        \pgfscope
            \pgf@circ@setlinewidth{bipoles}{\pgflinewidth}
            % clipping path: first a rectangle bigger then the shape
            % to avoid problems with the line thickness
            \pgfpathrectanglecorners{\pgfpoint{2\pgf@circ@res@left}{2\pgf@circ@res@down}}
            {\pgfpoint{2\pgf@circ@res@right}{2\pgf@circ@res@up}}
            % next the opening to the left
            \pgfpathrectanglecorners{\pgfpoint{2\pgf@circ@res@left}{.1\pgf@circ@res@down}}
            {\pgfpoint{0pt}{.1\pgf@circ@res@up}}
            % do the difference and clip before drawing
            \pgfseteorule
            \pgfusepath{clip}
            \pgfpathmoveto{\@plus}
            \pgfpatharc{105}{435}{\pgf@circ@res@step and \pgf@circ@res@other}
            \pgfusepath{draw}
        \endpgfscope
        % internal wire
        \pgfsetlinewidth{\pgfstartlinewidth}
        \pgfpathmoveto{\pgfpoint{\pgf@circ@res@left}{0pt}}
        \pgfpathlineto{\pgfpoint{0.3\pgf@circ@res@right}{0pt}}
        \pgfpathmoveto{\pgfpoint{0.5\pgf@circ@res@right}{0pt}}
        \pgfpathlineto{\pgfpoint{\pgf@circ@res@right}{0pt}}
        % and the contact line up
        % I use ...left and ---right as temporal lengths here to avoid defining more
        \pgfextractx{\pgf@circ@res@left}{\@plus}
        \pgfextractx{\pgf@circ@res@right}{\@minus}
        \pgfpathmoveto{\@plus}
        \pgfpathlineto{\pgfpoint{\pgf@circ@res@left}{\pgf@circ@res@up}}
        \pgfpathmoveto{\@minus}
        \pgfpathlineto{\pgfpoint{\pgf@circ@res@right}{\pgf@circ@res@up}}
        \pgfusepath{draw}
    \endpgfscope
}


%% Varistor
\pgfcircdeclarebipolescaled{resistors}
{}
{\ctikzvalof{bipoles/varistor/height}}
{varistor}
{\ctikzvalof{bipoles/varistor/height}}
{\ctikzvalof{bipoles/varistor/width}}
{
    \pgfscope
        \pgfpathrectanglecorners{\pgfpoint{\pgf@circ@res@left}{\ctikzvalof{bipoles/varistor/main}\pgf@circ@res@up}}{\pgfpoint{\pgf@circ@res@right}{-\ctikzvalof{bipoles/varistor/main}\pgf@circ@res@up}}
        \pgf@circ@setlinewidth{bipoles}{\pgfstartlinewidth}
        \pgf@circ@draworfill
    \endpgfscope

    \pgfpathmoveto{\pgfpoint{-\pgf@circ@res@left}{-\pgf@circ@res@down}}
    \pgfpathlineto{\pgfpoint{.4\pgf@circ@res@left}{\pgf@circ@res@down}}
    \pgfpathlineto{\pgfpoint{\pgf@circ@res@left}{\pgf@circ@res@down}}
    \pgfusepath{draw}

    \pgftext[top,x=.65\pgf@circ@res@left,y=1.2\pgf@circ@res@down]{{\pgf@circ@font@tiny\textsf{U}}}
}

%%%%%%%%%%%%%%
%% RF bipoles
%%%%%%%%%%%%%%

% transmission line
\pgfcircdeclarebipolescaled{RF}
{}
{\ctikzvalof{bipoles/tline/height}}
{tline}
{\ctikzvalof{bipoles/tline/height}}
{\ctikzvalof{bipoles/tline/width}}
{
    \pgf@circ@res@step=.2\pgf@circ@res@right % half x axis
    \begin{pgftransparencygroup}
        \pgf@circ@setlinewidth{bipoles}{\pgfstartlinewidth}
        \pgfpathmoveto{\pgfpoint{\pgf@circ@res@right-\pgf@circ@res@step}{\pgf@circ@res@up}}
        \pgfpathlineto{\pgfpoint{\pgf@circ@res@left+\pgf@circ@res@step}{\pgf@circ@res@up}}
        \pgfpatharc{-90}{90}{-\pgf@circ@res@step and -\pgf@circ@res@up}
        \pgfpathlineto{\pgfpoint{\pgf@circ@res@right-\pgf@circ@res@step}{\pgf@circ@res@down}}
        \pgf@circ@draworfill
        \pgfpathellipse{\pgfpoint{\pgf@circ@res@right-\pgf@circ@res@step}{0pt}}
        {\pgfpoint{\pgf@circ@res@step}{0pt}}{\pgfpoint{0pt}{-\pgf@circ@res@up}}
        \pgf@circ@draworfill
    \end{pgftransparencygroup}
    \pgfsetlinewidth{\pgfstartlinewidth}
    \pgfpathmoveto{\pgfpoint{\pgf@circ@res@right-\pgf@circ@res@step}{0pt}}
    \pgfpathlineto{\pgfpoint{\pgf@circ@res@right}{0pt}}
    \pgfusepath{stroke}
}

% microstrip transmission line
\pgfcircdeclarebipolescaled{RF}
{}
{\ctikzvalof{bipoles/mstline/height}}
{mstline}
{\ctikzvalof{bipoles/mstline/height}}
{\ctikzvalof{bipoles/mstline/width}}
{
    \pgfpathrectanglecorners{\pgfpoint{\pgf@circ@res@left}{\pgf@circ@res@up}}{\pgfpoint{\pgf@circ@res@right}{\pgf@circ@res@down}}
    \pgf@circ@setlinewidth{bipoles}{\pgfstartlinewidth}
    \pgf@circ@draworfill
}

%%%%%%%%%%%%%%%%%%%
%% Block diagrams
%%%%%%%%%%%%%%%%%%%

%% Draw the two-port fillable box
\def\pgf@circ@twoportbox{
    \pgfscope
        \pgf@circ@setlinewidth{bipoles}{\pgfstartlinewidth}
        \pgfpathrectanglecorners{\pgfpoint{\pgf@circ@res@left}{\pgf@circ@res@up}}{\pgfpoint{\pgf@circ@res@right}{\pgf@circ@res@down}}
        \pgf@circ@draworfill
    \endpgfscope
}

%% Generic two port box
\pgfcircdeclarebipolescaled{blocks}
{}
{\ctikzvalof{bipoles/twoport/height}}
{twoport}
{\ctikzvalof{bipoles/twoport/height}}
{\ctikzvalof{bipoles/twoport/width}}
{
    \pgf@circ@res@step = \ctikzvalof{bipoles/twoport/width}\pgf@circ@scaled@Rlen
    \divide \pgf@circ@res@step by 2

    \pgfpathmoveto{\pgfpoint{\pgf@circ@res@left}{\pgf@circ@res@zero}}
    \pgf@circ@res@other = \pgf@circ@res@left
    \advance\pgf@circ@res@other by \pgf@circ@res@step
    \ifpgf@circuit@dashed
        \pgfsetdash{{0.1cm}{0.1cm}}{0cm}
    \fi
    % draw outer box
    \pgf@circ@twoportbox
    \ifpgf@circuit@inputarrow
        {
            \advance \pgf@circ@res@left by -.5\ctikzvalof{bipoles/thickness}\pgfstartlinewidth
            \pgftransformshift{\pgfpoint{\pgf@circ@res@left}{0pt}}
            \pgfnode{inputarrow}{tip}{}{pgf@inputarrow}{\pgfusepath{fill}}
        }
    \fi

    % rotate inner symbol
    \def\pgfcircmathresult{\expandafter\pgf@circ@stripdecimals\pgf@circ@direction\pgf@nil}
    \ifnum \pgfcircmathresult > 45 \ifnum \pgfcircmathresult < 135
        \pgftransformrotate{270}
    \fi\fi
    \ifnum \pgfcircmathresult > 135 \ifnum \pgfcircmathresult < 225
        \pgftransformrotate{180}
    \fi\fi
    \ifnum \pgfcircmathresult > 225 \ifnum \pgfcircmathresult < 315
        \pgftransformrotate{90}
    \fi\fi

    % draw inner symbol
    \pgftext[center,x=0,y=0]{\ctikzvalof{bipoles/twoport/text}}

}

%% voltage controled oscillator
\pgfcircdeclarebipolescaled{blocks}
{}
{\ctikzvalof{bipoles/vco/width}}
{vco}
{\ctikzvalof{bipoles/twoport/width}}
{\ctikzvalof{bipoles/vco/width}}
{
    \pgf@circ@res@step = \ctikzvalof{bipoles/vco/width}\pgf@circ@scaled@Rlen
    \divide \pgf@circ@res@step by 2

    \pgfpathmoveto{\pgfpoint{\pgf@circ@res@left}{\pgf@circ@res@zero}}
    \pgf@circ@res@other = \pgf@circ@res@left
    \advance\pgf@circ@res@other by \pgf@circ@res@step

    \ifpgf@circuit@dashed
        \pgfsetdash{{0.1cm}{0.1cm}}{0cm}
    \fi
    % draw circle
    \pgfscope
        \pgf@circ@setlinewidth{bipoles}{\pgfstartlinewidth}
        \pgfpathcircle{\pgfpoint{0}{0}} {\pgf@circ@res@step}
        \pgf@circ@draworfill
    \endpgfscope
    \ifpgf@circuit@inputarrow
        {
            \advance \pgf@circ@res@left by -.5\ctikzvalof{bipoles/thickness}\pgfstartlinewidth
            \pgftransformshift{\pgfpoint{\pgf@circ@res@left}{0pt}}
            \pgfnode{inputarrow}{tip}{}{pgf@inputarrow}{\pgfusepath{fill}}
        }
    \fi

    % rotate inner symbol
    \def\pgfcircmathresult{\expandafter\pgf@circ@stripdecimals\pgf@circ@direction\pgf@nil}
    \ifnum \pgfcircmathresult > 45 \ifnum \pgfcircmathresult < 135
        \pgftransformrotate{270}
    \fi\fi
    \ifnum \pgfcircmathresult > 134 \ifnum \pgfcircmathresult < 225  % 134 degree, because >= 135 is not possible
        \pgftransformrotate{180}
    \fi\fi
    \ifnum \pgfcircmathresult > 224 \ifnum \pgfcircmathresult < 315
        \pgftransformrotate{90}
    \fi\fi

    % draw inner sine waves
    \pgfsetdash{}{0pt}	% always draw solid line for inner symbol
    \pgfsetlinewidth{\pgfstartlinewidth}
    \pgfpathmoveto{\pgfpoint{-.5\pgf@circ@res@step}{0\pgf@circ@res@step}}
    \pgfpathsine{\pgfpoint{.25\pgf@circ@res@step}{.25\pgf@circ@res@step}}
    \pgfpathcosine{\pgfpoint{.25\pgf@circ@res@step}{-.25\pgf@circ@res@step}}
    \pgfpathsine{\pgfpoint{.25\pgf@circ@res@step}{-.25\pgf@circ@res@step}}
    \pgfpathcosine{\pgfpoint{.25\pgf@circ@res@step}{.25\pgf@circ@res@step}}
    \pgfusepath{draw}
}

%% bandpass filter
\pgfcircdeclarebipolescaled{blocks}
{}
{\ctikzvalof{bipoles/bandpass/width}}
{bandpass}
{\ctikzvalof{bipoles/bandpass/width}}
{\ctikzvalof{bipoles/bandpass/width}}
{

    \pgf@circ@res@step = \ctikzvalof{bipoles/bandpass/width}\pgf@circ@scaled@Rlen
    \divide \pgf@circ@res@step by 2

    \pgfpathmoveto{\pgfpoint{\pgf@circ@res@left}{\pgf@circ@res@zero}}
    \pgf@circ@res@other = \pgf@circ@res@left
    \advance\pgf@circ@res@other by \pgf@circ@res@step

    \ifpgf@circuit@dashed
        \pgfsetdash{{0.1cm}{0.1cm}}{0cm}
    \fi

    % draw outer box
    \pgf@circ@twoportbox

    \ifpgf@circuit@inputarrow
        {
            \advance \pgf@circ@res@left by -.5\ctikzvalof{bipoles/thickness}\pgfstartlinewidth
            \pgftransformshift{\pgfpoint{\pgf@circ@res@left}{0pt}}
            \pgfnode{inputarrow}{tip}{}{pgf@inputarrow}{\pgfusepath{fill}}
        }
    \fi

    % rotate inner symbol
    \def\pgfcircmathresult{\expandafter\pgf@circ@stripdecimals\pgf@circ@direction\pgf@nil}
    \ifnum \pgfcircmathresult > 45 \ifnum \pgfcircmathresult < 135
        \pgftransformrotate{270}
    \fi\fi
    \ifnum \pgfcircmathresult > 134 \ifnum \pgfcircmathresult < 225  % 134 degree, because >= 135 is not possible
        \pgftransformrotate{180}
    \fi\fi
    \ifnum \pgfcircmathresult > 224 \ifnum \pgfcircmathresult < 315
        \pgftransformrotate{90}
    \fi\fi

    % draw inner symbol
    \pgfsetdash{}{0pt}	% always draw solid line for inner symbol
    \pgfsetarrows{-} %never draw arrows
    \pgfsetlinewidth{\pgfstartlinewidth}
    \pgfpathmoveto{\pgfpoint{-0.5\pgf@circ@res@step}{0.5\pgf@circ@res@step}}
    \pgfpathsine{\pgfpoint{.25\pgf@circ@res@step}{.25\pgf@circ@res@step}}
    \pgfpathcosine{\pgfpoint{.25\pgf@circ@res@step}{-.25\pgf@circ@res@step}}
    \pgfpathsine{\pgfpoint{.25\pgf@circ@res@step}{-.25\pgf@circ@res@step}}
    \pgfpathcosine{\pgfpoint{.25\pgf@circ@res@step}{.25\pgf@circ@res@step}}
    \pgfusepath{draw}
    \pgfpathmoveto{\pgfpoint{-0.15\pgf@circ@res@step}{0.35\pgf@circ@res@step}}
    \pgfpathlineto{\pgfpoint{0.15\pgf@circ@res@step}{0.65\pgf@circ@res@step}}
    \pgfusepath{draw}

    \pgfpathmoveto{\pgfpoint{-0.5\pgf@circ@res@step}{0}}
    \pgfpathsine{\pgfpoint{.25\pgf@circ@res@step}{.25\pgf@circ@res@step}}
    \pgfpathcosine{\pgfpoint{.25\pgf@circ@res@step}{-.25\pgf@circ@res@step}}
    \pgfpathsine{\pgfpoint{.25\pgf@circ@res@step}{-.25\pgf@circ@res@step}}
    \pgfpathcosine{\pgfpoint{.25\pgf@circ@res@step}{.25\pgf@circ@res@step}}
    \pgfusepath{draw}

    \pgfpathmoveto{\pgfpoint{-0.5\pgf@circ@res@step}{-0.5\pgf@circ@res@step}}
    \pgfpathsine{\pgfpoint{.25\pgf@circ@res@step}{.25\pgf@circ@res@step}}
    \pgfpathcosine{\pgfpoint{.25\pgf@circ@res@step}{-.25\pgf@circ@res@step}}
    \pgfpathsine{\pgfpoint{.25\pgf@circ@res@step}{-.25\pgf@circ@res@step}}
    \pgfpathcosine{\pgfpoint{.25\pgf@circ@res@step}{.25\pgf@circ@res@step}}
    \pgfusepath{draw}
    \pgfpathmoveto{\pgfpoint{-0.15\pgf@circ@res@step}{-0.65\pgf@circ@res@step}}
    \pgfpathlineto{\pgfpoint{0.15\pgf@circ@res@step}{-0.35\pgf@circ@res@step}}
    \pgfusepath{draw}
}

%% bandstop filter
\pgfcircdeclarebipolescaled{blocks}
{}
{\ctikzvalof{bipoles/bandstop/width}}
{bandstop}
{\ctikzvalof{bipoles/bandstop/width}}
{\ctikzvalof{bipoles/bandstop/width}}
{
    \pgf@circ@res@step = \ctikzvalof{bipoles/bandstop/width}\pgf@circ@scaled@Rlen
    \divide \pgf@circ@res@step by 2

    \pgfpathmoveto{\pgfpoint{\pgf@circ@res@left}{\pgf@circ@res@zero}}
    \pgf@circ@res@other = \pgf@circ@res@left
    \advance\pgf@circ@res@other by \pgf@circ@res@step

    \ifpgf@circuit@dashed
        \pgfsetdash{{0.1cm}{0.1cm}}{0cm}
    \fi

    % draw outer box
    \pgf@circ@twoportbox

    \ifpgf@circuit@inputarrow
        {
            \advance \pgf@circ@res@left by -.5\ctikzvalof{bipoles/thickness}\pgfstartlinewidth
            \pgftransformshift{\pgfpoint{\pgf@circ@res@left}{0pt}}
            \pgfnode{inputarrow}{tip}{}{pgf@inputarrow}{\pgfusepath{fill}}
        }
    \fi

    % rotate inner symbol
    \def\pgfcircmathresult{\expandafter\pgf@circ@stripdecimals\pgf@circ@direction\pgf@nil}
    \ifnum \pgfcircmathresult > 45 \ifnum \pgfcircmathresult < 135
        \pgftransformrotate{270}
    \fi\fi
    \ifnum \pgfcircmathresult > 134 \ifnum \pgfcircmathresult < 225% 134 degree, because >= 135 is not possible
        \pgftransformrotate{180}
    \fi\fi
    \ifnum \pgfcircmathresult > 224 \ifnum \pgfcircmathresult < 315
        \pgftransformrotate{90}
    \fi\fi

    % draw inner symbol
    \pgfsetdash{}{0pt}% always draw solid line for inner symbol
    \pgfsetarrows{-}%never draw arrows
    \pgfsetlinewidth{\pgfstartlinewidth}
    \pgfpathmoveto{\pgfpoint{-0.5\pgf@circ@res@step}{0.5\pgf@circ@res@step}}
    \pgfpathsine{\pgfpoint{.25\pgf@circ@res@step}{.25\pgf@circ@res@step}}
    \pgfpathcosine{\pgfpoint{.25\pgf@circ@res@step}{-.25\pgf@circ@res@step}}
    \pgfpathsine{\pgfpoint{.25\pgf@circ@res@step}{-.25\pgf@circ@res@step}}
    \pgfpathcosine{\pgfpoint{.25\pgf@circ@res@step}{.25\pgf@circ@res@step}}
    \pgfusepath{draw}

    \pgfpathmoveto{\pgfpoint{-0.5\pgf@circ@res@step}{0}}
    \pgfpathsine{\pgfpoint{.25\pgf@circ@res@step}{.25\pgf@circ@res@step}}
    \pgfpathcosine{\pgfpoint{.25\pgf@circ@res@step}{-.25\pgf@circ@res@step}}
    \pgfpathsine{\pgfpoint{.25\pgf@circ@res@step}{-.25\pgf@circ@res@step}}
    \pgfpathcosine{\pgfpoint{.25\pgf@circ@res@step}{.25\pgf@circ@res@step}}
    \pgfusepath{draw}
    \pgfpathmoveto{\pgfpoint{-0.15\pgf@circ@res@step}{-0.15\pgf@circ@res@step}}
    \pgfpathlineto{\pgfpoint{0.15\pgf@circ@res@step}{0.15\pgf@circ@res@step}}
    \pgfusepath{draw}

    \pgfpathmoveto{\pgfpoint{-0.5\pgf@circ@res@step}{-0.5\pgf@circ@res@step}}
    \pgfpathsine{\pgfpoint{.25\pgf@circ@res@step}{.25\pgf@circ@res@step}}
    \pgfpathcosine{\pgfpoint{.25\pgf@circ@res@step}{-.25\pgf@circ@res@step}}
    \pgfpathsine{\pgfpoint{.25\pgf@circ@res@step}{-.25\pgf@circ@res@step}}
    \pgfpathcosine{\pgfpoint{.25\pgf@circ@res@step}{.25\pgf@circ@res@step}}
    \pgfusepath{draw}
}

%% highpass filter
\pgfcircdeclarebipolescaled{blocks}
{}
{\ctikzvalof{bipoles/highpass/width}}
{highpass}
{\ctikzvalof{bipoles/highpass/width}}
{\ctikzvalof{bipoles/highpass/width}}
{
    \pgf@circ@res@step = \ctikzvalof{bipoles/highpass/width}\pgf@circ@scaled@Rlen
    \divide \pgf@circ@res@step by 2

    \pgfpathmoveto{\pgfpoint{\pgf@circ@res@left}{\pgf@circ@res@zero}}
    \pgf@circ@res@other = \pgf@circ@res@left
    \advance\pgf@circ@res@other by \pgf@circ@res@step

    \ifpgf@circuit@dashed
        \pgfsetdash{{0.1cm}{0.1cm}}{0cm}
    \fi

    % draw outer box
    \pgf@circ@twoportbox

    \ifpgf@circuit@inputarrow
        {
            \advance \pgf@circ@res@left by -.5\ctikzvalof{bipoles/thickness}\pgfstartlinewidth
            \pgftransformshift{\pgfpoint{\pgf@circ@res@left}{0pt}}
            \pgfnode{inputarrow}{tip}{}{pgf@inputarrow}{\pgfusepath{fill}}
        }
    \fi

    % rotate inner symbol
    \def\pgfcircmathresult{\expandafter\pgf@circ@stripdecimals\pgf@circ@direction\pgf@nil}
    \ifnum \pgfcircmathresult > 45 \ifnum \pgfcircmathresult < 135
        \pgftransformrotate{270}
    \fi\fi
    \ifnum \pgfcircmathresult > 134 \ifnum \pgfcircmathresult < 225  % 134 degree, because >= 135 is not possible
        \pgftransformrotate{180}
    \fi\fi
    \ifnum \pgfcircmathresult > 224 \ifnum \pgfcircmathresult < 315
        \pgftransformrotate{90}
    \fi\fi

    % draw inner symbol
    \pgfsetdash{}{0pt}	% always draw solid line for inner symbol
    \pgfsetarrows{-} %never draw arrows
    \pgfsetlinewidth{\pgfstartlinewidth}
    \pgfpathmoveto{\pgfpoint{-0.5\pgf@circ@res@step}{0.5\pgf@circ@res@step}}
    \pgfpathsine{\pgfpoint{.25\pgf@circ@res@step}{.25\pgf@circ@res@step}}
    \pgfpathcosine{\pgfpoint{.25\pgf@circ@res@step}{-.25\pgf@circ@res@step}}
    \pgfpathsine{\pgfpoint{.25\pgf@circ@res@step}{-.25\pgf@circ@res@step}}
    \pgfpathcosine{\pgfpoint{.25\pgf@circ@res@step}{.25\pgf@circ@res@step}}
    \pgfusepath{draw}

    \pgfpathmoveto{\pgfpoint{-0.5\pgf@circ@res@step}{0}}
    \pgfpathsine{\pgfpoint{.25\pgf@circ@res@step}{.25\pgf@circ@res@step}}
    \pgfpathcosine{\pgfpoint{.25\pgf@circ@res@step}{-.25\pgf@circ@res@step}}
    \pgfpathsine{\pgfpoint{.25\pgf@circ@res@step}{-.25\pgf@circ@res@step}}
    \pgfpathcosine{\pgfpoint{.25\pgf@circ@res@step}{.25\pgf@circ@res@step}}
    \pgfusepath{draw}
    \pgfpathmoveto{\pgfpoint{-0.15\pgf@circ@res@step}{-0.15\pgf@circ@res@step}}
    \pgfpathlineto{\pgfpoint{0.15\pgf@circ@res@step}{0.15\pgf@circ@res@step}}
    \pgfusepath{draw}

    \pgfpathmoveto{\pgfpoint{-0.5\pgf@circ@res@step}{-0.5\pgf@circ@res@step}}
    \pgfpathsine{\pgfpoint{.25\pgf@circ@res@step}{.25\pgf@circ@res@step}}
    \pgfpathcosine{\pgfpoint{.25\pgf@circ@res@step}{-.25\pgf@circ@res@step}}
    \pgfpathsine{\pgfpoint{.25\pgf@circ@res@step}{-.25\pgf@circ@res@step}}
    \pgfpathcosine{\pgfpoint{.25\pgf@circ@res@step}{.25\pgf@circ@res@step}}
    \pgfusepath{draw}
    \pgfpathmoveto{\pgfpoint{-0.15\pgf@circ@res@step}{-0.65\pgf@circ@res@step}}
    \pgfpathlineto{\pgfpoint{0.15\pgf@circ@res@step}{-0.35\pgf@circ@res@step}}
    \pgfusepath{draw}
}

%% lowpass filter
\pgfcircdeclarebipolescaled{blocks}
{}
{\ctikzvalof{bipoles/lowpass/width}}
{lowpass}
{\ctikzvalof{bipoles/lowpass/width}}
{\ctikzvalof{bipoles/lowpass/width}}
{
    \pgf@circ@res@step = \ctikzvalof{bipoles/lowpass/width}\pgf@circ@scaled@Rlen
    \divide \pgf@circ@res@step by 2

    \pgfpathmoveto{\pgfpoint{\pgf@circ@res@left}{\pgf@circ@res@zero}}
    \pgf@circ@res@other = \pgf@circ@res@left
    \advance\pgf@circ@res@other by \pgf@circ@res@step

    \ifpgf@circuit@dashed
        \pgfsetdash{{0.1cm}{0.1cm}}{0cm}
    \fi

    % draw outer box
    \pgf@circ@twoportbox

    \ifpgf@circuit@inputarrow
        {
            \advance \pgf@circ@res@left by -.5\ctikzvalof{bipoles/thickness}\pgfstartlinewidth
            \pgftransformshift{\pgfpoint{\pgf@circ@res@left}{0pt}}
            \pgfnode{inputarrow}{tip}{}{pgf@inputarrow}{\pgfusepath{fill}}
        }
    \fi

    % rotate inner symbol
    \def\pgfcircmathresult{\expandafter\pgf@circ@stripdecimals\pgf@circ@direction\pgf@nil}
    \ifnum \pgfcircmathresult > 45 \ifnum \pgfcircmathresult < 135
        \pgftransformrotate{270}
    \fi\fi
    \ifnum \pgfcircmathresult > 134 \ifnum \pgfcircmathresult < 225  % 134 degree, because >= 135 is not possible
        \pgftransformrotate{180}
    \fi\fi
    \ifnum \pgfcircmathresult > 224 \ifnum \pgfcircmathresult < 315
        \pgftransformrotate{90}
    \fi\fi

    % draw inner symbol
    \pgfsetdash{}{0pt}	% always draw solid line for inner symbol
    \pgfsetarrows{-} %never draw arrows
    \pgfsetlinewidth{\pgfstartlinewidth}
    \pgfpathmoveto{\pgfpoint{-0.5\pgf@circ@res@step}{0.5\pgf@circ@res@step}}
    \pgfpathsine{\pgfpoint{.25\pgf@circ@res@step}{.25\pgf@circ@res@step}}
    \pgfpathcosine{\pgfpoint{.25\pgf@circ@res@step}{-.25\pgf@circ@res@step}}
    \pgfpathsine{\pgfpoint{.25\pgf@circ@res@step}{-.25\pgf@circ@res@step}}
    \pgfpathcosine{\pgfpoint{.25\pgf@circ@res@step}{.25\pgf@circ@res@step}}
    \pgfusepath{draw}
    \pgfpathmoveto{\pgfpoint{-0.15\pgf@circ@res@step}{0.35\pgf@circ@res@step}}
    \pgfpathlineto{\pgfpoint{0.15\pgf@circ@res@step}{0.65\pgf@circ@res@step}}
    \pgfusepath{draw}

    \pgfpathmoveto{\pgfpoint{-0.5\pgf@circ@res@step}{0}}
    \pgfpathsine{\pgfpoint{.25\pgf@circ@res@step}{.25\pgf@circ@res@step}}
    \pgfpathcosine{\pgfpoint{.25\pgf@circ@res@step}{-.25\pgf@circ@res@step}}
    \pgfpathsine{\pgfpoint{.25\pgf@circ@res@step}{-.25\pgf@circ@res@step}}
    \pgfpathcosine{\pgfpoint{.25\pgf@circ@res@step}{.25\pgf@circ@res@step}}
    \pgfusepath{draw}
    \pgfpathmoveto{\pgfpoint{-0.15\pgf@circ@res@step}{-0.15\pgf@circ@res@step}}
    \pgfpathlineto{\pgfpoint{0.15\pgf@circ@res@step}{0.15\pgf@circ@res@step}}
    \pgfusepath{draw}

    \pgfpathmoveto{\pgfpoint{-0.5\pgf@circ@res@step}{-0.5\pgf@circ@res@step}}
    \pgfpathsine{\pgfpoint{.25\pgf@circ@res@step}{.25\pgf@circ@res@step}}
    \pgfpathcosine{\pgfpoint{.25\pgf@circ@res@step}{-.25\pgf@circ@res@step}}
    \pgfpathsine{\pgfpoint{.25\pgf@circ@res@step}{-.25\pgf@circ@res@step}}
    \pgfpathcosine{\pgfpoint{.25\pgf@circ@res@step}{.25\pgf@circ@res@step}}
    \pgfusepath{draw}
}

%% ADC
\pgfcircdeclarebipolescaled{blocks}
{}
{\ctikzvalof{bipoles/adc/width}}
{adc}
{\ctikzvalof{bipoles/adc/width}}
{\ctikzvalof{bipoles/adc/width}}
{
    \pgf@circ@res@step = \ctikzvalof{bipoles/adc/width}\pgf@circ@scaled@Rlen
    \divide \pgf@circ@res@step by 2

    \pgfpathmoveto{\pgfpoint{\pgf@circ@res@left}{\pgf@circ@res@zero}}
    \pgf@circ@res@other = \pgf@circ@res@left
    \advance\pgf@circ@res@other by \pgf@circ@res@step

    \ifpgf@circuit@dashed
        \pgfsetdash{{0.1cm}{0.1cm}}{0cm}
    \fi

    % draw outer box
    \pgf@circ@twoportbox

    \ifpgf@circuit@inputarrow
        {
            \advance \pgf@circ@res@left by -.5\ctikzvalof{bipoles/thickness}\pgfstartlinewidth
            \pgftransformshift{\pgfpoint{\pgf@circ@res@left}{0pt}}
            \pgfnode{inputarrow}{tip}{}{pgf@inputarrow}{\pgfusepath{fill}}
        }
    \fi

    % rotate inner symbol
    \def\texti{A}
    \def\textii{D}
    \def\pgfcircmathresult{\expandafter\pgf@circ@stripdecimals\pgf@circ@direction\pgf@nil}
    \ifnum \pgfcircmathresult > 45 \ifnum \pgfcircmathresult < 135
        \def\texti{D}
        \def\textii{A}
        \pgftransformrotate{270}
    \fi\fi
    \ifnum \pgfcircmathresult > 134 \ifnum \pgfcircmathresult < 225  % 134 degree, because >= 135 is not possible
        \def\texti{D}
        \def\textii{A}
        \pgftransformrotate{180}
    \fi\fi
    \ifnum \pgfcircmathresult > 225 \ifnum \pgfcircmathresult < 315
        \pgftransformrotate{90}
        \def\texti{A}
        \def\textii{D}
    \fi\fi

    % draw inner symbol
    \pgfsetdash{}{0pt}	% always draw solid line for inner symbol
    \pgfsetarrows{-} %never draw arrows
    \pgfsetlinewidth{\pgfstartlinewidth}
    \pgfpathmoveto{\pgfpoint{\pgf@circ@res@left}{\pgf@circ@res@down}}
    \pgfpathlineto{\pgfpoint{\pgf@circ@res@right}{\pgf@circ@res@up}}
    \pgfusepath{draw}

    \pgftext[center,x=0.45\pgf@circ@res@left,y=0.45\pgf@circ@res@up]{\textsf{\texti}}
    \pgftext[center,x=0.45\pgf@circ@res@right,y=0.45\pgf@circ@res@down]{\textsf{\textii}}
}

%% DAC
\pgfcircdeclarebipolescaled{blocks}
{}
{\ctikzvalof{bipoles/dac/width}}
{dac}
{\ctikzvalof{bipoles/dac/width}}
{\ctikzvalof{bipoles/dac/width}}
{
    \pgf@circ@res@step = \ctikzvalof{bipoles/dac/width}\pgf@circ@scaled@Rlen
    \divide \pgf@circ@res@step by 2

    \pgfpathmoveto{\pgfpoint{\pgf@circ@res@left}{\pgf@circ@res@zero}}
    \pgf@circ@res@other = \pgf@circ@res@left
    \advance\pgf@circ@res@other by \pgf@circ@res@step

    \ifpgf@circuit@dashed
        \pgfsetdash{{0.1cm}{0.1cm}}{0cm}
    \fi

    % draw outer box
    \pgf@circ@twoportbox

    \ifpgf@circuit@inputarrow
        {
            \advance \pgf@circ@res@left by -.5\ctikzvalof{bipoles/thickness}\pgfstartlinewidth
            \pgftransformshift{\pgfpoint{\pgf@circ@res@left}{0pt}}
            \pgfnode{inputarrow}{tip}{}{pgf@inputarrow}{\pgfusepath{fill}}
        }
    \fi

    % rotate inner symbol
    \def\texti{D}
    \def\textii{A}
    \def\pgfcircmathresult{\expandafter\pgf@circ@stripdecimals\pgf@circ@direction\pgf@nil}
    \ifnum \pgfcircmathresult > 45 \ifnum \pgfcircmathresult < 135
        \def\texti{A}
        \def\textii{D}
        \pgftransformrotate{270}
    \fi\fi
    \ifnum \pgfcircmathresult > 134 \ifnum \pgfcircmathresult < 225  % 134 degree, because >= 135 is not possible
        \def\texti{A}
        \def\textii{D}
        \pgftransformrotate{180}
    \fi\fi
    \ifnum \pgfcircmathresult > 225 \ifnum \pgfcircmathresult < 315
        \pgftransformrotate{90}
        \def\texti{D}
        \def\textii{A}
    \fi\fi

    % draw inner symbol
    \pgfsetdash{}{0pt}	% always draw solid line for inner symbol
    \pgfsetarrows{-} %never draw arrows
    \pgfsetlinewidth{\pgfstartlinewidth}
    \pgfpathmoveto{\pgfpoint{\pgf@circ@res@left}{\pgf@circ@res@down}}
    \pgfpathlineto{\pgfpoint{\pgf@circ@res@right}{\pgf@circ@res@up}}
    \pgfusepath{draw}

    \pgftext[center,x=0.45\pgf@circ@res@left,y=0.45\pgf@circ@res@up]{\textsf{\texti}}
    \pgftext[center,x=0.45\pgf@circ@res@right,y=0.45\pgf@circ@res@down]{\textsf{\textii}}
}

%% DSP
\pgfcircdeclarebipolescaled{blocks}
{}
{\ctikzvalof{bipoles/dsp/width}}
{dsp}
{\ctikzvalof{bipoles/dsp/width}}
{\ctikzvalof{bipoles/dsp/width}}
{
    \pgf@circ@res@step = \ctikzvalof{bipoles/dsp/width}\pgf@circ@scaled@Rlen
    \divide \pgf@circ@res@step by 2

    \pgfpathmoveto{\pgfpoint{\pgf@circ@res@left}{\pgf@circ@res@zero}}
    \pgf@circ@res@other = \pgf@circ@res@left
    \advance\pgf@circ@res@other by \pgf@circ@res@step

    \ifpgf@circuit@dashed
        \pgfsetdash{{0.1cm}{0.1cm}}{0cm}
    \fi

    % draw outer box
    \pgf@circ@twoportbox

    \ifpgf@circuit@inputarrow
        {
            \advance \pgf@circ@res@left by -.5\ctikzvalof{bipoles/thickness}\pgfstartlinewidth
            \pgftransformshift{\pgfpoint{\pgf@circ@res@left}{0pt}}
            \pgfnode{inputarrow}{tip}{}{pgf@inputarrow}{\pgfusepath{fill}}
        }
    \fi

    % rotate inner symbol
    \def\pgfcircmathresult{\expandafter\pgf@circ@stripdecimals\pgf@circ@direction\pgf@nil}
    \ifnum \pgfcircmathresult > 45 \ifnum \pgfcircmathresult < 135
        \pgftransformrotate{270}
    \fi\fi
    \ifnum \pgfcircmathresult > 135 \ifnum \pgfcircmathresult < 225
        \pgftransformrotate{180}
    \fi\fi
    \ifnum \pgfcircmathresult > 225 \ifnum \pgfcircmathresult < 315
        \pgftransformrotate{90}
    \fi\fi

    % draw inner symbol
    \pgftext[center,x=0,y=0]{\textsf{DSP}}
}

%% FFT
\pgfcircdeclarebipolescaled{blocks}
{}
{\ctikzvalof{bipoles/fft/width}}
{fft}
{\ctikzvalof{bipoles/fft/width}}
{\ctikzvalof{bipoles/fft/width}}
{
    \pgf@circ@res@step = \ctikzvalof{bipoles/fft/width}\pgf@circ@scaled@Rlen
    \divide \pgf@circ@res@step by 2

    \pgfpathmoveto{\pgfpoint{\pgf@circ@res@left}{\pgf@circ@res@zero}}
    \pgf@circ@res@other = \pgf@circ@res@left
    \advance\pgf@circ@res@other by \pgf@circ@res@step

    \ifpgf@circuit@dashed
        \pgfsetdash{{0.1cm}{0.1cm}}{0cm}
    \fi

    % draw outer box
    \pgf@circ@twoportbox

    \ifpgf@circuit@inputarrow
        {
            \advance \pgf@circ@res@left by -.5\ctikzvalof{bipoles/thickness}\pgfstartlinewidth
            \pgftransformshift{\pgfpoint{\pgf@circ@res@left}{0pt}}
            \pgfnode{inputarrow}{tip}{}{pgf@inputarrow}{\pgfusepath{fill}}
        }
    \fi

    % rotate inner symbol
    \def\pgfcircmathresult{\expandafter\pgf@circ@stripdecimals\pgf@circ@direction\pgf@nil}
    \ifnum \pgfcircmathresult > 45 \ifnum \pgfcircmathresult < 135
        \pgftransformrotate{270}
    \fi\fi
    \ifnum \pgfcircmathresult > 135 \ifnum \pgfcircmathresult < 225
        \pgftransformrotate{180}
    \fi\fi
    \ifnum \pgfcircmathresult > 225 \ifnum \pgfcircmathresult < 315
        \pgftransformrotate{90}
    \fi\fi

    % draw inner symbol
    \pgftext[center,x=0,y=0]{\textsf{FFT}}
}

%% Amplifier
\pgfcircdeclarebipolescaled{blocks}
{}
{
    \ifpgf@circuit@boxed
        \ctikzvalof{bipoles/twoport/width}
    \else
        \ctikzvalof{bipoles/amp/width}
    \fi
}
{amp}
{
    \ifpgf@circuit@boxed
        \ctikzvalof{bipoles/twoport/width}
    \else
        \ctikzvalof{bipoles/amp/width}
    \fi
}
{
    \ifpgf@circuit@boxed
        \ctikzvalof{bipoles/twoport/width}
    \else
        \ctikzvalof{bipoles/amp/width}
    \fi
}
{
    \pgf@circ@res@step = \ctikzvalof{bipoles/amp/width}\pgf@circ@scaled@Rlen

    \ifpgf@circuit@dashed
        \pgfsetdash{{0.1cm}{0.1cm}}{0cm}
    \fi

    % draw outer box
    \ifpgf@circuit@boxed
        \pgf@circ@setlinewidth{bipoles}{\pgfstartlinewidth}
        \pgfnode{blockbox}{center}{}{pgf@box}{\pgfusepath{draw}}
        \pgf@circ@draworfill
    \fi

    % draw input arrow
    \ifpgf@circuit@inputarrow
        {
            \advance \pgf@circ@res@left by -.5\ctikzvalof{bipoles/thickness}\pgfstartlinewidth
            \pgftransformshift{\pgfpoint{\pgf@circ@res@left}{0pt}}
            \pgfnode{inputarrow}{tip}{}{pgf@inputarrow}{\pgfusepath{fill}}
        }
    \fi

    \ifpgf@circuit@boxed
        \pgfsetlinewidth{\pgfstartlinewidth}
        \pgfsetdash{}{0pt}	% draw solid line for inner symbol if no box is drawn
        \pgf@circ@res@step=.7\pgf@circ@res@step % scale amp symbol when inside a box
    \else
        \pgf@circ@setlinewidth{bipoles}{\pgfstartlinewidth}
    \fi

    \pgfsetarrows{-} %never draw arrows

    \pgfpathmoveto{\pgfpoint{-0.5\pgf@circ@res@step}{0.55\pgf@circ@res@step}}
    \pgfpathlineto{\pgfpoint{0.5\pgf@circ@res@step}{0}}
    \pgfpathlineto{\pgfpoint{-0.5\pgf@circ@res@step}{-0.55\pgf@circ@res@step}}

    \pgfpathclose
    \pgf@circ@draworfill

    % draw inner text
    \pgftext[center,x=-0.12\pgf@circ@res@step,y=0]{\ctikzvalof{bipoles/twoport/text}}
}


%% variable amplifier
\pgfcircdeclarebipolescaled{blocks}
{}
{
    \ifpgf@circuit@boxed
        \ctikzvalof{bipoles/twoport/width}
    \else
        \ctikzvalof{bipoles/amp/width}
    \fi
}
{vamp}
{
    \ifpgf@circuit@boxed
        \ctikzvalof{bipoles/twoport/width}
    \else
        \ctikzvalof{bipoles/amp/width}
    \fi
}
{
    \ifpgf@circuit@boxed
        \ctikzvalof{bipoles/twoport/width}
    \else
        \ctikzvalof{bipoles/amp/width}
    \fi
}
{
    \pgf@circ@res@step = \ctikzvalof{bipoles/amp/width}\pgf@circ@scaled@Rlen

    \ifpgf@circuit@dashed
        \pgfsetdash{{0.1cm}{0.1cm}}{0cm}
    \fi

    % draw outer box
    \ifpgf@circuit@boxed
        \pgfnode{blockbox}{center}{}{pgf@box}{\pgfusepath{draw}}
    \fi

    % draw input arrow
    \ifpgf@circuit@inputarrow
        {
            \advance \pgf@circ@res@left by -.5\ctikzvalof{bipoles/thickness}\pgfstartlinewidth
            \pgftransformshift{\pgfpoint{\pgf@circ@res@left}{0pt}}
            \pgfnode{inputarrow}{tip}{}{pgf@inputarrow}{\pgfusepath{fill}}
        }
    \fi

    \ifpgf@circuit@boxed
        \pgfsetlinewidth{\pgfstartlinewidth}
        \pgfsetdash{}{0pt}	% draw solid line for inner symbol if no box is drawn
        \pgf@circ@res@step=.7\pgf@circ@res@step % scale amp symbol when inside a box
    \else
        \pgf@circ@setlinewidth{bipoles}{\pgfstartlinewidth}
    \fi


    \pgfsetdash{}{0pt}	% always draw solid line for inner symbol
    \pgfsetarrows{-} %never draw arrows

    \pgfpathmoveto{\pgfpoint{-0.5\pgf@circ@res@step}{0.55\pgf@circ@res@step}}
    \pgfpathlineto{\pgfpoint{0.5\pgf@circ@res@step}{0}}
    \pgfpathlineto{\pgfpoint{-0.5\pgf@circ@res@step}{-0.55\pgf@circ@res@step}}

    \pgfpathclose
    \pgf@circ@draworfill

    % draw inner text
    \pgftext[center,x=-0.12\pgf@circ@res@step,y=0]{\ctikzvalof{bipoles/twoport/text}}

    % draw arrow
    \pgfsetlinewidth{\pgfstartlinewidth}
    \pgfsetarrowsend{latex}
    \pgfpathmoveto{\pgfpoint{-0.8\pgf@circ@res@step}{-0.5\pgf@circ@res@step}}
    \pgfpathlineto{\pgfpoint{0.6\pgf@circ@res@step}{0.6\pgf@circ@res@step}}
    \pgfusepath{draw}
}

%% pi attenuator
\pgfcircdeclarebipolescaled{blocks}
{}
{\ctikzvalof{bipoles/piattenuator/width}}
{piattenuator}
{\ctikzvalof{bipoles/piattenuator/width}}
{\ctikzvalof{bipoles/piattenuator/width}}
{
    \pgf@circ@res@step = \ctikzvalof{bipoles/piattenuator/width}\pgf@circ@scaled@Rlen
    \divide \pgf@circ@res@step by 2

    \pgfpathmoveto{\pgfpoint{\pgf@circ@res@left}{\pgf@circ@res@zero}}
    \pgf@circ@res@other = \pgf@circ@res@left
    \advance\pgf@circ@res@other by \pgf@circ@res@step

    \ifpgf@circuit@dashed
        \pgfsetdash{{0.1cm}{0.1cm}}{0cm}
    \fi

    % draw outer box
    \pgf@circ@twoportbox

    \ifpgf@circuit@inputarrow
        {
            \advance \pgf@circ@res@left by -.5\ctikzvalof{bipoles/thickness}\pgfstartlinewidth
            \pgftransformshift{\pgfpoint{\pgf@circ@res@left}{0pt}}
            \pgfnode{inputarrow}{tip}{}{pgf@inputarrow}{\pgfusepath{fill}}
        }
    \fi

    % rotate inner symbol
    \def\pgfcircmathresult{\expandafter\pgf@circ@stripdecimals\pgf@circ@direction\pgf@nil}
    \ifnum \pgfcircmathresult > 45 \ifnum \pgfcircmathresult < 135
        \pgftransformrotate{270}
    \fi\fi
    \ifnum \pgfcircmathresult > 225 \ifnum \pgfcircmathresult < 315
        \pgftransformrotate{90}
    \fi\fi

    % draw inner symbol
    \pgfsetdash{}{0pt}	% always draw solid line for inner symbol
    \pgfsetarrows{-} %never draw arrows
    \pgfsetlinewidth{\pgfstartlinewidth}
    \pgfpathmoveto{\pgfpoint{0.4\pgf@circ@res@left}{0.4\pgf@circ@res@down}}
    \pgfpathlineto{\pgfpoint{0.4\pgf@circ@res@left}{0.4\pgf@circ@res@up}}
    \pgfusepath{draw}
    \pgfpathmoveto{\pgfpoint{0.4\pgf@circ@res@right}{0.4\pgf@circ@res@down}}
    \pgfpathlineto{\pgfpoint{0.4\pgf@circ@res@right}{0.4\pgf@circ@res@up}}
    \pgfusepath{draw}
    \pgfpathmoveto{\pgfpoint{0.6\pgf@circ@res@left}{0.4\pgf@circ@res@up}}
    \pgfpathlineto{\pgfpoint{0.6\pgf@circ@res@right}{0.4\pgf@circ@res@up}}
    \pgfusepath{draw}
}

%% variable pi attenuator
\pgfcircdeclarebipolescaled{blocks}
{}
{\ctikzvalof{bipoles/vpiattenuator/width}}
{vpiattenuator}
{\ctikzvalof{bipoles/vpiattenuator/width}}
{\ctikzvalof{bipoles/vpiattenuator/width}}
{
    \pgf@circ@res@step = \ctikzvalof{bipoles/vpiattenuator/width}\pgf@circ@scaled@Rlen
    \divide \pgf@circ@res@step by 2

    \pgfpathmoveto{\pgfpoint{\pgf@circ@res@left}{\pgf@circ@res@zero}}
    \pgf@circ@res@other = \pgf@circ@res@left
    \advance\pgf@circ@res@other by \pgf@circ@res@step

    \ifpgf@circuit@dashed
        \pgfsetdash{{0.1cm}{0.1cm}}{0cm}
    \fi

    % draw outer box
    \pgf@circ@twoportbox

    \ifpgf@circuit@inputarrow
        {
            \advance \pgf@circ@res@left by -.5\ctikzvalof{bipoles/thickness}\pgfstartlinewidth
            \pgftransformshift{\pgfpoint{\pgf@circ@res@left}{0pt}}
            \pgfnode{inputarrow}{tip}{}{pgf@inputarrow}{\pgfusepath{fill}}
        }
    \fi

    % rotate inner symbol
    \def\pgfcircmathresult{\expandafter\pgf@circ@stripdecimals\pgf@circ@direction\pgf@nil}
    \ifnum \pgfcircmathresult > 45 \ifnum \pgfcircmathresult < 135
        \pgftransformrotate{270}
    \fi\fi
    \ifnum \pgfcircmathresult > 225 \ifnum \pgfcircmathresult < 315
        \pgftransformrotate{90}
    \fi\fi

    % draw inner symbol
    \pgfsetdash{}{0pt}	% always draw solid line for inner symbol
    \pgfsetarrows{-} %never draw arrows
    \pgfsetlinewidth{\pgfstartlinewidth}
    \pgfpathmoveto{\pgfpoint{0.4\pgf@circ@res@left}{0.4\pgf@circ@res@down}}
    \pgfpathlineto{\pgfpoint{0.4\pgf@circ@res@left}{0.4\pgf@circ@res@up}}
    \pgfusepath{draw}
    \pgfpathmoveto{\pgfpoint{0.4\pgf@circ@res@right}{0.4\pgf@circ@res@down}}
    \pgfpathlineto{\pgfpoint{0.4\pgf@circ@res@right}{0.4\pgf@circ@res@up}}
    \pgfusepath{draw}
    \pgfpathmoveto{\pgfpoint{0.6\pgf@circ@res@left}{0.4\pgf@circ@res@up}}
    \pgfpathlineto{\pgfpoint{0.6\pgf@circ@res@right}{0.4\pgf@circ@res@up}}
    \pgfusepath{draw}

    \pgfsetarrowsend{latex}
    \pgfpathmoveto{\pgfpoint{0.7\pgf@circ@res@left}{0.5\pgf@circ@res@down}}
    \pgfpathlineto{\pgfpoint{0.7\pgf@circ@res@right}{0.8\pgf@circ@res@up}}
    \pgfusepath{draw}
}

%% T attenuator
\pgfcircdeclarebipolescaled{blocks}
{}
{\ctikzvalof{bipoles/tattenuator/width}}
{tattenuator}
{\ctikzvalof{bipoles/tattenuator/width}}
{\ctikzvalof{bipoles/tattenuator/width}}
{
    \pgf@circ@res@step = \ctikzvalof{bipoles/tattenuator/width}\pgf@circ@scaled@Rlen
    \divide \pgf@circ@res@step by 2

    \pgfpathmoveto{\pgfpoint{\pgf@circ@res@left}{\pgf@circ@res@zero}}
    \pgf@circ@res@other = \pgf@circ@res@left
    \advance\pgf@circ@res@other by \pgf@circ@res@step

    \ifpgf@circuit@dashed
        \pgfsetdash{{0.1cm}{0.1cm}}{0cm}
    \fi

    % draw outer box
    \pgf@circ@twoportbox

    \ifpgf@circuit@inputarrow
        {
            \advance \pgf@circ@res@left by -.5\ctikzvalof{bipoles/thickness}\pgfstartlinewidth
            \pgftransformshift{\pgfpoint{\pgf@circ@res@left}{0pt}}
            \pgfnode{inputarrow}{tip}{}{pgf@inputarrow}{\pgfusepath{fill}}
        }
    \fi

    % rotate inner symbol
    \def\pgfcircmathresult{\expandafter\pgf@circ@stripdecimals\pgf@circ@direction\pgf@nil}
    \ifnum \pgfcircmathresult > 45 \ifnum \pgfcircmathresult < 135
        \pgftransformrotate{270}
    \fi\fi
    \ifnum \pgfcircmathresult > 225 \ifnum \pgfcircmathresult < 315
        \pgftransformrotate{90}
    \fi\fi

    % draw inner symbol
    \pgfsetdash{}{0pt}	% always draw solid line for inner symbol
    \pgfsetarrows{-} %never draw arrows
    \pgfsetlinewidth{\pgfstartlinewidth}
    \pgfpathmoveto{\pgfpoint{0pt}{0.4\pgf@circ@res@down}}
    \pgfpathlineto{\pgfpoint{0pt}{0.4\pgf@circ@res@up}}
    \pgfusepath{draw}
    \pgfpathmoveto{\pgfpoint{0.6\pgf@circ@res@left}{0.4\pgf@circ@res@up}}
    \pgfpathlineto{\pgfpoint{0.6\pgf@circ@res@right}{0.4\pgf@circ@res@up}}
    \pgfusepath{draw}
}

%% variable T attenuator
\pgfcircdeclarebipolescaled{blocks}
{}
{\ctikzvalof{bipoles/vtattenuator/width}}
{vtattenuator}
{\ctikzvalof{bipoles/vtattenuator/width}}
{\ctikzvalof{bipoles/vtattenuator/width}}
{
    \pgf@circ@res@step = \ctikzvalof{bipoles/vtattenuator/width}\pgf@circ@scaled@Rlen
    \divide \pgf@circ@res@step by 2

    \pgfpathmoveto{\pgfpoint{\pgf@circ@res@left}{\pgf@circ@res@zero}}
    \pgf@circ@res@other = \pgf@circ@res@left
    \advance\pgf@circ@res@other by \pgf@circ@res@step

    \ifpgf@circuit@dashed
        \pgfsetdash{{0.1cm}{0.1cm}}{0cm}
    \fi

    % draw outer box
    \pgf@circ@twoportbox

    \ifpgf@circuit@inputarrow
        {
            \advance \pgf@circ@res@left by -.5\ctikzvalof{bipoles/thickness}\pgfstartlinewidth
            \pgftransformshift{\pgfpoint{\pgf@circ@res@left}{0pt}}
            \pgfnode{inputarrow}{tip}{}{pgf@inputarrow}{\pgfusepath{fill}}
        }
    \fi

    % rotate inner symbol
    \def\pgfcircmathresult{\expandafter\pgf@circ@stripdecimals\pgf@circ@direction\pgf@nil}
    \ifnum \pgfcircmathresult > 45 \ifnum \pgfcircmathresult < 135
        \pgftransformrotate{270}
    \fi\fi
    \ifnum \pgfcircmathresult > 225 \ifnum \pgfcircmathresult < 315
        \pgftransformrotate{90}
    \fi\fi

    % draw inner symbol
    \pgfsetdash{}{0pt}	% always draw solid line for inner symbol
    \pgfsetarrows{-} %never draw arrows
    \pgfsetlinewidth{\pgfstartlinewidth}
    \pgfpathmoveto{\pgfpoint{0pt}{0.4\pgf@circ@res@down}}
    \pgfpathlineto{\pgfpoint{0pt}{0.4\pgf@circ@res@up}}
    \pgfusepath{draw}
    \pgfpathmoveto{\pgfpoint{0.6\pgf@circ@res@left}{0.4\pgf@circ@res@up}}
    \pgfpathlineto{\pgfpoint{0.6\pgf@circ@res@right}{0.4\pgf@circ@res@up}}
    \pgfusepath{draw}

    \pgfsetarrowsend{latex}
    \pgfpathmoveto{\pgfpoint{0.7\pgf@circ@res@left}{0.5\pgf@circ@res@down}}
    \pgfpathlineto{\pgfpoint{0.7\pgf@circ@res@right}{0.8\pgf@circ@res@up}}
    \pgfusepath{draw}
}

%% phase shifter
\pgfcircdeclarebipolescaled{blocks}
{}
{\ctikzvalof{bipoles/phaseshifter/width}}
{phaseshifter}
{\ctikzvalof{bipoles/phaseshifter/width}}
{\ctikzvalof{bipoles/phaseshifter/width}}
{
    \pgf@circ@res@step = \ctikzvalof{bipoles/phaseshifter/width}\pgf@circ@scaled@Rlen
    \divide \pgf@circ@res@step by 2

    \pgfpathmoveto{\pgfpoint{\pgf@circ@res@left}{\pgf@circ@res@zero}}
    \pgf@circ@res@other = \pgf@circ@res@left
    \advance\pgf@circ@res@other by \pgf@circ@res@step

    \ifpgf@circuit@dashed
        \pgfsetdash{{0.1cm}{0.1cm}}{0cm}
    \fi

    % draw outer box
    \pgf@circ@twoportbox

    \ifpgf@circuit@inputarrow
        {
            \advance \pgf@circ@res@left by -.5\ctikzvalof{bipoles/thickness}\pgfstartlinewidth
            \pgftransformshift{\pgfpoint{\pgf@circ@res@left}{0pt}}
            \pgfnode{inputarrow}{tip}{}{pgf@inputarrow}{\pgfusepath{fill}}
        }
    \fi

    % rotate inner symbol
    \def\pgfcircmathresult{\expandafter\pgf@circ@stripdecimals\pgf@circ@direction\pgf@nil}
    \ifnum \pgfcircmathresult > 45 \ifnum \pgfcircmathresult < 135
        \pgftransformrotate{270}
    \fi\fi
    \ifnum \pgfcircmathresult > 135 \ifnum \pgfcircmathresult < 225
        \pgftransformrotate{180}
    \fi\fi
    \ifnum \pgfcircmathresult > 225 \ifnum \pgfcircmathresult < 315
        \pgftransformrotate{90}
    \fi\fi

    % inner symbol
    \pgftext[center,x=0,y=0]{\Large$\varphi$}
}

%% variable phase shifter
\pgfcircdeclarebipolescaled{blocks}
{}
{\ctikzvalof{bipoles/phaseshifter/width}}
{vphaseshifter}
{\ctikzvalof{bipoles/vphaseshifter/width}}
{\ctikzvalof{bipoles/vphaseshifter/width}}
{
    \pgf@circ@res@step = \ctikzvalof{bipoles/vphaseshifter/width}\pgf@circ@scaled@Rlen
    \divide \pgf@circ@res@step by 2

    \pgfpathmoveto{\pgfpoint{\pgf@circ@res@left}{\pgf@circ@res@zero}}
    \pgf@circ@res@other = \pgf@circ@res@left
    \advance\pgf@circ@res@other by \pgf@circ@res@step

    \ifpgf@circuit@dashed
        \pgfsetdash{{0.1cm}{0.1cm}}{0cm}
    \fi

    % draw outer box
    \pgf@circ@twoportbox

    \ifpgf@circuit@inputarrow
        {
            \advance \pgf@circ@res@left by -.5\ctikzvalof{bipoles/thickness}\pgfstartlinewidth
            \pgftransformshift{\pgfpoint{\pgf@circ@res@left}{0pt}}
            \pgfnode{inputarrow}{tip}{}{pgf@inputarrow}{\pgfusepath{fill}}
        }
    \fi

    % rotate inner symbol
    \def\pgfcircmathresult{\expandafter\pgf@circ@stripdecimals\pgf@circ@direction\pgf@nil}
    \ifnum \pgfcircmathresult > 45 \ifnum \pgfcircmathresult < 135
        \pgftransformrotate{270}
    \fi\fi
    \ifnum \pgfcircmathresult > 135 \ifnum \pgfcircmathresult < 225
        \pgftransformrotate{180}
    \fi\fi
    \ifnum \pgfcircmathresult > 225 \ifnum \pgfcircmathresult < 315
        \pgftransformrotate{90}
    \fi\fi

    % inner symbol
    \pgftext[center,x=0,y=0]{\Large$\varphi$}

    \pgfsetdash{}{0pt}	% always draw solid line for inner symbol
    \pgfsetlinewidth{\pgfstartlinewidth}
    \pgfsetarrowsend{latex}
    \pgfpathmoveto{\pgfpoint{0.75\pgf@circ@res@left}{0.65\pgf@circ@res@down}}
    \pgfpathlineto{\pgfpoint{0.7\pgf@circ@res@right}{0.65\pgf@circ@res@up}}
    \pgfusepath{draw}
}

%% detector
\pgfcircdeclarebipolescaled{blocks}
{}
{\ctikzvalof{bipoles/detector/width}}
{detector}
{\ctikzvalof{bipoles/detector/width}}
{\ctikzvalof{bipoles/detector/width}}
{
    \pgf@circ@res@step = \ctikzvalof{bipoles/detector/width}\pgf@circ@scaled@Rlen
    \divide \pgf@circ@res@step by 2

    \pgfpathmoveto{\pgfpoint{\pgf@circ@res@left}{\pgf@circ@res@zero}}
    \pgf@circ@res@other = \pgf@circ@res@left
    \advance\pgf@circ@res@other by \pgf@circ@res@step

    \ifpgf@circuit@dashed
        \pgfsetdash{{0.1cm}{0.1cm}}{0cm}
    \fi

    % draw outer box
    \pgf@circ@twoportbox

    \ifpgf@circuit@inputarrow
        {
            \advance \pgf@circ@res@left by -.5\ctikzvalof{bipoles/thickness}\pgfstartlinewidth
            \pgftransformshift{\pgfpoint{\pgf@circ@res@left}{0pt}}
            \pgfnode{inputarrow}{tip}{}{pgf@inputarrow}{\pgfusepath{fill}}
        }
    \fi

    % draw inner stuff
    \pgfsetdash{}{0pt}	% always draw solid line for inner symbol
    \pgfsetarrows{-} %never draw arrows
    \pgfsetlinewidth{0.8\pgfstartlinewidth}
    \pgfpathmoveto{\pgfpoint{0.75\pgf@circ@res@left}{0}}
    \pgfpathlineto{\pgfpoint{0.75\pgf@circ@res@right}{0}}
    \pgfusepath{draw}

    \ifpgf@circuit@fulldiode
        \pgfmathparse{2\pgf@circ@res@up / \pgf@circ@scaled@Rlen / \ctikzvalof{bipoles/generic/width}}
        \pgftransformscale{\pgfmathresult}
        \pgfnode{fulldiodeshape}{center}{}{pgf@fulldiode}{\pgfusepath{fill}}
    \else
        \pgfmathparse{2\pgf@circ@res@up / \pgf@circ@scaled@Rlen / \ctikzvalof{bipoles/resistor/width}}
        \pgftransformscale{\pgfmathresult}
        \pgfnode{emptydiodeshape}{center}{}{pgf@emptydiode}{\pgfusepath{fill}}
    \fi

}

%% single phase ac/dc converter
\pgfcircdeclarebipolescaled{blocks}
{
    \anchor{dc1}{
        \northeast
        \pgf@y=.4\pgf@y
    }
    \anchor{dc2}{
        \northeast
        \pgf@y=-.4\pgf@y
    }
}
{\ctikzvalof{bipoles/sacdc/width}}
{sacdc}
{\ctikzvalof{bipoles/sacdc/width}}
{\ctikzvalof{bipoles/sacdc/width}}
{
    \pgf@circ@res@step = \ctikzvalof{bipoles/sacdc/width}\pgf@circ@scaled@Rlen
    \divide \pgf@circ@res@step by 2

    \pgfpathmoveto{\pgfpoint{\pgf@circ@res@left}{\pgf@circ@res@zero}}
    \pgf@circ@res@other = \pgf@circ@res@left
    \advance\pgf@circ@res@other by \pgf@circ@res@step

    \ifpgf@circuit@dashed
        \pgfsetdash{{0.1cm}{0.1cm}}{0cm}
    \fi

    % draw outer box
    \pgf@circ@twoportbox

    \ifpgf@circuit@inputarrow
        {
            \advance \pgf@circ@res@left by -.5\ctikzvalof{bipoles/thickness}\pgfstartlinewidth
            \pgftransformshift{\pgfpoint{\pgf@circ@res@left}{0pt}}
            \pgfnode{inputarrow}{tip}{}{pgf@inputarrow}{\pgfusepath{fill}}
        }
    \fi

    % rotate inner symbol
    \def\pgfcircmathresult{\expandafter\pgf@circ@stripdecimals\pgf@circ@direction\pgf@nil}
    \ifnum \pgfcircmathresult > 45 \ifnum \pgfcircmathresult < 135
        \pgftransformrotate{270}
    \fi\fi
    \ifnum \pgfcircmathresult > 134 \ifnum \pgfcircmathresult < 225  % 134 degree, because >= 135 is not possible
        \pgftransformrotate{180}
    \fi\fi
    \ifnum \pgfcircmathresult > 225 \ifnum \pgfcircmathresult < 315
        \pgftransformrotate{90}
    \fi\fi

    % draw inner symbol
    \pgfsetdash{}{0pt}	% always draw solid line for inner symbol
    \pgfsetarrows{-} %never draw arrows
    \pgfsetlinewidth{\pgfstartlinewidth}
    \pgfpathmoveto{\pgfpoint{\pgf@circ@res@left}{\pgf@circ@res@down}}
    \pgfpathlineto{\pgfpoint{\pgf@circ@res@right}{\pgf@circ@res@up}}
    \pgfusepath{draw}
    
    % draw sin wave
    \pgfpathmoveto{\pgfpoint{-.76\pgf@circ@res@step}{.5\pgf@circ@res@step}}
    \pgfpathsine{\pgfpoint{.17\pgf@circ@res@step}{.17\pgf@circ@res@step}}
    \pgfpathcosine{\pgfpoint{.17\pgf@circ@res@step}{-.17\pgf@circ@res@step}}
    \pgfpathsine{\pgfpoint{.17\pgf@circ@res@step}{-.17\pgf@circ@res@step}}
    \pgfpathcosine{\pgfpoint{.17\pgf@circ@res@step}{.17\pgf@circ@res@step}}
    \pgfusepath{draw}
    
    % draw equal sign
    \pgfpathmoveto{\pgfpoint{.2\pgf@circ@res@step}{-.375\pgf@circ@res@step}}
    \pgfpathlineto{\pgfpoint{0.7\pgf@circ@res@step}{-0.375\pgf@circ@res@step}}
    \pgfusepath{draw}
    \pgfpathmoveto{\pgfpoint{.2\pgf@circ@res@step}{-.625\pgf@circ@res@step}}
    \pgfpathlineto{\pgfpoint{0.7\pgf@circ@res@step}{-0.625\pgf@circ@res@step}}
    \pgfusepath{draw}
}



%% single phase dc/ac converter
\pgfcircdeclarebipolescaled{blocks}
{
    \anchor{dc1}{
        \northeast
        \pgf@y=.4\pgf@y
        \pgf@x=-\pgf@x
    }
    \anchor{dc2}{
        \northeast
        \pgf@y=-.4\pgf@y
        \pgf@x=-\pgf@x
    }
}
{\ctikzvalof{bipoles/sdcac/width}}
{sdcac}
{\ctikzvalof{bipoles/sdcac/width}}
{\ctikzvalof{bipoles/sdcac/width}}
{
    \pgf@circ@res@step = \ctikzvalof{bipoles/sdcac/width}\pgf@circ@scaled@Rlen
    \divide \pgf@circ@res@step by 2

    \pgfpathmoveto{\pgfpoint{\pgf@circ@res@left}{\pgf@circ@res@zero}}
    \pgf@circ@res@other = \pgf@circ@res@left
    \advance\pgf@circ@res@other by \pgf@circ@res@step

    \ifpgf@circuit@dashed
        \pgfsetdash{{0.1cm}{0.1cm}}{0cm}
    \fi

    % draw outer box
    \pgf@circ@twoportbox

    \ifpgf@circuit@inputarrow
        {
            \advance \pgf@circ@res@left by -.5\ctikzvalof{bipoles/thickness}\pgfstartlinewidth
            \pgftransformshift{\pgfpoint{\pgf@circ@res@left}{0pt}}
            \pgfnode{inputarrow}{tip}{}{pgf@inputarrow}{\pgfusepath{fill}}
        }
    \fi

    % rotate inner symbol
    \def\pgfcircmathresult{\expandafter\pgf@circ@stripdecimals\pgf@circ@direction\pgf@nil}
    \ifnum \pgfcircmathresult > 45 \ifnum \pgfcircmathresult < 135
        \pgftransformrotate{270}
    \fi\fi
    \ifnum \pgfcircmathresult > 134 \ifnum \pgfcircmathresult < 225  % 134 degree, because >= 135 is not possible
        \pgftransformrotate{180}
    \fi\fi
    \ifnum \pgfcircmathresult > 225 \ifnum \pgfcircmathresult < 315
        \pgftransformrotate{90}
    \fi\fi

    % draw inner symbol
    \pgfsetdash{}{0pt}	% always draw solid line for inner symbol
    \pgfsetarrows{-} %never draw arrows
    \pgfsetlinewidth{\pgfstartlinewidth}
    \pgfpathmoveto{\pgfpoint{\pgf@circ@res@left}{\pgf@circ@res@down}}
    \pgfpathlineto{\pgfpoint{\pgf@circ@res@right}{\pgf@circ@res@up}}
    \pgfusepath{draw}
    
    % draw sin wave
    \pgfpathmoveto{\pgfpoint{.14\pgf@circ@res@step}{-.5\pgf@circ@res@step}}
    \pgfpathsine{\pgfpoint{.17\pgf@circ@res@step}{.17\pgf@circ@res@step}}
    \pgfpathcosine{\pgfpoint{.17\pgf@circ@res@step}{-.17\pgf@circ@res@step}}
    \pgfpathsine{\pgfpoint{.17\pgf@circ@res@step}{-.17\pgf@circ@res@step}}
    \pgfpathcosine{\pgfpoint{.17\pgf@circ@res@step}{.17\pgf@circ@res@step}}
    \pgfusepath{draw}
    
    % draw equal sign
    \pgfpathmoveto{\pgfpoint{-.2\pgf@circ@res@step}{.375\pgf@circ@res@step}}
    \pgfpathlineto{\pgfpoint{-0.7\pgf@circ@res@step}{0.375\pgf@circ@res@step}}
    \pgfusepath{draw}
    \pgfpathmoveto{\pgfpoint{-.2\pgf@circ@res@step}{.625\pgf@circ@res@step}}
    \pgfpathlineto{\pgfpoint{-0.7\pgf@circ@res@step}{0.625\pgf@circ@res@step}}
    \pgfusepath{draw}
}


%% threephase ac/dc converter
\pgfcircdeclarebipolescaled{blocks}
{
    \anchor{dc1}{
        \northeast
        \pgf@y=.4\pgf@y
    }
    \anchor{dc2}{
        \northeast
        \pgf@y=-.4\pgf@y
    }
    \anchor{ac1}{
        \northeast
        \pgf@y=.6\pgf@y
        \pgf@x=-\pgf@x
    }
    \anchor{ac2}{
        \northeast
        \pgf@y=0\pgf@y
        \pgf@x=-\pgf@x
    }
    \anchor{ac3}{
        \northeast
        \pgf@y=-.6\pgf@y
        \pgf@x=-\pgf@x
    }
}
{\ctikzvalof{bipoles/tacdc/width}}
{tacdc}
{\ctikzvalof{bipoles/tacdc/width}}
{\ctikzvalof{bipoles/tacdc/width}}
{
    \pgf@circ@res@step = \ctikzvalof{bipoles/tacdc/width}\pgf@circ@scaled@Rlen
    \divide \pgf@circ@res@step by 2

    \pgfpathmoveto{\pgfpoint{\pgf@circ@res@left}{\pgf@circ@res@zero}}
    \pgf@circ@res@other = \pgf@circ@res@left
    \advance\pgf@circ@res@other by \pgf@circ@res@step

    \ifpgf@circuit@dashed
        \pgfsetdash{{0.1cm}{0.1cm}}{0cm}
    \fi

    % draw outer box
    \pgf@circ@twoportbox

    \ifpgf@circuit@inputarrow
        {
            \advance \pgf@circ@res@left by -.5\ctikzvalof{bipoles/thickness}\pgfstartlinewidth
            \pgftransformshift{\pgfpoint{\pgf@circ@res@left}{0pt}}
            \pgfnode{inputarrow}{tip}{}{pgf@inputarrow}{\pgfusepath{fill}}
        }
    \fi

    % rotate inner symbol
    \def\pgfcircmathresult{\expandafter\pgf@circ@stripdecimals\pgf@circ@direction\pgf@nil}
    \ifnum \pgfcircmathresult > 45 \ifnum \pgfcircmathresult < 135
        \pgftransformrotate{270}
    \fi\fi
    \ifnum \pgfcircmathresult > 134 \ifnum \pgfcircmathresult < 225  % 134 degree, because >= 135 is not possible
        \pgftransformrotate{180}
    \fi\fi
    \ifnum \pgfcircmathresult > 225 \ifnum \pgfcircmathresult < 315
        \pgftransformrotate{90}
    \fi\fi

    % draw inner symbol
    \pgfsetdash{}{0pt}	% always draw solid line for inner symbol
    \pgfsetarrows{-} %never draw arrows
    \pgfsetlinewidth{\pgfstartlinewidth}
    \pgfpathmoveto{\pgfpoint{\pgf@circ@res@left}{\pgf@circ@res@down}}
    \pgfpathlineto{\pgfpoint{\pgf@circ@res@right}{\pgf@circ@res@up}}
    \pgfusepath{draw}
    
    % draw sin waves
    \pgfpathmoveto{\pgfpoint{-.76\pgf@circ@res@step}{.65\pgf@circ@res@step}}
    \pgfpathsine{\pgfpoint{.17\pgf@circ@res@step}{.17\pgf@circ@res@step}}
    \pgfpathcosine{\pgfpoint{.17\pgf@circ@res@step}{-.17\pgf@circ@res@step}}
    \pgfpathsine{\pgfpoint{.17\pgf@circ@res@step}{-.17\pgf@circ@res@step}}
    \pgfpathcosine{\pgfpoint{.17\pgf@circ@res@step}{.17\pgf@circ@res@step}}
    \pgfusepath{draw}
    \pgfpathmoveto{\pgfpoint{-.76\pgf@circ@res@step}{.5\pgf@circ@res@step}}
    \pgfpathsine{\pgfpoint{.17\pgf@circ@res@step}{.17\pgf@circ@res@step}}
    \pgfpathcosine{\pgfpoint{.17\pgf@circ@res@step}{-.17\pgf@circ@res@step}}
    \pgfpathsine{\pgfpoint{.17\pgf@circ@res@step}{-.17\pgf@circ@res@step}}
    \pgfpathcosine{\pgfpoint{.17\pgf@circ@res@step}{.17\pgf@circ@res@step}}
    \pgfusepath{draw}
    \pgfpathmoveto{\pgfpoint{-.76\pgf@circ@res@step}{.35\pgf@circ@res@step}}
    \pgfpathsine{\pgfpoint{.17\pgf@circ@res@step}{.17\pgf@circ@res@step}}
    \pgfpathcosine{\pgfpoint{.17\pgf@circ@res@step}{-.17\pgf@circ@res@step}}
    \pgfpathsine{\pgfpoint{.17\pgf@circ@res@step}{-.17\pgf@circ@res@step}}
    \pgfpathcosine{\pgfpoint{.17\pgf@circ@res@step}{.17\pgf@circ@res@step}}
    \pgfusepath{draw}
    
    
    
    
    % draw equal sign
    \pgfpathmoveto{\pgfpoint{.2\pgf@circ@res@step}{-.375\pgf@circ@res@step}}
    \pgfpathlineto{\pgfpoint{0.7\pgf@circ@res@step}{-0.375\pgf@circ@res@step}}
    \pgfusepath{draw}
    \pgfpathmoveto{\pgfpoint{.2\pgf@circ@res@step}{-.625\pgf@circ@res@step}}
    \pgfpathlineto{\pgfpoint{0.7\pgf@circ@res@step}{-0.625\pgf@circ@res@step}}
    \pgfusepath{draw}
}


%% threephase dc/ac converter
\pgfcircdeclarebipolescaled{blocks}
{
    \anchor{dc1}{
        \northeast
        \pgf@y=.4\pgf@y
        \pgf@x=-\pgf@x
    }
    \anchor{dc2}{
        \northeast
        \pgf@y=-.4\pgf@y
        \pgf@x=-\pgf@x
    }
    \anchor{ac1}{
        \northeast
        \pgf@y=.6\pgf@y
    }
    \anchor{ac2}{
        \northeast
        \pgf@y=0\pgf@y
    }
    \anchor{ac3}{
        \northeast
        \pgf@y=-.6\pgf@y
    }
}
{\ctikzvalof{bipoles/tdcac/width}}
{tdcac}
{\ctikzvalof{bipoles/tdcac/width}}
{\ctikzvalof{bipoles/tdcac/width}}
{
    \pgf@circ@res@step = \ctikzvalof{bipoles/tdcac/width}\pgf@circ@scaled@Rlen
    \divide \pgf@circ@res@step by 2

    \pgfpathmoveto{\pgfpoint{\pgf@circ@res@left}{\pgf@circ@res@zero}}
    \pgf@circ@res@other = \pgf@circ@res@left
    \advance\pgf@circ@res@other by \pgf@circ@res@step

    \ifpgf@circuit@dashed
        \pgfsetdash{{0.1cm}{0.1cm}}{0cm}
    \fi

    % draw outer box
    \pgf@circ@twoportbox

    \ifpgf@circuit@inputarrow
        {
            \advance \pgf@circ@res@left by -.5\ctikzvalof{bipoles/thickness}\pgfstartlinewidth
            \pgftransformshift{\pgfpoint{\pgf@circ@res@left}{0pt}}
            \pgfnode{inputarrow}{tip}{}{pgf@inputarrow}{\pgfusepath{fill}}
        }
    \fi

    % rotate inner symbol
    \def\pgfcircmathresult{\expandafter\pgf@circ@stripdecimals\pgf@circ@direction\pgf@nil}
    \ifnum \pgfcircmathresult > 45 \ifnum \pgfcircmathresult < 135
        \pgftransformrotate{270}
    \fi\fi
    \ifnum \pgfcircmathresult > 134 \ifnum \pgfcircmathresult < 225  % 134 degree, because >= 135 is not possible
        \pgftransformrotate{180}
    \fi\fi
    \ifnum \pgfcircmathresult > 225 \ifnum \pgfcircmathresult < 315
        \pgftransformrotate{90}
    \fi\fi

    % draw inner symbol
    \pgfsetdash{}{0pt}	% always draw solid line for inner symbol
    \pgfsetarrows{-} %never draw arrows
    \pgfsetlinewidth{\pgfstartlinewidth}
    \pgfpathmoveto{\pgfpoint{\pgf@circ@res@left}{\pgf@circ@res@down}}
    \pgfpathlineto{\pgfpoint{\pgf@circ@res@right}{\pgf@circ@res@up}}
    \pgfusepath{draw}
    
    % draw sin waves
    \pgfpathmoveto{\pgfpoint{.14\pgf@circ@res@step}{-.65\pgf@circ@res@step}}
    \pgfpathsine{\pgfpoint{.17\pgf@circ@res@step}{.17\pgf@circ@res@step}}
    \pgfpathcosine{\pgfpoint{.17\pgf@circ@res@step}{-.17\pgf@circ@res@step}}
    \pgfpathsine{\pgfpoint{.17\pgf@circ@res@step}{-.17\pgf@circ@res@step}}
    \pgfpathcosine{\pgfpoint{.17\pgf@circ@res@step}{.17\pgf@circ@res@step}}
    \pgfusepath{draw}
    \pgfpathmoveto{\pgfpoint{.14\pgf@circ@res@step}{-.5\pgf@circ@res@step}}
    \pgfpathsine{\pgfpoint{.17\pgf@circ@res@step}{.17\pgf@circ@res@step}}
    \pgfpathcosine{\pgfpoint{.17\pgf@circ@res@step}{-.17\pgf@circ@res@step}}
    \pgfpathsine{\pgfpoint{.17\pgf@circ@res@step}{-.17\pgf@circ@res@step}}
    \pgfpathcosine{\pgfpoint{.17\pgf@circ@res@step}{.17\pgf@circ@res@step}}
    \pgfusepath{draw}
    \pgfpathmoveto{\pgfpoint{.14\pgf@circ@res@step}{-.35\pgf@circ@res@step}}
    \pgfpathsine{\pgfpoint{.17\pgf@circ@res@step}{.17\pgf@circ@res@step}}
    \pgfpathcosine{\pgfpoint{.17\pgf@circ@res@step}{-.17\pgf@circ@res@step}}
    \pgfpathsine{\pgfpoint{.17\pgf@circ@res@step}{-.17\pgf@circ@res@step}}
    \pgfpathcosine{\pgfpoint{.17\pgf@circ@res@step}{.17\pgf@circ@res@step}}
    \pgfusepath{draw}
    
    
    
    
    % draw equal sign
    \pgfpathmoveto{\pgfpoint{-.2\pgf@circ@res@step}{.375\pgf@circ@res@step}}
    \pgfpathlineto{\pgfpoint{-0.7\pgf@circ@res@step}{0.375\pgf@circ@res@step}}
    \pgfusepath{draw}
    \pgfpathmoveto{\pgfpoint{-.2\pgf@circ@res@step}{.625\pgf@circ@res@step}}
    \pgfpathlineto{\pgfpoint{-0.7\pgf@circ@res@step}{0.625\pgf@circ@res@step}}
    \pgfusepath{draw}
}


%%%%%%%%%%%%%%%%%%%%%%%
%% MECHANICAL SYMBOLS
%%%%%%%%%%%%%%%%%%%%%%%

%% mechanical capacitance - stiffness/spring

\pgfcircdeclarebipolescaled{mechanicals}
{}
{\ctikzvalof{bipoles/spring/height}}
{spring}
{\ctikzvalof{bipoles/spring/height}}
{\ctikzvalof{bipoles/spring/width}}{
    \pgf@circ@setlinewidth{bipoles}{\pgfstartlinewidth}
    \pgftransformationadjustments

    \pgfmathsetlength{\pgf@circ@res@step}
    {(\ctikzvalof{bipoles/spring/width}*\pgf@circ@scaled@Rlen+\pgfhorizontaltransformationadjustment*.5*\pgflinewidth)/16}

    \pgfpathmoveto{\pgfpoint{\pgf@circ@res@left-\pgfhorizontaltransformationadjustment*0.5*\pgflinewidth}{\pgf@circ@res@zero}}
    \pgfsetcornersarced{\pgfpoint{.5\pgf@circ@res@up}{.5\pgf@circ@res@up}}
    \pgf@circ@res@other = \pgf@circ@res@left
    \advance\pgf@circ@res@other by \pgf@circ@res@step
    \pgfpathlineto{\pgfpoint{\pgf@circ@res@other}{\pgf@circ@res@up}}
    \advance\pgf@circ@res@other by 2\pgf@circ@res@step
    \pgfpathlineto{\pgfpoint{\pgf@circ@res@other}{\pgf@circ@res@down}}
    \advance\pgf@circ@res@other by 2\pgf@circ@res@step
    \pgfpathlineto{\pgfpoint{\pgf@circ@res@other}{\pgf@circ@res@up}}
    \advance\pgf@circ@res@other by 2\pgf@circ@res@step
    \pgfpathlineto{\pgfpoint{\pgf@circ@res@other}{\pgf@circ@res@down}}
    \advance\pgf@circ@res@other by 2\pgf@circ@res@step
    \pgfpathlineto{\pgfpoint{\pgf@circ@res@other}{\pgf@circ@res@up}}
    \advance\pgf@circ@res@other by 2\pgf@circ@res@step
    \pgfpathlineto{\pgfpoint{\pgf@circ@res@other}{\pgf@circ@res@down}}
    \advance\pgf@circ@res@other by 2\pgf@circ@res@step
    \pgfpathlineto{\pgfpoint{\pgf@circ@res@other}{\pgf@circ@res@up}}
    \advance\pgf@circ@res@other by 2\pgf@circ@res@step
    \pgfpathlineto{\pgfpoint{\pgf@circ@res@other}{\pgf@circ@res@down}}
    \advance\pgf@circ@res@other by \pgf@circ@res@step
    \pgfpathlineto{\pgfpoint{\pgf@circ@res@other}{\pgf@circ@res@zero}}
    \pgfsetbuttcap
    \pgfsetbeveljoin
    \pgfusepath{stroke}
}

%% mechanical capacitance - inerter
\pgfcircdeclarebipolescaled{mechanicals}
{}
{\ctikzvalof{bipoles/inerter/height}}
{inerter}
{\ctikzvalof{bipoles/inerter/height}}
{\ctikzvalof{bipoles/inerter/width}}
{
    \pgfpathrectanglecorners{\pgfpoint{\pgf@circ@res@left}{\pgf@circ@res@up}}{\pgfpoint{\pgf@circ@res@right}{\pgf@circ@res@down}}
    \pgf@circ@setlinewidth{bipoles}{\pgfstartlinewidth}
    \pgf@circ@draworfill
}


%% mechanical inductance - mass
\pgfcircdeclarebipolescaled{mechanicals}
{}
{\ctikzvalof{bipoles/mass/box height}}
{mass}
{\ctikzvalof{bipoles/mass/height}}
{\ctikzvalof{bipoles/mass/width}}
{
    \pgf@circ@setlinewidth{bipoles}{\pgfstartlinewidth}
    \pgftransformationadjustments
    \pgfpathrectangle{\pgfpoint{\pgf@circ@res@left}{\pgf@circ@res@down}}
        {\pgfpoint{-2\pgf@circ@res@down}{-2\pgf@circ@res@down}}
    \pgf@circ@draworfill

    \pgfpathmoveto{\pgfpoint{\pgf@circ@res@right}{\pgf@circ@res@down}}
    \pgfpathlineto{\pgfpoint{\pgf@circ@res@right}{\pgf@circ@res@up}}
    \pgfpathlineto{\pgfpoint{\pgf@circ@res@left}{\pgf@circ@res@up}}
    \pgfsetrectcap
    \pgfsetmiterjoin
    \pgfusepath{stroke}
}

%% mechanical resistor - damper
\pgfcircdeclarebipolescaled{mechanicals}
{}
{\ctikzvalof{bipoles/damper/height}}
{damper}
{\ctikzvalof{bipoles/damper/height}}
{\ctikzvalof{bipoles/damper/width}}
{
    \pgfpathrectanglecorners{\pgfpoint{\ctikzvalof{bipoles/damper/length}\pgf@circ@res@right}{\pgf@circ@res@down}}{\pgfpoint{\pgf@circ@res@right}{\pgf@circ@res@up}}
    \pgf@circ@maybefill

    % line into the damper
    \pgfpathmoveto{\pgfpoint{\pgf@circ@res@left}{\pgf@circ@res@zero}}
    \pgfpathlineto{\pgfpoint{\ctikzvalof{bipoles/damper/length}\pgf@circ@res@right}
        {\pgf@circ@res@zero}}
    \pgfusepath{stroke}

    % damper box
    \pgf@circ@setlinewidth{bipoles}{\pgfstartlinewidth}
    \pgfpathmoveto{\pgfpoint{\pgf@circ@res@left}{\pgf@circ@res@down}}
    \pgfpathlineto{\pgfpoint{\pgf@circ@res@right}{\pgf@circ@res@down}}
    \pgfpathlineto{\pgfpoint{\pgf@circ@res@right}{\pgf@circ@res@up}}
    \pgfpathlineto{\pgfpoint{\pgf@circ@res@left}{\pgf@circ@res@up}}

    \pgfsetrectcap
    \pgfsetmiterjoin
    \pgfusepath{stroke}

    % damper vertical element
    \pgfpathmoveto{\pgfpoint{\ctikzvalof{bipoles/damper/length}\pgf@circ@res@right}
        {.8\pgf@circ@res@down}}
    \pgfpathlineto{\pgfpoint{\ctikzvalof{bipoles/damper/length}\pgf@circ@res@right}
        {.8\pgf@circ@res@up}}
    \pgfsetbuttcap
    \pgfusepath{stroke}

}
%% mechanical viscoelastic element, suggested by @alex
%% in https://tex.stackexchange.com/questions/484268/combined-spring-damper-in-circuitikz
\pgfcircdeclarebipolescaled{mechanicals}
{}                                   % extra anchors
{\ctikzvalof{bipoles/damper/height}} % depth (under the path line)
{viscoe}                             % name
{\ctikzvalof{bipoles/damper/height}} % height (above the path line)
{\ctikzvalof{bipoles/damper/width}}  % width
{ % draw the bipole
    \pgfpathrectanglecorners{\pgfpoint{\ctikzvalof{bipoles/damper/length}\pgf@circ@res@right}{\pgf@circ@res@down}}{\pgfpoint{\pgf@circ@res@right}{\pgf@circ@res@up}}
    \pgf@circ@maybefill

    % spring into the damper
    \pgfscope
        \pgfpathmoveto{\pgfpoint{\pgf@circ@res@left}{\pgf@circ@res@zero}}
        \pgf@circ@setlinewidth{bipoles}{\pgfstartlinewidth}
        \pgfsetcornersarced{\pgfpoint{.25\pgf@circ@res@up}{.25\pgf@circ@res@up}}
        \pgfpathlineto{\pgfpoint{.75\pgf@circ@res@left}{.75\pgf@circ@res@up}}
        \pgfpathlineto{\pgfpoint{.5\pgf@circ@res@left}{-.75\pgf@circ@res@up}}
        \pgfpathlineto{\pgfpoint{.25\pgf@circ@res@left}{.75\pgf@circ@res@up}}
        \pgfpathlineto{\pgfpoint{0pt}{-.75\pgf@circ@res@up}}
        \pgfpathlineto{\pgfpoint{\ctikzvalof{bipoles/damper/length}\pgf@circ@res@right}{.75\pgf@circ@res@up}}
        \pgfusepath{stroke}
    \endpgfscope
    % damper box
    \pgf@circ@setlinewidth{bipoles}{\pgfstartlinewidth}
    \pgfpathmoveto{\pgfpoint{\pgf@circ@res@left}{\pgf@circ@res@down}}
    \pgfpathlineto{\pgfpoint{\pgf@circ@res@right}{\pgf@circ@res@down}}
    \pgfpathlineto{\pgfpoint{\pgf@circ@res@right}{\pgf@circ@res@up}}
    \pgfpathlineto{\pgfpoint{\pgf@circ@res@left}{\pgf@circ@res@up}}

    \pgfsetrectcap
    \pgfsetmiterjoin
    \pgfusepath{stroke}

    % damper vertical element
    \pgfpathmoveto{\pgfpoint{\ctikzvalof{bipoles/damper/length}\pgf@circ@res@right}
        {.8\pgf@circ@res@down}}
    \pgfpathlineto{\pgfpoint{\ctikzvalof{bipoles/damper/length}\pgf@circ@res@right}
        {.8\pgf@circ@res@up}}
    \pgfsetbuttcap
    \pgfusepath{stroke}

}

%%%%%%%%%%%%%%%%
%% Crossing
%%%%%%%%%%%%%%%%

%% crossing bipole (but see also nodes)
\pgfcircdeclarebipole
{}
{\ctikzvalof{bipoles/crossing/size}}
{crossing}
{\ctikzvalof{bipoles/crossing/size}}
{\ctikzvalof{bipoles/crossing/size}}{
    \pgfscope
        \pgfpathmoveto{\pgfpoint{\pgf@circ@res@left}{\pgf@circ@res@zero}}
        \pgfpathlineto{\pgfpoint{0.4\pgf@circ@res@left}{\pgf@circ@res@zero}}
        \pgfpatharc{0}{-180}{0.4*\pgf@circ@res@left}
        \pgfsetbeveljoin
        \pgfpathlineto{\pgfpoint{\pgf@circ@res@right}{\pgf@circ@res@zero}}
        \pgfusepath{draw}
    \endpgfscope
}

%%%%%%%%%%%%%%%%%%%%%%%%%
%% Miscellaneous bipoles
%%%%%%%%%%%%%%%%%%%%%%%%%

%% loudspeaker and microphone

\pgfcircdeclarebipolescaled{misc}
{}
{\ctikzvalof{bipoles/loudspeaker/depth}}
{loudspeaker}
{\ctikzvalof{bipoles/loudspeaker/height}}
{\ctikzvalof{bipoles/loudspeaker/width}}{

    \pgfpathrectanglecorners{\pgfpoint{\pgf@circ@res@left}{.4\pgf@circ@res@up}}
    {\pgfpoint{\pgf@circ@res@right}{\pgf@circ@res@down}}
    \pgfpathmoveto{\pgfpoint{0.5\pgf@circ@res@left}{.4\pgf@circ@res@up}}
    \pgfpathlineto{\pgfpoint{0.8\pgf@circ@res@left}{\pgf@circ@res@up}}
    \pgfpathlineto{\pgfpoint{0.8\pgf@circ@res@right}{\pgf@circ@res@up}}
    \pgfpathlineto{\pgfpoint{0.5\pgf@circ@res@right}{.4\pgf@circ@res@up}}
    \pgfpathclose
    \pgf@circ@setlinewidth{bipoles}{\pgfstartlinewidth}
    \pgf@circ@draworfill
}

\pgfcircdeclarebipolescaled{misc}
{}
{\ctikzvalof{bipoles/mic/depth}}
{mic}
{\ctikzvalof{bipoles/mic/height}}
{\ctikzvalof{bipoles/mic/width}}{

    \pgfscope
    \pgf@circ@setlinewidth{bipoles}{\pgfstartlinewidth}
    \pgfpathcircle{\pgfpoint{0pt}{.6\pgf@circ@res@up}}{.4\pgf@circ@res@up}
    \pgf@circ@draworfill
    \pgfpathmoveto{\pgfpoint{.6\pgf@circ@res@left}{\pgf@circ@res@up}}
    \pgfpathlineto{\pgfpoint{.6\pgf@circ@res@right}{\pgf@circ@res@up}}
    \pgfusepath{draw}
    \endpgfscope
    \pgfscope
    \pgfpathmoveto{\pgfpoint{\pgf@circ@res@left}{0pt}}
    \pgfpathlineto{\pgfpoint{-.2\pgf@circ@res@up}{0pt}}
    % 0.25358 is 0.6-0.4*cos(30)
    \pgfpathlineto{\pgfpoint{-.2\pgf@circ@res@up}{.25358\pgf@circ@res@up}}
    \pgfpathmoveto{\pgfpoint{.2\pgf@circ@res@up}{.25358\pgf@circ@res@up}}
    \pgfpathlineto{\pgfpoint{.2\pgf@circ@res@up}{0pt}}
    \pgfpathlineto{\pgfpoint{\pgf@circ@res@right}{0pt}}
    \pgfusepath{draw}

    \pgfusepath{draw}
    \endpgfscope
}

%% european gas filled surge arrester
\pgfcircdeclarebipolescaled{misc}
{}
{\ctikzvalof{bipoles/european gas filled surge arrester/height}}
{european gas filled surge arrester}
{\ctikzvalof{bipoles/european gas filled surge arrester/height}}
{\ctikzvalof{bipoles/european gas filled surge arrester/width}}
{

    \pgfpathrectanglecorners{\pgfpoint{\pgf@circ@res@left}{\pgf@circ@res@up}}{\pgfpoint{\pgf@circ@res@right}{\pgf@circ@res@down}}
    \pgf@circ@setlinewidth{bipoles}{\pgfstartlinewidth}

    \pgf@circ@draworfill

    \pgfscope
        \pgfsetarrowsend{latexslim}
        \pgfpathmoveto{\pgfpoint{\pgf@circ@res@left}{0pt}}
        \pgfpathlineto{\pgfpoint{\ctikzvalof{bipoles/european gas filled surge arrester/inside}\pgf@circ@res@left}{0pt}}
        \pgf@circ@setlinewidth{bipoles}{\pgfstartlinewidth}
        \pgfusepath{draw}

    \endpgfscope
}

%% american gas filled surge arrester
\pgfcircdeclarebipolescaled{misc}
{}
{\ctikzvalof{bipoles/american gas filled surge arrester/height}}
{american gas filled surge arrester}
{\ctikzvalof{bipoles/american gas filled surge arrester/height}}
{\ctikzvalof{bipoles/american gas filled surge arrester/width}}{

    \pgfpathmoveto{\pgfpoint{\pgf@circ@res@left}{\pgf@circ@res@zero}}
    \pgfpointorigin	\pgf@circ@res@other =  \pgf@x  \advance \pgf@circ@res@other by -\pgf@circ@res@up
    \pgfpathlineto{\pgfpoint{\pgf@circ@res@other}{\pgf@circ@res@zero}}
    \pgfusepath{draw}

    \pgfscope
        \pgf@circ@setlinewidth{bipoles}{\pgfstartlinewidth}
        \pgfpathcircle{\pgfpointorigin}{.9\pgf@circ@res@up}
        \pgf@circ@draworfill
    \endpgfscope

    \pgfpathmoveto{\pgfpoint{-\pgf@circ@res@other}{\pgf@circ@res@zero}}
    \pgfpathlineto{\pgfpoint{\pgf@circ@res@right}{\pgf@circ@res@zero}}
    \pgfusepath{draw}

    \pgfscope
        \pgfsetarrowsend{latex}

        \pgfpathmoveto{\pgfpoint{\pgf@circ@res@right}{\pgf@circ@res@zero}}
        \pgfpathlineto{\pgfpoint{\ctikzvalof{bipoles/american gas filled surge arrester/inside}\pgf@circ@res@right}{\pgf@circ@res@zero}}
        \pgfusepath{draw}

        \pgfpathmoveto{\pgfpoint{\pgf@circ@res@left}{\pgf@circ@res@zero}}
        \pgfpathlineto{\pgfpoint{\ctikzvalof{bipoles/american gas filled surge arrester/inside}\pgf@circ@res@left}{\pgf@circ@res@zero}}
        \pgfusepath{draw}
    \endpgfscope{}

    \pgfcircle{\pgfpoint{\ctikzvalof{bipoles/american gas filled surge arrester/dot x}\pgf@circ@res@left}{\ctikzvalof{bipoles/american gas filled surge arrester/dot y}\pgf@circ@res@down}}{\ctikzvalof{bipoles/american gas filled surge arrester/size}\pgf@circ@res@down}
    \pgfusepath{fill}
}

%% thermocouple
\pgfcircdeclarebipolescaled{misc}
{}
{\ctikzvalof{bipoles/thermocouple/height 2}}
{thermocouple}
{\ctikzvalof{bipoles/thermocouple/height}}
{\ctikzvalof{bipoles/thermocouple/width}}
{
    \pgfpathmoveto{\pgfpoint{\pgf@circ@res@left}{0}}
    \pgfpathlineto{\pgfpoint{\pgf@circ@res@left}{.5\pgf@circ@res@down}}
    \pgfusepath{draw}
    \pgfpathmoveto{\pgfpoint{\pgf@circ@res@left}{.5\pgf@circ@res@down}}
    \pgf@circ@setlinewidth{bipoles}{\pgfstartlinewidth}
    \pgfpathlineto{\pgfpoint{0}{\pgf@circ@res@down}}
    \pgfusepath{draw}
    \pgfsetlinewidth{\pgfstartlinewidth}
    \pgfpathmoveto{\pgfpoint{0}{\pgf@circ@res@down}}
    \pgfpathlineto{\pgfpoint{\pgf@circ@res@right}{.5\pgf@circ@res@down}}
    \pgfpathlineto{\pgfpoint{\pgf@circ@res@right}{0}}
    \pgfusepath{draw}
}

%% fuse
\pgfcircdeclarebipolescaled{misc}
{}
{\ctikzvalof{bipoles/fuse/height}}
{fuse}
{\ctikzvalof{bipoles/fuse/height}}
{\ctikzvalof{bipoles/fuse/width}}
{
    \pgfpathmoveto{\pgfpoint{\pgf@circ@res@left}{0}}
    \pgfpathlineto{\pgfpoint{\pgf@circ@res@right}{0}}
    \pgfusepath{draw}

    \pgfpathrectanglecorners{\pgfpoint{\pgf@circ@res@left}{\pgf@circ@res@up}}{\pgfpoint{\pgf@circ@res@right}{\pgf@circ@res@down}}
    \pgf@circ@setlinewidth{bipoles}{\pgfstartlinewidth}

    \pgf@circ@draworfill
}

%% asymmetric fuse
\pgfcircdeclarebipolescaled{misc}
{}
{\ctikzvalof{bipoles/afuse/height}}
{afuse}
{\ctikzvalof{bipoles/afuse/height}}
{\ctikzvalof{bipoles/afuse/width}}
{
    \pgfpathmoveto{\pgfpoint{\pgf@circ@res@left}{0}}
    \pgfpathlineto{\pgfpoint{\pgf@circ@res@right}{0}}
    \pgfusepath{draw}

    \pgfscope
        \pgfpathrectanglecorners{\pgfpoint{\pgf@circ@res@left}{\pgf@circ@res@up}}{\pgfpoint{\pgf@circ@res@right}{\pgf@circ@res@down}}
        \pgf@circ@setlinewidth{bipoles}{\pgfstartlinewidth}
        \pgf@circ@draworfill
    \endpgfscope

    \pgfpathrectanglecorners{\pgfpoint{.7\pgf@circ@res@right}{\pgf@circ@res@up}}{\pgfpoint{\pgf@circ@res@right}{\pgf@circ@res@down}}
    \pgfusepath{stroke,fill}
}

%% SQUID added by Cor Molenaar 5 March 2010
\pgfcircdeclarebipolescaled{misc}
{}
{\ctikzvalof{bipoles/squid/height}}
{squid}
{\ctikzvalof{bipoles/squid/height}}
{\ctikzvalof{bipoles/squid/width}}
{
    \pgf@circ@setlinewidth{bipoles}{\pgfstartlinewidth}

    \pgfpathellipse{\pgfpointorigin}{\pgfpoint{0}{\pgf@circ@res@up}}{\pgfpoint{\pgf@circ@res@left}{0}}
    \pgfpathmoveto{\pgfpoint{0.35*\pgf@circ@res@left}{1.35*\pgf@circ@res@up}}
    \pgfpathlineto{\pgfpoint{0.35*\pgf@circ@res@right}{0.65*\pgf@circ@res@up}}
    \pgfpathmoveto{\pgfpoint{0.35*\pgf@circ@res@left}{0.65*\pgf@circ@res@up}}
    \pgfpathlineto{\pgfpoint{0.35*\pgf@circ@res@right}{1.35*\pgf@circ@res@up}}
    \pgfpathmoveto{\pgfpoint{0.35*\pgf@circ@res@left}{1.35*\pgf@circ@res@down}}
    \pgfpathlineto{\pgfpoint{0.35*\pgf@circ@res@right}{0.65*\pgf@circ@res@down}}
    \pgfpathmoveto{\pgfpoint{0.35*\pgf@circ@res@left}{0.65*\pgf@circ@res@down}}
    \pgfpathlineto{\pgfpoint{0.35*\pgf@circ@res@right}{1.35*\pgf@circ@res@down}}

    \pgfusepath{draw}
}

% Generic barrier added by Cor Molenaar 5 March 2010
\pgfcircdeclarebipolescaled{misc}
{}
{\ctikzvalof{bipoles/barrier/height}}
{barrier}
{\ctikzvalof{bipoles/barrier/height}}
{\ctikzvalof{bipoles/barrier/width}}
{
    \pgf@circ@setlinewidth{bipoles}{\pgfstartlinewidth}

    \pgfpathmoveto{\pgfpoint{\pgf@circ@res@left}{\pgf@circ@res@step}}
    \pgfpathlineto{\pgfpoint{\pgf@circ@res@right}{\pgf@circ@res@step}}
    \pgfpathmoveto{\pgfpoint{0.35*\pgf@circ@res@left}{0.35*\pgf@circ@res@up}}
    \pgfpathlineto{\pgfpoint{0.35*\pgf@circ@res@right}{0.35*\pgf@circ@res@down}}
    \pgfpathmoveto{\pgfpoint{0.35*\pgf@circ@res@left}{0.35*\pgf@circ@res@down}}
    \pgfpathlineto{\pgfpoint{0.35*\pgf@circ@res@right}{0.35*\pgf@circ@res@up}}

    \pgfusepath{draw}
}

%
% open version of the barrier symbol
% suggested by Radványi Patrik Tamás <patrikradvanyi@gmail.com>
%
\pgfcircdeclarebipolescaled{misc}
{}
{\ctikzvalof{bipoles/barrier/height}}
{openbarrier}
{\ctikzvalof{bipoles/barrier/height}}
{\ctikzvalof{bipoles/barrier/width}}
{
    % this is set with normal wire linewidth
    \pgfpathmoveto{\pgfpoint{\pgf@circ@res@left}{0pt}}
    \pgfpathlineto{\pgfpoint{\ctikzvalof{bipoles/openbarrier/gap}*\pgf@circ@res@left}{0pt}}
    \pgfpathmoveto{\pgfpoint{\pgf@circ@res@right}{0pt}}
    \pgfpathlineto{\pgfpoint{\ctikzvalof{bipoles/openbarrier/gap}*\pgf@circ@res@right}{0pt}}
    \pgfusepath{draw}

    % do the cross part
    \pgf@circ@setlinewidth{bipoles}{\pgfstartlinewidth}

    \pgfpathmoveto{\pgfpoint{0.35*\pgf@circ@res@left}{0.35*\pgf@circ@res@up}}
    \pgfpathlineto{\pgfpoint{0.35*\pgf@circ@res@right}{0.35*\pgf@circ@res@down}}
    \pgfpathmoveto{\pgfpoint{0.35*\pgf@circ@res@left}{0.35*\pgf@circ@res@down}}
    \pgfpathlineto{\pgfpoint{0.35*\pgf@circ@res@right}{0.35*\pgf@circ@res@up}}

    \pgfusepath{draw}
}
%% Lamp
\pgfcircdeclarebipolescaled{misc}
{}
{\ctikzvalof{bipoles/lamp/height}}
{lamp}
{\ctikzvalof{bipoles/lamp/height}}
{\ctikzvalof{bipoles/lamp/width}}
{
    \pgf@circ@setlinewidth{bipoles}{\pgfstartlinewidth}

    \pgfpathellipse{\pgfpointorigin}{\pgfpoint{0}{\pgf@circ@res@up}}{\pgfpoint{\pgf@circ@res@left}{0}}
    \pgf@circ@draworfill
    \pgfpathmoveto{\pgfpoint{.7071*\pgf@circ@res@left}{.7071*\pgf@circ@res@up}}
    \pgfpathlineto{\pgfpoint{.7071*\pgf@circ@res@right}{.7071*\pgf@circ@res@down}}
    \pgfpathmoveto{\pgfpoint{.7071*\pgf@circ@res@left}{.7071*\pgf@circ@res@down}}
    \pgfpathlineto{\pgfpoint{.7071*\pgf@circ@res@right}{.7071*\pgf@circ@res@up}}
    \pgfusepath{draw}
}

%% bulb
\pgfcircdeclarebipolescaled{misc}
{}
{\ctikzvalof{bipoles/bulb/height}}
{bulb}
{\ctikzvalof{bipoles/bulb/height}}
{\ctikzvalof{bipoles/bulb/width}}
{%
    \pgf@circ@setlinewidth{bipoles}{\pgfstartlinewidth}
    \pgfpathellipse{\pgfpointorigin}{\pgfpoint{0}{0.8\pgf@circ@res@up}}{\pgfpoint{0.8\pgf@circ@res@left}{0}}
    \pgf@circ@draworfill
    \pgfpathmoveto{\pgfpoint{\pgf@circ@res@left}{\pgf@circ@res@zero}}
    \pgfpathlineto{\pgfpoint{0.4\pgf@circ@res@left}{\pgf@circ@res@zero}}
    \pgfpatharc{0}{-180}{0.4*\pgf@circ@res@left}
    \pgfsetbeveljoin
    \pgfpathlineto{\pgfpoint{\pgf@circ@res@right}{\pgf@circ@res@zero}}
    \pgfusepath{draw}
}


% end of pgfcircbipoles.tex
%%%---------- close: tex/pgfcircbipoles
%%%%%%%%%%% Springe nach tex/pgfcirctripoles
%%%---------- open: tex/pgfcirctripoles.tex
% Copyright 2018-2020 by Romano Giannetti
% Copyright 2015-2020 by Stefan Lindner
% Copyright 2013-2020 by Stefan Erhardt
% Copyright 2007-2020 by Massimo Redaelli
%
% This file may be distributed and/or modified
%
% 1. under the LaTeX Project Public License and/or
% 2. under the GNU Public License.
%
% See the files gpl-3.0_license.txt and lppl-1-3c_license.txt for more details.

%%%%%%%%%%%%%%%%%%%%%%%%%%%%%%%%%%%%%%
%% Tripoles

%%%%%%%%%%%%%
%% switches
%%%%%%%%%%%%%

% Legacy spdt
\pgfdeclareshape{spdt}{
    \savedmacro{\ctikzclass}{\edef\ctikzclass{switches}}
    \saveddimen{\scaledRlen}{\pgfmathsetlength{\pgf@x}{\ctikzvalof{\ctikzclass/scale}\pgf@circ@Rlen}}
    \savedanchor\northwest{%
        \pgfmathsetlength{\pgf@circ@scaled@Rlen}{\ctikzvalof{\ctikzclass/scale}\pgf@circ@Rlen}
        \pgf@y=\ctikzvalof{tripoles/spdt/height}\pgf@circ@scaled@Rlen
        \pgf@y=.5\pgf@y
        \pgf@x=-\ctikzvalof{tripoles/spdt/width}\pgf@circ@scaled@Rlen
        \pgf@x=.5\pgf@x
    }
    \anchor{left}{%
        \northwest
        \pgf@y=0pt
    }
    \anchor{in}{
        \northwest
        \pgf@y=0pt
    }
    \anchor{out 1}{
        \northwest
        \pgf@x=-\pgf@x
    }
    \anchor{out 2}{
        \northwest
        \pgf@x=-\pgf@x
        \pgf@y=-\pgf@y
    }
    \anchor{center}{
        \pgf@y=0pt
        \pgf@x=0pt
    }
    \anchor{east}{
        \northwest
        \pgf@y=0pt
        \pgf@x=-\pgf@x
    }
    \anchor{west}{
        \northwest
        \pgf@y=0pt
    }
    \anchor{south}{
        \northwest
        \pgf@x=0pt
        \pgf@y=-\pgf@y
    }
    \anchor{north}{
        \northwest
        \pgf@x=0pt
    }
    \anchor{south west}{
        \northwest
        \pgf@y=-\pgf@y
    }
    \anchor{north east}{
        \northwest
        \pgf@x=-\pgf@x
    }
    \anchor{north west}{
        \northwest
    }
    \anchor{south east}{
        \northwest
        \pgf@x=-\pgf@x
        \pgf@y=-\pgf@y
    }
    \backgroundpath{
        \pgfsetcolor{\ctikzvalof{color}}

        \northwest
        \pgf@circ@res@up = \pgf@y
        \pgf@circ@res@down = -\pgf@y
        \pgf@circ@res@right = -\pgf@x
        \pgf@circ@res@left = \pgf@x
        \pgf@circ@res@other = \ctikzvalof{tripoles/spdt/margin}\pgf@circ@res@left


        \pgfpathmoveto{\pgfpoint{\pgf@circ@res@right}{\pgf@circ@res@up}}
        \pgfpathlineto{\pgfpoint{-\pgf@circ@res@other}{\pgf@circ@res@up}}
        \pgfpathmoveto{\pgfpoint{\pgf@circ@res@right}{\pgf@circ@res@down}}
        \pgfpathlineto{\pgfpoint{-\pgf@circ@res@other}{\pgf@circ@res@down}}

        \pgfpathmoveto{\pgfpoint{\pgf@circ@res@left}{0pt}}
        \pgfpathlineto{\pgfpoint{\pgf@circ@res@other}{0pt}}

        \pgfusepath{draw}

        \pgfscope
            \pgftransformshift{\pgfpoint{-\pgf@circ@res@other}{\pgf@circ@res@up}}
            \pgfnode{ocirc}{center}{}{spdt1}{\pgfusepath{stroke}}
        \endpgfscope
        \pgfscope
            \pgftransformshift{\pgfpoint{-\pgf@circ@res@other}{\pgf@circ@res@down}}
            \pgfnode{ocirc}{center}{}{}{\pgfusepath{stroke}}
        \endpgfscope
        \pgfscope
            \pgftransformshift{\pgfpoint{\pgf@circ@res@other}{0pt}}
            \pgfnode{ocirc}{center}{}{spdt2}{\pgfusepath{stroke}}
        \endpgfscope

        \pgfscope
            \pgfpathmoveto{\pgfpointshapeborder{spdt2}{\pgfpointorigin}}
            \pgfpathlineto{
                \pgfpointadd{\pgfpointshapeborder{spdt1}{\pgfpoint{-\pgf@circ@res@other}{-100pt}}}
                {\pgfpoint{-.05\pgf@circ@res@up}{-.05\pgf@circ@res@up}}
            }
            \pgf@circ@setlinewidth{tripoles}{\pgflinewidth}
            \pgfusepath{draw}
        \endpgfscope
    }
}


% cute switch "node" shapes, matching with cute "to" shapes
% #1 -> name
% #2 -> barposition
% #3 -> arrowcode
\long\def\pgfcircdeclarecutespdt#1#2#3{
    \pgfdeclareshape{#1}
    {
        \savedmacro{\ctikzclass}{\edef\ctikzclass{switches}}
        \saveddimen{\scaledRlen}{\pgfmathsetlength{\pgf@x}{\ctikzvalof{\ctikzclass/scale}\pgf@circ@Rlen}}
        \savedanchor\northwest{%
            \pgfmathsetlength{\pgf@circ@scaled@Rlen}{\ctikzvalof{\ctikzclass/scale}\pgf@circ@Rlen}
            \pgf@y=\ctikzvalof{bipoles/cuteswitch/height}\pgf@circ@scaled@Rlen
            \pgf@y=.5\pgf@y
            \pgf@x=-\ctikzvalof{tripoles/spdt/width}\pgf@circ@scaled@Rlen
            \pgf@x=.25\pgf@x
        }
        \savedanchor\midlever{
            % these values are calculated when we create the definition of the shape.
            \pgfmathsetlength{\pgf@circ@scaled@Rlen}{\ctikzvalof{\ctikzclass/scale}\pgf@circ@Rlen}
            \pgf@y=\ctikzvalof{bipoles/cuteswitch/height}\pgf@circ@scaled@Rlen
            \pgf@circ@res@temp=\ctikzvalof{nodes width}\pgf@circ@scaled@Rlen
            \pgf@circ@res@temp=\ctikzvalof{bipoles/cuteswitch/thickness}\pgf@circ@res@temp
            \pgf@circ@res@down = -.5\pgf@y
            \pgf@circ@res@up = .5\pgf@y
            \pgfextracty{\pgf@circ@res@other}{#2}
            \pgf@x=0pt
            \pgf@y=.5\pgf@circ@res@other
        }
        % radius of the connector
        % This is the radius of the "ocirc" shape (see pgfcircshapes.tex)
        \saveddimen{\radius}{\pgfmathsetlength\pgf@x{\pgf@circ@Rlen*\ctikzvalof{nodes width}}}
        % shapename
        \savedmacro{\thisshape}{\def\thisshape{\tikz@fig@name}}
        % shape type
        \savedmacro{\cshape}{\def\cshape{\ctikzvalof{bipoles/cuteswitch/shape}}}
        % mid of the lever, to stack switches
        \anchor{mid}{\midlever}
        % center anchors
        \anchor{cin}{ \northwest \pgf@y=0pt}
        \anchor{cout 1}{ \northwest \pgf@x=-\pgf@x }
        \anchor{cout 2}{ \northwest \pgf@x=-\pgf@x \pgf@y=-\pgf@y }
        % horizontal angles
        \anchor{in}{ \northwest \pgf@y=0pt\advance\pgf@x by -\radius}
        \anchor{out 1}{ \northwest \pgf@x=-\pgf@x \advance\pgf@x by \radius}
        \anchor{out 2}{ \northwest \pgf@x=-\pgf@x \advance\pgf@x by \radius \pgf@y=-\pgf@y }

        \anchor{center}{ \pgf@y=0pt \pgf@x=0pt }
        \anchor{east}{ \northwest \pgf@y=0pt \pgf@x=-\pgf@x  }
        \anchor{west}{ \northwest \pgf@y=0pt }
        \anchor{south}{ \northwest \pgf@x=0pt \pgf@y=-\pgf@y }
        \anchor{north}{ \northwest \pgf@x=0pt }
        \anchor{south west}{ \northwest \pgf@y=-\pgf@y }
        \anchor{north east}{ \northwest \pgf@x=-\pgf@x }
        \anchor{north west}{ \northwest }
        \anchor{south east}{ \northwest \pgf@x=-\pgf@x \pgf@y=-\pgf@y }

        \backgroundpath{
            \pgfsetcolor{\ctikzvalof{color}}
            \northwest
            \pgf@circ@res@up = \pgf@y
            \pgf@circ@res@down = -\pgf@y
            \pgf@circ@res@right = -\pgf@x
            \pgf@circ@res@left = \pgf@x

            \pgfscope
            % This is the radius of the "ocirc" shape (see pgfcircshapes.tex)
            \pgf@circ@res@temp=\radius\relax
            \pgf@circ@res@temp=\ctikzvalof{bipoles/cuteswitch/thickness}\pgf@circ@res@temp
            \pgfsetlinewidth{2\pgf@circ@res@temp}
            \pgfpathmoveto{\pgfpoint{\pgf@circ@res@left}{0pt}}
            \pgfpathlineto{#2}
            \pgfsetroundcap\pgfusepath{draw}
            \endpgfscope
            \pgfscope % arrow
            #3
            \endpgfscope
            % terminals
            \pgfscope
            \pgftransformshift{\pgfpoint{\pgf@circ@res@right}{\pgf@circ@res@up}}
            \pgfnode{\cshape}{center}{}{\thisshape-out 1}{\pgfusepath{stroke}}
            \endpgfscope
            \pgfscope
            \pgftransformshift{\pgfpoint{\pgf@circ@res@right}{\pgf@circ@res@down}}
            \pgfnode{\cshape}{center}{}{\thisshape-out 2}{\pgfusepath{stroke}}
            \endpgfscope
            \pgfscope
            \pgftransformshift{\pgfpoint{\pgf@circ@res@left}{0pt}}
            \pgfnode{\cshape}{center}{}{\thisshape-in}{\pgfusepath{stroke}}
            \endpgfscope

        }
    }
}

\pgfcircdeclarecutespdt{cute spdt up}
{\pgfpoint{\pgf@circ@res@right}{\pgf@circ@res@up-1.5\pgf@circ@res@temp}}
{}

\pgfcircdeclarecutespdt{cute spdt mid}
{\pgfpoint{\pgf@circ@res@right}{0pt}}
{}

\pgfcircdeclarecutespdt{cute spdt down}
{\pgfpoint{\pgf@circ@res@right}{\pgf@circ@res@down+1.5\pgf@circ@res@temp}}
{}

\pgfcircdeclarecutespdt{cute spdt up arrow}
{\pgfpoint{\pgf@circ@res@right}{\pgf@circ@res@up-1.5\pgf@circ@res@temp}}
{
    \pgf@circ@setlinewidth{bipoles}{\pgflinewidth}
    \pgftransformshift{\pgfpoint{\pgf@circ@res@left}{0pt}} % in node
    \pgfpathmoveto{\pgfpointpolar{70}{1.5\pgf@circ@res@right}}
    \pgfpatharc{70}{-50}{1.5\pgf@circ@res@right}
    \pgfsetarrowsend{latexslim}
    \pgfusepath{draw}
}

\pgfcircdeclarecutespdt{cute spdt mid arrow}
{\pgfpoint{\pgf@circ@res@right}{0pt}}
{
    \pgf@circ@setlinewidth{bipoles}{\pgflinewidth}
    \pgftransformshift{\pgfpoint{\pgf@circ@res@left}{0pt}} % in node
    \pgfsetarrowsstart{latexslim}
    \pgfpathmoveto{\pgfpointpolar{-60}{1.5\pgf@circ@res@right}}
    \pgfpatharc{-60}{60}{1.5\pgf@circ@res@right}
    \pgfsetarrowsend{latexslim}
    \pgfusepath{draw}
}

\pgfcircdeclarecutespdt{cute spdt down arrow}
{\pgfpoint{\pgf@circ@res@right}{\pgf@circ@res@down+1.5\pgf@circ@res@temp}}
{
    \pgf@circ@setlinewidth{bipoles}{\pgflinewidth}
    \pgftransformshift{\pgfpoint{\pgf@circ@res@left}{0pt}} % in node
    \pgfpathmoveto{\pgfpointpolar{-50}{1.5\pgf@circ@res@right}}
    \pgfpatharc{-50}{70}{1.5\pgf@circ@res@right}
    \pgfsetarrowsend{latexslim}
    \pgfusepath{draw}
}

%%%%%%%%%%%%%%%%%%%%%%%%%%%%%%%%%%%%%%%%%%%%%%%%%%%%%%
%%%%%%%%%%%%%%%%%%%%%%%%%%%%%%%%%%%%%%%%%%%%%%%%%%%%%%
%%
%%       Logic ports
%%
%% Code from John Kormylo at tex.stackexchange.com
%% See https://tex.stackexchange.com/questions/372993/is-it-possible-to-implement-multiple-input-logic-ports-with-circuitikz
%% Integration and fixes from Romano Giannetti and TheTeXnician <38565529+TheTeXnician@users.noreply.github.com>
%%

\newcount\pgf@circ@res@count% reserve global register

\def\pgf@circ@logicport@input#1% #1 = \pgfmathcounter
{%
    \pgfextracty{\pgf@circ@res@up}{\northeast}%
    \step
    \pgf@circ@res@step=\dimexpr 2\pgf@y -2\pgf@circ@res@up\relax
    \advance\pgf@y by -#1\pgf@circ@res@step\relax
}%

% #1 = \pgfmathcounter #2=type #3 specificic port
% type is 1 for and,nand; 2 for or,nor; 3 for xor,xnor, 4 for european.
\def\pgf@circ@logicport@baseinput#1#2#3%
{%
    % and and nand
    \ifnum #2=1\relax
        \pgfextracty{\pgf@circ@res@up}{\northeast}%
        \pgfextractx{\pgf@circ@res@left}{\southwest}%
        \step
        \pgf@circ@res@step=\dimexpr 2\pgf@y -2\pgf@circ@res@up\relax
        \advance\pgf@y by -#1\pgf@circ@res@step\relax
        \pgf@x=\ctikzvalof{tripoles/american #3 port/port width}\pgf@circ@res@left
    \fi
    % or and nor
    \ifnum #2=2\relax
        \pgfextracty{\pgf@circ@res@up}{\northeast}%
        \pgfextractx{\pgf@circ@res@left}{\southwest}%
        \pgfextractx{\pgf@circ@res@right}{\northeast}%
        \step
        \pgf@circ@res@step=\dimexpr 2\pgf@y -2\pgf@circ@res@up\relax
        \advance\pgf@y by -#1\pgf@circ@res@step\relax
        \edef\pgf@circ@math@angle{\ctikzvalof{tripoles/american #3 port/angle}}%
        \pgf@circ@res@other=\ctikzvalof{tripoles/american #3 port/inner}\pgf@circ@res@right
        \pgfmathsetlengthmacro{\pgf@circ@math@xradius}{\pgf@circ@res@other /(1 - cos(\pgf@circ@math@angle)}%
        \pgfmathsetlengthmacro{\pgf@circ@math@yradius}{\pgf@circ@res@up /sin(\pgf@circ@math@angle)}%
        \pgf@circ@res@other=\ctikzvalof{tripoles/american #3 port/port width}\pgf@circ@res@right
        \pgfmathsetlengthmacro{\pgf@circ@math@xorigin}{\pgf@circ@res@other + \pgf@circ@math@xradius*cos(\pgf@circ@math@angle)}%
        \pgf@circ@res@temp=\pgf@y
        \pgfmathsetmacro{\pgf@circ@math@angle}{asin(\pgf@circ@res@temp/\pgf@circ@math@yradius)}%
        \pgfmathsetlength{\pgf@circ@res@other}{\pgf@circ@math@xradius*cos(\pgf@circ@math@angle)-\pgf@circ@math@xorigin}%
        \pgf@x=\pgf@circ@res@other
    \fi
    % xor and xnor
    \ifnum #2=3\relax
        \pgfextracty{\pgf@circ@res@up}{\northeast}%
        \pgfextractx{\pgf@circ@res@left}{\southwest}%
        \pgfextractx{\pgf@circ@res@right}{\northeast}%
        \pgfkeysgetvalue{/tikz/circuitikz/tripoles/american #3 port/angle}{\pgf@circ@math@angle}%
        \pgf@circ@res@other=\ctikzvalof{tripoles/american #3 port/inner}\pgf@circ@res@right
        \pgfmathsetlengthmacro{\pgf@circ@math@xradius}{\pgf@circ@res@other /(1 - cos(\pgf@circ@math@angle)}%
        \pgfmathsetlengthmacro{\pgf@circ@math@yradius}{\pgf@circ@res@up / sin(\pgf@circ@math@angle))}%
        \pgf@circ@res@other=\ctikzvalof{tripoles/american #3 port/port width}\pgf@circ@res@right
        \pgfmathsetlengthmacro{\pgf@circ@math@xorigin}{\pgf@circ@res@other + \pgf@circ@math@xradius*cos(\pgf@circ@math@angle)}%
        \pgf@circ@res@temp=\ctikzvalof{tripoles/american #3 port/distance}\pgf@circ@res@right
        \pgfmathsetlengthmacro{\pgf@circ@math@distance}{\pgf@circ@res@temp}
        % this compensates for the effect of the line width on the gap between the arcs
        \pgfmathsetlengthmacro{\pgf@circ@math@yradiusA}{\pgf@circ@math@yradius -2\pgflinewidth}%
        \pgfmathsetlengthmacro{\pgf@circ@math@xradiusA}{\pgf@circ@math@xradius -2\pgflinewidth}%

        \step
        \pgf@circ@res@step=\dimexpr 2\pgf@y -2\pgf@circ@res@up\relax
        \advance\pgf@y by -#1\pgf@circ@res@step\relax
         \pgf@circ@res@temp=\pgf@y
            \pgfmathsetmacro{\pgf@circ@math@angle}{asin(\pgf@circ@res@temp  / \pgf@circ@math@yradiusA)}%
        \pgfmathsetlength{\pgf@circ@res@other}{\pgf@circ@math@xradiusA*cos(\pgf@circ@math@angle)-\pgf@circ@math@xorigin}%
        \advance\pgf@circ@res@other by -\pgf@circ@math@distance
        \pgf@x=\pgf@circ@res@other
    \fi
    % european
    \ifnum #2=4\relax
        \pgfextracty{\pgf@circ@res@up}{\northeast}%
        \pgfextractx{\pgf@circ@res@other}{\left}%
        \step
        \pgf@circ@res@step=\dimexpr 2\pgf@y -2\pgf@circ@res@up\relax
        \advance\pgf@y by -#1\pgf@circ@res@step\relax
        \pgf@x=\pgf@circ@res@other
    \fi
}%

%%% american
\long\def\pgfcircdeclarelogicport#1#2#3{%
    \pgfdeclareshape{american #1 port}%
    {%
        \savedmacro{\ctikzclass}{\edef\ctikzclass{logic ports}}
        \saveddimen{\scaledRlen}{\pgfmathsetlength{\pgf@x}{\ctikzvalof{\ctikzclass/scale}\pgf@circ@Rlen}}
        \savedmacro\resize{% automatic
            \pgfmathsetlength{\pgf@circ@scaled@Rlen}{\ctikzvalof{\ctikzclass/scale}\pgf@circ@Rlen}
            \pgf@circ@res@up = \ctikzvalof{tripoles/american #1 port/height}\pgf@circ@scaled@Rlen
            \pgf@circ@res@up = .5\pgf@circ@res@up
            \pgf@circ@res@down = -\pgf@circ@res@up
            \pgf@circ@res@right = \ctikzvalof{tripoles/american #1 port/width}\pgf@circ@scaled@Rlen
            \pgf@circ@res@right = .5\pgf@circ@res@right
            \pgf@circ@res@left = -\pgf@circ@res@right
    }%
    \savedmacro\inputs{% get number of inputs
        \pgf@circ@res@count=\pgfkeysvalueof{/tikz/number inputs}\relax%
        \ifnum\pgf@circ@res@count=0
            \pgf@circ@res@count=\ctikzvalof{tripoles/american #1 port/inputs}\relax%
        \fi
        \ifnum\pgf@circ@res@count<2 \pgf@circ@res@count=2\fi
        \ifnum\pgf@circ@res@count>16 \pgf@circ@res@count=16\fi
        \def\inputs{\the\pgf@circ@res@count}%
    }%
    \savedanchor\step{% 1/2 gap at edges
        \pgfmathsetlength{\pgf@circ@scaled@Rlen}{\ctikzvalof{\ctikzclass/scale}\pgf@circ@Rlen}
        \pgf@circ@res@step = \ctikzvalof{tripoles/american #1 port/height}\pgf@circ@scaled@Rlen
        \divide\pgf@circ@res@step by \pgf@circ@res@count
        \pgfpoint{\pgf@circ@res@left}{\dimexpr\pgf@circ@res@up+0.5\pgf@circ@res@step}%
    }%
    \savedanchor\northeast{\pgfpoint{\pgf@circ@res@right}{\pgf@circ@res@up}}
    \savedanchor\southwest{\pgfpoint{\pgf@circ@res@left}{\pgf@circ@res@down}}
    \savedanchor\left{\pgfpoint{\ctikzvalof{tripoles/american #1 port/port width}\pgf@circ@res@left}{0pt}}
    \savedanchor\right{\pgfpoint{\ctikzvalof{tripoles/american #1 port/port width}\pgf@circ@res@right}{0pt}}
    \savedanchor\origin{\pgfpoint{\ctikzvalof{tripoles/american #1 port/origin}\pgf@circ@res@right}{0pt}}

    \anchor{center}{\origin}% for backwards compatibility
    \anchor{text}{\pgfpoint{-.5\wd\pgfnodeparttextbox}{\dimexpr.5\dp\pgfnodeparttextbox-.5\ht\pgfnodeparttextbox}}

    % create input anchors
    \expandafter\pgfutil@g@addto@macro\csname pgf@sh@s@american #1 port\endcsname{%
        \pgfmathloop%
        \ifnum\pgfmathcounter>\pgf@circ@res@count%
    \else%
        %\pgfutil@ifundefined{pgf@anchor@american #1 port@in \pgfmathcounter}{%
        \expandafter\xdef\csname pgf@anchor@american #1 port@in \pgfmathcounter\endcsname{%
            \noexpand\pgf@circ@logicport@input{\pgfmathcounter}% defined above
        }%
        \expandafter\xdef\csname pgf@anchor@american #1 port@bin \pgfmathcounter\endcsname{%
            \noexpand\pgf@circ@logicport@baseinput{\pgfmathcounter}{#2}{#1}% defined above
        }%
        %}{}%
        \repeatpgfmathloop%
    }

    \anchor{out}{\northeast\pgf@y=0pt}
    \anchor{bout}{\right\pgf@y=0pt}


    \anchor{left}{\left}% edges of component minus leads
    \anchor{right}{\right}

    \anchor{north east}{\northeast}% see \Compass macro
    \anchor{south west}{\southwest}
    \anchor{north}{\pgfextracty{\pgf@circ@res@up}{\northeast}%
    \pgfpoint{0cm}{\pgf@circ@res@up}}
    \anchor{north west}{\pgfextracty{\pgf@circ@res@up}{\northeast}%
        \pgfextractx{\pgf@circ@res@left}{\southwest}%
    \pgfpoint{\pgf@circ@res@left}{\pgf@circ@res@up}}
    \anchor{west}{\pgfextractx{\pgf@circ@res@left}{\southwest}%
    \pgfpoint{\pgf@circ@res@left}{0cm}}
    \anchor{south}{\pgfextracty{\pgf@circ@res@down}{\southwest}%
    \pgfpoint{0cm}{\pgf@circ@res@down}}
    \anchor{south east}{\pgfextracty{\pgf@circ@res@down}{\southwest}%
        \pgfextractx{\pgf@circ@res@right}{\northeast}%
    \pgfpoint{\pgf@circ@res@right}{\pgf@circ@res@down}}
    \anchor{east}{\pgfextractx{\pgf@circ@res@right}{\northeast}%
    \pgfpoint{\pgf@circ@res@right}{0cm}}

    \backgroundpath{
        \pgfsetcolor{\ctikzvalof{color}}
        #3
    }
}
}
%%% american and %%%%%%%%%%%%%%%%%%%%%%%%%%%%%%%%%%
\pgfcircdeclarelogicport{and}{1}{
    \pgfextracty{\pgf@circ@res@temp}{\step}%
    \pgf@circ@res@step = \dimexpr 2\pgf@circ@res@temp -2\pgf@circ@res@up\relax
    \pgf@circ@res@count = \inputs\relax
    \ifpgfcirc@draw@leads
        %input leads
        \loop\ifnum\pgf@circ@res@count>0
            \advance\pgf@circ@res@temp by -\pgf@circ@res@step
            \pgfpathmoveto{\pgfpoint{\pgf@circ@res@left}{\pgf@circ@res@temp}}%
            \pgfpathlineto{\pgfpoint
                {\ctikzvalof{tripoles/american and port/port width}\pgf@circ@res@left}
            {\pgf@circ@res@temp}}
            \advance\pgf@circ@res@count by -1
        \repeat
    \fi
    % output lead
    \pgfpathmoveto{\pgfpoint{\pgf@circ@res@right}{0pt}}
    \pgfpathlineto{\pgfpoint
        {\ctikzvalof{tripoles/american and port/port width}\pgf@circ@res@right}
    {0pt}}
    \pgfusepath{draw}


    \pgf@circ@setlinewidth{tripoles}{\pgflinewidth}
    \pgf@circ@res@other=\ctikzvalof{tripoles/american and port/port width}\pgf@circ@res@left

    \pgfpathmoveto{\pgfpoint{\pgf@circ@res@other}{\pgf@circ@res@up}}
    \pgfpathlineto{\pgfpoint{\pgf@circ@res@other}{\pgf@circ@res@down}}
    \pgfpatharc{-90}{90}{-2\pgf@circ@res@other and \pgf@circ@res@up}
    \pgfpathclose
    \pgf@circ@draworfill
    }
%%% american nand %%%%%%%%%%%%%%%%%%%%%%%%%%%%%%%%%%%%%%%%%%%%%
    \pgfcircdeclarelogicport{nand}{1}{
    \pgfextracty{\pgf@circ@res@temp}{\step}%
    \pgf@circ@res@step = \dimexpr 2\pgf@circ@res@temp -2\pgf@circ@res@up\relax
    \pgf@circ@res@count = \inputs\relax
    \ifpgfcirc@draw@leads
        %input leads
        \loop\ifnum\pgf@circ@res@count>0
            \advance\pgf@circ@res@temp by -\pgf@circ@res@step
            \pgfpathmoveto{\pgfpoint{\pgf@circ@res@left}{\pgf@circ@res@temp}}%
            \pgfpathlineto{\pgfpoint
                {\ctikzvalof{tripoles/american nand port/port width}\pgf@circ@res@left}
            {\pgf@circ@res@temp}}
            \advance\pgf@circ@res@count by -1
        \repeat
    \fi

    \pgfpathmoveto{\pgfpoint{\pgf@circ@res@right}{0pt}}
    \pgfpathlineto{\pgfpoint
        {\ctikzvalof{tripoles/american nand port/port width}\pgf@circ@res@right} {0pt}}

    \pgfusepath{draw}
    \pgf@circ@setlinewidth{tripoles}{\pgflinewidth}
    \pgf@circ@res@step = \ctikzvalof{tripoles/american nand port/circle width}\pgf@circ@res@right
    \pgf@circ@res@other = \ctikzvalof{tripoles/american nand port/port width}\pgf@circ@res@right
    \pgf@circ@res@temp = \dimexpr 2\pgf@circ@res@other - \pgf@circ@res@step\relax

    \pgfpathmoveto{\pgfpoint{-\pgf@circ@res@other}{\pgf@circ@res@up}}
    \pgfpathlineto{\pgfpoint{-\pgf@circ@res@other}{\pgf@circ@res@down}}
    \pgfpatharc{-90}{90}{\pgf@circ@res@temp and \pgf@circ@res@up}
    \pgfpathclose

    \pgfpathellipse
    {\pgfpoint{\pgf@circ@res@other-.5\pgf@circ@res@step}{0pt}}
    {\pgfpoint{.5\pgf@circ@res@step}{0pt}}
    {\pgfpoint{0pt}{.5\pgf@circ@res@step}}

    \pgf@circ@draworfill
}
%%% american nor %%%%%%%%%%%%%%%%%%%%%%%%%%%%%%%%%%%%%%%%%%%%%%%%%%%%%%%
\pgfcircdeclarelogicport{nor}{2}{
    \edef\pgf@circ@math@angle{\ctikzvalof{tripoles/american nor port/angle}}%
    \pgf@circ@res@other=\ctikzvalof{tripoles/american nor port/inner}\pgf@circ@res@right
    \pgfmathsetlengthmacro{\pgf@circ@math@xradius}{\pgf@circ@res@other /(1 - cos(\pgf@circ@math@angle)}%
    \pgfmathsetlengthmacro{\pgf@circ@math@yradius}{\pgf@circ@res@up /sin(\pgf@circ@math@angle)}%
    \pgf@circ@res@other=\ctikzvalof{tripoles/american nor port/port width}\pgf@circ@res@right
    \pgfmathsetlengthmacro{\pgf@circ@math@xorigin}{\pgf@circ@res@other + \pgf@circ@math@xradius*cos(\pgf@circ@math@angle)}%

    \pgfextracty{\pgf@circ@res@temp}{\step}%
    \pgf@circ@res@step = \dimexpr 2\pgf@circ@res@temp -2\pgf@circ@res@up\relax
    \pgf@circ@res@count = \inputs\relax
    \ifpgfcirc@draw@leads
        %input leads
        \loop\ifnum\pgf@circ@res@count>0
            \advance\pgf@circ@res@temp by -\pgf@circ@res@step
            \pgfpathmoveto{\pgfpoint{\pgf@circ@res@left}{\pgf@circ@res@temp}}%
            \pgfmathsetmacro{\pgf@circ@math@angle}{asin(\pgf@circ@res@temp/\pgf@circ@math@yradius)}%
            \pgfmathsetlength{\pgf@circ@res@other}{\pgf@circ@math@xradius*cos(\pgf@circ@math@angle)-\pgf@circ@math@xorigin}%
            \pgfpathlineto{\pgfpoint{\pgf@circ@res@other}{\pgf@circ@res@temp}}%
            \advance\pgf@circ@res@count by -1
        \repeat
    \fi

    \pgf@circ@res@other=\ctikzvalof{tripoles/american nor port/port width}\pgf@circ@res@right
    \pgfpathmoveto{\pgfpoint{\pgf@circ@res@right}{0pt}}
    \pgfpathlineto{\pgfpoint{\pgf@circ@res@other}{0pt}}

    \pgfusepath{draw}

    \pgf@circ@setlinewidth{tripoles}{\pgflinewidth}

    \edef\pgf@circ@math@angle{\ctikzvalof{tripoles/american nor port/angle}}%
    \pgf@circ@res@step = \ctikzvalof{tripoles/american nor port/circle width}\pgf@circ@res@right
    \pgf@circ@res@temp = \dimexpr 2\pgf@circ@res@other - \pgf@circ@res@step\relax
    \advance\pgf@circ@res@other by -\pgf@circ@res@step

    % main shape
    \ifpgfcirc@roundy@or@shapes
        \pgfpathmoveto{\pgfpoint{\pgf@circ@res@other}{0pt}}% for symmetry
        \pgfpatharc{0}{90}{\pgf@circ@res@temp and \pgf@circ@res@up}%
        \pgfpatharc{\pgf@circ@math@angle}{-\pgf@circ@math@angle}{\pgf@circ@math@xradius and \pgf@circ@math@yradius}%
        \pgfpatharc{-90}{0}{\pgf@circ@res@temp and \pgf@circ@res@up}%
        \pgfpathclose
    \else
        \pgfmathsetlength{\pgf@circ@res@temp}{\pgf@circ@math@xradius*cos(\pgf@circ@math@angle)-\pgf@circ@math@xorigin}%
        \pgfpathmoveto{\pgfpoint{\pgf@circ@res@temp}{\pgf@circ@res@up}}
        \pgfpatharc{\pgf@circ@math@angle}{-\pgf@circ@math@angle}{\pgf@circ@math@xradius and \pgf@circ@math@yradius}%
        \pgfpathcurveto{\pgfpoint{0.3\pgf@circ@res@right}{\pgf@circ@res@down}}{\pgfpoint{0.3\pgf@circ@res@right}{0.5\pgf@circ@res@down}}{\pgfpoint{\pgf@circ@res@other}{0pt}}
        \pgfpathcurveto{\pgfpoint{0.3\pgf@circ@res@right}{0.5\pgf@circ@res@up}}{\pgfpoint{0.3\pgf@circ@res@right}{\pgf@circ@res@up}}{\pgfpoint{\pgf@circ@res@temp}{\pgf@circ@res@up}}
        \pgfpathclose
    \fi

    % not dot
    \pgfpathellipse
    {\pgfpoint{\pgf@circ@res@other+.5\pgf@circ@res@step}{0pt}}
    {\pgfpoint{.5\pgf@circ@res@step}{0pt}}
    {\pgfpoint{0pt}{.5\pgf@circ@res@step}}

    \pgf@circ@draworfill
}
%%% american or %%%%%%%%%%%%%%%%%%%%%%%%%%%%%%%%%%%%%%%%%%%%%%%%%%%%
\pgfcircdeclarelogicport{or}{2}{
    \edef\pgf@circ@math@angle{\ctikzvalof{tripoles/american or port/angle}}%
    \pgf@circ@res@other=\ctikzvalof{tripoles/american or port/inner}\pgf@circ@res@right
    \pgfmathsetlengthmacro{\pgf@circ@math@xradius}{\pgf@circ@res@other /(1 - cos(\pgf@circ@math@angle)}%
    \pgfmathsetlengthmacro{\pgf@circ@math@yradius}{\pgf@circ@res@up /sin(\pgf@circ@math@angle)}%
    \pgf@circ@res@other=\ctikzvalof{tripoles/american or port/port width}\pgf@circ@res@right
    \pgfmathsetlengthmacro{\pgf@circ@math@xorigin}{\pgf@circ@res@other + \pgf@circ@math@xradius*cos(\pgf@circ@math@angle)}%

    \pgfextracty{\pgf@circ@res@temp}{\step}%
    \pgf@circ@res@step = \dimexpr 2\pgf@circ@res@temp -2\pgf@circ@res@up\relax
    \pgf@circ@res@count = \inputs\relax
    \ifpgfcirc@draw@leads
        %input leads
        \loop\ifnum\pgf@circ@res@count>0
            \advance\pgf@circ@res@temp by -\pgf@circ@res@step
            \pgfpathmoveto{\pgfpoint{\pgf@circ@res@left}{\pgf@circ@res@temp}}%
            \pgfmathsetmacro{\pgf@circ@math@angle}{asin(\pgf@circ@res@temp/\pgf@circ@math@yradius)}%
            \pgfmathsetlength{\pgf@circ@res@other}{\pgf@circ@math@xradius*cos(\pgf@circ@math@angle)-\pgf@circ@math@xorigin}%
            \pgfpathlineto{\pgfpoint{\pgf@circ@res@other}{\pgf@circ@res@temp}}%
            \advance\pgf@circ@res@count by -1
        \repeat
    \fi

    \pgf@circ@res@other=\ctikzvalof{tripoles/american or port/port width}\pgf@circ@res@right
    \pgfpathmoveto{\pgfpoint{\pgf@circ@res@right}{0pt}}
    \pgfpathlineto{\pgfpoint{\pgf@circ@res@other}{0pt}}

    \pgfusepath{draw}

    \pgf@circ@setlinewidth{tripoles}{\pgflinewidth}

    \edef\pgf@circ@math@angle{\ctikzvalof{tripoles/american or port/angle}}%

    % main shape
    \ifpgfcirc@roundy@or@shapes
        \pgfpathmoveto{\pgfpoint{\pgf@circ@res@other}{0pt}}% for symmetry
        \pgfpatharc{0}{90}{2\pgf@circ@res@other and \pgf@circ@res@up}%
        \pgfpatharc{\pgf@circ@math@angle}{-\pgf@circ@math@angle}{\pgf@circ@math@xradius and \pgf@circ@math@yradius}%
        \pgfpatharc{-90}{0}{2\pgf@circ@res@other and \pgf@circ@res@up}%
        \pgfpathclose
    \else
        \pgfmathsetlength{\pgf@circ@res@temp}{\pgf@circ@math@xradius*cos(\pgf@circ@math@angle)-\pgf@circ@math@xorigin}%
        \pgfpathmoveto{\pgfpoint{\pgf@circ@res@temp}{\pgf@circ@res@up}}
        \pgfpatharc{\pgf@circ@math@angle}{-\pgf@circ@math@angle}{\pgf@circ@math@xradius and \pgf@circ@math@yradius}%
        \pgfpathcurveto{\pgfpoint{0.3\pgf@circ@res@right}{\pgf@circ@res@down}}{\pgfpoint{0.3\pgf@circ@res@right}{0.5\pgf@circ@res@down}}{\pgfpoint{\pgf@circ@res@other}{0pt}}
        \pgfpathcurveto{\pgfpoint{0.3\pgf@circ@res@right}{0.5\pgf@circ@res@up}}{\pgfpoint{0.3\pgf@circ@res@right}{\pgf@circ@res@up}}{\pgfpoint{\pgf@circ@res@temp}{\pgf@circ@res@up}}
        \pgfpathclose
    \fi

    \pgf@circ@draworfill
}
%%% american xor %%%%%%%%%%%%%%%%%%%%%%%%%%%%%%%%%%%%%%%%%%%%%%%
\pgfcircdeclarelogicport{xor}{3}{
    \pgfkeysgetvalue{/tikz/circuitikz/tripoles/american xor port/angle}{\pgf@circ@math@angle}%
    \pgf@circ@res@other=\ctikzvalof{tripoles/american xor port/inner}\pgf@circ@res@right
    \pgfmathsetlengthmacro{\pgf@circ@math@xradius}{\pgf@circ@res@other /(1 - cos(\pgf@circ@math@angle)}%
    \pgfmathsetlengthmacro{\pgf@circ@math@yradius}{\pgf@circ@res@up / sin(\pgf@circ@math@angle))}%
    \pgf@circ@res@other=\ctikzvalof{tripoles/american xor port/port width}\pgf@circ@res@right
    \pgfmathsetlengthmacro{\pgf@circ@math@xorigin}{\pgf@circ@res@other + \pgf@circ@math@xradius*cos(\pgf@circ@math@angle)}%
    \pgf@circ@res@temp=\ctikzvalof{tripoles/american xor port/distance}\pgf@circ@res@right
    \pgfmathsetlengthmacro{\pgf@circ@math@distance}{\pgf@circ@res@temp}
    % this compensates for the effect of the line width on the gap between the arcs
    \pgfmathsetlengthmacro{\pgf@circ@math@yradiusA}{\pgf@circ@math@yradius -2\pgflinewidth}%
    \pgfmathsetlengthmacro{\pgf@circ@math@xradiusA}{\pgf@circ@math@xradius -2\pgflinewidth}%

    \pgfextracty{\pgf@circ@res@temp}{\step}%
    \pgf@circ@res@step = \dimexpr 2\pgf@circ@res@temp -2\pgf@circ@res@up\relax
    \pgf@circ@res@count = \inputs\relax
    \ifpgfcirc@draw@leads
        %input leads
        \loop\ifnum\pgf@circ@res@count>0
            \advance\pgf@circ@res@temp by -\pgf@circ@res@step
            \pgfpathmoveto{\pgfpoint{\pgf@circ@res@left}{\pgf@circ@res@temp}}%
            \pgfmathsetmacro{\pgf@circ@math@angle}{asin(\pgf@circ@res@temp  / \pgf@circ@math@yradiusA)}%
            \pgfmathsetlength{\pgf@circ@res@other}{\pgf@circ@math@xradiusA*cos(\pgf@circ@math@angle)-\pgf@circ@math@xorigin}%
            \advance\pgf@circ@res@other by -\pgf@circ@math@distance
            \pgfpathlineto{\pgfpoint{\pgf@circ@res@other}{\pgf@circ@res@temp}}%
            \advance\pgf@circ@res@count by -1
        \repeat
    \fi

    \pgf@circ@res@other=\ctikzvalof{tripoles/american xor port/port width}\pgf@circ@res@right
    \pgfpathmoveto{\pgfpoint{\pgf@circ@res@right}{0pt}}
    \pgfpathlineto{\pgfpoint{\pgf@circ@res@other}{0pt}}

    \pgfusepath{draw}

    \pgf@circ@setlinewidth{tripoles}{\pgflinewidth}

    \edef\pgf@circ@math@angle{\ctikzvalof{tripoles/american xor port/angle}}%

    % main shape
    \ifpgfcirc@roundy@or@shapes
        \pgfpathmoveto{\pgfpoint{\pgf@circ@res@other}{0pt}}% for symmetry
        \pgfpatharc{0}{90}{2\pgf@circ@res@other and \pgf@circ@res@up}%
        \pgfpatharc{\pgf@circ@math@angle}{-\pgf@circ@math@angle}{\pgf@circ@math@xradius and \pgf@circ@math@yradius}%
        \pgfpatharc{-90}{0}{2\pgf@circ@res@other and \pgf@circ@res@up}%
        \pgfpathclose
    \else
        \pgfmathsetlength{\pgf@circ@res@temp}{\pgf@circ@math@xradius*cos(\pgf@circ@math@angle)-\pgf@circ@math@xorigin}%
        \pgfpathmoveto{\pgfpoint{\pgf@circ@res@temp}{\pgf@circ@res@up}}
        \pgfpatharc{\pgf@circ@math@angle}{-\pgf@circ@math@angle}{\pgf@circ@math@xradius and \pgf@circ@math@yradius}%
        \pgfpathcurveto{\pgfpoint{0.3\pgf@circ@res@right}{\pgf@circ@res@down}}{\pgfpoint{0.3\pgf@circ@res@right}{0.5\pgf@circ@res@down}}{\pgfpoint{\pgf@circ@res@other}{0pt}}
        \pgfpathcurveto{\pgfpoint{0.3\pgf@circ@res@right}{0.5\pgf@circ@res@up}}{\pgfpoint{0.3\pgf@circ@res@right}{\pgf@circ@res@up}}{\pgfpoint{\pgf@circ@res@temp}{\pgf@circ@res@up}}
        \pgfpathclose
    \fi
    \pgf@circ@draworfill

    \pgfmathsetlength{\pgf@circ@res@temp}{(\pgf@circ@math@yradiusA)*sin(\pgf@circ@math@angle)}%

    \pgfpathmoveto{\pgfpoint{-\pgf@circ@res@other-\pgf@circ@math@distance}{\pgf@circ@res@temp}}% first arc
    \pgfpatharc{\pgf@circ@math@angle}{-\pgf@circ@math@angle}{\pgf@circ@math@xradiusA and \pgf@circ@math@yradiusA}%

    \pgfusepath{draw}
}
%%% american xnor %%%%%%%%%%%%%%%%%%%%%%%%%%%%%%%%%%%%%%%%%%%%%%%%
\pgfcircdeclarelogicport{xnor}{3}{
    \pgfkeysgetvalue{/tikz/circuitikz/tripoles/american xnor port/angle}{\pgf@circ@math@angle}%
    \pgf@circ@res@other=\ctikzvalof{tripoles/american xnor port/inner}\pgf@circ@res@right
    \pgfmathsetlengthmacro{\pgf@circ@math@xradius}{\pgf@circ@res@other /(1 - cos(\pgf@circ@math@angle)}%
    \pgfmathsetlengthmacro{\pgf@circ@math@yradius}{\pgf@circ@res@up / sin(\pgf@circ@math@angle))}%
    \pgf@circ@res@other=\ctikzvalof{tripoles/american xnor port/port width}\pgf@circ@res@right
    \pgfmathsetlengthmacro{\pgf@circ@math@xorigin}{\pgf@circ@res@other + \pgf@circ@math@xradius*cos(\pgf@circ@math@angle)}%
    \pgf@circ@res@temp=\ctikzvalof{tripoles/american xor port/distance}\pgf@circ@res@right
    \pgfmathsetlengthmacro{\pgf@circ@math@distance}{\pgf@circ@res@temp}
    % this compensates for the effect of the line width on the gap between the arcs
    \pgfmathsetlengthmacro{\pgf@circ@math@yradiusA}{\pgf@circ@math@yradius -2\pgflinewidth}%
    \pgfmathsetlengthmacro{\pgf@circ@math@xradiusA}{\pgf@circ@math@xradius -2\pgflinewidth}%

    \pgfextracty{\pgf@circ@res@temp}{\step}%
    \pgf@circ@res@step = \dimexpr 2\pgf@circ@res@temp -2\pgf@circ@res@up\relax
    \pgf@circ@res@count = \inputs\relax
    \ifpgfcirc@draw@leads
        %input leads
        \loop\ifnum\pgf@circ@res@count>0
            \advance\pgf@circ@res@temp by -\pgf@circ@res@step
            \pgfpathmoveto{\pgfpoint{\pgf@circ@res@left}{\pgf@circ@res@temp}}%
            \pgfmathsetmacro{\pgf@circ@math@angle}{asin(\pgf@circ@res@temp  / \pgf@circ@math@yradiusA)}%
            \pgfmathsetlength{\pgf@circ@res@other}{\pgf@circ@math@xradiusA*cos(\pgf@circ@math@angle)-\pgf@circ@math@xorigin}%
            \advance\pgf@circ@res@other by -\pgf@circ@math@distance
            \pgfpathlineto{\pgfpoint{\pgf@circ@res@other}{\pgf@circ@res@temp}}%
            \advance\pgf@circ@res@count by -1
        \repeat
    \fi

    \pgf@circ@res@other=\ctikzvalof{tripoles/american xnor port/port width}\pgf@circ@res@right
    \pgfpathmoveto{\pgfpoint{\pgf@circ@res@right}{0pt}}
    \pgfpathlineto{\pgfpoint{\pgf@circ@res@other}{0pt}}

    \pgfusepath{draw}

    \pgf@circ@setlinewidth{tripoles}{\pgflinewidth}

    \edef\pgf@circ@math@angle{\ctikzvalof{tripoles/american xnor port/angle}}%
    \pgf@circ@res@step = \ctikzvalof{tripoles/american xnor port/circle width}\pgf@circ@res@right
    \pgf@circ@res@temp = \dimexpr 2\pgf@circ@res@other - \pgf@circ@res@step\relax
    \advance\pgf@circ@res@other by -\pgf@circ@res@step

    % main shape
    \ifpgfcirc@roundy@or@shapes
        \pgfpathmoveto{\pgfpoint{\pgf@circ@res@other}{0pt}}% for symmetry
        \pgfpatharc{0}{90}{\pgf@circ@res@temp and \pgf@circ@res@up}%
        \pgfpatharc{\pgf@circ@math@angle}{-\pgf@circ@math@angle}{\pgf@circ@math@xradius and \pgf@circ@math@yradius}%
        \pgfpatharc{-90}{0}{\pgf@circ@res@temp and \pgf@circ@res@up}%
        \pgfpathclose
    \else
        \pgfmathsetlength{\pgf@circ@res@temp}{\pgf@circ@math@xradius*cos(\pgf@circ@math@angle)-\pgf@circ@math@xorigin}%
        \pgfpathmoveto{\pgfpoint{\pgf@circ@res@temp}{\pgf@circ@res@up}}
        \pgfpatharc{\pgf@circ@math@angle}{-\pgf@circ@math@angle}{\pgf@circ@math@xradius and \pgf@circ@math@yradius}%
        \pgfpathcurveto{\pgfpoint{0.3\pgf@circ@res@right}{\pgf@circ@res@down}}{\pgfpoint{0.3\pgf@circ@res@right}{0.5\pgf@circ@res@down}}{\pgfpoint{\pgf@circ@res@other}{0pt}}
        \pgfpathcurveto{\pgfpoint{0.3\pgf@circ@res@right}{0.5\pgf@circ@res@up}}{\pgfpoint{0.3\pgf@circ@res@right}{\pgf@circ@res@up}}{\pgfpoint{\pgf@circ@res@temp}{\pgf@circ@res@up}}
        \pgfpathclose
    \fi

    \pgfpathellipse
    {\pgfpoint{\pgf@circ@res@other+.5\pgf@circ@res@step}{0pt}}
    {\pgfpoint{.5\pgf@circ@res@step}{0pt}}
    {\pgfpoint{0pt}{.5\pgf@circ@res@step}}
    \pgf@circ@draworfill

    \pgf@circ@res@other=\ctikzvalof{tripoles/american xnor port/port width}\pgf@circ@res@left
    \pgfmathsetlength{\pgf@circ@res@temp}{(\pgf@circ@math@yradiusA)*sin(\pgf@circ@math@angle)}%

    \pgfpathmoveto{\pgfpoint{\pgf@circ@res@other-\pgf@circ@math@distance}{\pgf@circ@res@temp}}% first arc
    \pgfpatharc{\pgf@circ@math@angle}{-\pgf@circ@math@angle}{\pgf@circ@math@xradiusA and \pgf@circ@math@yradiusA}%

    \pgfusepath{draw}
}

%%% Original one-input ports

\pgfdeclareshape{american not port}{
    \savedmacro{\ctikzclass}{\edef\ctikzclass{logic ports}}
    \saveddimen{\scaledRlen}{\pgfmathsetlength{\pgf@x}{\ctikzvalof{\ctikzclass/scale}\pgf@circ@Rlen}}
    \anchor{center}{\pgfpointorigin}
    \savedanchor\northwest{%
        \pgfmathsetlength{\pgf@circ@scaled@Rlen}{\ctikzvalof{\ctikzclass/scale}\pgf@circ@Rlen}
        \pgf@y=\ctikzvalof{bipoles/not port/height}\pgf@circ@scaled@Rlen
        \pgf@y=.5\pgf@y
        \pgf@x=-\ctikzvalof{bipoles/not port/width}\pgf@circ@scaled@Rlen
        \pgf@x=.5\pgf@x
    }
    \anchor{in}{
        \northwest
        \pgf@y=0pt
    }
    \anchor{in 1}{
        \northwest
        \pgf@y=0pt
    }
    \anchor{bin}{
        \northwest
        \pgf@y=0pt
    }
    \anchor{bin 1}{
        \northwest
        \pgf@y=0pt
        \pgf@x=0.7\pgf@x
    }
    \anchor{out}{
        \northwest
        \pgf@y=0pt
        \pgf@x=-\pgf@x
    }
    \anchor{bout}{
        \northwest
        \pgf@y=0pt
        \pgf@x=-0.7\pgf@x
    }
    \anchor{center}{
        \pgfpointorigin
    }
    \anchor{east}{
        \northwest
        \pgf@y=0pt
        \pgf@x=-\pgf@x
    }
    \anchor{west}{
        \northwest
        \pgf@y=0pt
    }
    \anchor{south}{
        \northwest
        \pgf@x=0pt
        \pgf@y=-\pgf@y
    }
    \anchor{north}{
        \northwest
        \pgf@x=0pt
    }
    \anchor{south west}{
        \northwest
        \pgf@y=-\pgf@y
    }
    \anchor{north east}{
        \northwest
        \pgf@x=-\pgf@x
    }
    \anchor{north west}{
        \northwest
    }
    \anchor{south east}{
        \northwest
        \pgf@x=-\pgf@x
        \pgf@y=-\pgf@y
    }
    \backgroundpath{
        \pgfsetcolor{\ctikzvalof{color}}

        \northwest
        \pgf@circ@res@up = \pgf@y
        \pgf@circ@res@down = -\pgf@y
        \pgf@circ@res@right = -\pgf@x
        \pgf@circ@res@left = \pgf@x

        \pgf@circ@res@other = \ctikzvalof{bipoles/not port/circle width}\pgf@circ@res@right

        \pgfscope
            \pgf@circ@setlinewidth{tripoles}{\pgflinewidth}
            \pgftransformxshift{.7\pgf@circ@res@left}
            \pgf@circ@res@step=\pgf@circ@res@right
            \advance\pgf@circ@res@step by -\pgf@circ@res@left
            \pgf@circ@res@step=.7\pgf@circ@res@step

            \pgfpathmoveto{\pgfpoint{\pgf@circ@res@step-\pgf@circ@res@other}{0pt}}
            \pgfpathlineto{\pgfpoint{0pt}{\pgf@circ@res@up}}
            \pgfpathlineto{\pgfpoint{0pt}{\pgf@circ@res@down}}
            \pgfpathlineto{\pgfpoint{\pgf@circ@res@step-\pgf@circ@res@other}{0pt}}
            \pgf@circ@draworfill
            \pgfpathellipse
            {\pgfpoint{\pgf@circ@res@step-.5\pgf@circ@res@other}{0pt}}
            {\pgfpoint{.5\pgf@circ@res@other}{0pt}}
            {\pgfpoint{0pt}{.5\pgf@circ@res@other}}
            \pgf@circ@draworfill
        \endpgfscope

        \ifpgfcirc@draw@leads
            %input leads
            \pgfpathmoveto{\pgfpoint{\pgf@circ@res@left}{0pt}}
            \pgfpathlineto{\pgfpoint{.7\pgf@circ@res@left}{0pt}}
        \fi

        \pgfpathmoveto{\pgfpoint{\pgf@circ@res@right}{0pt}}
        \pgfpathlineto{\pgfpoint{.7\pgf@circ@res@right}{0pt}}

        \pgfusepath{draw}
    }
}

\pgfdeclareshape{american buffer port}{
    \savedmacro{\ctikzclass}{\edef\ctikzclass{logic ports}}
    \saveddimen{\scaledRlen}{\pgfmathsetlength{\pgf@x}{\ctikzvalof{\ctikzclass/scale}\pgf@circ@Rlen}}
    \anchor{center}{\pgfpointorigin}
    \savedanchor\northwest{%
        \pgfmathsetlength{\pgf@circ@scaled@Rlen}{\ctikzvalof{\ctikzclass/scale}\pgf@circ@Rlen}
        \pgf@y=\ctikzvalof{bipoles/not port/height}\pgf@circ@scaled@Rlen
        \pgf@y=.5\pgf@y
        \pgf@x=-\ctikzvalof{bipoles/not port/width}\pgf@circ@scaled@Rlen
        \pgf@x=.5\pgf@x
    }
    \anchor{in}{
        \northwest
        \pgf@y=0pt
    }
    \anchor{in 1}{
        \northwest
        \pgf@y=0pt
    }
    \anchor{bin}{
        \northwest
        \pgf@y=0pt
    }
    \anchor{bin 1}{
        \northwest
        \pgf@y=0pt
        \pgf@x=0.7\pgf@x
    }
    \anchor{out}{
        \northwest
        \pgf@y=0pt
        \pgf@x=-\pgf@x
    }
    \anchor{bout}{
        \northwest
        \pgf@y=0pt
        \pgf@x=-0.7\pgf@x
    }
    \anchor{center}{
        \pgfpointorigin
    }
    \anchor{east}{
        \northwest
        \pgf@y=0pt
        \pgf@x=-\pgf@x
    }
    \anchor{west}{
        \northwest
        \pgf@y=0pt
    }
    \anchor{south}{
        \northwest
        \pgf@x=0pt
        \pgf@y=-\pgf@y
    }
    \anchor{north}{
        \northwest
        \pgf@x=0pt
    }
    \anchor{south west}{
        \northwest
        \pgf@y=-\pgf@y
    }
    \anchor{north east}{
        \northwest
        \pgf@x=-\pgf@x
    }
    \anchor{north west}{
        \northwest
    }
    \anchor{south east}{
        \northwest
        \pgf@x=-\pgf@x
        \pgf@y=-\pgf@y
    }
    \backgroundpath{
        \pgfsetcolor{\ctikzvalof{color}}

        \northwest
        \pgf@circ@res@up = \pgf@y
        \pgf@circ@res@down = -\pgf@y
        \pgf@circ@res@right = -\pgf@x
        \pgf@circ@res@left = \pgf@x

        \pgfscope
            \pgf@circ@setlinewidth{tripoles}{\pgflinewidth}
            \pgftransformxshift{.7\pgf@circ@res@left}
            \pgf@circ@res@step=\pgf@circ@res@right
            \advance\pgf@circ@res@step by -\pgf@circ@res@left
            \pgf@circ@res@step=.7\pgf@circ@res@step

            \pgfpathmoveto{\pgfpoint{\pgf@circ@res@step}{0pt}}
            \pgfpathlineto{\pgfpoint{0pt}{\pgf@circ@res@up}}
            \pgfpathlineto{\pgfpoint{0pt}{\pgf@circ@res@down}}
            \pgfpathlineto{\pgfpoint{\pgf@circ@res@step}{0pt}}
            \pgf@circ@draworfill
        \endpgfscope

        \ifpgfcirc@draw@leads
            %input leads
            \pgfpathmoveto{\pgfpoint{\pgf@circ@res@left}{0pt}}
            \pgfpathlineto{\pgfpoint{.7\pgf@circ@res@left}{0pt}}
        \fi

        \pgfpathmoveto{\pgfpoint{\pgf@circ@res@right}{0pt}}
        \pgfpathlineto{\pgfpoint{.7\pgf@circ@res@right}{0pt}}

        \pgfusepath{draw}
    }
}
\pgfdeclareshape{invschmitt}{
    \savedmacro{\ctikzclass}{\edef\ctikzclass{logic ports}}
    \saveddimen{\scaledRlen}{\pgfmathsetlength{\pgf@x}{\ctikzvalof{\ctikzclass/scale}\pgf@circ@Rlen}}
    \anchor{center}{\pgfpointorigin}
    \savedanchor\northwest{%
        \pgfmathsetlength{\pgf@circ@scaled@Rlen}{\ctikzvalof{\ctikzclass/scale}\pgf@circ@Rlen}
        \pgf@y=\ctikzvalof{bipoles/not port/height}\pgf@circ@scaled@Rlen
        \pgf@y=.5\pgf@y
        \pgf@x=-\ctikzvalof{bipoles/not port/width}\pgf@circ@scaled@Rlen
        \pgf@x=.5\pgf@x
    }
    \anchor{in}{
        \northwest
        \pgf@y=0pt
    }
    \anchor{in 1}{
        \northwest
        \pgf@y=0pt
    }
    \anchor{bin}{
        \northwest
        \pgf@y=0pt
        \pgf@x=0.7\pgf@x
    }
    \anchor{bin 1}{
        \northwest
        \pgf@y=0pt
        \pgf@x=0.7\pgf@x
    }
    \anchor{out}{
        \northwest
        \pgf@y=0pt
        \pgf@x=-\pgf@x
    }
    \anchor{bout}{
        \northwest
        \pgf@y=0pt
        \pgf@x=-0.7\pgf@x
    }
    \anchor{center}{
        \pgfpointorigin
    }
    \anchor{east}{
        \northwest
        \pgf@y=0pt
        \pgf@x=-\pgf@x
    }
    \anchor{west}{
        \northwest
        \pgf@y=0pt
    }
    \anchor{south}{
        \northwest
        \pgf@x=0pt
        \pgf@y=-\pgf@y
    }
    \anchor{north}{
        \northwest
        \pgf@x=0pt
    }
    \anchor{south west}{
        \northwest
        \pgf@y=-\pgf@y
    }
    \anchor{north east}{
        \northwest
        \pgf@x=-\pgf@x
    }
    \anchor{north west}{
        \northwest
    }
    \anchor{south east}{
        \northwest
        \pgf@x=-\pgf@x
        \pgf@y=-\pgf@y
    }
    \backgroundpath{
        \pgfsetcolor{\ctikzvalof{color}}

        \northwest
        \pgf@circ@res@up = \pgf@y
        \pgf@circ@res@down = -\pgf@y
        \pgf@circ@res@right = -\pgf@x
        \pgf@circ@res@left = \pgf@x

        \pgf@circ@res@other = \ctikzvalof{bipoles/not port/circle width}\pgf@circ@res@right

        \pgfscope
            \pgf@circ@setlinewidth{tripoles}{\pgflinewidth}
            \pgftransformxshift{.7\pgf@circ@res@left}
            \pgf@circ@res@step=\pgf@circ@res@right
            \advance\pgf@circ@res@step by -\pgf@circ@res@left
            \pgf@circ@res@step=.7\pgf@circ@res@step

            \pgfpathmoveto{\pgfpoint{\pgf@circ@res@step-\pgf@circ@res@other}{0pt}}
            \pgfpathlineto{\pgfpoint{0pt}{\pgf@circ@res@up}}
            \pgfpathlineto{\pgfpoint{0pt}{\pgf@circ@res@down}}
            \pgfpathclose
            \pgf@circ@draworfill
            \pgfpathellipse
            {\pgfpoint{\pgf@circ@res@step-.5\pgf@circ@res@other}{0pt}}
            {\pgfpoint{.5\pgf@circ@res@other}{0pt}}
            {\pgfpoint{0pt}{.5\pgf@circ@res@other}}
            \pgf@circ@draworfill
        \endpgfscope

        \ifpgfcirc@draw@leads
            %input leads
            \pgfpathmoveto{\pgfpoint{\pgf@circ@res@left}{0pt}}
            \pgfpathlineto{\pgfpoint{.7\pgf@circ@res@left}{0pt}}
        \fi

        \pgfpathmoveto{\pgfpoint{\pgf@circ@res@right}{0pt}}
        \pgfpathlineto{\pgfpoint{.7\pgf@circ@res@right}{0pt}}
        \pgfusepath{draw}
        %draw inner shape

        \pgf@circ@setlinewidth{tripoles}{\pgflinewidth}

        \pgfpathmoveto{\pgfpoint{.6\pgf@circ@res@left}{.3\pgf@circ@res@down}}
        \pgfpathlineto{\pgfpoint{.4\pgf@circ@res@left}{.3\pgf@circ@res@down}}
        \pgfpathlineto{\pgfpoint{.4\pgf@circ@res@left}{.3\pgf@circ@res@up}}
        \pgfpathlineto{\pgfpoint{.05\pgf@circ@res@left}{.3\pgf@circ@res@up}}
        \pgfpathmoveto{\pgfpoint{.4\pgf@circ@res@left}{.3\pgf@circ@res@down}}
        \pgfpathlineto{\pgfpoint{.25\pgf@circ@res@left}{.3\pgf@circ@res@down}}
        \pgfpathlineto{\pgfpoint{.25\pgf@circ@res@left}{.3\pgf@circ@res@up}}
        \pgfusepath{draw}

    }
}

\pgfdeclareshape{schmitt}{
    \savedmacro{\ctikzclass}{\edef\ctikzclass{logic ports}}
    \saveddimen{\scaledRlen}{\pgfmathsetlength{\pgf@x}{\ctikzvalof{\ctikzclass/scale}\pgf@circ@Rlen}}
    \anchor{center}{\pgfpointorigin}
    \savedanchor\northwest{%
        \pgfmathsetlength{\pgf@circ@scaled@Rlen}{\ctikzvalof{\ctikzclass/scale}\pgf@circ@Rlen}
        \pgf@y=\ctikzvalof{bipoles/not port/height}\pgf@circ@scaled@Rlen
        \pgf@y=.5\pgf@y
        \pgf@x=-\ctikzvalof{bipoles/not port/width}\pgf@circ@scaled@Rlen
        \pgf@x=.5\pgf@x
    }
    \anchor{in}{
        \northwest
        \pgf@y=0pt
    }
    \anchor{in 1}{
        \northwest
        \pgf@y=0pt
    }
    \anchor{bin}{
        \northwest
        \pgf@y=0pt
        \pgf@x=0.7\pgf@x
    }
    \anchor{bin 1}{
        \northwest
        \pgf@y=0pt
        \pgf@x=0.7\pgf@x
    }
    \anchor{out}{
        \northwest
        \pgf@y=0pt
        \pgf@x=-\pgf@x
    }
    \anchor{bout}{
        \northwest
        \pgf@y=0pt
        \pgf@x=-.7\pgf@x
    }
    \anchor{center}{
        \pgfpointorigin
    }
    \anchor{east}{
        \northwest
        \pgf@y=0pt
        \pgf@x=-\pgf@x
    }
    \anchor{west}{
        \northwest
        \pgf@y=0pt
    }
    \anchor{south}{
        \northwest
        \pgf@x=0pt
        \pgf@y=-\pgf@y
    }
    \anchor{north}{
        \northwest
        \pgf@x=0pt
    }
    \anchor{south west}{
        \northwest
        \pgf@y=-\pgf@y
    }
    \anchor{north east}{
        \northwest
        \pgf@x=-\pgf@x
    }
    \anchor{north west}{
        \northwest
    }
    \anchor{south east}{
        \northwest
        \pgf@x=-\pgf@x
        \pgf@y=-\pgf@y
    }
    \backgroundpath{
        \pgfsetcolor{\ctikzvalof{color}}


        \northwest
        \pgf@circ@res@up = \pgf@y
        \pgf@circ@res@down = -\pgf@y
        \pgf@circ@res@right = -\pgf@x
        \pgf@circ@res@left = \pgf@x

        \pgfscope
            \pgf@circ@setlinewidth{tripoles}{\pgflinewidth}
            \pgftransformxshift{.7\pgf@circ@res@left}
            \pgf@circ@res@step=\pgf@circ@res@right
            \advance\pgf@circ@res@step by -\pgf@circ@res@left
            \pgf@circ@res@step=.7\pgf@circ@res@step

            \pgfpathmoveto{\pgfpoint{\pgf@circ@res@step}{0pt}}
            \pgfpathlineto{\pgfpoint{0pt}{\pgf@circ@res@up}}
            \pgfpathlineto{\pgfpoint{0pt}{\pgf@circ@res@down}}
            \pgfpathclose
            \pgf@circ@draworfill
        \endpgfscope

        \ifpgfcirc@draw@leads
            %input leads
            \pgfpathmoveto{\pgfpoint{\pgf@circ@res@left}{0pt}}
            \pgfpathlineto{\pgfpoint{.7\pgf@circ@res@left}{0pt}}
        \fi

        \pgfpathmoveto{\pgfpoint{\pgf@circ@res@right}{0pt}}
        \pgfpathlineto{\pgfpoint{.7\pgf@circ@res@right-\pgf@circ@res@other}{0pt}}
        \pgfusepath{draw}
        %draw inner shape

        \pgf@circ@setlinewidth{tripoles}{\pgflinewidth}

        \pgfpathmoveto{\pgfpoint{.6\pgf@circ@res@left}{.3\pgf@circ@res@down}}
        \pgfpathlineto{\pgfpoint{.4\pgf@circ@res@left}{.3\pgf@circ@res@down}}
        \pgfpathlineto{\pgfpoint{.4\pgf@circ@res@left}{.3\pgf@circ@res@up}}
        \pgfpathlineto{\pgfpoint{.05\pgf@circ@res@left}{.3\pgf@circ@res@up}}
        \pgfpathmoveto{\pgfpoint{.4\pgf@circ@res@left}{.3\pgf@circ@res@down}}
        \pgfpathlineto{\pgfpoint{.25\pgf@circ@res@left}{.3\pgf@circ@res@down}}
        \pgfpathlineto{\pgfpoint{.25\pgf@circ@res@left}{.3\pgf@circ@res@up}}
        \pgfusepath{draw}

    }
}


%%% start european logic ports, from John Kormylo
%%%%%%%%%%%%%%%%%%%%%%%%%%%%%%%%%%%%%%%%%%%%%%%%%%%%%%%%%%%%%%%
%%

% #1 - name
% #2 - text inside
% #3 - number of inputs
% #4 = is it a not?
%
% option to add a circle for not-output, see
% https://github.com/circuitikz/circuitikz/issues/385
%
\newif\ifpgf@circ@european@port@circle\pgf@circ@european@port@circlefalse
\newif\ifpgf@circ@european@port@circle@ieee\pgf@circ@european@port@circle@ieeefalse
\ctikzset{tripoles/european not shape/.initial=ocirc}
\ctikzset{tripoles/european not symbol/.is choice}
\ctikzset{tripoles/european not symbol/triangle/.code={\pgf@circ@european@port@circlefalse}}
\ctikzset{tripoles/european not symbol/circle/.code={%
\pgf@circ@european@port@circletrue\pgf@circ@european@port@circle@ieeefalse\ctikzset{tripoles/european not shape=ocirc}}}
\ctikzset{tripoles/european not symbol/ieee circle/.code={%
\pgf@circ@european@port@circletrue\pgf@circ@european@port@circle@ieeetrue\ctikzset{tripoles/european not shape=notcirc}}}

\long\def\pgfcircdeclareeurologicport#1#2#3#4{
    \pgfdeclareshape{european #1 port}
    {
        \savedmacro{\ctikzclass}{\edef\ctikzclass{logic ports}}
        \saveddimen{\scaledRlen}{\pgfmathsetlength{\pgf@x}{\ctikzvalof{\ctikzclass/scale}\pgf@circ@Rlen}}
        \saveddimen{\boutshift}{%
            \ifpgf@circ@european@port@circle
                \pgfmathsetlength{\pgf@circ@scaled@Rlen}{\ctikzvalof{\ctikzclass/scale}\pgf@circ@Rlen}
                \edef\pgf@temp{ocirc}
                \edef\pgf@circ@temp{\ctikzvalof{tripoles/european not shape}}
                \ifx\pgf@temp\pgf@circ@temp % it's ocirc
                    \pgfmathsetlength{\pgf@x}{2*\ctikzvalof{nodes width}*\pgf@circ@Rlen}
                \else % it's ieee not circ
                    \pgf@circ@notradius % defined together with ieeestd ports
                    \pgf@x=2\pgf@circ@res@temp
                \fi
            \else
                \pgf@x=0pt
            \fi
        }
        \savedmacro\resize{% automatic
            \pgfmathsetlength{\pgf@circ@scaled@Rlen}{\ctikzvalof{\ctikzclass/scale}\pgf@circ@Rlen}
            \pgf@circ@res@up = \ctikzvalof{tripoles/european #1 port/height}\pgf@circ@scaled@Rlen
            \pgf@circ@res@up = .5\pgf@circ@res@up
            \pgf@circ@res@down = -\pgf@circ@res@up
            \pgf@circ@res@right = \ctikzvalof{tripoles/european #1 port/width}\pgf@circ@scaled@Rlen
            \pgf@circ@res@right = .5\pgf@circ@res@right
            \pgf@circ@res@left = -\pgf@circ@res@right
        }%
        \savedmacro\inputs{% get number of inputs
            \pgf@circ@res@count=\pgfkeysvalueof{/tikz/number inputs}\relax%
            \ifnum\pgf@circ@res@count=0
                \pgf@circ@res@count=\ctikzvalof{tripoles/european #1 port/inputs}\relax%
            \fi
        \ifnum\pgf@circ@res@count<2 \pgf@circ@res@count=2\fi
    \ifnum\pgf@circ@res@count>16 \pgf@circ@res@count=16\fi
        \def\inputs{\the\pgf@circ@res@count}%
    }%
    \savedanchor\step{% 1/2 gap at edges
        \pgfmathsetlength{\pgf@circ@scaled@Rlen}{\ctikzvalof{\ctikzclass/scale}\pgf@circ@Rlen}
        \pgf@circ@res@step = \ctikzvalof{tripoles/european #1 port/height}\pgf@circ@scaled@Rlen
        \divide\pgf@circ@res@step by #3
        \pgfpoint{\pgf@circ@res@left}{\dimexpr\pgf@circ@res@up+0.5\pgf@circ@res@step}%
    }%
    \savedanchor\northeast{\pgfpoint{\pgf@circ@res@right}{\pgf@circ@res@up}}%
    \savedanchor\southwest{\pgfpoint{\pgf@circ@res@left}{\pgf@circ@res@down}}%
    \savedanchor\left{\pgfpoint{\ctikzvalof{tripoles/european #1 port/reserved}\pgf@circ@res@left}{0pt}}%
    \savedanchor\right{\pgfpoint{\ctikzvalof{tripoles/european #1 port/reserved}\pgf@circ@res@right}{0pt}}%
    \savedanchor\origin{\pgfpoint{\ctikzvalof{tripoles/european #1 port/origin}\pgf@circ@res@right}{0pt}}%

    \anchor{center}{\origin}% for backwards compatibility
    % the text anchor overlaps the logic symbol
    \anchor{text}{\pgfpoint{-.5\wd\pgfnodeparttextbox}{\dimexpr.5\dp\pgfnodeparttextbox-.5\ht\pgfnodeparttextbox}}%
    % create input anchors
    \ifnum#3=1\relax
        \anchor{in}{\southwest\pgfpoint{\pgf@x}{0pt}}% or \step
        \anchor{in 1}{\southwest\pgfpoint{\pgf@x}{0pt}}% or \step
        \anchor{bin}{\left\pgfpoint{\pgf@x}{0pt}}% or \step
        \anchor{bin 1}{\left\pgfpoint{\pgf@x}{0pt}}% or \step
    \else
        \expandafter\pgfutil@g@addto@macro\csname pgf@sh@s@european #1 port\endcsname{%
            \pgfmathloop%
            \ifnum\pgfmathcounter>#3%
        \else%
            %\pgfutil@ifundefined{pgf@anchor@european #1 port@in \pgfmathcounter}{% redundant
            \expandafter\xdef\csname pgf@anchor@european #1 port@in \pgfmathcounter\endcsname{%
                \noexpand\pgf@circ@logicport@input{\pgfmathcounter}% defined above
            }%
            \expandafter\xdef\csname pgf@anchor@european #1 port@bin \pgfmathcounter\endcsname{%
                \noexpand\pgf@circ@logicport@baseinput{\pgfmathcounter}{4}{#1}% defined above
            }%
            %}{}%
            \repeatpgfmathloop%
        }
    \fi
    \anchor{out}{\northeast\pgf@y=0pt}
    \anchor{bout}{\right\advance\pgf@x by \boutshift\pgf@y=0pt}

    \anchor{left}{\left}% edges of component minus leads
    \anchor{right}{\right}

    \anchor{north east}{\northeast}% see \Compass macro
    \anchor{south west}{\southwest}
    \anchor{north}{\pgfextracty{\pgf@circ@res@up}{\northeast}%
    \pgfpoint{0cm}{\pgf@circ@res@up}}
    \anchor{north west}{\pgfextracty{\pgf@circ@res@up}{\northeast}%
        \pgfextractx{\pgf@circ@res@left}{\southwest}%
    \pgfpoint{\pgf@circ@res@left}{\pgf@circ@res@up}}
    \anchor{west}{\pgfextractx{\pgf@circ@res@left}{\southwest}%
    \pgfpoint{\pgf@circ@res@left}{0cm}}
    \anchor{south}{\pgfextracty{\pgf@circ@res@down}{\southwest}%
    \pgfpoint{0cm}{\pgf@circ@res@down}}
    \anchor{south east}{\pgfextracty{\pgf@circ@res@down}{\southwest}%
        \pgfextractx{\pgf@circ@res@right}{\northeast}%
    \pgfpoint{\pgf@circ@res@right}{\pgf@circ@res@down}}
    \anchor{east}{\pgfextractx{\pgf@circ@res@right}{\northeast}%
    \pgfpoint{\pgf@circ@res@right}{0cm}}

    \backgroundpath{
        \pgfsetcolor{\ctikzvalof{color}}
        \pgfstartlinewidth=\pgflinewidth
        \pgfscope
            \pgf@circ@setlinewidth{tripoles}{\pgflinewidth}
            \pgfpathrectanglecorners
            {\pgfpoint{\ctikzvalof{tripoles/european #1 port/reserved}\pgf@circ@res@left}{\pgf@circ@res@up}}
            {\pgfpoint{\ctikzvalof{tripoles/european #1 port/reserved}\pgf@circ@res@right}{\pgf@circ@res@down}}
            \pgf@circ@draworfill
        \endpgfscope
        \ifpgfcirc@draw@leads
            %input leads
            \ifnum#3=1\relax
                \pgfpathmoveto{\pgfpoint{\pgf@circ@res@left}{0pt}}%
                \pgfpathlineto{\pgfpoint{\ctikzvalof{tripoles/european #1 port/reserved}\pgf@circ@res@left}{0pt}}%
            \else
                \pgfextracty{\pgf@circ@res@temp}{\step}%
                \pgf@circ@res@step = \dimexpr 2\pgf@circ@res@temp -2\pgf@circ@res@up\relax
                %\pgf@circ@res@count = #3\relax% redundant
                \loop\ifnum\pgf@circ@res@count>0
                    \advance\pgf@circ@res@temp by -\pgf@circ@res@step
                    \pgfpathmoveto{\pgfpoint{\pgf@circ@res@left}{\pgf@circ@res@temp}}%
                    \pgfpathlineto{\pgfpoint
                        {\ctikzvalof{tripoles/european #1 port/reserved}\pgf@circ@res@left}
                    {\pgf@circ@res@temp}}
                    \advance\pgf@circ@res@count by -1
                \repeat
            \fi
        \fi
        %
        \pgfpathmoveto{\pgfpoint{\pgf@circ@res@right}{0pt}}
        \pgfpathlineto{%
        \pgfpoint{\ctikzvalof{tripoles/european #1 port/reserved}\pgf@circ@res@right}{0pt}}
        %
        \pgfusepath{draw}
        %
        \edef\pgf@temp{not}
        \edef\pgf@circ@temp{#4}
        \ifx\pgf@temp\pgf@circ@temp % is a not
            \ifpgf@circ@european@port@circle
                \pgfscope
                    \pgf@circ@setlinewidth{tripoles}{\pgflinewidth}
                    \pgftransformxshift{\ctikzvalof{tripoles/european #1 port/reserved}\pgf@circ@res@right}
                    % reset  linewidth for IEEE ports, otherwise they will multiply...
                    \ifpgf@circ@european@port@circle@ieee\pgfsetlinewidth{\pgfstartlinewidth}\fi
                    \pgfnode{\ctikzvalof{tripoles/european not shape}}{west}{}{NOT}{\pgfusepath{stroke}}
                \endpgfscope
            \else
                \pgfpathmoveto{\pgfpoint{\ctikzvalof{tripoles/european #1 port/not width}\pgf@circ@res@right}{0pt}}
                \pgfpathlineto{\pgfpoint{\ctikzvalof{tripoles/european #1 port/reserved}\pgf@circ@res@right}%
                {\ctikzvalof{tripoles/european #1 port/not height}\pgf@circ@res@up}}
                \pgfusepath{draw}
            \fi
        \fi
        %
        \pgfpathmoveto{\pgfpointorigin}
        \pgftext{#2}
        }
    }
}
\pgfcircdeclareeurologicport{and}{\&}{\pgf@circ@res@count}{}
\pgfcircdeclareeurologicport{or}{$\ge 1$}{\pgf@circ@res@count}{}
\pgfcircdeclareeurologicport{xor}{$=1$}{\pgf@circ@res@count}{}
\pgfcircdeclareeurologicport{not}{$1$}{1}{not}
\pgfcircdeclareeurologicport{buffer}{$1$}{1}{}
\pgfcircdeclareeurologicport{nand}{\&}{\pgf@circ@res@count}{not}
\pgfcircdeclareeurologicport{nor}{$\ge 1$}{\pgf@circ@res@count}{not}
\pgfcircdeclareeurologicport{xnor}{$=1$}{\pgf@circ@res@count}{not}

%% end european logic ports

%%%%%%%%%%%%%%%%%%%%%%%%
%% Transistors
%%%%%%%%%%%%%%%%%%%%%%%%


\long\def\pgfcircdeclaretransistor#1#2#3{
    \pgfdeclareshape{#1}
    {
        \savedmacro{\ctikzclass}{\edef\ctikzclass{transistors}}
        \saveddimen{\scaledRlen}{\pgfmathsetlength{\pgf@x}{\ctikzvalof{\ctikzclass/scale}\pgf@circ@Rlen}}
        \anchor{center}{
            \pgfpointorigin
        }
        \savedanchor\northeast{% upper right
            \pgfmathsetlength{\pgf@circ@scaled@Rlen}{\ctikzvalof{\ctikzclass/scale}\pgf@circ@Rlen}
            \pgf@y=\ctikzvalof{tripoles/#1/height}\pgf@circ@scaled@Rlen
            \pgf@y=.5\pgf@y
            \pgf@x=0pt
        }
        \savedanchor\left{%center left
            \pgf@y=0pt
            \pgfmathsetlength{\pgf@circ@scaled@Rlen}{\ctikzvalof{\ctikzclass/scale}\pgf@circ@Rlen}
            \pgf@x=-\ctikzvalof{tripoles/#1/width}\pgf@circ@scaled@Rlen
        }
        \savedanchor\right{%center right -- added by Burak Kelleci % this is really 0,0
            \pgf@y=0pt
            \pgf@x=0pt
        }
        \anchor{text}{
            \northeast
            \pgfmathsetlength{\pgf@circ@scaled@Rlen}{\ctikzvalof{\ctikzclass/scale}\pgf@circ@Rlen}
            \ifpgf@circuit@transisors@fixlabels
                \ifpgf@circuit@fet@bodydiode
                    % try to put the text to the right of the flyback diode
                    \pgfmathsetlength{\pgf@circ@res@other}{(
                        \ctikzvalof{tripoles/#1/bodydiode distance}*
                        \ctikzvalof{tripoles/#1/width} +
                        \ctikzvalof{tripoles/#1/bodydiode scale}*
                        \ctikzvalof{bipoles/diode/height}/2
                        )*\pgf@circ@scaled@Rlen}
                    \advance\pgf@x by \pgf@circ@res@other
                \fi
                % add a bit of space to avoid central (substrate) terminal if drawn
                \advance\pgf@x by 0.05\pgf@circ@scaled@Rlen\relax
                \pgf@y=\dimexpr.5\dp\pgfnodeparttextbox-.5\ht\pgfnodeparttextbox\relax
            \else
                \pgf@y=.7\pgf@y
                \pgf@x= \pgf@circ@scaled@Rlen
                \pgf@x=0.1\pgf@x
            \fi
        }
        \anchor{pathstart}{ % south
            \northeast
            \pgf@y=-\pgf@y
        }
        \anchor{pathend}{
            \northeast
        }
        \anchor{north}{
            \northeast
            \pgf@circ@res@step=\pgf@y
            \left
            \pgf@y=\pgf@circ@res@step
            \pgf@x=.5\pgf@x
        }
        \anchor{west}{
            \left
        }
        \anchor{east}{
            \northeast
            \pgf@y=0pt
        }
        \anchor{south}{
            \northeast
            \pgf@circ@res@step=\pgf@y
            \left
            \pgf@y=-\pgf@circ@res@step
            \pgf@x=.5\pgf@x
        }
        \anchor{south west}{
            \northeast
            \pgf@circ@res@step=\pgf@y
            \left
            \pgf@y=-\pgf@circ@res@step
        }
        \anchor{north east}{
            \northeast
        }
        \anchor{north west}{
            \northeast
            \pgf@circ@res@step=\pgf@y
            \left
            \pgf@y=\pgf@circ@res@step
        }
        \anchor{south east}{
            \northeast
            \pgf@y=-\pgf@y
        }
        \anchor{B}{
            \northeast
            \pgf@circ@res@step=\ctikzvalof{tripoles/#1/conn height}\pgf@y
            \left
            \pgf@y=-\ctikzvalof{tripoles/#1/curr direction}\pgf@circ@res@step
        }
        \anchor{base}{
            \northeast
            \pgf@circ@res@step=\ctikzvalof{tripoles/#1/conn height}\pgf@y
            \left
            \pgf@y=-\ctikzvalof{tripoles/#1/curr direction}\pgf@circ@res@step
        }
        \anchor{bulk}{ %added by Burak Kelleci
            \northeast
            \pgf@circ@res@step=\ctikzvalof{tripoles/#1/conn height}\pgf@y
            \right
            \pgf@y=-\ctikzvalof{tripoles/#1/curr direction}\pgf@circ@res@step
        }
        \anchor{nobulk}{ %added by Burak Kelleci
            \left
            \pgf@x=\ctikzvalof{tripoles/#1/base width}\pgf@x
        }
        \anchor{G}{
            \northeast
            \pgf@circ@res@step=\ctikzvalof{tripoles/#1/conn height}\pgf@y
            \left
            \pgf@y=-\ctikzvalof{tripoles/#1/curr direction}\pgf@circ@res@step
        }
        \anchor{gate}{
            \northeast
            \pgf@circ@res@step=\ctikzvalof{tripoles/#1/conn height}\pgf@y
            \left
            \pgf@y=-\ctikzvalof{tripoles/#1/curr direction}\pgf@circ@res@step
        }
        \anchor{nobase}{
            \left
            \pgf@x=\ctikzvalof{tripoles/#1/base width}\pgf@x
        }
        \anchor{nogate}{
            \left
            \pgf@x=\ctikzvalof{tripoles/#1/gate width}\pgf@x
        }
        \anchor{E}{
            \northeast
            \pgf@y=-\ctikzvalof{tripoles/#1/curr direction}\pgf@y
        }
        \anchor{emitter}{
            \northeast
            \pgf@y=-\ctikzvalof{tripoles/#1/curr direction}\pgf@y
        }
        \anchor{C}{
            \northeast
            \pgf@y=\ctikzvalof{tripoles/#1/curr direction}\pgf@y
        }
        \anchor{collector}{
            \northeast
            \pgf@y=\ctikzvalof{tripoles/#1/curr direction}\pgf@y
        }
        \anchor{S}{
            \northeast
            \pgf@y=-\ctikzvalof{tripoles/#1/curr direction}\pgf@y
        }
        \anchor{source}{
            \northeast
            \pgf@y=-\ctikzvalof{tripoles/#1/curr direction}\pgf@y
        }
        \anchor{D}{
            \northeast
            \pgf@y=\ctikzvalof{tripoles/#1/curr direction}\pgf@y
        }
        \anchor{drain}{
            \northeast
            \pgf@y=\ctikzvalof{tripoles/#1/curr direction}\pgf@y
        }
        \anchor{body C in}{
            \northeast
            \pgf@y=\ctikzvalof{tripoles/#1/curr direction}\pgf@y
            \pgf@y=\ctikzvalof{tripoles/#1/bodydiode conn}\pgf@y
        }
        \anchor{body E in}{
            \northeast
            \pgf@y=-\ctikzvalof{tripoles/#1/curr direction}\pgf@y
            \pgf@y=\ctikzvalof{tripoles/#1/bodydiode conn}\pgf@y
        }
        \anchor{body C out}{
            \northeast
            \pgf@ya=\ctikzvalof{tripoles/#1/curr direction}\pgf@y
            \pgf@ya=\ctikzvalof{tripoles/#1/bodydiode conn}\pgf@ya
            \pgf@y=\pgf@ya
            \left
            \pgf@x=-\ctikzvalof{tripoles/#1/bodydiode distance}\pgf@x
            \pgf@y=\pgf@ya
        }
        \anchor{body E out}{
            \northeast
            \pgf@ya=-\ctikzvalof{tripoles/#1/curr direction}\pgf@y
            \pgf@ya=\ctikzvalof{tripoles/#1/bodydiode conn}\pgf@ya
            \pgf@y=\pgf@ya
            \left
            \pgf@x=-\ctikzvalof{tripoles/#1/bodydiode distance}\pgf@x
            \pgf@y=\pgf@ya
        }
        #2
        \backgroundpath{
            \pgftransformationadjustments
            \pgfsetcolor{\ctikzvalof{color}}
            %
            \ifnum \ctikzvalof{tripoles/#1/curr direction} > 0
            \pgf@circuit@trans@ntypetrue
            \else
            \pgf@circuit@trans@ntypefalse
        \fi
        \northeast
        \pgf@circ@res@up = \pgf@y
        \pgf@circ@res@down = -\pgf@y
        \pgf@circ@res@right = \pgf@x
        \left
        \pgf@circ@res@left = \pgf@x
        \pgf@circ@scaled@Rlen=\scaledRlen
        %
        #3
        % BODY DIODE
        \ifpgf@circuit@fet@bodydiode
            \drawbodydiode{#1}
        \fi
        %
    }
}
}

\long\def\drawbodydiode#1{
    \pgfscope
        \pgftransformshift{\pgfpoint{-\ctikzvalof{tripoles/#1/bodydiode distance}\pgf@circ@res@left}{\pgf@circ@res@up+\pgf@circ@res@down}}
        \pgftransformrotate{90}
        % diode scale and bodydiode scale interacts. We want the size of the diode
        % proportional to the transistor, so we will:
        % 1) undo diode scale 2) apply transistor scale (using the current class) 3) apply bodydiode scale
        \pgfmathsetmacro{\@@BDscale}{\ctikzvalof{tripoles/#1/bodydiode scale}* \ctikzvalof{\ctikzclass/scale}/\ctikzvalof{diodes/scale}}
        \pgftransformscale{\@@BDscale}
        \ifpgf@circuit@fulldiode
            \pgfnode{fulldiodeshape}{center}{}{pgf@bodydiode}{\pgfusepath{fill}}
        \else
            \pgfnode{emptydiodeshape}{center}{}{pgf@bodydiode}{\pgfusepath{fill}}
        \fi
    \endpgfscope
    % Draw stroke line
    \ifpgf@circuit@strokediode
        \pgfpathmoveto{\pgfpointanchor{pgf@bodydiode}{west}}
        \pgfpathlineto{\pgfpointanchor{pgf@bodydiode}{east}}
        \pgfusepath{stroke}
    \fi
    %Draw upper connection to body diode
    \pgfpathmoveto{\pgfpoint{\pgf@circ@res@right}
    {\ctikzvalof{tripoles/#1/bodydiode conn}\pgf@circ@res@up}}
    \pgfpathlineto{\pgfpoint{-\ctikzvalof{tripoles/#1/bodydiode distance}\pgf@circ@res@left}
    {\ctikzvalof{tripoles/#1/bodydiode conn}\pgf@circ@res@up}}
    \pgfpathlineto{\pgfpointanchor{pgf@bodydiode}{east}}
    \pgfusepath{draw}
    \pgfscope
        \pgftransformshift{\pgfpoint{\pgf@circ@res@right}
        {\ctikzvalof{tripoles/#1/bodydiode conn}\pgf@circ@res@up}}
        \pgftransformscale{0.5}
        \pgfnode{circ}{center}{}{}{\pgfusepath{fill}}
    \endpgfscope{}
    %Draw lower connection to body diode
    \pgfpathmoveto{\pgfpoint{\pgf@circ@res@right}
    {\ctikzvalof{tripoles/#1/bodydiode conn}\pgf@circ@res@down}}
    \pgfpathlineto{\pgfpoint{-\ctikzvalof{tripoles/#1/bodydiode distance}\pgf@circ@res@left}{\ctikzvalof{tripoles/#1/bodydiode conn}\pgf@circ@res@down}}
    \pgfpathlineto{\pgfpointanchor{pgf@bodydiode}{west}}
    \pgfusepath{draw}
    \pgfscope
        \pgftransformshift{\pgfpoint{\pgf@circ@res@right}		        	       {\ctikzvalof{tripoles/#1/bodydiode conn}\pgf@circ@res@down}}
        \pgftransformscale{0.5}
        \pgfnode{circ}{center}{}{}{\pgfusepath{fill}}
    \endpgfscope
}

\long\def\declarebpt#1{
    \pgfcircdeclaretransistor{#1}{
        \anchor{inner up}{
            \northeast
            \pgf@y=\ctikzvalof{tripoles/#1/base height}\pgf@y
        }
        \anchor{inner down}{
            \northeast
            \pgf@y=-\ctikzvalof{tripoles/#1/base height}\pgf@y
        }
        }{
        \pgfpathmoveto{\pgfpoint{\pgf@circ@res@right}{\pgf@circ@res@up+\pgfverticaltransformationadjustment*.5*\pgflinewidth}}
        \pgfpathlineto{\pgfpoint{\pgf@circ@res@right}
        {\ctikzvalof{tripoles/#1/base height}\pgf@circ@res@up}}
        \pgfpathlineto{\pgfpoint
            {\ctikzvalof{tripoles/#1/base width}\pgf@circ@res@left}
        {\ctikzvalof{tripoles/#1/base height 2}\pgf@circ@res@up}}
        \pgfusepath{draw}

        % drawing base
        \pgfscope
            \pgfpathmoveto{\pgfpoint
                {\ctikzvalof{tripoles/#1/base width}\pgf@circ@res@left}
            {\ctikzvalof{tripoles/#1/base height}\pgf@circ@res@up}}
            \pgfpathlineto{\pgfpoint
                {\ctikzvalof{tripoles/#1/base width}\pgf@circ@res@left}
            {\ctikzvalof{tripoles/#1/base height}\pgf@circ@res@down}}
            \pgf@circ@setlinewidth{tripoles}{\pgflinewidth}
            \ifpgf@circuit@bpt@schottky
                % upper
                \pgfpathmoveto{\pgfpoint
                    {\ctikzvalof{tripoles/#1/base width}\pgf@circ@res@left}
                    {\ctikzvalof{tripoles/#1/base height}\pgf@circ@res@up}}
                \pgfpathlineto{\pgfpoint
                    {\ctikzvalof{tripoles/#1/base width}\pgf@circ@res@left}
                    {\ctikzvalof{tripoles/#1/base height}\pgf@circ@res@up+
                    \ctikzvalof{tripoles/schottky base size}\pgf@circ@scaled@Rlen}}
                \pgfpathlineto{\pgfpoint
                    {\ctikzvalof{tripoles/#1/base width}\pgf@circ@res@left+
                        \ctikzvalof{tripoles/schottky base size}\pgf@circ@scaled@Rlen}
                    {\ctikzvalof{tripoles/#1/base height}\pgf@circ@res@up+
                        \ctikzvalof{tripoles/schottky base size}\pgf@circ@scaled@Rlen}}
                \pgfpathlineto{\pgfpoint
                    {\ctikzvalof{tripoles/#1/base width}\pgf@circ@res@left+
                        \ctikzvalof{tripoles/schottky base size}\pgf@circ@scaled@Rlen}
                    {\ctikzvalof{tripoles/#1/base height}\pgf@circ@res@up}}
                % % lower
                \pgfpathmoveto{\pgfpoint
                    {\ctikzvalof{tripoles/#1/base width}\pgf@circ@res@left}
                    {\ctikzvalof{tripoles/#1/base height}\pgf@circ@res@down}}
                \pgfpathlineto{\pgfpoint
                    {\ctikzvalof{tripoles/#1/base width}\pgf@circ@res@left}
                    {\ctikzvalof{tripoles/#1/base height}\pgf@circ@res@down-
                    \ctikzvalof{tripoles/schottky base size}\pgf@circ@scaled@Rlen}}
                \pgfpathlineto{\pgfpoint
                    {\ctikzvalof{tripoles/#1/base width}\pgf@circ@res@left-
                        \ctikzvalof{tripoles/schottky base size}\pgf@circ@scaled@Rlen}
                    {\ctikzvalof{tripoles/#1/base height}\pgf@circ@res@down-
                    \ctikzvalof{tripoles/schottky base size}\pgf@circ@scaled@Rlen}}
                \pgfpathlineto{\pgfpoint
                    {\ctikzvalof{tripoles/#1/base width}\pgf@circ@res@left-
                        \ctikzvalof{tripoles/schottky base size}\pgf@circ@scaled@Rlen}
                    {\ctikzvalof{tripoles/#1/base height}\pgf@circ@res@down}}
                \fi
            \pgfusepath{draw}
        \endpgfscope

        \pgfpathmoveto{\pgfpoint
            {\ctikzvalof{tripoles/#1/base width}\pgf@circ@res@left}
        {\ctikzvalof{tripoles/#1/base height 2}\pgf@circ@res@down}}
        \pgfpathlineto{\pgfpoint{\pgf@circ@res@right}
        {\ctikzvalof{tripoles/#1/base height}\pgf@circ@res@down}}
        \pgfpathlineto{\pgfpoint{\pgf@circ@res@right}{\pgf@circ@res@down-\pgfverticaltransformationadjustment*.5*\pgflinewidth}}
        \pgfusepath{draw}
        %draw arrow depending on type of transistor
        \pgfscope
            \pgfslopedattimetrue
            \pgfallowupsidedownattimetrue
            \pgfresetnontranslationattimefalse
            \edef\@@anchor{center}
            \ifpgf@circuit@trans@ntype
                \ifpgf@circuit@trans@arrowatend
                    \edef\@@anchor{btip}
                    \pgftransformlineattime{1.0}{%
                        \pgfpoint%
                        {\ctikzvalof{tripoles/#1/base width}\pgf@circ@res@left}%
                        {\ctikzvalof{tripoles/#1/base height 2}\pgf@circ@res@down}%
                        }{%
                        \pgfpoint{\pgf@circ@res@right}%
                        {\ctikzvalof{tripoles/#1/base height}\pgf@circ@res@down}%
                    }
                \else
                    \pgftransformlineattime{\ctikzvalof{tripoles/#1/arrow pos}}{%
                        \pgfpoint%
                        {\ctikzvalof{tripoles/#1/base width}\pgf@circ@res@left}%
                        {\ctikzvalof{tripoles/#1/base height 2}\pgf@circ@res@down}%
                        }{%
                        \pgfpoint{\pgf@circ@res@right}%
                        {\ctikzvalof{tripoles/#1/base height}\pgf@circ@res@down}%
                    }
                \fi
            \else % p-type
                \ifpgf@circuit@trans@arrowatend
                    \edef\@@anchor{tip}
                    \pgftransformlineattime{1.0}{%
                        \pgfpoint{\pgf@circ@res@right}%
                        {\ctikzvalof{tripoles/#1/base height}\pgf@circ@res@up}%
                        }{%
                        \pgfpoint{\ctikzvalof{tripoles/#1/base width}\pgf@circ@res@left}%
                        {\ctikzvalof{tripoles/#1/base height 2}\pgf@circ@res@up}%
                    }
                \else
                    \pgftransformlineattime{\ctikzvalof{tripoles/#1/arrow pos}}{%
                        \pgfpoint{\pgf@circ@res@right}%
                        {\ctikzvalof{tripoles/#1/base height}\pgf@circ@res@up}%
                        }{%
                        \pgfpoint{\ctikzvalof{tripoles/#1/base width}\pgf@circ@res@left}%
                        {\ctikzvalof{tripoles/#1/base height 2}\pgf@circ@res@up}%
                    }
                \fi
            \fi
            \pgfnode{trarrow}{\@@anchor}{}{}{\pgfusepath{stroke}}
        \endpgfscope

        \ifpgf@circuit@bpt@drawphoto
            \pgfscope
                \pgfsetarrowsstart{latexslim}
                \pgfpathmoveto{\pgfpointadd{\pgfpoint
                        {\ctikzvalof{tripoles/#1/base width}\pgf@circ@res@left}
                    {\pgf@circ@res@up+\pgf@circ@res@down}}
                {\pgfpoint{0.05\pgf@circ@res@left}{0.1\pgf@circ@res@up}}}
                \pgfpathlineto{\pgfpointadd{\pgfpoint
                        {\ctikzvalof{tripoles/#1/base width}\pgf@circ@res@left}
                    {\pgf@circ@res@up+\pgf@circ@res@down}}
                {\pgfpoint{0.5\pgf@circ@res@left}{0.3\pgf@circ@res@up}}}
                \pgfusepath{draw}
                \pgfpathmoveto{\pgfpointadd{\pgfpoint
                        {\ctikzvalof{tripoles/#1/base width}\pgf@circ@res@left}
                    {\pgf@circ@res@up+\pgf@circ@res@down}}
                {\pgfpoint{0.05\pgf@circ@res@left}{-0.1\pgf@circ@res@up}}}
                \pgfpathlineto{\pgfpointadd{\pgfpoint
                        {\ctikzvalof{tripoles/#1/base width}\pgf@circ@res@left}
                    {\pgf@circ@res@up+\pgf@circ@res@down}}
                {\pgfpoint{0.5\pgf@circ@res@left}{0.1\pgf@circ@res@up}}}
                \pgfusepath{draw}
            \endpgfscope
            \else
            \ifpgf@circuit@bpt@drawbase
                \pgfpathmoveto{\pgfpoint
                    {\ctikzvalof{tripoles/#1/base width}\pgf@circ@res@left}
                {\pgf@circ@res@up+\pgf@circ@res@down}}
                \pgfpathlineto{\pgfpoint{\pgf@circ@res@left-\pgfhorizontaltransformationadjustment*.5*\pgflinewidth}{\pgf@circ@res@up+\pgf@circ@res@down}}
                \pgfusepath{draw}
            \fi
        \fi
    }
}

\declarebpt{npn}
\declarebpt{pnp}
%
% multi-emitter and multi-collector BJTs by Romano Giannetti
%
\def\pgf@circ@bjt@C@anchor#1{% #1: collector number
    \pgfextractx{\pgf@circ@res@temp}{\basedimension}
    \pgfextracty{\pgf@circ@res@other}{\basedimension}
    \ifnum\cdir>0 % NPN, above
        \pgfpoint{\pgf@circ@res@temp}{%
        \pgf@circ@res@other+\pgfverticaltransformationadjustment*.5*\pgflinewidth+(#1-1)*\multistep}
    \else % PNP, below
        \pgfpoint{\pgf@circ@res@temp}{%
        -\pgf@circ@res@other-\pgfverticaltransformationadjustment*.5*\pgflinewidth-(#1-1)*\multistep}
    \fi
}
\def\pgf@circ@bjt@E@anchor#1{% #1: collector number
    \pgfextractx{\pgf@circ@res@temp}{\basedimension}
    \pgfextracty{\pgf@circ@res@other}{\basedimension}
    \ifnum\cdir<0 % PNP, above
        \pgfpoint{\pgf@circ@res@temp}{%
        \pgf@circ@res@other+\pgfverticaltransformationadjustment*.5*\pgflinewidth+(#1-1)*\multistep}
    \else % PNP, below
        \pgfpoint{\pgf@circ@res@temp}{%
        -\pgf@circ@res@other-\pgfverticaltransformationadjustment*.5*\pgflinewidth-(#1-1)*\multistep}
    \fi
}

\long\def\declarebjt#1{
    \pgfdeclareshape{bjt#1}{
        \savedmacro{\ctikzclass}{\edef\ctikzclass{transistors}}
        \saveddimen{\scaledRlen}{\pgfmathsetlength{\pgf@x}{\ctikzvalof{\ctikzclass/scale}\pgf@circ@Rlen}}
        % \cdir is 1 for npn, -1 for pnp
        \savedmacro{\cdir}{\edef\cdir{\ctikzvalof{tripoles/bjt/#1/curr direction}}}
        \savedmacro{\numE}{\edef\numE{\ctikzvalof{tripoles/bjt/emitters}}}
        \savedmacro{\numC}{\edef\numC{\ctikzvalof{tripoles/bjt/collectors}}}
        % step up or down for the additional C/Es
        \saveddimen{\multistep}{\pgfmathsetlength{\pgf@x}{%
            \ctikzvalof{tripoles/bjt/height}*\ctikzvalof{tripoles/bjt/multi height}*
            \ctikzvalof{\ctikzclass/scale}*\pgf@circ@Rlen}}
        \saveddimen{\external}{\pgfmathsetlength{\pgf@x}{%
            \ctikzvalof{tripoles/bjt/pins width}*\ctikzvalof{\ctikzclass/scale}*\pgf@circ@Rlen}}
        \savedanchor\basedimension{% these are the dimensions if nC=1 y nE=1
            \pgfmathsetlength{\pgf@circ@scaled@Rlen}{\ctikzvalof{\ctikzclass/scale}\pgf@circ@Rlen}
            \pgf@y=\ctikzvalof{tripoles/bjt/height}\pgf@circ@scaled@Rlen
            \pgf@y=.5\pgf@y
            \pgf@x=\ctikzvalof{tripoles/bjt/width}\pgf@circ@scaled@Rlen
            \pgf@x=.5\pgf@x
        }
        \savedanchor\northeast{% upper right
            \pgfmathsetlength{\pgf@circ@scaled@Rlen}{\ctikzvalof{\ctikzclass/scale}\pgf@circ@Rlen}
            \pgfmathsetlength{\pgf@circ@res@step}{%
                \ctikzvalof{tripoles/bjt/height}*\ctikzvalof{tripoles/bjt/multi height}*
                \ctikzvalof{\ctikzclass/scale}*\pgf@circ@Rlen}
            \ifnum\cdir> 0
                \edef\numup{\numC}\edef\numdown{\numE}
            \else
                \edef\numup{\numE}\edef\numdown{\numC}
            \fi
            \pgfmathsetlength{\pgf@y}{0.5*\ctikzvalof{tripoles/bjt/height}\pgf@circ@scaled@Rlen
                + (\numup-1)*\pgf@circ@res@step+\pgfverticaltransformationadjustment*.5*\pgflinewidth
                + \ctikzvalof{tripoles/bjt/pins width}*\ctikzvalof{\ctikzclass/scale}*\pgf@circ@Rlen}
            \pgf@x=\ctikzvalof{tripoles/bjt/width}\pgf@circ@scaled@Rlen
            \pgf@x=.5\pgf@x
        }
        \savedanchor\southeast{% lower right
            \pgfmathsetlength{\pgf@circ@scaled@Rlen}{\ctikzvalof{\ctikzclass/scale}\pgf@circ@Rlen}
            \pgfmathsetlength{\pgf@circ@res@step}{%
                \ctikzvalof{tripoles/bjt/height}*\ctikzvalof{tripoles/bjt/multi height}*
                \ctikzvalof{\ctikzclass/scale}*\pgf@circ@Rlen}
            \ifnum\cdir> 0
                \edef\numup{\numC}\edef\numdown{\numE}
            \else
                \edef\numup{\numE}\edef\numdown{\numC}
            \fi
            \pgfmathsetlength{\pgf@y}{-0.5*\ctikzvalof{tripoles/bjt/height}\pgf@circ@scaled@Rlen
                - (\numdown-1)*\pgf@circ@res@step-\pgfverticaltransformationadjustment*.5*\pgflinewidth
                - \ctikzvalof{tripoles/bjt/pins width}*\ctikzvalof{\ctikzclass/scale}*\pgf@circ@Rlen}
            \pgf@x=\ctikzvalof{tripoles/bjt/width}\pgf@circ@scaled@Rlen
            \pgf@x=.5\pgf@x
        }
        \savedanchor\southwest{% lower left
            \pgfmathsetlength{\pgf@circ@scaled@Rlen}{\ctikzvalof{\ctikzclass/scale}\pgf@circ@Rlen}
            \pgfmathsetlength{\pgf@circ@res@step}{%
                \ctikzvalof{tripoles/bjt/height}*\ctikzvalof{tripoles/bjt/multi height}*
                \ctikzvalof{\ctikzclass/scale}*\pgf@circ@Rlen}
            \ifnum\cdir> 0
                \edef\numup{\numC}\edef\numdown{\numE}
            \else
                \edef\numup{\numE}\edef\numdown{\numC}
            \fi
            \pgfmathsetlength{\pgf@y}{-0.5*\ctikzvalof{tripoles/bjt/height}\pgf@circ@scaled@Rlen
                - (\numdown-1)*\pgf@circ@res@step-\pgfverticaltransformationadjustment*.5*\pgflinewidth
                - \ctikzvalof{tripoles/bjt/pins width}*\ctikzvalof{\ctikzclass/scale}*\pgf@circ@Rlen}
            \pgfmathsetlength{\pgf@x}{-0.5*\ctikzvalof{tripoles/bjt/width}*\pgf@circ@scaled@Rlen
                - \ctikzvalof{tripoles/bjt/pins width}*\ctikzvalof{\ctikzclass/scale}*\pgf@circ@Rlen}
        }
        \anchor{center}{\pgfpointorigin}
        \anchor{north}{\northeast\pgf@x=0cm\relax}
        \anchor{east}{\northeast\pgf@y=0cm\relax}
        \anchor{south}{\southwest\pgf@x=0cm\relax}
        \anchor{west}{\southwest\pgf@y=0cm}
        \anchor{north east}{\northeast}
        \anchor{north west}{\northeast\pgf@ya=\pgf@y\southwest\pgf@y=\pgf@ya}
        \anchor{south west}{\southwest}
        \anchor{south east}{\southeast}

        \anchor{text}{\northeast\pgf@y=\dimexpr.5\dp\pgfnodeparttextbox-.5\ht\pgfnodeparttextbox\relax}
        \anchor{B}{\northeast\pgf@y=0cm\pgf@x=-\pgf@x\advance\pgf@x by -\external}
        \anchor{base}{\basedimension\pgf@y=0cm\pgf@x=-\pgf@x\advance\pgf@x by -\external}
        \anchor{nobase}{\basedimension\pgf@y=0cm\pgf@x=-\pgf@x}
        % center of the base "bar"
        \anchor{cbase}{\basedimension\pgf@x=-\pgf@x
            \pgfmathsetlength{\pgf@y}{\cdir*(\numC-\numE)*\multistep/2}%
        }
        % geometrical centers
        \anchor{vcenter}{\pgf@x=0cm\relax
            \pgfmathsetlength{\pgf@y}{\cdir*(\numC-\numE)*\multistep/2}%
        }
        \anchor{gcenter}{%
            \northeast\pgf@xa=0.5\pgf@x
            \southwest\advance\pgf@xa by 0.5\pgf@x
            \pgf@x=\pgf@xa
            \pgfmathsetlength{\pgf@y}{\cdir*(\numC-\numE)*\multistep/2}%
        }
        % external connections
        \anchor{E}{
            \ifnum\cdir>0% npn, emitter down
            \southeast
            \else
            \northeast
            \fi
        }% first emitter
        \anchor{emitter}{\ifnum\cdir>0\southeast\else\northeast\fi}% first emitter
        \anchor{C}{\ifnum\cdir<0\southeast\else\northeast\fi}
        \anchor{collector}{\ifnum\cdir<0\southeast\else\northeast\fi}

        \backgroundpath{
            \pgftransformationadjustments
            \pgfsetcolor{\ctikzvalof{color}}
            %
            % set the type and up and down number of connections
            %
            \ifnum\cdir> 0
                \pgf@circuit@trans@ntypetrue
                \edef\numup{\numC}
                \edef\numdown{\numE}
            \else
                \pgf@circuit@trans@ntypefalse
                \edef\numup{\numE}
                \edef\numdown{\numC}
            \fi
            \basedimension
            \pgf@circ@res@up = \pgf@y
            \pgf@circ@res@down = -\pgf@y
            \pgf@circ@res@right = \pgf@x
            \pgf@circ@res@left = -\pgf@x
            \pgf@circ@scaled@Rlen=\scaledRlen
            \pgf@circ@res@step=\multistep
            %
            % set arrow positions options
            %
            \edef\@@anchor{center}\edef\@@pos{\ctikzvalof{tripoles/#1/arrow pos}}
            \ifpgf@circuit@trans@arrowatend
                \edef\@@pos{1.0}
                \ifpgf@circuit@trans@ntype % arrow is toward outside, push it a bit
                    \edef\@@anchor{btip}
                \else
                    \edef\@@anchor{tip}
                \fi
            \fi
            %
            % Drawing upper connections
            %
            \pgfscope
            \pgf@circ@count@a=\numup\relax
            \pgfmathloop%
            \ifnum\pgf@circ@count@a>0
            \advance\pgf@circ@count@a-1\relax
            \pgfmathsetlength{\pgf@circ@res@other}{\the\pgf@circ@count@a*\multistep}%
            \ifnum\pgf@circ@count@a=\numexpr\numup-1\relax % draw the external pin connection
            \pgfpathmoveto{\pgfpoint
                {\pgf@circ@res@right}%
                {\pgf@circ@res@up+\external+\pgfverticaltransformationadjustment*.5*\pgflinewidth+\pgf@circ@res@other}}%
            \pgfpathlineto{\pgfpoint
                {\pgf@circ@res@right}%
                {\pgf@circ@res@up+\pgfverticaltransformationadjustment*.5*\pgflinewidth+\pgf@circ@res@other}}%
            \else
            \pgfpathmoveto{\pgfpoint
                {\pgf@circ@res@right}%
                {\pgf@circ@res@up+\pgfverticaltransformationadjustment*.5*\pgflinewidth+\pgf@circ@res@other}}%
            \fi
            \pgfpathlineto{\pgfpoint
                {\ctikzvalof{tripoles/bjt/base width}\pgf@circ@res@left}%
                {\ctikzvalof{tripoles/bjt/base height 2}\pgf@circ@res@up+\pgf@circ@res@other}}%
            \pgfsetroundcap % better when connecting to sloped lines
            \pgfusepath{draw}
            \ifpgf@circuit@trans@ntype\else % it's a PNP; draw arrow
            \pgfscope
                \pgfslopedattimetrue
                \pgfallowupsidedownattimetrue
                \pgfresetnontranslationattimefalse
                \pgftransformlineattime{\@@pos}{%
                    \pgfpoint
                        {\pgf@circ@res@right}%
                        {\pgf@circ@res@up+\pgfverticaltransformationadjustment*.5*\pgflinewidth+\pgf@circ@res@other}%
                    }{\pgfpoint
                        {\ctikzvalof{tripoles/bjt/base width}\pgf@circ@res@left}%
                        {\ctikzvalof{tripoles/bjt/base height 2}\pgf@circ@res@up+\pgf@circ@res@other}%
                    }
                    \pgfnode{trarrow}{\@@anchor}{}{}{\pgfusepath{stroke}}
            \endpgfscope
            \fi
            \repeatpgfmathloop
            \endpgfscope
            %
            % Drawing base
            %
            \pgfscope
                \pgfpathmoveto{\pgfpoint
                    {\ctikzvalof{tripoles/bjt/base width}\pgf@circ@res@left}
                    {\ctikzvalof{tripoles/bjt/base height}\pgf@circ@res@up+(\numup-1)*\multistep}}
                \pgfpathlineto{\pgfpoint
                    {\ctikzvalof{tripoles/bjt/base width}\pgf@circ@res@left}
                    {\ctikzvalof{tripoles/bjt/base height}\pgf@circ@res@down-(\numdown-1)*\multistep}}
                \pgf@circ@setlinewidth{tripoles}{\pgflinewidth}
                \ifpgf@circuit@bpt@schottky
                    % upper
                    \pgfpathmoveto{\pgfpoint
                        {\ctikzvalof{tripoles/bjt/base width}\pgf@circ@res@left}
                        {\ctikzvalof{tripoles/bjt/base height}\pgf@circ@res@up+(\numup-1)*\multistep}}
                    \pgfpathlineto{\pgfpoint
                        {\ctikzvalof{tripoles/bjt/base width}\pgf@circ@res@left}
                        {\ctikzvalof{tripoles/bjt/base height}\pgf@circ@res@up+(\numup-1)*\multistep+
                        \ctikzvalof{tripoles/schottky base size}\pgf@circ@scaled@Rlen}}
                    \pgfpathlineto{\pgfpoint
                        {\ctikzvalof{tripoles/bjt/base width}\pgf@circ@res@left+
                            \ctikzvalof{tripoles/schottky base size}\pgf@circ@scaled@Rlen}
                        {\ctikzvalof{tripoles/bjt/base height}\pgf@circ@res@up+(\numup-1)*\multistep+
                            \ctikzvalof{tripoles/schottky base size}\pgf@circ@scaled@Rlen}}
                    \pgfpathlineto{\pgfpoint
                        {\ctikzvalof{tripoles/bjt/base width}\pgf@circ@res@left+
                            \ctikzvalof{tripoles/schottky base size}\pgf@circ@scaled@Rlen}
                        {\ctikzvalof{tripoles/bjt/base height}\pgf@circ@res@up+(\numup-1)*\multistep}}
                    % lower
                    \pgfpathmoveto{\pgfpoint
                        {\ctikzvalof{tripoles/bjt/base width}\pgf@circ@res@left}
                        {\ctikzvalof{tripoles/bjt/base height}\pgf@circ@res@down-(\numdown-1)*\multistep}}
                    \pgfpathlineto{\pgfpoint
                        {\ctikzvalof{tripoles/bjt/base width}\pgf@circ@res@left}
                        {\ctikzvalof{tripoles/bjt/base height}\pgf@circ@res@down-(\numdown-1)*\multistep-
                        \ctikzvalof{tripoles/schottky base size}\pgf@circ@scaled@Rlen}}
                    \pgfpathlineto{\pgfpoint
                        {\ctikzvalof{tripoles/bjt/base width}\pgf@circ@res@left-
                            \ctikzvalof{tripoles/schottky base size}\pgf@circ@scaled@Rlen}
                        {\ctikzvalof{tripoles/bjt/base height}\pgf@circ@res@down-(\numdown-1)*\multistep-
                        \ctikzvalof{tripoles/schottky base size}\pgf@circ@scaled@Rlen}}
                    \pgfpathlineto{\pgfpoint
                        {\ctikzvalof{tripoles/bjt/base width}\pgf@circ@res@left-
                            \ctikzvalof{tripoles/schottky base size}\pgf@circ@scaled@Rlen}
                        {\ctikzvalof{tripoles/bjt/base height}\pgf@circ@res@down-(\numdown-1)*\multistep}}
                \fi
                \pgfsetroundcap % I like it more...
                \pgfusepath{draw}
            \endpgfscope
            %
            % draw base external connection
            %
            \pgfpathmoveto{\pgfpoint{\ctikzvalof{tripoles/bjt/base width}\pgf@circ@res@left}{0pt}}
            \pgfpathlineto{\pgfpoint{\ctikzvalof{tripoles/bjt/base width}\pgf@circ@res@left-\external}{0pt}}
            \pgfusepath{draw}
            %
            % Drawing lower connections
            %
            \pgfscope
            \pgf@circ@count@a=\numdown\relax
            \pgfmathloop%
            \ifnum\pgf@circ@count@a>0
            \advance\pgf@circ@count@a-1\relax
            \pgfmathsetlength{\pgf@circ@res@other}{\the\pgf@circ@count@a*\multistep}%
            \pgfpathmoveto{\pgfpoint
                {\ctikzvalof{tripoles/bjt/base width}\pgf@circ@res@left}
                {\ctikzvalof{tripoles/bjt/base height 2}\pgf@circ@res@down-\pgf@circ@res@other}}
            \pgfpathlineto{\pgfpoint
                {\pgf@circ@res@right}
                {\pgf@circ@res@down-\pgfverticaltransformationadjustment*.5*\pgflinewidth-\pgf@circ@res@other}}
            \ifnum\pgf@circ@count@a=\numexpr\numdown-1\relax % draw the external pin connection
            \pgfpathlineto{\pgfpoint
                {\pgf@circ@res@right}
                {\pgf@circ@res@down-\pgfverticaltransformationadjustment*.5*\pgflinewidth-\pgf@circ@res@other-\external}}
            \fi
            \pgfsetroundcap % better when connecting to sloped lines
            \pgfusepath{draw}
            \ifpgf@circuit@trans@ntype % it's a NPN; draw arrow
            \pgfscope
                \pgfslopedattimetrue
                \pgfallowupsidedownattimetrue
                \pgfresetnontranslationattimefalse
                \pgftransformlineattime{\@@pos}{%
                    \pgfpoint
                        {\ctikzvalof{tripoles/bjt/base width}\pgf@circ@res@left}%
                        {\ctikzvalof{tripoles/bjt/base height 2}\pgf@circ@res@down-\pgf@circ@res@other}%
                    }{\pgfpoint
                        {\pgf@circ@res@right}%
                        {\pgf@circ@res@down-\pgfverticaltransformationadjustment*.5*\pgflinewidth-\pgf@circ@res@other}%
                    }
                    \pgfnode{trarrow}{\@@anchor}{}{}{\pgfusepath{stroke}}
            \endpgfscope
            \fi
            \repeatpgfmathloop
            \endpgfscope
        }
        % \pgf@sh@s@<name of the shape here> contains all the code for the shape
        % and is executed just before a node is drawn.
        \expandafter\pgfutil@g@addto@macro\csname pgf@sh@s@bjt#1\endcsname{%
            % Start with the maximum collector number and go backwards.
            \pgf@circ@count@a=\numC\relax
            \pgfmathloop%
            \ifnum\pgf@circ@count@a>0
                % we will create two anchors per pin: the "normal one" like `pin 1` for the
                % electrical contact, and the "border one" like `bpin 1` for labels.
                % they will coincide if `external pins width` is set to 0.
                \expandafter\xdef\csname pgf@anchor@bjt#1@C\the\pgf@circ@count@a\endcsname{%
                    \noexpand\pgf@circ@bjt@C@anchor{\the\pgf@circ@count@a}%
                }
            \advance\pgf@circ@count@a by -1\relax%
            \repeatpgfmathloop%
            % and emitters
            \pgf@circ@count@a=\numE\relax
            \pgfmathloop%
            \ifnum\pgf@circ@count@a>0
                % we will create two anchors per pin: the "normal one" like `pin 1` for the
                % electrical contact, and the "border one" like `bpin 1` for labels.
                % they will coincide if `external pins width` is set to 0.
                \expandafter\xdef\csname pgf@anchor@bjt#1@E\the\pgf@circ@count@a\endcsname{%
                    \noexpand\pgf@circ@bjt@E@anchor{\the\pgf@circ@count@a}%
                }
            \advance\pgf@circ@count@a by -1\relax%
            \repeatpgfmathloop%
            }%
    }
}

\declarebjt{npn}
\declarebjt{pnp}

% end of multi-bjts

\long\def\declareigbt#1{
    \pgfcircdeclaretransistor{#1}{
        \anchor{inner up}{
            \northeast
            \pgf@y=\ctikzvalof{tripoles/#1/gate height}\pgf@y
        }
        \anchor{inner down}{
            \northeast
            \pgf@y=-\ctikzvalof{tripoles/#1/gate height}\pgf@y
        }
    }
    {
        %draw upper connection
        \pgfpathmoveto{\pgfpoint{\pgf@circ@res@right}{\pgf@circ@res@up+\pgfverticaltransformationadjustment*.5*\pgflinewidth}}
        \pgfpathlineto{\pgfpoint{\pgf@circ@res@right}
        {\ctikzvalof{tripoles/#1/gate height}\pgf@circ@res@up}}
        \pgfpathlineto{\pgfpoint
            {\ctikzvalof{tripoles/#1/base width}\pgf@circ@res@left}
        {\ctikzvalof{tripoles/#1/gate height 2}\pgf@circ@res@up}}
        \pgfusepath{draw}

        %draw thicker gate lines
        \pgfscope
            \pgfscope
                \pgfpathmoveto{\pgfpoint
                    {\ctikzvalof{tripoles/#1/gate width}\pgf@circ@res@left}
                {\ctikzvalof{tripoles/#1/outer base height}\pgf@circ@res@up+\pgfverticaltransformationadjustment*.5\pgflinewidth}}
                \pgfpathlineto{\pgfpoint
                    {\ctikzvalof{tripoles/#1/gate width}\pgf@circ@res@left}
                {\ctikzvalof{tripoles/#1/outer base height}\pgf@circ@res@down-\pgfverticaltransformationadjustment*.5\pgflinewidth}}
                % set the normal thickness
                \pgf@circ@setlinewidth{tripoles}{\pgflinewidth}
                \edef\@@extrat{\ctikzvalof{tripoles/#1/outer base thickness}}
                \pgfsetlinewidth{\@@extrat\pgflinewidth}
                \pgfusepath{draw}
            \endpgfscope
            \pgfpathmoveto{\pgfpoint
                {\ctikzvalof{tripoles/#1/base width}\pgf@circ@res@left}
            {\ctikzvalof{tripoles/#1/base height}\pgf@circ@res@up+\pgfverticaltransformationadjustment*.5\pgflinewidth}}
            \pgfpathlineto{\pgfpoint
                {\ctikzvalof{tripoles/#1/base width}\pgf@circ@res@left}
            {\ctikzvalof{tripoles/#1/base height}\pgf@circ@res@down-\pgfverticaltransformationadjustment*.5\pgflinewidth}}
            \pgf@circ@setlinewidth{tripoles}{\pgflinewidth}
            \pgfusepath{draw}
        \endpgfscope
        %draw lower connection
        \pgfpathmoveto{\pgfpoint
            {\ctikzvalof{tripoles/#1/base width}\pgf@circ@res@left}
        {\ctikzvalof{tripoles/#1/gate height 2}\pgf@circ@res@down}}
        \pgfpathlineto{\pgfpoint{\pgf@circ@res@right}
        {\ctikzvalof{tripoles/#1/gate height}\pgf@circ@res@down}}
        \pgfpathlineto{\pgfpoint{\pgf@circ@res@right}{\pgf@circ@res@down-\pgfverticaltransformationadjustment*.5*\pgflinewidth}}
        \pgfusepath{draw}
        %draw arrow depending on type of transistor
        \pgfscope
            \pgfslopedattimetrue
            \pgfallowupsidedownattimetrue
            \pgfresetnontranslationattimefalse
            \ifpgf@circuit@trans@arrowatend
                \ifpgf@circuit@trans@ntype
                    \edef\@@anchor{btip}\edef\@@pos{1.0}
                \else
                    \edef\@@anchor{tip}\edef\@@pos{1.0}
                \fi
            \else
                \edef\@@anchor{center}\edef\@@pos{0.5}
            \fi
            \ifpgf@circuit@trans@ntype
                \pgftransformlineattime{\@@pos}{%
                    \pgfpoint%
                    {\ctikzvalof{tripoles/#1/base width}\pgf@circ@res@left}%
                    {\ctikzvalof{tripoles/#1/gate height 2}\pgf@circ@res@down}%
                    }{%
                    \pgfpoint{\pgf@circ@res@right}%
                    {\ctikzvalof{tripoles/#1/gate height}\pgf@circ@res@down}%
                }
            \else
                \pgftransformlineattime{\@@pos}{%
                    \pgfpoint{\pgf@circ@res@right}%
                    {\ctikzvalof{tripoles/#1/gate height}\pgf@circ@res@up}%
                    }{%
                    \pgfpoint{\ctikzvalof{tripoles/#1/base width}\pgf@circ@res@left}%
                    {\ctikzvalof{tripoles/#1/gate height 2}\pgf@circ@res@up}%
                }
            \fi
            \pgfnode{trarrow}{\@@anchor}{}{}{\pgfusepath{stroke}}
        \endpgfscope
        %draw gate
        \ifpgf@circuit@trans@ntype
            \pgfpathmoveto{\pgfpoint
                {\ctikzvalof{tripoles/#1/gate width}\pgf@circ@res@left}
            {\ctikzvalof{tripoles/#1/conn height}\pgf@circ@res@down}}
            \pgfpathlineto{\pgfpoint{\pgf@circ@res@left-\pgfhorizontaltransformationadjustment*.5*\pgflinewidth}%
            {\ctikzvalof{tripoles/#1/conn height}\pgf@circ@res@down}}
        \else
            \pgfpathmoveto{\pgfpoint
                {\ctikzvalof{tripoles/#1/gate width}\pgf@circ@res@left}
            {\ctikzvalof{tripoles/#1/conn height}\pgf@circ@res@up}}
            \pgfpathlineto{\pgfpoint{\pgf@circ@res@left-\pgfhorizontaltransformationadjustment*.5*\pgflinewidth}%
            {\ctikzvalof{tripoles/#1/conn height}\pgf@circ@res@up}}
        \fi
        \pgfusepath{draw}
    }
}

\declareigbt{pigbt}
\declareigbt{nigbt}
\declareigbt{Lnigbt}
\declareigbt{Lpigbt}

\pgfcircdeclaretransistor{nmos}{
        \anchor{inner up}{
            \northeast
            \pgf@y=\ctikzvalof{tripoles/nmos/gate height}\pgf@y
        }
        \anchor{inner down}{
            \northeast
            \pgf@y=-\ctikzvalof{tripoles/nmos/gate height}\pgf@y
        }
    }{%
    \pgfpathmoveto{\pgfpoint{\pgf@circ@res@right}{\pgf@circ@res@up+\pgfverticaltransformationadjustment*.5*\pgflinewidth}}
    \pgfpathlineto{\pgfpoint{\pgf@circ@res@right}
    {\ctikzvalof{tripoles/nmos/gate height}\pgf@circ@res@up}}
    \pgfpathlineto{\pgfpoint
        {\ctikzvalof{tripoles/nmos/base width}\pgf@circ@res@left}
    {\ctikzvalof{tripoles/nmos/gate height}\pgf@circ@res@up}}

    \pgfpathmoveto{\pgfpoint
        {\ctikzvalof{tripoles/nmos/base width}\pgf@circ@res@left}
    {\ctikzvalof{tripoles/nmos/gate height}\pgf@circ@res@down}}
    \pgfpathlineto{\pgfpoint{\pgf@circ@res@right}
    {\ctikzvalof{tripoles/nmos/gate height}\pgf@circ@res@down}}
    \pgfpathlineto{\pgfpoint{\pgf@circ@res@right}{\pgf@circ@res@down-\pgfverticaltransformationadjustment*.5*\pgflinewidth}}
    \pgfusepath{draw}

    \pgfscope
        \pgfpathmoveto{\pgfpoint
            {\ctikzvalof{tripoles/nmos/base width}\pgf@circ@res@left}
        {\ctikzvalof{tripoles/nmos/base height}\pgf@circ@res@up}}
        \pgfpathlineto{\pgfpoint
            {\ctikzvalof{tripoles/nmos/base width}\pgf@circ@res@left}
        {\ctikzvalof{tripoles/nmos/base height}\pgf@circ@res@down}}
        \pgfpathmoveto{\pgfpoint
            {\ctikzvalof{tripoles/nmos/gate width}\pgf@circ@res@left}
        {\ctikzvalof{tripoles/nmos/gate height}\pgf@circ@res@up}}
        \pgfpathlineto{\pgfpoint
            {\ctikzvalof{tripoles/nmos/gate width}\pgf@circ@res@left}
        {\ctikzvalof{tripoles/nmos/gate height}\pgf@circ@res@down}}
        \pgf@circ@setlinewidth{tripoles}{\pgflinewidth}
        \pgfusepath{draw}
    \endpgfscope
    \ifpgf@circuit@mos@arrows
        \pgfscope
            \ifpgf@circuit@trans@arrowatend
                \pgftransformshift{\pgfpoint
                    {\pgf@circ@res@right}%
                    {\ctikzvalof{tripoles/pmos/gate height}\pgf@circ@res@down}%
                }
                \pgfnode{trarrow}{btip}{}{}{\pgfusepath{stroke}}
            \else
                \pgfslopedattimetrue
                \pgfallowupsidedownattimetrue
                \pgfresetnontranslationattimefalse
                \pgftransformlineattime{\ctikzvalof{tripoles/nmos/arrow pos}}{%
                    \pgfpoint%
                    {\ctikzvalof{tripoles/nmos/gate width}\pgf@circ@res@left}%
                    {\ctikzvalof{tripoles/nmos/gate height}\pgf@circ@res@down}%
                    }{%
                    \pgfpoint
                    {\pgf@circ@res@right}%
                    {\ctikzvalof{tripoles/nmos/gate height}\pgf@circ@res@down}%
                }
                \pgfnode{currarrow}{center}{}{}{\pgfusepath{stroke}}
            \fi
        \endpgfscope
    \fi

    \ifpgf@circuit@bpt@drawgate
        \pgfpathmoveto{\pgfpoint
            {\ctikzvalof{tripoles/nmos/gate width}\pgf@circ@res@left}
        {\pgf@circ@res@up+\pgf@circ@res@down}}
        \pgfpathlineto{\pgfpoint{\pgf@circ@res@left-\pgfhorizontaltransformationadjustment*.5*\pgflinewidth}{\pgf@circ@res@up+\pgf@circ@res@down}}
        \pgfusepath{draw}
    \fi

    \ifpgf@circuit@bpt@drawbulk % added by Burak Kelleci
        \pgfpathmoveto{\pgfpoint{\ctikzvalof{tripoles/nmos/base width}\pgf@circ@res@left}
                {\pgf@circ@res@up+\pgf@circ@res@down}}
        \pgfpathlineto{\pgfpoint{\pgf@circ@res@right+\pgfhorizontaltransformationadjustment*.5*\pgflinewidth}
                {\pgf@circ@res@up+\pgf@circ@res@down}}
        \pgfusepath{draw}
    \fi
}

\pgfcircdeclaretransistor{pmos}{
        \anchor{inner up}{
            \northeast
            \pgf@y=\ctikzvalof{tripoles/pmos/gate height}\pgf@y
        }
        \anchor{inner down}{
            \northeast
            \pgf@y=-\ctikzvalof{tripoles/pmos/gate height}\pgf@y
        }
    }{%
    \pgfpathmoveto{\pgfpoint{\pgf@circ@res@right}{\pgf@circ@res@up+\pgfverticaltransformationadjustment*.5*\pgflinewidth}}
    \pgfpathlineto{\pgfpoint{\pgf@circ@res@right}
    {\ctikzvalof{tripoles/pmos/gate height}\pgf@circ@res@up}}
    \pgfpathlineto{\pgfpoint
        {\ctikzvalof{tripoles/pmos/base width}\pgf@circ@res@left}
    {\ctikzvalof{tripoles/pmos/gate height}\pgf@circ@res@up}}
    \pgfusepath{draw}

    \ifpgf@circuit@mos@arrows
        \pgfscope
            \ifpgf@circuit@trans@arrowatend
                \pgftransformshift{\pgfpoint
                    {\ctikzvalof{tripoles/pmos/base width}\pgf@circ@res@left}%
                    {\ctikzvalof{tripoles/pmos/gate height}\pgf@circ@res@up}%
                }
                \pgftransformrotate{180}
                \pgfnode{trarrow}{tip}{}{}{\pgfusepath{stroke}}
            \else
                \pgfslopedattimetrue
                \pgfallowupsidedownattimetrue
                \pgfresetnontranslationattimefalse
                \pgftransformlineattime{\ctikzvalof{tripoles/pmos/arrow pos}}{%
                    \pgfpoint%
                    {\pgf@circ@res@right}%
                    {\ctikzvalof{tripoles/pmos/gate height}\pgf@circ@res@up}%
                    }{%
                    \pgfpoint
                    {\ctikzvalof{tripoles/pmos/gate width}\pgf@circ@res@left}%
                    {\ctikzvalof{tripoles/pmos/gate height}\pgf@circ@res@up}%
                }
                \pgfnode{currarrow}{center}{}{}{\pgfusepath{stroke}}
            \fi
        \endpgfscope
    \fi

    \pgfscope
        \pgfpathmoveto{\pgfpoint
            {\ctikzvalof{tripoles/pmos/base width}\pgf@circ@res@left}
        {\ctikzvalof{tripoles/pmos/base height}\pgf@circ@res@up}}
        \pgfpathlineto{\pgfpoint
            {\ctikzvalof{tripoles/pmos/base width}\pgf@circ@res@left}
        {\ctikzvalof{tripoles/pmos/base height}\pgf@circ@res@down}}
        \pgfpathmoveto{\pgfpoint
            {\ctikzvalof{tripoles/pmos/gate width}\pgf@circ@res@left}
        {\ctikzvalof{tripoles/pmos/gate height}\pgf@circ@res@up}}
        \pgfpathlineto{\pgfpoint
            {\ctikzvalof{tripoles/pmos/gate width}\pgf@circ@res@left}
        {\ctikzvalof{tripoles/pmos/gate height}\pgf@circ@res@down}}
        \pgf@circ@setlinewidth{tripoles}{\pgflinewidth}
        \pgfusepath{draw}
    \endpgfscope

    \pgfpathmoveto{\pgfpoint
        {\ctikzvalof{tripoles/pmos/base width}\pgf@circ@res@left}
    {\ctikzvalof{tripoles/pmos/gate height}\pgf@circ@res@down}}
    \pgfpathlineto{\pgfpoint{\pgf@circ@res@right}
    {\ctikzvalof{tripoles/pmos/gate height}\pgf@circ@res@down}}
    \pgfpathlineto{\pgfpoint{\pgf@circ@res@right}{\pgf@circ@res@down-\pgfverticaltransformationadjustment*.5*\pgflinewidth}}


    \pgfusepath{draw}
    \ifpgf@circuit@bpt@drawgate
        \pgfpathmoveto{\pgfpoint
            {\ctikzvalof{tripoles/pmos/gate width}\pgf@circ@res@left}
        {\pgf@circ@res@up+\pgf@circ@res@down}}
        \pgfpathlineto{\pgfpoint{\pgf@circ@res@left-\pgfhorizontaltransformationadjustment*.5*\pgflinewidth}{\pgf@circ@res@up+\pgf@circ@res@down}}
        \pgfusepath{draw}
    \fi

    \ifpgf@circuit@bpt@drawbulk % added by Burak Kelleci
        \pgfpathmoveto{\pgfpoint{\ctikzvalof{tripoles/pmos/base width}\pgf@circ@res@left}
                {\pgf@circ@res@up+\pgf@circ@res@down}}
        \pgfpathlineto{\pgfpoint{\pgf@circ@res@right+\pgfhorizontaltransformationadjustment*.5*\pgflinewidth}
                {\pgf@circ@res@up+\pgf@circ@res@down}}
        \pgfusepath{draw}
    \fi

    \ifpgf@circuit@pmos@nocircle\else
        % we are not scaling the circle with the MOS --- I think it's better to have it
        % coherent with the poles/nodes of the rest of the circuit.
        \pgfpathcircle{\pgfpoint
            {\ctikzvalof{tripoles/pmos/gate width}\pgf@circ@res@left - \ctikzvalof{nodes width}*\pgf@circ@Rlen}
        {\pgf@circ@res@up+\pgf@circ@res@down}}{\ctikzvalof{nodes width}*\pgf@circ@Rlen}
        \ifpgf@circuit@pmos@emptycircle
            \pgfsetfillcolor{white}
        \fi
        \pgfusepath{draw,fill}
    \fi
}

%% HEMT FET Transistor
\pgfcircdeclaretransistor{hemt}{
        \anchor{inner up}{
            \northeast
            \pgf@y=\ctikzvalof{tripoles/hemt/gate height}\pgf@y
        }
        \anchor{inner down}{
            \northeast
            \pgf@y=-\ctikzvalof{tripoles/hemt/gate height}\pgf@y
        }
    }{%
    \pgfpathmoveto{\pgfpoint{\pgf@circ@res@right}{\pgf@circ@res@up+\pgfverticaltransformationadjustment*.5*\pgflinewidth}}
    \pgfpathlineto{\pgfpoint{\pgf@circ@res@right}
    {\ctikzvalof{tripoles/hemt/gate height}\pgf@circ@res@up}}
    \pgfpathlineto{\pgfpoint
        {\ctikzvalof{tripoles/hemt/base width}\pgf@circ@res@left}
    {\ctikzvalof{tripoles/hemt/gate height}\pgf@circ@res@up}}

    \pgfpathmoveto{\pgfpoint
        {\ctikzvalof{tripoles/hemt/base width}\pgf@circ@res@left}
    {\ctikzvalof{tripoles/hemt/gate height}\pgf@circ@res@down}}
    \pgfpathlineto{\pgfpoint{\pgf@circ@res@right}
    {\ctikzvalof{tripoles/hemt/gate height}\pgf@circ@res@down}}
    \pgfpathlineto{\pgfpoint{\pgf@circ@res@right}{\pgf@circ@res@down-\pgfverticaltransformationadjustment*.5*\pgflinewidth}}
    \pgfusepath{draw}

    \pgfscope
        \pgfpathmoveto{\pgfpoint
            {\ctikzvalof{tripoles/hemt/base width}\pgf@circ@res@left}
        {\ctikzvalof{tripoles/hemt/base height}\pgf@circ@res@up}}
        \pgfpathlineto{\pgfpoint
            {\ctikzvalof{tripoles/hemt/base width}\pgf@circ@res@left}
        {\ctikzvalof{tripoles/hemt/base height}\pgf@circ@res@down}}
        \pgf@circ@setlinewidth{tripoles}{\pgflinewidth}
        \pgfusepath{draw}
    \endpgfscope

    \pgfpathmoveto{\pgfpoint
        {\ctikzvalof{tripoles/hemt/base width}\pgf@circ@res@left}
    {\pgf@circ@res@up+\pgf@circ@res@down}}
    \pgfpathlineto{\pgfpoint{\pgf@circ@res@left-\pgfhorizontaltransformationadjustment*.5*\pgflinewidth}{\pgf@circ@res@up+\pgf@circ@res@down}}
    \pgfusepath{draw}
}

\long\def\drawfetcore#1{
    \pgftransformationadjustments
    %top connection
    \pgfpathmoveto{\pgfpoint{\pgf@circ@res@right}{\pgf@circ@res@up+\pgfverticaltransformationadjustment*.5*\pgflinewidth}}
    \pgfpathlineto{\pgfpoint{\pgf@circ@res@right}
    {\ctikzvalof{tripoles/#1/gate height}\pgf@circ@res@up}}
    \pgfpathlineto{\pgfpoint
        {\ctikzvalof{tripoles/#1/base width}\pgf@circ@res@left}
    {\ctikzvalof{tripoles/#1/gate height}\pgf@circ@res@up}}
    \pgfusepath{draw}
    \pgfscope
        \pgfpathmoveto{\pgfpoint
            {\ctikzvalof{tripoles/#1/base width}\pgf@circ@res@left}
        {\ctikzvalof{tripoles/#1/base height}\pgf@circ@res@up}}
        \ifpgf@circuit@trans@depletiontype
            \pgfpathlineto{\pgfpoint
                {\ctikzvalof{tripoles/#1/base width}\pgf@circ@res@left}
            {\ctikzvalof{tripoles/#1/base height}\pgf@circ@res@down}}
        \else

            \pgfpathlineto{\pgfpoint
                {\ctikzvalof{tripoles/#1/base width}\pgf@circ@res@left}
            {\ctikzvalof{tripoles/#1/base height}\pgf@circ@res@up*0.45}}
            \pgfpathmoveto{\pgfpoint
                {\ctikzvalof{tripoles/#1/base width}\pgf@circ@res@left}
            {\ctikzvalof{tripoles/#1/base height}\pgf@circ@res@up*0.25}}
            \pgfpathlineto{\pgfpoint
                {\ctikzvalof{tripoles/#1/base width}\pgf@circ@res@left}
            {\ctikzvalof{tripoles/#1/base height}\pgf@circ@res@down*0.25}}
            \pgfpathmoveto{\pgfpoint
                {\ctikzvalof{tripoles/#1/base width}\pgf@circ@res@left}
            {\ctikzvalof{tripoles/nfet/base height}\pgf@circ@res@down*0.45}}
            \pgfpathlineto{\pgfpoint
                {\ctikzvalof{tripoles/#1/base width}\pgf@circ@res@left}
            {\ctikzvalof{tripoles/#1/base height}\pgf@circ@res@down}}
        \fi
        \pgf@circ@setlinewidth{tripoles}{\pgflinewidth}
        \pgfusepath{draw}
    \endpgfscope
    %Bulk connection line
    \pgfpathmoveto{\pgfpoint
        {\ctikzvalof{tripoles/#1/base width}\pgf@circ@res@left}
    {\pgf@circ@res@up+\pgf@circ@res@down}}
    \pgfpathlineto{\pgfpoint{\pgf@circ@res@right+\pgfhorizontaltransformationadjustment*.5*\pgflinewidth}
    {\pgf@circ@res@up+\pgf@circ@res@down}}

    %bottom connection
    \pgfpathmoveto{\pgfpoint
        {\ctikzvalof{tripoles/#1/base width}\pgf@circ@res@left}
    {\ctikzvalof{tripoles/#1/gate height}\pgf@circ@res@down}}
    \pgfpathlineto{\pgfpoint{\pgf@circ@res@right}
    {\ctikzvalof{tripoles/#1/gate height}\pgf@circ@res@down}}
    \pgfpathlineto{\pgfpoint{\pgf@circ@res@right}{\pgf@circ@res@down-\pgfverticaltransformationadjustment*.5*\pgflinewidth}}
    \pgfusepath{draw}

    %draw thick gate line
    \pgfscope
        \pgfpathmoveto{\pgfpoint
            {\ctikzvalof{tripoles/#1/gate width}\pgf@circ@res@left}
        {\ctikzvalof{tripoles/#1/gate height}\pgf@circ@res@up+\pgfverticaltransformationadjustment*.5*\pgflinewidth}}
        \pgfpathlineto{\pgfpoint
            {\ctikzvalof{tripoles/#1/gate width}\pgf@circ@res@left}
        {\ctikzvalof{tripoles/#1/gate height}\pgf@circ@res@down-\pgfverticaltransformationadjustment*.5*\pgflinewidth}}
        \pgf@circ@setlinewidth{tripoles}{\pgflinewidth}
        \pgfusepath{draw}
    \endpgfscope

    % arrows
    \pgfscope
        \pgfslopedattimetrue
        \pgfallowupsidedownattimetrue
        \pgfresetnontranslationattimefalse
        \ifpgf@circuit@trans@arrowatend
                \ifpgf@circuit@trans@ntype
                    \edef\@@anchor{tip}
                    \edef\@@pos{1.0}
                \else
                    \edef\@@anchor{btip}
                    \edef\@@pos{0.0}
                \fi
        \else
            \edef\@@anchor{center}\edef\@@pos{0.6}
        \fi
        \pgftransformlineattime{\@@pos}{%
            \pgfpoint
            {\pgf@circ@res@right}%
            {\pgf@circ@res@up+\pgf@circ@res@down}%
            }{%
            \pgfpoint%
            {\ctikzvalof{tripoles/#1/base width}\pgf@circ@res@left}%
            {\pgf@circ@res@up+\pgf@circ@res@down}%
        }
        \ifpgf@circuit@trans@ntype
        \else
            \pgftransformrotate{180}
        \fi
        \pgfnode{trarrow}{\@@anchor}{}{}{\pgfusepath{stroke}}
    \endpgfscope

% GATE CONNECTION
\ifpgf@circuit@bpt@drawgate
    \ifpgf@circuit@trans@ntype
        \pgfpathmoveto{\pgfpoint
            {\ctikzvalof{tripoles/#1/gate width}\pgf@circ@res@left}
        {\ctikzvalof{tripoles/#1/conn height}\pgf@circ@res@down}}
        \pgfpathlineto{\pgfpoint{\pgf@circ@res@left-\pgfhorizontaltransformationadjustment*.5*\pgflinewidth}{\ctikzvalof{tripoles/#1/conn height}\pgf@circ@res@down}}
    \else
        \pgfpathmoveto{\pgfpoint
            {\ctikzvalof{tripoles/#1/gate width}\pgf@circ@res@left}
        {-\ctikzvalof{tripoles/#1/conn height}\pgf@circ@res@down}}
        \pgfpathlineto{\pgfpoint{\pgf@circ@res@left-\pgfhorizontaltransformationadjustment*.5*\pgflinewidth}{-\ctikzvalof{tripoles/#1/conn height}\pgf@circ@res@down}}
    \fi
    \pgfusepath{draw}
\fi

}

\long\def\pgfdeclaretransistorwrapperaddbulk#1#2#3
{\pgfcircdeclaretransistor{#1}{
        \anchor{bulk}{\left\pgf@x=0pt}
        \anchor{B}{\left\pgf@x=0pt}%override Base anchor from npn&igbt
        \anchor{inner up}{
            \northeast
            \pgf@y=\ctikzvalof{tripoles/#1/gate height}\pgf@y
        }
        \anchor{inner down}{
            \northeast
            \pgf@y=-\ctikzvalof{tripoles/#1/gate height}\pgf@y
        }
        #2
    }
    {#3}
}

\pgfdeclaretransistorwrapperaddbulk{nfet}{}{%
	\pgf@circuit@trans@depletiontypefalse
	\drawfetcore{nfet}
}

\pgfdeclaretransistorwrapperaddbulk{pfet}{}{%
    \pgf@circuit@trans@depletiontypefalse
    \drawfetcore{pfet}
}

% N-CHANNEL IGFET ENHANCEMENT TYPE
\pgfdeclaretransistorwrapperaddbulk{nigfete}{}{%
    \pgf@circuit@trans@depletiontypefalse
    \drawfetcore{nigfete}

    % SOURCE CONNECTION
    \pgfpathmoveto{\pgfpoint{\pgf@circ@res@right}{0pt}}
    \pgfpathlineto{\pgfpoint{\pgf@circ@res@right}{\pgf@circ@res@down}}
    \pgfusepath{draw}

    % SOLDER DOT at source-bulk connection
    \ifpgf@circuit@fet@solderdot
        \pgfscope
            \pgftransformshift{\pgfpoint{\pgf@circ@res@right}{
            \ctikzvalof{tripoles/nigfete/gate height}\pgf@circ@res@down}}
            \pgfnode{circ}{center}{}{}{}
    \endpgfscope{}
\fi
}

% N-CHANNEL IGFET ENHANCEMENT TYPE with Bulk connector
\pgfdeclaretransistorwrapperaddbulk{nigfetebulk}{}{%
    \pgf@circuit@trans@depletiontypefalse
    \drawfetcore{nigfetebulk}
    % SOLDER DOT at source-bulk connection
    \ifpgf@circuit@fet@solderdot
        \pgfscope
            \pgftransformshift{\pgfpoint{\pgf@circ@res@right}{
            \ctikzvalof{tripoles/nigfetebulk/gate height}\pgf@circ@res@down}}
            \pgfnode{circ}{center}{}{}{}
        \endpgfscope{}
    \fi
}

% N-CHANNEL IGFET DEPLETION TYPE
\pgfdeclaretransistorwrapperaddbulk{nigfetd}{}{%
    \pgf@circuit@trans@depletiontypetrue
    \drawfetcore{nigfetd}

    % SOURCE CONNECTION
    \pgfpathmoveto{\pgfpoint{\pgf@circ@res@right}{0pt}}
    \pgfpathlineto{\pgfpoint{\pgf@circ@res@right}{\pgf@circ@res@down}}
    \pgfusepath{draw}

    % SOLDER DOT at source-bulk connection
    \ifpgf@circuit@fet@solderdot
        \pgfscope
            \pgftransformshift{\pgfpoint{\pgf@circ@res@right}{
            \ctikzvalof{tripoles/nigfete/gate height}\pgf@circ@res@down}}
            \pgfnode{circ}{center}{}{}{}
        \endpgfscope{}
    \fi
}

% P-CHANNEL IGFET ENHANCEMENT TYPE
\pgfdeclaretransistorwrapperaddbulk{pigfete}{}{%
    \pgf@circuit@trans@depletiontypefalse
    \drawfetcore{pigfete}
    \pgfpathmoveto{\pgfpoint{\pgf@circ@res@right}{\pgf@circ@res@up}}

    % SOURCE CONNECTION
    \pgfpathmoveto{\pgfpoint{\pgf@circ@res@right}{0pt}}
    \pgfpathlineto{\pgfpoint{\pgf@circ@res@right}{\pgf@circ@res@up}}
    \pgfusepath{draw}


    % SOLDER DOT at source-bulk connection
    \ifpgf@circuit@fet@solderdot
        \pgfscope
            \pgftransformshift{\pgfpoint{\pgf@circ@res@right}{
            \ctikzvalof{tripoles/pigfete/gate height}\pgf@circ@res@up}}
            \pgfnode{circ}{center}{}{}{}
        \endpgfscope
    \fi
}

% P-CHANNEL IGFET ENHANCEMENT TYPE with bulk connector
\pgfdeclaretransistorwrapperaddbulk{pigfetebulk}{}{%
    \pgf@circuit@trans@depletiontypefalse
    \drawfetcore{pigfetebulk}
}

% P-CHANNEL IGFET DEPLETION TYPE
\pgfdeclaretransistorwrapperaddbulk{pigfetd}{}{%
    \pgf@circuit@trans@depletiontypetrue
    \drawfetcore{pigfetd}

    % SOURCE CONNECTION
    \pgfpathmoveto{\pgfpoint{\pgf@circ@res@right}{0pt}}
    \pgfpathlineto{\pgfpoint{\pgf@circ@res@right}{\pgf@circ@res@up}}
    \pgfusepath{draw}


    % SOLDER DOT at source-bulk connection
    \ifpgf@circuit@fet@solderdot
        \pgfscope
            \pgftransformshift{\pgfpoint{\pgf@circ@res@right}{
            \ctikzvalof{tripoles/nigfete/gate height}\pgf@circ@res@up}}
            \pgfnode{circ}{center}{}{}{}
        \endpgfscope{}
    \fi
}

\pgfcircdeclaretransistor{njfet}{
        \anchor{inner up}{
            \northeast
            \pgf@y=\ctikzvalof{tripoles/njfet/gate height 2}\pgf@y
        }
        \anchor{inner down}{
            \northeast
            \pgf@y=-\ctikzvalof{tripoles/njfet/gate height 2}\pgf@y
        }
    }{%
    \pgfpathmoveto{\pgfpoint{\pgf@circ@res@right}{\pgf@circ@res@up+\pgfverticaltransformationadjustment*.5*\pgflinewidth}}
    \pgfpathlineto{\pgfpoint{\pgf@circ@res@right}
    {\ctikzvalof{tripoles/njfet/gate height 2}\pgf@circ@res@up}}
    \pgfpathlineto{\pgfpoint
        {\ctikzvalof{tripoles/njfet/gate width}\pgf@circ@res@left}
    {\ctikzvalof{tripoles/njfet/gate height 2}\pgf@circ@res@up}}

    \pgfpathmoveto{\pgfpoint
        {\ctikzvalof{tripoles/njfet/gate width}\pgf@circ@res@left}
    {\ctikzvalof{tripoles/njfet/gate height 2}\pgf@circ@res@down}}
    \pgfpathlineto{\pgfpoint{\pgf@circ@res@right}
    {\ctikzvalof{tripoles/njfet/gate height 2}\pgf@circ@res@down}}
    \pgfpathlineto{\pgfpoint{\pgf@circ@res@right}{\pgf@circ@res@down-\pgfverticaltransformationadjustment*.5*\pgflinewidth}}
    \pgfusepath{draw}

    \pgfscope
        \pgfpathmoveto{\pgfpoint
            {\ctikzvalof{tripoles/njfet/gate width}\pgf@circ@res@left}
        {\ctikzvalof{tripoles/njfet/gate height}\pgf@circ@res@up}}
        \pgfpathlineto{\pgfpoint
            {\ctikzvalof{tripoles/njfet/gate width}\pgf@circ@res@left}
        {\ctikzvalof{tripoles/njfet/gate height}\pgf@circ@res@down}}
        \pgf@circ@setlinewidth{tripoles}{\pgflinewidth}
        \pgfusepath{draw}
    \endpgfscope

    % arrow
    \pgfscope
        \pgfslopedattimetrue
        \pgfallowupsidedownattimetrue
        \pgfresetnontranslationattimefalse
        \ifpgf@circuit@trans@arrowatend
            \edef\@@anchor{tip}\edef\@@pos{1.0}
        \else
            \edef\@@anchor{center}\edef\@@pos{0.6}
        \fi
        \pgftransformlineattime{\@@pos}{%
            \pgfpoint{\pgf@circ@res@left}%
            {\ctikzvalof{tripoles/njfet/gate height 2}\pgf@circ@res@down}%
            }{%
            \pgfpoint
            {\ctikzvalof{tripoles/njfet/gate width}\pgf@circ@res@left}%
            {\ctikzvalof{tripoles/njfet/gate height 2}\pgf@circ@res@down}%
        }
        \pgfnode{trarrow}{\@@anchor}{}{}{\pgfusepath{stroke}}
    \endpgfscope

    \pgfpathmoveto{\pgfpoint
        {\ctikzvalof{tripoles/njfet/gate width}\pgf@circ@res@left}
    {\ctikzvalof{tripoles/njfet/gate height 2}\pgf@circ@res@down}}
    \pgfpathlineto{\pgfpoint{\pgf@circ@res@left-\pgfhorizontaltransformationadjustment*.5*\pgflinewidth}
    {\ctikzvalof{tripoles/njfet/gate height 2}\pgf@circ@res@down}}
    \pgfusepath{draw}
}

\pgfcircdeclaretransistor{pjfet}{
        \anchor{inner up}{
            \northeast
            \pgf@y=\ctikzvalof{tripoles/pjfet/gate height 2}\pgf@y
        }
        \anchor{inner down}{
            \northeast
            \pgf@y=-\ctikzvalof{tripoles/pjfet/gate height 2}\pgf@y
        }
    }{%
    \pgfpathmoveto{\pgfpoint{\pgf@circ@res@right}{\pgf@circ@res@up+\pgfverticaltransformationadjustment*.5*\pgflinewidth}}
    \pgfpathlineto{\pgfpoint{\pgf@circ@res@right}
    {\ctikzvalof{tripoles/pjfet/gate height 2}\pgf@circ@res@up}}
    \pgfpathlineto{\pgfpoint
        {\ctikzvalof{tripoles/pjfet/gate width}\pgf@circ@res@left}
    {\ctikzvalof{tripoles/pjfet/gate height 2}\pgf@circ@res@up}}

    \pgfpathmoveto{\pgfpoint
        {\ctikzvalof{tripoles/pjfet/gate width}\pgf@circ@res@left}
    {\ctikzvalof{tripoles/pjfet/gate height 2}\pgf@circ@res@down}}
    \pgfpathlineto{\pgfpoint{\pgf@circ@res@right}
    {\ctikzvalof{tripoles/pjfet/gate height 2}\pgf@circ@res@down}}
    \pgfpathlineto{\pgfpoint{\pgf@circ@res@right}{\pgf@circ@res@down-\pgfverticaltransformationadjustment*.5*\pgflinewidth}}
    \pgfusepath{draw}

    \pgfscope
        \pgfpathmoveto{\pgfpoint
            {\ctikzvalof{tripoles/pjfet/gate width}\pgf@circ@res@left}
        {\ctikzvalof{tripoles/pjfet/gate height}\pgf@circ@res@up}}
        \pgfpathlineto{\pgfpoint
            {\ctikzvalof{tripoles/pjfet/gate width}\pgf@circ@res@left}
        {\ctikzvalof{tripoles/pjfet/gate height}\pgf@circ@res@down}}
        \pgf@circ@setlinewidth{tripoles}{\pgflinewidth}
        \pgfusepath{draw}
    \endpgfscope

    \pgfscope
        \pgfslopedattimetrue
        \pgfallowupsidedownattimetrue
        \pgfresetnontranslationattimefalse
        \ifpgf@circuit@trans@arrowatend
            \edef\@@anchor{btip}\edef\@@pos{1.0}
        \else
            \edef\@@anchor{center}\edef\@@pos{0.4}
        \fi
        \pgftransformlineattime{\@@pos}{%
            \pgfpoint%
            {\ctikzvalof{tripoles/pjfet/gate width}\pgf@circ@res@left}%
            {\ctikzvalof{tripoles/pjfet/gate height 2}\pgf@circ@res@up}%
            }{%
            \pgfpoint{\pgf@circ@res@left}%
            {\ctikzvalof{tripoles/pjfet/gate height 2}\pgf@circ@res@up}%
        }
        \pgfnode{trarrow}{\@@anchor}{}{}{\pgfusepath{stroke}}
    \endpgfscope

    \pgfpathmoveto{\pgfpoint
        {\ctikzvalof{tripoles/pjfet/gate width}\pgf@circ@res@left}
    {\ctikzvalof{tripoles/pjfet/gate height 2}\pgf@circ@res@up}}
    \pgfpathlineto{\pgfpoint{\pgf@circ@res@left-\pgfhorizontaltransformationadjustment*.5*\pgflinewidth}
    {\ctikzvalof{tripoles/pjfet/gate height 2}\pgf@circ@res@up}}
    \pgfusepath{draw}
}

\pgfdeclaretransistorwrapperaddbulk{isfet}{}{%
    \pgfpathmoveto{\pgfpoint{\pgf@circ@res@right}{\pgf@circ@res@up+\pgfverticaltransformationadjustment*.5*\pgflinewidth}}
    % DRAIN CONNECTION
    \pgfpathlineto{\pgfpoint{\pgf@circ@res@right}
    {\ctikzvalof{tripoles/isfet/gate height}\pgf@circ@res@up}}
    % DRAIN
    \pgfpathlineto{\pgfpoint
        {\ctikzvalof{tripoles/isfet/base width}\pgf@circ@res@left}
    {\ctikzvalof{tripoles/isfet/gate height}\pgf@circ@res@up}}
    \pgfusepath{draw}

    % GATE, DEPLETION TYPE
    \pgfscope %% added
        \pgfpathmoveto{\pgfpoint
            {\ctikzvalof{tripoles/isfet/base width}\pgf@circ@res@left}
        {\ctikzvalof{tripoles/isfet/base height}\pgf@circ@res@up}}
        \pgfpathlineto{\pgfpoint
            {\ctikzvalof{tripoles/isfet/base width}\pgf@circ@res@left}
        {\ctikzvalof{tripoles/isfet/base height}\pgf@circ@res@down}}
        \pgf@circ@setlinewidth{tripoles}{\pgflinewidth} %% added
        \pgfusepath{draw} %% added
    \endpgfscope %% added

    % BULK
    \pgfpathmoveto{\pgfpoint
        {\ctikzvalof{tripoles/isfet/base width}\pgf@circ@res@left}
    {\pgf@circ@res@up+\pgf@circ@res@down}}
    \pgfpathlineto{\pgfpoint{\pgf@circ@res@right+.5\pgflinewidth}
    {\pgf@circ@res@up+\pgf@circ@res@down}}

    % SOURCE
    \pgfpathmoveto{\pgfpoint
        {\ctikzvalof{tripoles/isfet/base width}\pgf@circ@res@left}
    {\ctikzvalof{tripoles/isfet/gate height}\pgf@circ@res@down}}
    \pgfpathlineto{\pgfpoint{\pgf@circ@res@right}
    {\ctikzvalof{tripoles/isfet/gate height}\pgf@circ@res@down}}
    % SOURCE CONNECTION
    \pgfpathmoveto{\pgfpoint{\pgf@circ@res@right}{0pt}}
    \pgfpathlineto{\pgfpoint{\pgf@circ@res@right}{\pgf@circ@res@down-\pgfverticaltransformationadjustment*.5*\pgflinewidth}}
    \pgfusepath{draw}

    % SOLDER DOT at source-bulk connection
    \ifpgf@circuit@fet@solderdot
        \pgfscope
            \pgftransformshift{\pgfpoint{\pgf@circ@res@right}{
            \ctikzvalof{tripoles/nigfete/gate height}\pgf@circ@res@down}}
            \pgfnode{circ}{center}{}{}{}
        \endpgfscope{}
    \fi
    % ARROW
    \pgfscope
        \pgfslopedattimetrue
        \pgfallowupsidedownattimetrue
        \pgfresetnontranslationattimefalse
        \ifpgf@circuit@trans@arrowatend
            \edef\@@anchor{tip}\edef\@@pos{1.0}
        \else
            \edef\@@anchor{center}\edef\@@pos{0.6}
        \fi
        \pgftransformlineattime{\@@pos}{%
            \pgfpoint
            {\pgf@circ@res@right}%
            {\pgf@circ@res@up+\pgf@circ@res@down}%
            }{%
            \pgfpoint%
            {\ctikzvalof{tripoles/isfet/base width}\pgf@circ@res@left}%
            {\pgf@circ@res@up+\pgf@circ@res@down}%
        }
        \pgfnode{trarrow}{\@@anchor}{}{}{\pgfusepath{stroke}}
    \endpgfscope
    \pgfusepath{draw}

    % Wavy lines
    \pgfscope
        \pgfpathmoveto{\pgfpoint{-\ctikzvalof{tripoles/isfet/waves x sep}\pgf@circ@res@up}{\ctikzvalof{tripoles/isfet/waves y sep}\pgf@circ@res@up}}
        \pgfpathsine{\pgfpoint{-\ctikzvalof{tripoles/isfet/wave width}\pgf@circ@res@up}{-\ctikzvalof{tripoles/isfet/wave amp}\pgf@circ@res@up}}
        \pgfpathcosine{\pgfpoint{-\ctikzvalof{tripoles/isfet/wave width}\pgf@circ@res@up}{\ctikzvalof{tripoles/isfet/wave amp}\pgf@circ@res@up}}
        \pgfpathsine{\pgfpoint{-\ctikzvalof{tripoles/isfet/wave width}\pgf@circ@res@up}{\ctikzvalof{tripoles/isfet/wave amp}\pgf@circ@res@up}}
        \pgfpathcosine{\pgfpoint{-\ctikzvalof{tripoles/isfet/wave width}\pgf@circ@res@up}{-\ctikzvalof{tripoles/isfet/wave amp}\pgf@circ@res@up}}
        \pgfusepath{draw}

        \pgfpathmoveto{\pgfpoint{-\ctikzvalof{tripoles/isfet/waves x sep}\pgf@circ@res@up}{0cm}}
        \pgfpathsine{\pgfpoint{-\ctikzvalof{tripoles/isfet/wave width}\pgf@circ@res@up}{-\ctikzvalof{tripoles/isfet/wave amp}\pgf@circ@res@up}}
        \pgfpathcosine{\pgfpoint{-\ctikzvalof{tripoles/isfet/wave width}\pgf@circ@res@up}{\ctikzvalof{tripoles/isfet/wave amp}\pgf@circ@res@up}}
        \pgfpathsine{\pgfpoint{-\ctikzvalof{tripoles/isfet/wave width}\pgf@circ@res@up}{\ctikzvalof{tripoles/isfet/wave amp}\pgf@circ@res@up}}
        \pgfpathcosine{\pgfpoint{-\ctikzvalof{tripoles/isfet/wave width}\pgf@circ@res@up}{-\ctikzvalof{tripoles/isfet/wave amp}\pgf@circ@res@up}}
        \pgfusepath{draw}

        \pgfpathmoveto{\pgfpoint{-\ctikzvalof{tripoles/isfet/waves x sep}\pgf@circ@res@up}{-\ctikzvalof{tripoles/isfet/waves y sep}\pgf@circ@res@up}}
        \pgfpathsine{\pgfpoint{-\ctikzvalof{tripoles/isfet/wave width}\pgf@circ@res@up}{-\ctikzvalof{tripoles/isfet/wave amp}\pgf@circ@res@up}}
        \pgfpathcosine{\pgfpoint{-\ctikzvalof{tripoles/isfet/wave width}\pgf@circ@res@up}{\ctikzvalof{tripoles/isfet/wave amp}\pgf@circ@res@up}}
        \pgfpathsine{\pgfpoint{-\ctikzvalof{tripoles/isfet/wave width}\pgf@circ@res@up}{\ctikzvalof{tripoles/isfet/wave amp}\pgf@circ@res@up}}
        \pgfpathcosine{\pgfpoint{-\ctikzvalof{tripoles/isfet/wave width}\pgf@circ@res@up}{-\ctikzvalof{tripoles/isfet/wave amp}\pgf@circ@res@up}}
        \pgfusepath{draw}
    \endpgfscope
}

% end of transistors

%%%%%%%%%%%%%
%% Switches
%%%%%%%%%%%%%

\pgfcircdeclarebipole{
	\anchor{out 1}{
		\northeast
		\pgf@y=0cm
	}
	\anchor{out 2}{
		\northeast
		\pgf@y=.8\pgf@y
	}
}
{\ctikzvalof{tripoles/toggleswitch/height 2}}
{toggleswitch}
{\ctikzvalof{tripoles/toggleswitch/height}}
{\ctikzvalof{tripoles/toggleswitch/width}}
{

    \pgfpathmoveto{\pgfpoint{\pgf@circ@res@left}{0pt}}
    \pgfpathlineto{\pgfpoint{.3\pgf@circ@res@left}{0pt}}
    \pgfusepath{draw}
    \pgfscope
        \pgfpathmoveto{\pgfpoint{.3\pgf@circ@res@left}{0pt}}
        \pgf@circ@setlinewidth{bipoles}{\pgfstartlinewidth}
        \pgfpathlineto{\pgfpoint{.7\pgf@circ@res@right}{.5\pgf@circ@res@up}}
        \pgfusepath{draw}
    \endpgfscope

    \pgfpathmoveto{\pgfpoint{\pgf@circ@res@right}{0pt}}
    \pgfpathlineto{\pgfpoint{.5\pgf@circ@res@right}{0}}

    \pgfpathmoveto{\pgfpoint{\pgf@circ@res@right}{.8\pgf@circ@res@up}}
    \pgfpathlineto{\pgfpoint{.5\pgf@circ@res@right}{.8\pgf@circ@res@up}}
    \pgfpathlineto{\pgfpoint{.5\pgf@circ@res@right}{.15\pgf@circ@res@up}}
    \pgfusepath{draw}


    \pgfpathmoveto{\pgfpoint{.2\pgf@circ@res@left}{\pgf@circ@res@up}}
    \pgfpathlineto{\pgfpoint{.2\pgf@circ@res@right}{\pgf@circ@res@up}}
    \pgfusepath{draw}

    \pgfsetdash{{.08\pgf@circ@res@up}{.04\pgf@circ@res@up}{.7\pgf@circ@res@up}{.04\pgf@circ@res@up}{.8\pgf@circ@res@up}}{0cm}
    \pgfpathmoveto{\pgfpoint{0}{\pgf@circ@res@up}}
    \pgfpathlineto{\pgfpoint{0}{.8\pgf@circ@res@up}}
    \pgfpathlineto{\pgfpoint{.2\pgf@circ@res@left}{.6\pgf@circ@res@up}}
    \pgfpathlineto{\pgfpoint{0}{.4\pgf@circ@res@up}}
    \pgfpathlineto{\pgfpoint{0}{.2\pgf@circ@res@up}}
    \pgfusepath{draw}
    \pgfsetdash{}{0cm}
}

%%%%%%%%%%%%%%%%%%%%%%%%%%%%%%%%%%%%%%%%%%%%%%
%% operational and instrumentation amplifiers
%%%%%%%%%%%%%%%%%%%%%%%%%%%%%%%%%%%%%%%%%%%%%%

\pgfdeclareshape{op amp}
{
    \savedmacro{\ctikzclass}{\edef\ctikzclass{amplifiers}}
    \saveddimen{\scaledRlen}{\pgfmathsetlength{\pgf@x}{\ctikzvalof{\ctikzclass/scale}\pgf@circ@Rlen}}
    \anchor{center}{\pgfpointorigin}
    \savedanchor\northwest{%
        \pgfmathsetlength{\pgf@circ@scaled@Rlen}{\ctikzvalof{\ctikzclass/scale}\pgf@circ@Rlen}
        \pgf@y=\ctikzvalof{tripoles/op amp/height}\pgf@circ@scaled@Rlen
        \pgf@y=.5\pgf@y
        \pgf@x=-\ctikzvalof{tripoles/op amp/width}\pgf@circ@scaled@Rlen
        \pgf@x=.5\pgf@x
    }
    \savedanchor\left{%
        \pgfmathsetlength{\pgf@circ@scaled@Rlen}{\ctikzvalof{\ctikzclass/scale}\pgf@circ@Rlen}
        \pgf@x=-\ctikzvalof{tripoles/op amp/width}\pgf@circ@scaled@Rlen
        \pgf@x=.5\pgf@x
        \pgf@y=0pt
    }
    \anchor{south}{
        \northwest
        \pgf@y=-\pgf@y
        \pgf@x=0pt
    }
    \anchor{north}{
        \northwest
        \pgf@x=0pt
    }
    \savedanchor\left{%
        \pgfmathsetlength{\pgf@circ@scaled@Rlen}{\ctikzvalof{\ctikzclass/scale}\pgf@circ@Rlen}
        \pgf@x=-\ctikzvalof{tripoles/op amp/width}\pgf@circ@scaled@Rlen
        \pgf@x=.5\pgf@x
        \pgf@y=0pt
    }
    \savedanchor\inOneFixed{%
        \pgfmathsetlength{\pgf@circ@scaled@Rlen}{\ctikzvalof{\ctikzclass/scale}\pgf@circ@Rlen}
        \pgf@y=\ctikzvalof{tripoles/op amp/height}\pgf@circ@scaled@Rlen
        \pgf@y=.5\pgf@y
        \pgf@y=\ctikzvalof{tripoles/op amp/input height}\pgf@y
        \pgf@x=-\ctikzvalof{tripoles/op amp/width}\pgf@circ@scaled@Rlen
        \pgf@x=.5\pgf@x
    }
    \anchor{in up}{
        \inOneFixed
    }
    \anchor{in down}{
        \inOneFixed
        \pgf@y=-\pgf@y
    }
    \savedanchor\inOne{%
        \pgfmathsetlength{\pgf@circ@scaled@Rlen}{\ctikzvalof{\ctikzclass/scale}\pgf@circ@Rlen}
        \pgf@y=\ctikzvalof{tripoles/op amp/height}\pgf@circ@scaled@Rlen
        \pgf@y=.5\pgf@y
        \pgf@y=\ctikzvalof{tripoles/op amp/input height}\pgf@y
        \pgf@x=-\ctikzvalof{tripoles/op amp/width}\pgf@circ@scaled@Rlen
        \pgf@x=.5\pgf@x
    \ifpgf@circuit@oa@iplusup\pgf@y=-\pgf@y\fi
    }
    \anchor{-}{
        \inOne
    }
    \anchor{+}{
        \inOne
        \pgf@y=-\pgf@y
    }
    \savedanchor\up{%
        \pgfmathsetlength{\pgf@circ@scaled@Rlen}{\ctikzvalof{\ctikzclass/scale}\pgf@circ@Rlen}
        \pgf@y=\ctikzvalof{tripoles/op amp/height}\pgf@circ@scaled@Rlen
        \pgf@y=.5\pgf@y
        \pgf@x=-\ctikzvalof{tripoles/op amp/width}\pgf@circ@scaled@Rlen
        \pgf@x=.5\pgf@x
        \pgf@circ@res@up = \pgf@y
        \pgf@circ@res@right = -\pgf@x
        \pgf@circ@res@left = \pgf@x
        \pgfpointlineattime{
            \ctikzvalof{tripoles/op amp/up pos}}{
            \pgfpoint{
            \ctikzvalof{tripoles/op amp/port width}\pgf@circ@res@left}
        {\pgf@circ@res@up}}
        {\pgfpoint{\ctikzvalof{tripoles/op amp/port width}\pgf@circ@res@right}{0pt}}
    }
    \anchor{up}{
        \up
    }
    \anchor{down}{
        \up
        \pgf@y=-\pgf@y
    }
    \anchor{out}{
        \northwest
        \pgf@y=0pt
        \pgf@x=-\pgf@x
    }
    \anchor{west}{
        \left
    }
    \anchor{east}{
        \left
        \pgf@x=-\pgf@x
    }

    % support for border anchors
    \savedanchor\leftedge{%
        \pgfmathsetlength{\pgf@circ@scaled@Rlen}{\ctikzvalof{\ctikzclass/scale}\pgf@circ@Rlen}
        \pgf@x=-\ctikzvalof{tripoles/op amp/width}\pgf@circ@scaled@Rlen
        \pgf@x=.5\pgf@x
        \pgf@x = \ctikzvalof{tripoles/op amp/port width}\pgf@x
        \pgf@y=0pt
    }
    \anchor{leftedge}{\leftedge}
    \anchor{rightedge}{\leftedge \pgf@x =-\pgf@x }
    \anchor{bin up}{
        \inOneFixed
        \pgf@ya=\pgf@y \leftedge \pgf@y=\pgf@ya
    }
    \anchor{bin down}{
        \inOneFixed
        \pgf@y=-\pgf@y
        \pgf@ya=\pgf@y \leftedge \pgf@y=\pgf@ya
    }
    \anchor{b-}{
        \inOne
        \pgf@ya=\pgf@y \leftedge \pgf@y=\pgf@ya
    }
    \anchor{b+}{
        \inOne
        \pgf@y=-\pgf@y
        \pgf@ya=\pgf@y \leftedge \pgf@y=\pgf@ya
    }
    \anchor{bout}{
        \leftedge
        \pgf@y=0pt
        \pgf@x=-\pgf@x
    }
    % end border anchors
    \anchor{south west}{ \northwest \pgf@y=-\pgf@y }
    \anchor{north east}{ \northwest \pgf@x=-\pgf@x }
    \anchor{north west}{ \northwest }
    \anchor{south east}{ \northwest \pgf@x=-\pgf@x \pgf@y=-\pgf@y }

    \anchor{text}{
        \ifpgf@circ@center@text
            \pgf@x=\dimexpr -.6\wd\pgfnodeparttextbox\relax
        \else
            \leftedge
            \pgfmathsetlength{\pgf@x}{\pgf@x + \ctikzvalof{left text distance}}
        \fi
        \pgf@y=\dimexpr.5\dp\pgfnodeparttextbox-.5\ht\pgfnodeparttextbox\relax
    }

    \backgroundpath{
        \pgfsetcolor{\ctikzvalof{color}}

        \northwest
        \pgf@circ@res@up = \pgf@y
        \pgf@circ@res@down = -\pgf@y
        \pgf@circ@res@right = -\pgf@x
        \pgf@circ@res@left = \pgf@x
        \pgf@circ@scaled@Rlen=\scaledRlen

				% Triangle
        \pgfscope
            \pgf@circ@setlinewidth{tripoles}{\pgflinewidth}
            \pgftransformxshift{\ctikzvalof{tripoles/op amp/port width}\pgf@circ@res@left}
            \pgf@circ@res@step=\pgf@circ@res@right
            \advance\pgf@circ@res@step by -\pgf@circ@res@left
            \pgf@circ@res@step=\ctikzvalof{tripoles/op amp/port width}\pgf@circ@res@step

            \pgfpathmoveto{\pgfpoint{\pgf@circ@res@step}{0pt}}
            \pgfpathlineto{\pgfpoint{0pt}{\pgf@circ@res@up}}
            \pgfpathlineto{\pgfpoint{0pt}{\pgf@circ@res@down}}
            \pgfpathclose
            \pgf@circ@draworfill
        \endpgfscope

				% Negative input terminal
        \pgfpathmoveto{\pgfpoint
            {\pgf@circ@res@left}
            {\ctikzvalof{tripoles/op amp/input height}\pgf@circ@res@up}}
        \pgfpathlineto{\pgfpoint
            {\ctikzvalof{tripoles/op amp/port width}\pgf@circ@res@left}
            {\ctikzvalof{tripoles/op amp/input height}\pgf@circ@res@up}}
        \pgftext[left, at=\pgfpoint{\ctikzvalof{tripoles/op amp/port width}\pgf@circ@res@left}{\ctikzvalof{tripoles/op amp/input height}\pgf@circ@res@up}]{\ctikzvalof{tripoles/op amp/font} \ifpgf@circuit@oa@iplusup\pgf@circ@ampli@plus\else\pgf@circ@ampli@minus\fi}

				% Positive input terminal
        \pgfpathmoveto{\pgfpoint
            {\pgf@circ@res@left}
            {\ctikzvalof{tripoles/op amp/input height}\pgf@circ@res@down}}
        \pgfpathlineto{\pgfpoint
            {\ctikzvalof{tripoles/op amp/port width}\pgf@circ@res@left}
            {\ctikzvalof{tripoles/op amp/input height}\pgf@circ@res@down}}
        \pgftext[left, at=\pgfpoint{\ctikzvalof{tripoles/op amp/port width}\pgf@circ@res@left}{\ctikzvalof{tripoles/op amp/input height}\pgf@circ@res@down}]{\ctikzvalof{tripoles/op amp/font} \ifpgf@circuit@oa@iplusup\pgf@circ@ampli@minus\else\pgf@circ@ampli@plus\fi}

        % Output terminal
        \pgfpathmoveto{\pgfpoint{\pgf@circ@res@right}{0pt}}
        \pgfpathlineto{\pgfpoint{\ctikzvalof{tripoles/op amp/port width}\pgf@circ@res@right}{0pt}}
        \pgfsetrectcap
        \pgfusepath{draw}
    }
}

% Op amp shape as in european standard EN 60617
\pgfdeclareshape{en amp}
{
    \savedmacro{\ctikzclass}{\edef\ctikzclass{amplifiers}}
    \saveddimen{\scaledRlen}{\pgfmathsetlength{\pgf@x}{\ctikzvalof{\ctikzclass/scale}\pgf@circ@Rlen}}
    \anchor{center}{\pgfpointorigin}
    \savedanchor\northwest{%
        \pgfmathsetlength{\pgf@circ@scaled@Rlen}{\ctikzvalof{\ctikzclass/scale}\pgf@circ@Rlen}
        \pgf@y=\ctikzvalof{tripoles/en amp/height}\pgf@circ@scaled@Rlen
        \pgf@y=.5\pgf@y
        \pgf@x=-\ctikzvalof{tripoles/en amp/width}\pgf@circ@scaled@Rlen
        \pgf@x=.5\pgf@x
    }
    \savedanchor\left{%
        \pgfmathsetlength{\pgf@circ@scaled@Rlen}{\ctikzvalof{\ctikzclass/scale}\pgf@circ@Rlen}
        \pgf@x=-\ctikzvalof{tripoles/en amp/width}\pgf@circ@scaled@Rlen
        \pgf@x=.5\pgf@x
        \pgf@y=0pt
    }
    \anchor{south}{
        \northwest
        \pgf@y=-\pgf@y
        \pgf@x=0pt
    }
    \anchor{north}{
        \northwest
        \pgf@x=0pt
    }
    \savedanchor\inOneFixed{%
        \pgfmathsetlength{\pgf@circ@scaled@Rlen}{\ctikzvalof{\ctikzclass/scale}\pgf@circ@Rlen}
        \pgf@y=\ctikzvalof{tripoles/en amp/height}\pgf@circ@scaled@Rlen
        \pgf@y=.5\pgf@y
        \pgf@y=\ctikzvalof{tripoles/en amp/input height}\pgf@y
        \pgf@x=-\ctikzvalof{tripoles/en amp/width}\pgf@circ@scaled@Rlen
        \pgf@x=.5\pgf@x
    }
    \anchor{in up}{
        \inOneFixed
    }
    \anchor{in down}{
        \inOneFixed
        \pgf@y=-\pgf@y
    }
    \savedanchor\inOne{%
        \pgfmathsetlength{\pgf@circ@scaled@Rlen}{\ctikzvalof{\ctikzclass/scale}\pgf@circ@Rlen}
        \pgf@y=\ctikzvalof{tripoles/en amp/height}\pgf@circ@scaled@Rlen
        \pgf@y=.5\pgf@y
        \pgf@y=\ctikzvalof{tripoles/en amp/input height}\pgf@y
        \pgf@x=-\ctikzvalof{tripoles/en amp/width}\pgf@circ@scaled@Rlen
        \pgf@x=.5\pgf@x
    \ifpgf@circuit@oa@iplusup\pgf@y=-\pgf@y\fi
    }
    \anchor{-}{
        \inOne
    }
    \anchor{+}{
        \inOne
        \pgf@y=-\pgf@y
    }
    \anchor{up}{
        \northwest
        \pgf@x=0pt
    }
    \anchor{down}{
        \northwest
        \pgf@x=0pt
        \pgf@y=-\pgf@y
    }
    \anchor{out}{
        \northwest
        \pgf@y=0pt
        \pgf@x=-\pgf@x
    }
    \anchor{west}{
        \left
    }
    \anchor{east}{
        \left
        \pgf@x=-\pgf@x
    }
    % support for border anchors
    \savedanchor\leftedge{%
        \pgfmathsetlength{\pgf@circ@scaled@Rlen}{\ctikzvalof{\ctikzclass/scale}\pgf@circ@Rlen}
        \pgf@x=-\ctikzvalof{tripoles/en amp/width}\pgf@circ@scaled@Rlen
        \pgf@x=.5\pgf@x
        \pgf@x = \ctikzvalof{tripoles/en amp/port width}\pgf@x
        \pgf@y=0pt
    }
    \anchor{leftedge}{\leftedge}
    \anchor{rightedge}{\leftedge \pgf@x =-\pgf@x }
    \anchor{bin up}{
        \inOneFixed
        \pgf@ya=\pgf@y \leftedge \pgf@y=\pgf@ya
    }
    \anchor{bin down}{
        \inOneFixed
        \pgf@y=-\pgf@y
        \pgf@ya=\pgf@y \leftedge \pgf@y=\pgf@ya
    }
    \anchor{b-}{
        \inOne
        \pgf@ya=\pgf@y \leftedge \pgf@y=\pgf@ya
    }
    \anchor{b+}{
        \inOne
        \pgf@y=-\pgf@y
        \pgf@ya=\pgf@y \leftedge \pgf@y=\pgf@ya
    }
    \anchor{bout}{
        \leftedge
        \pgf@y=0pt
        \pgf@x=-\pgf@x
    }
    % end border anchors
    \anchor{south west}{ \northwest \pgf@y=-\pgf@y }
    \anchor{north east}{ \northwest \pgf@x=-\pgf@x }
    \anchor{north west}{ \northwest }
    \anchor{south east}{ \northwest \pgf@x=-\pgf@x \pgf@y=-\pgf@y }
    \anchor{text}{
        \ifpgf@circ@center@text
            \pgf@x=\dimexpr -.6\wd\pgfnodeparttextbox\relax
        \else
            \leftedge
            \pgfmathsetlength{\pgf@x}{\pgf@x + \ctikzvalof{left text distance}}
        \fi
        \pgf@y=\dimexpr.5\dp\pgfnodeparttextbox-.5\ht\pgfnodeparttextbox\relax
    }

    \backgroundpath{
        \pgfsetcolor{\ctikzvalof{color}}
        \northwest
        \pgf@circ@res@up = \pgf@y
        \pgf@circ@res@down = -\pgf@y
        \pgf@circ@res@right = -\pgf@x
        \pgf@circ@res@left = \pgf@x
        \pgf@circ@scaled@Rlen=\scaledRlen

        \pgfscope
            \pgf@circ@setlinewidth{tripoles}{\pgflinewidth}
            \pgfpathmoveto{\pgfpoint{\ctikzvalof{tripoles/en amp/port width}\pgf@circ@res@left}{\pgf@circ@res@down}}
            \pgfpathlineto{\pgfpoint{\ctikzvalof{tripoles/en amp/port width}\pgf@circ@res@left}{\pgf@circ@res@up}}
            \pgfpathlineto{\pgfpoint{\ctikzvalof{tripoles/en amp/port width}\pgf@circ@res@right}{\pgf@circ@res@up}}
            \pgfpathlineto{\pgfpoint{\ctikzvalof{tripoles/en amp/port width}\pgf@circ@res@right}{\pgf@circ@res@down}}
            \pgfpathclose
            \pgf@circ@draworfill
        \endpgfscope

        \pgfpathmoveto{\pgfpoint
            {\pgf@circ@res@left}
            {\ctikzvalof{tripoles/en amp/input height}\pgf@circ@res@up}}
        \pgfpathlineto{\pgfpoint
            {\ctikzvalof{tripoles/en amp/port width}\pgf@circ@res@left}
            {\ctikzvalof{tripoles/en amp/input height}\pgf@circ@res@up}}
        \pgftext[left, at=\pgfpoint{\ctikzvalof{tripoles/en amp/port width}\pgf@circ@res@left}{\ctikzvalof{tripoles/en amp/input height}\pgf@circ@res@up}]{\ctikzvalof{tripoles/en amp/font} \ifpgf@circuit@oa@iplusup\pgf@circ@ampli@plus\else\pgf@circ@ampli@minus\fi}

        \pgfpathmoveto{\pgfpoint
            {\pgf@circ@res@left}
            {\ctikzvalof{tripoles/en amp/input height}\pgf@circ@res@down}}
        \pgfpathlineto{\pgfpoint
            {\ctikzvalof{tripoles/en amp/port width}\pgf@circ@res@left}
            {\ctikzvalof{tripoles/en amp/input height}\pgf@circ@res@down}}
        \pgftext[left, at=\pgfpoint{\ctikzvalof{tripoles/en amp/port width}\pgf@circ@res@left}{\ctikzvalof{tripoles/en amp/input height}\pgf@circ@res@down}]{\ctikzvalof{tripoles/en amp/font} \ifpgf@circuit@oa@iplusup\pgf@circ@ampli@minus\else\pgf@circ@ampli@plus\fi}

        \pgfpathmoveto{\pgfpoint{\pgf@circ@res@right}{0pt}}
        \pgfpathlineto{\pgfpoint{.7\pgf@circ@res@right}{0pt}}
        \pgfsetrectcap
        \pgfusepath{draw}


        \pgftext[top, y=-.5ex, at=\pgfpoint{0pt}{\pgf@circ@res@up}]{\hbox{\ctikzvalof{tripoles/en amp/font2}\ctikzvalof{tripoles/en amp/text}}}
        % \pgftext[top, y=-.5ex, at=\pgfpoint{0pt}{\pgf@circ@res@up}]{\ctikzvalof{tripoles/en amp/font2}$\mathstrut{\triangleright}\,\mathrm{A}$}
    }
}

% Fully differential output op amp
% Contributed by Kristofer M. Monisit
\pgfdeclareshape{fd op amp}
{
    \savedmacro{\ctikzclass}{\edef\ctikzclass{amplifiers}}
    \saveddimen{\scaledRlen}{\pgfmathsetlength{\pgf@x}{\ctikzvalof{\ctikzclass/scale}\pgf@circ@Rlen}}
    \anchor{center}{\pgfpointorigin}
    \savedanchor\northwest{%
        \pgfmathsetlength{\pgf@circ@scaled@Rlen}{\ctikzvalof{\ctikzclass/scale}\pgf@circ@Rlen}
        \pgf@y=\ctikzvalof{tripoles/fd op amp/height}\pgf@circ@scaled@Rlen
        \pgf@y=.5\pgf@y
        \pgf@x=-\ctikzvalof{tripoles/fd op amp/width}\pgf@circ@scaled@Rlen
        \pgf@x=.5\pgf@x
    }
    \savedanchor\outline{%
        \pgfmathsetlength{\pgf@circ@scaled@Rlen}{\ctikzvalof{\ctikzclass/scale}\pgf@circ@Rlen}
        \pgf@y=\ctikzvalof{tripoles/fd op amp/height}\pgf@circ@scaled@Rlen
        \pgf@y=.5\pgf@y
        \pgf@x=-\ctikzvalof{tripoles/fd op amp/width}\pgf@circ@scaled@Rlen
        \pgf@x=.5\pgf@x
    \ifpgf@circuit@oa@oplusup\else\pgf@y=-\pgf@y\fi
    }
    \anchor{south}{
        \northwest
        \pgf@y=-\pgf@y\pgf@x=0pt\relax
    }
    \anchor{north}{
        \northwest\pgf@x=0pt\relax
    }
    \savedanchor\left{%
        \pgfmathsetlength{\pgf@circ@scaled@Rlen}{\ctikzvalof{\ctikzclass/scale}\pgf@circ@Rlen}
        \pgf@x=-\ctikzvalof{tripoles/fd op amp/width}\pgf@circ@scaled@Rlen
        \pgf@x=.5\pgf@x
        \pgf@y=0pt
    }
    \savedanchor\inOneFixed{%
        \pgfmathsetlength{\pgf@circ@scaled@Rlen}{\ctikzvalof{\ctikzclass/scale}\pgf@circ@Rlen}
        \pgf@y=\ctikzvalof{tripoles/fd op amp/height}\pgf@circ@scaled@Rlen
        \pgf@y=.5\pgf@y
        \pgf@y=\ctikzvalof{tripoles/fd op amp/input height}\pgf@y
        \pgf@x=-\ctikzvalof{tripoles/fd op amp/width}\pgf@circ@scaled@Rlen
        \pgf@x=.5\pgf@x
    }
    \anchor{in up}{
        \inOneFixed
    }
    \anchor{in down}{
        \inOneFixed
        \pgf@y=-\pgf@y
    }
    \savedanchor\inOne{%
        \pgfmathsetlength{\pgf@circ@scaled@Rlen}{\ctikzvalof{\ctikzclass/scale}\pgf@circ@Rlen}
        \pgf@y=\ctikzvalof{tripoles/fd op amp/height}\pgf@circ@scaled@Rlen
        \pgf@y=.5\pgf@y
        \pgf@y=\ctikzvalof{tripoles/fd op amp/input height}\pgf@y
        \pgf@x=-\ctikzvalof{tripoles/fd op amp/width}\pgf@circ@scaled@Rlen
        \pgf@x=.5\pgf@x
    \ifpgf@circuit@oa@iplusup\pgf@y=-\pgf@y\fi
    }
    \anchor{-}{
        \inOne
    }
    \anchor{+}{
        \inOne
        \pgf@y=-\pgf@y
    }
    \savedanchor\up{%
        \pgfmathsetlength{\pgf@circ@scaled@Rlen}{\ctikzvalof{\ctikzclass/scale}\pgf@circ@Rlen}
        \pgf@y=\ctikzvalof{tripoles/fd op amp/height}\pgf@circ@scaled@Rlen
        \pgf@y=.5\pgf@y
        \pgf@x=-\ctikzvalof{tripoles/fd op amp/width}\pgf@circ@scaled@Rlen
        \pgf@x=.5\pgf@x
        \pgf@circ@res@up = \pgf@y
        \pgf@circ@res@right = -\pgf@x
        \pgf@circ@res@left = \pgf@x
        \pgfpointlineattime{
            \ctikzvalof{tripoles/fd op amp/up pos}}{
            \pgfpoint{
            \ctikzvalof{tripoles/fd op amp/port width}\pgf@circ@res@left}
        {\pgf@circ@res@up}}
        {\pgfpoint{\ctikzvalof{tripoles/fd op amp/port width}\pgf@circ@res@right}{0pt}}
    }
    \anchor{up}{
        \up
    }
    \anchor{down}{
        \up
        \pgf@y=-\pgf@y
    }
    \anchor{out up}{
        \northwest
        \pgf@y=\ctikzvalof{tripoles/fd op amp/output height}\pgf@y
        \pgf@x=-\ctikzvalof{tripoles/fd op amp/port width}\pgf@x
    }
    \anchor{out down}{
        \northwest
        \pgf@y=-\ctikzvalof{tripoles/fd op amp/output height}\pgf@y
        \pgf@x=-\ctikzvalof{tripoles/fd op amp/port width}\pgf@x
    }
    \anchor{out +}{
        \outline
        \pgf@y=\ctikzvalof{tripoles/fd op amp/output height}\pgf@y
        \pgf@x=-\ctikzvalof{tripoles/fd op amp/port width}\pgf@x
    }
    \anchor{out -}{
        \outline
        \pgf@y=-\ctikzvalof{tripoles/fd op amp/output height}\pgf@y
        \pgf@x=-\ctikzvalof{tripoles/fd op amp/port width}\pgf@x
    }
    % support for border anchors
    \savedanchor\leftedge{%
        \pgfmathsetlength{\pgf@circ@scaled@Rlen}{\ctikzvalof{\ctikzclass/scale}\pgf@circ@Rlen}
        \pgf@x=-\ctikzvalof{tripoles/fd op amp/width}\pgf@circ@scaled@Rlen
        \pgf@x=.5\pgf@x
        \pgf@x = \ctikzvalof{tripoles/fd op amp/port width}\pgf@x
        \pgf@y=0pt
    }
    \anchor{leftedge}{\leftedge}
    \anchor{rightedge}{\leftedge \pgf@x =-\pgf@x }
    \anchor{bin up}{
        \inOneFixed
        \pgf@ya=\pgf@y \leftedge \pgf@y=\pgf@ya
    }
    \anchor{bin down}{
        \inOneFixed
        \pgf@y=-\pgf@y
        \pgf@ya=\pgf@y \leftedge \pgf@y=\pgf@ya
    }
    \anchor{b-}{
        \inOne
        \pgf@ya=\pgf@y \leftedge \pgf@y=\pgf@ya
    }
    \anchor{b+}{
        \inOne
        \pgf@y=-\pgf@y
        \pgf@ya=\pgf@y \leftedge \pgf@y=\pgf@ya
    }
    \anchor{bout}{
        \leftedge
        \pgf@y=0pt
        \pgf@x=-\pgf@x
    }
    \anchor{bout +}{
        \outline
        \pgf@xa=\pgf@x\pgf@ya=\pgf@y
        \pgfpointlineattime{\ctikzvalof{tripoles/fd op amp/output height}}
            {\pgfpoint{\pgf@xa}{0pt}}
            {\pgfpoint{0pt}{\pgf@ya}}
    }
    \anchor{bout -}{
        \outline
        \pgf@xa=\pgf@x\pgf@ya=\pgf@y
        \pgfpointlineattime{\ctikzvalof{tripoles/fd op amp/output height}}
            {\pgfpoint{\pgf@xa}{0pt}}
            {\pgfpoint{0pt}{\pgf@ya}}
        \pgf@y=-\pgf@y
    }
    \anchor{bout up}{
        \northwest
        \pgf@xa=\pgf@x\pgf@ya=\pgf@y
        \pgfpointlineattime{\ctikzvalof{tripoles/fd op amp/output height}}
            {\pgfpoint{\pgf@xa}{0pt}}
            {\pgfpoint{0pt}{\pgf@ya}}
    }
    \anchor{bout down}{
        \northwest
        \pgf@xa=\pgf@x\pgf@ya=\pgf@y
        \pgfpointlineattime{\ctikzvalof{tripoles/fd op amp/output height}}
            {\pgfpoint{\pgf@xa}{0pt}}
            {\pgfpoint{0pt}{\pgf@ya}}
        \pgf@y=-\pgf@y
    }
    % end border anchors
    \anchor{west}{
        \left
    }
    \anchor{east}{
        \left
        \pgf@x=-\ctikzvalof{tripoles/fd op amp/port width}\pgf@x
    }
    \anchor{out}{% should not be used
        \left
        \pgf@x=-\ctikzvalof{tripoles/fd op amp/port width}\pgf@x
    }
    \anchor{south west}{ \northwest \pgf@y=-\pgf@y }
    \anchor{north east}{ \northwest \pgf@x=-\ctikzvalof{tripoles/fd op amp/port width}\pgf@x }
    \anchor{north west}{ \northwest }
    \anchor{south east}{ \northwest \pgf@x=-\ctikzvalof{tripoles/fd op amp/port width}\pgf@x \pgf@y=-\pgf@y }
    \anchor{text}{
        \ifpgf@circ@center@text
            \pgf@x=\dimexpr -.6\wd\pgfnodeparttextbox\relax
        \else
            \leftedge
            \pgfmathsetlength{\pgf@x}{\pgf@x + \ctikzvalof{left text distance}}
        \fi
        \pgf@y=\dimexpr.5\dp\pgfnodeparttextbox-.5\ht\pgfnodeparttextbox\relax
    }
    \backgroundpath{
        \pgfsetcolor{\ctikzvalof{color}}

        \northwest
        \pgf@circ@res@up = \pgf@y
        \pgf@circ@res@down = -\pgf@y
        \pgf@circ@res@right = -\pgf@x
        \pgf@circ@res@left = \pgf@x
        \pgf@circ@scaled@Rlen=\scaledRlen

        % Triangle
        % Includes output terminals in the traingular shape
        % to ensure that diagonal joins are properly displayed
        % we first draw the main triangle and the leads with normal thickness,
        % and the redraw the main triangle with the component shape
        \pgfscope
            % shift origin a bit to ease calculations
            \pgftransformxshift{\ctikzvalof{tripoles/fd op amp/port width}\pgf@circ@res@left}
            \pgf@circ@res@step=\pgf@circ@res@right
            \advance\pgf@circ@res@step by -\pgf@circ@res@left
            \pgf@circ@res@step=\ctikzvalof{tripoles/fd op amp/port width}\pgf@circ@res@step

            % Initial point (right vertex)
            \pgfpathmoveto{\pgfpoint{\pgf@circ@res@step}{0pt}}

            % Negative output terminal
            \pgfpathlineto{\pgfpointlineattime{\ctikzvalof{tripoles/fd op amp/output height}}{\pgfpoint{\pgf@circ@res@step}{0pt}}{\pgfpoint{0pt}{\pgf@circ@res@up}}}
            \pgfpathlineto{\pgfpoint{\pgf@circ@res@step}{\ctikzvalof{tripoles/fd op amp/output height}\pgf@circ@res@up}}
            \pgfpathlineto{\pgfpointlineattime{\ctikzvalof{tripoles/fd op amp/output height}}{\pgfpoint{\pgf@circ@res@step}{0pt}}{\pgfpoint{0pt}{\pgf@circ@res@up}}}

            % Top vertex
            \pgfpathlineto{\pgfpoint{0pt}{\pgf@circ@res@up}}

            % Bottom vertex
            \pgfpathlineto{\pgfpoint{0pt}{\pgf@circ@res@down}}

            % Positive output terminal
            \pgfpathlineto{\pgfpointlineattime{\ctikzvalof{tripoles/fd op amp/output height}}{\pgfpoint{\pgf@circ@res@step}{0pt}}{\pgfpoint{0pt}{\pgf@circ@res@down}}}
            \pgfpathlineto{\pgfpoint{\pgf@circ@res@step}{\ctikzvalof{tripoles/fd op amp/output height}\pgf@circ@res@down}}
            \pgfpathlineto{\pgfpointlineattime{\ctikzvalof{tripoles/fd op amp/output height}}{\pgfpoint{\pgf@circ@res@step}{0pt}}{\pgfpoint{0pt}{\pgf@circ@res@down}}}

            % Right vertex
            \pgfpathlineto{\pgfpoint{\pgf@circ@res@step}{0pt}}

            \pgfpathclose
            \pgfusepath{stroke}

            % ok, now we'll redraw the triangle with the class specific
            % thickness and optionally fill

            \pgfscope
                \pgf@circ@setlinewidth{quadpoles}{\pgflinewidth}

                % Initial point (right vertex)
                \pgfpathmoveto{\pgfpoint{\pgf@circ@res@step}{0pt}}
                % Top vertex
                \pgfpathlineto{\pgfpoint{0pt}{\pgf@circ@res@up}}
                % Bottom vertex
                \pgfpathlineto{\pgfpoint{0pt}{\pgf@circ@res@down}}

                \pgfpathclose
                \pgf@circ@draworfill
            \endpgfscope % thick and fill
        \endpgfscope % shift

        % Negative input terminal
        \pgfpathmoveto{\pgfpoint
            {\pgf@circ@res@left}
            {\ctikzvalof{tripoles/fd op amp/input height}\pgf@circ@res@up}}
        \pgfpathlineto{\pgfpoint
            {\ctikzvalof{tripoles/fd op amp/port width}\pgf@circ@res@left}
            {\ctikzvalof{tripoles/fd op amp/input height}\pgf@circ@res@up}}
        \pgftext[left, at=\pgfpoint{\ctikzvalof{tripoles/fd op amp/port width}\pgf@circ@res@left}{\ctikzvalof{tripoles/fd op amp/input height}\pgf@circ@res@up}]{\ctikzvalof{tripoles/fd op amp/font} \ifpgf@circuit@oa@iplusup\pgf@circ@ampli@plus\else\pgf@circ@ampli@minus\fi}

        % Positive input terminal
        \pgfpathmoveto{\pgfpoint
            {\pgf@circ@res@left}
            {\ctikzvalof{tripoles/fd op amp/input height}\pgf@circ@res@down}}
        \pgfpathlineto{\pgfpoint
            {\ctikzvalof{tripoles/fd op amp/port width}\pgf@circ@res@left}
            {\ctikzvalof{tripoles/fd op amp/input height}\pgf@circ@res@down}}
        \pgftext[left, at=\pgfpoint{\ctikzvalof{tripoles/fd op amp/port width}\pgf@circ@res@left}{\ctikzvalof{tripoles/fd op amp/input height}\pgf@circ@res@down}]{\ctikzvalof{tripoles/fd op amp/font} \ifpgf@circuit@oa@iplusup\pgf@circ@ampli@minus\else\pgf@circ@ampli@plus\fi}

        % Negative output terminal label
        \pgftext[right, bottom, x=3pt, y=1pt, at=\pgfpoint{0pt}{0.425\pgf@circ@res@down}]{\ctikzvalof{tripoles/fd op amp/font} \ifpgf@circuit@oa@oplusup\pgf@circ@ampli@minus\else\pgf@circ@ampli@plus\fi}

        % Positive output terminal label
        \pgftext[right, top, x=3pt, y=-1pt, at=\pgfpoint{0pt}{0.425\pgf@circ@res@up}]{\ctikzvalof{tripoles/fd op amp/font} \ifpgf@circuit@oa@oplusup\pgf@circ@ampli@plus\else\pgf@circ@ampli@minus\fi}

        \pgfsetrectcap
        \pgfusepath{draw}
    }
}


% Instrumentation amplifier with differential output
\pgfdeclareshape{fd inst amp}
{
    \savedmacro{\ctikzclass}{\edef\ctikzclass{amplifiers}}
    \saveddimen{\scaledRlen}{\pgfmathsetlength{\pgf@x}{\ctikzvalof{\ctikzclass/scale}\pgf@circ@Rlen}}
    % when tikz calls the anchor it wants the relative position in the lengths
    % \pgf@x  \pgf@y
    % \pgfpoint* functions set that variables
    % anchors are visible outside and run on use
    \anchor{center}{\pgfpointorigin}
    % savedanchors are internals and run on node creation (not use)
    % bounding-box top left
    \savedanchor\northwest{%
        \pgfmathsetlength{\pgf@circ@scaled@Rlen}{\ctikzvalof{\ctikzclass/scale}\pgf@circ@Rlen}
        \pgf@y=\ctikzvalof{tripoles/fd inst amp/height}\pgf@circ@scaled@Rlen
        \pgf@y=.5\pgf@y
        \pgf@x=-\ctikzvalof{tripoles/fd inst amp/width}\pgf@circ@scaled@Rlen
        \pgf@x=.5\pgf@x
    }
    \anchor{nw}{
        \northwest
    }
    \anchor{south}{
        \northwest
        \pgf@y=-\pgf@y\pgf@x=0pt\relax
    }
    \anchor{north}{
        \northwest\pgf@x=0pt\relax
    }
    \savedanchor\left{%
        \pgf@y=0pt
    }
    \anchor{leftedge}
    {\left
        \pgf@x = \ctikzvalof{tripoles/fd inst amp/port width}\pgf@x
    }
    \savedanchor\inOneFixed{%
        \pgfmathsetlength{\pgf@circ@scaled@Rlen}{\ctikzvalof{\ctikzclass/scale}\pgf@circ@Rlen}
          \pgf@y=\ctikzvalof{tripoles/fd inst amp/height}\pgf@circ@scaled@Rlen
          \pgf@y=.5\pgf@y
          \pgf@y=\ctikzvalof{tripoles/fd inst amp/input height}\pgf@y
          \pgf@x=-\ctikzvalof{tripoles/fd inst amp/width}\pgf@circ@scaled@Rlen
          \pgf@x=.5\pgf@x
    }
    \anchor{in up}{
        \inOneFixed
    }
    \anchor{in down}{
        \inOneFixed
        \pgf@y=-\pgf@y
    }
    \savedanchor\inOne{%
        \pgfmathsetlength{\pgf@circ@scaled@Rlen}{\ctikzvalof{\ctikzclass/scale}\pgf@circ@Rlen}
        \pgf@y=\ctikzvalof{tripoles/fd inst amp/height}\pgf@circ@scaled@Rlen
        \pgf@y=.5\pgf@y
        \pgf@y=\ctikzvalof{tripoles/fd inst amp/input height}\pgf@y
        \pgf@x=-\ctikzvalof{tripoles/fd inst amp/width}\pgf@circ@scaled@Rlen
        \pgf@x=.5\pgf@x
        \ifpgf@circuit@oa@iplusup\pgf@y=-\pgf@y\fi
    }
    \anchor{-}{
        \inOne
    }
    \anchor{+}{
        \inOne
        \pgf@y=-\pgf@y
    }
    \savedanchor\up{%
        \pgfmathsetlength{\pgf@circ@scaled@Rlen}{\ctikzvalof{\ctikzclass/scale}\pgf@circ@Rlen}
        \pgf@y=\ctikzvalof{tripoles/fd inst amp/height}\pgf@circ@scaled@Rlen
        \pgf@y=.5\pgf@y
        \pgf@x=-\ctikzvalof{tripoles/fd inst amp/width}\pgf@circ@scaled@Rlen
        \pgf@x=.5\pgf@x
        \pgf@circ@res@up = \pgf@y
        \pgf@circ@res@right = -\pgf@x
        \pgf@circ@res@left = \pgf@x
        \pgfpointlineattime{
            \ctikzvalof{tripoles/fd inst amp/up pos}}{
            \pgfpoint{
            \ctikzvalof{tripoles/fd inst amp/port width}\pgf@circ@res@left}
        {\pgf@circ@res@up}}
        {\pgfpoint{\ctikzvalof{tripoles/fd inst amp/port width}\pgf@circ@res@right}{\ctikzvalof{tripoles/fd inst amp/height 2}\pgf@circ@res@up}}
    }
    \anchor{up}{
        \up
    }
    \anchor{down}{
        \up
        \pgf@y=-\pgf@y
    }
    % reference voltage input anchors.
    \savedanchor\refv{%
        \pgfmathsetlength{\pgf@circ@scaled@Rlen}{\ctikzvalof{\ctikzclass/scale}\pgf@circ@Rlen}
        \pgf@y=\ctikzvalof{tripoles/fd inst amp/height}\pgf@circ@scaled@Rlen
        \pgf@y=.5\pgf@y
        \pgf@x=-\ctikzvalof{tripoles/fd inst amp/width}\pgf@circ@scaled@Rlen
        \pgf@x=.5\pgf@x
        \pgf@circ@res@up = \pgf@y
        \pgf@circ@res@right = -\pgf@x
        \pgf@circ@res@left = \pgf@x
        \pgfpointlineattime{
            \ctikzvalof{tripoles/fd inst amp/refv pos}}{
            \pgfpoint{
            \ctikzvalof{tripoles/fd inst amp/port width}\pgf@circ@res@left}
        {\pgf@circ@res@up}}
        {\pgfpoint{\ctikzvalof{tripoles/fd inst amp/port width}\pgf@circ@res@right}{\ctikzvalof{tripoles/fd inst amp/height 2}\pgf@circ@res@up}}
    }
    % we need both because they are normally drawn under the amp, and if you
    % mirror it vertically you need them
    \anchor{refv up}{
        \refv
    }
    \anchor{refv down}{
        \refv
        \pgf@y=-\pgf@y
    }
    \savedanchor\outport{
        \pgfmathsetlength{\pgf@circ@scaled@Rlen}{\ctikzvalof{\ctikzclass/scale}\pgf@circ@Rlen}
        \pgf@y=\ctikzvalof{tripoles/fd inst amp/height}\pgf@circ@scaled@Rlen
        \pgf@y=.5\pgf@y
        \pgf@y=\ctikzvalof{tripoles/fd inst amp/input height}\pgf@y
        \pgf@x=-\ctikzvalof{tripoles/fd inst amp/width}\pgf@circ@scaled@Rlen
        \pgf@x=-.5\pgf@x
        \ifpgf@circuit@oa@oplusup\else\pgf@y=-\pgf@y\fi
    }
    \anchor{out}{
        \outport
        \pgf@y=0pt
    }
    \anchor{out +}{
        \outport
    }
    \anchor{out -}{
        \outport
        \pgf@y=-\pgf@y
    }
    \savedanchor\outportfixed{
        \pgfmathsetlength{\pgf@circ@scaled@Rlen}{\ctikzvalof{\ctikzclass/scale}\pgf@circ@Rlen}
        \pgf@y=\ctikzvalof{tripoles/fd inst amp/height}\pgf@circ@scaled@Rlen
        \pgf@y=.5\pgf@y
        \pgf@y=\ctikzvalof{tripoles/fd inst amp/input height}\pgf@y
        \pgf@x=-\ctikzvalof{tripoles/fd inst amp/width}\pgf@circ@scaled@Rlen
        \pgf@x=-.5\pgf@x
    }
    \anchor{out up}{
        \outportfixed
    }
    \anchor{out down}{
        \outportfixed
        \pgf@y=-\pgf@y
    }
    % support for border anchors
    \savedanchor\leftedge{%
        \pgfmathsetlength{\pgf@circ@scaled@Rlen}{\ctikzvalof{\ctikzclass/scale}\pgf@circ@Rlen}
        \pgf@x=-\ctikzvalof{tripoles/fd inst amp/width}\pgf@circ@scaled@Rlen
        \pgf@x=.5\pgf@x
        \pgf@x = \ctikzvalof{tripoles/fd inst amp/port width}\pgf@x
        \pgf@y=0pt
    }
    \anchor{leftedge}{\leftedge}
    \anchor{rightedge}{\leftedge \pgf@x =-\pgf@x }
    \anchor{bin up}{
        \inOneFixed
        \pgf@ya=\pgf@y \leftedge \pgf@y=\pgf@ya
    }
    \anchor{bin down}{
        \inOneFixed
        \pgf@y=-\pgf@y
        \pgf@ya=\pgf@y \leftedge \pgf@y=\pgf@ya
    }
    \anchor{b-}{
        \inOne
        \pgf@ya=\pgf@y \leftedge \pgf@y=\pgf@ya
    }
    \anchor{b+}{
        \inOne
        \pgf@y=-\pgf@y
        \pgf@ya=\pgf@y \leftedge \pgf@y=\pgf@ya
    }
    \anchor{bout}{
        \leftedge
        \pgf@x=-\pgf@x \pgf@y=0pt
    }
    \anchor{bout +}{
        \outport
        \pgf@ya=\pgf@y \leftedge \pgf@x=-\pgf@x \pgf@y=\pgf@ya
    }
    \anchor{bout -}{
        \outport
        \pgf@y=-\pgf@y
        \pgf@ya=\pgf@y \leftedge \pgf@x=-\pgf@x \pgf@y=\pgf@ya
    }
    \anchor{bout up}{
        \outportfixed
        \pgf@ya=\pgf@y \leftedge \pgf@x=-\pgf@x \pgf@y=\pgf@ya
    }
    \anchor{bout down}{
        \outportfixed
        \pgf@y=-\pgf@y
        \pgf@ya=\pgf@y \leftedge \pgf@x=-\pgf@x \pgf@y=\pgf@ya
    }
    % end border anchors
    %
    \anchor{west}{
        \left
    }
    \anchor{east}{
        \left
        \pgf@x=-\pgf@x
    }
    \anchor{south west}{ \northwest \pgf@y=-\pgf@y }
    \anchor{north east}{ \northwest \pgf@x=-\pgf@x }
    \anchor{north west}{ \northwest }
    \anchor{south east}{ \northwest \pgf@x=-\pgf@x \pgf@y=-\pgf@y }
    \anchor{text}{
        \ifpgf@circ@center@text
            \pgf@x=\dimexpr -.6\wd\pgfnodeparttextbox\relax
        \else
            \leftedge
            \pgfmathsetlength{\pgf@x}{\pgf@x + \ctikzvalof{left text distance}}
        \fi
        \pgf@y=\dimexpr.5\dp\pgfnodeparttextbox-.5\ht\pgfnodeparttextbox\relax
    }

    % let's start drawing the component
    \backgroundpath{
        \pgfsetcolor{\ctikzvalof{color}}
        %
        \northwest
        \pgf@circ@res@up = \pgf@y
        \pgf@circ@res@down = -\pgf@y
        \pgf@circ@res@right = -\pgf@x
        \pgf@circ@res@left = \pgf@x
        \pgf@circ@scaled@Rlen=\scaledRlen
        % main component, normally in thicker lines
        \pgfscope
            \newdimen\pgf@circ@res@right@double
						\pgf@circ@res@right@double=2\pgf@circ@res@right

            \pgf@circ@setlinewidth{tripoles}{\pgflinewidth}
            \pgftransformxshift{\ctikzvalof{tripoles/fd inst amp/port width}\pgf@circ@res@left}
            \pgf@circ@res@step=\pgf@circ@res@right
            \advance\pgf@circ@res@step by -\pgf@circ@res@left
            \pgf@circ@res@step=\ctikzvalof{tripoles/fd inst amp/port width}\pgf@circ@res@step
            %first point (near output)
            \pgfpathmoveto{\pgfpoint{\ctikzvalof{tripoles/fd inst amp/port width}\pgf@circ@res@right@double}{0}}
            %from the exit to the top (short side)... (note that the .6 must be copied in \up and \refv anchors
            \pgfpathlineto{\pgfpoint{\ctikzvalof{tripoles/fd inst amp/port width}\pgf@circ@res@right@double}{\ctikzvalof{tripoles/fd inst amp/height 2}\pgf@circ@res@up}}
            % and then to the input "front up", "down", to the output short side "down"
            \pgfpathlineto{\pgfpoint{0pt}{\pgf@circ@res@up}}
            \pgfpathlineto{\pgfpoint{0pt}{\pgf@circ@res@down}}
            \pgfpathlineto{\pgfpoint{\ctikzvalof{tripoles/fd inst amp/port width}\pgf@circ@res@right@double}{\ctikzvalof{tripoles/fd inst amp/height 2}\pgf@circ@res@down}}
            % ...and close
            \pgfpathclose
            \pgf@circ@draworfill
        \endpgfscope

        % input terminal up
        \pgfpathmoveto{\pgfpoint
            {\pgf@circ@res@left}
        {\ctikzvalof{tripoles/fd inst amp/input height}\pgf@circ@res@up}}
        %
        \pgfpathlineto{\pgfpoint
            {\ctikzvalof{tripoles/fd inst amp/port width}\pgf@circ@res@left}
        {\ctikzvalof{tripoles/fd inst amp/input height}\pgf@circ@res@up}}
        %
        \pgftext[left, at=\pgfpoint{\ctikzvalof{tripoles/fd inst amp/port width}\pgf@circ@res@left}{\ctikzvalof{tripoles/fd inst amp/input height}\pgf@circ@res@up}]{\ctikzvalof{tripoles/fd inst amp/font} \ifpgf@circuit@oa@iplusup\pgf@circ@ampli@plus\else\pgf@circ@ampli@minus\fi}

        % input terminal down
        \pgfpathmoveto{\pgfpoint
            {\pgf@circ@res@left}
        {\ctikzvalof{tripoles/fd inst amp/input height}\pgf@circ@res@down}}
        %
        \pgfpathlineto{\pgfpoint
            {\ctikzvalof{tripoles/fd inst amp/port width}\pgf@circ@res@left}
        {\ctikzvalof{tripoles/fd inst amp/input height}\pgf@circ@res@down}}
        \pgftext[left, at=\pgfpoint{\ctikzvalof{tripoles/fd inst amp/port width}\pgf@circ@res@left}{\ctikzvalof{tripoles/fd inst amp/input height}\pgf@circ@res@down}]{\ctikzvalof{tripoles/fd inst amp/font} \ifpgf@circuit@oa@iplusup\pgf@circ@ampli@minus\else\pgf@circ@ampli@plus\fi}

        % output leads down and up
        \pgfpathmoveto{\pgfpoint{\pgf@circ@res@right}{\ctikzvalof{tripoles/fd inst amp/output height}\pgf@circ@res@down}}
        \pgfpathlineto{\pgfpoint{\ctikzvalof{tripoles/fd inst amp/port width}\pgf@circ@res@right}{\ctikzvalof{tripoles/fd inst amp/output height}\pgf@circ@res@down}} %
        \pgftext[right, at=\pgfpoint{\ctikzvalof{tripoles/fd inst amp/port width}\pgf@circ@res@right}{\ctikzvalof{tripoles/fd inst amp/output height}\pgf@circ@res@down}]{\ctikzvalof{tripoles/fd inst amp/font}\ifpgf@circuit@oa@oplusup$-\;$\else$+\;$\fi}

        \pgfpathmoveto{\pgfpoint{\pgf@circ@res@right}{\ctikzvalof{tripoles/fd inst amp/output height}\pgf@circ@res@up}}
        \pgfpathlineto{\pgfpoint{\ctikzvalof{tripoles/fd inst amp/port width}\pgf@circ@res@right}{\ctikzvalof{tripoles/fd inst amp/output height}\pgf@circ@res@up}} %
        \pgftext[right, at=\pgfpoint{\ctikzvalof{tripoles/fd inst amp/port width}\pgf@circ@res@right}{\ctikzvalof{tripoles/fd inst amp/output height}\pgf@circ@res@up}]{\ctikzvalof{tripoles/fd inst amp/font}\ifpgf@circuit@oa@oplusup$+\;$\else$-\;$\fi}
        %
        \pgfsetrectcap
        \pgfusepath{draw}
    }
}

% Transconductance amplifier (Transkonduktanzverstärker)
\pgfdeclareshape{gm amp}
{
    \savedmacro{\ctikzclass}{\edef\ctikzclass{amplifiers}}
    \saveddimen{\scaledRlen}{\pgfmathsetlength{\pgf@x}{\ctikzvalof{\ctikzclass/scale}\pgf@circ@Rlen}}
    \anchor{center}{\pgfpointorigin}
    \savedanchor\northwest{%
        \pgfmathsetlength{\pgf@circ@scaled@Rlen}{\ctikzvalof{\ctikzclass/scale}\pgf@circ@Rlen}
        \pgf@y=\ctikzvalof{tripoles/gm amp/height}\pgf@circ@scaled@Rlen
        \pgf@y=.5\pgf@y
        \pgf@x=-\ctikzvalof{tripoles/gm amp/width}\pgf@circ@scaled@Rlen
        \pgf@x=.5\pgf@x
    }
    \anchor{south}{
        \northwest
        \pgf@y=-\pgf@y
    }
    \anchor{north}{
        \northwest
    }
    \savedanchor\left{%
        \pgf@y=0pt
    }
    \anchor{leftedge}
    {\left
        \pgf@x = \ctikzvalof{tripoles/op amp/port width}\pgf@x
    }
    \savedanchor\inOneFixed{%
        \pgfmathsetlength{\pgf@circ@scaled@Rlen}{\ctikzvalof{\ctikzclass/scale}\pgf@circ@Rlen}
        \pgf@y=\ctikzvalof{tripoles/gm amp/height}\pgf@circ@scaled@Rlen
        \pgf@y=.5\pgf@y
        \pgf@y=\ctikzvalof{tripoles/gm amp/input height}\pgf@y
        \pgf@x=-\ctikzvalof{tripoles/gm amp/width}\pgf@circ@scaled@Rlen
        \pgf@x=.5\pgf@x
    }
    \anchor{in up}{
        \inOneFixed
    }
    \anchor{in down}{
        \inOneFixed
        \pgf@y=-\pgf@y
    }
    \savedanchor\inOne{%
        \pgfmathsetlength{\pgf@circ@scaled@Rlen}{\ctikzvalof{\ctikzclass/scale}\pgf@circ@Rlen}
        \pgf@y=\ctikzvalof{tripoles/gm amp/height}\pgf@circ@scaled@Rlen
        \pgf@y=.5\pgf@y
        \pgf@y=\ctikzvalof{tripoles/gm amp/input height}\pgf@y
        \pgf@x=-\ctikzvalof{tripoles/gm amp/width}\pgf@circ@scaled@Rlen
        \pgf@x=.5\pgf@x
        \ifpgf@circuit@oa@iplusup\pgf@y=-\pgf@y\fi
    }
    \anchor{-}{
        \inOne
    }
    \anchor{+}{
        \inOne
        \pgf@y=-\pgf@y
    }
        \savedanchor\up{%
        \pgfmathsetlength{\pgf@circ@scaled@Rlen}{\ctikzvalof{\ctikzclass/scale}\pgf@circ@Rlen}
        \pgf@y=\ctikzvalof{tripoles/gm amp/height}\pgf@circ@scaled@Rlen
        \pgf@y=.5\pgf@y
        \pgf@x=-\ctikzvalof{tripoles/gm amp/width}\pgf@circ@scaled@Rlen
        \pgf@x=.5\pgf@x
        \pgf@circ@res@up = \pgf@y
        \pgf@circ@res@right = -\pgf@x
        \pgf@circ@res@left = \pgf@x
        \pgfpointlineattime{
            \ctikzvalof{tripoles/gm amp/up pos}}{
            \pgfpoint{
            \ctikzvalof{tripoles/gm amp/port width}\pgf@circ@res@left}
        {\pgf@circ@res@up}}
        {\pgfpoint{\ctikzvalof{tripoles/gm amp/port width}\pgf@circ@res@right}{\ctikzvalof{tripoles/gm amp/height 2}\pgf@circ@res@up}}
    }
    \anchor{up}{
        \up
    }
    \anchor{down}{
        \up
        \pgf@y=-\pgf@y
    }
    \anchor{out}{
        \northwest
        \pgf@y=0pt
        \pgf@x=-\pgf@x
    }
    % support for border anchors
    \savedanchor\leftedge{%
        \pgfmathsetlength{\pgf@circ@scaled@Rlen}{\ctikzvalof{\ctikzclass/scale}\pgf@circ@Rlen}
        \pgf@x=-\ctikzvalof{tripoles/gm amp/width}\pgf@circ@scaled@Rlen
        \pgf@x=.5\pgf@x
        \pgf@x = \ctikzvalof{tripoles/gm amp/port width}\pgf@x
        \pgf@y=0pt
    }
    \anchor{leftedge}{\leftedge}
    \anchor{rightedge}{\leftedge \pgf@x =-\pgf@x }
    \anchor{bin up}{
        \inOneFixed
        \pgf@ya=\pgf@y \leftedge \pgf@y=\pgf@ya
    }
    \anchor{bin down}{
        \inOneFixed
        \pgf@y=-\pgf@y
        \pgf@ya=\pgf@y \leftedge \pgf@y=\pgf@ya
    }
    \anchor{b-}{
        \inOne
        \pgf@ya=\pgf@y \leftedge \pgf@y=\pgf@ya
    }
    \anchor{b+}{
        \inOne
        \pgf@y=-\pgf@y
        \pgf@ya=\pgf@y \leftedge \pgf@y=\pgf@ya
    }
    \anchor{bout}{
        \leftedge
        \pgf@y=0pt
        \pgf@x=-\pgf@x
    }
    % end border anchors
    \anchor{west}{
        \left
    }
    \anchor{east}{
        \left
        \pgf@x=-\pgf@x
    }
    \anchor{south west}{ \northwest \pgf@y=-\pgf@y }
    \anchor{north east}{ \northwest \pgf@x=-\pgf@x }
    \anchor{north west}{ \northwest }
    \anchor{south east}{ \northwest \pgf@x=-\pgf@x \pgf@y=-\pgf@y }

    \anchor{text}{
        \ifpgf@circ@center@text
            \pgf@x=\dimexpr -.6\wd\pgfnodeparttextbox\relax
        \else
            \leftedge
            \pgfmathsetlength{\pgf@x}{\pgf@x + \ctikzvalof{left text distance}}
        \fi
        \pgf@y=\dimexpr.5\dp\pgfnodeparttextbox-.5\ht\pgfnodeparttextbox\relax
    }
    \backgroundpath{
        \pgfsetcolor{\ctikzvalof{color}}

        \northwest
        \pgf@circ@res@up = \pgf@y
        \pgf@circ@res@down = -\pgf@y
        \pgf@circ@res@right = -\pgf@x
        \pgf@circ@res@left = \pgf@x
        \pgf@circ@scaled@Rlen=\scaledRlen

        \pgfscope
            \newdimen\pgf@circ@res@right@double
						\pgf@circ@res@right@double=2\pgf@circ@res@right

            \pgf@circ@setlinewidth{tripoles}{\pgflinewidth}
            \pgftransformxshift{\ctikzvalof{tripoles/gm amp/port width}\pgf@circ@res@left}
            \pgf@circ@res@step=\pgf@circ@res@right
            \advance\pgf@circ@res@step by -\pgf@circ@res@left
            \pgf@circ@res@step=\ctikzvalof{tripoles/gm amp/port width}\pgf@circ@res@step
            %Umrandung:
            \pgfpathmoveto{\pgfpoint{\ctikzvalof{tripoles/gm amp/port width}\pgf@circ@res@right@double}{0}} %geändert startpunkt neu am ausgangsstrich
            \pgfpathlineto{\pgfpoint{\ctikzvalof{tripoles/gm amp/port width}\pgf@circ@res@right@double}{\ctikzvalof{tripoles/gm amp/height 2}\pgf@circ@res@up}}%vom Ausgang nach oben
            \pgfpathlineto{\pgfpoint{0pt}{\pgf@circ@res@up}} %neu ecke links oben nach rechts oben
            \pgfpathlineto{\pgfpoint{0pt}{\pgf@circ@res@down}} %bei deneigängen runter
            \pgfpathlineto{\pgfpoint{\ctikzvalof{tripoles/gm amp/port width}\pgf@circ@res@right@double}{\ctikzvalof{tripoles/gm amp/height 2}\pgf@circ@res@down}}%ecke links unten nach rechts unten
            \pgfpathclose
            \pgf@circ@draworfill
        \endpgfscope

        \pgfpathmoveto{\pgfpoint
            {\pgf@circ@res@left}
            {\ctikzvalof{tripoles/gm amp/input height}\pgf@circ@res@up}}
        \pgfpathlineto{\pgfpoint
            {\ctikzvalof{tripoles/gm amp/port width}\pgf@circ@res@left}
            {\ctikzvalof{tripoles/gm amp/input height}\pgf@circ@res@up}}
        \pgftext[left, at=\pgfpoint{\ctikzvalof{tripoles/gm amp/port width}\pgf@circ@res@left}{\ctikzvalof{tripoles/gm amp/input height}\pgf@circ@res@up}]{\ctikzvalof{tripoles/gm amp/font} \ifpgf@circuit@oa@iplusup\pgf@circ@ampli@plus\else\pgf@circ@ampli@minus\fi}


        \pgfpathmoveto{\pgfpoint
            {\pgf@circ@res@left}
            {\ctikzvalof{tripoles/gm amp/input height}\pgf@circ@res@down}}
        \pgfpathlineto{\pgfpoint
            {\ctikzvalof{tripoles/gm amp/port width}\pgf@circ@res@left}
            {\ctikzvalof{tripoles/gm amp/input height}\pgf@circ@res@down}}
        \pgftext[left, at=\pgfpoint{\ctikzvalof{tripoles/gm amp/port width}\pgf@circ@res@left}{\ctikzvalof{tripoles/gm amp/input height}\pgf@circ@res@down}]{\ctikzvalof{tripoles/gm amp/font} \ifpgf@circuit@oa@iplusup\pgf@circ@ampli@minus\else\pgf@circ@ampli@plus\fi}

        \pgfpathmoveto{\pgfpoint{\pgf@circ@res@right}{0pt}}
        \pgfpathlineto{\pgfpoint{\ctikzvalof{tripoles/gm amp/port width}\pgf@circ@res@right}{0pt}} %
        \pgfsetrectcap
        \pgfusepath{draw}

    }
}

% Instrumentation amplifier
\pgfdeclareshape{inst amp}
{
    \savedmacro{\ctikzclass}{\edef\ctikzclass{amplifiers}}
    \saveddimen{\scaledRlen}{\pgfmathsetlength{\pgf@x}{\ctikzvalof{\ctikzclass/scale}\pgf@circ@Rlen}}
    % when tikz calls the anchor it wants the relative position in the lengths
    % \pgf@x  \pgf@y
    % \pgfpoint* functions set that variables
    % anchors are visible outside and run on use
    \anchor{center}{\pgfpointorigin}
    % savedanchors are internals and run on node creation (not use)
    % bounding-box top left
    \savedanchor\northwest{%
        \pgfmathsetlength{\pgf@circ@scaled@Rlen}{\ctikzvalof{\ctikzclass/scale}\pgf@circ@Rlen}
        \pgf@y=\ctikzvalof{tripoles/inst amp/height}\pgf@circ@scaled@Rlen
        \pgf@y=.5\pgf@y
        \pgf@x=-\ctikzvalof{tripoles/inst amp/width}\pgf@circ@scaled@Rlen
        \pgf@x=.5\pgf@x
    }
    \anchor{nw}{
        \northwest
    }
    \anchor{south}{
        \northwest
        \pgf@y=-\pgf@y
    }
    \anchor{north}{
        \northwest
    }
    \savedanchor\left{%
        \pgf@y=0pt
    }
    \anchor{leftedge}
    {\left
        \pgf@x = \ctikzvalof{tripoles/op amp/port width}\pgf@x
    }
    \savedanchor\inOneFixed{%
        \pgfmathsetlength{\pgf@circ@scaled@Rlen}{\ctikzvalof{\ctikzclass/scale}\pgf@circ@Rlen}
          \pgf@y=\ctikzvalof{tripoles/inst amp/height}\pgf@circ@scaled@Rlen
          \pgf@y=.5\pgf@y
          \pgf@y=\ctikzvalof{tripoles/inst amp/input height}\pgf@y
          \pgf@x=-\ctikzvalof{tripoles/inst amp/width}\pgf@circ@scaled@Rlen
          \pgf@x=.5\pgf@x
    }
    \anchor{in up}{
        \inOneFixed
    }
    \anchor{in down}{
        \inOneFixed
        \pgf@y=-\pgf@y
    }
    \savedanchor\inOne{%
        \pgfmathsetlength{\pgf@circ@scaled@Rlen}{\ctikzvalof{\ctikzclass/scale}\pgf@circ@Rlen}
        \pgf@y=\ctikzvalof{tripoles/inst amp/height}\pgf@circ@scaled@Rlen
        \pgf@y=.5\pgf@y
        \pgf@y=\ctikzvalof{tripoles/inst amp/input height}\pgf@y
        \pgf@x=-\ctikzvalof{tripoles/inst amp/width}\pgf@circ@scaled@Rlen
        \pgf@x=.5\pgf@x
        \ifpgf@circuit@oa@iplusup\pgf@y=-\pgf@y\fi
    }
    \anchor{-}{
        \inOne
    }
    \anchor{+}{
        \inOne
        \pgf@y=-\pgf@y
    }
    \savedanchor\up{%
        \pgfmathsetlength{\pgf@circ@scaled@Rlen}{\ctikzvalof{\ctikzclass/scale}\pgf@circ@Rlen}
        \pgf@y=\ctikzvalof{tripoles/inst amp/height}\pgf@circ@scaled@Rlen
        \pgf@y=.5\pgf@y
        \pgf@x=-\ctikzvalof{tripoles/inst amp/width}\pgf@circ@scaled@Rlen
        \pgf@x=.5\pgf@x
        \pgf@circ@res@up = \pgf@y
        \pgf@circ@res@right = -\pgf@x
        \pgf@circ@res@left = \pgf@x
        \pgfpointlineattime{
            \ctikzvalof{tripoles/inst amp/up pos}}{
            \pgfpoint{
            \ctikzvalof{tripoles/inst amp/port width}\pgf@circ@res@left}
        {\pgf@circ@res@up}}
        {\pgfpoint{\ctikzvalof{tripoles/inst amp/port width}\pgf@circ@res@right}{\ctikzvalof{tripoles/inst amp/height 2}\pgf@circ@res@up}}
    }
    \anchor{up}{
        \up
    }
    \anchor{down}{
        \up
        \pgf@y=-\pgf@y
    }
    % reference voltage input anchors.
    \savedanchor\refv{%
        \pgfmathsetlength{\pgf@circ@scaled@Rlen}{\ctikzvalof{\ctikzclass/scale}\pgf@circ@Rlen}
        \pgf@y=\ctikzvalof{tripoles/inst amp/height}\pgf@circ@scaled@Rlen
        \pgf@y=.5\pgf@y
        \pgf@x=-\ctikzvalof{tripoles/inst amp/width}\pgf@circ@scaled@Rlen
        \pgf@x=.5\pgf@x
        \pgf@circ@res@up = \pgf@y
        \pgf@circ@res@right = -\pgf@x
        \pgf@circ@res@left = \pgf@x
        \pgfpointlineattime{
            \ctikzvalof{tripoles/inst amp/refv pos}}{
            \pgfpoint{
            \ctikzvalof{tripoles/inst amp/port width}\pgf@circ@res@left}
        {\pgf@circ@res@up}}
        {\pgfpoint{\ctikzvalof{tripoles/inst amp/port width}\pgf@circ@res@right}{\ctikzvalof{tripoles/inst amp/height 2}\pgf@circ@res@up}}
    }
    % we need both because they are normally drawn under the amp, and if you
    % mirror it vertically you need them
    \anchor{refv up}{
        \refv
    }
    \anchor{refv down}{
        \refv
        \pgf@y=-\pgf@y
    }
    \anchor{out}{
        \northwest
        \pgf@y=0pt
        \pgf@x=-\pgf@x
    }
    % support for border anchors
    \savedanchor\leftedge{%
        \pgfmathsetlength{\pgf@circ@scaled@Rlen}{\ctikzvalof{\ctikzclass/scale}\pgf@circ@Rlen}
        \pgf@x=-\ctikzvalof{tripoles/inst amp/width}\pgf@circ@scaled@Rlen
        \pgf@x=.5\pgf@x
        \pgf@x = \ctikzvalof{tripoles/inst amp/port width}\pgf@x
        \pgf@y=0pt
    }
    \anchor{leftedge}{\leftedge}
    \anchor{rightedge}{\leftedge \pgf@x =-\pgf@x }
    \anchor{bin up}{
        \inOneFixed
        \pgf@ya=\pgf@y \leftedge \pgf@y=\pgf@ya
    }
    \anchor{bin down}{
        \inOneFixed
        \pgf@y=-\pgf@y
        \pgf@ya=\pgf@y \leftedge \pgf@y=\pgf@ya
    }
    \anchor{b-}{
        \inOne
        \pgf@ya=\pgf@y \leftedge \pgf@y=\pgf@ya
    }
    \anchor{b+}{
        \inOne
        \pgf@y=-\pgf@y
        \pgf@ya=\pgf@y \leftedge \pgf@y=\pgf@ya
    }
    \anchor{bout}{
        \leftedge
        \pgf@y=0pt
        \pgf@x=-\pgf@x
    }
    % end border anchors
    \anchor{west}{
        \left
    }
    \anchor{east}{
        \left
        \pgf@x=-\pgf@x
    }
    \anchor{south west}{ \northwest \pgf@y=-\pgf@y }
    \anchor{north east}{ \northwest \pgf@x=-\pgf@x }
    \anchor{north west}{ \northwest }
    \anchor{south east}{ \northwest \pgf@x=-\pgf@x \pgf@y=-\pgf@y }

    \anchor{text}{
        \ifpgf@circ@center@text
            \pgf@x=\dimexpr -.6\wd\pgfnodeparttextbox\relax
        \else
            \leftedge
            \pgfmathsetlength{\pgf@x}{\pgf@x + \ctikzvalof{left text distance}}
        \fi
        \pgf@y=\dimexpr.5\dp\pgfnodeparttextbox-.5\ht\pgfnodeparttextbox\relax
    }

    % let's start drawing the component
    \backgroundpath{
        \pgfsetcolor{\ctikzvalof{color}}
        %
        \northwest
        \pgf@circ@res@up = \pgf@y
        \pgf@circ@res@down = -\pgf@y
        \pgf@circ@res@right = -\pgf@x
        \pgf@circ@res@left = \pgf@x
        \pgf@circ@scaled@Rlen=\scaledRlen

        % main component, normally in thicker lines
        \pgfscope
            \newdimen\pgf@circ@res@right@double
						\pgf@circ@res@right@double=2\pgf@circ@res@right

            \pgf@circ@setlinewidth{tripoles}{\pgflinewidth}
            \pgftransformxshift{\ctikzvalof{tripoles/inst amp/port width}\pgf@circ@res@left}
            \pgf@circ@res@step=\pgf@circ@res@right
            \advance\pgf@circ@res@step by -\pgf@circ@res@left
            \pgf@circ@res@step=\ctikzvalof{tripoles/inst amp/port width}\pgf@circ@res@step
            %first point (near output)
            \pgfpathmoveto{\pgfpoint{\ctikzvalof{tripoles/inst amp/port width}\pgf@circ@res@right@double}{0}}
            %from the exit to the top (short side)... (note that the .6 must be copied in \up and \refv anchors
            \pgfpathlineto{\pgfpoint{\ctikzvalof{tripoles/inst amp/port width}\pgf@circ@res@right@double}{\ctikzvalof{tripoles/inst amp/height 2}\pgf@circ@res@up}}
            % and then to the input "front up", "down", to the output short side "down"
            \pgfpathlineto{\pgfpoint{0pt}{\pgf@circ@res@up}}
            \pgfpathlineto{\pgfpoint{0pt}{\pgf@circ@res@down}}
            \pgfpathlineto{\pgfpoint{\ctikzvalof{tripoles/inst amp/port width}\pgf@circ@res@right@double}{\ctikzvalof{tripoles/inst amp/height 2}\pgf@circ@res@down}}
            % ...and close
            \pgfpathclose
            \pgf@circ@draworfill
        \endpgfscope

        % Negative input terminal
        \pgfpathmoveto{\pgfpoint
            {\pgf@circ@res@left}
        {\ctikzvalof{tripoles/inst amp/input height}\pgf@circ@res@up}}
        %
        \pgfpathlineto{\pgfpoint
            {\ctikzvalof{tripoles/inst amp/port width}\pgf@circ@res@left}
        {\ctikzvalof{tripoles/inst amp/input height}\pgf@circ@res@up}}
        %
        \pgftext[left, at=\pgfpoint{\ctikzvalof{tripoles/inst amp/port width}\pgf@circ@res@left}{\ctikzvalof{tripoles/inst amp/input height}\pgf@circ@res@up}]{\ctikzvalof{tripoles/inst amp/font} \ifpgf@circuit@oa@iplusup\pgf@circ@ampli@plus\else\pgf@circ@ampli@minus\fi}

        % Positive input terminal
        \pgfpathmoveto{\pgfpoint
            {\pgf@circ@res@left}
        {\ctikzvalof{tripoles/inst amp/input height}\pgf@circ@res@down}}
        %
        \pgfpathlineto{\pgfpoint
            {\ctikzvalof{tripoles/inst amp/port width}\pgf@circ@res@left}
        {\ctikzvalof{tripoles/inst amp/input height}\pgf@circ@res@down}}
    \pgftext[left, at=\pgfpoint{\ctikzvalof{tripoles/inst amp/port width}\pgf@circ@res@left}{\ctikzvalof{tripoles/inst amp/input height}\pgf@circ@res@down}]{\ctikzvalof{tripoles/inst amp/font} \ifpgf@circuit@oa@iplusup\pgf@circ@ampli@minus\else\pgf@circ@ampli@plus\fi}

        % Output terminal
        \pgfpathmoveto{\pgfpoint{\pgf@circ@res@right}{0pt}}
        \pgfpathlineto{\pgfpoint{\ctikzvalof{tripoles/inst amp/port width}\pgf@circ@res@right}{0pt}} %
        %
        \pgfsetrectcap
        \pgfusepath{draw}
    }
}

% Instrumentation amplifier with terminals for gain resistance between inputs
\pgfdeclareshape{inst amp ra}
{
    \savedmacro{\ctikzclass}{\edef\ctikzclass{amplifiers}}
    \saveddimen{\scaledRlen}{\pgfmathsetlength{\pgf@x}{\ctikzvalof{\ctikzclass/scale}\pgf@circ@Rlen}}
    \anchor{center}{\pgfpointorigin}
    % bounding-box top left
    \savedanchor\northwest{%
        \pgfmathsetlength{\pgf@circ@scaled@Rlen}{\ctikzvalof{\ctikzclass/scale}\pgf@circ@Rlen}
        \pgf@y=\ctikzvalof{tripoles/inst amp ra/height}\pgf@circ@scaled@Rlen
        \pgf@y=.5\pgf@y
        \pgf@x=-\ctikzvalof{tripoles/inst amp ra/width}\pgf@circ@scaled@Rlen
        \pgf@x=.5\pgf@x
    }
    \anchor{nw}{
        \northwest
    }
    \anchor{south}{
        \northwest
        \pgf@y=-\pgf@y\pgf@x=0pt\relax
    }
    \anchor{north}{
        \northwest\pgf@x=0pt\relax
    }
    \savedanchor\left{%
        \pgf@y=0pt
    }
    \anchor{leftedge}
    {\left
        \pgf@x = \ctikzvalof{tripoles/op amp/port width}\pgf@x
    }
    % inputs (+-)
    \savedanchor\inOneFixed{%
        \pgfmathsetlength{\pgf@circ@scaled@Rlen}{\ctikzvalof{\ctikzclass/scale}\pgf@circ@Rlen}
          \pgf@y=\ctikzvalof{tripoles/inst amp ra/height}\pgf@circ@scaled@Rlen
          \pgf@y=.5\pgf@y
          \pgf@y=\ctikzvalof{tripoles/inst amp ra/input height}\pgf@y
          \pgf@x=-\ctikzvalof{tripoles/inst amp ra/width}\pgf@circ@scaled@Rlen
          \pgf@x=.5\pgf@x
    }
    \anchor{in up}{
        \inOneFixed
    }
    \anchor{in down}{
        \inOneFixed
        \pgf@y=-\pgf@y
    }
    \savedanchor\inOne{%
        \pgfmathsetlength{\pgf@circ@scaled@Rlen}{\ctikzvalof{\ctikzclass/scale}\pgf@circ@Rlen}
        \pgf@y=\ctikzvalof{tripoles/inst amp ra/height}\pgf@circ@scaled@Rlen
        \pgf@y=.5\pgf@y
        \pgf@y=\ctikzvalof{tripoles/inst amp ra/input height}\pgf@y
        \pgf@x=-\ctikzvalof{tripoles/inst amp ra/width}\pgf@circ@scaled@Rlen
        \pgf@x=.5\pgf@x
        \ifpgf@circuit@oa@iplusup\pgf@y=-\pgf@y\fi
    }
    \anchor{-}{
        \inOne
    }
    \anchor{+}{
        \inOne
        \pgf@y=-\pgf@y
    }
    % R ampli anchors. They are by default at 20% more than R-length distance
    % you can change that with the `ra pos` key (use 0.5 for one-R).
    \savedanchor\raOneFixed{%
        \pgfmathsetlength{\pgf@circ@scaled@Rlen}{\ctikzvalof{\ctikzclass/scale}\pgf@circ@Rlen}
        \pgf@y=\pgf@circ@scaled@Rlen
        \pgf@y=\ctikzvalof{tripoles/inst amp ra/ra pos}\pgf@y
        \pgf@x=-\ctikzvalof{tripoles/inst amp ra/width}\pgf@circ@scaled@Rlen
        \pgf@x=.5\pgf@x
    }
    \anchor{ra up}{
        \raOneFixed
    }
    \anchor{ra down}{
        \raOneFixed
        \pgf@y=-\pgf@y
    }
    \savedanchor\raOne{%
        \pgfmathsetlength{\pgf@circ@scaled@Rlen}{\ctikzvalof{\ctikzclass/scale}\pgf@circ@Rlen}
        \pgf@y=\pgf@circ@scaled@Rlen
        \pgf@y=\ctikzvalof{tripoles/inst amp ra/ra pos}\pgf@y
        \pgf@x=-\ctikzvalof{tripoles/inst amp ra/width}\pgf@circ@scaled@Rlen
        \pgf@x=.5\pgf@x
        \ifpgf@circuit@oa@iplusup\pgf@y=-\pgf@y\fi
    }
    \anchor{ra-}{
        \raOne
    }
    \anchor{ra+}{
        \raOne
        \pgf@y=-\pgf@y
    }
    % power supplies
    \savedanchor\up{%
        \pgfmathsetlength{\pgf@circ@scaled@Rlen}{\ctikzvalof{\ctikzclass/scale}\pgf@circ@Rlen}
        \pgf@y=\ctikzvalof{tripoles/inst amp ra/height}\pgf@circ@scaled@Rlen
        \pgf@y=.5\pgf@y
        \pgf@x=-\ctikzvalof{tripoles/inst amp ra/width}\pgf@circ@scaled@Rlen
        \pgf@x=.5\pgf@x
        \pgf@circ@res@up = \pgf@y
        \pgf@circ@res@right = -\pgf@x
        \pgf@circ@res@left = \pgf@x
        \pgfpointlineattime{
            \ctikzvalof{tripoles/inst amp ra/up pos}}{
            \pgfpoint{
            \ctikzvalof{tripoles/inst amp ra/port width}\pgf@circ@res@left}
        {\pgf@circ@res@up}}
        {\pgfpoint{\ctikzvalof{tripoles/inst amp ra/port width}\pgf@circ@res@right}{\ctikzvalof{tripoles/inst amp ra/height 2}\pgf@circ@res@up}}
    }
    \anchor{up}{
        \up
    }
    \anchor{down}{
        \up
        \pgf@y=-\pgf@y
    }
    % reference voltage input anchors.
    \savedanchor\refv{%
        \pgfmathsetlength{\pgf@circ@scaled@Rlen}{\ctikzvalof{\ctikzclass/scale}\pgf@circ@Rlen}
        \pgf@y=\ctikzvalof{tripoles/inst amp ra/height}\pgf@circ@scaled@Rlen
        \pgf@y=0.5\pgf@y
        \pgf@x=-\ctikzvalof{tripoles/inst amp ra/width}\pgf@circ@scaled@Rlen
        \pgf@x=0.5\pgf@x
        \pgf@circ@res@up = \pgf@y
        \pgf@circ@res@right = -\pgf@x
        \pgf@circ@res@left = \pgf@x
        \pgfpointlineattime{
            \ctikzvalof{tripoles/inst amp ra/refv pos}}{
            \pgfpoint{
            \ctikzvalof{tripoles/inst amp ra/port width}\pgf@circ@res@left}
        {\pgf@circ@res@up}}
        {\pgfpoint{\ctikzvalof{tripoles/inst amp ra/port width}\pgf@circ@res@right}{\ctikzvalof{tripoles/inst amp ra/height 2}\pgf@circ@res@up}}
    }
    % we need both because they are normally drawn under the amp, and if you
    % mirror it vertically you need them
    \anchor{refv up}{
        \refv
    }
    \anchor{refv down}{
        \refv
        \pgf@y=-\pgf@y
    }
    \anchor{out}{
        \northwest
        \pgf@y=0pt
        \pgf@x=-\pgf@x
    }
    % support for border anchors
    \savedanchor\leftedge{%
        \pgfmathsetlength{\pgf@circ@scaled@Rlen}{\ctikzvalof{\ctikzclass/scale}\pgf@circ@Rlen}
        \pgf@x=-\ctikzvalof{tripoles/inst amp ra/width}\pgf@circ@scaled@Rlen
        \pgf@x=.5\pgf@x
        \pgf@x = \ctikzvalof{tripoles/inst amp ra/port width}\pgf@x
        \pgf@y=0pt
    }
    \anchor{leftedge}{\leftedge}
    \anchor{rightedge}{\leftedge \pgf@x =-\pgf@x }
    \anchor{bin up}{
        \inOneFixed
        \pgf@ya=\pgf@y \leftedge \pgf@y=\pgf@ya
    }
    \anchor{bin down}{
        \inOneFixed
        \pgf@y=-\pgf@y
        \pgf@ya=\pgf@y \leftedge \pgf@y=\pgf@ya
    }
    \anchor{b-}{
        \inOne
        \pgf@ya=\pgf@y \leftedge \pgf@y=\pgf@ya
    }
    \anchor{b+}{
        \inOne
        \pgf@y=-\pgf@y
        \pgf@ya=\pgf@y \leftedge \pgf@y=\pgf@ya
    }
    \anchor{bra up}{
        \raOneFixed
        \pgf@ya=\pgf@y \leftedge \pgf@y=\pgf@ya
    }
    \anchor{bra down}{
        \raOneFixed
        \pgf@y=-\pgf@y
        \pgf@ya=\pgf@y \leftedge \pgf@y=\pgf@ya
    }
    \anchor{bra-}{
        \raOne
        \pgf@ya=\pgf@y \leftedge \pgf@y=\pgf@ya
    }
    \anchor{bra+}{
        \raOne
        \pgf@y=-\pgf@y
        \pgf@ya=\pgf@y \leftedge \pgf@y=\pgf@ya
    }
    \anchor{bout}{
        \leftedge
        \pgf@y=0pt
        \pgf@x=-\pgf@x
    }
    % end border anchors
    \anchor{west}{
        \left
    }
    \anchor{east}{
        \left
        \pgf@x=-\pgf@x
    }
    \anchor{south west}{ \northwest \pgf@y=-\pgf@y }
    \anchor{north east}{ \northwest \pgf@x=-\pgf@x }
    \anchor{north west}{ \northwest }
    \anchor{south east}{ \northwest \pgf@x=-\pgf@x \pgf@y=-\pgf@y }

    \anchor{text}{
        \ifpgf@circ@center@text
            \pgf@x=\dimexpr -.6\wd\pgfnodeparttextbox\relax
        \else
            \leftedge
            \pgfmathsetlength{\pgf@x}{\pgf@x + \ctikzvalof{left text distance}}
        \fi
        \pgf@y=\dimexpr.5\dp\pgfnodeparttextbox-.5\ht\pgfnodeparttextbox\relax
    }

    % drawing of the component
    \backgroundpath{
        \pgfsetcolor{\ctikzvalof{color}}

        \northwest
        \pgf@circ@res@up = \pgf@y
        \pgf@circ@res@down = -\pgf@y
        \pgf@circ@res@right = -\pgf@x
        \pgf@circ@res@left = \pgf@x
        \pgf@circ@scaled@Rlen=\scaledRlen

        \newdimen\pgf@circ@res@right@double
				\pgf@circ@res@right@double=2\pgf@circ@res@right

        % main component, normally in thicker lines
        \pgfscope
            \pgf@circ@setlinewidth{tripoles}{\pgflinewidth}
            \pgftransformxshift{\ctikzvalof{tripoles/inst amp ra/port width}\pgf@circ@res@left}
            \pgf@circ@res@step=\pgf@circ@res@right
            \advance\pgf@circ@res@step by -\pgf@circ@res@left
            \pgf@circ@res@step=\ctikzvalof{tripoles/inst amp ra/port width}\pgf@circ@res@step
            %primer punto: la linea de salida (lado componente)
            \pgfpathmoveto{\pgfpoint{\ctikzvalof{tripoles/inst amp ra/port width}\pgf@circ@res@right@double}{0}}
            %from the exit to the top (short side)... (note that the .6 must be copied in \up anchor
            \pgfpathlineto{\pgfpoint{\ctikzvalof{tripoles/inst amp ra/port width}\pgf@circ@res@right@double}{\ctikzvalof{tripoles/inst amp ra/height 2}\pgf@circ@res@up}}
            % and then to the input "front up", "down", to the output short side "down"
            \pgfpathlineto{\pgfpoint{0pt}{\pgf@circ@res@up}}
            \pgfpathlineto{\pgfpoint{0pt}{\pgf@circ@res@down}}
            \pgfpathlineto{\pgfpoint{\ctikzvalof{tripoles/inst amp ra/port width}\pgf@circ@res@right@double}{\ctikzvalof{tripoles/inst amp ra/height 2}\pgf@circ@res@down}}
            % ...and close
            \pgfpathclose
            \pgf@circ@draworfill
        \endpgfscope

        % ra terminal -
        \pgfpathmoveto{\pgfpoint
            {\pgf@circ@res@left}
        {\ctikzvalof{tripoles/inst amp ra/ra pos}\pgf@circ@Rlen}}
        %
        \pgfpathlineto{\pgfpoint
            {\ctikzvalof{tripoles/inst amp ra/port width}\pgf@circ@res@left}
        {\ctikzvalof{tripoles/inst amp ra/ra pos}\pgf@circ@Rlen}}
        % ra terminal +
        \pgfpathmoveto{\pgfpoint
            {\pgf@circ@res@left}
        {-\ctikzvalof{tripoles/inst amp ra/ra pos}\pgf@circ@Rlen}}
        %
        \pgfpathlineto{\pgfpoint
            {\ctikzvalof{tripoles/inst amp ra/port width}\pgf@circ@res@left}
        {-\ctikzvalof{tripoles/inst amp ra/ra pos}\pgf@circ@Rlen}}

        % Negative input terminal
        \pgfpathmoveto{\pgfpoint
            {\pgf@circ@res@left}
        {\ctikzvalof{tripoles/inst amp ra/input height}\pgf@circ@res@up}}
        %
        \pgfpathlineto{\pgfpoint
            {\ctikzvalof{tripoles/inst amp ra/port width}\pgf@circ@res@left}
        {\ctikzvalof{tripoles/inst amp ra/input height}\pgf@circ@res@up}}
        %
        \pgftext[left, at=\pgfpoint{\ctikzvalof{tripoles/inst amp ra/port width}\pgf@circ@res@left}{\ctikzvalof{tripoles/inst amp ra/input height}\pgf@circ@res@up}]{\ctikzvalof{tripoles/inst amp ra/font} \ifpgf@circuit@oa@iplusup\pgf@circ@ampli@plus\else\pgf@circ@ampli@minus\fi}

        % Positive input terminal
        \pgfpathmoveto{\pgfpoint
            {\pgf@circ@res@left}
        {\ctikzvalof{tripoles/inst amp ra/input height}\pgf@circ@res@down}}
        %
        \pgfpathlineto{\pgfpoint
            {\ctikzvalof{tripoles/inst amp ra/port width}\pgf@circ@res@left}
        {\ctikzvalof{tripoles/inst amp ra/input height}\pgf@circ@res@down}}
        \pgftext[left, at=\pgfpoint{\ctikzvalof{tripoles/inst amp ra/port width}\pgf@circ@res@left}{\ctikzvalof{tripoles/inst amp ra/input height}\pgf@circ@res@down}]{\ctikzvalof{tripoles/inst amp ra/font} \ifpgf@circuit@oa@iplusup\pgf@circ@ampli@minus\else\pgf@circ@ampli@plus\fi}

        % Output terminal
        \pgfpathmoveto{\pgfpoint{\pgf@circ@res@right}{0pt}}
        \pgfpathlineto{\pgfpoint{\ctikzvalof{tripoles/inst amp ra/port width}\pgf@circ@res@right}{0pt}} %
        %
        \pgfsetrectcap
        \pgfusepath{draw}
    }
}

% Buffer
% Contributed by Danilo Piazzalunga
\pgfdeclareshape{buffer}
{
    \savedmacro{\ctikzclass}{\edef\ctikzclass{amplifiers}}
    \saveddimen{\scaledRlen}{\pgfmathsetlength{\pgf@x}{\ctikzvalof{\ctikzclass/scale}\pgf@circ@Rlen}}
    \anchor{center}{\pgfpointorigin}
    \savedanchor\northwest{%
        \pgfmathsetlength{\pgf@circ@scaled@Rlen}{\ctikzvalof{\ctikzclass/scale}\pgf@circ@Rlen}
        \pgf@y=\ctikzvalof{bipoles/buffer/height}\pgf@circ@scaled@Rlen
        \pgf@y=.5\pgf@y
        \pgf@x=-\ctikzvalof{bipoles/buffer/width}\pgf@circ@scaled@Rlen
        \pgf@x=.5\pgf@x
    }
    \anchor{east}{ \northwest \pgf@y=0pt \pgf@x=-\pgf@x  }
    \anchor{west}{ \northwest \pgf@y=0pt }
    \anchor{south}{ \northwest \pgf@x=0pt \pgf@y=-\pgf@y }
    \anchor{north}{ \northwest \pgf@x=0pt }
    \anchor{south west}{ \northwest \pgf@y=-\pgf@y }
    \anchor{north east}{ \northwest \pgf@x=-\pgf@x }
    \anchor{north west}{ \northwest }
    \anchor{south east}{ \northwest \pgf@x=-\pgf@x \pgf@y=-\pgf@y }
    \savedanchor\left{%
        \pgf@y=0pt
    }
    \anchor{in}{
        \northwest
        \pgf@y=0pt
    }
    \anchor{out}{
        \northwest
        \pgf@y=0pt
        \pgf@x=-\pgf@x
    }
    \anchor{bin}{
        \northwest
        \pgf@y=0pt
        \pgf@x=0.7\pgf@x
    }
    \anchor{bout}{
        \northwest
        \pgf@y=0pt
        \pgf@x=-0.7\pgf@x
    }

    \anchor{text}{
        \ifpgf@circ@center@text
            \pgf@x=\dimexpr -.6\wd\pgfnodeparttextbox\relax
        \else
            \northwest
            \pgfmathsetlength{\pgf@x}{0.7*\pgf@x + \ctikzvalof{left text distance}}
        \fi
        \pgf@y=\dimexpr.5\dp\pgfnodeparttextbox-.5\ht\pgfnodeparttextbox\relax
    }

    \backgroundpath{
        \pgfsetcolor{\ctikzvalof{color}}

        \northwest
        \pgf@circ@res@up = \pgf@y
        \pgf@circ@res@down = -\pgf@y
        \pgf@circ@res@right = -\pgf@x
        \pgf@circ@res@left = \pgf@x
        \pgf@circ@scaled@Rlen=\scaledRlen
        \pgfscope
            \pgf@circ@setlinewidth{tripoles}{\pgflinewidth}
            \pgftransformxshift{.7\pgf@circ@res@left}
            \pgf@circ@res@step=\pgf@circ@res@right
            \advance\pgf@circ@res@step by -\pgf@circ@res@left
            \pgf@circ@res@step=.7\pgf@circ@res@step

            \pgfpathmoveto{\pgfpoint{\pgf@circ@res@step}{0pt}}
            \pgfpathlineto{\pgfpoint{0pt}{\pgf@circ@res@up}}
            \pgfpathlineto{\pgfpoint{0pt}{\pgf@circ@res@down}}
            \pgfpathclose
            \pgf@circ@draworfill
        \endpgfscope

        \pgfpathmoveto{\pgfpoint{\pgf@circ@res@left}{0pt}}
        \pgfpathlineto{\pgfpoint{.7\pgf@circ@res@left}{0pt}}

        \pgfpathmoveto{\pgfpoint{\pgf@circ@res@right}{0pt}}
        \pgfpathlineto{\pgfpoint{.7\pgf@circ@res@right}{0pt}}

        \pgfusepath{draw}
    }
}

% plain amplifier, no symbols
\pgfdeclareshape{plain amp}
{
    \savedmacro{\ctikzclass}{\edef\ctikzclass{amplifiers}}
    \saveddimen{\scaledRlen}{\pgfmathsetlength{\pgf@x}{\ctikzvalof{\ctikzclass/scale}\pgf@circ@Rlen}}
    \anchor{center}{\pgfpointorigin}
    \savedanchor\northwest{%
        \pgfmathsetlength{\pgf@circ@scaled@Rlen}{\ctikzvalof{\ctikzclass/scale}\pgf@circ@Rlen}
        \pgf@y=\ctikzvalof{tripoles/plain amp/height}\pgf@circ@scaled@Rlen
        \pgf@y=.5\pgf@y
        \pgf@x=-\ctikzvalof{tripoles/plain amp/width}\pgf@circ@scaled@Rlen
        \pgf@x=.5\pgf@x
    }
    \anchor{east}{ \northwest \pgf@y=0pt \pgf@x=-\pgf@x  }
    \anchor{west}{ \northwest \pgf@y=0pt }
    \anchor{south}{ \northwest \pgf@x=0pt \pgf@y=-\pgf@y }
    \anchor{north}{ \northwest \pgf@x=0pt }
    \anchor{south west}{ \northwest \pgf@y=-\pgf@y }
    \anchor{north east}{ \northwest \pgf@x=-\pgf@x }
    \anchor{north west}{ \northwest }
    \anchor{south east}{ \northwest \pgf@x=-\pgf@x \pgf@y=-\pgf@y }
    \savedanchor\left{%
        \pgf@y=0pt
    }
    \savedanchor\inOne{%
        \pgfmathsetlength{\pgf@circ@scaled@Rlen}{\ctikzvalof{\ctikzclass/scale}\pgf@circ@Rlen}
        \pgf@y=\ctikzvalof{tripoles/plain amp/height}\pgf@circ@scaled@Rlen
        \pgf@y=.5\pgf@y
        \pgf@y=\ctikzvalof{tripoles/plain amp/input height}\pgf@y
        \pgf@x=-\ctikzvalof{tripoles/plain amp/width}\pgf@circ@scaled@Rlen
        \pgf@x=.5\pgf@x
        \ifpgf@circuit@oa@iplusup\pgf@y=-\pgf@y\fi
    }
    \savedanchor\inOneFixed{%
        \pgfmathsetlength{\pgf@circ@scaled@Rlen}{\ctikzvalof{\ctikzclass/scale}\pgf@circ@Rlen}
        \pgf@y=\ctikzvalof{tripoles/plain amp/height}\pgf@circ@scaled@Rlen
        \pgf@y=.5\pgf@y
        \pgf@y=\ctikzvalof{tripoles/plain amp/input height}\pgf@y
        \pgf@x=-\ctikzvalof{tripoles/plain amp/width}\pgf@circ@scaled@Rlen
        \pgf@x=.5\pgf@x
    }
    \anchor{-}{
        \inOne
    }
    \anchor{+}{
        \inOne
        \pgf@y=-\pgf@y
    }
    \anchor{in up}{
        \inOneFixed
    }
    \anchor{in down}{
        \inOneFixed
        \pgf@y=-\pgf@y
    }
    % support for border anchors
    \savedanchor\leftedge{%
        \pgfmathsetlength{\pgf@circ@scaled@Rlen}{\ctikzvalof{\ctikzclass/scale}\pgf@circ@Rlen}
        \pgf@x=-\ctikzvalof{tripoles/plain amp/width}\pgf@circ@scaled@Rlen
        \pgf@x=.5\pgf@x
        \pgf@x = \ctikzvalof{tripoles/plain amp/port width}\pgf@x
        \pgf@y=0pt
    }
    \anchor{leftedge}{\leftedge}
    \anchor{rightedge}{\leftedge \pgf@x =-\pgf@x }
    \anchor{bin up}{
        \inOneFixed
        \pgf@ya=\pgf@y \leftedge \pgf@y=\pgf@ya
    }
    \anchor{bin down}{
        \inOneFixed
        \pgf@y=-\pgf@y
        \pgf@ya=\pgf@y \leftedge \pgf@y=\pgf@ya
    }
    \anchor{b-}{
        \inOne
        \pgf@ya=\pgf@y \leftedge \pgf@y=\pgf@ya
    }
    \anchor{b+}{
        \inOne
        \pgf@y=-\pgf@y
        \pgf@ya=\pgf@y \leftedge \pgf@y=\pgf@ya
    }
    \anchor{bout}{
        \leftedge
        \pgf@y=0pt
        \pgf@x=-\pgf@x
    }
    % end border anchors
    \savedanchor\up{%
        \pgfmathsetlength{\pgf@circ@scaled@Rlen}{\ctikzvalof{\ctikzclass/scale}\pgf@circ@Rlen}
        \pgf@y=\ctikzvalof{tripoles/plain amp/height}\pgf@circ@scaled@Rlen
        \pgf@y=.5\pgf@y
        \pgf@x=-\ctikzvalof{tripoles/plain amp/width}\pgf@circ@scaled@Rlen
        \pgf@x=.5\pgf@x
        \pgf@circ@res@up = \pgf@y
        \pgf@circ@res@right = -\pgf@x
        \pgf@circ@res@left = \pgf@x
        \pgfpointlineattime{
            \ctikzvalof{tripoles/plain amp/up pos}}{
            \pgfpoint{
            \ctikzvalof{tripoles/plain amp/port width}\pgf@circ@res@left}
        {\pgf@circ@res@up}}
        {\pgfpoint{\ctikzvalof{tripoles/plain amp/port width}\pgf@circ@res@right}{0pt}}
    }
    \anchor{up}{
        \up
    }
    \anchor{down}{
        \up
        \pgf@y=-\pgf@y
    }
    \anchor{out}{
        \northwest
        \pgf@y=0pt
        \pgf@x=-\pgf@x
    }

    \anchor{text}{
        \ifpgf@circ@center@text
            \pgf@x=\dimexpr -.6\wd\pgfnodeparttextbox\relax
        \else
            \leftedge
            \pgfmathsetlength{\pgf@x}{\pgf@x + \ctikzvalof{left text distance}}
        \fi
        \pgf@y=\dimexpr.5\dp\pgfnodeparttextbox-.5\ht\pgfnodeparttextbox\relax
    }

    \backgroundpath{
        \pgfsetcolor{\ctikzvalof{color}}

        \northwest
        \pgf@circ@res@up = \pgf@y
        \pgf@circ@res@down = -\pgf@y
        \pgf@circ@res@right = -\pgf@x
        \pgf@circ@res@left = \pgf@x

        \pgfscope
            \pgf@circ@setlinewidth{tripoles}{\pgflinewidth}
            \pgftransformxshift{\ctikzvalof{tripoles/plain amp/port width}\pgf@circ@res@left}
            \pgf@circ@res@step=\pgf@circ@res@right
            \advance\pgf@circ@res@step by -\pgf@circ@res@left
            \pgf@circ@res@step=\ctikzvalof{tripoles/plain amp/port width}\pgf@circ@res@step

            \pgfpathmoveto{\pgfpoint{\pgf@circ@res@step}{0pt}}
            \pgfpathlineto{\pgfpoint{0pt}{\pgf@circ@res@up}}
            \pgfpathlineto{\pgfpoint{0pt}{\pgf@circ@res@down}}
            \pgfpathclose
            \pgf@circ@draworfill
        \endpgfscope
        \pgfpathmoveto{\pgfpoint
            {\pgf@circ@res@left}
        {\ctikzvalof{tripoles/plain amp/input height}\pgf@circ@res@up}}
        \pgfpathlineto{\pgfpoint
            {\ctikzvalof{tripoles/plain amp/port width}\pgf@circ@res@left}
        {\ctikzvalof{tripoles/plain amp/input height}\pgf@circ@res@up}}


        \pgfpathmoveto{\pgfpoint
            {\pgf@circ@res@left}
        {\ctikzvalof{tripoles/plain amp/input height}\pgf@circ@res@down}}
        \pgfpathlineto{\pgfpoint
            {\ctikzvalof{tripoles/plain amp/port width}\pgf@circ@res@left}
        {\ctikzvalof{tripoles/plain amp/input height}\pgf@circ@res@down}}

        \pgfpathmoveto{\pgfpoint{\pgf@circ@res@right}{0pt}}
        \pgfpathlineto{\pgfpoint{\ctikzvalof{tripoles/plain amp/port width}\pgf@circ@res@right}{0pt}}

        \pgfusepath{draw}
    }
}

% plain amplifier, no symbols, one input
\pgfdeclareshape{plain mono amp}
{
    \savedmacro{\ctikzclass}{\edef\ctikzclass{amplifiers}}
    \saveddimen{\scaledRlen}{\pgfmathsetlength{\pgf@x}{\ctikzvalof{\ctikzclass/scale}\pgf@circ@Rlen}}
    \anchor{center}{\pgfpointorigin}
    \savedanchor\northwest{%
        \pgfmathsetlength{\pgf@circ@scaled@Rlen}{\ctikzvalof{\ctikzclass/scale}\pgf@circ@Rlen}
        \pgf@y=\ctikzvalof{tripoles/plain amp/height}\pgf@circ@scaled@Rlen
        \pgf@y=.5\pgf@y
        \pgf@x=-\ctikzvalof{tripoles/plain amp/width}\pgf@circ@scaled@Rlen
        \pgf@x=.5\pgf@x
    }
    \anchor{east}{ \northwest \pgf@y=0pt \pgf@x=-\pgf@x  }
    \anchor{west}{ \northwest \pgf@y=0pt }
    \anchor{south}{ \northwest \pgf@x=0pt \pgf@y=-\pgf@y }
    \anchor{north}{ \northwest \pgf@x=0pt }
    \anchor{south west}{ \northwest \pgf@y=-\pgf@y }
    \anchor{north east}{ \northwest \pgf@x=-\pgf@x }
    \anchor{north west}{ \northwest }
    \anchor{south east}{ \northwest \pgf@x=-\pgf@x \pgf@y=-\pgf@y }
    \savedanchor\left{%
        \pgf@y=0pt
    }
    \savedanchor\inOne{%
        \pgfmathsetlength{\pgf@circ@scaled@Rlen}{\ctikzvalof{\ctikzclass/scale}\pgf@circ@Rlen}
        \pgf@x=-\ctikzvalof{tripoles/plain amp/width}\pgf@circ@scaled@Rlen
        \pgf@x=.5\pgf@x
        \pgf@y=0pt\relax
    }
    \anchor{in}{
        \inOne
    }
    % support for border anchors
    \savedanchor\leftedge{%
        \pgfmathsetlength{\pgf@circ@scaled@Rlen}{\ctikzvalof{\ctikzclass/scale}\pgf@circ@Rlen}
        \pgf@x=-\ctikzvalof{tripoles/plain amp/width}\pgf@circ@scaled@Rlen
        \pgf@x=.5\pgf@x
        \pgf@x = \ctikzvalof{tripoles/plain amp/port width}\pgf@x
        \pgf@y=0pt
    }
    \anchor{leftedge}{\leftedge}
    \anchor{rightedge}{\leftedge \pgf@x =-\pgf@x }
    \anchor{bin}{
        \inOne
        \pgf@ya=\pgf@y \leftedge \pgf@y=\pgf@ya
    }
    \anchor{bout}{
        \leftedge
        \pgf@y=0pt
        \pgf@x=-\pgf@x
    }
    % end border anchors
    \savedanchor\up{%
        \pgfmathsetlength{\pgf@circ@scaled@Rlen}{\ctikzvalof{\ctikzclass/scale}\pgf@circ@Rlen}
        \pgf@y=\ctikzvalof{tripoles/plain amp/height}\pgf@circ@scaled@Rlen
        \pgf@y=.5\pgf@y
        \pgf@x=-\ctikzvalof{tripoles/plain amp/width}\pgf@circ@scaled@Rlen
        \pgf@x=.5\pgf@x
        \pgf@circ@res@up = \pgf@y
        \pgf@circ@res@right = -\pgf@x
        \pgf@circ@res@left = \pgf@x
        \pgfpointlineattime{
            \ctikzvalof{tripoles/plain amp/up pos}}{
            \pgfpoint{
            \ctikzvalof{tripoles/plain amp/port width}\pgf@circ@res@left}
        {\pgf@circ@res@up}}
        {\pgfpoint{\ctikzvalof{tripoles/plain amp/port width}\pgf@circ@res@right}{0pt}}
    }
    \anchor{up}{
        \up
    }
    \anchor{down}{
        \up
        \pgf@y=-\pgf@y
    }
    \anchor{out}{
        \northwest
        \pgf@y=0pt
        \pgf@x=-\pgf@x
    }

    \anchor{text}{
        \ifpgf@circ@center@text
            \pgf@x=\dimexpr -.6\wd\pgfnodeparttextbox\relax
        \else
            \leftedge
            \pgfmathsetlength{\pgf@x}{\pgf@x + \ctikzvalof{left text distance}}
        \fi
        \pgf@y=\dimexpr.5\dp\pgfnodeparttextbox-.5\ht\pgfnodeparttextbox\relax
    }
    \backgroundpath{
        \pgfsetcolor{\ctikzvalof{color}}

        \northwest
        \pgf@circ@res@up = \pgf@y
        \pgf@circ@res@down = -\pgf@y
        \pgf@circ@res@right = -\pgf@x
        \pgf@circ@res@left = \pgf@x

        \pgfscope
            \pgf@circ@setlinewidth{tripoles}{\pgflinewidth}
            \pgftransformxshift{\ctikzvalof{tripoles/plain amp/port width}\pgf@circ@res@left}
            \pgf@circ@res@step=\pgf@circ@res@right
            \advance\pgf@circ@res@step by -\pgf@circ@res@left
            \pgf@circ@res@step=\ctikzvalof{tripoles/plain amp/port width}\pgf@circ@res@step

            \pgfpathmoveto{\pgfpoint{\pgf@circ@res@step}{0pt}}
            \pgfpathlineto{\pgfpoint{0pt}{\pgf@circ@res@up}}
            \pgfpathlineto{\pgfpoint{0pt}{\pgf@circ@res@down}}
            \pgfpathclose
            \pgf@circ@draworfill
        \endpgfscope
        \pgfpathmoveto{\pgfpoint
            {\pgf@circ@res@left}
        {0pt}}
        \pgfpathlineto{\pgfpoint
            {\ctikzvalof{tripoles/plain amp/port width}\pgf@circ@res@left}
        {0pt}}

        \pgfpathmoveto{\pgfpoint{\pgf@circ@res@right}{0pt}}
        \pgfpathlineto{\pgfpoint{\ctikzvalof{tripoles/plain amp/port width}\pgf@circ@res@right}{0pt}}

        \pgfusepath{draw}
    }
}
%%%%%%%%%%%%%%%%%%%%%%%%%%%%%%%%%%%%%%%%%%%%%
%% electromechanical device (motor/generator)
%%%%%%%%%%%%%%%%%%%%%%%%%%%%%%%%%%%%%%%%%%%%%

\pgfdeclareshape{elmech}
{
    \savedmacro{\ctikzclass}{\edef\ctikzclass{electromechanicals}}
    \saveddimen{\scaledRlen}{\pgfmathsetlength{\pgf@x}{\ctikzvalof{\ctikzclass/scale}\pgf@circ@Rlen}}
    \savedanchor\northwest{
        \pgfmathsetlength{\pgf@circ@scaled@Rlen}{\ctikzvalof{\ctikzclass/scale}\pgf@circ@Rlen}
        \pgf@y=\ctikzvalof{tripoles/elmech/height}\pgf@circ@scaled@Rlen
        \pgf@y=.5\pgf@y
        \pgf@x=-\ctikzvalof{tripoles/elmech/width}\pgf@circ@scaled@Rlen
        \pgf@x=.5\pgf@x
    }
    \anchor{center}{
        \pgfpointorigin
    }
    \anchor{text}{
        \pgfpointorigin
        \advance \pgf@x by -.5\wd\pgfnodeparttextbox
        \advance \pgf@y by -.5\ht\pgfnodeparttextbox
    }
    \anchor{left}{%
        \northwest
        \pgf@y=0pt
    }
    \anchor{right}{%
        \northwest
        \pgf@y=0pt
        \pgf@x=-\pgf@x
    }
    \anchor{top}{
        \northwest
        \pgf@x=0pt
    }
    \anchor{pathstart}{
        \northwest
        \pgf@x=0pt
    }
    \anchor{pathend}{
        \northwest
        \pgf@x=0pt
        \pgf@y=-\pgf@y
    }
    \anchor{bottom}{
        \northwest
        \pgf@x=0pt
        \pgf@y=-\pgf@y
    }
    \anchor{center}{
        \pgf@y=0pt
        \pgf@x=0pt
    }
    \anchor{east}{
        \northwest
        \pgf@y=0pt
        \pgf@x=-\pgf@x
    }
    \anchor{west}{
        \northwest
        \pgf@y=0pt
    }
    \anchor{south}{
        \northwest
        \pgf@x=0pt
        \pgf@y=-\pgf@y
    }
    \anchor{north}{
        \northwest
        \pgf@x=0pt
    }
    \anchor{south west}{ \northwest \pgf@y=-\pgf@y }
    \anchor{north east}{ \northwest \pgf@x=-\pgf@x }
    \anchor{north west}{ \northwest }
    \anchor{south east}{ \northwest \pgf@x=-\pgf@x \pgf@y=-\pgf@y }
    \anchorborder{%
        \@tempdima=\pgf@x\@tempdimb=\pgf@y
        \northwest\pgf@circ@res@other=-\pgf@x
        \pgfpointborderellipse{\pgfqpoint{\@tempdima}{\@tempdimb}}{\pgfqpoint{\pgf@circ@res@other}{\pgf@circ@res@other}}
    }
    \anchor{block north west}{\northwest\pgf@x=0.5\pgf@x}
    \anchor{block south west}{\northwest\pgf@x=0.5\pgf@x\pgf@y=-\pgf@y}
    \anchor{block north east}{\northwest\pgf@x=-0.5\pgf@x}
    \anchor{block south east}{\northwest\pgf@x=-0.5\pgf@x\pgf@y=-\pgf@y}
    \anchor{block up right}{
        \northwest
        % remember that pgf@x is negative
        % center of the block is at 0.5*H+W*cos(30)/2
        \pgf@y=\dimexpr0.5\pgf@y - 0.433\pgf@x\relax
        \pgf@x=-0.5\pgf@x
    }
    \anchor{block up left}{
        \northwest
        % remember that pgf@x is negative
        % center of the block is at 0.5*H+W*cos(30)/2
        \pgf@y=\dimexpr0.5\pgf@y - 0.433\pgf@x\relax
        \pgf@x=0.5\pgf@x
    }
    \anchor{block down right}{
        \northwest
        % remember that pgf@x is negative
        % center of the block is at 0.5*H+W*cos(30)/2
        \pgf@y=\dimexpr0.5\pgf@y - 0.433\pgf@x\relax
        \pgf@y=-\pgf@y
        \pgf@x=-0.5\pgf@x
    }
    \anchor{block down left}{
        \northwest
        % remember that pgf@x is negative
        % center of the block is at 0.5*H+W*cos(30)/2
        \pgf@y=\dimexpr0.5\pgf@y - 0.433\pgf@x\relax
        \pgf@y=-\pgf@y
        \pgf@x=0.5\pgf@x
    }
    \behindbackgroundpath{
        \pgfsetcolor{\ctikzvalof{color}}
        \pgf@circ@scaled@Rlen=\scaledRlen
        \pgf@circ@res@step=\ctikzvalof{tripoles/elmech/width}\pgf@circ@scaled@Rlen
        \pgf@circ@res@up=\ctikzvalof{tripoles/elmech/height}\pgf@circ@scaled@Rlen
        \pgfscope
            \pgfstartlinewidth=\pgflinewidth
            \pgf@circ@setlinewidth{bipoles}{\pgfstartlinewidth}
            \pgfscope % clip the bar: whole size minus the circle
                \pgfpathrectanglecorners{\pgfpoint{-.5\pgf@circ@res@step}{-.5\pgf@circ@res@up}}{\pgfpoint{.5\pgf@circ@res@step}{.5\pgf@circ@res@up}}
                \pgfpathcircle{\pgfpoint{0}{0}} {0.5\pgf@circ@res@step}
                \pgfseteorule
                \pgfusepath{clip}
                \pgfpathrectangle{\pgfpoint{-.25\pgf@circ@res@step}{-.5\pgf@circ@res@up}}{\pgfpoint{.5\pgf@circ@res@step}{\pgf@circ@res@up}}
                \pgfsetfillcolor{\ctikzvalof{color}}
                \pgfusepath{fill, draw}
            \endpgfscope
            \pgfpathcircle{\pgfpoint{0}{0}} {0.5\pgf@circ@res@step}
            \ifx\tikz@fillcolor\pgfutil@empty
                % set the default fill color to white
                \pgfsetfillcolor{white}
                % ...but override it if the class is defined!
                \pgf@circ@setifdefinedfill{draw, fill}{draw, fill}
            \else
                \pgfsetfillcolor{\tikz@fillcolor}
                \pgfusepath{draw, fill}
            \fi
        \endpgfscope
    }
}

%%%%%%%%%%%%%%%%%%%
%% Magnetron
%%%%%%%%%%%%%%%%%%%

\pgfdeclareshape{magnetron}
{
    \savedmacro{\ctikzclass}{\edef\ctikzclass{tubes}}  % class of these components
    \saveddimen{\scaledRlen}{\pgfmathsetlength{\pgf@x}{\ctikzvalof{\ctikzclass/scale}\pgf@circ@Rlen}}
    \savedanchor\northwest{
        \pgfmathsetlength{\pgf@circ@scaled@Rlen}{\ctikzvalof{\ctikzclass/scale}\pgf@circ@Rlen}
        \pgf@y=\ctikzvalof{tripoles/magnetron/width}\pgf@circ@scaled@Rlen
        \pgf@y=.5\pgf@y
        \pgf@x=-\ctikzvalof{tripoles/magnetron/width}\pgf@circ@scaled@Rlen
        \pgf@x=.5\pgf@x
    }
    \anchor{center}{
        \pgfpointorigin
    }
    \anchor{anode}{
        \northwest
        \pgf@x=0pt
        \pgf@y=-\pgf@y
    }
    \anchor{cathode1}{
        \northwest
        \pgf@circ@res@step=\pgf@y
        \pgfmathparse{cos(105)}
        \pgf@x=\pgfmathresult\pgf@circ@res@step
        \pgfmathparse{sin(105)}
        \pgf@y=\pgfmathresult\pgf@circ@res@step
        %\pgfpointorigin
        %\pgfpathmoveto{\pgfpointpolar{105}{\pgf@circ@res@step}}%not working in a scaled tikzpicture
    }
    \anchor{cathode2}{
        \northwest
        \pgf@circ@res@step=\pgf@y
        \pgfmathparse{cos(75)}
        \pgf@x=\pgfmathresult\pgf@circ@res@step
        \pgfmathparse{sin(75)}
        \pgf@y=\pgfmathresult\pgf@circ@res@step
        %\pgfpointorigin
        %\pgfpathmoveto{\pgfpointpolar{75}{\pgf@circ@res@step}}%not working in a scaled tikzpicture
    }
    \anchor{text}{
        \pgfpointorigin
        \advance \pgf@x by -.5\wd\pgfnodeparttextbox
        \advance \pgf@y by -.5\ht\pgfnodeparttextbox
    }
    \anchor{left}{%
        \northwest
        \pgf@y=0pt
    }
    \anchor{right}{%
        \northwest
        \pgf@y=0pt
        \pgf@x=-\pgf@x
    }
    \anchor{top}{
        \northwest
        \pgf@x=0pt
    }
    \anchor{pathstart}{
        \northwest
        \pgf@x=0pt
    }
    \anchor{pathend}{
        \northwest
        \pgf@x=0pt
        \pgf@y=-\pgf@y
    }
    \anchor{bottom}{
        \northwest
        \pgf@x=0pt
        \pgf@y=-\pgf@y
    }
    \anchor{center}{
        \pgf@y=0pt
        \pgf@x=0pt
    }
    \anchor{east}{
        \northwest
        \pgf@y=0pt
        \pgf@x=-\pgf@x
    }
    \anchor{west}{
        \northwest
        \pgf@y=0pt
    }
    \anchor{south}{
        \northwest
        \pgf@x=0pt
        \pgf@y=-\pgf@y
    }
    \anchor{north}{
        \northwest
        \pgf@x=0pt
    }
    \anchor{south west}{ \northwest \pgf@y=-\pgf@y }
    \anchor{north east}{ \northwest \pgf@x=-\pgf@x }
    \anchor{north west}{ \northwest }
    \anchor{south east}{ \northwest \pgf@x=-\pgf@x \pgf@y=-\pgf@y }
    \backgroundpath{
        \pgfsetcolor{\ctikzvalof{color}}
        \pgf@circ@scaled@Rlen=\scaledRlen
        \pgf@circ@res@step=\ctikzvalof{tripoles/magnetron/width}\pgf@circ@scaled@Rlen
        \northwest
        \pgf@circ@res@up = \pgf@y
        \pgf@circ@res@down = -\pgf@y
        \pgf@circ@res@right = -\pgf@x
        \pgf@circ@res@left = \pgf@x
        \pgfmathsetlength{\pgf@circ@res@other}{sin(15)*\pgf@circ@res@up}

        \pgfscope
            \pgfstartlinewidth=\pgflinewidth
            \pgf@circ@setlinewidth{bipoles}{\pgfstartlinewidth}
            %create outer circle
            \pgfpathcircle{\pgfpoint{0}{0}} {\pgf@circ@res@right}
            \pgf@circ@draworfill
            %create chambers
            \foreach \angle in {45,135,225,315}{
                \pgfpathmoveto{ \pgfpointpolar{\angle}{0.6\pgf@circ@res@right}}
                \pgfpathlineto{ \pgfpointpolar{\angle}{\pgf@circ@res@right}}
            }
            \pgfsetroundcap
            \pgfusepath{draw}
            \pgfscope
                %draw connection from outside
                %anode
                \pgfsetlinewidth{\pgfstartlinewidth}
                \pgfpathmoveto{\pgfpoint{0\pgf@circ@res@left}{\pgf@circ@res@down}}
                \pgfpathlineto{\pgfpoint{0\pgf@circ@res@right}{.5\pgf@circ@res@down}}
                %cathodes
                \pgfpathmoveto{\pgfpointpolar{105}{\pgf@circ@res@up}}
                \pgfpathlineto{\pgfpoint{-\pgf@circ@res@other}{.5\pgf@circ@res@up}}
                \pgfpathmoveto{\pgfpointpolar{75}{\pgf@circ@res@up}}
                \pgfpathlineto{\pgfpoint{\pgf@circ@res@other}{.5\pgf@circ@res@up}}
                \pgfsetbuttcap
                \pgfusepath{draw}
            \endpgfscope
            %create cathode
            \pgfpathmoveto{\pgfpoint{\pgf@circ@res@other}{.5\pgf@circ@res@up}}
            \pgfpathlineto{\pgfpoint{0}{.15\pgf@circ@res@up}}
            \pgfpathlineto{\pgfpoint{-\pgf@circ@res@other}{.5\pgf@circ@res@up}}
            \pgfusepath{draw}
            %create anode
            \pgfpathmoveto{\pgfpoint{0.3\pgf@circ@res@left}{.5\pgf@circ@res@down}}
            \pgfpathlineto{\pgfpoint{0.3\pgf@circ@res@right}{.5\pgf@circ@res@down}}

            \pgfsetbuttcap
            \pgfusepath{draw}
        \endpgfscope
    }
}

%%%%%%%%%%%%%%%%%%%%%%%%%%%%%%%%%%%%%%%%%%%%%%%%%
% Electronic tubes, submitted by J. op den Brouw
%%%%%%%%%%%%%%%%%%%%%%%%%%%%%%%%%%%%%%%%%%%%%%%%%

% Draw tube outline
\def\pgf@circ@tubes@drawtube{%
	\ifdim\ctikzvalof{tubes/width}pt>\ctikzvalof{tubes/height}pt\relax
	\pgfpathmoveto{\pgfpoint{0pt}{\pgf@circ@res@up}}
	\pgfutil@tempdima=\pgf@circ@res@right
	\advance\pgfutil@tempdima by -\pgf@circ@res@up
	\pgfpathlineto{\pgfpoint{\pgfutil@tempdima}{\pgf@circ@res@up}}
	\pgfpatharc{90}{-90}{\pgf@circ@res@up}
	\pgfpathlineto{\pgfpoint{-\pgfutil@tempdima}{-\pgf@circ@res@up}}
	\pgfpatharc{270}{90}{\pgf@circ@res@up}
	\else
	\pgfpathmoveto{\pgfpoint{-\pgf@circ@res@right}{0pt}}
	\pgfutil@tempdima=\pgf@circ@res@up
	\advance\pgfutil@tempdima by -\pgf@circ@res@right
	\pgfpathlineto{\pgfpoint{-\pgf@circ@res@right}{\pgfutil@tempdima}}
	\pgfpatharc{180}{0}{\pgf@circ@res@right}
	\pgfpathlineto{\pgfpoint{\pgf@circ@res@right}{-\pgfutil@tempdima}}
	\pgfpatharc{180}{0}{-\pgf@circ@res@right}
	\fi
	\pgfpathclose
}

%% The diode (tube), triode, tetrode and pentode only differ in the
%% number of grids. So we construct a generic declare function in
%% which we can put code for the grid anchors and grid drawing code
%% \pgfcircdeclaretube{tube name}{grid anchors}{grid drawing code}
\long\def\pgfcircdeclaretube#1#2#3{%
    \pgfdeclareshape{#1}{
        \savedmacro{\ctikzclass}{\edef\ctikzclass{tubes}}
        \saveddimen{\scaledRlen}{\pgfmathsetlength{\pgf@x}{\ctikzvalof{\ctikzclass/scale}\pgf@circ@Rlen}}
        \anchor{center}{
            \pgfpointorigin
        }
        \savedanchor\northwest{%
            \pgfmathsetlength{\pgf@circ@scaled@Rlen}{\ctikzvalof{\ctikzclass/scale}\pgf@circ@Rlen}
            \pgf@circ@res@up=\ctikzvalof{tubes/height}\pgf@circ@scaled@Rlen
            \pgf@circ@res@right=\ctikzvalof{tubes/width}\pgf@circ@scaled@Rlen
            % x and y should be half the Rlen
            \pgf@y=\pgf@circ@res@up
            \pgf@y=.5\pgf@y
            \pgf@x=-\pgf@circ@res@right
            \pgf@x=.5\pgf@x
        }
        \anchor{north} {%
            \northwest
            \pgf@x=0pt
        }
        \anchor{east}{%
            \northwest
            \pgf@x=-\pgf@x
            \pgf@y=0pt
        }
        \anchor{south}{%
            \northwest
            \pgf@y=-\pgf@y
            \pgf@x=0pt
        }
        \anchor{west}{%
            \northwest
            \pgf@y=0pt
        }
        \anchor{north west}{%
            \northwest
        }
        \anchor{north east}{%
            \northwest
            \pgf@x=-\pgf@x
        }
        \anchor{south east}{
            \northwest
            \pgf@x=-\pgf@x
            \pgf@y=-\pgf@y
        }
        \anchor{south west}{
            \northwest
            \pgf@y=-\pgf@y
        }
        \anchor{anode} {%
            \northwest
            \pgf@x=0pt
        }
        \anchor{cathode}{%
            \northwest
            \pgf@y=-\pgf@y
            \pgf@x=\ctikzvalof{tubes/cathode width}\pgf@x
        }
        \anchor{cathode 1}{%
            \northwest
            \pgf@y=-\pgf@y
            \pgf@x=\ctikzvalof{tubes/cathode width}\pgf@x
        }
        \anchor{cathode 2}{%
            \northwest
            \pgf@y=-\pgf@y
            \pgf@x=-\ctikzvalof{tubes/cathode width}\pgf@x
        }
        \anchor{filament 1}{%
            \northwest
            \pgfmathparse{(\ctikzvalof{tubes/tube radius}*sin(\ctikzvalof{tubes/filament angle})}
            \pgf@x=\pgfmathresult\pgf@x
            \pgf@y=-\pgf@y
        }
        \anchor{filament 2}{%
            \northwest
            \pgfmathparse{(\ctikzvalof{tubes/tube radius}*sin(\ctikzvalof{tubes/filament angle})}
            \pgf@x=-\pgfmathresult\pgf@x
            \pgf@y=-\pgf@y
        }

        % Extra anchors
        #2

        \backgroundpath{
            \pgfscope
                % Line width for tripoles
                \pgf@circ@setlinewidth{tripoles}{\pgflinewidth}
                \pgf@circ@scaled@Rlen=\scaledRlen

                % Setup to draw tube
                \pgf@circ@res@up=\ctikzvalof{tubes/height}\pgf@circ@scaled@Rlen
                \pgf@circ@res@right=\ctikzvalof{tubes/width}\pgf@circ@scaled@Rlen
                \pgf@circ@res@up=\ctikzvalof{tubes/tube radius}\pgf@circ@res@up
                \pgf@circ@res@right=\ctikzvalof{tubes/tube radius}\pgf@circ@res@right

                % Tube outline
                \pgf@circ@tubes@drawtube

                % Setup to draw grid, filament, anode and cathode
                \pgf@circ@res@up=\ctikzvalof{tubes/height}\pgf@circ@scaled@Rlen
                \pgf@circ@res@right=\ctikzvalof{tubes/width}\pgf@circ@scaled@Rlen
                \pgf@circ@res@up=0.5\pgf@circ@res@up
                \pgf@circ@res@right=0.5\pgf@circ@res@right

                % Tube fill color (if any)
                \pgf@circ@draworfill

                % Grid drawing
                #3

                % Filament (is not drawn by default)
                \ifpgf@circuit@tubes@filament
                    \pgf@circ@res@temp=-\ctikzvalof{tubes/cathode distance}\pgf@circ@res@up
                    \advance\pgf@circ@res@temp by -\ctikzvalof{tubes/filament distance}\pgf@circ@res@up
                    \pgfmathparse{(\ctikzvalof{tubes/tube radius}*sin(\ctikzvalof{tubes/filament angle})}
                    \pgf@xa=\pgfmathresult\pgf@circ@res@right
                    \pgfmathparse{\ctikzvalof{tubes/tube radius}+\ctikzvalof{tubes/tube radius}*cos(\ctikzvalof{tubes/filament angle}}
                    \pgf@ya=\pgfmathresult\pgf@circ@res@up
                    \pgfpathmoveto{\pgfpoint{0pt}{\pgf@circ@res@temp}}
                    \pgfpathlineto{\pgfpoint{-\pgf@xa}{-\pgf@ya}}
                    \pgfpathlineto{\pgfpoint{-\pgf@xa}{-\pgf@circ@res@up}}
                    \pgfpathmoveto{\pgfpoint{0pt}{\pgf@circ@res@temp}}
                    \pgfpathlineto{\pgfpoint{\pgf@xa}{-\pgf@ya}}
                    \pgfpathlineto{\pgfpoint{\pgf@xa}{-\pgf@circ@res@up}}
                    \pgf@circuit@tubes@filamentfalse
                \fi

                % Anode (or plate)
                \pgfpathmoveto{\pgfpoint{0pt}{\pgf@circ@res@up}} % north
                \pgfpathlineto{\pgfpoint{0pt}{\ctikzvalof{tubes/anode distance}\pgf@circ@res@up}}
                \pgfpathmoveto{\pgfpoint{-\ctikzvalof{tubes/anode width}\pgf@circ@res@right}{\ctikzvalof{tubes/anode distance}\pgf@circ@res@up}}
                \pgfpathlineto{\pgfpoint{\ctikzvalof{tubes/anode width}\pgf@circ@res@right}{\ctikzvalof{tubes/anode distance}\pgf@circ@res@up}}

                % Cathode
                \ifpgf@circuit@tubes@nocathode
                    \pgf@circuit@tubes@nocathodefalse
                \else
                    \pgfsetcornersarced{\pgfpoint{\ctikzvalof{tubes/cathode corners}\pgf@circ@res@up}{\ctikzvalof{tubes/cathode corners}\pgf@circ@res@up}}
                    \pgfpathmoveto{\pgfpoint{-\ctikzvalof{tubes/cathode width}\pgf@circ@res@right}{-\pgf@circ@res@up}}
                    \pgfpathlineto{\pgfpoint{-\ctikzvalof{tubes/cathode width}\pgf@circ@res@right}{-\ctikzvalof{tubes/cathode distance}\pgf@circ@res@up}}
                    \pgfpathlineto{\pgfpoint{\ctikzvalof{tubes/cathode width}\pgf@circ@res@right}{-\ctikzvalof{tubes/cathode distance}\pgf@circ@res@up}}
                    \ifpgf@circuit@tubes@fullcathode
                        \pgfpathlineto{\pgfpoint{\ctikzvalof{tubes/cathode width}\pgf@circ@res@right}{-\pgf@circ@res@up}}
                        \pgf@circuit@tubes@fullcathodefalse
                    \else
                        \pgfpathlineto{\pgfpoint{\ctikzvalof{tubes/cathode width}\pgf@circ@res@right}{-\ctikzvalof{tubes/cathode distance}\pgf@circ@res@up-\ctikzvalof{tubes/cathode right extend}\pgf@circ@res@up}}
                    \fi
                \fi

                % Draw the background
                \pgfusepath{draw}
            \endpgfscope
        }
    }
}

\pgfcircdeclaretube{diodetube}{}{} % shape diode already exists

\pgfcircdeclaretube{triode}
{
	\anchor{grid} {% should not be used
		\northwest
		\pgf@y=\ctikzvalof{tubes/grid shift}\pgf@y
	}
	\anchor{control} {%
		\northwest
		\pgf@y=\ctikzvalof{tubes/grid shift}\pgf@y
	}
}
{
	% Grid protrusion
	\pgf@xa=-\ctikzvalof{tubes/tube radius}\pgf@circ@res@right
	\advance\pgf@xa by -\ctikzvalof{tubes/grid protrusion}\pgf@circ@res@right
	\pgfpathmoveto{\pgfpoint{-\pgf@circ@res@right}{\ctikzvalof{tubes/grid shift}\pgf@circ@res@up}}
	\pgfpathlineto{\pgfpoint{\pgf@xa}{\ctikzvalof{tubes/grid shift}\pgf@circ@res@up}}
	% Grid dashes: calculations
	\pgf@xb=2\pgf@circ@res@right
	\pgf@circ@res@step=\ctikzvalof{tubes/tube radius}\pgf@xb
	\pgf@circ@count@a=\ctikzvalof{tubes/grid dashes}  % dashes*2+1
	\multiply\pgf@circ@count@a by 2\relax
	\advance\pgf@circ@count@a by 1\relax
	\advance\pgf@circ@res@step by -\pgf@xa
	\divide\pgf@circ@res@step by \pgf@circ@count@a
	% Grid dashes: draw
	\pgf@circ@res@temp=\pgf@xa
	\pgf@circ@count@a=\ctikzvalof{tubes/grid dashes}
	\loop
	\advance\pgf@circ@res@temp by\pgf@circ@res@step
	\pgfpathmoveto{\pgfpoint{\pgf@circ@res@temp}{\ctikzvalof{tubes/grid shift}\pgf@circ@res@up}}
	\advance\pgf@circ@res@temp by\pgf@circ@res@step
	\pgfpathlineto{\pgfpoint{\pgf@circ@res@temp}{\ctikzvalof{tubes/grid shift}\pgf@circ@res@up}}
	\advance\pgf@circ@count@a by-1
	\ifnum\pgf@circ@count@a>0\relax
	\repeat
}

\pgfcircdeclaretube{tetrode}
{
	\anchor{grid} {% should not be used
		\northwest
		\pgfutil@tempdima=\pgf@y
		\pgf@y=-\ctikzvalof{tubes/grid separation}\pgf@y
		\pgf@y=0.5\pgf@y
		\advance\pgf@y by \ctikzvalof{tubes/grid shift}\pgfutil@tempdima
	}
	\anchor{control} {%
		\northwest
		\pgfutil@tempdima=\pgf@y
		\pgf@y=-\ctikzvalof{tubes/grid separation}\pgf@y
		\pgf@y=0.5\pgf@y
		\advance\pgf@y by \ctikzvalof{tubes/grid shift}\pgfutil@tempdima
	}
	\anchor{screen} {%
		\northwest
		\pgfutil@tempdima=\pgf@y
		\pgf@y=\ctikzvalof{tubes/grid separation}\pgf@y
		\pgf@y=0.5\pgf@y
		\advance\pgf@y by \ctikzvalof{tubes/grid shift}\pgfutil@tempdima
	}
}
{
	% Grid x/y points
	\pgf@xa=-\ctikzvalof{tubes/tube radius}\pgf@circ@res@right
	\advance\pgf@xa by -\ctikzvalof{tubes/grid protrusion}\pgf@circ@res@right
	\pgfutil@tempdima=\ctikzvalof{tubes/grid separation}\pgf@circ@res@up
	\pgfutil@tempdimb=-\pgfutil@tempdima
	\pgfutil@tempdima=0.5\pgfutil@tempdima
	\advance\pgfutil@tempdima by \ctikzvalof{tubes/grid shift}\pgf@circ@res@up
	\pgfutil@tempdimb=0.5\pgfutil@tempdimb
	\advance\pgfutil@tempdimb by \ctikzvalof{tubes/grid shift}\pgf@circ@res@up
	% Grid protrusion
	\pgfpathmoveto{\pgfpoint{-\pgf@circ@res@right}{\pgfutil@tempdimb}}
	\pgfpathlineto{\pgfpoint{\pgf@xa}{\pgfutil@tempdimb}}
	\pgfpathmoveto{\pgfpoint{-\pgf@circ@res@right}{\pgfutil@tempdima}}
	\pgfpathlineto{\pgfpoint{\pgf@xa}{\pgfutil@tempdima}}
	% Grid dashes: calculations
	\pgf@xb=2\pgf@circ@res@right
	\pgf@circ@res@step=\ctikzvalof{tubes/tube radius}\pgf@xb
	\pgf@circ@count@a=\ctikzvalof{tubes/grid dashes}  % dashes*2+1
	\multiply\pgf@circ@count@a by 2\relax
	\advance\pgf@circ@count@a by 1\relax
	\advance\pgf@circ@res@step by -\pgf@xa
	\divide\pgf@circ@res@step by \pgf@circ@count@a
	% Grid dashes: draw
	\pgf@circ@res@temp=\pgf@xa
	\pgf@circ@count@a=\ctikzvalof{tubes/grid dashes}
	\loop
	\advance\pgf@circ@res@temp by\pgf@circ@res@step
	\pgfpathmoveto{\pgfpoint{\pgf@circ@res@temp}{\pgfutil@tempdima}}
	\pgfpathlineto{\pgfpoint{\pgf@circ@res@temp+\pgf@circ@res@step}{\pgfutil@tempdima}}
	\pgfpathmoveto{\pgfpoint{\pgf@circ@res@temp}{\pgfutil@tempdimb}}
	\pgfpathlineto{\pgfpoint{\pgf@circ@res@temp+\pgf@circ@res@step}{\pgfutil@tempdimb}}
	\advance\pgf@circ@res@temp by\pgf@circ@res@step
	\advance\pgf@circ@count@a by-1
	\ifnum\pgf@circ@count@a>0\relax
	\repeat
}

\pgfcircdeclaretube{pentode}
{
	\anchor{grid} {% should not be used
		\northwest
		\pgfutil@tempdima=\pgf@y
		\pgf@y=-\ctikzvalof{tubes/grid separation}\pgf@y
		\advance\pgf@y by \ctikzvalof{tubes/grid shift}\pgfutil@tempdima
	}
	\anchor{control} {%
		\northwest
		\pgfutil@tempdima=\pgf@y
		\pgf@y=-\ctikzvalof{tubes/grid separation}\pgf@y
		\advance\pgf@y by \ctikzvalof{tubes/grid shift}\pgfutil@tempdima
	}
	\anchor{screen} {%
		\northwest
		\pgf@y=\ctikzvalof{tubes/grid shift}\pgf@y
	}
	\anchor{suppressor} {%
		\northwest
		\pgfutil@tempdima=\pgf@y
		\pgf@y=\ctikzvalof{tubes/grid separation}\pgf@y
		\advance\pgf@y by \ctikzvalof{tubes/grid shift}\pgfutil@tempdima
	}
}
{
	% Grid x/y points
	\pgf@xa=-\ctikzvalof{tubes/tube radius}\pgf@circ@res@right
	\advance\pgf@xa by -\ctikzvalof{tubes/grid protrusion}\pgf@circ@res@right
	\pgfutil@tempdima=\ctikzvalof{tubes/grid separation}\pgf@circ@res@up
	\pgfutil@tempdimb=-\pgfutil@tempdima
	\advance\pgfutil@tempdima by \ctikzvalof{tubes/grid shift}\pgf@circ@res@up
	\advance\pgfutil@tempdimb by \ctikzvalof{tubes/grid shift}\pgf@circ@res@up
	\pgf@circ@res@other=\ctikzvalof{tubes/grid shift}\pgf@circ@res@up
	% Grid protrusion
	\pgfpathmoveto{\pgfpoint{-\pgf@circ@res@right}{\pgfutil@tempdimb}}
	\pgfpathlineto{\pgfpoint{\pgf@xa}{\pgfutil@tempdimb}}
	\pgfpathmoveto{\pgfpoint{-\pgf@circ@res@right}{\pgfutil@tempdima}}
	\pgfpathlineto{\pgfpoint{\pgf@xa}{\pgfutil@tempdima}}
	\pgfpathmoveto{\pgfpoint{-\pgf@circ@res@right}{\pgf@circ@res@other}}
	\pgfpathlineto{\pgfpoint{\pgf@xa}{\pgf@circ@res@other}}
	% Grid dashes: calculations
	\pgf@xb=2\pgf@circ@res@right
	\pgf@circ@res@step=\ctikzvalof{tubes/tube radius}\pgf@xb
	\pgf@circ@count@a=\ctikzvalof{tubes/grid dashes}  % dashes*2+1
	\multiply\pgf@circ@count@a by 2\relax
	\advance\pgf@circ@count@a by 1\relax
	\advance\pgf@circ@res@step by -\pgf@xa
	\divide\pgf@circ@res@step by \pgf@circ@count@a
	% Grid dashes: draw
	\pgf@circ@res@temp=\pgf@xa
	\pgf@circ@count@a=\ctikzvalof{tubes/grid dashes}
	\loop
	\advance\pgf@circ@res@temp by\pgf@circ@res@step
	\pgfpathmoveto{\pgfpoint{\pgf@circ@res@temp}{\pgfutil@tempdima}}
	\pgfpathlineto{\pgfpoint{\pgf@circ@res@temp+\pgf@circ@res@step}{\pgfutil@tempdima}}
	\pgfpathmoveto{\pgfpoint{\pgf@circ@res@temp}{\pgfutil@tempdimb}}
	\pgfpathlineto{\pgfpoint{\pgf@circ@res@temp+\pgf@circ@res@step}{\pgfutil@tempdimb}}
	\pgfpathmoveto{\pgfpoint{\pgf@circ@res@temp}{\pgf@circ@res@other}}
	\pgfpathlineto{\pgfpoint{\pgf@circ@res@temp+\pgf@circ@res@step}{\pgf@circ@res@other}}
	\advance\pgf@circ@res@temp by\pgf@circ@res@step
	\advance\pgf@circ@count@a by-1
	\ifnum\pgf@circ@count@a>0\relax
	\repeat
}

\pgfcircdeclaretube{pentode suppressor to cathode}
{
	\anchor{grid} {% should not be used
		\northwest
		\pgfutil@tempdima=\pgf@y
		\pgf@y=-\ctikzvalof{tubes/grid separation}\pgf@y
		\advance\pgf@y by \ctikzvalof{tubes/grid shift}\pgfutil@tempdima
	}
	\anchor{control} {%
		\northwest
		\pgfutil@tempdima=\pgf@y
		\pgf@y=-\ctikzvalof{tubes/grid separation}\pgf@y
		\advance\pgf@y by \ctikzvalof{tubes/grid shift}\pgfutil@tempdima
	}
	\anchor{screen} {%
		\northwest
		\pgf@y=\ctikzvalof{tubes/grid shift}\pgf@y
	}
}
{
	% Grid x/y points
	\pgf@xa=-\ctikzvalof{tubes/tube radius}\pgf@circ@res@right
	\advance\pgf@xa by -\ctikzvalof{tubes/grid protrusion}\pgf@circ@res@right
	\pgfutil@tempdima=\ctikzvalof{tubes/grid separation}\pgf@circ@res@up
	\pgfutil@tempdimb=-\pgfutil@tempdima
	\advance\pgfutil@tempdima by \ctikzvalof{tubes/grid shift}\pgf@circ@res@up
	\advance\pgfutil@tempdimb by \ctikzvalof{tubes/grid shift}\pgf@circ@res@up
	\pgf@circ@res@other=\ctikzvalof{tubes/grid shift}\pgf@circ@res@up
	% Grid protrusion
	\pgfpathmoveto{\pgfpoint{-\pgf@circ@res@right}{\pgfutil@tempdimb}}
	\pgfpathlineto{\pgfpoint{\pgf@xa}{\pgfutil@tempdimb}}
	\pgfpathmoveto{\pgfpoint{-\pgf@circ@res@right}{\pgf@circ@res@other}}
	\pgfpathlineto{\pgfpoint{\pgf@xa}{\pgf@circ@res@other}}
	% Grid dashes: calculations
	\pgf@xb=2\pgf@circ@res@right
	\pgf@circ@res@step=\ctikzvalof{tubes/tube radius}\pgf@xb
	\pgf@circ@count@a=\ctikzvalof{tubes/grid dashes}  % dashes*2+1
	\multiply\pgf@circ@count@a by 2\relax
	\advance\pgf@circ@count@a by 1\relax
	\advance\pgf@circ@res@step by -\pgf@xa
	\divide\pgf@circ@res@step by \pgf@circ@count@a
	% Grid dashes: draw
	\pgf@circ@res@temp=\pgf@xa
	\pgf@circ@count@a=\ctikzvalof{tubes/grid dashes}
	\loop
	\advance\pgf@circ@res@temp by\pgf@circ@res@step
	\ifnum\pgf@circ@count@a>1\relax
	\pgfpathmoveto{\pgfpoint{\pgf@circ@res@temp}{\pgfutil@tempdimb}}
	\pgfpathlineto{\pgfpoint{\pgf@circ@res@temp+\pgf@circ@res@step}{\pgfutil@tempdimb}}
	\pgfpathmoveto{\pgfpoint{\pgf@circ@res@temp}{\pgf@circ@res@other}}
	\pgfpathlineto{\pgfpoint{\pgf@circ@res@temp+\pgf@circ@res@step}{\pgf@circ@res@other}}
	\fi
	\pgfpathmoveto{\pgfpoint{\pgf@circ@res@temp}{\pgfutil@tempdima}}
	\pgfpathlineto{\pgfpoint{\pgf@circ@res@temp+\pgf@circ@res@step}{\pgfutil@tempdima}}
	\advance\pgf@circ@res@temp by\pgf@circ@res@step
	\advance\pgf@circ@count@a by-1
	\ifnum\pgf@circ@count@a>0\relax
	\repeat
	% Grid: connection from suppressor to cathode
	\pgfsetcornersarced{\pgfpoint{\ctikzvalof{tubes/cathode corners}\pgf@circ@res@up}{\ctikzvalof{tubes/cathode corners}\pgf@circ@res@up}}
	\pgfpathlineto{\pgfpoint{\pgf@circ@res@temp}{\pgfutil@tempdima-2*\ctikzvalof{tubes/grid separation}\pgf@circ@res@up}}
	\pgfpathlineto{\pgfpoint{\ctikzvalof{tubes/cathode width}\pgf@circ@res@right-0.4142136*\ctikzvalof{tubes/cathode corners}\pgf@circ@res@right}{-\ctikzvalof{tubes/cathode distance}\pgf@circ@res@up-0.4142136*\ctikzvalof{tubes/cathode corners}\pgf@circ@res@up}}

}

%%%---------- close: tex/pgfcirctripoles
%%%%%%%%%%% Springe nach tex/pgfcircquadpoles
%%%---------- open: tex/pgfcircquadpoles.tex
% Copyright 2018-2020 by Romano Giannetti
% Copyright 2015-2020 by Stefan Lindner
% Copyright 2013-2020 by Stefan Erhardt
% Copyright 2007-2020 by Massimo Redaelli
%
% This file may be distributed and/or modified
%
% 1. under the LaTeX Project Public License and/or
% 2. under the GNU Public License.
%
% See the files gpl-3.0_license.txt and lppl-1-3c_license.txt for more details.

%%%%%%%%%%%%%%%%%%%%%%%%%%%%%%%%%%%%%%
%% Quadripoles

\long\def\pgfcircdeclarequadpole#1#2#3{
    \pgfdeclareshape{#1}
    {
        \savedmacro{\ctikzclass}{\edef\ctikzclass{inductors}}
        \saveddimen{\scaledRlen}{\pgfmathsetlength{\pgf@x}{\ctikzvalof{\ctikzclass/scale}\pgf@circ@Rlen}}
        % shapename
        \savedmacro{\thisshape}{\def\thisshape{\tikz@fig@name}}
        \anchor{center}{
            \northwest
            \pgf@x=0pt
            \pgf@y=0pt
        }
        \savedmacro{\stretto}{\def\stretto{\ctikzvalof{quadpoles/#1/inner}}}
        \savedanchor\northwest{%
            \pgfmathsetlength{\pgf@circ@scaled@Rlen}{\ctikzvalof{\ctikzclass/scale}\pgf@circ@Rlen}
            \pgf@y=\ctikzvalof{quadpoles/#1/height}\pgf@circ@scaled@Rlen
            \pgf@y=.5\pgf@y
            \pgf@x=.5\pgf@circ@scaled@Rlen
            \pgf@x=-\ctikzvalof{quadpoles/#1/width}\pgf@x
        }
        %% we define the upper right (positive coord) dot (which is B1)
        \savedanchor{\innerdot}{%
            \pgfmathsetlength{\pgf@circ@scaled@Rlen}{\ctikzvalof{\ctikzclass/scale}\pgf@circ@Rlen}
            \pgf@xa=.5\pgf@circ@scaled@Rlen
            \pgf@xa=-\ctikzvalof{quadpoles/#1/width}\pgf@xa
            % by default use the cute inductor size
            \pgfmathsetlength\pgf@x{-\stretto*\pgf@xa-\ctikzvalof{bipoles/cuteinductor/height}*\pgf@circ@scaled@Rlen/2}
            % check if it's american
            \edef\pgf@circ@temp{\ctikzvalof{inductor}}
            \edef\pgf@temp{american}
            \ifx\pgf@circ@temp\pgf@temp
                \pgfmathsetlength\pgf@x{-\stretto*\pgf@xa-\ctikzvalof{bipoles/americaninductor/height}*\pgf@circ@scaled@Rlen/2}
            \fi
            % check if it's european
            \edef\pgf@temp{european}
            \ifx\pgf@circ@temp\pgf@temp
                \pgfmathsetlength\pgf@x{-\stretto*\pgf@xa-\ctikzvalof{bipoles/fullgeneric/height}*\pgf@circ@scaled@Rlen/2}
            \fi
            \pgfmathsetlength\pgf@y{0.5*\pgf@circ@scaled@Rlen}
        }
        \savedanchor{\outerdot}{%
            \pgfmathsetlength{\pgf@circ@scaled@Rlen}{\ctikzvalof{\ctikzclass/scale}\pgf@circ@Rlen}
            \pgf@xa=.5\pgf@circ@scaled@Rlen
            \pgf@xa=-\ctikzvalof{quadpoles/#1/width}\pgf@xa
            % by default use the cute inductor size
            \pgfmathsetlength\pgf@x{-\stretto*\pgf@xa+\ctikzvalof{bipoles/cuteinductor/height}*\pgf@circ@scaled@Rlen/2}
            % check if it's american
            \edef\pgf@circ@temp{\ctikzvalof{inductor}}
            \edef\pgf@temp{american}
            \ifx\pgf@circ@temp\pgf@temp
                \pgfmathsetlength\pgf@x{-\stretto*\pgf@xa+\ctikzvalof{bipoles/americaninductor/height}*\pgf@circ@scaled@Rlen/2}
            \fi
            % check if it's european
            \edef\pgf@temp{european}
            \ifx\pgf@circ@temp\pgf@temp
                \pgfmathsetlength\pgf@x{-\stretto*\pgf@xa+\ctikzvalof{bipoles/fullgeneric/height}*\pgf@circ@scaled@Rlen/2}
            \fi
            \pgfmathsetlength\pgf@y{0.5*\pgf@circ@scaled@Rlen}
        }
        \anchor{A2}{
            \northwest
            \pgf@y=-\pgf@y
        }
        \anchor{B1}{
            \northwest
            \pgf@x=-\pgf@x
        }
        \anchor{A1}{
            \northwest
        }
        \anchor{B2}{
            \northwest
            \pgf@x=-\pgf@x
            \pgf@y=-\pgf@y
        }
        %% dot's anchors
        \anchor{inner dot A1}{\innerdot\pgf@x=-\pgf@x}
        \anchor{outer dot A1}{\outerdot\pgf@x=-\pgf@x}
        \anchor{inner dot A2}{\innerdot\pgf@x=-\pgf@x\pgf@y=-\pgf@y}
        \anchor{outer dot A2}{\outerdot\pgf@x=-\pgf@x\pgf@y=-\pgf@y}
        \anchor{inner dot B1}{\innerdot}
        \anchor{outer dot B1}{\outerdot}
        \anchor{inner dot B2}{\innerdot\pgf@y=-\pgf@y}
        \anchor{outer dot B2}{\outerdot\pgf@y=-\pgf@y}
        % geographical
        \anchor{north}{
            \northwest
            \pgf@x=0pt
        }
        \anchor{south}{
            \northwest
            \pgf@x=0pt
            \pgf@y=-\pgf@y
        }
        \anchor{west}{
            \northwest
            \pgf@y=0pt
        }
        \anchor{east}{
            \northwest
            \pgf@y=0pt
            \pgf@x=-\pgf@x
        }
        \anchor{south west}{
            \northwest
            \pgf@y=-\pgf@y
        }
        \anchor{north east}{
            \northwest
            \pgf@x=-\pgf@x
        }
        \anchor{north west}{
            \northwest
        }
        \anchor{south east}{
            \northwest
            \pgf@x=-\pgf@x
            \pgf@y=-\pgf@y
        }
        \anchor{base}{
            \northwest
            \pgf@x=0pt
        }
        #3
        \backgroundpath{
            \pgfsetcolor{\ctikzvalof{color}}

            \northwest
            \pgf@circ@res@up = \pgf@y
            \pgf@circ@res@down = -\pgf@y
            \pgf@circ@res@right = -\pgf@x
            \pgf@circ@res@left = \pgf@x
            #2
        }
    }
}


\def\pgf@circ@drawtransformerbasicanchor{
    \ctikzvalof{quadpoles/trans/height}
    \anchor{AA2}{
        \northwest
        \pgf@x=\ctikzvalof{quadpoles/transformer/width1}\pgf@x
        \pgf@x=.7\pgf@x
        \pgf@y=-\pgf@y
        \pgf@y=\ctikzvalof{quadpoles/transformer/height1}\pgf@y
    }
    \anchor{BB1}{
        \northwest
        \pgf@x=-\pgf@x
        \pgf@x=\ctikzvalof{quadpoles/transformer/width1}\pgf@x
        \pgf@x=.7\pgf@x
        \pgf@y=\ctikzvalof{quadpoles/transformer/height1}\pgf@y
    }
    \anchor{AA1}{
        \northwest
        \pgf@x=\ctikzvalof{quadpoles/transformer/width1}\pgf@x
        \pgf@x=.7\pgf@x
        \pgf@y=\ctikzvalof{quadpoles/transformer/height1}\pgf@y
    }
    \anchor{BB2}{
        \northwest
        \pgf@x=-\pgf@x
        \pgf@x=\ctikzvalof{quadpoles/transformer/width1}\pgf@x
        \pgf@x=.7\pgf@x
        \pgf@y=-\pgf@y
        \pgf@y=\ctikzvalof{quadpoles/transformer/height1}\pgf@y
    }
}

%% Null styles that can be used to change individually the L1 and L2
%% inductors of the transformer.

\ctikzset{transformer L1/.style={}}
\ctikzset{transformer L2/.style={}}

\def\pgf@circ@drawtransformerbasicbody{
    \pgfscope
        \pgfslopedattimetrue
        \pgfallowupsidedownattimetrue
        \pgftransformlineattime{.5}{%
            \pgfpoint%
            {\stretto\pgf@circ@res@left}%
            {\pgf@circ@res@up}%
            }{%
            \pgfpoint
            {\stretto\pgf@circ@res@left}%
            {\pgf@circ@res@down}%
        }

        \pgfkeys{\circuitikzbasekey/.cd, transformer L1}
        \edef\pgf@circ@temp{\ctikzvalof{inductor}}%
        \def\pgf@temp{european}%
        \ifx\pgf@temp\pgf@circ@temp%
            \pgfnode{fullgenericshape}{center}{}{\thisshape-L1}{\pgfusepath{stroke}}
        \else%
            \def\pgf@temp{cute}
            \ifx\pgf@temp\pgf@circ@temp%
                \pgfnode{cuteinductorshape}{center}{}{\thisshape-L1}{\pgfusepath{stroke}}
            \else%
                \pgfnode{americaninductorshape}{center}{}{\thisshape-L1}{\pgfusepath{stroke}}
            \fi%
        \fi%


    \endpgfscope
    \pgfscope
        \pgfslopedattimetrue
        \pgfallowupsidedownattimetrue
        \pgftransformlineattime{.5}{%
            \pgfpoint%
            {\stretto\pgf@circ@res@right}%
            {\pgf@circ@res@down}%
            }{%
            \pgfpoint
            {\stretto\pgf@circ@res@right}%
            {\pgf@circ@res@up}%
        }

        \pgfkeys{\circuitikzbasekey/.cd, transformer L2}
        \edef\pgf@circ@temp{\ctikzvalof{inductor}}%
        \def\pgf@temp{european}%
        \ifx\pgf@temp\pgf@circ@temp%
            \pgfnode{fullgenericshape}{center}{}{\thisshape-L2}{\pgfusepath{stroke}}
        \else%
            \def\pgf@temp{cute}
            \ifx\pgf@temp\pgf@circ@temp%
                \pgfnode{cuteinductorshape}{center}{}{\thisshape-L2}{\pgfusepath{stroke}}
            \else%
                \pgfnode{americaninductorshape}{center}{}{\thisshape-L2}{\pgfusepath{stroke}}
            \fi%
        \fi%

    \endpgfscope

    \pgfpathmoveto{\pgfpoint{\pgf@circ@res@left}{\pgf@circ@res@up}}
    \pgfpathlineto{\pgfpoint{\stretto\pgf@circ@res@left}{\pgf@circ@res@up}}
    \pgfpathlineto{\pgfpointanchor{\thisshape-L1}{b}}

    \pgfpathmoveto{\pgfpointanchor{\thisshape-L1}{a}}
    \pgfpathlineto{\pgfpoint{\stretto\pgf@circ@res@left}{\pgf@circ@res@down}}
    \pgfpathlineto{\pgfpoint{\pgf@circ@res@left}{\pgf@circ@res@down}}

    \pgfpathmoveto{\pgfpoint{\pgf@circ@res@right}{\pgf@circ@res@up}}
    \pgfpathlineto{\pgfpoint{\stretto\pgf@circ@res@right}{\pgf@circ@res@up}}
    \pgfpathlineto{\pgfpointanchor{\thisshape-L2}{a}}

    \pgfpathmoveto{\pgfpointanchor{\thisshape-L2}{b}}
    \pgfpathlineto{\pgfpoint{\stretto\pgf@circ@res@right}{\pgf@circ@res@down}}
    \pgfpathlineto{\pgfpoint{\pgf@circ@res@right}{\pgf@circ@res@down}}

    \pgfusepath{draw}

}


\pgfcircdeclarequadpole{transformer}{
    \pgf@circ@drawtransformerbasicbody
}{\pgf@circ@drawtransformerbasicanchor}

\pgfcircdeclarequadpole{transformer core}{

    \pgf@circ@drawtransformerbasicbody

    % use the chocke line thickness
    \pgfsetlinewidth{\ctikzvalof{bipoles/cutechoke/cthick}\pgflinewidth}

    % Find the distance from center for the lines representing the core
    % the 2.5 is for backward compatibility --- the distance was calculated as a fraction
    % of the whole component, now as a fraction of the distance between coils, to be
    % compatible with the quadpoles "inner" style.
    \pgfmathsetlength{\pgf@circ@res@other}{2.5*\stretto*\ctikzvalof{quadpoles/transformer core/core width}*\pgf@circ@res@right}

    \pgfmoveto{\pgfpoint%
        {\pgf@circ@res@other}%
        {\ctikzvalof{quadpoles/transformer core/core height}\pgf@circ@res@down}%
    }
    \pgflineto{
        \pgfpoint%
        {\pgf@circ@res@other}%
        {\ctikzvalof{quadpoles/transformer core/core height}\pgf@circ@res@up}%
    }

    %% this should be just -\pgf@circ@res@other, but in case someone define an asymmetric trafo someday...
    \pgfmathsetlength{\pgf@circ@res@other}{2.5*\stretto*\ctikzvalof{quadpoles/transformer core/core width}*\pgf@circ@res@left}
    \pgfmoveto{\pgfpoint%
        {\pgf@circ@res@other}%
        {\ctikzvalof{quadpoles/transformer core/core height}\pgf@circ@res@down}%
    }
    \pgflineto{
        \pgfpoint%
        {\pgf@circ@res@other}%
        {\ctikzvalof{quadpoles/transformer core/core height}\pgf@circ@res@up}%
    }

    \pgf@circ@setlinewidth{bipoles}{\pgflinewidth}
    \pgfusepath{draw}
}{\pgf@circ@drawtransformerbasicanchor}


\pgfcircdeclarequadpole{gyrator}{

    \pgfpathmoveto{\pgfpoint{\pgf@circ@res@left}{\pgf@circ@res@up}}
    \pgfpathlineto{\pgfpoint{\stretto\pgf@circ@res@left}{\pgf@circ@res@up}}
    \pgfpathlineto{\pgfpoint{\stretto\pgf@circ@res@left}{\pgf@circ@res@down}}
    \pgfpathlineto{\pgfpoint{\pgf@circ@res@left}{\pgf@circ@res@down}}

    \pgfpathmoveto{\pgfpoint{\pgf@circ@res@right}{\pgf@circ@res@up}}
    \pgfpathlineto{\pgfpoint{\stretto\pgf@circ@res@right}{\pgf@circ@res@up}}
    \pgfpathlineto{\pgfpoint{\stretto\pgf@circ@res@right}{\pgf@circ@res@down}}
    \pgfpathlineto{\pgfpoint{\pgf@circ@res@right}{\pgf@circ@res@down}}

    \pgfusepath{draw}

    \pgf@circ@setlinewidth{quadpoles}{\pgflinewidth}
    \pgfmathsetlength{\pgf@circ@res@other}{min(.7*\stretto*\pgf@circ@res@up, .8*\pgf@circ@res@right)} % radius
    \pgfpathmoveto{\pgfpoint{\stretto\pgf@circ@res@left}{-\pgf@circ@res@other}}
    \pgfpatharc{-90}{90}{\pgf@circ@res@other}
    \pgfpathclose
    \pgf@circ@draworfill

    \pgfpathmoveto{\pgfpoint{\stretto\pgf@circ@res@right}{\pgf@circ@res@other}}
    \pgfpatharc{90}{270}{\pgf@circ@res@other}
    \pgfpathclose
    \pgf@circ@draworfill
}{}

%%%%%%%%%%%%%%%%%%%%
%% Block elements
%%%%%%%%%%%%%%%%%%%%

\pgfdeclareshape{mixer}
{
    \savedmacro{\ctikzclass}{\edef\ctikzclass{blocks}}
    \saveddimen{\scaledRlen}{\pgfmathsetlength{\pgf@x}{\ctikzvalof{\ctikzclass/scale}\pgf@circ@Rlen}}
    \savedanchor\northwest{
        \pgfmathsetlength{\pgf@circ@scaled@Rlen}{\ctikzvalof{\ctikzclass/scale}\pgf@circ@Rlen}
        \ifpgf@circuit@boxed
            \pgf@y=\ctikzvalof{bipoles/twoport/width}\pgf@circ@scaled@Rlen
            \pgf@y=.5\pgf@y
            \pgf@x=-\ctikzvalof{bipoles/twoport/width}\pgf@circ@scaled@Rlen
            \pgf@x=.5\pgf@x
        \else
            \pgf@y=\ctikzvalof{tripoles/mixer/width}\pgf@circ@scaled@Rlen
            \pgf@y=.5\pgf@y
            \pgf@x=-\ctikzvalof{tripoles/mixer/width}\pgf@circ@scaled@Rlen
            \pgf@x=.5\pgf@x
        \fi
    }
    \anchor{center}{
        \pgfpointorigin
    }
    \anchor{left}{%
        \northwest
        \pgf@y=0pt
    }
    \anchor{1}{
        \northwest
        \pgf@y=0pt
    }
    \anchor{2}{
        \northwest
        \pgf@y=-\pgf@y
        \pgf@x=0pt
    }
    \anchor{3}{
        \northwest
        \pgf@y=0pt
        \pgf@x=-\pgf@x
    }
    \anchor{4}{
        \northwest
        \pgf@y=\pgf@y
        \pgf@x=0pt
    }
    \anchor{in 1}{
        \northwest
        \pgf@y=0pt
    }
    \anchor{in1}{
        \northwest
        \pgf@y=0pt
    }
    \anchor{in}{
        \northwest
        \pgf@y=0pt
    }
    \anchor{in 2}{
        \northwest
        \pgf@y=-\pgf@y
        \pgf@x=0pt
    }
    \anchor{in2}{
        \northwest
        \pgf@y=-\pgf@y
        \pgf@x=0pt
    }
    \anchor{out}{
        \northwest
        \pgf@y=0pt
        \pgf@x=-\pgf@x
    }
    \anchor{center}{
        \pgf@y=0pt
        \pgf@x=0pt
    }
    \anchor{east}{
        \northwest
        \pgf@y=0pt
        \pgf@x=-\pgf@x
    }
    \anchor{west}{
        \northwest
        \pgf@y=0pt
    }
    \anchor{south}{
        \northwest
        \pgf@x=0pt
        \pgf@y=-\pgf@y
    }
    \anchor{north}{
        \northwest
        \pgf@x=0pt
    }
    \anchor{south west}{
        \northwest
        \pgf@y=-\pgf@y
    }
    \anchor{north east}{
        \northwest
        \pgf@x=-\pgf@x
    }
    \anchor{north west}{
        \northwest
    }
    \anchor{south east}{
        \northwest
        \pgf@x=-\pgf@x
        \pgf@y=-\pgf@y
    }
    \backgroundpath{
        \pgfsetcolor{\ctikzvalof{color}}
        \pgf@circ@scaled@Rlen=\scaledRlen

        \pgf@circ@res@step=\ctikzvalof{tripoles/mixer/width}\pgf@circ@scaled@Rlen

        \pgfscope
            \pgfstartlinewidth=\pgflinewidth

            % draw outer box
            \ifpgf@circuit@boxed
                \pgfnode{blockbox}{center}{}{pgf@box}{\pgfusepath{draw}}
            \fi

            % draw outer circle
            \ifpgf@circuit@boxed
                \pgf@circ@res@step=.7\pgf@circ@res@step
                \pgfsetdash{}{0pt}	% draw solid circle if boxed
            \else
                \pgf@circ@setlinewidth{bipoles}{\pgfstartlinewidth}
            \fi
            \pgfpathcircle{\pgfpoint{0}{0}} {0.5\pgf@circ@res@step}
            \pgf@circ@draworfill

            % draw inner stuff
            \pgfsetdash{}{0pt}	% always draw solid line for inner symbol
            \pgfsetlinewidth{\pgfstartlinewidth}
            \pgfpathmoveto{\pgfpointorigin}
            \pgfpathmoveto{\pgfpointpolar{135}{0.5\pgf@circ@res@step}}
            \pgfpathlineto{\pgfpointpolar{-45}{0.5\pgf@circ@res@step}}
            \pgfpathmoveto{\pgfpointorigin}
            \pgfpathmoveto{\pgfpointpolar{45}{0.5\pgf@circ@res@step}}
            \pgfpathlineto{\pgfpointpolar{-135}{0.5\pgf@circ@res@step}}
            \pgfusepath{draw}

        \endpgfscope
    }
}

\pgfdeclareshape{adder}
{
    \savedmacro{\ctikzclass}{\edef\ctikzclass{blocks}}
    \saveddimen{\scaledRlen}{\pgfmathsetlength{\pgf@x}{\ctikzvalof{\ctikzclass/scale}\pgf@circ@Rlen}}
    \savedanchor\northwest{
        \pgfmathsetlength{\pgf@circ@scaled@Rlen}{\ctikzvalof{\ctikzclass/scale}\pgf@circ@Rlen}
        \ifpgf@circuit@boxed
            \pgf@y=\ctikzvalof{bipoles/twoport/width}\pgf@circ@scaled@Rlen
            \pgf@y=.5\pgf@y
            \pgf@x=-\ctikzvalof{bipoles/twoport/width}\pgf@circ@scaled@Rlen
            \pgf@x=.5\pgf@x
        \else
            \pgf@y=\ctikzvalof{tripoles/adder/width}\pgf@circ@scaled@Rlen
            \pgf@y=.5\pgf@y
            \pgf@x=-\ctikzvalof{tripoles/adder/width}\pgf@circ@scaled@Rlen
            \pgf@x=.5\pgf@x
        \fi
    }
    \anchor{center}{
        \pgfpointorigin
    }
    \anchor{left}{%
        \northwest
        \pgf@y=0pt
    }
    \anchor{1}{
        \northwest
        \pgf@y=0pt
    }
    \anchor{2}{
        \northwest
        \pgf@y=-\pgf@y
        \pgf@x=0pt
    }
    \anchor{3}{
        \northwest
        \pgf@y=0pt
        \pgf@x=-\pgf@x
    }
    \anchor{4}{
        \northwest
        \pgf@y=\pgf@y
        \pgf@x=0pt
    }
    \anchor{in 1}{
        \northwest
        \pgf@y=0pt
    }
    \anchor{in1}{
        \northwest
        \pgf@y=0pt
    }
    \anchor{in}{
        \northwest
        \pgf@y=0pt
    }
    \anchor{in 2}{
        \northwest
        \pgf@y=-\pgf@y
        \pgf@x=0pt
    }
    \anchor{in2}{
        \northwest
        \pgf@y=-\pgf@y
        \pgf@x=0pt
    }
    \anchor{out}{
        \northwest
        \pgf@y=0pt
        \pgf@x=-\pgf@x
    }
    \anchor{center}{
        \pgf@y=0pt
        \pgf@x=0pt
    }
    \anchor{east}{
        \northwest
        \pgf@y=0pt
        \pgf@x=-\pgf@x
    }
    \anchor{west}{
        \northwest
        \pgf@y=0pt
    }
    \anchor{south}{
        \northwest
        \pgf@x=0pt
        \pgf@y=-\pgf@y
    }
    \anchor{north}{
        \northwest
        \pgf@x=0pt
    }
    \anchor{south west}{
        \northwest
        \pgf@y=-\pgf@y
    }
    \anchor{north east}{
        \northwest
        \pgf@x=-\pgf@x
    }
    \anchor{north west}{
        \northwest
    }
    \anchor{south east}{
        \northwest
        \pgf@x=-\pgf@x
        \pgf@y=-\pgf@y
    }
    \backgroundpath{
        \pgfsetcolor{\ctikzvalof{color}}
        \pgf@circ@scaled@Rlen=\scaledRlen

        \pgf@circ@res@step=\ctikzvalof{tripoles/adder/width}\pgf@circ@scaled@Rlen

        \pgfscope
            \pgfstartlinewidth=\pgflinewidth

            % draw outer box
            \ifpgf@circuit@boxed
                \pgfnode{blockbox}{center}{}{pgf@box}{\pgfusepath{draw}}
            \fi

            % draw outer circle
            \ifpgf@circuit@boxed
                \pgf@circ@res@step=.7\pgf@circ@res@step{}
                \pgfsetdash{}{0pt}	% draw solid circle if boxed
            \else
                \pgf@circ@setlinewidth{bipoles}{\pgfstartlinewidth}
            \fi
            \pgfpathcircle{\pgfpoint{0}{0}} {0.5\pgf@circ@res@step}
            \pgf@circ@draworfill

            % draw inner stuff
            \pgfsetdash{}{0pt}	% always draw solid line for inner symbol
            \pgfsetlinewidth{\pgfstartlinewidth}

            \pgfpathmoveto{\pgfpointorigin}
            \pgfpathmoveto{\pgfpointpolar{0}{0.3\pgf@circ@res@step}}
            \pgfpathlineto{\pgfpointpolar{180}{0.3\pgf@circ@res@step}}
            \pgfpathmoveto{\pgfpointorigin}
            \pgfpathmoveto{\pgfpointpolar{90}{0.3\pgf@circ@res@step}}
            \pgfpathlineto{\pgfpointpolar{270}{0.3\pgf@circ@res@step}}
            \pgf@circ@setlinewidth{tripoles}{\pgflinewidth}
            \pgfusepath{draw}

        \endpgfscope
    }
}

\pgfdeclareshape{oscillator}
{
    \savedmacro{\ctikzclass}{\edef\ctikzclass{blocks}}
    \saveddimen{\scaledRlen}{\pgfmathsetlength{\pgf@x}{\ctikzvalof{\ctikzclass/scale}\pgf@circ@Rlen}}
    \savedanchor\northwest{
        \pgfmathsetlength{\pgf@circ@scaled@Rlen}{\ctikzvalof{\ctikzclass/scale}\pgf@circ@Rlen}
        \ifpgf@circuit@boxed
            \pgf@y=\ctikzvalof{bipoles/twoport/width}\pgf@circ@scaled@Rlen
            \pgf@y=.5\pgf@y
            \pgf@x=-\ctikzvalof{bipoles/twoport/width}\pgf@circ@scaled@Rlen
        \else
            \pgf@y=\ctikzvalof{tripoles/oscillator/width}\pgf@circ@scaled@Rlen
            \pgf@y=.5\pgf@y
            \pgf@x=-\ctikzvalof{tripoles/oscillator/width}\pgf@circ@scaled@Rlen
        \fi
    }
    \anchor{center}{
        \pgfpointorigin
    }
    \anchor{north}{
        \northwest
        \pgf@x=.5\pgf@x
    }
    \anchor{south}{
        \northwest
        \pgf@x=.5\pgf@x
        \pgf@y=-\pgf@y
    }
    \anchor{east}{
        \northwest
        \pgf@y=0pt
        \pgf@x=0pt
    }
    \anchor{west}{
        \northwest
        \pgf@y=0pt
    }
    \anchor{south west}{ \northwest \pgf@y=-\pgf@y}
    \anchor{north east}{ \northwest \pgf@x=0pt\relax}
    \anchor{north west}{ \northwest }
    \anchor{south east}{ \northwest \pgf@x=0pt \pgf@y=-\pgf@y }
    \anchor{text}{
        \pgf@x=-2\pgf@x
        \advance \pgf@x by -.5\wd\pgfnodeparttextbox
        \advance \pgf@y by -1.5\ht\pgfnodeparttextbox
    }
    \backgroundpath{
        \pgfsetcolor{\ctikzvalof{color}}
        \pgf@circ@scaled@Rlen=\scaledRlen

        \pgf@circ@res@step=\ctikzvalof{tripoles/oscillator/width}\pgf@circ@scaled@Rlen{}

        \pgfscope
            \pgfstartlinewidth=\pgflinewidth

            \pgftransformxshift{-0.5\pgf@circ@res@step} % The oscillator is shifted to the left, so a connection comes out of the anchor "east"

            % draw outer box
            \ifpgf@circuit@boxed{}
                \pgfnode{blockbox}{center}{}{pgf@box}{\pgfusepath{draw}}
            \fi

            % draw outer circle
            \ifpgf@circuit@boxed
                \pgf@circ@res@step=.7\pgf@circ@res@step{}
                \pgfsetdash{}{0pt}	% draw solid circle if boxed
            \else
                \pgf@circ@setlinewidth{bipoles}{\pgfstartlinewidth}
            \fi
            \pgfpathcircle{\pgfpoint{0}{0}} {0.5\pgf@circ@res@step}
            \pgf@circ@draworfill

            % draw inner sine waves
            \pgfsetdash{}{0pt}	% always draw solid line for inner symbol
            \pgfsetlinewidth{\pgfstartlinewidth}
            \pgfsetcornersarced{\pgfpointorigin}% do not use rounded corners!
            \pgfpathmoveto{\pgfpoint{-0.3\pgf@circ@res@step}{0\pgf@circ@res@step}}
            \pgfpathsine{\pgfpoint{.15\pgf@circ@res@step}{.15\pgf@circ@res@step}}
            \pgfpathcosine{\pgfpoint{.15\pgf@circ@res@step}{-.15\pgf@circ@res@step}}
            \pgfpathsine{\pgfpoint{.15\pgf@circ@res@step}{-.15\pgf@circ@res@step}}
            \pgfpathcosine{\pgfpoint{.15\pgf@circ@res@step}{.15\pgf@circ@res@step}}

            \pgfusepath{draw}

        \endpgfscope
    }
}

\pgfdeclareshape{circulator}
{
    \savedmacro{\ctikzclass}{\edef\ctikzclass{blocks}}
    \saveddimen{\scaledRlen}{\pgfmathsetlength{\pgf@x}{\ctikzvalof{\ctikzclass/scale}\pgf@circ@Rlen}}
    \savedanchor\northwest{
        \pgfmathsetlength{\pgf@circ@scaled@Rlen}{\ctikzvalof{\ctikzclass/scale}\pgf@circ@Rlen}
        \ifpgf@circuit@boxed
            \pgf@y=\ctikzvalof{bipoles/twoport/width}\pgf@circ@scaled@Rlen
            \pgf@y=.5\pgf@y
            \pgf@x=-\ctikzvalof{bipoles/twoport/width}\pgf@circ@scaled@Rlen
            \pgf@x=.5\pgf@x
        \else
            \pgf@y=\ctikzvalof{tripoles/circulator/width}\pgf@circ@scaled@Rlen
            \pgf@y=.5\pgf@y
            \pgf@x=-\ctikzvalof{tripoles/circulator/width}\pgf@circ@scaled@Rlen
            \pgf@x=.5\pgf@x
        \fi
    }
    \anchor{center}{
        \pgfpointorigin
    }
    \anchor{left}{%
        \northwest
        \pgf@y=0pt
    }
    \anchor{1}{
        \northwest
        \pgf@y=0pt
    }
    \anchor{2}{
        \northwest
        \pgf@y=0pt
        \pgf@x=-\pgf@x
    }
    \anchor{3}{
        \northwest
        \pgf@y=-\pgf@y
        \pgf@x=0pt
    }
    \anchor{east}{
        \northwest
        \pgf@y=0pt
        \pgf@x=-\pgf@x
    }
    \anchor{west}{
        \northwest
        \pgf@y=0pt
    }
    \anchor{south}{
        \northwest
        \pgf@x=0pt
        \pgf@y=-\pgf@y
    }
    \anchor{north}{
        \northwest
        \pgf@x=0pt
    }
    \anchor{south west}{
        \northwest
        \pgf@y=-\pgf@y
    }
    \anchor{north east}{
        \northwest
        \pgf@x=-\pgf@x
    }
    \anchor{north west}{
        \northwest
    }
    \anchor{south east}{
        \northwest
        \pgf@x=-\pgf@x
        \pgf@y=-\pgf@y
    }
    \backgroundpath{
        \pgfsetcolor{\ctikzvalof{color}}
        \pgf@circ@scaled@Rlen=\scaledRlen

        \pgf@circ@res@step=\ctikzvalof{tripoles/circulator/width}\pgf@circ@scaled@Rlen

        \pgfscope
            \pgfstartlinewidth=\pgflinewidth

            % draw outer box
            \ifpgf@circuit@boxed
                \pgfnode{blockbox}{center}{}{pgf@box}{\pgfusepath{draw}}
            \fi

            % draw outer circle
            \ifpgf@circuit@boxed{}
                \pgf@circ@res@step=.7\pgf@circ@res@step{}
                \pgfsetdash{}{0pt}	% draw solid circle if boxed
            \else
                \pgf@circ@setlinewidth{bipoles}{\pgfstartlinewidth}
            \fi
            \pgfpathcircle{\pgfpoint{0}{0}} {0.5\pgf@circ@res@step}
            \pgf@circ@draworfill

            % inner arrow
            \pgfsetdash{}{0pt}	% always draw solid line for inner symbol
            \pgfsetlinewidth{\pgfstartlinewidth}

            \pgfsetarrowsend{latex}
            \pgfpathmoveto{\pgfpoint{-0.25\pgf@circ@res@step}{0}}
            \pgfpatharc{180}{-90} {0.25\pgf@circ@res@step}
            \pgfpathlineto{\pgfpoint{-5pt}{-0.2\pgf@circ@res@step}}
            \pgfusepath{draw}

            \endpgfscope
        }
}

%% gridnode
\pgfdeclareshape{gridnode}
{
    \savedmacro{\ctikzclass}{\edef\ctikzclass{blocks}}
    \saveddimen{\scaledRlen}{\pgfmathsetlength{\pgf@x}{\ctikzvalof{\ctikzclass/scale}\pgf@circ@Rlen}}
    \savedanchor\northwest{
        \pgfmathsetlength{\pgf@circ@scaled@Rlen}{\ctikzvalof{\ctikzclass/scale}\pgf@circ@Rlen}
            \pgf@y=\ctikzvalof{quadpoles/gridnode/width}\pgf@circ@scaled@Rlen
            \pgf@y=.5\pgf@y
            \pgf@x=-\ctikzvalof{quadpoles/gridnode/width}\pgf@circ@scaled@Rlen
            \pgf@x=.5\pgf@x
    }
    \anchor{center}{
        \pgfpointorigin
    }

    \anchor{north}{
        \northwest
        \pgf@x=0pt
    }
    \anchor{up}{
        \northwest
        \pgf@x=0pt
    }
    \anchor{south}{
        \northwest
        \pgf@x=0pt
        \pgf@y=-\pgf@y
    }
    \anchor{down}{
        \northwest
        \pgf@x=0pt
        \pgf@y=-\pgf@y
    }
    \anchor{east}{
        \northwest
        \pgf@y=0pt
        \pgf@x=-\pgf@x
    }
    \anchor{right}{
        \northwest
        \pgf@y=0pt
        \pgf@x=-\pgf@x
    }
    \anchor{west}{
        \northwest
        \pgf@y=0pt
    }    
    \anchor{left}{
        \northwest
        \pgf@y=0pt
    }
    \anchor{south west}{ 
        \northwest 
        \pgf@y=-\pgf@y
    }
    \anchor{north east}{
        \northwest 
        \pgf@x=-\pgf@x
        \relax
    }
    \anchor{north west}{ 
        \northwest 
    }
    \anchor{south east}{ 
        \northwest 
        \pgf@x=-\pgf@x
        \pgf@y=-\pgf@y
    }
    \anchor{text}{
        \pgf@x=-2\pgf@x
        \advance \pgf@x by -.5\wd\pgfnodeparttextbox
        \advance \pgf@y by -1.5\ht\pgfnodeparttextbox
    }
    \backgroundpath{
        \pgfsetcolor{\ctikzvalof{color}}
        \pgf@circ@scaled@Rlen=\scaledRlen

        \pgf@circ@res@step=\ctikzvalof{quadpoles/gridnode/width}\pgf@circ@scaled@Rlen
    
        \northwest
        \pgf@circ@res@up = \pgf@y
        \pgf@circ@res@down = -\pgf@y
        \pgf@circ@res@right = -\pgf@x
        \pgf@circ@res@left = \pgf@x
        
        \pgf@circ@res@step = \ctikzvalof{quadpoles/gridnode/width}\pgf@circ@scaled@Rlen
        \divide \pgf@circ@res@step by 2

        \pgfpathmoveto{\pgfpoint{\pgf@circ@res@left}{\pgf@circ@res@zero}}
        \pgf@circ@res@other = \pgf@circ@res@left
        \advance\pgf@circ@res@other by \pgf@circ@res@step

        \ifpgf@circuit@dashed
            \pgfsetdash{{0.1cm}{0.1cm}}{0cm}
        \fi

        % draw outer box
        \pgf@circ@twoportbox
        
        
        \ifpgf@circuit@inputarrow
            {
                \advance \pgf@circ@res@left by -.5\ctikzvalof{bipoles/thickness}\pgfstartlinewidth
                \pgftransformshift{\pgfpoint{\pgf@circ@res@left}{0pt}}
                \pgfnode{inputarrow}{tip}{}{pgf@inputarrow}{\pgfusepath{fill}}
            }
        \fi
    
        \pgfsetdash{}{0pt}	% always draw solid line for inner symbol
        \pgfsetarrows{-} %never draw arrows
        \pgfsetlinewidth{0.05mm}
        
        % draw grid
        \foreach \line in {-1,-.5,...,1}
        {
            \pgfpathmoveto{\pgfpoint{\line\pgf@circ@res@right}{\pgf@circ@res@up}}
            \pgfpathlineto{\pgfpoint{\pgf@circ@res@right}{\line\pgf@circ@res@up}}
            
            \pgfpathmoveto{\pgfpoint{\line\pgf@circ@res@right}{\pgf@circ@res@up}}
            \pgfpathlineto{\pgfpoint{\pgf@circ@res@left}{\line\pgf@circ@res@down}}
        }
        
        %prevent from draw the inner cross twice
        \foreach \line in {-.5,0,...,.5}
        {
            \pgfpathmoveto{\pgfpoint{\line\pgf@circ@res@right}{\pgf@circ@res@down}}
            \pgfpathlineto{\pgfpoint{\pgf@circ@res@left}{\line\pgf@circ@res@up}}
            
            \pgfpathmoveto{\pgfpoint{\line\pgf@circ@res@right}{\pgf@circ@res@down}}
            \pgfpathlineto{\pgfpoint{\pgf@circ@res@right}{\line\pgf@circ@res@down}}
        }
        \pgfusepath{draw}
    }
}
    

% Wilkinson divider
\pgfdeclareshape{wilkinson}{
    \savedmacro{\ctikzclass}{\edef\ctikzclass{blocks}}
    \saveddimen{\scaledRlen}{\pgfmathsetlength{\pgf@x}{\ctikzvalof{\ctikzclass/scale}\pgf@circ@Rlen}}
    \savedanchor\northwest{%
        \pgfmathsetlength{\pgf@circ@scaled@Rlen}{\ctikzvalof{\ctikzclass/scale}\pgf@circ@Rlen}
        \pgf@y=\ctikzvalof{tripoles/wilkinson/height}\pgf@circ@scaled@Rlen
        \pgf@y=.5\pgf@y
        \pgf@x= \pgf@circ@scaled@Rlen
        \pgf@x=.5\pgf@x
        \pgf@x=-\ctikzvalof{tripoles/wilkinson/width}\pgf@x
    }
    \anchor{center}{
        \northwest
        \pgf@x=0pt
        \pgf@y=0pt
    }
    \anchor{north}{
        \northwest
        \pgf@x=0pt
    }
    \anchor{south}{
        \northwest
        \pgf@x=0pt
        \pgf@y=-\pgf@y
    }
    \anchor{west}{
        \northwest
        \pgf@y=0pt
    }
    \anchor{east}{
        \northwest
        \pgf@y=0pt
        \pgf@x=-\pgf@x
    }
    \anchor{south west}{
        \northwest
        \pgf@y=-\pgf@y
    }
    \anchor{north east}{
        \northwest
        \pgf@x=-\pgf@x
    }
    \anchor{north west}{
        \northwest
    }
    \anchor{south east}{
        \northwest
        \pgf@x=-\pgf@x
        \pgf@y=-\pgf@y
    }
    \anchor{in}{
        \northwest
        \pgf@y=0pt
    }
    \anchor{out1}{
        \northwest
        \pgf@x=-\pgf@x
        \pgf@y=-0.5\pgf@y
    }
    \anchor{out2}{
        \northwest
        \pgf@x=-\pgf@x
        \pgf@y=0.5\pgf@y
    }
    \anchor{text}{
        \northwest
        \advance \pgf@y by 0.5\ht\pgfnodeparttextbox
        \pgf@x=-.5\wd\pgfnodeparttextbox
    }
    \backgroundpath{
        \pgfsetcolor{\ctikzvalof{color}}
        \pgf@circ@scaled@Rlen=\scaledRlen

        \northwest
        \pgf@circ@res@up = \pgf@y
        \pgf@circ@res@down = -\pgf@y
        \pgf@circ@res@right = -\pgf@x
        \pgf@circ@res@left = \pgf@x

        \pgfstartlinewidth=\pgflinewidth

        % draw outer box
        \pgf@circ@twoportbox

        % draw inner stuff
        \pgfsetdash{}{0pt}	% always draw solid line for inner symbol
        \pgfsetarrows{-} %never draw arrows
        \pgfsetlinewidth{\pgfstartlinewidth}
        \pgfpathmoveto{\pgfpoint{\pgf@circ@res@left}{0pt}}
        \pgfpathlineto{\pgfpoint{0.5\pgf@circ@res@left}{0pt}}
        \pgfpathmoveto{\pgfpoint{\pgf@circ@res@right}{0.5\pgf@circ@res@up}}
        \pgfpathlineto{\pgfpoint{0.5\pgf@circ@res@right}{0.5\pgf@circ@res@up}}
        \pgfpathlineto{\pgfpoint{0.5\pgf@circ@res@left}{0pt}}
        \pgfpathlineto{\pgfpoint{0.5\pgf@circ@res@right}{0.5\pgf@circ@res@down}}
        \pgfpathlineto{\pgfpoint{\pgf@circ@res@right}{0.5\pgf@circ@res@down}}

        \pgfusepath{draw}

        \pgfsetdash{}{0pt}	% always draw solid line for inner symbol
        % draw inner resisitor - european or american style is recognised
        {
            \pgftransformshift{\pgfpoint{0.5\pgf@circ@res@right}{0pt}}
            \pgftransformrotate{90}

            % calculate size of resistor
            \ifpgf@circuit@europeanresistor
                \pgfmathparse{\pgf@circ@res@up / \pgf@circ@scaled@Rlen / \ctikzvalof{bipoles/generic/width} / 2}
                \pgftransformscale{\pgfmathresult}
                \pgfnode{genericshape}{center}{}{wilk@int@R}{\pgfusepath{fill}}
            \else
                \pgfmathparse{\pgf@circ@res@up / \pgf@circ@scaled@Rlen / \ctikzvalof{bipoles/resistor/width} / 2}
                \pgftransformscale{\pgfmathresult}
                \pgfnode{resistorshape}{center}{}{wilk@int@R}{\pgfusepath{fill}}
            \fi
        }

        \pgfpathmoveto{\pgfpoint{0.5\pgf@circ@res@right}{0.5\pgf@circ@res@up}}
        \pgfpathlineto{\pgfpointanchor{wilk@int@R}{right}}

        \pgfpathmoveto{\pgfpointanchor{wilk@int@R}{left}}
        \pgfpathlineto{\pgfpoint{0.5\pgf@circ@res@right}{0.5\pgf@circ@res@down}}
        \pgfusepath{draw}

    }
}

%% couplers generics
\long\def\pgfcircdeclarefourport#1#2{

    \pgfdeclareshape{#1}{
        \savedmacro{\ctikzclass}{\edef\ctikzclass{blocks}}
        \saveddimen{\scaledRlen}{\pgfmathsetlength{\pgf@x}{\ctikzvalof{\ctikzclass/scale}\pgf@circ@Rlen}}
        \anchor{center}{
            \northwest
            \pgf@x=0pt
            \pgf@y=0pt
        }
        \savedanchor\northwest{%
            \pgfmathsetlength{\pgf@circ@scaled@Rlen}{\ctikzvalof{\ctikzclass/scale}\pgf@circ@Rlen}
            \pgf@y=\ctikzvalof{quadpoles/#1/height}\pgf@circ@scaled@Rlen
            \pgf@y=.5\pgf@y
            \pgf@x=.5\pgf@circ@scaled@Rlen
            \pgf@x=-\ctikzvalof{quadpoles/#1/width}\pgf@x
        }
        \anchor{north}{
            \northwest
            \pgf@x=0pt
        }
        \anchor{south}{
            \northwest
            \pgf@x=0pt
            \pgf@y=-\pgf@y
        }
        \anchor{west}{
            \northwest
            \pgf@y=0pt
        }
        \anchor{east}{
            \northwest
            \pgf@y=0pt
            \pgf@x=-\pgf@x
        }
        \anchor{south west}{
            \northwest
            \pgf@y=-\pgf@y
        }
        \anchor{north east}{
            \northwest
            \pgf@x=-\pgf@x
        }
        \anchor{north west}{
            \northwest
        }
        \anchor{south east}{
            \northwest
            \pgf@x=-\pgf@x
            \pgf@y=-\pgf@y
        }
        \anchor{port1}{
            \northwest
            \pgf@y=-0.5\pgf@y
        }
        \anchor{port2}{
            \northwest
            \pgf@x=-\pgf@x
            \pgf@y=-0.5\pgf@y
        }
        \anchor{port3}{
            \northwest
            \pgf@x=-\pgf@x
            \pgf@y=0.5\pgf@y
        }
        \anchor{port4}{
            \northwest
            \pgf@y=0.5\pgf@y
        }
        \anchor{1}{
            \northwest
            \pgf@y=-0.5\pgf@y
        }
        \anchor{2}{
            \northwest
            \pgf@x=-\pgf@x
            \pgf@y=-0.5\pgf@y
        }
        \anchor{3}{
            \northwest
            \pgf@x=-\pgf@x
            \pgf@y=0.5\pgf@y
        }
        \anchor{4}{
            \northwest
            \pgf@y=0.5\pgf@y
        }

        \anchor{text}{
            \northwest
            \advance \pgf@y by 0.5\ht\pgfnodeparttextbox
            \pgf@x=-.5\wd\pgfnodeparttextbox
        }
        \backgroundpath{
            \pgfsetcolor{\ctikzvalof{color}}

            \northwest
            \pgf@circ@res@up = \pgf@y
            \pgf@circ@res@down = -\pgf@y
            \pgf@circ@res@right = -\pgf@x
            \pgf@circ@res@left = \pgf@x
            \pgf@circ@scaled@Rlen=\scaledRlen

            \pgfstartlinewidth=\pgflinewidth

            % draw outer box
            \pgf@circ@setlinewidth{bipoles}{\pgfstartlinewidth}
            \pgfpathrectanglecorners{\pgfpoint{\pgf@circ@res@left}{\pgf@circ@res@up}}{\pgfpoint{\pgf@circ@res@right}{\pgf@circ@res@down}}
            \pgf@circ@draworfill

            % draw inner stuff
            #2

            % draw inner text
            \pgftext[center,x=-0.15\pgf@circ@res@step,y=0]{\ctikzvalof{bipoles/twoport/text}}
        }
    }
}

% four-port
\pgfcircdeclarefourport{fourport}{}

% straight coupler
\pgfcircdeclarefourport{coupler}{
    \pgfsetlinewidth{\pgfstartlinewidth}
    \pgfpathmoveto{\pgfpoint{\pgf@circ@res@left}{0.5\pgf@circ@res@up}}
    \pgfpathlineto{\pgfpoint{\pgf@circ@res@right}{0.5\pgf@circ@res@up}}
    \pgfpathmoveto{\pgfpoint{\pgf@circ@res@left}{0.5\pgf@circ@res@down}}
    \pgfpathlineto{\pgfpoint{\pgf@circ@res@right}{0.5\pgf@circ@res@down}}
    \pgfusepath{draw}

    \pgfpathmoveto{\pgfpoint{0.7\pgf@circ@res@left}{0.4\pgf@circ@res@down}}
    \pgfpathlineto{\pgfpoint{0.7\pgf@circ@res@right}{0.4\pgf@circ@res@up}}
    \pgfsetarrows{latex-latex}
    \pgfusepath{draw}
    \pgfpathmoveto{\pgfpoint{0.7\pgf@circ@res@left}{0.4\pgf@circ@res@up}}
    \pgfpathlineto{\pgfpoint{0.7\pgf@circ@res@right}{0.4\pgf@circ@res@down}}
    \pgfsetarrows{latex-latex}
    \pgfusepath{draw}
}

% "bended" coupler
\pgfcircdeclarefourport{coupler2}{
    \pgfsetlinewidth{\pgfstartlinewidth}
    \pgfpathmoveto{\pgfpoint{\pgf@circ@res@left}{0.5\pgf@circ@res@up}}
    \pgfpathlineto{\pgfpoint{\pgf@circ@res@right}{0.5\pgf@circ@res@up}}
    \pgfpathmoveto{\pgfpoint{\pgf@circ@res@left}{0.5\pgf@circ@res@down}}
    \pgfpathlineto{\pgfpoint{\pgf@circ@res@right}{0.5\pgf@circ@res@down}}
    \pgfusepath{draw}

    \pgfscope
        \pgfpathmoveto{\pgfpoint{0.1\pgf@circ@res@left}{0pt}}
        \pgfpatharc{0}{90} {0.4\pgf@circ@res@up}
        \pgfsetarrowsend{latex}
        \pgfusepath{draw}
    \endpgfscope
    \pgfscope
        \pgfpathmoveto{\pgfpoint{0.1\pgf@circ@res@left}{0pt}}
        \pgfpatharc{0}{-90} {0.4\pgf@circ@res@up}
        \pgfsetarrowsend{latex}
        \pgfusepath{draw}
    \endpgfscope
    \pgfscope
        \pgfpathmoveto{\pgfpoint{0.1\pgf@circ@res@right}{0pt}}
        \pgfpatharc{180}{90} {0.4\pgf@circ@res@up}
        \pgfsetarrowsend{latex}
        \pgfusepath{draw}
    \endpgfscope
    \pgfscope
        \pgfpathmoveto{\pgfpoint{0.1\pgf@circ@res@right}{0pt}}
        \pgfpatharc{-180}{-90} {0.4\pgf@circ@res@up}
        \pgfsetarrowsend{latex}
        \pgfusepath{draw}
    \endpgfscope
}

%%%---------- close: tex/pgfcircquadpoles
%%%%%%%%%%% Springe nach tex/pgfcircmultipoles
%%%---------- open: tex/pgfcircmultipoles.tex
% Copyright 2018-2020 by Romano Giannetti
% Copyright 2015-2020 by Stefan Lindner
% Copyright 2013-2020 by Stefan Erhardt
% Copyright 2007-2020 by Massimo Redaelli
%
% This file may be distributed and/or modified
%
% 1. under the LaTeX Project Public License and/or
% 2. under the GNU Public License.
%
% See the files gpl-3.0_license.txt and lppl-1-3c_license.txt for more details.

%%%%%%%%%%%%%%%%%%%%%%%%%%%%%%%%%%%%%%
%% Multipoles by Romano Giannetti
%%%%%%%%%%%%%%%%%%%%%%%%%%%%%%%%%%%%%%

%%%%%%%%%
%% Chips
%%%%%%%%%

% let's use the same shifts everywhere, no magic numbers
\def\pgf@circ@dip@pin@shift{0.5}
\def\pgf@circ@qfp@pin@shift{0.25}

% derived from https://tex.stackexchange.com/a/146753/38080
% original author Mark Wibrow
% Thanks also to John Kormylo https://tex.stackexchange.com/a/372996/38080
% a lot of thanks to @marmot  for the un-rotation hint
% https://tex.stackexchange.com/a/473571/38080

% DIP (dual in line package) chips

\pgfdeclareshape{dipchip}{
    \savedmacro{\ctikzclass}{\edef\ctikzclass{chips}}
    \saveddimen{\scaledRlen}{\pgfmathsetlength{\pgf@x}{\ctikzvalof{\ctikzclass/scale}\pgf@circ@Rlen}}
    \savedmacro\numpins{%
            \pgf@circ@count@a=\ctikzvalof{multipoles/dipchip/num pins}%
            \def\numpins{\the\pgf@circ@count@a}
    }
    \savedanchor\centerpoint{%
        \pgf@x=-.5\wd\pgfnodeparttextbox%
        \pgf@y=-.5\ht\pgfnodeparttextbox%
        \advance\pgf@y by+.5\dp\pgfnodeparttextbox%
    }%
    \savedanchor\origin{\pgfpoint{0pt}{0pt}}
    \anchor{center}{\origin}
    \anchor{text}{\centerpoint}% to adjust text
    \saveddimen\height{%
        \pgfmathsetlength{\pgf@circ@scaled@Rlen}{\ctikzvalof{\ctikzclass/scale}\pgf@circ@Rlen}
        \pgfmathsetlength\pgf@x{((\numpins)
        *\ctikzvalof{multipoles/dipchip/pin spacing})*\pgf@circ@scaled@Rlen/2}%
    }%
    \saveddimen{\chipspacing}{
        \pgfmathsetlength{\pgf@circ@scaled@Rlen}{\ctikzvalof{\ctikzclass/scale}\pgf@circ@Rlen}
        \pgfmathsetlength\pgf@x{\pgf@circ@scaled@Rlen*\ctikzvalof{multipoles/dipchip/pin spacing}}}
    \saveddimen{\width}{
        \pgfmathsetlength{\pgf@circ@scaled@Rlen}{\ctikzvalof{\ctikzclass/scale}\pgf@circ@Rlen}
        \pgfmathsetlength\pgf@x{\pgf@circ@scaled@Rlen*\ctikzvalof{multipoles/dipchip/width}}}
    \saveddimen{\extshift}{
        \pgfmathsetlength{\pgf@circ@scaled@Rlen}{\ctikzvalof{\ctikzclass/scale}\pgf@circ@Rlen}
        \pgfmathsetlength\pgf@x{\pgf@circ@scaled@Rlen*\ctikzvalof{multipoles/external pins width}}}
    % standard anchors
    \savedanchor\northwest{%
        \pgfmathsetlength{\pgf@circ@scaled@Rlen}{\ctikzvalof{\ctikzclass/scale}\pgf@circ@Rlen}
        \pgfmathsetlength\pgf@y{0.5*((\numpins)
        *\ctikzvalof{multipoles/dipchip/pin spacing})*\pgf@circ@scaled@Rlen/2}%
        \pgfmathsetlength\pgf@x{-0.5*\pgf@circ@scaled@Rlen*\ctikzvalof{multipoles/dipchip/width}}
    }
    \anchor{dot}{\northwest
        \pgfmathsetlength\pgf@x{\pgf@x + 0.3*\chipspacing}
        \pgfmathsetlength\pgf@y{\pgf@y - 0.3*\chipspacing}
    }
    \anchor{nw}{\northwest}
    \anchor{ne}{\northwest\pgf@x=-\pgf@x}
    \anchor{se}{\northwest\pgf@x=-\pgf@x\pgf@y=-\pgf@y}
    \anchor{sw}{\northwest\pgf@y=-\pgf@y}
    \anchor{north west}{\northwest}
    \anchor{north east}{\northwest\pgf@x=-\pgf@x}
    \anchor{south east}{\northwest\pgf@x=-\pgf@x \pgf@y=-\pgf@y}
    \anchor{south west}{\northwest\pgf@y=-\pgf@y}
    \anchor{n}{\northwest\pgf@x=0pt }
    \anchor{e}{\northwest\pgf@x=-\pgf@x\pgf@y=0pt }
    \anchor{s}{\northwest\pgf@x=0pt\pgf@y=-\pgf@y}
    \anchor{w}{\northwest\pgf@y=0pt }
    \anchor{north}{\northwest\pgf@x=0pt }
    \anchor{east}{\northwest\pgf@x=-\pgf@x\pgf@y=0pt }
    \anchor{south}{\northwest\pgf@x=0pt\pgf@y=-\pgf@y}
    \anchor{west}{\northwest\pgf@y=0pt }
    % start drawing
    \backgroundpath{%
        \northwest
        \pgf@circ@res@up = \pgf@y
        \pgf@circ@res@down = -\pgf@y
        \pgf@circ@res@right = -\pgf@x
        \pgf@circ@res@left = \pgf@x
        \pgf@circ@scaled@Rlen=\scaledRlen
        \pgf@circ@res@step = \ctikzvalof{multipoles/dipchip/pin spacing}\pgf@circ@scaled@Rlen
        \pgf@circ@res@other = \ctikzvalof{multipoles/external pins width}\pgf@circ@scaled@Rlen
        \pgfscope% (for the line width)
        \pgf@circ@setlinewidth{multipoles}{\pgflinewidth}
        \pgfpathrectanglecorners{\pgfpoint{-\width/2}{-\height/2}}{\pgfpoint{\width/2}{\height/2}}%
        \pgf@circ@draworfill
        %% upside mark
        \ifpgf@circuit@chip@topmark
            \pgfpathmoveto{\pgfpoint{0.2*\pgf@circ@res@left}{\pgf@circ@res@up}}
            \pgfpatharc{0}{180}{0.2*\pgf@circ@res@left}
        \fi
        \pgfusepath{stroke}%
        \pgfsetcolor{\ctikzvalof{color}}
        % Adding the pin number
        \ifpgf@circuit@chip@shownumbers
            \pgf@circ@count@a=\numpins\relax
            \divide\pgf@circ@count@a by 2 \pgf@circ@count@b=\pgf@circ@count@a
            % thanks to @marmot: https://tex.stackexchange.com/a/473571/38080
            \ifpgf@circuit@chip@straightnumbers
                \pgfgettransformentries\a\b\temp\temp\temp\temp
                \pgfmathsetmacro{\rot}{-atan2(\b,\a)}
                \pgfmathtruncatemacro{\quadrant}{mod(4+int(360+(\rot+45)/90),4)}
            \else
                \pgfmathsetmacro{\rot}{0}
                \pgfmathsetmacro{\quadrant}{0}
            \fi
            \def\pgf@circ@strut{\vrule width 0pt height 1em depth 0.4em\relax}
            \def\mytext{\ctikzvalof{multipoles/font}\space\pgf@circ@strut\the\pgf@circ@count@c\space}
            \pgfmathloop%
            \ifnum\pgf@circ@count@a>0
                \ifcase\quadrant % rotation 0
                    % left
                    \pgf@circ@count@c=\pgf@circ@count@a
                    \pgftext[left,
                        at=\pgfpoint{\pgf@circ@res@left}{\pgf@circ@res@up+(\pgf@circ@dip@pin@shift-\the\pgf@circ@count@a)*\pgf@circ@res@step},
                        rotate=\rot]{\mytext}
                    % right
                    \pgf@circ@count@c=\numexpr2*\pgf@circ@count@b-\pgf@circ@count@a+1\relax
                    \pgftext[right,
                        at=\pgfpoint{\pgf@circ@res@right}{\pgf@circ@res@up+(\pgf@circ@dip@pin@shift-\the\pgf@circ@count@a)*\pgf@circ@res@step},
                        rotate=\rot]{\mytext}
                \or % rotation -90
                    % left
                    \pgf@circ@count@c=\pgf@circ@count@a
                    \pgftext[top,
                        at=\pgfpoint{\pgf@circ@res@left}{\pgf@circ@res@up+(\pgf@circ@dip@pin@shift-\the\pgf@circ@count@a)*\pgf@circ@res@step},
                        rotate=\rot]{\mytext}
                    % right
                    \pgf@circ@count@c=\numexpr2*\pgf@circ@count@b-\pgf@circ@count@a+1\relax
                    \pgftext[bottom,
                        at=\pgfpoint{\pgf@circ@res@right}{\pgf@circ@res@up+(\pgf@circ@dip@pin@shift-\the\pgf@circ@count@a)*\pgf@circ@res@step},
                        rotate=\rot]{\mytext}
                \or %rotation 180
                    % left
                    \pgf@circ@count@c=\pgf@circ@count@a
                    \pgftext[right,
                        at=\pgfpoint{\pgf@circ@res@left}{\pgf@circ@res@up+(\pgf@circ@dip@pin@shift-\the\pgf@circ@count@a)*\pgf@circ@res@step},
                        rotate=\rot]{\mytext}
                    % right
                    \pgf@circ@count@c=\numexpr2*\pgf@circ@count@b-\pgf@circ@count@a+1\relax
                    \pgftext[left,
                        at=\pgfpoint{\pgf@circ@res@right}{\pgf@circ@res@up+(\pgf@circ@dip@pin@shift-\the\pgf@circ@count@a)*\pgf@circ@res@step},
                        rotate=\rot]{\mytext}
                \or % rotation +90
                    % left
                    \pgf@circ@count@c=\pgf@circ@count@a
                    \pgftext[bottom,
                        at=\pgfpoint{\pgf@circ@res@left}{\pgf@circ@res@up+(\pgf@circ@dip@pin@shift-\the\pgf@circ@count@a)*\pgf@circ@res@step},
                        rotate=\rot]{\mytext}
                    % right
                    \pgf@circ@count@c=\numexpr2*\pgf@circ@count@b-\pgf@circ@count@a+1\relax
                    \pgftext[top,
                        at=\pgfpoint{\pgf@circ@res@right}{\pgf@circ@res@up+(\pgf@circ@dip@pin@shift-\the\pgf@circ@count@a)*\pgf@circ@res@step},
                        rotate=\rot]{\mytext}
                \fi
                \advance\pgf@circ@count@a-1\relax%
                \repeatpgfmathloop
            \fi
            \endpgfscope
            \ifdim\pgf@circ@res@other>0pt
            \pgfscope
                \pgfsetlinewidth{\ctikzvalof{multipoles/external pins thickness}\pgflinewidth}
                \pgf@circ@count@a=\numpins\relax
                \divide\pgf@circ@count@a by 2 \pgf@circ@count@b=\pgf@circ@count@a
                \pgfmathloop%
                \ifnum\pgf@circ@count@a>0
                    \edef\padfrac{\ctikzvalof{multipoles/external pad fraction}}
                    \ifnum\padfrac>0
                        \pgf@circ@res@temp=\pgf@circ@res@step\divide\pgf@circ@res@temp by \padfrac
                        % left side pads
                        \pgfpathmoveto{\pgfpoint{\pgf@circ@res@left}{\pgf@circ@res@temp+\pgf@circ@res@up+(\pgf@circ@dip@pin@shift-\the\pgf@circ@count@a)*\pgf@circ@res@step}}
                        \pgfpathlineto{\pgfpoint{\pgf@circ@res@left-\pgf@circ@res@other}{\pgf@circ@res@temp+\pgf@circ@res@up+(\pgf@circ@dip@pin@shift-\the\pgf@circ@count@a)*\pgf@circ@res@step}}
                        \pgfpathlineto{\pgfpoint{\pgf@circ@res@left-\pgf@circ@res@other}{-\pgf@circ@res@temp+\pgf@circ@res@up+(\pgf@circ@dip@pin@shift-\the\pgf@circ@count@a)*\pgf@circ@res@step}}
                        \pgfpathlineto{\pgfpoint{\pgf@circ@res@left}{-\pgf@circ@res@temp+\pgf@circ@res@up+(\pgf@circ@dip@pin@shift-\the\pgf@circ@count@a)*\pgf@circ@res@step}}
                        % right side pads
                        \pgfpathmoveto{\pgfpoint{\pgf@circ@res@right}{\pgf@circ@res@temp+\pgf@circ@res@up+(\pgf@circ@dip@pin@shift-\the\pgf@circ@count@a)*\pgf@circ@res@step}}
                        \pgfpathlineto{\pgfpoint{\pgf@circ@res@right+\pgf@circ@res@other}{\pgf@circ@res@temp+\pgf@circ@res@up+(\pgf@circ@dip@pin@shift-\the\pgf@circ@count@a)*\pgf@circ@res@step}}
                        \pgfpathlineto{\pgfpoint{\pgf@circ@res@right+\pgf@circ@res@other}{-\pgf@circ@res@temp+\pgf@circ@res@up+(\pgf@circ@dip@pin@shift-\the\pgf@circ@count@a)*\pgf@circ@res@step}}
                        \pgfpathlineto{\pgfpoint{\pgf@circ@res@right}{-\pgf@circ@res@temp+\pgf@circ@res@up+(\pgf@circ@dip@pin@shift-\the\pgf@circ@count@a)*\pgf@circ@res@step}}
                    \else
                        % left side pins
                        \pgfpathmoveto{\pgfpoint{\pgf@circ@res@left}{\pgf@circ@res@up+(\pgf@circ@dip@pin@shift-\the\pgf@circ@count@a)*\pgf@circ@res@step}}
                        \pgfpathlineto{\pgfpoint{\pgf@circ@res@left-\pgf@circ@res@other}{\pgf@circ@res@up+(\pgf@circ@dip@pin@shift-\the\pgf@circ@count@a)*\pgf@circ@res@step}}
                        % right side pins
                        \pgfpathmoveto{\pgfpoint{\pgf@circ@res@right}{\pgf@circ@res@up+(\pgf@circ@dip@pin@shift-\the\pgf@circ@count@a)*\pgf@circ@res@step}}
                        \pgfpathlineto{\pgfpoint{\pgf@circ@res@right+\pgf@circ@res@other}{\pgf@circ@res@up+(\pgf@circ@dip@pin@shift-\the\pgf@circ@count@a)*\pgf@circ@res@step}}
                    \fi
                    \advance\pgf@circ@count@a by -1\relax%
                \repeatpgfmathloop
                \pgfusepath{stroke}
            \endpgfscope
            \fi
        }%
        % \pgf@sh@s@<name of the shape here> contains all the code for the shape
        % and is executed just before a node is drawn.
        \pgfutil@g@addto@macro\pgf@sh@s@dipchip{%
            % Start with the maximum pin number and go backwards.
            \pgf@circ@count@a=\numpins\relax
            \pgfmathloop%
            \ifnum\pgf@circ@count@a>0
                % we will create two anchors per pin: the "normal one" like `pin 1` for the
                % electrical contact, and the "border one" like `bpin 1` for labels.
                % they will coincide if `external pins width` is set to 0.
                \expandafter\xdef\csname pgf@anchor@dipchip@pin\space\the\pgf@circ@count@a\endcsname{%
                    \noexpand\pgf@circ@dippinanchor{\the\pgf@circ@count@a}{1}%
                }
                \expandafter\xdef\csname pgf@anchor@dipchip@bpin\space\the\pgf@circ@count@a\endcsname{%
                    \noexpand\pgf@circ@dippinanchor{\the\pgf@circ@count@a}{0}%
                }
                \advance\pgf@circ@count@a by -1\relax%
                \repeatpgfmathloop%
            }%
        }

% QFP (quad flat package) chips

\pgfdeclareshape{qfpchip}{
    \savedmacro{\ctikzclass}{\edef\ctikzclass{chips}}
    \saveddimen{\scaledRlen}{\pgfmathsetlength{\pgf@x}{\ctikzvalof{\ctikzclass/scale}\pgf@circ@Rlen}}
    \savedmacro\numpins{%
            \pgf@circ@count@a=\ctikzvalof{multipoles/qfpchip/num pins}%
            \def\numpins{\the\pgf@circ@count@a}
    }
    \savedanchor\centerpoint{%
        \pgf@x=-.5\wd\pgfnodeparttextbox%
        \pgf@y=-.5\ht\pgfnodeparttextbox%
        \advance\pgf@y by+.5\dp\pgfnodeparttextbox%
    }%
    \savedanchor\origin{\pgfpoint{0pt}{0pt}}
    \anchor{center}{\origin}
    \anchor{text}{\centerpoint}% to adjust text
    \saveddimen\height{%
        \pgfmathsetlength{\pgf@circ@scaled@Rlen}{\ctikzvalof{\ctikzclass/scale}\pgf@circ@Rlen}
        \pgfmathsetlength\pgf@x{((\numpins+2)
        *\ctikzvalof{multipoles/qfpchip/pin spacing})*\pgf@circ@scaled@Rlen/4}%
    }%
    \saveddimen\width{%
        \pgfmathsetlength{\pgf@circ@scaled@Rlen}{\ctikzvalof{\ctikzclass/scale}\pgf@circ@Rlen}
        \pgfmathsetlength\pgf@x{((\numpins+2)
        *\ctikzvalof{multipoles/qfpchip/pin spacing})*\pgf@circ@scaled@Rlen/4}%
    }%
    \saveddimen{\chipspacing}{
        \pgfmathsetlength{\pgf@circ@scaled@Rlen}{\ctikzvalof{\ctikzclass/scale}\pgf@circ@Rlen}
        \pgfmathsetlength\pgf@x{\pgf@circ@scaled@Rlen*\ctikzvalof{multipoles/qfpchip/pin spacing}}}
    \saveddimen{\extshift}{
        \pgfmathsetlength{\pgf@circ@scaled@Rlen}{\ctikzvalof{\ctikzclass/scale}\pgf@circ@Rlen}
        \pgfmathsetlength\pgf@x{\pgf@circ@scaled@Rlen*\ctikzvalof{multipoles/external pins width}}}
    % standard anchors
    \savedanchor\northwest{%
        \pgfmathsetlength{\pgf@circ@scaled@Rlen}{\ctikzvalof{\ctikzclass/scale}\pgf@circ@Rlen}
        \pgfmathsetlength\pgf@y{0.5*((\numpins+2)
        *\ctikzvalof{multipoles/qfpchip/pin spacing})*\pgf@circ@scaled@Rlen/4}%
        \pgf@x=-\pgf@y
    }
    \anchor{dot}{\northwest
        \pgfmathsetlength\pgf@x{\pgf@x + 0.3*\chipspacing}
        \pgfmathsetlength\pgf@y{\pgf@y - 0.3*\chipspacing}
    }
    \anchor{nw}{\northwest}
    \anchor{ne}{\northwest\pgf@x=-\pgf@x}
    \anchor{se}{\northwest\pgf@x=-\pgf@x\pgf@y=-\pgf@y}
    \anchor{sw}{\northwest\pgf@y=-\pgf@y}
    \anchor{north west}{\northwest}
    \anchor{north east}{\northwest\pgf@x=-\pgf@x}
    \anchor{south east}{\northwest\pgf@x=-\pgf@x \pgf@y=-\pgf@y}
    \anchor{south west}{\northwest\pgf@y=-\pgf@y}
    \anchor{n}{\northwest\pgf@x=0pt }
    \anchor{e}{\northwest\pgf@x=-\pgf@x\pgf@y=0pt }
    \anchor{s}{\northwest\pgf@x=0pt\pgf@y=-\pgf@y}
    \anchor{w}{\northwest\pgf@y=0pt }
    \anchor{north}{\northwest\pgf@x=0pt }
    \anchor{east}{\northwest\pgf@x=-\pgf@x\pgf@y=0pt }
    \anchor{south}{\northwest\pgf@x=0pt\pgf@y=-\pgf@y}
    \anchor{west}{\northwest\pgf@y=0pt }
    % start drawing
    \backgroundpath{%
        \northwest
        \pgf@circ@res@up = \pgf@y
        \pgf@circ@res@down = -\pgf@y
        \pgf@circ@res@right = -\pgf@x
        \pgf@circ@res@left = \pgf@x
        \pgf@circ@scaled@Rlen=\scaledRlen
        \pgf@circ@res@step = \ctikzvalof{multipoles/qfpchip/pin spacing}\pgf@circ@scaled@Rlen
        \pgf@circ@res@other = \ctikzvalof{multipoles/external pins width}\pgf@circ@scaled@Rlen
        \pgfscope% (for the line width)
        \pgf@circ@setlinewidth{multipoles}{\pgflinewidth}
        %% upside mark
        \ifpgf@circuit@chip@topmark
            \pgfpathmoveto{\pgfpoint{-\width/2}{\height/2-\pgf@circ@res@step/2}}
            \pgfpathlineto{\pgfpoint{-\width/2+\pgf@circ@res@step/2}{\height/2}}
        \else
            \pgfpathmoveto{\pgfpoint{-\width/2}{\height/2}}
        \fi
        %% rest of the shape
        \pgfpathlineto{\pgfpoint{\width/2}{\height/2}}
        \pgfpathlineto{\pgfpoint{\width/2}{-\height/2}}
        \pgfpathlineto{\pgfpoint{-\width/2}{-\height/2}}
        \pgfpathclose
        \pgf@circ@draworfill
        % Adding the pin number
        \pgfsetcolor{\ctikzvalof{color}}
        \ifpgf@circuit@chip@shownumbers
            \pgf@circ@count@a=\numpins%
            \divide\pgf@circ@count@a by 4 \pgf@circ@count@b=\pgf@circ@count@a
            % thanks to @marmot: https://tex.stackexchange.com/a/473571/38080
            \ifpgf@circuit@chip@straightnumbers
                \pgfgettransformentries\a\b\temp\temp\temp\temp
                \pgfmathsetmacro{\rot}{-atan2(\b,\a)}
                \pgfmathtruncatemacro{\quadrant}{mod(4+int(360+(\rot+45)/90),4)}
            \else
                \pgfmathsetmacro{\rot}{0}
                \pgfmathsetmacro{\quadrant}{0}
            \fi
            \def\pgf@circ@strut{\vrule width 0pt height 1em depth 0.4em\relax}
            \def\mytext{\ctikzvalof{multipoles/font}\space\pgf@circ@strut\the\pgf@circ@count@c\space}
            \pgfmathloop%
            \ifnum\pgf@circ@count@a>0
                \ifcase\quadrant % rotation 0
                    % left
                    \pgf@circ@count@c=\pgf@circ@count@a
                    \pgftext[left,
                        at=\pgfpoint{\pgf@circ@res@left}{\pgf@circ@res@up+(\pgf@circ@qfp@pin@shift-\the\pgf@circ@count@a)*\pgf@circ@res@step},
                        rotate=\rot]{\mytext}
                    % bottom
                    \pgf@circ@count@c=\numexpr\pgf@circ@count@b+\pgf@circ@count@a\relax
                    \pgftext[bottom,
                        at=\pgfpoint{\pgf@circ@res@left-(\pgf@circ@qfp@pin@shift-\the\pgf@circ@count@a)*\pgf@circ@res@step}{\pgf@circ@res@down},
                        rotate=\rot]{\mytext}
                    % right
                    \pgf@circ@count@c=\numexpr3*\pgf@circ@count@b-\pgf@circ@count@a+1\relax
                    \pgftext[right,
                        at=\pgfpoint{\pgf@circ@res@right}{\pgf@circ@res@up+(\pgf@circ@qfp@pin@shift-\the\pgf@circ@count@a)*\pgf@circ@res@step},
                        rotate=\rot]{\mytext}
                    % top
                    \pgf@circ@count@c=\numexpr3*\pgf@circ@count@b+\pgf@circ@count@a\relax
                    \pgftext[top,
                        at=\pgfpoint{\pgf@circ@res@right+(\pgf@circ@qfp@pin@shift-\the\pgf@circ@count@a)*\pgf@circ@res@step}{\pgf@circ@res@up},
                        rotate=\rot]{\mytext}
                \or % rotation -90
                    % left
                    \pgf@circ@count@c=\pgf@circ@count@a
                    \pgftext[top,
                        at=\pgfpoint{\pgf@circ@res@left}{\pgf@circ@res@up+(\pgf@circ@qfp@pin@shift-\the\pgf@circ@count@a)*\pgf@circ@res@step},
                        rotate=\rot]{\mytext}
                    % bottom
                    \pgf@circ@count@c=\numexpr\pgf@circ@count@b+\pgf@circ@count@a\relax
                    \pgftext[left,
                        at=\pgfpoint{\pgf@circ@res@left-(\pgf@circ@qfp@pin@shift-\the\pgf@circ@count@a)*\pgf@circ@res@step}{\pgf@circ@res@down},
                        rotate=\rot]{\mytext}
                    % right
                    \pgf@circ@count@c=\numexpr3*\pgf@circ@count@b-\pgf@circ@count@a+1\relax
                    \pgftext[bottom,
                        at=\pgfpoint{\pgf@circ@res@right}{\pgf@circ@res@up+(\pgf@circ@qfp@pin@shift-\the\pgf@circ@count@a)*\pgf@circ@res@step},
                        rotate=\rot]{\mytext}
                    % top
                    \pgf@circ@count@c=\numexpr3*\pgf@circ@count@b+\pgf@circ@count@a\relax
                    \pgftext[right,
                        at=\pgfpoint{\pgf@circ@res@right+(\pgf@circ@qfp@pin@shift-\the\pgf@circ@count@a)*\pgf@circ@res@step}{\pgf@circ@res@up},
                        rotate=\rot]{\mytext}
                \or %rotation 180
                    % left
                    \pgf@circ@count@c=\pgf@circ@count@a
                    \pgftext[right,
                        at=\pgfpoint{\pgf@circ@res@left}{\pgf@circ@res@up+(\pgf@circ@qfp@pin@shift-\the\pgf@circ@count@a)*\pgf@circ@res@step},
                        rotate=\rot]{\mytext}
                    % bottom
                    \pgf@circ@count@c=\numexpr\pgf@circ@count@b+\pgf@circ@count@a\relax
                    \pgftext[top,
                        at=\pgfpoint{\pgf@circ@res@left-(\pgf@circ@qfp@pin@shift-\the\pgf@circ@count@a)*\pgf@circ@res@step}{\pgf@circ@res@down},
                        rotate=\rot]{\mytext}
                    % right
                    \pgf@circ@count@c=\numexpr3*\pgf@circ@count@b-\pgf@circ@count@a+1\relax
                    \pgftext[left,
                        at=\pgfpoint{\pgf@circ@res@right}{\pgf@circ@res@up+(\pgf@circ@qfp@pin@shift-\the\pgf@circ@count@a)*\pgf@circ@res@step},
                        rotate=\rot]{\mytext}
                    % top
                    \pgf@circ@count@c=\numexpr3*\pgf@circ@count@b+\pgf@circ@count@a\relax
                    \pgftext[bottom,
                        at=\pgfpoint{\pgf@circ@res@right+(\pgf@circ@qfp@pin@shift-\the\pgf@circ@count@a)*\pgf@circ@res@step}{\pgf@circ@res@up},
                        rotate=\rot]{\mytext}
                \or % rotation +90
                    % left
                    \pgf@circ@count@c=\pgf@circ@count@a
                    \pgftext[bottom,
                        at=\pgfpoint{\pgf@circ@res@left}{\pgf@circ@res@up+(\pgf@circ@qfp@pin@shift-\the\pgf@circ@count@a)*\pgf@circ@res@step},
                        rotate=\rot]{\mytext}
                    % bottom
                    \pgf@circ@count@c=\numexpr\pgf@circ@count@b+\pgf@circ@count@a\relax
                    \pgftext[right,
                        at=\pgfpoint{\pgf@circ@res@left-(\pgf@circ@qfp@pin@shift-\the\pgf@circ@count@a)*\pgf@circ@res@step}{\pgf@circ@res@down},
                        rotate=\rot]{\mytext}
                    % right
                    \pgf@circ@count@c=\numexpr3*\pgf@circ@count@b-\pgf@circ@count@a+1\relax
                    \pgftext[top,
                        at=\pgfpoint{\pgf@circ@res@right}{\pgf@circ@res@up+(\pgf@circ@qfp@pin@shift-\the\pgf@circ@count@a)*\pgf@circ@res@step},
                        rotate=\rot]{\mytext}
                    % top
                    \pgf@circ@count@c=\numexpr3*\pgf@circ@count@b+\pgf@circ@count@a\relax
                    \pgftext[left,
                        at=\pgfpoint{\pgf@circ@res@right+(\pgf@circ@qfp@pin@shift-\the\pgf@circ@count@a)*\pgf@circ@res@step}{\pgf@circ@res@up},
                        rotate=\rot]{\mytext}
                \fi
                \advance\pgf@circ@count@a-1\relax%
                \repeatpgfmathloop
            \fi
            \endpgfscope
            \ifdim\pgf@circ@res@other>0pt
            \pgfscope
                \pgfsetlinewidth{\ctikzvalof{multipoles/external pins thickness}\pgflinewidth}
                \pgf@circ@count@a=\numpins%
                \divide\pgf@circ@count@a by 4 \pgf@circ@count@b=\pgf@circ@count@a
                \pgfmathloop%
                \ifnum\pgf@circ@count@a>0
                    \edef\padfrac{\ctikzvalof{multipoles/external pad fraction}}
                    \ifnum\padfrac>0
                        \pgf@circ@res@temp=\pgf@circ@res@step\divide\pgf@circ@res@temp by \padfrac
                        % left side pads
                        \pgfpathmoveto{\pgfpoint{\pgf@circ@res@left}{\pgf@circ@res@temp+\pgf@circ@res@up+(\pgf@circ@qfp@pin@shift-\the\pgf@circ@count@a)*\pgf@circ@res@step}}
                        \pgfpathlineto{\pgfpoint{\pgf@circ@res@left-\pgf@circ@res@other}{\pgf@circ@res@temp+\pgf@circ@res@up+(\pgf@circ@qfp@pin@shift-\the\pgf@circ@count@a)*\pgf@circ@res@step}}
                        \pgfpathlineto{\pgfpoint{\pgf@circ@res@left-\pgf@circ@res@other}{-\pgf@circ@res@temp+\pgf@circ@res@up+(\pgf@circ@qfp@pin@shift-\the\pgf@circ@count@a)*\pgf@circ@res@step}}
                        \pgfpathlineto{\pgfpoint{\pgf@circ@res@left}{-\pgf@circ@res@temp+\pgf@circ@res@up+(\pgf@circ@qfp@pin@shift-\the\pgf@circ@count@a)*\pgf@circ@res@step}}
                        % bottom side pads
                        \pgfpathmoveto{\pgfpoint{-\pgf@circ@res@temp+\pgf@circ@res@left-(\pgf@circ@qfp@pin@shift-\the\pgf@circ@count@a)*\pgf@circ@res@step}{\pgf@circ@res@down}}
                        \pgfpathlineto{\pgfpoint{-\pgf@circ@res@temp+\pgf@circ@res@left-(\pgf@circ@qfp@pin@shift-\the\pgf@circ@count@a)*\pgf@circ@res@step}{\pgf@circ@res@down-\pgf@circ@res@other}}
                        \pgfpathlineto{\pgfpoint{\pgf@circ@res@temp+\pgf@circ@res@left-(\pgf@circ@qfp@pin@shift-\the\pgf@circ@count@a)*\pgf@circ@res@step}{\pgf@circ@res@down-\pgf@circ@res@other}}
                        \pgfpathlineto{\pgfpoint{\pgf@circ@res@temp+\pgf@circ@res@left-(\pgf@circ@qfp@pin@shift-\the\pgf@circ@count@a)*\pgf@circ@res@step}{\pgf@circ@res@down}}
                        % right side pads
                        \pgfpathmoveto{\pgfpoint{\pgf@circ@res@right}{\pgf@circ@res@temp+\pgf@circ@res@up+(\pgf@circ@qfp@pin@shift-\the\pgf@circ@count@a)*\pgf@circ@res@step}}
                        \pgfpathlineto{\pgfpoint{\pgf@circ@res@right+\pgf@circ@res@other}{\pgf@circ@res@temp+\pgf@circ@res@up+(\pgf@circ@qfp@pin@shift-\the\pgf@circ@count@a)*\pgf@circ@res@step}}
                        \pgfpathlineto{\pgfpoint{\pgf@circ@res@right+\pgf@circ@res@other}{-\pgf@circ@res@temp+\pgf@circ@res@up+(\pgf@circ@qfp@pin@shift-\the\pgf@circ@count@a)*\pgf@circ@res@step}}
                        \pgfpathlineto{\pgfpoint{\pgf@circ@res@right}{-\pgf@circ@res@temp+\pgf@circ@res@up+(\pgf@circ@qfp@pin@shift-\the\pgf@circ@count@a)*\pgf@circ@res@step}}
                        % top side pads
                        \pgfpathmoveto{\pgfpoint{\pgf@circ@res@temp+\pgf@circ@res@right+(\pgf@circ@qfp@pin@shift-\the\pgf@circ@count@a)*\pgf@circ@res@step}{\pgf@circ@res@up}}
                        \pgfpathlineto{\pgfpoint{\pgf@circ@res@temp+\pgf@circ@res@right+(\pgf@circ@qfp@pin@shift-\the\pgf@circ@count@a)*\pgf@circ@res@step}{\pgf@circ@res@up+\pgf@circ@res@other}}
                        \pgfpathlineto{\pgfpoint{-\pgf@circ@res@temp+\pgf@circ@res@right+(\pgf@circ@qfp@pin@shift-\the\pgf@circ@count@a)*\pgf@circ@res@step}{\pgf@circ@res@up+\pgf@circ@res@other}}
                        \pgfpathlineto{\pgfpoint{-\pgf@circ@res@temp+\pgf@circ@res@right+(\pgf@circ@qfp@pin@shift-\the\pgf@circ@count@a)*\pgf@circ@res@step}{\pgf@circ@res@up}}
                    \else
                        % left side pins
                        \pgfpathmoveto{\pgfpoint{\pgf@circ@res@left}{\pgf@circ@res@up+(\pgf@circ@qfp@pin@shift-\the\pgf@circ@count@a)*\pgf@circ@res@step}}
                        \pgfpathlineto{\pgfpoint{\pgf@circ@res@left-\pgf@circ@res@other}{\pgf@circ@res@up+(\pgf@circ@qfp@pin@shift-\the\pgf@circ@count@a)*\pgf@circ@res@step}}
                        % bottom side pins
                        \pgfpathmoveto{\pgfpoint{\pgf@circ@res@left-(\pgf@circ@qfp@pin@shift-\the\pgf@circ@count@a)*\pgf@circ@res@step}{\pgf@circ@res@down}}
                        \pgfpathlineto{\pgfpoint{\pgf@circ@res@left-(\pgf@circ@qfp@pin@shift-\the\pgf@circ@count@a)*\pgf@circ@res@step}{\pgf@circ@res@down-\pgf@circ@res@other}}
                        % right side pins
                        \pgfpathmoveto{\pgfpoint{\pgf@circ@res@right}{\pgf@circ@res@up+(\pgf@circ@qfp@pin@shift-\the\pgf@circ@count@a)*\pgf@circ@res@step}}
                        \pgfpathlineto{\pgfpoint{\pgf@circ@res@right+\pgf@circ@res@other}{\pgf@circ@res@up+(\pgf@circ@qfp@pin@shift-\the\pgf@circ@count@a)*\pgf@circ@res@step}}
                        % top side pins
                        \pgfpathmoveto{\pgfpoint{\pgf@circ@res@right+(\pgf@circ@qfp@pin@shift-\the\pgf@circ@count@a)*\pgf@circ@res@step}{\pgf@circ@res@up}}
                        \pgfpathlineto{\pgfpoint{\pgf@circ@res@right+(\pgf@circ@qfp@pin@shift-\the\pgf@circ@count@a)*\pgf@circ@res@step}{\pgf@circ@res@up+\pgf@circ@res@other}}
                    \fi
                    \advance\pgf@circ@count@a-1\relax%
                \repeatpgfmathloop
                \pgfusepath{stroke}
            \endpgfscope
            \fi
        }%
        % \pgf@sh@s@<name of the shape here> contains all the code for the shape
        % and is executed just before a node is drawn.
        \pgfutil@g@addto@macro\pgf@sh@s@qfpchip{%
            % Start with the maximum pin number and go backwards.
            \pgf@circ@count@a=\numpins%
            \pgfmathloop%
            \ifnum\pgf@circ@count@a>0
                \expandafter\xdef\csname pgf@anchor@qfpchip@pin\space\the\pgf@circ@count@a\endcsname{%
                    \noexpand\pgf@circ@qfppinanchor{\the\pgf@circ@count@a}{1}%
                }
                \expandafter\xdef\csname pgf@anchor@qfpchip@bpin\space\the\pgf@circ@count@a\endcsname{%
                    \noexpand\pgf@circ@qfppinanchor{\the\pgf@circ@count@a}{0}%
                }
                \advance\pgf@circ@count@a-1\relax%
                \repeatpgfmathloop%
            }%
        }

%% anchors for DIP
\def\pgf@circ@dippinanchor#1#2{% #1: pin number #2: 0 for border pin, 1 for external pin
    \c@pgf@countc=\numpins\relax
    \divide\c@pgf@countc by 2
    \ifnum #1 > \the\c@pgf@countc
        % right side
        \pgfpoint{\width/2+#2*\extshift}{-\height/2+(\pgf@circ@dip@pin@shift-\c@pgf@countc+#1-1)*\chipspacing}
    \else
        \pgfpoint{-\width/2-#2*\extshift}{\height/2+(\pgf@circ@dip@pin@shift-#1)*\chipspacing}
\fi
}

%% anchors for QFP
\def\pgf@circ@qfppinanchor#1#2{% #1: pin number #2: 0 for border pin, 1 for external pin
    \c@pgf@countc=\numpins\relax
    \divide\c@pgf@countc by 4
    \ifnum #1 > \the\c@pgf@countc
        \c@pgf@countb=\c@pgf@countc \multiply \c@pgf@countb by 2
        \ifnum #1 > \the\c@pgf@countb
            \c@pgf@countb=\c@pgf@countc \multiply \c@pgf@countb by 3
            \ifnum #1 > \the\c@pgf@countb
                % 3*npins/4 < pin, top side
                \pgfpoint{\width/2+(\pgf@circ@qfp@pin@shift+\c@pgf@countb-#1)*\chipspacing}{\height/2+#2*\extshift}%
            \else
                % 2*npins/4 < pin <= 3*npins/4, right side
                \pgfpoint{\width/2+#2*\extshift}{\height/2+(\pgf@circ@qfp@pin@shift-\c@pgf@countb+#1-1)*\chipspacing}%
            \fi
        \else
            %  npins/4 < pin <= 2*npins/4, bottom side
            \pgfpoint{\width/2+(\pgf@circ@qfp@pin@shift-\c@pgf@countb+#1-1)*\chipspacing}{-\height/2-#2*\extshift}%
        \fi
    \else
        % <= npins/4, left side
        \pgfpoint{-\width/2-#2*\extshift}{\height/2+(\pgf@circ@qfp@pin@shift-#1)*\chipspacing}%
    \fi
}

%%%%%%%%%%%%%%%%%
%% Rotary Switch
%%%%%%%%%%%%%%%%%

\pgfdeclareshape{rotaryswitch}
{
    \savedmacro{\ctikzclass}{\edef\ctikzclass{switches}}
    \saveddimen{\scaledRlen}{\pgfmathsetlength{\pgf@x}{\ctikzvalof{\ctikzclass/scale}\pgf@circ@Rlen}}
    \savedanchor\northeast{%
        \pgfmathsetlength{\pgf@circ@scaled@Rlen}{\ctikzvalof{\ctikzclass/scale}\pgf@circ@Rlen}
        % this strange value makes the 2-pole rotary switch equal to the 2 poles cute spdt
        % the magic number is 0.25/cos(35)
        % try to recalculate it for the actual switch
        \pgf@circ@res@temp=\ctikzvalof{tripoles/spdt/width}\pgf@circ@scaled@Rlen
        \pgf@circ@res@temp=.3052\pgf@circ@res@temp
        \edef\a{\ctikzvalof{multipoles/rotary/angle}}
        \edef\r{\ctikzvalof{nodes width}}
        \pgfmathsetlength{\pgf@y}{\r*\pgf@circ@scaled@Rlen +(\a>90 ? 2 : 2*sin(\a))*\pgf@circ@res@temp}
        \pgfmathsetlength{\pgf@x}{\r*\pgf@circ@scaled@Rlen + \pgf@circ@res@temp}
    }
    \savedanchor\northwest{%
        \pgfmathsetlength{\pgf@circ@scaled@Rlen}{\ctikzvalof{\ctikzclass/scale}\pgf@circ@Rlen}
        % this strange value makes the 2-pole rotary switch equal to the 2 poles cute spdt
        % the magic number is 0.25/cos(35)
        % try to recalculate it for the actual switch
        \pgf@circ@res@temp=\ctikzvalof{tripoles/spdt/width}\pgf@circ@scaled@Rlen
        \pgf@circ@res@temp=.3052\pgf@circ@res@temp
        \edef\a{\ctikzvalof{multipoles/rotary/angle}}
        \edef\r{\ctikzvalof{nodes width}}
        \pgfmathsetlength{\pgf@y}{\r*\pgf@circ@scaled@Rlen +(\a>90 ? 2 : 2*sin(\a))*\pgf@circ@res@temp}
        \pgfmathsetlength{\pgf@x}{-\r*\pgf@circ@scaled@Rlen - (\a<90 ? 1 : 1-2*cos(\a))*\pgf@circ@res@temp}
    }
    \savedanchor\central{%
        \pgfmathsetlength{\pgf@circ@scaled@Rlen}{\ctikzvalof{\ctikzclass/scale}\pgf@circ@Rlen}
        % this strange value makes the 2-pole rotary switch equal to the 2 poles cute spdt
        % the magic number is 0.25/cos(35)
        % try to recalculate it for the actual switch
        \pgf@circ@res@temp=\ctikzvalof{tripoles/spdt/width}\pgf@circ@scaled@Rlen
        \pgf@circ@res@temp=.3052\pgf@circ@res@temp
        \edef\a{\ctikzvalof{multipoles/rotary/angle}}
        \edef\r{\ctikzvalof{nodes width}}
        \pgfmathsetlength{\pgf@y}{\r*\pgf@circ@scaled@Rlen +(\a>90 ? 2 : 2*sin(\a))*\pgf@circ@res@temp}
        \pgfmathsetlength{\pgf@x}{(\a<90 ? 0 : cos(\a))*\pgf@circ@res@temp}
    }
    % external square limits
    \savedanchor\extnorthwest{%
        \pgfmathsetlength{\pgf@circ@scaled@Rlen}{\ctikzvalof{\ctikzclass/scale}\pgf@circ@Rlen}
        \pgf@x=-\ctikzvalof{tripoles/spdt/width}\pgf@circ@scaled@Rlen
        % this strange value makes the 2-pole rotary switch equal to the 2 poles cute spdt
        \pgf@x=.3052\pgf@x % the magic number is 0.25/cos(35)
        \pgf@x=2.5\pgf@x % external square size
        \pgf@y=-\pgf@x %square thing when angle=180?
    }
    \saveddimen{\width}{
        \pgfmathsetlength{\pgf@circ@scaled@Rlen}{\ctikzvalof{\ctikzclass/scale}\pgf@circ@Rlen}
        \pgfmathsetlength\pgf@x{0.3052*\pgf@circ@scaled@Rlen*\ctikzvalof{tripoles/spdt/width}}}
    % radius of the connector
    % This is the radius of the "ocirc" shape (see pgfcircshapes.tex)
    \saveddimen{\radius}{\pgfmathsetlength\pgf@x{\pgf@circ@Rlen*\ctikzvalof{nodes width}}}
    % shapename
    \savedmacro{\thisshape}{\def\thisshape{\tikz@fig@name}}
    % shape type
    \savedmacro{\cshape}{\def\cshape{\ctikzvalof{multipoles/rotary/shape}}}
    \savedmacro{\channels}{\def\channels{\ctikzvalof{multipoles/rotary/channels}}}
    \savedmacro{\angle}{\def\angle{\ctikzvalof{multipoles/rotary/angle}}}
    \savedmacro{\wiper}{\def\wiper{\ctikzvalof{multipoles/rotary/wiper}}}
    \savedmacro{\stepa}{\pgfmathsetmacro{\stepa}{2*\ctikzvalof{multipoles/rotary/angle}/(\ctikzvalof{multipoles/rotary/channels}-1)}}
    % mid of the lever, to stack switches
    %\anchor{mid}{\midlever}
    \anchor{mid}{\northwest
        \pgf@circ@res@temp=-\pgf@x
        \pgfmathsetlength{\pgf@x}{\pgf@circ@res@temp*(-1+cos(\wiper))}
        \pgfmathsetlength{\pgf@y}{\pgf@circ@res@temp*sin(\wiper)}
    }
    % center anchors
    \anchor{cin}{ \northwest \pgf@y=0pt\advance\pgf@x by \radius}
    % horizontal angles
    \anchor{in}{ \northwest \pgf@y=0pt}
    \anchor{ain}{ \northwest \pgf@y=0pt}

    \anchor{center}{ \central \pgf@y=0pt }
    \anchor{east}{ \northeast \pgf@y=0pt }
    \anchor{west}{ \northwest \pgf@y=0pt }
    \anchor{south}{ \central \pgf@y=-\pgf@y }
    \anchor{north}{ \central }
    \anchor{south west}{ \northwest \pgf@y=-\pgf@y }
    \anchor{north east}{ \northeast }
    \anchor{north west}{ \northwest }
    \anchor{south east}{ \northeast \pgf@y=-\pgf@y }

    \anchor{ext center}{ \pgf@y=0pt \pgf@x=0pt \advance\pgf@x by -\width}
    \anchor{ext east}{ \extnorthwest \pgf@y=0pt \pgf@x=-\pgf@x \advance\pgf@x by -\width}
    \anchor{ext west}{ \extnorthwest \pgf@y=0pt \advance\pgf@x by -\width}
    \anchor{ext south}{ \extnorthwest \pgf@x=0pt \pgf@y=-\pgf@y \advance\pgf@x by -\width}
    \anchor{ext north}{ \extnorthwest \pgf@x=0pt \advance\pgf@x by -\width}
    \anchor{ext south west}{ \extnorthwest \pgf@y=-\pgf@y \advance\pgf@x by -\width}
    \anchor{ext north east}{ \extnorthwest \pgf@x=-\pgf@x \advance\pgf@x by -\width}
    \anchor{ext north west}{ \extnorthwest \advance\pgf@x by -\width}
    \anchor{ext south east}{ \extnorthwest \pgf@x=-\pgf@x \pgf@y=-\pgf@y \advance\pgf@x by -\width}

    \backgroundpath{
        \pgfsetcolor{\ctikzvalof{color}}
        \pgf@circ@res@right = \width
        \pgf@circ@res@left = -\width

        \pgfscope %wiper
        % This is the radius of the "ocirc" shape (see pgfcircshapes.tex)
        \pgf@circ@res@temp=\radius\relax
        \pgf@circ@res@temp=\ctikzvalof{multipoles/rotary/thickness}\pgf@circ@res@temp
        \pgfsetlinewidth{2\pgf@circ@res@temp}
        \pgfpathmoveto{\pgfpoint{\pgf@circ@res@left}{0pt}}
        \pgfpathlineto{\pgfpointadd{\pgfpoint{\pgf@circ@res@left}{0pt}}{\pgfpointpolar{\wiper}{2\pgf@circ@res@right}}}
        \pgfsetroundcap\pgfusepath{draw}
        \endpgfscope

        \ifpgf@circ@rotaryarrow
            \pgfscope % arrow
                \ifpgf@circ@rotaryarrow@ccw\pgfsetarrowsstart{latexslim}\fi
                \pgf@circ@setlinewidth{bipoles}{\pgflinewidth}
                \pgftransformshift{\pgfpoint{\pgf@circ@res@left}{0pt}} % center of cin node
                \pgftransformrotate{\wiper}
                \pgfpathmoveto{\pgfpointpolar{50}{1.0\pgf@circ@res@right}}
                \pgfpatharc{50}{-50}{1.0\pgf@circ@res@right}
                \ifpgf@circ@rotaryarrow@cw\pgfsetarrowsend{latexslim}\fi
                \pgfusepath{draw}
            \endpgfscope
        \fi

        % \typeout{CHANNELS\space\channels\space ANGLE\space\angle STEPA\space\stepa}
        \pgf@circ@count@a=\channels\relax
        \pgfmathsetmacro{\currenta}{-\angle}
        \pgfmathloop%
        \ifnum\pgf@circ@count@a>0
            % \typeout{LOOPIN\space\space\the\pgf@circ@count@a\space CURRENTA\space\currenta\space RIGHT\space\the\pgf@circ@res@right}
            \pgfscope
                \pgftransformshift{\pgfpointadd{\pgfpoint{\pgf@circ@res@left}{0pt}}{\pgfpointpolar{\currenta}{2\pgf@circ@res@right}}}
                \pgfnode{\cshape}{center}{}{\thisshape-out \the\pgf@circ@count@a}{\pgfusepath{stroke}}
            \endpgfscope
            \pgfmathsetmacro{\currenta}{\currenta+\stepa}
            % \typeout{LOOPOUT\space\the\pgf@circ@count@a\space CURRENTA\space\currenta\space RIGHT\space\the\pgf@circ@res@right}
            \advance\pgf@circ@count@a by -1\relax%
        \repeatpgfmathloop

        \pgfscope % input
        \pgftransformshift{\pgfpoint{\pgf@circ@res@left}{0pt}}
        \pgfnode{\cshape}{center}{}{\thisshape-in}{\pgfusepath{stroke}}
        \endpgfscope
    }
    % \pgf@sh@s@<name of the shape here> contains all the code for the shape
    % and is executed just before a node is drawn.
    \pgfutil@g@addto@macro\pgf@sh@s@rotaryswitch{%
        % Start with the maximum pin number and go backwards.
        \pgf@circ@count@a=\channels\relax
        \pgfmathloop%
        \ifnum\pgf@circ@count@a>0
        % we will create two anchors per pin: the "normal one" like `pin 1` for the
        % electrical contact, and the "border one" like `bpin 1` for labels.
        % they will coincide if `external pins width` is set to 0.
        \expandafter\xdef\csname pgf@anchor@rotaryswitch@out\space\the\pgf@circ@count@a\endcsname{%
            \noexpand\pgf@circ@rotaryanchor{\the\pgf@circ@count@a}{1}{0}%
        }
        \expandafter\xdef\csname pgf@anchor@rotaryswitch@cout\space\the\pgf@circ@count@a\endcsname{%
            \noexpand\pgf@circ@rotaryanchor{\the\pgf@circ@count@a}{0}{0}%
        }
        \expandafter\xdef\csname pgf@anchor@rotaryswitch@aout\space\the\pgf@circ@count@a\endcsname{%
            \noexpand\pgf@circ@rotaryanchor{\the\pgf@circ@count@a}{0}{1}%
        }
        \expandafter\xdef\csname pgf@anchor@rotaryswitch@sqout\space\the\pgf@circ@count@a\endcsname{%
            \noexpand\pgf@circ@rotarysqanchor{\the\pgf@circ@count@a}{0}%
        }
        \advance\pgf@circ@count@a by -1\relax%
        \repeatpgfmathloop%
    }%
}

\def\pgf@circ@rotaryanchor#1#2#3{% #1: numero del pin; #2: 1 - x pos, 0 - center; #3 0: inner, 1 outer
    \pgf@circ@res@temp=\width
    \pgfmathsetmacro{\myangle}{\angle-(#1-1)*\stepa}
    \pgfmathsetlength{\pgf@x}{2*(\pgf@circ@res@temp+#3*\radius/2)*cos(\myangle))+#2*\radius}
    \pgfmathsetlength{\pgf@y}{2*(\pgf@circ@res@temp+#3*\radius/2)*sin(\myangle)}
    \advance\pgf@x by -\pgf@circ@res@temp
}

\def\pgf@circ@rotarysqanchor#1{% external square anchors
    \pgf@circ@res@temp=\width
    \pgfmathsetmacro{\myangle}{\angle-(#1-1)*\stepa}
    \pgfpointborderrectangle{\pgfpointpolar{\myangle}{1pt}}{\pgfpoint{2.5\pgf@circ@res@temp}{2.5\pgf@circ@res@temp}}
    \advance\pgf@x by -\pgf@circ@res@temp
}

%%%%%%%%%%%%%%%%%%%%%%%%%%
% Seven segments displays
%%%%%%%%%%%%%%%%%%%%%%%%%%

\pgfdeclareshape{bare7seg}{
    \savedmacro{\ctikzclass}{\edef\ctikzclass{displays}}
    \saveddimen{\scaledRlen}{\pgfmathsetlength{\pgf@x}{\ctikzvalof{\ctikzclass/scale}\pgf@circ@Rlen}}
    \savedmacro{\dotstatus}{\edef\dotstatus{\pgf@circ@sevenseg@dotstate}}
    \saveddimen{\dotspace}{% the dot is on the right, and occupy the same as the thickness
        \ifpgf@circ@sevenseg@dot
            \pgfmathsetlength{\pgf@x}{\ctikzvalof{seven seg/thickness}}
        \else
            \pgf@x=0pt
        \fi
    }
    % The object extension is more or less (-width/2,-width) to (width/2,width)
    % and adjusted for line thickness (both sides) and eventually the dot
    \saveddimen{\width}{
        \pgfmathsetlength{\pgf@circ@scaled@Rlen}{\ctikzvalof{\ctikzclass/scale}\pgf@circ@Rlen}
        \pgfmathsetlength{\pgf@x}{\ctikzvalof{seven seg/width}*\pgf@circ@scaled@Rlen}}
    \saveddimen{\gap}{\pgfmathsetlength{\pgf@x}{\ctikzvalof{seven seg/segment sep}}}
    \saveddimen{\boxgap}{\pgfmathsetlength{\pgf@x}{\ctikzvalof{seven seg/box sep}}}
    \savedanchor{\southwest}{% both negative
        \pgfmathsetlength{\pgf@circ@scaled@Rlen}{\ctikzvalof{\ctikzclass/scale}\pgf@circ@Rlen}
        \pgfmathsetlength{\pgf@x}{-0.5*\ctikzvalof{seven seg/width}*\pgf@circ@scaled@Rlen
        -0.5*\ctikzvalof{seven seg/thickness}-\ctikzvalof{seven seg/box sep}}
        \pgfmathsetlength{\pgf@y}{-\ctikzvalof{seven seg/width}*\pgf@circ@scaled@Rlen
        -0.5*\ctikzvalof{seven seg/thickness}-\ctikzvalof{seven seg/box sep}}
    }
    \savedanchor{\northeast}{% both positive
        \pgfmathsetlength{\pgf@circ@scaled@Rlen}{\ctikzvalof{\ctikzclass/scale}\pgf@circ@Rlen}
        \ifpgf@circ@sevenseg@dot
            \pgfmathsetlength{\pgf@circ@res@other}{\ctikzvalof{seven seg/thickness}}
        \else
            \pgf@circ@res@other=0pt
        \fi
        \pgfmathsetlength{\pgf@x}{0.5*\ctikzvalof{seven seg/width}*\pgf@circ@scaled@Rlen
        +0.5*\ctikzvalof{seven seg/thickness}+\pgf@circ@res@other+\ctikzvalof{seven seg/box sep}}
        \pgfmathsetlength{\pgf@y}{\ctikzvalof{seven seg/width}*\pgf@circ@scaled@Rlen
        +0.5*\ctikzvalof{seven seg/thickness}+\ctikzvalof{seven seg/box sep}}
    }
    \savedanchor{\topright}{% anchor without the box sep and the thickness
        \pgfmathsetlength{\pgf@circ@scaled@Rlen}{\ctikzvalof{\ctikzclass/scale}\pgf@circ@Rlen}
        \pgfmathsetlength{\pgf@x}{0.5*\ctikzvalof{seven seg/width}*\pgf@circ@scaled@Rlen}
        \pgfmathsetlength{\pgf@y}{\ctikzvalof{seven seg/width}*\pgf@circ@scaled@Rlen}
    }
    \anchor{center}{\pgfpointorigin}
    \anchor{north west}{\southwest\pgf@y=-\pgf@y}
    \anchor{north east}{\northeast}
    \anchor{south east}{\northeast\pgf@y=-\pgf@y}
    \anchor{south west}{\southwest}
    \anchor{north}{\northeast\pgf@x=0pt}
    \anchor{east}{\northeast\pgf@y=0pt}
    \anchor{south}{\southwest\pgf@x=0pt}
    \anchor{west}{\southwest\pgf@y=0pt}
    \anchor{a}{\topright\pgf@x=0pt}
    \anchor{b}{\topright\pgf@y=0.5\pgf@y}
    \anchor{c}{\topright\pgf@y=-0.5\pgf@y}
    \anchor{d}{\topright\pgf@y=-\pgf@y\pgf@x=0pt}
    \anchor{e}{\topright\pgf@x=-\pgf@x\pgf@y=-0.5\pgf@y}
    \anchor{f}{\topright\pgf@x=-\pgf@x\pgf@y=0.5\pgf@y}
    \anchor{g}{\pgfpointorigin}
    \anchor{dot}{\topright\pgf@y=-\pgf@y\advance\pgf@x by \dotspace}
    \behindbackgroundpath{%
        \southwest % I do not want the dot here, it will stick out
        \pgf@circ@res@up = -\pgf@y
        \pgf@circ@res@down = \pgf@y
        \pgf@circ@res@right = \pgf@x
        \pgf@circ@res@left = -\pgf@x
        \pgfscope
        \pgf@circ@setlinewidth{multipoles}{\pgflinewidth}
        \pgfsetcolor{\ctikzvalof{color}}
        \pgfpathrectanglecorners%
        {\pgfpoint{\pgf@circ@res@right}{\pgf@circ@res@down}}
        {\pgfpoint{\pgf@circ@res@left+\dotspace}{\pgf@circ@res@up}}
        \ifpgf@circ@sevenseg@box
            \pgf@circ@draworfill
        \else
            \pgf@circ@maybefill
        \fi
        \endpgfscope
        \edef\bits{\ctikzvalof{seven seg/bits}}
        \pgfscope
            \pfg@circ@sseg@drawbits{\bits}
        \endpgfscope
        \pgfscope
            \ifpgf@circ@sevenseg@dot
                \pgf@circ@sseg@drawdots
            \fi
        \endpgfscope
    }
}

\def\pgf@circ@sseg@splitbits#1#2#3#4#5#6#7\relax{%split the seven bits
    \edef\@@a{#1}\edef\@@b{#2}\edef\@@c{#3}\edef\@@d{#4}\edef\@@e{#5}\edef\@@f{#6}\edef\@@g{#7}%
}
\def\pgf@circ@sseg@drawone#1#2#3#4#5{% #1 on off the x1, y1, x2 , y2
    \ifnum #1 > 0\relax
        \pgfsetcolor{\ctikzvalof{seven seg/color on}}
    \else
        \pgfsetcolor{\ctikzvalof{seven seg/color off}}
    \fi
    \pgfpathmoveto{\pgfpoint{#2}{#3}}
    \pgfpathlineto{\pgfpoint{#4}{#5}}
    \pgfusepath{draw}
}
\def\pfg@circ@sseg@drawbits#1{% #1 must be 7 bits
    \expandafter\pgf@circ@sseg@splitbits#1\relax% a bit of magic...
    \pgfmathsetlength{\pgf@circ@res@other}{0.5*\ctikzvalof{seven seg/thickness}}
    \pgfsetlinewidth{\ctikzvalof{seven seg/thickness}}
    % \pgfsetroundcap
    \pgfsetarrowsstart{Triangle Cap[]}
    \pgfsetarrowsend{Triangle Cap[]}
    % segments
    \pgf@circ@sseg@drawone{\@@a}{-\width/2+\gap}{\width}{\width/2-\gap}{\width}
    \pgf@circ@sseg@drawone{\@@b}{\width/2}{\width-\gap}{\width/2}{0pt+\gap}
    \pgf@circ@sseg@drawone{\@@c}{\width/2}{0pt-\gap}{\width/2}{-\width+\gap}
    \pgf@circ@sseg@drawone{\@@d}{\width/2-\gap}{-\width}{-\width/2+\gap}{-\width}
    \pgf@circ@sseg@drawone{\@@e}{-\width/2}{-\width+\gap}{-\width/2}{0pt-\gap}
    \pgf@circ@sseg@drawone{\@@f}{-\width/2}{0pt+\gap}{-\width/2}{\width-\gap}
    \pgf@circ@sseg@drawone{\@@g}{-\width/2+\gap}{0pt}{\width/2-\gap}{0pt}
}
\def\pgf@circ@sseg@drawdots{% dots
    \edef\what{empty}
    \ifx\what\pgf@circ@sevenseg@dotstate
        % do nothing
    \else
        \pgfmathsetlength{\pgf@circ@res@other}{0.5*\ctikzvalof{seven seg/thickness}}
        \edef\what{off}
        \ifx\what\pgf@circ@sevenseg@dotstate
            % dot off
            \pgfsetfillcolor{\ctikzvalof{seven seg/color off}}
            \pgfsetcolor{\ctikzvalof{seven seg/color off}}
        \else
            % dot on
            \pgfsetfillcolor{\ctikzvalof{seven seg/color on}}
            \pgfsetcolor{\ctikzvalof{seven seg/color on}}
        \fi
        \pgfpathcircle{\pgfpoint{\width/2+2*\pgf@circ@res@other}{-\width}}{\pgf@circ@res@other}
        \pgfusepath{draw,fill}
    \fi
}


%%%%%%
%%%%%% Flip-flops
%%%%%%

% Flip flops are a specialized kind of dipchip.
% they have a class by themselves

%% flip flop definitions --- by default empty
%% pin texts
\ctikzset{multipoles/flipflop/t1/.initial={}}
\ctikzset{multipoles/flipflop/t2/.initial={}}
\ctikzset{multipoles/flipflop/t3/.initial={}}
\ctikzset{multipoles/flipflop/t4/.initial={}}
\ctikzset{multipoles/flipflop/t5/.initial={}}
\ctikzset{multipoles/flipflop/t6/.initial={}}
\ctikzset{multipoles/flipflop/tu/.initial={}}
\ctikzset{multipoles/flipflop/td/.initial={}}
% pin clock wedge flags
\ctikzset{multipoles/flipflop/c1/.initial={0}}
\ctikzset{multipoles/flipflop/c2/.initial={0}}
\ctikzset{multipoles/flipflop/c3/.initial={0}}
\ctikzset{multipoles/flipflop/c4/.initial={0}}
\ctikzset{multipoles/flipflop/c5/.initial={0}}
\ctikzset{multipoles/flipflop/c6/.initial={0}}
\ctikzset{multipoles/flipflop/cu/.initial={0}}
\ctikzset{multipoles/flipflop/cd/.initial={0}}
% pin negation circle flags
\ctikzset{multipoles/flipflop/n1/.initial={0}}
\ctikzset{multipoles/flipflop/n2/.initial={0}}
\ctikzset{multipoles/flipflop/n3/.initial={0}}
\ctikzset{multipoles/flipflop/n4/.initial={0}}
\ctikzset{multipoles/flipflop/n5/.initial={0}}
\ctikzset{multipoles/flipflop/n6/.initial={0}}
\ctikzset{multipoles/flipflop/nu/.initial={0}}
\ctikzset{multipoles/flipflop/nd/.initial={0}}


% Thanks to @marmot
\tikzset{flipflop def/.code=\pgfqkeys{\circuitikzbasekey/multipoles/flipflop}{#1}}

% default set of flip flops
\tikzset{
    % async
    latch/.style={flipflop, flipflop def={t1=D, t6=Q, t3=CLK, t4=\ctikztextnot{Q}}},
    flipflop SR/.style={flipflop, flipflop def={t1=S, t3=R, t6=Q, t4=\ctikztextnot{Q}}},
    % sync
    flipflop D/.style={flipflop, flipflop def={t1=D, t6=Q, c3=1, t4=\ctikztextnot{Q}}},
    flipflop T/.style={flipflop, flipflop def={t1=T, t6=Q, c3=1, t4=\ctikztextnot{Q}}},
    flipflop JK/.style={flipflop, flipflop def={t1=J, t3=K, c2=1, t6=Q, t4=\ctikztextnot{Q}}},
    % additional features
    add async SR/.style={flipflop def={tu={\ctikztextnot{SET}}, td={\ctikztextnot{RST}}}},
    dot on notQ/.style={flipflop def={t4={Q}, n4=1}},
}

%
% commodity macro to draw the clock wedges. They leave the size of the
% wedge in \pgf@circ@res@temp so that a possible label can be displaced.
%
\def\pgf@circ@do@wedge@left{
    \pgf@circ@res@temp=0pt\relax
    \ifnum\ctikzvalof{multipoles/flipflop/c\the\pgf@circ@count@c}>0
        \pgfpathmoveto{\pgfpoint{\pgf@circ@res@left}{\pgf@circ@res@up+(\pgf@circ@dip@pin@shift-\the\pgf@circ@count@a)*\pgf@circ@res@step+\wedge}}
        \pgfpathlineto{\pgfpoint{\pgf@circ@res@left+\wedge}{\pgf@circ@res@up+(\pgf@circ@dip@pin@shift-\the\pgf@circ@count@a)*\pgf@circ@res@step}}
        \pgfpathlineto{\pgfpoint{\pgf@circ@res@left}{\pgf@circ@res@up+(\pgf@circ@dip@pin@shift-\the\pgf@circ@count@a)*\pgf@circ@res@step-\wedge}}
        \pgfusepath{stroke}
        \pgf@circ@res@temp=\wedge
    \fi
}
\def\pgf@circ@do@wedge@right{
    \pgf@circ@res@temp=0pt\relax
    \ifnum\ctikzvalof{multipoles/flipflop/c\the\pgf@circ@count@c}>0
        \pgfpathmoveto{\pgfpoint{\pgf@circ@res@right}{\pgf@circ@res@up+(\pgf@circ@dip@pin@shift-\the\pgf@circ@count@a)*\pgf@circ@res@step+\wedge}}
        \pgfpathlineto{\pgfpoint{\pgf@circ@res@right-\wedge}{\pgf@circ@res@up+(\pgf@circ@dip@pin@shift-\the\pgf@circ@count@a)*\pgf@circ@res@step}}
        \pgfpathlineto{\pgfpoint{\pgf@circ@res@right}{\pgf@circ@res@up+(\pgf@circ@dip@pin@shift-\the\pgf@circ@count@a)*\pgf@circ@res@step-\wedge}}
        \pgfusepath{stroke}
        \pgf@circ@res@temp=-\wedge
    \fi
}
\def\pgf@circ@do@wedge@up{
    \pgf@circ@res@temp=0pt\relax
    \ifnum\ctikzvalof{multipoles/flipflop/cu}>0
        \pgfpathmoveto{\pgfpoint{-\wedge}{\pgf@circ@res@up}}
        \pgfpathlineto{\pgfpoint{0pt}{\pgf@circ@res@up-\wedge}}
        \pgfpathlineto{\pgfpoint{\wedge}{\pgf@circ@res@up}}
        \pgfusepath{stroke}
        \pgf@circ@res@temp=-\wedge
    \fi
}
\def\pgf@circ@do@wedge@down{
    \pgf@circ@res@temp=0pt\relax
    \ifnum\ctikzvalof{multipoles/flipflop/cd}>0
        \pgfpathmoveto{\pgfpoint{-\wedge}{\pgf@circ@res@down}}
        \pgfpathlineto{\pgfpoint{0pt}{\pgf@circ@res@down+\wedge}}
        \pgfpathlineto{\pgfpoint{\wedge}{\pgf@circ@res@down}}
        \pgfusepath{stroke}
        \pgf@circ@res@temp=\wedge
    \fi
}
% generic flip-flop shape
\pgfdeclareshape{flipflop}{
    \savedmacro{\ctikzclass}{\edef\ctikzclass{flipflops}}
    \saveddimen{\scaledRlen}{\pgfmathsetlength{\pgf@x}{\ctikzvalof{\ctikzclass/scale}\pgf@circ@Rlen}}
    \savedmacro{\thisshape}{\def\thisshape{\tikz@fig@name}}
    \savedmacro\numpins{\def\numpins{6}}
    \savedanchor\centerpoint{%
        \pgf@x=-.5\wd\pgfnodeparttextbox%
        \pgf@y=-.5\ht\pgfnodeparttextbox%
        \advance\pgf@y by+.5\dp\pgfnodeparttextbox%
    }%
    \savedanchor\origin{\pgfpoint{0pt}{0pt}}
    \anchor{center}{\origin}
    \anchor{text}{\centerpoint}% to adjust text
    \saveddimen\height{%
        \pgfmathsetlength{\pgf@circ@scaled@Rlen}{\ctikzvalof{\ctikzclass/scale}\pgf@circ@Rlen}
        \pgfmathsetlength\pgf@x{((\numpins)
        *\ctikzvalof{multipoles/flipflop/pin spacing})*\pgf@circ@scaled@Rlen/2}%
    }%
    \saveddimen{\chipspacing}{
        \pgfmathsetlength{\pgf@circ@scaled@Rlen}{\ctikzvalof{\ctikzclass/scale}\pgf@circ@Rlen}
        \pgfmathsetlength\pgf@x{\pgf@circ@scaled@Rlen*\ctikzvalof{multipoles/flipflop/pin spacing}}}
    \saveddimen{\width}{
        \pgfmathsetlength{\pgf@circ@scaled@Rlen}{\ctikzvalof{\ctikzclass/scale}\pgf@circ@Rlen}
        \pgfmathsetlength\pgf@x{\pgf@circ@scaled@Rlen*\ctikzvalof{multipoles/flipflop/width}}}
    \saveddimen{\extshift}{
        \pgfmathsetlength{\pgf@circ@scaled@Rlen}{\ctikzvalof{\ctikzclass/scale}\pgf@circ@Rlen}
        \pgfmathsetlength\pgf@x{\pgf@circ@scaled@Rlen*\ctikzvalof{multipoles/external pins width}}}
    % standard anchors
    \savedanchor\northwest{%
        \pgfmathsetlength{\pgf@circ@scaled@Rlen}{\ctikzvalof{\ctikzclass/scale}\pgf@circ@Rlen}
        \pgfmathsetlength\pgf@y{0.5*((\numpins)
        *\ctikzvalof{multipoles/flipflop/pin spacing})*\pgf@circ@scaled@Rlen/2}%
        \pgfmathsetlength\pgf@x{-0.5*\pgf@circ@scaled@Rlen*\ctikzvalof{multipoles/flipflop/width}}
    }
    \anchor{dot}{\northwest
        \pgfmathsetlength\pgf@x{\pgf@x + 0.3*\chipspacing}
        \pgfmathsetlength\pgf@y{\pgf@y - 0.3*\chipspacing}
    }
    \anchor{nw}{\northwest}
    \anchor{ne}{\northwest\pgf@x=-\pgf@x}
    \anchor{se}{\northwest\pgf@x=-\pgf@x\pgf@y=-\pgf@y}
    \anchor{sw}{\northwest\pgf@y=-\pgf@y}
    \anchor{north west}{\northwest}
    \anchor{north east}{\northwest\pgf@x=-\pgf@x}
    \anchor{south east}{\northwest\pgf@x=-\pgf@x \pgf@y=-\pgf@y}
    \anchor{south west}{\northwest\pgf@y=-\pgf@y}
    \anchor{n}{\northwest\pgf@x=0pt }
    \anchor{e}{\northwest\pgf@x=-\pgf@x\pgf@y=0pt }
    \anchor{s}{\northwest\pgf@x=0pt\pgf@y=-\pgf@y}
    \anchor{w}{\northwest\pgf@y=0pt }
    \anchor{north}{\northwest\pgf@x=0pt }
    \anchor{east}{\northwest\pgf@x=-\pgf@x\pgf@y=0pt }
    \anchor{south}{\northwest\pgf@x=0pt\pgf@y=-\pgf@y}
    \anchor{west}{\northwest\pgf@y=0pt }
    % upper and lower pin
    \anchor{up}{\northwest\pgf@x=0pt\advance\pgf@y by\extshift }
    \anchor{down}{\northwest\pgf@x=0pt\pgf@y=-\pgf@y\advance\pgf@y by-\extshift}
    \anchor{bup}{\northwest\pgf@x=0pt }
    \anchor{bdown}{\northwest\pgf@x=0pt\pgf@y=-\pgf@y}
    % start drawing
    \backgroundpath{%
        \northwest
        \pgf@circ@res@up = \pgf@y
        \pgf@circ@res@down = -\pgf@y
        \pgf@circ@res@right = -\pgf@x
        \pgf@circ@res@left = \pgf@x
        \pgf@circ@scaled@Rlen=\scaledRlen
        \pgf@circ@res@step = \ctikzvalof{multipoles/flipflop/pin spacing}\pgf@circ@scaled@Rlen
        \pgf@circ@res@other = \ctikzvalof{multipoles/external pins width}\pgf@circ@scaled@Rlen
        \pgfsetcolor{\ctikzvalof{color}}
        \pgfscope% (for the line width)
            \pgf@circ@setlinewidth{multipoles}{\pgflinewidth}
            \pgfpathrectanglecorners{\pgfpoint{-\width/2}{-\height/2}}{\pgfpoint{\width/2}{\height/2}}%
            \pgf@circ@draworfill
            \pgfusepath{stroke}%
        \endpgfscope
        % Adding the pin number
        \pgf@circ@count@a=\numpins\relax
        \divide\pgf@circ@count@a by 2 \pgf@circ@count@b=\pgf@circ@count@a
        % thanks to @marmot: https://tex.stackexchange.com/a/473571/38080
        \ifpgf@circuit@chip@straightnumbers
            \pgfgettransformentries\a\b\temp\temp\temp\temp
            \pgfmathsetmacro{\rot}{-atan2(\b,\a)}
            \pgfmathtruncatemacro{\quadrant}{mod(4+int(360+(\rot+45)/90),4)}
        \else
            \pgfmathsetmacro{\rot}{0}
            \pgfmathsetmacro{\quadrant}{0}
        \fi
        \def\pgf@circ@strut{\vrule width 0pt height 1em depth 0.4em\relax}
        % text
        \def\mytext{\ctikzvalof{multipoles/flipflop/font}\space
                \ctikzvalof{multipoles/flipflop/t\the\pgf@circ@count@c}%
                \pgf@circ@strut\space}
        % \typeout{TEXT\space\mytext}
        \pgfmathloop%
        \def\wedge{\ctikzvalof{multipoles/flipflop/clock wedge size}\pgf@circ@res@step}
        \pgf@circ@res@temp=0pt\relax
        \ifnum\pgf@circ@count@a>0
            \ifcase\quadrant % rotation 0
                % left
                \pgf@circ@count@c=\pgf@circ@count@a
                \pgf@circ@do@wedge@left
                % \typeout{TEXT Left Q1\space\mytext}
                \pgftext[left,
                    at=\pgfpoint{\pgf@circ@res@left+\pgf@circ@res@temp}{\pgf@circ@res@up+(\pgf@circ@dip@pin@shift-\the\pgf@circ@count@a)*\pgf@circ@res@step},
                    rotate=\rot]{\mytext}
                % right
                \pgf@circ@count@c=\numexpr2*\pgf@circ@count@b-\pgf@circ@count@a+1\relax
                \pgf@circ@do@wedge@right
                \pgftext[right,
                    at=\pgfpoint{\pgf@circ@res@right+\pgf@circ@res@temp}{\pgf@circ@res@up+(\pgf@circ@dip@pin@shift-\the\pgf@circ@count@a)*\pgf@circ@res@step},
                    rotate=\rot]{\mytext}
            \or % rotation -90
                % left
                \pgf@circ@count@c=\pgf@circ@count@a
                \pgf@circ@do@wedge@left
                \pgftext[top,
                    at=\pgfpoint{\pgf@circ@res@left+\pgf@circ@res@temp}{\pgf@circ@res@up+(\pgf@circ@dip@pin@shift-\the\pgf@circ@count@a)*\pgf@circ@res@step},
                    rotate=\rot]{\mytext}
                % right
                \pgf@circ@count@c=\numexpr2*\pgf@circ@count@b-\pgf@circ@count@a+1\relax
                \pgf@circ@do@wedge@right
                \pgftext[bottom,
                    at=\pgfpoint{\pgf@circ@res@right+\pgf@circ@res@temp}{\pgf@circ@res@up+(\pgf@circ@dip@pin@shift-\the\pgf@circ@count@a)*\pgf@circ@res@step},
                    rotate=\rot]{\mytext}
            \or %rotation 180
                % left
                \pgf@circ@count@c=\pgf@circ@count@a
                \pgf@circ@do@wedge@left
                \pgftext[right,
                    at=\pgfpoint{\pgf@circ@res@left+\pgf@circ@res@temp}{\pgf@circ@res@up+(\pgf@circ@dip@pin@shift-\the\pgf@circ@count@a)*\pgf@circ@res@step},
                    rotate=\rot]{\mytext}
                % right
                \pgf@circ@count@c=\numexpr2*\pgf@circ@count@b-\pgf@circ@count@a+1\relax
                \pgf@circ@do@wedge@right
                \pgftext[left,
                    at=\pgfpoint{\pgf@circ@res@right+\pgf@circ@res@temp}{\pgf@circ@res@up+(\pgf@circ@dip@pin@shift-\the\pgf@circ@count@a)*\pgf@circ@res@step},
                    rotate=\rot]{\mytext}
            \or % rotation +90
                % left
                \pgf@circ@count@c=\pgf@circ@count@a
                \pgf@circ@do@wedge@left
                \pgftext[bottom,
                    at=\pgfpoint{\pgf@circ@res@left+\pgf@circ@res@temp}{\pgf@circ@res@up+(\pgf@circ@dip@pin@shift-\the\pgf@circ@count@a)*\pgf@circ@res@step},
                    rotate=\rot]{\mytext}
                % right
                \pgf@circ@count@c=\numexpr2*\pgf@circ@count@b-\pgf@circ@count@a+1\relax
                \pgf@circ@do@wedge@right
                \pgftext[top,
                    at=\pgfpoint{\pgf@circ@res@right+\pgf@circ@res@temp}{\pgf@circ@res@up+(\pgf@circ@dip@pin@shift-\the\pgf@circ@count@a)*\pgf@circ@res@step},
                    rotate=\rot]{\mytext}
            \fi
            \advance\pgf@circ@count@a-1\relax%
            \repeatpgfmathloop
            % Now the up and down text
            % up
            \def\mytext{\ctikzvalof{multipoles/flipflop/fontud}\space\pgf@circ@strut\ctikzvalof{multipoles/flipflop/tu}\space}
            \pgf@circ@do@wedge@up
            \ifcase\quadrant % rotation 0
                \pgftext[top,
                    at=\pgfpoint{0pt}{\pgf@circ@res@up+\pgf@circ@res@temp},
                    rotate=\rot]{\mytext}
            \or % rotation -90
                \pgftext[right,
                    at=\pgfpoint{0pt}{\pgf@circ@res@up+\pgf@circ@res@temp},
                    rotate=\rot]{\mytext}
            \or %rotation 180
                \pgftext[bottom,
                    at=\pgfpoint{0pt}{\pgf@circ@res@up+\pgf@circ@res@temp},
                    rotate=\rot]{\mytext}
            \or % rotation +90
                \pgftext[left,
                    at=\pgfpoint{0pt}{\pgf@circ@res@up+\pgf@circ@res@temp},
                    rotate=\rot]{\mytext}
            \fi
            % down
            \def\mytext{\ctikzvalof{multipoles/flipflop/fontud}\space\pgf@circ@strut\ctikzvalof{multipoles/flipflop/td}\space}
            \pgf@circ@do@wedge@down
            \ifcase\quadrant % rotation 0
                \pgftext[bottom,
                    at=\pgfpoint{0pt}{\pgf@circ@res@down+\pgf@circ@res@temp},
                    rotate=\rot]{\mytext}
            \or % rotation -90
                \pgftext[left,
                    at=\pgfpoint{0pt}{\pgf@circ@res@down+\pgf@circ@res@temp},
                    rotate=\rot]{\mytext}
            \or %rotation 180
                \pgftext[top,
                    at=\pgfpoint{0pt}{\pgf@circ@res@down+\pgf@circ@res@temp},
                    rotate=\rot]{\mytext}
            \or % rotation +90
                \pgftext[right,
                    at=\pgfpoint{0pt}{\pgf@circ@res@down+\pgf@circ@res@temp},
                    rotate=\rot]{\mytext}
            \fi
            % external pins
            \ifdim\pgf@circ@res@other>0pt
            \pgfscope
                \pgfsetlinewidth{\ctikzvalof{multipoles/external pins thickness}\pgflinewidth}
                \pgf@circ@count@a=\numpins\relax
                \divide\pgf@circ@count@a by 2 \pgf@circ@count@b=\pgf@circ@count@a
                \pgfmathloop%
                \ifnum\pgf@circ@count@a>0
                    % left side pins
                    \pgf@circ@count@c=\pgf@circ@count@a
                    %% we draw the pin only if it's defined either a text, a clock wedge or a not pin
                    %% Or'ing tests in core TeX is tough
                    \edef\@@or{0}
                    % Just expand the key the minimum needed
                    \edef\@@tmp{x\unexpandedvalueof{/tikz/circuitikz/multipoles/flipflop/t\the\pgf@circ@count@c}}\edef\@@x{x}
                    \ifx\@@tmp\@@x\else\edef\@@or{1}\fi
                    \edef\@@tmp{\ctikzvalof{multipoles/flipflop/c\the\pgf@circ@count@c}}
                    \ifnum\@@tmp>0\edef\@@or{1}\fi
                    \edef\@@tmp{\ctikzvalof{multipoles/flipflop/n\the\pgf@circ@count@c}}
                    \ifnum\@@tmp>0\edef\@@or{1}\fi
                    \ifnum\@@or>0
                        \pgfpathmoveto{\pgfpoint{\pgf@circ@res@left}{\pgf@circ@res@up+(\pgf@circ@dip@pin@shift-\the\pgf@circ@count@a)*\pgf@circ@res@step}}
                        \pgfpathlineto{\pgfpoint{\pgf@circ@res@left-\pgf@circ@res@other}{\pgf@circ@res@up+(\pgf@circ@dip@pin@shift-\the\pgf@circ@count@a)*\pgf@circ@res@step}}
                        \pgfusepath{stroke}
                    \fi
                    % right side pins
                    \pgf@circ@count@c=\numexpr2*\pgf@circ@count@b-\pgf@circ@count@a+1\relax
                    \edef\@@or{0}
                    \edef\@@tmp{x\unexpandedvalueof{/tikz/circuitikz/multipoles/flipflop/t\the\pgf@circ@count@c}}\edef\@@x{x}
                    \ifx\@@tmp\@@x\else\edef\@@or{1}\fi
                    \edef\@@tmp{\ctikzvalof{multipoles/flipflop/c\the\pgf@circ@count@c}}
                    \ifnum\@@tmp>0\edef\@@or{1}\fi
                    \edef\@@tmp{\ctikzvalof{multipoles/flipflop/n\the\pgf@circ@count@c}}
                    \ifnum\@@tmp>0\edef\@@or{1}\fi
                    % \typeout{TEST\space\@@tmp\space\@@x}
                    \ifnum\@@or>0
                        \pgfpathmoveto{\pgfpoint{\pgf@circ@res@right}{\pgf@circ@res@up+(\pgf@circ@dip@pin@shift-\the\pgf@circ@count@a)*\pgf@circ@res@step}}
                        \pgfpathlineto{\pgfpoint{\pgf@circ@res@right+\pgf@circ@res@other}{\pgf@circ@res@up+(\pgf@circ@dip@pin@shift-\the\pgf@circ@count@a)*\pgf@circ@res@step}}
                        \pgfusepath{stroke}
                    \fi
                    \advance\pgf@circ@count@a by -1\relax%
                \repeatpgfmathloop
                % up side
                \edef\@@or{0}
                \edef\@@tmp{x\unexpandedvalueof{/tikz/circuitikz/multipoles/flipflop/tu}}\edef\@@x{x}
                \ifx\@@tmp\@@x\else\edef\@@or{1}\fi
                \edef\@@tmp{\ctikzvalof{multipoles/flipflop/cu}}
                \ifnum\@@tmp>0\edef\@@or{1}\fi
                \edef\@@tmp{\ctikzvalof{multipoles/flipflop/nu}}
                \ifnum\@@tmp>0\edef\@@or{1}\fi
                % \typeout{TEST\space\@@tmp\space\@@x}
                \ifnum\@@or>0
                    \pgfpathmoveto{\pgfpoint{0pt}{\pgf@circ@res@up}}
                    \pgfpathlineto{\pgfpoint{0pt}{\pgf@circ@res@up+\pgf@circ@res@other}}
                    \pgfusepath{stroke}
                \fi
                % down side
                \edef\@@or{0}
                \edef\@@tmp{x\unexpandedvalueof{/tikz/circuitikz/multipoles/flipflop/td}}\edef\@@x{x}
                \ifx\@@tmp\@@x\else\edef\@@or{1}\fi
                \edef\@@tmp{\ctikzvalof{multipoles/flipflop/cd}}
                \ifnum\@@tmp>0\edef\@@or{1}\fi
                \edef\@@tmp{\ctikzvalof{multipoles/flipflop/nu}}
                \ifnum\@@tmp>0\edef\@@or{1}\fi
                % \typeout{TEST\space\@@tmp\space\@@x}
                \ifnum\@@or>0
                    \pgfpathmoveto{\pgfpoint{0pt}{\pgf@circ@res@down}}
                    \pgfpathlineto{\pgfpoint{0pt}{\pgf@circ@res@down-\pgf@circ@res@other}}
                    \pgfusepath{stroke}
                \fi
            \endpgfscope
            \fi % external pin width >0
            %
            % draw "inverting" circles on outputs, they must be last
            %
            \pgfscope
                \pgfsetlinewidth{\ctikzvalof{multipoles/external pins thickness}\pgflinewidth}
                \pgf@circ@count@a=\numpins\relax
                \divide\pgf@circ@count@a by 2 \pgf@circ@count@b=\pgf@circ@count@a
                \pgfmathloop%
                \ifnum\pgf@circ@count@a>0
                    % left side pins
                    \pgf@circ@count@c=\pgf@circ@count@a
                    \edef\@@tmp{\ctikzvalof{multipoles/flipflop/n\the\pgf@circ@count@c}}
                    \ifnum\@@tmp>0\pgfscope
                        \pgftransformshift{\pgfpoint{\pgf@circ@res@left}{\pgf@circ@res@up+(\pgf@circ@dip@pin@shift-\the\pgf@circ@count@a)*\pgf@circ@res@step}}
                        \pgfnode{ocirc}{east}{}{\thisshape-N\the\pgf@circ@count@c}{\pgfusepath{stroke}}
                    \endpgfscope\fi
                    % right side pins
                    \pgf@circ@count@c=\numexpr2*\pgf@circ@count@b-\pgf@circ@count@a+1\relax
                    \edef\@@tmp{\ctikzvalof{multipoles/flipflop/n\the\pgf@circ@count@c}}
                    \ifnum\@@tmp>0\pgfscope
                        \pgftransformshift{\pgfpoint{\pgf@circ@res@right}{\pgf@circ@res@up+(\pgf@circ@dip@pin@shift-\the\pgf@circ@count@a)*\pgf@circ@res@step}}
                        \pgfnode{ocirc}{west}{}{\thisshape-N\the\pgf@circ@count@c}{\pgfusepath{stroke}}
                    \endpgfscope\fi
                    \advance\pgf@circ@count@a by -1\relax%
                \repeatpgfmathloop
                % up side
                \edef\@@tmp{\ctikzvalof{multipoles/flipflop/nu}}
                \ifnum\@@tmp>0\pgfscope
                    \pgftransformshift{\pgfpoint{0pt}{\pgf@circ@res@up}}
                    \pgfnode{ocirc}{south}{}{\thisshape-Nu}{\pgfusepath{stroke}}
                \endpgfscope\fi
                % down side
                \edef\@@tmp{\ctikzvalof{multipoles/flipflop/nd}}
                \ifnum\@@tmp>0\pgfscope
                    \pgftransformshift{\pgfpoint{0pt}{\pgf@circ@res@down}}
                    \pgfnode{ocirc}{north}{}{\thisshape-Nd}{\pgfusepath{stroke}}
                \endpgfscope\fi
            \endpgfscope
        }%
        % \pgf@sh@s@<name of the shape here> contains all the code for the shape
        % and is executed just before a node is drawn.
        \pgfutil@g@addto@macro\pgf@sh@s@flipflop{%
            % Start with the maximum pin number and go backwards.
            \pgf@circ@count@a=\numpins\relax
            \pgfmathloop%
            \ifnum\pgf@circ@count@a>0
                % we will create two anchors per pin: the "normal one" like `pin 1` for the
                % electrical contact, and the "border one" like `bpin 1` for labels.
                % they will coincide if `external pins width` is set to 0.
                \expandafter\xdef\csname pgf@anchor@flipflop@pin\space\the\pgf@circ@count@a\endcsname{%
                    \noexpand\pgf@circ@dippinanchor{\the\pgf@circ@count@a}{1}%
                }
                \expandafter\xdef\csname pgf@anchor@flipflop@bpin\space\the\pgf@circ@count@a\endcsname{%
                    \noexpand\pgf@circ@dippinanchor{\the\pgf@circ@count@a}{0}%
                }
                \advance\pgf@circ@count@a by -1\relax%
                \repeatpgfmathloop%
            }%
}

%
% MUX-DEMUXES
%
% Thanks to @marmot
\tikzset{muxdemux def/.code=\pgfqkeys{\circuitikzbasekey/multipoles/muxdemux}{#1}}
\tikzset{demux/.style={muxdemux, muxdemux def={Lh=4, Rh=8, NL=1, NB=3, NR=8}}}
\tikzset{one bit adder/.style={muxdemux,
         muxdemux def={Lh=4, NL=2, Rh=2, NR=1, NB=1, w=1.5,
         inset w=0.5, inset Lh=2, inset Rh=1.5}}}
\tikzset{ALU/.style={muxdemux,
         muxdemux def={Lh=5, NL=2, Rh=2, NR=1, NB=2, NT=1, w=2,
         inset w=1, inset Lh=2, inset Rh=0, square pins=1}}}
%generic mux-demux shape
\pgfdeclareshape{muxdemux}{
    \savedmacro{\ctikzclass}{\edef\ctikzclass{muxdemuxes}}
    \saveddimen{\scaledRlen}{\pgfmathsetlength{\pgf@x}{\ctikzvalof{\ctikzclass/scale}\pgf@circ@Rlen}}
    \savedmacro{\thisshape}{\def\thisshape{\tikz@fig@name}}
    % pins on the four sides
    % \savedmacro\NL{%
    %         \pgf@circ@count@a=\ctikzvalof{multipoles/muxdemux/NL}%
    %         \def\NL{\the\pgf@circ@count@a}
    % }
    \savedmacro\NL{\edef\NL{\ctikzvalof{multipoles/muxdemux/NL}}}
    \savedmacro\NR{\edef\NR{\ctikzvalof{multipoles/muxdemux/NR}}}
    \savedmacro\NT{\edef\NT{\ctikzvalof{multipoles/muxdemux/NT}}}
    \savedmacro\NB{\edef\NB{\ctikzvalof{multipoles/muxdemux/NB}}}
    \savedmacro\squarepins{\edef\squarepins{\ctikzvalof{multipoles/muxdemux/square pins}}}
    % topleft and topright sizes
    \savedanchor{\topleft}{%
        \pgfmathsetlength{\pgf@circ@scaled@Rlen}{\ctikzvalof{\ctikzclass/scale}\pgf@circ@Rlen}
        \pgfmathsetlength\pgf@y{\ctikzvalof{multipoles/muxdemux/base len}*\ctikzvalof{multipoles/muxdemux/Lh}*\pgf@circ@scaled@Rlen/2}
        \pgfmathsetlength\pgf@x{-\ctikzvalof{multipoles/muxdemux/base len}*\ctikzvalof{multipoles/muxdemux/w}*\pgf@circ@scaled@Rlen/2}
    }
    \savedanchor{\topright}{%
        \pgfmathsetlength{\pgf@circ@scaled@Rlen}{\ctikzvalof{\ctikzclass/scale}\pgf@circ@Rlen}
        \pgfmathsetlength\pgf@y{\ctikzvalof{multipoles/muxdemux/base len}*\ctikzvalof{multipoles/muxdemux/Rh}*\pgf@circ@scaled@Rlen/2}
        \pgfmathsetlength\pgf@x{\ctikzvalof{multipoles/muxdemux/base len}*\ctikzvalof{multipoles/muxdemux/w}*\pgf@circ@scaled@Rlen/2}
    }
    \savedanchor{\insetnortheast}{%
        \pgfmathsetlength{\pgf@circ@scaled@Rlen}{\ctikzvalof{\ctikzclass/scale}\pgf@circ@Rlen}
        \pgfmathsetlength\pgf@y{\ctikzvalof{multipoles/muxdemux/base len}*\ctikzvalof{multipoles/muxdemux/inset Lh}*\pgf@circ@scaled@Rlen/2}
        \pgfmathsetlength\pgf@x{-\ctikzvalof{multipoles/muxdemux/base len}*
        (\ctikzvalof{multipoles/muxdemux/w}-2*\ctikzvalof{multipoles/muxdemux/inset w})*\pgf@circ@scaled@Rlen/2}
    }
    \saveddimen{\insethright}{
        \pgfmathsetlength{\pgf@circ@scaled@Rlen}{\ctikzvalof{\ctikzclass/scale}\pgf@circ@Rlen}
        \pgfmathsetlength\pgf@x{\ctikzvalof{multipoles/muxdemux/base len}*\ctikzvalof{multipoles/muxdemux/inset Rh}*\pgf@circ@scaled@Rlen/2}}
    \saveddimen{\extshift}{
        \pgfmathsetlength{\pgf@circ@scaled@Rlen}{\ctikzvalof{\ctikzclass/scale}\pgf@circ@Rlen}
        \pgfmathsetlength\pgf@x{\pgf@circ@scaled@Rlen*\ctikzvalof{multipoles/external pins width}}}
    \savedanchor{\northwest}{%
        \pgfmathsetlength{\pgf@circ@scaled@Rlen}{\ctikzvalof{\ctikzclass/scale}\pgf@circ@Rlen}
        \pgfmathsetlength\pgf@y{\ctikzvalof{multipoles/muxdemux/base len}*max(\ctikzvalof{multipoles/muxdemux/Rh},\ctikzvalof{multipoles/muxdemux/Lh})*\pgf@circ@scaled@Rlen/2}
        \pgfmathsetlength\pgf@x{-\ctikzvalof{multipoles/muxdemux/base len}*\ctikzvalof{multipoles/muxdemux/w}*\pgf@circ@scaled@Rlen/2}
    }
    \anchor{nw}{\northwest}
    \anchor{ne}{\northwest\pgf@x=-\pgf@x}
    \anchor{se}{\northwest\pgf@x=-\pgf@x\pgf@y=-\pgf@y}
    \anchor{sw}{\northwest\pgf@y=-\pgf@y}
    \anchor{north west}{\northwest}
    \anchor{north east}{\northwest\pgf@x=-\pgf@x}
    \anchor{south east}{\northwest\pgf@x=-\pgf@x \pgf@y=-\pgf@y}
    \anchor{south west}{\northwest\pgf@y=-\pgf@y}
    \anchor{n}{\northwest\pgf@x=0pt }
    \anchor{e}{\northwest\pgf@x=-\pgf@x\pgf@y=0pt }
    \anchor{s}{\northwest\pgf@x=0pt\pgf@y=-\pgf@y}
    \anchor{w}{\northwest\pgf@y=0pt }
    \anchor{north}{\northwest\pgf@x=0pt }
    \anchor{east}{\northwest\pgf@x=-\pgf@x\pgf@y=0pt }
    \anchor{south}{\northwest\pgf@x=0pt\pgf@y=-\pgf@y}
    \anchor{west}{\northwest\pgf@y=0pt }
    \anchor{center}{\pgfpointorigin}
    \anchor{top right}{\topright}
    \anchor{bottom right}{\topright\pgf@y=-\pgf@y}
    \anchor{right}{\topright\pgf@y=0pt\relax}
    \anchor{top left}{\topleft}
    \anchor{bottom left}{\topleft\pgf@y=-\pgf@y}
    \anchor{left}{\topleft\pgf@y=0pt\relax}
    \anchor{top}{\topright\pgf@ya=\pgf@y \topleft \advance\pgf@y by \pgf@ya
        \divide\pgf@y by 2 \pgf@x=0pt\relax}
    \anchor{bottom}{\topright\pgf@ya=\pgf@y \topleft \advance\pgf@y by \pgf@ya
        \divide\pgf@y by 2 \pgf@y=-\pgf@y \pgf@x=0pt\relax}
    \anchor{inset top right}{\pgf@ya=\insethright\insetnortheast\advance\pgf@y by -0.5\pgf@ya}
    \anchor{inset bottom right}{\pgf@ya=\insethright\insetnortheast\advance\pgf@y by -0.5\pgf@ya\pgf@y=-\pgf@y}
    \anchor{inset right}{\insetnortheast\pgf@y=0pt\relax}
    \anchor{inset top left}{\insetnortheast\pgf@ya=\pgf@y\topleft\pgf@y=\pgf@ya}
    \anchor{inset bottom left}{\insetnortheast\pgf@ya=\pgf@y\topleft\pgf@y=-\pgf@ya}
    \anchor{inset left}{\topleft\pgf@y=0pt\relax}
    \anchor{inset bottom}{\topleft\pgf@xa=\pgf@x\pgf@ya=\insethright
        \insetnortheast\pgf@xb=\pgf@x\pgf@yb=\pgf@x
        \pgfpoint{(\pgf@xa+\pgf@xb)/2}{-\pgf@ya+\pgf@yb/2}}
    \anchor{inset top}{\topleft\pgf@xa=\pgf@x\pgf@ya=\insethright
        \insetnortheast\pgf@xb=\pgf@x\pgf@yb=\pgf@x
        \pgfpoint{(\pgf@xa+\pgf@xb)/2}{\pgf@ya-\pgf@yb/2}}
    \anchor{inset center}{\topleft\pgf@xa=\pgf@x\insetnortheast
        \advance\pgf@x by \pgf@xa \divide\pgf@x by 2 \pgf@y=0pt\relax}
    \anchor{narrow center}{\insetnortheast\pgf@xa=\pgf@x\topright
        \advance\pgf@x by \pgf@xa \divide\pgf@x by 2\pgf@y=0pt\relax}
    \anchor{center up}{\topright\pgf@ya=\pgf@y \topleft \advance\pgf@y by \pgf@ya
        \divide\pgf@y by 2
        \pgf@yb = \insethright \advance\pgf@y by \pgf@yb
        \divide\pgf@y by 2 \pgf@x=0pt\relax}
    \anchor{center down}{\topright\pgf@ya=\pgf@y \topleft \advance\pgf@y by \pgf@ya
        \divide\pgf@y by 2
        \pgf@yb = \insethright \advance\pgf@y by \pgf@yb
        \divide\pgf@y by 2 \pgf@y=-\pgf@y \pgf@x=0pt\relax}
    \anchor{text}{%
        \pgfmathsetlength{\pgf@circ@scaled@Rlen}{\ctikzvalof{\ctikzclass/scale}\pgf@circ@Rlen}
        \pgfmathsetlength\pgf@x{\ctikzvalof{multipoles/muxdemux/base len}*
          \ctikzvalof{multipoles/muxdemux/inset w}*\pgf@circ@scaled@Rlen/2}
        \advance\pgf@x by -.5\wd\pgfnodeparttextbox%
        \pgf@y=-.5\ht\pgfnodeparttextbox%
        \advance\pgf@y by+.5\dp\pgfnodeparttextbox%
    }%
    \backgroundpath{%
        \topleft
        \pgf@circ@res@up = \pgf@y
        \pgf@circ@res@down = -\pgf@y
        \pgf@circ@res@left = \pgf@x
        \topright
        \pgf@circ@res@other = \pgf@y
        \pgf@circ@res@right = \pgf@x
        \insetnortheast
        \pgf@circ@res@step = \pgf@x
        \pgf@circ@res@temp = \pgf@y
        %
        % external block
        %
        \pgfscope% (for the line width)
            \pgf@circ@setlinewidth{multipoles}{\pgflinewidth}
            \pgfpathmoveto{\pgfpoint{\pgf@circ@res@left}{\pgf@circ@res@up}}
            \pgfpathlineto{\pgfpoint{\pgf@circ@res@right}{\pgf@circ@res@other}}
            \pgfpathlineto{\pgfpoint{\pgf@circ@res@right}{-\pgf@circ@res@other}}
            \pgfpathlineto{\pgfpoint{\pgf@circ@res@left}{\pgf@circ@res@down}}
            % inset, starting down
            \ifdim\pgf@circ@res@temp>0pt % inset
                % \typeout{INSETw\space\the\pgf@circ@res@right\space x\space\the\pgf@circ@res@step\space  y\space\the\pgf@circ@res@temp}
                \pgfpathlineto{\pgfpoint{\pgf@circ@res@left}{-\pgf@circ@res@temp}}
                \pgfpathlineto{\pgfpoint{\pgf@circ@res@step}{-\insethright}}
                \pgfpathlineto{\pgfpoint{\pgf@circ@res@step}{\insethright}}
                \pgfpathlineto{\pgfpoint{\pgf@circ@res@left}{\pgf@circ@res@temp}}
            \fi
            \pgfpathclose
            \pgf@circ@draworfill
        \endpgfscope
        % now we have to draw the pins, if needed
        \ifdim\extshift>0pt\ifpgfcirc@draw@leads\pgfscope % let's avoid too much indent
        % Ok, we have to draw the leads (a.k.a. pins)
            \pgfsetlinewidth{\ctikzvalof{multipoles/external pins thickness}\pgflinewidth}
            % We mimic the anchors here --- probably there is a better way
            % left pins
            \ifnum\NL>0\relax % not indented, closed on \repeatpgfmathloop
            \pgf@circ@count@a=\NL\relax
            \pgf@circ@count@b=\NL \divide\pgf@circ@count@b by 2 % see https://tex.stackexchange.com/questions/146523/why-does-numexpr-integer-division-round-rather-than-truncate
            \topleft\pgf@circ@res@left=\pgf@x \pgf@circ@res@up=\pgf@y
            \insetnortheast\pgf@circ@res@right=\pgf@x \pgf@circ@res@down=\pgf@y
            \ifdim\pgf@circ@res@down>0pt % check if we have an inset
            % we have to check oddity
                \ifodd\NL
                    \ifnum\NL=1
                        % only centerpin, step should not be used, but anyway...
                        \pgfmathsetlength{\pgf@circ@res@step}{2*(\pgf@circ@res@up-\pgf@circ@res@down)/(\NL)}
                    \else
                        \pgfmathsetlength{\pgf@circ@res@step}{2*(\pgf@circ@res@up-\pgf@circ@res@down)/(\NL-1)}
                    \fi
                \else
                    \pgfmathsetlength{\pgf@circ@res@step}{2*(\pgf@circ@res@up-\pgf@circ@res@down)/\NL}
                \fi
            \else % no inset
                \pgfmathsetlength{\pgf@circ@res@step}{2*\pgf@circ@res@up/\NL}
            \fi
            \pgfmathloop%
            \ifnum\pgf@circ@count@a>0
                %%%%%
                \ifdim\pgf@circ@res@down>0pt % check if we have an inset
                    \ifnum\pgf@circ@count@a>\pgf@circ@count@b\relax
                        % for lower pins we have to shift them down
                        % \typeout{DEBUGTEST1\space #1\space entering\space \NL}
                        \ifodd\NL
                            % odd number of pins
                            \ifnum\pgf@circ@count@a=\numexpr\the\pgf@circ@count@b+1\relax
                                % centerpin!
                                \pgfpathmoveto{\pgfpoint{\pgf@circ@res@right}{0pt}}
                                \ifnum\squarepins>0
                                    \pgfpathlineto{\pgfpoint{\pgf@circ@res@left-\extshift}{0pt}}
                                \else
                                    \pgfpathlineto{\pgfpoint{\pgf@circ@res@right-\extshift}{0pt}}
                                \fi
                            \else
                                \pgfpathmoveto{\pgfpoint{\pgf@circ@res@left}{\pgf@circ@res@up+(0.5-\pgf@circ@count@a+1)*\pgf@circ@res@step-2*\pgf@circ@res@down}}
                                \pgfpathlineto{\pgfpoint{\pgf@circ@res@left-\extshift}{\pgf@circ@res@up+(0.5-\pgf@circ@count@a+1)*\pgf@circ@res@step-2*\pgf@circ@res@down}}
                            \fi
                        \else
                            % even numer of pins: just go down
                            \pgfpathmoveto{\pgfpoint{\pgf@circ@res@left}{\pgf@circ@res@up+(0.5-\pgf@circ@count@a)*\pgf@circ@res@step-2*\pgf@circ@res@down}}
                            \pgfpathlineto{\pgfpoint{\pgf@circ@res@left-\extshift}{\pgf@circ@res@up+(0.5-\pgf@circ@count@a)*\pgf@circ@res@step-2*\pgf@circ@res@down}}
                        \fi
                    \else
                        % nothing need for #1<=NL/2
                        \pgfpathmoveto{\pgfpoint{\pgf@circ@res@left}{\pgf@circ@res@up+(0.5-\pgf@circ@count@a)*\pgf@circ@res@step}}
                        \pgfpathlineto{\pgfpoint{\pgf@circ@res@left-\extshift}{\pgf@circ@res@up+(0.5-\pgf@circ@count@a)*\pgf@circ@res@step}}
                    \fi
                \else
                % no inset
                    \pgfpathmoveto{\pgfpoint{\pgf@circ@res@left}{\pgf@circ@res@up+(0.5-\pgf@circ@count@a)*\pgf@circ@res@step}}
                    \pgfpathlineto{\pgfpoint{\pgf@circ@res@left-\extshift}{\pgf@circ@res@up+(0.5-\pgf@circ@count@a)*\pgf@circ@res@step}}
                \fi
                %%%%%
                \advance\pgf@circ@count@a by -1\relax%
            \repeatpgfmathloop\fi%
            % right pins
            \ifnum\NR>0\pgf@circ@count@a=\NR\relax
            \pgfmathloop%
            \topright\pgf@circ@res@right=\pgf@x \pgf@circ@res@up=\pgf@y
            \pgfmathsetlength{\pgf@circ@res@step}{2*\pgf@circ@res@up/\NR}
            \ifnum\pgf@circ@count@a>0
                \pgfpathmoveto{\pgfpoint{\pgf@circ@res@right}{\pgf@circ@res@up+(0.5-\pgf@circ@count@a)*\pgf@circ@res@step}}
                \pgfpathlineto{\pgfpoint{\pgf@circ@res@right+\extshift}{\pgf@circ@res@up+(0.5-\pgf@circ@count@a)*\pgf@circ@res@step}}
            \advance\pgf@circ@count@a by -1\relax%
            \repeatpgfmathloop\fi%
            % bottom pins
            \ifnum\NB>0\pgf@circ@count@a=\NB\relax %%%
            \pgfmathloop%
            \topleft\pgf@circ@res@left=\pgf@x \pgf@circ@res@up=\pgf@y
            \topright\pgf@circ@res@right=\pgf@x \pgf@circ@res@down=\pgf@y
            \pgfmathsetlength{\pgf@circ@res@step}{2*\pgf@circ@res@right/\NB}
            \pgfmathsetlength{\pgf@circ@res@other}{(\pgf@circ@res@down-\pgf@circ@res@up)/(\pgf@circ@res@right-\pgf@circ@res@left)*\pgf@circ@res@step}
            \ifnum\pgf@circ@count@a>0
                \pgfpathmoveto{\pgfpoint{\pgf@circ@res@left+(\pgf@circ@count@a-0.5)*\pgf@circ@res@step}
                {-\pgf@circ@res@down+(\NB-\pgf@circ@count@a+0.5)*\pgf@circ@res@other}}
                \ifnum\squarepins>0
                    \pgfpathlineto{\pgfpoint{\pgf@circ@res@left+(\pgf@circ@count@a-0.5)*\pgf@circ@res@step}
                    {-max(\pgf@circ@res@down, \pgf@circ@res@up)-\extshift}}
                \else
                    \pgfpathlineto{\pgfpoint{\pgf@circ@res@left+(\pgf@circ@count@a-0.5)*\pgf@circ@res@step}
                    {-\pgf@circ@res@down+(\NB-\pgf@circ@count@a+0.5)*\pgf@circ@res@other-\extshift}}
                \fi
            \advance\pgf@circ@count@a by -1\relax%
            \repeatpgfmathloop\fi%
            % top pins
            \ifnum\NT>0\pgf@circ@count@a=\NT\relax
            \pgfmathloop%
            \topleft\pgf@circ@res@left=\pgf@x \pgf@circ@res@up=\pgf@y
            \topright\pgf@circ@res@right=\pgf@x \pgf@circ@res@down=\pgf@y
            \pgfmathsetlength{\pgf@circ@res@step}{2*\pgf@circ@res@right/\NT}
            \pgfmathsetlength{\pgf@circ@res@other}{(\pgf@circ@res@down-\pgf@circ@res@up)/(\pgf@circ@res@right-\pgf@circ@res@left)*\pgf@circ@res@step}
            \ifnum\pgf@circ@count@a>0
                \pgfpathmoveto{\pgfpoint{\pgf@circ@res@left+(\pgf@circ@count@a-0.5)*\pgf@circ@res@step}
                {\pgf@circ@res@down-(\NT-\pgf@circ@count@a+0.5)*\pgf@circ@res@other}}
                \ifnum\squarepins>0
                    \pgfpathlineto{\pgfpoint{\pgf@circ@res@left+(\pgf@circ@count@a-0.5)*\pgf@circ@res@step}
                    {max(\pgf@circ@res@down, \pgf@circ@res@up)+\extshift}}
                \else
                    \pgfpathlineto{\pgfpoint{\pgf@circ@res@left+(\pgf@circ@count@a-0.5)*\pgf@circ@res@step}
                    {\pgf@circ@res@down-(\NT-\pgf@circ@count@a+0.5)*\pgf@circ@res@other+\extshift}}
                \fi
            \advance\pgf@circ@count@a by -1\relax%
            \repeatpgfmathloop\fi%
        % end drawing pins; stroke them
        \pgfusepath{stroke}
        \endpgfscope\fi\fi
    }
    % let's start adding anchors
    \pgfutil@g@addto@macro\pgf@sh@s@muxdemux{%
        % left side anchors
        \pgf@circ@count@a=\NL\relax
        % \typeout{STARTGENERATINGLEFT\space\the\pgf@circ@count@a\space FOR\space\thisshape\space\NL}
        \pgfmathloop%
        \ifnum\pgf@circ@count@a>0
        % \typeout{GENERATINGLEFT\space\the\pgf@circ@count@a\space FOR\space\thisshape\space\NL}
            % we will create two anchors per pin: the "normal one" like `lpin 1` for the
            % external leads, and the "border one" like `blpin 1` for internal ones.
            % they will coincide if `external pins width` is set to 0.
            \expandafter\xdef\csname pgf@anchor@muxdemux@lpin\space\the\pgf@circ@count@a\endcsname{%
                \noexpand\pgf@circ@muxdemux@L@anchor{\the\pgf@circ@count@a}{1}%
            }
            \expandafter\xdef\csname pgf@anchor@muxdemux@blpin\space\the\pgf@circ@count@a\endcsname{%
                \noexpand\pgf@circ@muxdemux@L@anchor{\the\pgf@circ@count@a}{0}%
            }
            \advance\pgf@circ@count@a by -1\relax%
        \repeatpgfmathloop%
        % right anchors
        \pgf@circ@count@a=\NR\relax
        \pgfmathloop%
        \ifnum\pgf@circ@count@a>0
            % we will create two anchors per pin: the "normal one" like `rpin 1` for the
            % external leads, and the "border one" like `brpin 1` for internal ones.
            % they will coincide if `external pins width` is set to 0.
            \expandafter\xdef\csname pgf@anchor@muxdemux@rpin\space\the\pgf@circ@count@a\endcsname{%
                \noexpand\pgf@circ@muxdemux@R@anchor{\the\pgf@circ@count@a}{1}%
            }
            \expandafter\xdef\csname pgf@anchor@muxdemux@brpin\space\the\pgf@circ@count@a\endcsname{%
                \noexpand\pgf@circ@muxdemux@R@anchor{\the\pgf@circ@count@a}{0}%
            }
            \advance\pgf@circ@count@a by -1\relax%
        \repeatpgfmathloop%
        % bottom anchors
        \pgf@circ@count@a=\NB\relax
        \pgfmathloop%
        \ifnum\pgf@circ@count@a>0
            % we will create two anchors per pin: the "normal one" like `bpin 1` for the
            % external leads, and the "border one" like `bbpin 1` for internal ones.
            % they will coincide if `external pins width` is set to 0.
            \expandafter\xdef\csname pgf@anchor@muxdemux@bpin\space\the\pgf@circ@count@a\endcsname{%
                \noexpand\pgf@circ@muxdemux@B@anchor{\the\pgf@circ@count@a}{1}%
            }
            \expandafter\xdef\csname pgf@anchor@muxdemux@bbpin\space\the\pgf@circ@count@a\endcsname{%
                \noexpand\pgf@circ@muxdemux@B@anchor{\the\pgf@circ@count@a}{0}%
            }
            \advance\pgf@circ@count@a by -1\relax%
        \repeatpgfmathloop%
        % top anchors
        \pgf@circ@count@a=\NT\relax
        \pgfmathloop%
        \ifnum\pgf@circ@count@a>0
            % we will create two anchors per pin: the "normal one" like `tpin 1` for the
            % external leads, and the "border one" like `btpin 1` for internal ones.
            % they will coincide if `external pins width` is set to 0.
            \expandafter\xdef\csname pgf@anchor@muxdemux@tpin\space\the\pgf@circ@count@a\endcsname{%
                \noexpand\pgf@circ@muxdemux@T@anchor{\the\pgf@circ@count@a}{1}%
            }
            \expandafter\xdef\csname pgf@anchor@muxdemux@btpin\space\the\pgf@circ@count@a\endcsname{%
                \noexpand\pgf@circ@muxdemux@T@anchor{\the\pgf@circ@count@a}{0}%
            }
            \advance\pgf@circ@count@a by -1\relax%
        \repeatpgfmathloop%
    }%
}

%% left anchors for muxdemux

\def\pgf@circ@muxdemux@L@anchor#1#2{% #1: pin number #2: 0 for border pin, 1 for external pin
    \topleft
    \pgf@xa=\pgf@x
    \pgf@ya=\pgf@y
    \insetnortheast
    \pgf@xb=\pgf@x
    \pgf@yb=\pgf@y
    \ifnum#1>\NL
        \PackageError{circuitikz}{%
            You requested left pin #1 for mux/demux shape \thisshape\space \MessageBreak
            which has been defined with \NL\space left pins%
        }{Please check the manual about mux/demux shapes; if you press return I'll try to continue}
    \fi
    \pgf@circ@count@a=\NL \divide\pgf@circ@count@a by 2 % see https://tex.stackexchange.com/questions/146523/why-does-numexpr-integer-division-round-rather-than-truncate
    % \typeout{LEFT \the\pgf@xa \space \the\pgf@ya \space \NL}
    \ifnum\NL>1
        \ifdim\pgf@yb>0pt % check if we have an inset
        % we have to check oddity
            \ifodd\NL
                \pgfmathsetlength{\pgf@circ@res@step}{2*(\pgf@ya-\pgf@yb)/(\NL-1)}
            \else
                \pgfmathsetlength{\pgf@circ@res@step}{2*(\pgf@ya-\pgf@yb)/\NL}
            \fi
        \else % no inset
            \pgfmathsetlength{\pgf@circ@res@step}{2*\pgf@ya/\NL}
        \fi
        \ifdim\pgf@yb>0pt % check if we have an inset
            \ifnum#1>\pgf@circ@count@a\relax
                % for lower pins we have to shift them down
                % \typeout{DEBUGTEST1\space #1\space entering\space \NL}
                \ifodd\NL
                    % odd number of pins
                    \ifnum#1=\numexpr\the\pgf@circ@count@a+1\relax
                        % centerpin!
                        \ifnum#2=0\relax
                            \pgfpoint{\pgf@xb}{0pt}
                        \else
                            \ifnum\squarepins>0
                                \pgfpoint{\pgf@xa-#2*\extshift}{0pt}
                            \else
                                \pgfpoint{\pgf@xb-#2*\extshift}{0pt}
                            \fi
                        \fi
                    \else
                        \pgfpoint{\pgf@xa-#2*\extshift}{\pgf@ya+(0.5-#1+1)*\pgf@circ@res@step-2*\pgf@yb}
                    \fi
                \else
                    % even numer of pins: just go down
                    \pgfpoint{\pgf@xa-#2*\extshift}{\pgf@ya+(0.5-#1)*\pgf@circ@res@step-2*\pgf@yb}
                \fi
            \else
                % nothing need for #1<=NL/2
                \pgfpoint{\pgf@xa-#2*\extshift}{\pgf@ya+(0.5-#1)*\pgf@circ@res@step}
            \fi
        \else
        % no inset
            \pgfpoint{\pgf@xa-#2*\extshift}{\pgf@ya+(0.5-#1)*\pgf@circ@res@step}
        \fi
    \else
        \pgfpoint{\pgf@xa-#2*\extshift}{0pt}
    \fi
}

% right anchors
\def\pgf@circ@muxdemux@R@anchor#1#2{% #1: pin number #2: 0 for border pin, 1 for external pin
    \topright
    \pgf@xa=\pgf@x
    \pgf@ya=\pgf@y
    \ifnum#1>\NR
        \PackageError{circuitikz}{%
            You requested right pin #1 for mux/demux shape \thisshape\space \MessageBreak
            which has been defined with \NR\space right pins%
        }{Please check the manual about mux/demux shapes; if you press return I'll try to continue}
    \fi
    \ifnum\NR>1
        \pgfmathsetlength{\pgf@circ@res@step}{2*\pgf@ya/\NR}
        \pgfpoint{\pgf@xa+#2*\extshift}{\pgf@ya+(0.5-#1)*\pgf@circ@res@step}
    \else
        \pgfpoint{\pgf@xa+#2*\extshift}{0pt}
    \fi
}

% bottom anchors
\def\pgf@circ@muxdemux@B@anchor#1#2{% #1: pin number #2: 0 for border pin, 1 for external pin
    \topleft
    \pgf@xa=\pgf@x
    \pgf@ya=\pgf@y
    \topright
    \pgf@xb=\pgf@x
    \pgf@yb=\pgf@y
    \ifnum#1>\NB
        \PackageError{circuitikz}{%
            You requested bottom pin #1 for mux/demux shape \thisshape\space \MessageBreak
            which has been defined with \NB\space bottom pins%
        }{Please check the manual about mux/demux shapes; if you press return I'll try to continue}
    \fi
    \ifnum\NB>0
        % \typeout{DEBUGTESTtopleft\space\the\pgf@ya \space topright\space\the\pgf@yb \space\NB}
        \pgfmathsetlength{\pgf@circ@res@step}{2*\pgf@xb/\NB}
        \pgfmathsetlength{\pgf@circ@res@other}{(\pgf@yb-\pgf@ya)/(\pgf@xb-\pgf@xa)*\pgf@circ@res@step}
        \pgfmathsetlength\pgf@x{\pgf@xa+(#1-0.5)*\pgf@circ@res@step}
        \ifnum#2=0\relax
            \pgfmathsetlength\pgf@y{-\pgf@yb+(\NB-#1+0.5)*\pgf@circ@res@other}
        \else
            \ifnum\squarepins>0\relax
                \pgfmathsetlength\pgf@y{-max(\pgf@ya,\pgf@yb)-\extshift}
            \else
                \pgfmathsetlength\pgf@y{-\pgf@yb+(\NB-#1+0.5)*\pgf@circ@res@other-\extshift}
            \fi
        \fi
    \else
        % should not happen, give the same as pin 1 anyway
        \ifnum#2=0\relax
        \pgfpoint{0pt}{-\pgf@yb+(\pgf@yb-\pgf@ya)/2}
        \else
            \pgfpoint{0pt}{-max(\pgf@ya,\pgf@yb)-\extshift}
        \fi
    \fi
}

% top anchors
\def\pgf@circ@muxdemux@T@anchor#1#2{% #1: pin number #2: 0 for border pin, 1 for external pin
    \topleft
    \pgf@xa=\pgf@x
    \pgf@ya=\pgf@y
    \topright
    \pgf@xb=\pgf@x
    \pgf@yb=\pgf@y
    \ifnum#1>\NT
        \PackageError{circuitikz}{%
            You requested top pin #1 for mux/demux shape \thisshape\space \MessageBreak
            which has been defined with \NT\space top pins%
        }{Please check the manual about mux/demux shapes; if you press return I'll try to continue}
    \fi
    \ifnum\NT>0
        \pgfmathsetlength{\pgf@circ@res@step}{2*\pgf@xb/\NT}
        \pgfmathsetlength{\pgf@circ@res@other}{(\pgf@yb-\pgf@ya)/(\pgf@xb-\pgf@xa)*\pgf@circ@res@step}
        \pgfmathsetlength\pgf@x{\pgf@xa+(#1-0.5)*\pgf@circ@res@step}
        \ifnum#2=0\relax
            \pgfmathsetlength\pgf@y{\pgf@yb-(\NT-#1+0.5)*\pgf@circ@res@other}
        \else
            \ifnum\squarepins>0
                \pgfmathsetlength\pgf@y{max(\pgf@ya,\pgf@yb)+\extshift}
            \else
                \pgfmathsetlength\pgf@y{\pgf@yb-(\NT-#1+0.5)*\pgf@circ@res@other+\extshift}
            \fi
        \fi
    \else
        % should not happen, give the same as pin 1 anyway
        \ifnum#2=0\relax
        \pgfpoint{0pt}{\pgf@yb-(\pgf@yb-\pgf@ya)/2}
        \else
            \pgfpoint{0pt}{max(\pgf@ya,\pgf@yb)+\extshift}
        \fi
    \fi
}

%% IEEE standard logic ports module
%%
%% Original multi-input code from John Kormylo at tex.stackexchange.com
%% Help by TheTeXnician <38565529+TheTeXnician@users.noreply.github.com>
%% Suggested idea and example code by Jason Sachs <jmsachs@gmail.com>
%% Please see https://github.com/circuitikz/circuitikz/issues/383 for a lot of details
%% Most of the code, all errors and bugs by Romano Giannetti <romano.giannetti@gmail.com>
%% Everything is in the same place here --- more or less; first step to move towards
%% a module interface for circutikz
%%

% base parameters for ieeestd ports.

\ctikzset{ieeestd ports/.is family}
% baselen is relative to pgfcirc@Rlen as ever; scaled if the class says so.
% the value of 0.4 is the standard pin distance for a port with height=num pins
% and matches the chip distance
\ctikzset{ieeestd ports/baselen/.initial=0.4}
% these are in term of baselen; width depends on height (fixed proportions)
\ctikzset{ieeestd ports/height/.initial=2}
\ctikzset{ieeestd ports/pin length/.initial=0.7}
% the standard "not" circle should be 1/6.5 of height (diameter);
% so radius/baselen=1/3.25/2    --- using 0.1 and no scaling is as a pole
\ctikzset{ieeestd ports/not radius/.initial=0.154}
\ctikzset{ieeestd ports/not radius fill/.initial=1}% change ony if you know why
% the suggested xnor distance is is 1.24, so 1.25/3.25/2
% xor/xnor leads go full in in IEEE; let this be optional
\ctikzset{ieeestd ports/xor bar distance/.initial=0.192}%
\ctikzset{ieeestd ports/xor leads in/.initial=1}%
%
% base size of a small external schmitt symbol
%
\ctikzset{ieeestd ports/schmitt symbol size/.initial=0.3}%
%
% input management
% we are using the same /tikz/number inputs than the legacy ports
%
\tikzset{/tikz/inner inputs/.initial=0} % using 0 means that all inputs are inner
%
% integrate with the other logic ports
%
\ctikzset{logic ports/ieee/.code= {\pgf@circuit@europeanlogicportfalse
    \tikzset{and port/.style={shape=ieeestd and port}}%
    \tikzset{or port/.style={shape=ieeestd or port}}%
    \tikzset{xor port/.style={shape=ieeestd xor port}}%
    \tikzset{buffer port/.style={shape=ieeestd buffer port}}%
    \tikzset{not port/.style={shape=ieeestd not port}}%
    \tikzset{nand port/.style={shape=ieeestd nand port}}%
    \tikzset{nor port/.style={shape=ieeestd nor port}}%
    \tikzset{xnor port/.style={shape=ieeestd xnor port}}%
    \tikzset{schmitt port/.style={shape=ieeestd schmitt port}}%
    \tikzset{inv schmitt port/.style={shape=ieeestd schmitt port}}%
}}
\tikzset{ieee ports/.style = {\circuitikzbasekey/logic ports = ieee}}
%
% the base angle for the or port. See the drawings. This will not change with height
%
\pgfmathsetmacro{\pgf@circ@orangle}{atan(3.25/6.5)}
% \typeout{ANGLE-IS\space\pgf@circ@orangle}
\def\pgf@circ@ieeeport@input#1% #1 = \pgfmathcounter
{%
    \ifnum#1>\inputs
        \PackageError{circuitikz}{%
            You requested input pin #1 for logic port shape \thisshape\space \MessageBreak
            which has been defined with \inputs\space pins%
        }{Please check the manual about logic ports; if you press return I'll try to continue}
    \fi
    \pgfmathsetlength{\pgf@circ@res@up}{(\inputs/2)*\pind+0.5*\pind}% pin "0", above the rack/port
    \pgfextractx{\pgf@circ@res@left}{\bodyleft}
    \pgf@circ@res@step=\pind
    \pgf@y=\pgf@circ@res@up\advance\pgf@y by -#1\pgf@circ@res@step\relax
    \pgf@x=\pgf@circ@res@left\advance\pgf@x by -\pinlen
}%

% #1 = \pgfmathcounter #2=type
% type is 1 for and,nand; 2 for or,nor; 3 for xor,xnor
\def\pgf@circ@ieeeport@baseinput#1#2%
{%
    \ifnum#1>\inputs
        \PackageError{circuitikz}{%
            You requested border input pin #1 for logic port shape \thisshape\space \MessageBreak
            which has been defined with \inputs\space pins%
        }{Please check the manual about logic ports; if you press return I'll try to continue}
    \fi
    % Find the vertical position (this is the same for any port)
    \pgfmathsetlength{\pgf@circ@res@up}{(\inputs/2)*\pind+0.5*\pind}% pin "0", above the rack/port
    \pgf@circ@res@step=\pind\advance\pgf@circ@res@up by -#1\pgf@circ@res@step\relax
    % rack (extended) pins; they are the same for all the ports
    % call K = (inputs-inner)/2, rounded up; pins on the rack are:
    %      above: 1..K (included)
    %      below: inputs-K..inputs
    % Find the pins on the rack; they are 1...
    \pgf@circ@count@a=\numexpr (\inputs - \inners)/2\relax       % =K; numexpr rounds up!
    \pgf@circ@count@b=\numexpr \inputs - \pgf@circ@count@a +1 \relax % =inputs - K +1
    % border anchors for rack should be ok
    \pgfextractx{\pgf@circ@res@left}{\topleft}
    \pgfextractx{\pgf@circ@res@right}{\bodyleft}
    \pgf@y=\pgf@circ@res@up\pgf@x=\pgf@circ@res@left
    % we have finished if we are in the rack
    \ifnum #1 > \pgf@circ@count@a \ifnum #1 < \pgf@circ@count@b
        % we are on the inner ports; we have to do the hard work here
        % and and nand
        \ifnum #2=1
            \relax % It's an and/nand, all border ports are on the rack line
        \fi
        % or and nor
        \ifnum #2=2
            \pgfmathsetlength{\pgf@x}{\pgf@circ@res@right-2*\stdH*(1-cos(atan(\pgf@circ@res@up/(2*\stdH))))}
        \fi
        % xor and xnor
        \ifnum #2=3\relax
            \pgfmathsetlength{\pgf@x}{\pgf@circ@res@right-\xorbar-2*\stdH*(1-cos(atan(\pgf@circ@res@up/(2*\stdH))))}
        \fi
    \fi\fi
}%
% inner base ports for xor types port
% #1 = \pgfmathcounter #2=type
% type is 1 for and,nand; 2 for or,nor; 3 for xor,xnor
\def\pgf@circ@ieeeport@innerbaseinput#1%
{%
    \ifnum#1>\inputs
        \PackageError{circuitikz}{%
            You requested border input pin #1 for logic port shape \thisshape\space \MessageBreak
            which has been defined with \inputs\space pins%
        }{Please check the manual about logic ports; if you press return I'll try to continue}
    \fi
    % Find the vertical position (this is the same for any port)
    \pgfmathsetlength{\pgf@circ@res@up}{(\inputs/2)*\pind+0.5*\pind}% pin "0", above the rack/port
    \pgf@circ@res@step=\pind\advance\pgf@circ@res@up by -#1\pgf@circ@res@step\relax
    % rack (extended) pins; they are the same for all the ports
    % call K = (inputs-inner)/2, rounded up; pins on the rack are:
    %      above: 1..K (included)
    %      below: inputs-K..inputs
    % Find the pins on the rack; they are 1...
    \pgf@circ@count@a=\numexpr (\inputs - \inners)/2\relax       % =K; numexpr rounds up!
    \pgf@circ@count@b=\numexpr \inputs - \pgf@circ@count@a +1 \relax % =inputs - K +1
    % border anchors for rack should be ok
    \pgfextractx{\pgf@circ@res@left}{\topleft}
    \pgfextractx{\pgf@circ@res@right}{\bodyleft}
    \pgf@y=\pgf@circ@res@up\pgf@x=\pgf@circ@res@left
    % we have finished if we are in the rack
    \ifnum #1 > \pgf@circ@count@a \ifnum #1 < \pgf@circ@count@b
        % we are on the inner ports; we have to do the hard work here
        \pgfmathsetlength{\pgf@x}{\pgf@circ@res@right-2*\stdH*(1-cos(atan(\pgf@circ@res@up/(2*\stdH))))}
    \fi\fi
}%

%%% macro to find basic lenghts --- they leave it in \pgf@circ@res@temp
\def\pgf@circ@ieeestd@baselen{%
    \pgfmathsetlength{\pgf@circ@scaled@Rlen}{\ctikzvalof{\ctikzclass/scale}*\pgf@circ@Rlen}
    \pgfmathsetlength{\pgf@circ@res@temp}{\ctikzvalof{ieeestd ports/baselen}*\pgf@circ@scaled@Rlen}
}
\def\pgf@circ@ieeestd@stdH{%
    \pgfmathsetlength{\pgf@circ@scaled@Rlen}{\ctikzvalof{\ctikzclass/scale}*\pgf@circ@Rlen}
    \pgfmathsetlength{\pgf@circ@res@temp}{0.5*\ctikzvalof{ieeestd ports/baselen}*
        \ctikzvalof{ieeestd ports/height}*\pgf@circ@scaled@Rlen}
}
\def\pgf@circ@ieeestd@pinlen{%
    \pgfmathsetlength{\pgf@circ@scaled@Rlen}{\ctikzvalof{\ctikzclass/scale}*\pgf@circ@Rlen}
    \pgfmathsetlength{\pgf@circ@res@temp}{\ctikzvalof{ieeestd ports/baselen}*
        \ctikzvalof{ieeestd ports/pin length}*\pgf@circ@scaled@Rlen}
}
\def\pgf@circ@ieeestd@xorbar{%
    \pgfmathsetlength{\pgf@circ@scaled@Rlen}{\ctikzvalof{\ctikzclass/scale}*\pgf@circ@Rlen}
    \pgfmathsetlength{\pgf@circ@res@temp}{\ctikzvalof{ieeestd ports/baselen}*
        \ctikzvalof{ieeestd ports/xor bar distance}*\pgf@circ@scaled@Rlen}
}
%% Not circle
\def\pgf@circ@notradius{
    \pgf@circ@ieeestd@stdH % got the standard length. Notice that his is 3.25H for IEEE
    \pgfmathsetlength{\pgf@circ@res@temp}{\ctikzvalof{ieeestd ports/not radius}*\pgf@circ@res@temp}
    % \typeout{NOTRADIUS\space\the\pgf@circ@res@temp}
}
%% Find (in ...@other) the height of the rack
\def\pgf@circ@findrackH{%
    \pgf@circ@count@a=\pgfkeysvalueof{/tikz/number inputs}\relax%
    \pgf@circ@count@b=\pgfkeysvalueof{/tikz/inner inputs}\relax%
    \ifnum\pgf@circ@count@a=0 \pgf@circ@count@a=2\fi  % default pins
    \ifnum\pgf@circ@count@a<2 \pgf@circ@count@a=2\fi %
    \ifnum\pgf@circ@count@b=0 \pgf@circ@count@b=\pgf@circ@count@a\fi%
    \pgf@circ@ieeestd@stdH
    \multiply\pgf@circ@res@temp by 2\relax% full height
    \divide\pgf@circ@res@temp by \pgf@circ@count@b % the pin spacing
    \pgfmathsetlength{\pgf@circ@res@other}{(\pgf@circ@count@a/2)*\pgf@circ@res@temp} %top of the rack/port
    % \typeout{RACK-\thisshape\space\the\pgf@circ@res@other}
}
%%
\def\pgf@circ@find@ieeeport@up{% leave it in up
    % Normal port limits
    \pgf@circ@ieeestd@stdH
    \pgfmathsetlength{\pgf@circ@res@up}{\pgf@circ@res@temp}
    % rack top size
    \pgf@circ@findrackH
    \ifdim\pgf@circ@res@other > \pgf@circ@res@up
        \pgf@circ@res@up=\pgf@circ@res@other
    \else
    \fi
}
\def\pgf@circ@find@ieeeport@left#1{% leave it in left; #1 is type
    % Normal port limits
    \pgf@circ@ieeestd@stdH
    \pgfmathsetlength{\pgf@circ@res@left}{8*\pgf@circ@res@temp/6.5}
    \pgf@circ@res@right=\pgf@circ@res@left % save the border value
    \pgf@circ@res@step=\pgf@circ@res@temp % save the stdH value
    \pgf@circ@ieeestd@pinlen\advance\pgf@circ@res@left by \pgf@circ@res@temp
    % \typeout{LEFT1-\thisshape\space L\space\the\pgf@circ@res@left\space R\space\the\pgf@circ@res@right}
    % this is the normal left border
    % For the or or xor port, the limit can be the pointy thing (in case the
    % pinlen is zero or too small)
    % add to the body margin the or/nor peak:
    \ifnum #1 > 1\relax% "or", "nor", "xor", "xnor" gates.
        \pgfmathsetlength{\pgf@circ@res@other}{2*\pgf@circ@res@step*(1-cos(\pgf@circ@orangle))}
        \advance\pgf@circ@res@right by \pgf@circ@res@other
        % \typeout{LEFT2-\thisshape\space L\space\the\pgf@circ@res@left\space R\space\the\pgf@circ@res@right}
    \fi
    % add to the body margin the xor/xnor distance
    \ifnum #1 = 3\relax% "xor" or "xnor" gates.
        \pgf@circ@ieeestd@xorbar
        \advance\pgf@circ@res@right by \pgf@circ@res@temp
        % \typeout{LEFT3-\thisshape\space L\space\the\pgf@circ@res@left\space R\space\the\pgf@circ@res@right}
    \fi
    % and if this exceeds the normal margin, this is it
    % \typeout{LEFT4-\thisshape\space L\space\the\pgf@circ@res@left\space R\space\the\pgf@circ@res@right}
    \ifdim \pgf@circ@res@right > \pgf@circ@res@left
        \pgf@circ@res@left=\pgf@circ@res@right
    \fi
    % \typeout{LEFT5-\thisshape\space L\space\the\pgf@circ@res@left\space R\space\the\pgf@circ@res@right}
    \pgf@circ@res@left=-\pgf@circ@res@left
}
\def\pgf@circ@find@ieeeport@right#1{% leave it in right; #1 is plain or negated
    % Normal port limits
    \pgf@circ@ieeestd@stdH
    \pgfmathsetlength{\pgf@circ@res@right}{8*\pgf@circ@res@temp/6.5}
    \pgf@circ@notradius
    \pgfmathsetlength{\pgf@circ@res@other}{\pgf@circ@res@right+2*#1*\pgf@circ@res@temp}
    \pgf@circ@ieeestd@pinlen\advance\pgf@circ@res@right by \pgf@circ@res@temp
    \ifdim\pgf@circ@res@other > \pgf@circ@res@right
        \pgf@circ@res@right=\pgf@circ@res@other
    \fi
}
\def\pgf@circ@find@ieeeport@not@right#1{% leave it in right; #1 is plain or negated
    % Normal port limits
    \pgf@circ@ieeestd@stdH
    % notice 0.8660254 is cos(30)
    \pgfmathsetlength{\pgf@circ@res@right}{0.8660254*\pgf@circ@res@temp}
    \pgf@circ@notradius
    \pgfmathsetlength{\pgf@circ@res@other}{\pgf@circ@res@right+2*#1*\pgf@circ@res@temp}
    \pgf@circ@ieeestd@pinlen\advance\pgf@circ@res@right by \pgf@circ@res@temp
    \ifdim\pgf@circ@res@other > \pgf@circ@res@right
        \pgf@circ@res@right=\pgf@circ@res@other
    \fi
}


%%% ieeestd multi-input ports
%%% #1: name
%%% #2: type: 1 for and,nand; 2 for or,nor; 3 for xor,xnor
%%% #3: polarity: 0 for direct, 1 for inverted (not at the output)
%%% #4: drawing for the port
\long\def\pgfcircdeclareieeeport#1#2#3#4{%
    \pgfdeclareshape{ieeestd #1 port}%
    {%
        \savedmacro{\ctikzclass}{\edef\ctikzclass{logic ports}}
        \saveddimen{\scaledRlen}{\pgfmathsetlength{\pgf@x}{\ctikzvalof{\ctikzclass/scale}\pgf@circ@Rlen}}
        \savedmacro{\thisshape}{\def\thisshape{\tikz@fig@name}}
        \saveddimen{\baselen}{%
            \pgf@circ@ieeestd@baselen\pgf@x=\pgf@circ@res@temp
        }
        \saveddimen{\stdH}{% This is HALF the height of the inner port
            \pgf@circ@ieeestd@stdH\pgf@x=\pgf@circ@res@temp
        }
        \saveddimen{\notdiameter}{
            \pgf@circ@notradius\pgf@x=2\pgf@circ@res@temp
        }
        \saveddimen{\pind}{% pin distance;
            \pgf@circ@count@a=\pgfkeysvalueof{/tikz/number inputs}\relax%
            \pgf@circ@count@b=\pgfkeysvalueof{/tikz/inner inputs}\relax%
            \ifnum\pgf@circ@count@a=0 \pgf@circ@count@a=2\fi  % default pins
            \ifnum\pgf@circ@count@a<2 \pgf@circ@count@a=2\fi %
            \ifnum\pgf@circ@count@b=0 \pgf@circ@count@b=\pgf@circ@count@a\fi%
            \pgf@circ@ieeestd@stdH\pgf@x=2\pgf@circ@res@temp % full height
            \divide\pgf@x by \pgf@circ@count@b
        }
        \saveddimen{\pinlen}{%
            \pgf@circ@ieeestd@pinlen\pgf@x=\pgf@circ@res@temp
        }
        \saveddimen{\xorbar}{%
            \pgf@circ@ieeestd@xorbar\pgf@x=\pgf@circ@res@temp
        }
        % anchors for the body (no pins included here)
        \savedanchor{\bodyleft}{% This DOES NOT take into account the pointy or/xor thing
            \pgf@circ@ieeestd@stdH
            \pgfmathsetlength{\pgf@x}{-8*\pgf@circ@res@temp/6.5}
            \pgfmathsetlength{\pgf@y}{\pgf@circ@res@temp}
        }
        \savedanchor{\topleft}{%
            \pgf@circ@ieeestd@xorbar\pgf@circ@res@right=\pgf@circ@res@temp % save \xorbar
            \pgf@circ@ieeestd@stdH
            \pgf@circ@res@other=0pt\relax
            \ifnum #2 = 2\relax% "or" or "nor" gates.
                \pgfmathsetlength{\pgf@circ@res@other}{2*\pgf@circ@res@temp*(1-cos(\pgf@circ@orangle))}
            \fi
            \ifnum #2 = 3\relax% "xor" or "xnor" gates.
                \pgfmathsetlength{\pgf@circ@res@other}{2*\pgf@circ@res@temp*(1-cos(\pgf@circ@orangle))
                    +\pgf@circ@res@right}
            \fi
            \pgfmathsetlength{\pgf@x}{-8*\pgf@circ@res@temp/6.5-\pgf@circ@res@other}
            \pgfmathsetlength{\pgf@y}{\pgf@circ@res@temp}
        }
        \savedanchor{\bodyright}{% This DOES NOT take into account the "NOT" circle
            \pgf@circ@ieeestd@stdH
            \pgfmathsetlength{\pgf@x}{8*\pgf@circ@res@temp/6.5}
            \pgfmathsetlength{\pgf@y}{-\pgf@circ@res@temp}
        }
        \savedanchor{\bottomright}{% This DOES take into account the "NOT" circle
            \pgf@circ@notradius\pgf@circ@res@other=\pgf@circ@res@temp
            \pgf@circ@ieeestd@stdH
            % #3 is =1 if the port is a negated output one
            \pgfmathsetlength{\pgf@x}{8*\pgf@circ@res@temp/6.5+#3*2*\pgf@circ@res@other}
            \pgfmathsetlength{\pgf@y}{-\pgf@circ@res@temp}
        }
        % geographical anchors --- must be rectangulars!
        \savedanchor{\northwest}{%
            \pgf@circ@find@ieeeport@up
            \pgf@circ@find@ieeeport@left{#2}
            % \typeout{ANCH5-\thisshape\space L\space\the\pgf@circ@res@left\space U\space\the\pgf@circ@res@up}
            \pgf@x=\pgf@circ@res@left
            \pgf@y=\pgf@circ@res@up
        }
        \savedanchor{\southwest}{%
            \pgf@circ@find@ieeeport@up
            \pgf@circ@find@ieeeport@left{#2}
            \pgf@x=\pgf@circ@res@left
            \pgf@y=-\pgf@circ@res@up
        }
        \savedanchor{\southeast}{%
            \pgf@circ@find@ieeeport@up
            \pgf@circ@find@ieeeport@right{#3}
            \pgf@x=\pgf@circ@res@right
            \pgf@y=-\pgf@circ@res@up
        }
        \savedanchor{\northeast}{%
            \pgf@circ@find@ieeeport@up
            \pgf@circ@find@ieeeport@right{#3}
            \pgf@x=\pgf@circ@res@right
            \pgf@y=\pgf@circ@res@up
        }
        \savedmacro\inputs{% get number of inputs
            \pgf@circ@count@a=\pgfkeysvalueof{/tikz/number inputs}\relax
            \ifnum\pgf@circ@count@a=0\pgf@circ@count@a=2\fi    % default
            \ifnum\pgf@circ@count@a<2 \pgf@circ@count@a=2\fi   % minimum pins
            % \ifnum\pgf@circ@count@a>16 \pgf@circ@count@a=16\fi
            \def\inputs{\the\pgf@circ@count@a}%
        }%
        \savedmacro\inners{% get number of "inner" inputs (for racks)
            \pgf@circ@count@a=\pgfkeysvalueof{/tikz/number inputs}\relax
            \pgf@circ@count@b=\pgfkeysvalueof{/tikz/inner inputs}\relax
            \ifnum\pgf@circ@count@a=0 \pgf@circ@count@a=2\fi  % default pins
            \ifnum\pgf@circ@count@a<2 \pgf@circ@count@a=2\fi  % minimum pins
            \ifnum\pgf@circ@count@b=0 \pgf@circ@count@b=\pgf@circ@count@a\fi
            % \typeout{INNER is \the\pgf@circ@count@b}%
            % \ifnum\pgf@circ@count@a>16 \pgf@circ@count@a=16\fi
            \def\inners{\the\pgf@circ@count@b}%
        }%
        \anchor{center}{\pgfpointorigin}
        \anchor{text}{
            \ifpgf@circ@center@text
                \pgfpoint{-.5\wd\pgfnodeparttextbox}{\dimexpr.5\dp\pgfnodeparttextbox-.5\ht\pgfnodeparttextbox}
            \else
                \pgf@circ@ieeestd@stdH
                \pgfmathsetlength{\pgf@circ@res@left}{-8*\pgf@circ@res@temp/6.5} % left border
                \pgfpoint{\pgf@circ@res@left + \ctikzvalof{left text distance}}{\dimexpr.5\dp\pgfnodeparttextbox-.5\ht\pgfnodeparttextbox}
            \fi
        }
        % create input anchors
        \expandafter\pgfutil@g@addto@macro\csname pgf@sh@s@ieeestd #1 port\endcsname{%
            \pgf@circ@count@a=\inputs
            \pgfmathloop%
            \ifnum\pgfmathcounter>\pgf@circ@count@a%
            \else%
            %\pgfutil@ifundefined{pgf@anchor@american #1 port@in \pgfmathcounter}{%
            \expandafter\xdef\csname pgf@anchor@ieeestd #1 port@in \pgfmathcounter\endcsname{%
                \noexpand\pgf@circ@ieeeport@input{\pgfmathcounter}% defined above
            }%
            \expandafter\xdef\csname pgf@anchor@ieeestd #1 port@bin \pgfmathcounter\endcsname{%
                \noexpand\pgf@circ@ieeeport@baseinput{\pgfmathcounter}{#2}% defined above
            }%
            \ifnum #2 = 3\relax % xor/xnor inner border pins
                \expandafter\xdef\csname pgf@anchor@ieeestd #1 port@ibin \pgfmathcounter\endcsname{%
                    \noexpand\pgf@circ@ieeeport@innerbaseinput{\pgfmathcounter}% defined above
                }%
            \fi
            %}{}%
            \repeatpgfmathloop%
        }
        % output anchor
        \anchor{out}{%
            \pgfextractx{\pgf@circ@res@other}{\bodyright}   %body border, without not ball
            \advance\pgf@circ@res@other by\pinlen
            \pgfextractx{\pgf@circ@res@temp}{\bottomright}   %body + ball border
            \ifdim \pgf@circ@res@temp > \pgf@circ@res@other
                \pgf@circ@res@other = \pgf@circ@res@temp % do not enter in the ball...
            \fi
            \pgf@x=\pgf@circ@res@other\pgf@y=0pt
        }
        \anchor{bout}{\bottomright\pgf@y=0pt}

        \anchor{body right}{\bodyright\pgf@y=0pt}
        \anchor{right}{\bottomright\pgf@y=0pt}
        \anchor{body left}{\bodyleft\pgf@y=0pt}% central edge of the body
        \anchor{left}{% central edge of the component
            \bodyleft\pgf@y=0pt
            \ifnum #2=3\relax
                \advance\pgf@x by -\xorbar
            \fi
        }
        \anchor{up}{%
            \bodyleft
            \ifnum #2 > 1 % pointy shapes
                \pgf@circ@ieeestd@stdH
                % horizontal coordinate where the right semicircle starts
                \pgfmathsetlength{\pgf@circ@res@other}{-8*\pgf@circ@res@temp/6.5+2*\pgf@circ@res@temp*cos(\pgf@circ@orangle)}
                % vertical drop of the circle at the above coordinate
                \pgfmathsetlength{\pgf@circ@res@step}{2*\pgf@circ@res@temp*(1-cos(atan(\pgf@circ@res@other/(2*\pgf@circ@res@temp))))}
                \advance \pgf@y by -\pgf@circ@res@step
            \fi
            \pgf@x=0pt
        }
        \anchor{down}{%
            \bodyleft
            \ifnum #2 > 1 % pointy shapes
                \pgf@circ@ieeestd@stdH
                % horizontal coordinate where the right semicircle starts
                \pgfmathsetlength{\pgf@circ@res@other}{-8*\pgf@circ@res@temp/6.5+2*\pgf@circ@res@temp*cos(\pgf@circ@orangle)}
                % vertical drop of the circle at the above coordinate
                \pgfmathsetlength{\pgf@circ@res@step}{2*\pgf@circ@res@temp*(1-cos(atan(\pgf@circ@res@other/(2*\pgf@circ@res@temp))))}
                \advance \pgf@y by -\pgf@circ@res@step
            \fi
            \pgf@y=-\pgf@y\pgf@x=0pt
        }

        % geographical anchors
        \anchor{nw}{\northwest}
        \anchor{ne}{\northeast}
        \anchor{se}{\southeast}
        \anchor{sw}{\southwest}
        \anchor{north west}{\northwest}
        \anchor{north east}{\northeast}
        \anchor{south east}{\southeast}
        \anchor{south west}{\southwest}
        % over 0,0 even if asymmetric
        % will break if the geocoords are not rectangular
        \anchor{n}{\northwest\pgf@x=0pt\relax}
        \anchor{e}{\northeast\pgf@y=0pt\relax}
        \anchor{s}{\southwest\pgf@x=0pt\relax}
        \anchor{w}{\northwest\pgf@y=0pt\relax}
        \anchor{north}{\northwest\pgf@x=0pt\relax}
        \anchor{east}{\northeast\pgf@y=0pt\relax}
        \anchor{south}{\southwest\pgf@x=0pt\relax}
        \anchor{west}{\northwest\pgf@y=0pt\relax}

        \backgroundpath{
            \pgfscope
                \pgfsetcolor{\ctikzvalof{color}}
                #4
            \endpgfscope
            % output lead:
            \pgfextractx{\pgf@circ@res@right}{\bottomright} %body+ball border
            \pgfextractx{\pgf@circ@res@other}{\bodyright}   %body border, without "not" ball
            \advance\pgf@circ@res@other by \pinlen\relax
            \ifdim \pgf@circ@res@other > \pgf@circ@res@right
                \pgfpathmoveto{\pgfpoint{\pgf@circ@res@right}{0pt}}
                \pgfpathlineto{\pgfpoint{\pgf@circ@res@other}{0pt}}
                \pgfusepath{draw}
            \fi
            \ifnum #3=1\relax\pgfscope
                \pgftransformshift{\pgfpoint{\pgf@circ@res@right}{0pt}}
                \pgfnode{notcirc}{east}{}{\thisshape-not}{\pgfusepath{stroke}}
            \endpgfscope\fi
        }
    }
}

%%% #1 direct name #2 negated name #3 type #4 drawing (without output)
\long\def\pgfcircdeclareieeeportpair#1#2#3#4{%
    \pgfcircdeclareieeeport{#1}{#3}{0}{#4}% direct
    \pgfcircdeclareieeeport{#2}{#3}{1}{#4}% negated
}
%
% ieeestd "and" and "nand"
%
\pgfcircdeclareieeeportpair{and}{nand}{1}{%
    \pgf@circ@count@a = \inputs\relax
    \pgfmathsetlength{\pgf@circ@res@up}{(\inputs/2)*\pind} %top of the rack/port
    \pgfmathsetlength{\pgf@circ@res@temp}{\pgf@circ@res@up+0.5*\pind}
    \pgfextractx{\pgf@circ@res@left}{\bodyleft}
    \pgfextracty{\pgf@circ@res@down}{\bodyleft}
    \ifpgfcirc@draw@leads
        %input leads --- all the same for AND ports
        \loop\ifnum\pgf@circ@count@a>0
        \advance\pgf@circ@res@temp by -\pind
        \pgfpathmoveto{\pgfpoint{\pgf@circ@res@left}{\pgf@circ@res@temp}}%
        \pgfpathlineto{\pgfpoint{\pgf@circ@res@left-\pinlen}{\pgf@circ@res@temp}}%
        \advance\pgf@circ@count@a by -1
        \repeat
    \fi
    \pgfusepath{draw}
    %% Body. let's start from the top left
    \pgfscope
        \pgf@circ@setlinewidth{tripoles}{\pgflinewidth}
        \pgfpathmoveto{\pgfpoint{-8*\stdH/6.5}{\stdH}}
        \pgfpathlineto{\pgfpoint{1.5*\stdH/6.5}{\stdH}}
        \pgfpatharc{90}{-90}{\stdH}
        \pgfpathlineto{\pgfpoint{-8*\stdH/6.5}{-\stdH}}
        \pgfpathclose
        \pgf@circ@draworfill
        % rack now; skip if not needed.
        % \typeout{WHAT\space \inputs\space \inners}
        \ifnum\inputs>\inners
            \pgfpathmoveto{\pgfpoint{\pgf@circ@res@left}{\pgf@circ@res@up}}
            \pgfpathlineto{\pgfpoint{\pgf@circ@res@left}{\pgf@circ@res@down}}
            \pgfpathmoveto{\pgfpoint{\pgf@circ@res@left}{-\pgf@circ@res@down}}
            \pgfpathlineto{\pgfpoint{\pgf@circ@res@left}{-\pgf@circ@res@up}}
            \pgfusepath{draw}
        \fi
    \endpgfscope
}
%
% or/nor and xor/xnor are practically the same. Let factor out everything
% the argument #1 is put just before the drawing of the inner pins
% the argument #2 is put after the drawing of the body
%
\long\def\pgf@circ@ieeeport@orxor#1#2{%
    \pgf@circ@count@a = \inputs\relax
    \pgfmathsetlength{\pgf@circ@res@up}{(\inputs/2)*\pind} %top of the rack/port
    \pgfmathsetlength{\pgf@circ@res@temp}{\pgf@circ@res@up+0.5*\pind}
    \pgfextractx{\pgf@circ@res@left}{\bodyleft}
    \pgfextracty{\pgf@circ@res@down}{\bodyleft}
    % rack (extended) pins; they are the same for all the ports
    % call K = (inputs-inner)/2, rounded up; pins on the rack are:
    %      above: 1..K (included)
    %      below: inputs-K..inputs
    % Find the pins on the rack; they are 1...
    \pgf@circ@count@b=\numexpr (\inputs - \inners)/2\relax       % =K; numexpr rounds up!
    \pgf@circ@count@c=\numexpr \inputs - \pgf@circ@count@b +1 \relax % =inputs - K +1
    \ifpgfcirc@draw@leads
        %input leads --- for or ports
        \loop\ifnum\pgf@circ@count@a>0
        \pgfextractx{\pgf@circ@res@right}{\topleft}
        \advance\pgf@circ@res@temp by -\pind
        % this is the height; let's find the "right" position
        \ifnum \pgf@circ@count@a > \pgf@circ@count@b \ifnum \pgf@circ@count@a < \pgf@circ@count@c
            % inner pins
            % \typeout{INNER\space\pgf@circ@count@a}
            \pgfmathsetlength{\pgf@circ@res@right}{\pgf@circ@res@left-2*\stdH*(1-cos(atan(\pgf@circ@res@temp/(2*\stdH))))}
            % hook for xor/xnor
            #1
        \fi\fi
        \pgfpathmoveto{\pgfpoint{\pgf@circ@res@right}{\pgf@circ@res@temp}}%
        \pgfpathlineto{\pgfpoint{\pgf@circ@res@left-\pinlen}{\pgf@circ@res@temp}}%
        \pgfusepath{draw}
        \advance\pgf@circ@count@a by -1
        \repeat
    \fi
    %% Body. let's start from the top left
    \pgfscope
        \pgf@circ@setlinewidth{tripoles}{\pgflinewidth}
        % it should start here, but with this trick the close comes out better.
        % \pgfpathmoveto{\pgfpoint{-8*\stdH/6.5-2*\stdH*(1-cos(\pgf@circ@orangle))}{\stdH}}
        \pgfpathmoveto{\pgfpoint{-8*\stdH/6.5}{\stdH}}
        \pgfpathlineto{\pgfpoint{8*\stdH/6.5-2*\stdH*cos(\pgf@circ@orangle)}{\stdH}} %
        \pgfpatharcto{2*\stdH}{2*\stdH}{0}{0}{0}{\pgfpoint{8*\stdH/6.5}{0pt}}
        \pgfpatharcto{2*\stdH}{2*\stdH}{0}{0}{0}{\pgfpoint{8*\stdH/6.5-2*\stdH*cos(\pgf@circ@orangle)}{-\stdH}}
        \pgfpathlineto{\pgfpoint{-8*\stdH/6.5-2*\stdH*(1-cos(\pgf@circ@orangle))}{-\stdH}}
        %% this should be 2 and 2; but the round part is not a perfect circle that way
        %% so the 2.15 is ajusted "by taste" to touch the anchors exactly.
        \pgfpatharcto{2*\stdH}{2.2*\stdH}{0}{0}{1}{\pgfpoint{-8*\stdH/6.5-2*\stdH*(1-cos(\pgf@circ@orangle))}{\stdH}}
        \pgfpathclose
        \pgf@circ@draworfill
        % hook for xor/xnor
        #2
        % rack now; skip if not needed.
        % \typeout{WHAT\space \inputs\space \inners}
        \ifnum\inputs>\inners
            \pgfextractx{\pgf@circ@res@left}{\topleft}
            \pgfextracty{\pgf@circ@res@down}{\topleft}
            \pgfpathmoveto{\pgfpoint{\pgf@circ@res@left}{\pgf@circ@res@up}}
            \pgfpathlineto{\pgfpoint{\pgf@circ@res@left}{\pgf@circ@res@down}}
            \pgfpathmoveto{\pgfpoint{\pgf@circ@res@left}{-\pgf@circ@res@down}}
            \pgfpathlineto{\pgfpoint{\pgf@circ@res@left}{-\pgf@circ@res@up}}
            \pgfusepath{draw}
        \fi
    \endpgfscope
}
%
% ieeestd "or" and "nor"
%
\pgfcircdeclareieeeportpair{or}{nor}{2}{%
    \pgf@circ@ieeeport@orxor{}{}
}
\pgfcircdeclareieeeportpair{xor}{xnor}{3}{%
    \pgf@circ@ieeeport@orxor{
        \edef\@@tmp{\ctikzvalof{ieeestd ports/xor leads in}}
        \ifnum\@@tmp=0\relax
            % move pin start to the left to leave the xor gap free (not standard)
            \advance\pgf@circ@res@right by -\xorbar
        \fi
        }{%
        % add the xor/xnor bar
        \pgfpathmoveto{\pgfpoint{-\xorbar-8*\stdH/6.5-2*\stdH*(1-cos(\pgf@circ@orangle))}{-\stdH}}
        % see the comment on the main body about the 2.2
        \pgfpatharcto{2*\stdH}{2.2*\stdH}{0}{0}{1}{\pgfpoint{-\xorbar -8*\stdH/6.5-2*\stdH*(1-cos(\pgf@circ@orangle))}{\stdH}}
        \pgfusepath{draw}
    }
}
%
% Buffer and inverters
%
% #1: name
% #2: polarity
% #3: content
\long\def\pgfcircdeclareieeebufferport#1#2#3{%
    \pgfdeclareshape{ieeestd #1 port}%
    {%
        \savedmacro{\ctikzclass}{\edef\ctikzclass{logic ports}}
        \saveddimen{\scaledRlen}{\pgfmathsetlength{\pgf@x}{\ctikzvalof{\ctikzclass/scale}\pgf@circ@Rlen}}
        \savedmacro{\thisshape}{\def\thisshape{\tikz@fig@name}}
        \saveddimen{\baselen}{%
            \pgf@circ@ieeestd@baselen\pgf@x=\pgf@circ@res@temp
        }
        \saveddimen{\stdH}{% This is HALF the height of the inner port
            \pgf@circ@ieeestd@stdH\pgf@x=\pgf@circ@res@temp
        }
        \saveddimen{\notdiameter}{
            \pgf@circ@notradius\pgf@x=2\pgf@circ@res@temp
        }
        \saveddimen{\pinlen}{%
            \pgf@circ@ieeestd@pinlen\pgf@x=\pgf@circ@res@temp
        }
        % anchors for the body (no pins included here)
        % Notice that 0.8660254 is cos(30)
        \savedanchor{\bodyleft}{% This DOES NOT take into account the pointy or/xor thing
            \pgf@circ@ieeestd@stdH
            \pgfmathsetlength{\pgf@x}{-0.8660254*\pgf@circ@res@temp}
            \pgfmathsetlength{\pgf@y}{\pgf@circ@res@temp}
        }
        \savedanchor{\topleft}{%
            \pgf@circ@ieeestd@stdH
            \pgfmathsetlength{\pgf@x}{-0.8660254*\pgf@circ@res@temp}
            \pgfmathsetlength{\pgf@y}{\pgf@circ@res@temp}
        }
        \savedanchor{\bodyright}{% This DOES NOT take into account the "NOT" circle
            \pgf@circ@ieeestd@stdH
            \pgfmathsetlength{\pgf@x}{0.8660254*\pgf@circ@res@temp}
            \pgfmathsetlength{\pgf@y}{\pgf@circ@res@temp}
        }
        \savedanchor{\bottomright}{% This DOES take into account the "NOT" circle
            \pgf@circ@notradius\pgf@circ@res@other=\pgf@circ@res@temp
            \pgf@circ@ieeestd@stdH
            % #2 is =1 if the port is a negated output one
            \pgfmathsetlength{\pgf@x}{0.8660254*\pgf@circ@res@temp+#2*2*\pgf@circ@res@other}
            \pgfmathsetlength{\pgf@y}{-\pgf@circ@res@temp}
        }
        % geographical anchors --- must be rectangulars!
        \savedanchor{\northwest}{%
            \pgf@circ@ieeestd@stdH
            \pgfmathsetlength{\pgf@x}{-0.8660254*\pgf@circ@res@temp}
            \pgfmathsetlength{\pgf@y}{\pgf@circ@res@temp}
        }
        \savedanchor{\southwest}{%
            \pgf@circ@ieeestd@stdH
            \pgfmathsetlength{\pgf@x}{-0.8660254*\pgf@circ@res@temp}
            \pgfmathsetlength{\pgf@y}{-\pgf@circ@res@temp}
        }
        \savedanchor{\southeast}{%
            \pgf@circ@ieeestd@stdH
            \pgf@circ@res@up=\pgf@circ@res@temp
            \pgf@circ@find@ieeeport@not@right{#2}
            \pgf@x=\pgf@circ@res@right
            \pgf@y=-\pgf@circ@res@up
        }
        \savedanchor{\northeast}{%
            \pgf@circ@ieeestd@stdH
            \pgf@circ@res@up=\pgf@circ@res@temp
            \pgf@circ@find@ieeeport@not@right{#2}
            \pgf@x=\pgf@circ@res@right
            \pgf@y=\pgf@circ@res@up
        }
        \anchor{center}{\pgfpointorigin}
        \anchor{text}{
            \ifpgf@circ@center@text
                \pgfpoint{-.5\wd\pgfnodeparttextbox}{\dimexpr.5\dp\pgfnodeparttextbox-.5\ht\pgfnodeparttextbox}
            \else
                \pgf@circ@ieeestd@stdH
                \pgfpoint{-0.8660254*\pgf@circ@res@temp + \ctikzvalof{left text distance}}{\dimexpr.5\dp\pgfnodeparttextbox-.5\ht\pgfnodeparttextbox}
            \fi
        }
        % input anchors
        \anchor{in}{\bodyleft\pgf@y=0pt\advance\pgf@x by-\pinlen}
        \anchor{in 1}{\bodyleft\pgf@y=0pt\advance\pgf@x by-\pinlen}
        \anchor{bin}{\bodyleft\pgf@y=0pt\relax}
        \anchor{bin 1}{\bodyleft\pgf@y=0pt\relax}
        % output anchors
        \anchor{out}{%
            \pgfextractx{\pgf@circ@res@other}{\bodyright}   %body border, without not ball
            \advance\pgf@circ@res@other by\pinlen
            \pgfextractx{\pgf@circ@res@temp}{\bottomright}   %body + ball border
            \ifdim \pgf@circ@res@temp > \pgf@circ@res@other
                \pgf@circ@res@other = \pgf@circ@res@temp % do not enter in the ball...
            \fi
            \pgf@x=\pgf@circ@res@other\pgf@y=0pt
        }
        \anchor{bout}{\bottomright\pgf@y=0pt}

        \anchor{body right}{\bodyright\pgf@y=0pt}
        \anchor{right}{\bottomright\pgf@y=0pt}
        \anchor{body left}{\bodyleft\pgf@y=0pt}% central edge of the body
        \anchor{left}{% central edge of the component
            \pgf@circ@ieeestd@xorbar% save \xorbar in temp
            \bodyleft\pgf@y=0pt\advance\pgf@x by -\pgf@circ@res@temp
        }
        \anchor{up}{%
            \bodyleft
            \pgf@y=+0.5\pgf@y\pgf@x=0pt
        }
        \anchor{down}{%
            \bodyleft
            \pgf@y=-0.5\pgf@y\pgf@x=0pt
        }

        % geographical anchors
        \anchor{nw}{\northwest}
        \anchor{ne}{\northeast}
        \anchor{se}{\southeast}
        \anchor{sw}{\southwest}
        \anchor{north west}{\northwest}
        \anchor{north east}{\northeast}
        \anchor{south east}{\southeast}
        \anchor{south west}{\southwest}
        % over 0,0 even if asymmetric
        % will break if the geocoords are not rectangular
        \anchor{n}{\northwest\pgf@x=0pt\relax}
        \anchor{e}{\northeast\pgf@y=0pt\relax}
        \anchor{s}{\southwest\pgf@x=0pt\relax}
        \anchor{w}{\northwest\pgf@y=0pt\relax}
        \anchor{north}{\northwest\pgf@x=0pt\relax}
        \anchor{east}{\northeast\pgf@y=0pt\relax}
        \anchor{south}{\southwest\pgf@x=0pt\relax}
        \anchor{west}{\northwest\pgf@y=0pt\relax}

        \backgroundpath{
            \pgfscope
                \pgfsetcolor{\ctikzvalof{color}}
                #3
            \endpgfscope
            % output lead:
            \pgfextractx{\pgf@circ@res@right}{\bottomright} %body+ball border
            \pgfextractx{\pgf@circ@res@other}{\bodyright}   %body border, without "not" ball
            \advance\pgf@circ@res@other by \pinlen\relax
            \ifdim \pgf@circ@res@other > \pgf@circ@res@right
                \pgfpathmoveto{\pgfpoint{\pgf@circ@res@right}{0pt}}
                \pgfpathlineto{\pgfpoint{\pgf@circ@res@other}{0pt}}
                \pgfusepath{draw}
            \fi
            \ifnum #2=1\relax\pgfscope
                \pgftransformshift{\pgfpoint{\pgf@circ@res@right}{0pt}}
                \pgfnode{notcirc}{east}{}{\thisshape-not}{\pgfusepath{stroke}}
            \endpgfscope\fi
        }
    }
}

%%% #1 direct name #2 negated name #3 drawing (without output)
\long\def\pgfcircdeclareieeebufferportpair#1#2#3{%
    \pgfcircdeclareieeebufferport{#1}{0}{#3}% direct
    \pgfcircdeclareieeebufferport{#2}{1}{#3}% negated
}

\pgfcircdeclareieeebufferportpair{buffer}{not}{%
    \pgfextractx{\pgf@circ@res@left}{\bodyleft}
    \pgfextracty{\pgf@circ@res@up}{\bodyleft}
    \pgfextractx{\pgf@circ@res@right}{\bodyright}
    % \draw input pin
    \pgfpathmoveto{\pgfpoint{\pgf@circ@res@left}{0pt}}
    \pgfpathlineto{\pgfpoint{\pgf@circ@res@left-\pinlen}{0pt}}
    \pgfusepath{draw}
    \pgfscope
        \pgf@circ@setlinewidth{tripoles}{\pgflinewidth}
        \pgfpathmoveto{\pgfpoint{\pgf@circ@res@left}{\pgf@circ@res@up}}
        \pgfpathlineto{\pgfpoint{\pgf@circ@res@right}{0pt}}
        \pgfpathlineto{\pgfpoint{\pgf@circ@res@left}{-\pgf@circ@res@up}}
        \pgfpathclose
        \pgf@circ@draworfill
    \endpgfscope
}

\pgfcircdeclareieeebufferportpair{schmitt}{inv schmitt}{%
    \pgfextractx{\pgf@circ@res@left}{\bodyleft}
    \pgfextracty{\pgf@circ@res@up}{\bodyleft}
    \pgfextractx{\pgf@circ@res@right}{\bodyright}
    % \draw input pin
    \pgfpathmoveto{\pgfpoint{\pgf@circ@res@left}{0pt}}
    \pgfpathlineto{\pgfpoint{\pgf@circ@res@left-\pinlen}{0pt}}
    \pgfusepath{draw}
    \pgfscope
        \pgf@circ@setlinewidth{tripoles}{\pgflinewidth}
        \pgfpathmoveto{\pgfpoint{\pgf@circ@res@left}{\pgf@circ@res@up}}
        \pgfpathlineto{\pgfpoint{\pgf@circ@res@right}{0pt}}
        \pgfpathlineto{\pgfpoint{\pgf@circ@res@left}{-\pgf@circ@res@up}}
        \pgfpathclose
        \pgf@circ@draworfill
    \endpgfscope
    % draw schmitt symbol in normal line thickness
    \pgfpathmoveto{\pgfpoint{0.75*\pgf@circ@res@left}{-0.25*\pgf@circ@res@up}}
    \pgfpathlineto{\pgfpoint{0.50*\pgf@circ@res@left}{-0.25*\pgf@circ@res@up}}
    \pgfpathlineto{\pgfpoint{0.50*\pgf@circ@res@left}{0.25*\pgf@circ@res@up}}
    \pgfpathlineto{\pgfpoint{0.00*\pgf@circ@res@left}{0.25*\pgf@circ@res@up}}
    \pgfpathmoveto{\pgfpoint{0.25*\pgf@circ@res@left}{0.25*\pgf@circ@res@up}}
    \pgfpathlineto{\pgfpoint{0.25*\pgf@circ@res@left}{-0.25*\pgf@circ@res@up}}
    \pgfpathlineto{\pgfpoint{0.5*\pgf@circ@res@left}{-0.25*\pgf@circ@res@up}}
    \pgfusepath{draw}
}


\pgfdeclareshape{schmitt symbol}{
    \savedmacro{\ctikzclass}{\edef\ctikzclass{logic ports}}
    \anchor{center}{\pgfpointorigin}
    \savedanchor\northwest{%
        \pgf@circ@ieeestd@stdH
        \pgf@y=\ctikzvalof{ieeestd ports/schmitt symbol size}\pgf@circ@res@temp
        \pgf@x=-1.5\pgf@y
    }
    \anchor{center}{\pgf@y=0pt \pgf@x=0pt}
    \anchor{east}{\northwest\pgf@y=0pt\pgf@x=-\pgf@x}
    \anchor{e}{\northwest\pgf@y=0pt \pgf@x=-\pgf@x}
    \anchor{west}{\northwest\pgf@y=0pt}
    \anchor{w}{\northwest \pgf@y=0pt}
    \anchor{south}{\northwest \pgf@x=0pt\pgf@y=-\pgf@y}
    \anchor{s}{\northwest\pgf@x=0pt\pgf@y=-\pgf@y}
    \anchor{north}{\northwest\pgf@x=0pt}
    \anchor{n}{\northwest\pgf@x=0pt}
    \anchor{south west}{\northwest\pgf@y=-\pgf@y}
    \anchor{north east}{\northwest\pgf@x=-\pgf@x}
    \anchor{north west}{\northwest}
    \anchor{south east}{\northwest\pgf@x=-\pgf@x\pgf@y=-\pgf@y}
    \backgroundpath{
        \pgfscope
            \pgfsetcolor{\ctikzvalof{color}}
            \pgfextractx{\pgf@circ@res@left}{\northwest}
            \pgfextracty{\pgf@circ@res@up}{\northwest}
            \pgf@circ@res@left=0.7\pgf@circ@res@left
            \pgf@circ@res@up=0.7\pgf@circ@res@up
            \pgfpathmoveto{\pgfpoint{\pgf@circ@res@left}{-\pgf@circ@res@up}}
            \pgfpathlineto{\pgfpoint{\pgf@circ@res@left/3}{-\pgf@circ@res@up}}
            \pgfpathlineto{\pgfpoint{\pgf@circ@res@left/3}{\pgf@circ@res@up}}
            \pgfpathlineto{\pgfpoint{-\pgf@circ@res@left}{\pgf@circ@res@up}}
            \pgfpathmoveto{\pgfpoint{-\pgf@circ@res@left/3}{\pgf@circ@res@up}}
            \pgfpathlineto{\pgfpoint{-\pgf@circ@res@left/3}{-\pgf@circ@res@up}}
            \pgfpathlineto{\pgfpoint{\pgf@circ@res@left}{-\pgf@circ@res@up}}
            \pgfusepath{draw}
        \endpgfscope
    }
}

\pgfdeclareshape{notcirc}{
    \savedmacro{\ctikzclass}{\edef\ctikzclass{logic ports}}
    \anchor{center}{\pgfpointorigin}
    \savedanchor\northwest{%
        \pgf@circ@notradius
        \pgf@y=\pgf@circ@res@temp
        \pgf@x=-\pgf@y
    }
    \anchor{center}{\pgf@y=0pt \pgf@x=0pt}
    \anchor{right}{\northwest\pgf@y=0pt\pgf@x=-\pgf@x}
    \anchor{east}{\northwest\pgf@y=0pt\pgf@x=-\pgf@x}
    \anchor{e}{\northwest\pgf@y=0pt \pgf@x=-\pgf@x}
    \anchor{left}{\northwest\pgf@y=0pt}
    \anchor{west}{\northwest\pgf@y=0pt}
    \anchor{w}{\northwest \pgf@y=0pt}
    \anchor{south}{\northwest \pgf@x=0pt\pgf@y=-\pgf@y}
    \anchor{s}{\northwest\pgf@x=0pt\pgf@y=-\pgf@y}
    \anchor{north}{\northwest\pgf@x=0pt}
    \anchor{n}{\northwest\pgf@x=0pt}
    \anchor{south west}{\northwest\pgf@y=-\pgf@y}
    \anchor{north east}{\northwest\pgf@x=-\pgf@x}
    \anchor{north west}{\northwest}
    \anchor{south east}{\northwest\pgf@x=-\pgf@x\pgf@y=-\pgf@y}
    \anchorborder{
        \pgf@circ@res@left=\pgf@x
        \pgf@circ@res@up=\pgf@y
        \northwest\pgf@circ@res@temp=\pgf@y
        \pgfpointborderellipse{\pgfpoint{\pgf@circ@res@left}{\pgf@circ@res@up}}%
        {\pgfpoint{\pgf@circ@res@temp}{\pgf@circ@res@temp}}
    }
    \behindforegroundpath{
        \pgfscope
            \northwest\pgf@circ@res@temp=\pgf@y
            \pgfsetcolor{\ctikzvalof{color}}
            \pgf@circ@setlinewidth{tripoles}{\pgflinewidth}
            \pgfpathcircle{\pgfpointorigin}{\pgf@circ@res@temp}
            \ifx\tikz@fillcolor\pgfutil@empty
                % set the default fill color to white
                \pgfsetfillcolor{white}
                % ...but override it if the class is defined!
                \pgf@circ@setifdefinedfill{draw, fill}{draw, fill}
            \else
                \pgfsetfillcolor{\tikz@fillcolor}
            \fi
            \pgfsetfillopacity{\ctikzvalof{ieeestd ports/not radius fill}}% normally 1.0
            \pgfusepath{draw,fill}
        \endpgfscope
    }
}


%%%---------- close: tex/pgfcircmultipoles

%%%%%%%%%%% Springe nach tex/pgfcirclabel
%%%---------- open: tex/pgfcirclabel.tex
% Copyright 2018-2020 by Romano Giannetti
% Copyright 2015-2020 by Stefan Lindner
% Copyright 2013-2020 by Stefan Erhardt
% Copyright 2007-2020 by Massimo Redaelli
%
% This file may be distributed and/or modified
%
% 1. under the LaTeX Project Public License and/or
% 2. under the GNU Public License.
%
% See the files gpl-3.0_license.txt and lppl-1-3c_license.txt for more details.

%%%%%%%%%%%%%%%%%%%%%%%%%%%%%%%%%%%%%%%%%%%
%% Bipole label positioning

%% bipole labels and annotation extra style

\ctikzset{bipole label style/.style={}}
\tikzset{bipole label style/.code={
        \ctikzset{bipole label style/.style={#1}}
}}
\tikzset{bipole label append style/.code={
        \ctikzset{bipole label style/.append style={#1}}
}}
\ctikzset{bipole annotation style/.style={}}
\tikzset{bipole annotation style/.code={
        \ctikzset{bipole annotation style/.style={#1}}
}}
\tikzset{bipole annotation append style/.code={
        \ctikzset{bipole annotation style/.append style={#1}}
}}

%% Options
\ctikzset{label/.style = { l=#1 } }
\ctikzset{l/.code = {
        \pgfkeys{/tikz/circuitikz/bipole/label/name=#1}
        \ctikzsetvalof{bipole/label/unit}{}
        \ifpgf@circ@siunitx
            \pgf@circ@handleSI{#1}
            \ifpgf@circ@siunitx@res
                \edef\pgf@temp{\pgf@circ@handleSI@val}
                \pgfkeyslet{/tikz/circuitikz/bipole/label/name}{\pgf@temp}
                \edef\pgf@temp{\pgf@circ@handleSI@unit}
                \pgfkeyslet{/tikz/circuitikz/bipole/label/unit}{\pgf@temp}
            \else
        \fi
        \else
    \fi
}}

\ctikzset{label above/.code = {
        l=#1,
    \circuitikzbasekey/bipole/label/position=90 }
}

\ctikzset{l^/.style = {
        l=#1,
    \circuitikzbasekey/bipole/label/position=90 }
}

\ctikzset{label below/.code = {
        l=#1,
    \circuitikzbasekey/bipole/label/position=-90 }
}

\ctikzset{l_/.style = {
        l=#1,
    \circuitikzbasekey/bipole/label/position=-90 }
}

\ctikzset{annotation/.style = { a=#1 } }
\ctikzset{a/.code = {
    \pgfkeys{/tikz/circuitikz/bipole/annotation/name=#1}
    \ctikzsetvalof{bipole/annotation/unit}{}
    \ifpgf@circ@siunitx
        \pgf@circ@handleSI{#1}
        \ifpgf@circ@siunitx@res
            \edef\pgf@temp{\pgf@circ@handleSI@val}
            \pgfkeyslet{/tikz/circuitikz/bipole/annotation/name}{\pgf@temp}
            \edef\pgf@temp{\pgf@circ@handleSI@unit}
            \pgfkeyslet{/tikz/circuitikz/bipole/annotation/unit}{\pgf@temp}
        \else
        \fi
    \else
    \fi
}}

\ctikzset{annotation above/.code = {
        a=#1,
    \circuitikzbasekey/bipole/annotation/position=90 }
}

\ctikzset{a^/.style = {
        a=#1,
    \circuitikzbasekey/bipole/annotation/position=90 }
}

\ctikzset{annotation below/.code = {
        a=#1,
    \circuitikzbasekey/bipole/annotation/position=-90 }
}
\ctikzset{a_/.style = {
        a=#1,
    \circuitikzbasekey/bipole/annotation/position=-90 }
}

% This is to adjust spacing for the labels so that they are not cramped on components
\def\pgf@circ@ls{.75ex} % labelspace to have just one point to change

\def\pgf@circ@drawlabels#1{
    \pgfextra{
        % This function will be called with argument #1 equal
        % to "label" or "annotation" form pgfcircpath.tex.
        % pgf@circ@direction is the direction of the path,
        % its value is set in pgfcircpath.tex
        \pgfmathsubtract{\pgf@circ@direction}{90}
        \pgfmathround{\pgfmathresult} % avoid precision loss errors
        \edef\pgf@circ@labanc{\expandafter\pgf@circ@stripdecimals\pgfmathresult\pgf@nil}

        \edef\pgf@circ@temp{\ctikzvalof{bipole/#1/position}}
        \ifnum \pgf@circ@temp < 0
                \pgfmathadd{\pgf@circ@labanc}{180}
                \edef\pgf@circ@labanc{\expandafter\pgf@circ@stripdecimals\pgfmathresult\pgf@nil}
        \fi
        % \typeout{INI: TEMP\space\pgf@circ@temp\space LABANC\space\pgf@circ@labanc}
        %
        % normalize the angle values
        %
        \pgfmathmod{\pgf@circ@labanc}{360}
        \edef\pgf@circ@labanc{\expandafter\pgf@circ@stripdecimals\pgfmathresult\pgf@nil}
        %
        \ifpgf@circuit@bipole@inverted
                \pgfmathadd{\pgf@circ@temp}{180} %If shape is inverted, use opposite anchor
                \edef\pgf@circ@temp{\expandafter\pgf@circ@stripdecimals\pgfmathresult\pgf@nil}
        \fi
        %
        \ifnum \ctikzvalof{mirror value} = -1
                \pgfmathadd{\pgf@circ@temp}{180} %If shape is mirrored, use opposite anchor
                \edef\pgf@circ@temp{\expandafter\pgf@circ@stripdecimals\pgfmathresult\pgf@nil}
        \fi
        % \typeout{FIN: TEMP\space\pgf@circ@temp\space LABANC\space\pgf@circ@labanc}
    }
    %Firstly, place a coordinate directly at the edge of the shape
    (\ctikzvalof{bipole/name}.\pgf@circ@temp) coordinate (labelcoor)
    %now decide, which labels should be drawn
    \pgfextra{
            \edef\pgf@temp{\ctikzvalof{label/align}}
            \def\pgf@circ@temp{straight}
    }
    \ifx\pgf@temp\pgf@circ@temp %straight
            \pgf@circ@drawreglabels{#1}
    \else
            \pgfextra{\def\pgf@circ@temp{rotate}}
            \ifx\pgf@temp\pgf@circ@temp %rotate
                    \pgf@circ@drawrotlabels{#1}
            \else% smart
                    \pgf@circ@drawsmartlabels{#1}
            \fi
    \fi
}


\def\pgf@circ@drawsmartlabels#1{
    \pgfextra{
        \pgfmathmod{\pgf@circ@direction}{90}
        \edef\pgfcircmathresult{\expandafter\pgf@circ@stripdecimals\pgfmathresult\pgf@nil}
    }
    \ifnum \pgfcircmathresult > 20
    \ifnum \pgfcircmathresult < 70
        \pgf@circ@drawrotlabels{#1}
    \else
        \pgf@circ@drawreglabels{#1}
    \fi
    \else
        \pgf@circ@drawreglabels{#1}
    \fi
    }


\def\pgf@circ@drawrotlabels#1{
    \pgfextra{
        % scale the distances in function of zoom, so that they are not
        % dependent on it but on font size. Thanks to @marmot
        % https://tex.stackexchange.com/a/476018/38080
        % the coeffcient is adjusted so that the distance is more or less
        % the same for rotated labels and straight ones (although it will
        % depend on the font, so it's not exact).
        \pgfgettransformentries{\tmpa}{\tmpb}{\tmpc}{\tmpd}{\tmp}{\tmp}%
        \pgfmathsetmacro{\myscale}{sqrt(abs(\tmpa*\tmpd-\tmpb*\tmpc))}% abs should not be needed
        % \typeout{ROT\tmpa\space\tmpb\space\tmpc\space\tmpd\space\myscale}
        \pgfmathsetlength\pgf@circ@res@temp{1.5*\pgf@circ@ls/\myscale}
        \ifnum \ctikzvalof{bipole/#1/position}>0
        %we need some more space for placement below, due to mid-anchor
            \else % we do not have <= in \ifnum...
                \pgf@circ@res@temp=1.5\pgf@circ@res@temp
        \fi
        %Calculate rotation of the label from direction and strip decimals
        \edef\pgfcirclabrot{\expandafter\pgf@circ@stripdecimals\pgf@circ@direction\pgf@nil}
        % rotate the label at second or third quadrant:
        \ifnum \pgfcirclabrot > 90 \ifnum \pgfcirclabrot < 270
                \pgfmathsubtract{\pgf@circ@direction}{180}
                \edef\pgfcirclabrot{\expandafter\pgf@circ@stripdecimals\pgfmathresult\pgf@nil}
                %invert the space relationships due to rotated strings
                \ifnum \ctikzvalof{bipole/#1/position}>0
                        \pgf@circ@res@temp=1.5\pgf@circ@res@temp
                \fi
        \fi\fi
        \pgfmathparse{\ctikzvalof{bipole/#1/position}>0?\pgf@circ@direction+90:\pgf@circ@direction-90}
        \edef\pgf@circ@labposangle{\expandafter\pgf@circ@stripdecimals\pgfmathresult\pgf@nil}
    }
    % reset cm is not working correctly here
    (labelcoor)++(\pgf@circ@labposangle:\the\pgf@circ@res@temp) coordinate(labelcoor)
    node[anchor=mid, rotate=\pgfcirclabrot, \circuitikzbasekey/bipole #1 style]
    (\ctikzvalof{bipole/name}#1){\pgf@circ@finallabels{#1}}
}

\def\pgf@circ@drawreglabels#1{
    %Now calculate all shape positions
    %Use mid-anchor at x-axis and base-anchor at y-axis, respectively.
    %All points between will be addressed by angled-anchors:
    \pgfextra{
        % scale ex-distance to make it independent on scale
        % thanks @marmot see https://tex.stackexchange.com/a/476018/38080
        \pgfgettransformentries{\tmpa}{\tmpb}{\tmpc}{\tmpd}{\tmp}{\tmp}%
        \pgfmathsetmacro{\myscale}{sqrt(abs(\tmpa*\tmpd-\tmpb*\tmpc))}% abs should not be needed
        % \typeout{ROT\tmpa\space\tmpb\space\tmpc\space\tmpd\space\myscale}
        \pgfmathsetlength\pgf@circ@res@temp{\pgf@circ@ls/\myscale}
        \pgfmathadd{\pgf@circ@labanc}{90}
        \pgfmathround{\pgfmathresult}
        \def\pgf@circ@labanctext{\pgf@circ@labanc}
        \edef\pgf@circ@temp{\expandafter\pgf@circ@stripdecimals\pgfmathresult\pgf@nil}
        \pgfmathparse{mod(\pgf@circ@temp,180)>135?mod(\pgf@circ@temp,180)-180:mod(\pgf@circ@temp,180)}
        \edef\pgfcircmathresult{\expandafter\pgf@circ@stripdecimals\pgfmathresult\pgf@nil}
    }
    %Values around 90 are at both y-axis
    \ifnum \pgfcircmathresult > 84 \ifnum \pgfcircmathresult< 96
        \pgfextra{\edef\pgf@circ@labpos{\expandafter\pgf@circ@stripdecimals\pgf@circ@direction\pgf@nil}}
        \ifnum \pgf@circ@labpos > 180
            \ifnum \ctikzvalof{bipole/#1/position} > 0
                    \pgfextra{\def\pgf@circ@labanctext{mid west}}
            \else
                    \pgfextra{\def\pgf@circ@labanctext{mid east}}
            \fi
        \else
            \ifnum \ctikzvalof{bipole/#1/position} > 0
                    \pgfextra{\def\pgf@circ@labanctext{mid east}}
            \else
                    \pgfextra{\def\pgf@circ@labanctext{mid west}}
            \fi
        \fi
    \fi\fi
    %Values between -5 and 5 are at pos /neg x-axis
    \pgfextra{\def\uffa{}\newdimen\realshift\realshift=\dimexpr1pt\relax}
        \ifnum \pgfcircmathresult <6 \ifnum \pgfcircmathresult > -6
            \ifnum \ctikzvalof{bipole/#1/position} < 0
                \ifnum \pgf@circ@labanc > 90
                    % using base coordinate instead of south to naturally align
                    % symbols with descendants; but this invalidate the effect of
                    % the inner sep, so recover it by shifting the anchor
                    % reset cm is not working sometime, use @marmot solution
                    % see https://tex.stackexchange.com/a/476018/38080
                    (labelcoor) ++(-\pgf@circ@labanc:\pgf@circ@res@temp) coordinate(labelcoor)
                    \pgfextra{\def\pgf@circ@labanctext{base}}%base
                \else
                    \pgfextra{\def\pgf@circ@labanctext{north}}%north
                \fi
             \else
                \ifnum \pgf@circ@labanc < 90
                    % shift, as above
                    (labelcoor) ++(-\pgf@circ@labanc:\pgf@circ@res@temp) coordinate(labelcoor)
                    \pgfextra{\def\pgf@circ@labanctext{base}}%base
                \else
                    \ifnum \pgf@circ@labanc > 180
                        % this shouldn't  happen, but somehow it does (270 degree anchors)
                        % shift, as above
                        (labelcoor) ++(-\pgf@circ@labanc:\pgf@circ@res@temp) coordinate(labelcoor)
                         \pgfextra{\def\pgf@circ@labanctext{base}}%base
                    \else
                      \pgfextra{\def\pgf@circ@labanctext{north}}%north
                   \fi
                \fi
            \fi
        \fi\fi
    (labelcoor) node[anchor=\pgf@circ@labanctext,
    inner sep=0.5\pgf@circ@res@temp, outer sep=0pt, \circuitikzbasekey/bipole #1 style,
        ](\ctikzvalof{bipole/name}#1){\strut\pgf@circ@finallabels{#1}%
    }
}

\def\pgf@circ@finallabels#1{%
    \edef\pgf@temp{}%
    \edef\pgf@circ@temp{\ctikzvalof{bipole/#1/unit}}%
    \ifx\pgf@temp\pgf@circ@temp%
        \ctikzvalof{bipole/#1/name}%
    \else%
        $\SI{\ctikzvalof{bipole/#1/name}}{\ctikzvalof{bipole/#1/unit}}$%
    \fi%
}

\ctikzset{text/.style={t=#1}}
\ctikzset{t/.code={%
        \ctikzsetvalof{bipoles/twoport/text}{#1}%
}}

%%%% Stacked labels
%
% stacked labels by Romano Giannetti romano@rgtti.com
% heavily based on Claudo Fiandrinos's https://tex.stackexchange.com/a/65792/38080
% \expandafter trick inspired by Matthew Leingang's https://tex.stackexchange.com/a/12272/38080
%
% labels are in a tabular, globally aligned:
%        vertically with key l2 valign (default c)
%        c: center t: top b: bottom
%        horizontally with key l2 align (default l)
%        l: left c: centered r: right
% you can switch sides using l2_=... and l2^=...
% syntax is l2_ = line1 and line2 (same for l2^)
%
\ctikzset{%
    l2 valign/.store in=\ltwo@valign, l2 valign=c,
    l2 halign/.store in=\ltwo@halign, l2 halign=l,
}
\ctikzset{l2base/.code n args={2}{
  \pgfkeys{/tikz/circuitikz/bipole/label/name=%
        \bgroup
        \setlength{\tabcolsep}{2pt}%
        \def\ltwo@tabu{\tabular[\ltwo@valign]}%
        \expandafter\ltwo@tabu\expandafter{\ltwo@halign}%
        #1\\ #2%
        \endtabular
        \egroup
    }%
    \ctikzsetvalof{bipole/label/unit}{}
    \ifpgf@circ@siunitx
        \pgf@circ@handleSI{#2}
        \ifpgf@circ@siunitx@res
            \edef\pgf@temp{\pgf@circ@handleSI@val}
            \pgfkeyslet{/tikz/circuitikz/bipole/label/name}{\pgf@temp}
            \edef\pgf@temp{\pgf@circ@handleSI@unit}
            \pgfkeyslet{/tikz/circuitikz/bipole/label/unit}{\pgf@temp}
        \else
        \fi
    \else
    \fi
}}
\ctikzset{l2/.style args={#1 and #2}{
        l2base={#1}{#2},
    \circuitikzbasekey/bipole/label/position=90 }
}
\ctikzset{l2 above/.style args={#1 and #2}{
        l2base={#1}{#2},
    \circuitikzbasekey/bipole/label/position=90 }
}
\ctikzset{l2^/.style args={#1 and #2}{
        l2base={#1}{#2},
    \circuitikzbasekey/bipole/label/position=90 }
}
\ctikzset{l2 below/.style args={#1 and #2}{
        l2base={#1}{#2},
    \circuitikzbasekey/bipole/label/position=-90 }
}
\ctikzset{l2_/.style args={#1 and #2}{
        l2base={#1}{#2},
    \circuitikzbasekey/bipole/label/position=-90 }
}

%%%---------- close: tex/pgfcirclabel
%%%%%%%%%%% Springe nach tex/pgfcircvoltage
%%%---------- open: tex/pgfcircvoltage.tex
% Copyright 2018-2020 by Romano Giannetti
% Copyright 2015-2020 by Stefan Lindner
% Copyright 2013-2020 by Stefan Erhardt
% Copyright 2007-2020 by Massimo Redaelli
%
% This file may be distributed and/or modified
%
% 1. under the LaTeX Project Public License and/or
% 2. under the GNU Public License.
%
% See the files gpl-3.0_license.txt and lppl-1-3c_license.txt for more details.

%%%%%%%%%%%%%%%%%%%%%%%%%%%%%%%%%%%%%%%%%%
%%  Voltage management

%% styles
\ctikzset{bipole voltage style/.style={}}
\tikzset{bipole voltage style/.code={
        \ctikzset{bipole voltage style/.style={#1}}
}}
\tikzset{bipole voltage append style/.code={
        \ctikzset{bipole voltage style/.append style={#1}}
}}

\ctikzset{v^>/.style = {
        v = #1,
        \circuitikzbasekey/bipole/voltage/direction = forward,
        \circuitikzbasekey/bipole/voltage/position = above
    }
}

\ctikzset{v^</.style = {
        v = #1,
        \circuitikzbasekey/bipole/voltage/direction = backward,
        \circuitikzbasekey/bipole/voltage/position = above
    }
}

\ctikzset{v_>/.style = {
        v = #1,
        \circuitikzbasekey/bipole/voltage/direction = forward,
        \circuitikzbasekey/bipole/voltage/position = below
    }
}

\ctikzset{v_</.style = {
        v = #1,
        \circuitikzbasekey/bipole/voltage/direction = backward,
        \circuitikzbasekey/bipole/voltage/position = below
    }
}

\ctikzset{v_/.style = {v = #1, \circuitikzbasekey/bipole/voltage/position = below} }
\ctikzset{v^/.style = {v = #1, \circuitikzbasekey/bipole/voltage/position  = above} }
\ctikzset{v>/.style = {v = #1, \circuitikzbasekey/bipole/voltage/direction = forward} }
\ctikzset{v</.style = {v = #1, \circuitikzbasekey/bipole/voltage/direction = backward} }

% Default position varies whether the component is a voltage source
% or not
\ctikzset{v/.code = {
        \ifpgf@circuit@bipole@isvoltage
            \pgfkeys{\circuitikzbasekey/bipole/voltage/position=above,
            \circuitikzbasekey/bipole/voltage/direction=forward}
        \else
            \ifpgf@circ@oldvoltagedirection
                \pgfkeys{\circuitikzbasekey/bipole/voltage/position=below,
                \circuitikzbasekey/bipole/voltage/direction=backward}
            \else
                \pgfkeys{\circuitikzbasekey/bipole/voltage/position=below,
                \circuitikzbasekey/bipole/voltage/direction=forward}
            \fi
        \fi
        \ifpgf@circ@oldvoltagedirection
            \ifpgf@circuit@bipole@iscurrent\ifpgf@circ@fixbatteries
                \pgfkeys{\circuitikzbasekey/bipole/voltage/position=below,
                \circuitikzbasekey/bipole/voltage/direction=forward}
        \fi\fi
        \else
        \ifpgf@circuit@bipole@iscurrent
            \ifpgf@circuit@bipole@current@backward
                \pgfkeys{\circuitikzbasekey/bipole/voltage/position=below,
                \circuitikzbasekey/bipole/voltage/direction=forward}
            \else
                \pgfkeys{\circuitikzbasekey/bipole/voltage/position=below,
                \circuitikzbasekey/bipole/voltage/direction=backward}
            \fi\fi\fi
            \pgfkeys{/tikz/circuitikz/bipole/voltage/label/name=#1}
            \ctikzsetvalof{bipole/voltage/label/unit}{}
            \ifpgf@circ@siunitx
                \pgf@circ@handleSI{#1}
                \ifpgf@circ@siunitx@res
                    \edef\pgf@temp{\pgf@circ@handleSI@val}
                    \pgfkeyslet{/tikz/circuitikz/bipole/voltage/label/name}{\pgf@temp}
                    \edef\pgf@temp{\pgf@circ@handleSI@unit}
                    \pgfkeyslet{/tikz/circuitikz/bipole/voltage/label/unit}{\pgf@temp}
                \else
            \fi
            \else
        \fi
    }
}

% american voltage font selection and symbol definition
% the default font command is {} --- nothing
\def\pgf@circ@avfont{\ctikzvalof{voltage/american font}}
%
% plus and minus symbols (default is $+$ and $-$, see pgfcirc.defines.tex)
%
\def\pgf@circ@avplus{\ctikzvalof{voltage/american plus}}
\def\pgf@circ@avminus{\ctikzvalof{voltage/american minus}}

%%
\def\setscaledRlenforclass{%
    \csname pgf@sh@ma@\ctikzvalof{bipole/name}\endcsname
    \ifdefined\ctikzclass
        \edef\pgf@temp{/tikz/circuitikz/\ctikzclass/scale}
        \pgfkeysifdefined{\pgf@temp}
            {\pgf@circ@scaled@Rlen=\ctikzvalof{\ctikzclass/scale}\pgf@circ@Rlen}
            {\pgf@circ@scaled@Rlen=\pgf@circ@Rlen}
    \else
        \pgf@circ@scaled@Rlen=\pgf@circ@Rlen
    \fi
}

%% Output routine for generic bipoles

\def\pgf@circ@drawvoltagegeneric{
    \pgfextra{
        % \typeout{NAME:\ctikzvalof{bipole/name}}
        \edef\pgf@temp{/tikz/circuitikz/bipoles/\ctikzvalof{bipole/kind}/voltage/straight label distance}
        \pgfkeysifdefined{\pgf@temp}
        {
            \edef\partheight{\ctikzvalof{bipoles/\ctikzvalof{bipole/kind}/voltage/straight label distance}}
            \edef\tmpdistfromline{(\partheight\pgf@circ@scaled@Rlen)}
        }
        {
            \pgfkeysifdefined{/tikz/circuitikz/bipoles/voltage/straight label distance}
            {
                \edef\partheight{\ctikzvalof{bipoles/voltage/straight label distance}}
                \edef\tmpdistfromline{(\partheight\pgf@circ@scaled@Rlen)}
            }
            {%calculate default value from part height
                \edef\pgf@temp{/tikz/circuitikz/bipoles/\ctikzvalof{bipole/kind}/height}
                \pgfkeysifdefined{\pgf@temp}
                {
                    \edef\partheight{0.5*\ctikzvalof{bipoles/\ctikzvalof{bipole/kind}/height}}
                    \edef\tmpdistfromline{(\partheight\pgf@circ@scaled@Rlen+0.2\pgf@circ@scaled@Rlen)}
                }
                {
                    \edef\tmpdistfromline{(.5\pgf@circ@scaled@Rlen)} %fallback to fixed value
                }
            }
        }
        % \typeout{KIND:\ctikzvalof{bipole/kind}\space RLEN:\the\pgf@circ@Rlen\space SCALED:\the\pgf@circ@scaled@Rlen}
        \ifnum \ctikzvalof{mirror value}=-1
        \ifpgf@circuit@bipole@inverted
            \ifpgf@circuit@bipole@voltage@straight
                \def\distfromline{\tmpdistfromline}
            \else
                \def\distfromline{\ctikzvalof{voltage/distance from line}\pgf@circ@scaled@Rlen}
            \fi
            \else
            \ifpgf@circuit@bipole@voltage@straight
                \def\distfromline{-\tmpdistfromline}
            \else
                \def\distfromline{-\ctikzvalof{voltage/distance from line}\pgf@circ@scaled@Rlen}
            \fi
        \fi
        \else
            \ifpgf@circuit@bipole@inverted
                \ifpgf@circuit@bipole@voltage@straight
                    \def\distfromline{-\tmpdistfromline}
                \else
                    \def\distfromline{-\ctikzvalof{voltage/distance from line}\pgf@circ@scaled@Rlen}
                \fi
                \else
                \ifpgf@circuit@bipole@voltage@straight
                    \def\distfromline{\tmpdistfromline}
                \else
                    \def\distfromline{\ctikzvalof{voltage/distance from line}\pgf@circ@scaled@Rlen}
                \fi
            \fi
        \fi
        \ifpgf@circuit@bipole@voltage@below
            \def\pgf@circ@voltage@angle{90}
        \else
            \def\pgf@circ@voltage@angle{-90}
        \fi
        \edef\pgf@temp{/tikz/circuitikz/bipoles/\ctikzvalof{bipole/kind}/voltage/distance from node}
        \pgfkeysifdefined{\pgf@temp}
            { \edef\distancefromnode{\ctikzvalof{bipoles/\ctikzvalof{bipole/kind}/voltage/distance from node}} }
            { \edef\distancefromnode{\ctikzvalof{voltage/distance from node}} }
        \edef\pgf@temp{/tikz/circuitikz/bipoles/\ctikzvalof{bipole/kind}/voltage/bump b}
        \pgfkeysifdefined{\pgf@temp}
            { \edef\bumpb{\ctikzvalof{bipoles/\ctikzvalof{bipole/kind}/voltage/bump b}} }
            { \edef\bumpb{\ctikzvalof{voltage/bump b}} }
        \edef\shiftv{\ctikzvalof{voltage/shift}}
        % additional per-bipole voltage shift (internal)
        \edef\pgf@temp{/tikz/circuitikz/bipoles/\ctikzvalof{bipole/kind}/voltage/additional label shift}
        \pgfkeysifdefined{\pgf@temp}
        {
            \edef\addvshift{\ctikzvalof{bipoles/\ctikzvalof{bipole/kind}/voltage/additional label shift}}
        }
        {
            \edef\addvshift{0}
        }
        \newdimen{\absvshift}
        \pgfmathsetlength{\absvshift}{(1+\shiftv+\addvshift)*(\distfromline)}
        % put this to true to see the voltage label coordinate anchors
        \newif\ifpgf@circ@debugv\pgf@circ@debugvfalse
    }
    % %\pgf@circ@Rlen/\ctikzvalof{current arrow scale} is equal to the length of the currarrow
    coordinate (pgfcirc@midtmp) at ($(\tikztostart) ! \pgf@circ@Rlen/\ctikzvalof{current arrow scale} ! (anchorstartnode)$) %absolute move, minimum space is length of arrowhead
    coordinate (pgfcirc@midtmp) at ($(pgfcirc@midtmp) ! \distancefromnode ! (anchorstartnode)$)
    coordinate (pgfcirc@Vfrom@flat) at (pgfcirc@midtmp)
    coordinate (pgfcirc@Vfrom) at ($(pgfcirc@midtmp) ! -\distfromline ! \pgf@circ@voltage@angle:(anchorstartnode)$)

    coordinate (pgfcirc@midtmp) at ($(\tikztotarget) ! \pgf@circ@Rlen/\ctikzvalof{current arrow scale} ! (anchorendnode)$)%absolute move, minimum space is length of arrowhead
    coordinate (pgfcirc@midtmp) at ($(pgfcirc@midtmp) ! \distancefromnode ! (anchorendnode)$)
    coordinate (pgfcirc@Vto@flat) at (pgfcirc@midtmp)
    coordinate (pgfcirc@Vto) at ($(pgfcirc@midtmp) ! \distfromline ! \pgf@circ@voltage@angle : (anchorendnode)$)

    \ifpgf@circuit@bipole@voltage@below
        \ifpgf@circ@debugv
            node [ocirc, fill=red] at (anchorstartnode) {}
            node [ocirc, fill=blue] at (anchorendnode) {}
            node [ocirc, fill=green] at (pgfcirc@Vto) {}
            node [ocirc, fill=yellow] at (pgfcirc@Vfrom) {}
            node [odiamondpole, fill=green!50!black] at (pgfcirc@Vto@flat) {}
            node [odiamondpole, fill=orange] at (pgfcirc@Vfrom@flat) {}
        \fi
        coordinate (pgfcirc@Vto) at ($(pgfcirc@Vto@flat) ! \absvshift!90 :  (anchorendnode)$)
        coordinate (pgfcirc@Vfrom) at ($(pgfcirc@Vfrom@flat) ! \absvshift!-90 :  (anchorstartnode)$)
        coordinate (pgfcirc@Vcont1t) at ($(\ctikzvalof{bipole/name}.center) ! \bumpb ! (\ctikzvalof{bipole/name}.-110)$)
        coordinate (pgfcirc@Vcont2t) at ($(\ctikzvalof{bipole/name}.center) ! \bumpb ! (\ctikzvalof{bipole/name}.-70)$)
        coordinate (pgfcirc@Vcont1) at ($(pgfcirc@Vcont1t) ! -\absvshift!90 : (pgfcirc@Vcont2t)$)
        coordinate (pgfcirc@Vcont2) at ($(pgfcirc@Vcont2t) ! -\absvshift!-90 : (pgfcirc@Vcont1t)$)
        \ifpgf@circ@debugv
            node [odiamondpole, fill=green] at (pgfcirc@Vto) {}
            node [odiamondpole, fill=yellow] at (pgfcirc@Vfrom) {}
            node [osquarepole, fill=red] at (pgfcirc@Vcont1) {}
            node [osquarepole, fill=blue] at (pgfcirc@Vcont2) {}
        \fi
    \else
        \ifpgf@circ@debugv
            node [ocirc, fill=red] at (anchorstartnode) {}
            node [ocirc, fill=blue] at (anchorendnode) {}
            node [ocirc, fill=green] at (pgfcirc@Vto) {}
            node [ocirc, fill=yellow] at (pgfcirc@Vfrom) {}
            node [odiamondpole, fill=green] at (pgfcirc@Vto@flat) {}
            node [odiamondpole, fill=yellow] at (pgfcirc@Vfrom@flat) {}
        \fi
        coordinate (pgfcirc@Vto) at ($(pgfcirc@Vto@flat) ! -\absvshift!90 :  (anchorendnode)$)
        coordinate (pgfcirc@Vfrom) at ($(pgfcirc@Vfrom@flat) ! -\absvshift!-90 :  (anchorstartnode)$)
        coordinate (pgfcirc@Vcont1t) at ($(\ctikzvalof{bipole/name}.center) ! \bumpb ! (\ctikzvalof{bipole/name}.110)$)
        coordinate (pgfcirc@Vcont2t) at ($(\ctikzvalof{bipole/name}.center) ! \bumpb ! (\ctikzvalof{bipole/name}.70)$)
        coordinate (pgfcirc@Vcont1) at ($(pgfcirc@Vcont1t) ! \absvshift!90 : (pgfcirc@Vcont2t)$)
        coordinate (pgfcirc@Vcont2) at ($(pgfcirc@Vcont2t) ! \absvshift!-90 : (pgfcirc@Vcont1t)$)
        \ifpgf@circ@debugv
            node [odiamondpole, fill=green] at (pgfcirc@Vto) {}
            node [odiamondpole, fill=yellow] at (pgfcirc@Vfrom) {}
            node [osquarepole, fill=red] at (pgfcirc@Vcont1) {}
            node [osquarepole, fill=blue] at (pgfcirc@Vcont2) {}
        \fi
    \fi

    \ifpgf@circuit@europeanvoltage
        \ifpgf@circuit@bipole@voltage@straight
            \ifpgf@circuit@bipole@voltage@backward
                (pgfcirc@Vto) --(pgfcirc@Vfrom) node[currarrow, sloped,  allow upside down, pos=1,anchor=tip] {}
            \else
                (pgfcirc@Vfrom) --(pgfcirc@Vto) node[currarrow, sloped,  allow upside down, pos=1,anchor=tip] {}
            \fi
        \else
            \ifpgf@circuit@bipole@voltage@backward
                (pgfcirc@Vto) .. controls (pgfcirc@Vcont2)  and (pgfcirc@Vcont1) ..
                node[currarrow, sloped,  allow upside down, pos=1, anchor=tip] {}
                (pgfcirc@Vfrom)
            \else
                (pgfcirc@Vfrom) .. controls (pgfcirc@Vcont1)  and (pgfcirc@Vcont2) ..
                node[currarrow, sloped,  allow upside down, pos=1, anchor=tip] {}
                (pgfcirc@Vto)
            \fi
        \fi
    \else % american
        \ifx\@@kind\@@open % open circuit; put + and - directly on it
            \ifpgf@circuit@bipole@voltage@backward
                \ifpgf@circ@oldvoltagedirection
                    (pgfcirc@Vfrom@flat) node[inner sep=0, node font=\pgf@circ@avfont,
                        anchor=center]{\pgf@circ@avplus}
                    (pgfcirc@Vto@flat) node[inner sep=0, node font=\pgf@circ@avfont,
                        anchor=center]{\pgf@circ@avminus}
                \else
                    (pgfcirc@Vfrom@flat) node[inner sep=0, node font=\pgf@circ@avfont,
                        anchor=center]{\pgf@circ@avminus}
                    (pgfcirc@Vto@flat) node[inner sep=0, node font=\pgf@circ@avfont,
                        anchor=center]{\pgf@circ@avplus}
                \fi
                \else
                \ifpgf@circ@oldvoltagedirection
                    (pgfcirc@Vfrom@flat) node[inner sep=0, node font=\pgf@circ@avfont,
                        anchor=center]{\pgf@circ@avminus}
                    (pgfcirc@Vto@flat) node[inner sep=0, node font=\pgf@circ@avfont,
                        anchor=center]{\pgf@circ@avplus}
                \else
                    (pgfcirc@Vfrom@flat) node[inner sep=0, node font=\pgf@circ@avfont,
                        anchor=center]{\pgf@circ@avplus}
                    (pgfcirc@Vto@flat) node[inner sep=0, node font=\pgf@circ@avfont,
                        anchor=center]{\pgf@circ@avminus}
                \fi
            \fi
        \else % not open circuit
            \ifpgf@circuit@bipole@voltage@backward
                \ifpgf@circ@oldvoltagedirection
                    (pgfcirc@Vfrom) node[inner sep=0, node font=\pgf@circ@avfont,
                        anchor=\pgf@circ@bipole@voltage@label@anchor]{\pgf@circ@avplus}
                    (pgfcirc@Vto) node[inner sep=0, node font=\pgf@circ@avfont,
                        anchor=\pgf@circ@bipole@voltage@label@anchor]{\pgf@circ@avminus}
                \else
                    (pgfcirc@Vfrom) node[inner sep=0, node font=\pgf@circ@avfont,
                        anchor=\pgf@circ@bipole@voltage@label@anchor]{\pgf@circ@avminus}
                    (pgfcirc@Vto) node[inner sep=0, node font=\pgf@circ@avfont,
                        anchor=\pgf@circ@bipole@voltage@label@anchor]{\pgf@circ@avplus}
                \fi
                \else
                \ifpgf@circ@oldvoltagedirection
                    (pgfcirc@Vfrom) node[inner sep=0, node font=\pgf@circ@avfont,
                        anchor=\pgf@circ@bipole@voltage@label@anchor]{\pgf@circ@avminus}
                    (pgfcirc@Vto) node[inner sep=0, node font=\pgf@circ@avfont,
                        anchor=\pgf@circ@bipole@voltage@label@anchor]{\pgf@circ@avplus}
                \else
                    (pgfcirc@Vfrom) node[inner sep=0, node font=\pgf@circ@avfont,
                        anchor=\pgf@circ@bipole@voltage@label@anchor]{\pgf@circ@avplus}
                    (pgfcirc@Vto) node[inner sep=0, node font=\pgf@circ@avfont,
                        anchor=\pgf@circ@bipole@voltage@label@anchor]{\pgf@circ@avminus}
                \fi
            \fi
        \fi
    \fi
}

%% Output routine for voltage sources
\def\pgf@circ@drawvoltagegenerator{
    % the following is affected indirectly by voltage/shift, you can move the arrow with voltage/bump a.
    % it's not perfect, but I can't find the way to do it correctly...
    \pgfextra{
        \edef\shiftv{\ctikzvalof{voltage/shift}}
        \edef\bumpa{\ctikzvalof{voltage/bump a}}
        \pgfmathsetmacro{\bumpaplus}{\bumpa + 0.5*\shiftv} % coefficient added "by feel". Sorry.
    }
    \ifpgf@circuit@bipole@voltage@below
        coordinate (pgfcirc@Vfrom) at ($(\ctikzvalof{bipole/name}.center) ! \bumpaplus ! (\ctikzvalof{bipole/name}.-120)$)
        coordinate (pgfcirc@Vto) at ($(\ctikzvalof{bipole/name}.center) ! \bumpaplus ! (\ctikzvalof{bipole/name}.-60)$)
    \else
        coordinate (pgfcirc@Vfrom) at ($ (\ctikzvalof{bipole/name}.center) ! \bumpaplus ! (\ctikzvalof{bipole/name}.120)$)
        coordinate (pgfcirc@Vto) at ($ (\ctikzvalof{bipole/name}.center) ! \bumpaplus ! (\ctikzvalof{bipole/name}.60)$)
    \fi
    % fix the (unused in this case) Vcont1/2 coords for label placement along the line
    coordinate (pgfcirc@Vcont1) at (pgfcirc@Vto)
    coordinate (pgfcirc@Vcont2) at (pgfcirc@Vfrom)
    \ifpgf@circuit@europeanvoltage
        \ifpgf@circuit@bipole@voltage@backward
            (pgfcirc@Vto)  -- node[currarrow, sloped,  allow upside down, pos=1, anchor=tip] {} (pgfcirc@Vfrom)
        \else
            (pgfcirc@Vfrom)  -- node[currarrow, sloped,  allow upside down, pos=1, anchor=tip] {} (pgfcirc@Vto)
        \fi
        \else% american voltage
        \ifpgf@circuit@bipole@voltageoutsideofsymbol
            % if it is a battery, must put + and -

            \ifpgf@circ@fixbatteries
                \ifpgf@circuit@bipole@voltage@backward
                    (pgfcirc@Vfrom)  node[node font=\pgf@circ@avfont] {\pgf@circ@avplus}
                    (pgfcirc@Vto) node[node font=\pgf@circ@avfont] {\pgf@circ@avminus}
                \else
                    (pgfcirc@Vfrom)  node[node font=\pgf@circ@avfont] {\pgf@circ@avminus}
                    (pgfcirc@Vto) node[node font=\pgf@circ@avfont] {\pgf@circ@avplus}
                \fi
                \else
                \ifpgf@circuit@bipole@voltage@backward
                    (pgfcirc@Vfrom)  node[node font=\pgf@circ@avfont] {\pgf@circ@avminus}
                    (pgfcirc@Vto) node[node font=\pgf@circ@avfont] {\pgf@circ@avplus}
                \else
                    (pgfcirc@Vfrom)  node[node font=\pgf@circ@avfont] {\pgf@circ@avplus}
                    (pgfcirc@Vto) node[node font=\pgf@circ@avfont] {\pgf@circ@avminus}
                \fi
            \fi
        \fi
    \fi
}

%% Output routine
%% this is the entry point
%%
\def\pgf@circ@drawvoltage{% node name
    \pgfextra{ %WARNING: indentation is probably wrong
        \edef\pgfcircmathresult{\expandafter\pgf@circ@stripdecimals\pgf@circ@direction\pgf@nil}
        \ifnum\pgfcircmathresult >4 \ifnum\pgfcircmathresult <86
        \ifpgf@circuit@bipole@voltage@below
            \def\pgf@circ@bipole@voltage@label@anchor{north west}
        \else
            \def\pgf@circ@bipole@voltage@label@anchor{south east}
        \fi
        \fi\fi
        \ifnum\pgfcircmathresult >85 \ifnum\pgfcircmathresult <95
        \ifpgf@circuit@bipole@voltage@below
            \def\pgf@circ@bipole@voltage@label@anchor{west}
        \else
            \def\pgf@circ@bipole@voltage@label@anchor{east}
        \fi
        \fi\fi
        \ifnum\pgfcircmathresult >94 \ifnum\pgfcircmathresult <176
        \ifpgf@circuit@bipole@voltage@below
            \def\pgf@circ@bipole@voltage@label@anchor{south west}
        \else \def\pgf@circ@bipole@voltage@label@anchor{north east}
        \fi
        \fi\fi
        \ifnum\pgfcircmathresult >175 \ifnum\pgfcircmathresult <185
        \ifpgf@circuit@bipole@voltage@below
            \def\pgf@circ@bipole@voltage@label@anchor{south}
        \else\def\pgf@circ@bipole@voltage@label@anchor{north}
        \fi
        \fi\fi
        \ifnum\pgfcircmathresult >184 \ifnum\pgfcircmathresult <266
        \ifpgf@circuit@bipole@voltage@below
            \def\pgf@circ@bipole@voltage@label@anchor{south east}
        \else\def\pgf@circ@bipole@voltage@label@anchor{north west}
        \fi
        \fi\fi
        \ifnum\pgfcircmathresult >265 \ifnum\pgfcircmathresult <275
        \ifpgf@circuit@bipole@voltage@below
            \def\pgf@circ@bipole@voltage@label@anchor{east}
        \else \def\pgf@circ@bipole@voltage@label@anchor{west}
        \fi
        \fi\fi
        \ifnum\pgfcircmathresult >274 \ifnum\pgfcircmathresult <356
        \ifpgf@circuit@bipole@voltage@below
            \def\pgf@circ@bipole@voltage@label@anchor{north east}
        \else\def\pgf@circ@bipole@voltage@label@anchor{south west}
        \fi
        \fi\fi
        \ifnum\pgfcircmathresult >-1 \ifnum\pgfcircmathresult <5
        \ifpgf@circuit@bipole@voltage@below
            \def\pgf@circ@bipole@voltage@label@anchor{north}
        \else\def\pgf@circ@bipole@voltage@label@anchor{south}
        \fi
        \fi\fi
        \ifnum\pgfcircmathresult >355 \ifnum\pgfcircmathresult <361
        \ifpgf@circuit@bipole@voltage@below
            \def\pgf@circ@bipole@voltage@label@anchor{north}
        \else\def\pgf@circ@bipole@voltage@label@anchor{south}
        \fi
        \fi\fi

        % this must be set *before* changing for mirroring and inverting; in that case
        % the xscale/yscale parameters take it into account
        \ifpgf@circuit@bipole@voltage@below
            \def\pgf@circ@bipole@voltage@label@where{-90}
        \else
            \def\pgf@circ@bipole@voltage@label@where{90}
        \fi

        % magic to counteract the scale and yscale effects (there should be a better way...)
        \ifnum \ctikzvalof{mirror value}=-1
            \ifpgf@circuit@bipole@voltage@below
                \pgf@circuit@bipole@voltage@belowfalse
            \else
                \pgf@circuit@bipole@voltage@belowtrue
            \fi
        \fi

        \ifpgf@circuit@bipole@inverted
            \ifpgf@circuit@bipole@voltage@below
                \pgf@circuit@bipole@voltage@belowfalse
            \else
                \pgf@circuit@bipole@voltage@belowtrue
            \fi
        \fi

        % take into account scaling
        \setscaledRlenforclass

        \edef\pgf@temp{/tikz/circuitikz/bipoles/\ctikzvalof{bipole/kind}/voltage/european label distance}
        \pgfkeysifdefined{\pgf@temp}
            { \edef\eudist{\ctikzvalof{bipoles/\ctikzvalof{bipole/kind}/voltage/european label distance}} }
            { \edef\eudist{\ctikzvalof{voltage/european label distance}} }
        % find the height of the bipole or use a default value
        \edef\pgf@temp{/tikz/circuitikz/bipoles/\ctikzvalof{bipole/kind}/height}
        \pgfkeysifdefined{\pgf@temp}
            {\pgfmathsetmacro{\partheightf}{0.5*\ctikzvalof{bipoles/\ctikzvalof{bipole/kind}/height}}
             \edef\partheight{\partheightf\pgf@circ@scaled@Rlen}}
            {\edef\partheight{(.5\pgf@circ@scaled@Rlen)}} %fallback to fixed value
        \newdimen{\alshift}
        % this is more or less the same of the legacy code; we shift the american label a bit
        % outside the (+) -- (-) line
        \pgfmathsetlength{\alshift}{(\ctikzvalof{voltage/american label distance}-0.6)*\partheight}
        \pgfsetcornersarced{\pgfpointorigin}% do not use rounded corners!
        % set the macro for detecting open
        \edef\@@kind{\ctikzvalof{bipole/kind}}\edef\@@open{open}
        \ifpgf@adjust@open@voltage\else\edef\@@open{this-will-nEver-match}\fi
        % \typeout{KIND\space\@@kind}
    }%end pgfextra

    \ifpgf@circuit@bipole@isvoltage
        \pgf@circ@drawvoltagegenerator
    \else
        \pgf@circ@drawvoltagegeneric
    \fi

    % \pgfextra{\typeout{LABEL\space KIND:\@@kind\space OPEN:\@@open}}

    \ifpgf@circuit@bipole@voltage@straight
        coordinate (Vlab) at ($(pgfcirc@Vto)!0.5!(pgfcirc@Vfrom) $)
        node [anchor=\pgf@circ@bipole@voltage@label@anchor, inner sep=2pt,
        \circuitikzbasekey/bipole voltage style](\ctikzvalof{bipole/name}voltage)
        at (Vlab) {\pgf@circ@finallabels{voltage/label}}
    \else
        \ifpgf@circuit@europeanvoltage
            coordinate (Vlab) at ($(pgfcirc@Vcont1)!0.5!(pgfcirc@Vcont2)$)
        \else
            coordinate (Vlab) at ($(pgfcirc@Vfrom)!0.5!(pgfcirc@Vto)$)
            \ifpgf@circuit@bipole@isvoltage\else
            % add a bit of space for american labels above their symbols in the normal case. You can avoid that
            % with voltage/american label distance=0.5 (it's measured from the center of the component, in heights)
                coordinate (Vlab) at ($(Vlab) ! \alshift ! \pgf@circ@bipole@voltage@label@where :(pgfcirc@Vto)$)
            \fi
        \fi

        \ifpgf@circuit@europeanvoltage
            node [anchor=\pgf@circ@bipole@voltage@label@anchor, inner sep=2pt,
            \circuitikzbasekey/bipole voltage style](\ctikzvalof{bipole/name}voltage)
            at (Vlab) {\pgf@circ@finallabels{voltage/label}}
        \else % american voltages
            \ifx\@@kind\@@open
                coordinate (Vlab) at ($(pgfcirc@Vfrom@flat)!0.5!(pgfcirc@Vto@flat)$)
                node [anchor=center, inner sep=2pt,
                \circuitikzbasekey/bipole voltage style](\ctikzvalof{bipole/name}voltage)
                at (Vlab) {\pgf@circ@finallabels{voltage/label}}
            \else
                node [anchor=\pgf@circ@bipole@voltage@label@anchor, inner sep=2pt,
                \circuitikzbasekey/bipole voltage style](\ctikzvalof{bipole/name}voltage)
                at (Vlab) {\pgf@circ@finallabels{voltage/label}}
            \fi
        \fi
    \fi
}%end drawvoltages
%%%---------- close: tex/pgfcircvoltage
%%%%%%%%%%% Springe nach tex/pgfcirccurrent
%%%---------- open: tex/pgfcirccurrent.tex
% Copyright 2018-2020 by Romano Giannetti
% Copyright 2015-2020 by Stefan Lindner
% Copyright 2013-2020 by Stefan Erhardt
% Copyright 2007-2020 by Massimo Redaelli
%
% This file may be distributed and/or modified
%
% 1. under the LaTeX Project Public License and/or
% 2. under the GNU Public License.
%
% See the files gpl-3.0_license.txt and lppl-1-3c_license.txt for more details.

%%%%%%%%%%%%%%%%%%%%%%%%%%%%%%%%%%%%%%%%
%% Current handling

%% styles
\ctikzset{bipole current style/.style={}}
\tikzset{bipole current style/.code={
        \ctikzset{bipole current style/.style={#1}}
}}
\tikzset{bipole current append style/.code={
        \ctikzset{bipole current style/.append style={#1}}
}}

%% Options
\ctikzset{i^>/.style = {
        i = #1,
        \circuitikzbasekey/bipole/current/direction = forward,
        \circuitikzbasekey/bipole/current/x position = after,
        \circuitikzbasekey/bipole/current/y position = above
    }
}

\ctikzset{i_>/.style = {
        i = #1,
        \circuitikzbasekey/bipole/current/direction = forward,
        \circuitikzbasekey/bipole/current/x position = after,
        \circuitikzbasekey/bipole/current/y position = below
    }
}

\ctikzset{i>^/.style = {
        i = #1,
        \circuitikzbasekey/bipole/current/direction = forward,
        \circuitikzbasekey/bipole/current/x position = before,
        \circuitikzbasekey/bipole/current/y position = above
    }
}

\ctikzset{i>_/.style = {
        i = #1,
        \circuitikzbasekey/bipole/current/direction = forward,
        \circuitikzbasekey/bipole/current/x position = before,
        \circuitikzbasekey/bipole/current/y position = below
    }
}

\ctikzset{i^</.style = {
        i = #1,
        \circuitikzbasekey/bipole/current/direction = backward,
        \circuitikzbasekey/bipole/current/x position = after,
        \circuitikzbasekey/bipole/current/y position = above
    }
}

\ctikzset{i_</.style = {
        i = #1,
        \circuitikzbasekey/bipole/current/direction = backward,
        \circuitikzbasekey/bipole/current/x position = after,
        \circuitikzbasekey/bipole/current/y position = below
    }
}

\ctikzset{i<^/.style = {
        i = #1,
        \circuitikzbasekey/bipole/current/direction = backward,
        \circuitikzbasekey/bipole/current/x position = before,
        \circuitikzbasekey/bipole/current/y position = above
    }
}

\ctikzset{i<_/.style = {
        i = #1,
        \circuitikzbasekey/bipole/current/direction = backward,
        \circuitikzbasekey/bipole/current/x position = before,
        \circuitikzbasekey/bipole/current/y position = below
    }
}

\ctikzset{i/.code = {
        \pgfkeys{\circuitikzbasekey/bipole/current/direction = forward,
            \circuitikzbasekey/bipole/current/x position = after,
        \circuitikzbasekey/bipole/current/y position = above }
        \pgfkeys{/tikz/circuitikz/bipole/current/label/name=#1}
        \ctikzsetvalof{bipole/current/label/unit}{}
        \ifpgf@circ@siunitx
            \pgf@circ@handleSI{#1}
            \ifpgf@circ@siunitx@res
                \edef\pgf@temp{\pgf@circ@handleSI@val}
                \pgfkeyslet{/tikz/circuitikz/bipole/current/label/name}{\pgf@temp}
                \edef\pgf@temp{\pgf@circ@handleSI@unit}
                \pgfkeyslet{/tikz/circuitikz/bipole/current/label/unit}{\pgf@temp}
            \else
        \fi
        \else
    \fi
    %reverse current direction for voltage sources
    \ifpgf@circ@oldvoltagedirection\else
        \ifpgf@circuit@bipole@isvoltage
            \ifpgf@circuit@bipole@voltage@backward
                \pgfkeys{\circuitikzbasekey/bipole/current/direction = forward,
                    \circuitikzbasekey/bipole/current/x position = after,
                \circuitikzbasekey/bipole/current/y position = below }
            \else
                \pgfkeys{\circuitikzbasekey/bipole/current/direction = backward,
                    \circuitikzbasekey/bipole/current/x position = before,
                \circuitikzbasekey/bipole/current/y position = above }
            \fi\fi\fi
    }
}

\ifpgf@circ@oldvoltagedirection
    \ctikzset{i</.style = { i = #1, \circuitikzbasekey/bipole/current/direction = backward } }
\else
    \ctikzset{i</.style = { i = #1, \circuitikzbasekey/bipole/current/direction = backward, \circuitikzbasekey/bipole/current/x position = before } }
\fi
\ctikzset{i_/.style = { i = #1, \circuitikzbasekey/bipole/current/y position = below } }
\ctikzset{i>/.style = {i = #1, \circuitikzbasekey/bipole/current/direction = forward } }
\ctikzset{i^/.style = { i= #1, \circuitikzbasekey/bipole/current/y position = above } }

%% Output routine
\def\pgf@circ@drawcurrent{
    \pgfextra{
        \edef\pgf@circ@ffffff{\pgf@circ@direction}
        \def\pgfcircmathresult{\expandafter\pgf@circ@stripdecimals\pgf@circ@ffffff\pgf@nil}

        \ifnum\pgfcircmathresult >4 \ifnum\pgfcircmathresult <86
        \ifpgf@circuit@bipole@current@below
            \def\pgf@circ@dir{north west} \else \def\pgf@circ@dir{south east}
        \fi
        \fi\fi
        \ifnum\pgfcircmathresult >85 \ifnum\pgfcircmathresult <95
        \ifpgf@circuit@bipole@current@below
            \def\pgf@circ@dir{west} \else \def\pgf@circ@dir{east}
        \fi
        \fi\fi
        \ifnum\pgfcircmathresult >94 \ifnum\pgfcircmathresult <176
        \ifpgf@circuit@bipole@current@below
            \def\pgf@circ@dir{south west}\else \def\pgf@circ@dir{north east}
        \fi
        \fi\fi
        \ifnum\pgfcircmathresult >175 \ifnum\pgfcircmathresult <185
        \ifpgf@circuit@bipole@current@below
            \def\pgf@circ@dir{south}\else\def\pgf@circ@dir{north}
        \fi
        \fi\fi
        \ifnum\pgfcircmathresult >184 \ifnum\pgfcircmathresult <266
        \ifpgf@circuit@bipole@current@below
            \def\pgf@circ@dir{south east}\else\def\pgf@circ@dir{north west}
        \fi
        \fi\fi
        \ifnum\pgfcircmathresult >265 \ifnum\pgfcircmathresult <275
        \ifpgf@circuit@bipole@current@below
            \def\pgf@circ@dir{east}\else \def\pgf@circ@dir{west}
        \fi
        \fi\fi
        \ifnum\pgfcircmathresult >274 \ifnum\pgfcircmathresult <356
        \ifpgf@circuit@bipole@current@below
            \def\pgf@circ@dir{north east}\else\def\pgf@circ@dir{south west}
        \fi
        \fi\fi
        \ifnum\pgfcircmathresult <5
        \ifpgf@circuit@bipole@current@below
            \def\pgf@circ@dir{north}\else\def\pgf@circ@dir{south}
        \fi
        \fi
        \ifnum\pgfcircmathresult >355
        \ifpgf@circuit@bipole@current@below
            \def\pgf@circ@dir{north}\else\def\pgf@circ@dir{south}
        \fi
    \fi

    \ifpgf@circuit@bipole@current@below
        \def\pgf@circ@bipole@current@label@where{-90}
    \else
        \def\pgf@circ@bipole@current@label@where{+90}
    \fi
    }

    \pgfextra{\def\pgf@temp{short}\edef\pgf@circ@temp{\ctikzvalof{bipole/kind}}}
    \ifx\pgf@circ@temp\pgf@temp%draw current at a short at middle of the line
        (\tikztostart)--(\tikztotarget)
    \else% normal bipole or source
        \ifpgf@circuit@bipole@current@before
            (\tikztostart)--(anchorstartnode)
        \else
            (anchorendnode)--(\tikztotarget)
        \fi
    \fi
    \ifpgf@circuit@bipole@current@backward
        \pgfextra{
            \pgfmathsubtract{\pgf@circ@ffffff}{180}
            \edef\pgf@circ@ffffff{\expandafter\pgf@circ@stripdecimals\pgfmathresult\pgf@nil}
        }
    \fi
    coordinate[currarrow,pos=\ctikzvalof{current/distance},rotate=\pgf@circ@ffffff,
    anchor=center](Iarrow)
    (Iarrow.\pgf@circ@bipole@current@label@where)
    node[anchor=\pgf@circ@dir, \circuitikzbasekey/bipole current style]
    (\ctikzvalof{bipole/name}current){\pgf@circ@finallabels{current/label}}
}

%%%---------- close: tex/pgfcirccurrent
%%%%%%%%%%% Springe nach tex/pgfcircflow
%%%---------- open: tex/pgfcircflow.tex
% Copyright 2018-2020 by Romano Giannetti
% Copyright 2015-2020 by Stefan Lindner
% Copyright 2013-2020 by Stefan Erhardt
% Copyright 2007-2020 by Massimo Redaelli
%
% This file may be distributed and/or modified
%
% 1. under the LaTeX Project Public License and/or
% 2. under the GNU Public License.
%
% See the files gpl-3.0_license.txt and lppl-1-3c_license.txt for more details.

%%%%%%%%%%%%%%%%%%%%%%%%%%%%%%%%%%%%%%%%
%% flow handling

%% styles
\ctikzset{bipole flow style/.style={}}
\tikzset{bipole flow style/.code={
        \ctikzset{bipole flow style/.style={#1}}
}}
\tikzset{bipole flow append style/.code={
        \ctikzset{bipole flow style/.append style={#1}}
}}

%% Options
\ctikzset{f^>/.style = {
        f = #1,
        \circuitikzbasekey/bipole/flow/direction = forward,
        \circuitikzbasekey/bipole/flow/x position = after,
        \circuitikzbasekey/bipole/curflowrent/y position = above
    }
}

\ctikzset{f_>/.style = {
        f = #1,
        \circuitikzbasekey/bipole/flow/direction = forward,
        \circuitikzbasekey/bipole/flow/x position = after,
        \circuitikzbasekey/bipole/flow/y position = below
    }
}

\ctikzset{f>^/.style = {
        f = #1,
        \circuitikzbasekey/bipole/flow/direction = forward,
        \circuitikzbasekey/bipole/flow/x position = before,
        \circuitikzbasekey/bipole/flow/y position = above
    }
}

\ctikzset{f>_/.style = {
        f = #1,
        \circuitikzbasekey/bipole/flow/direction = forward,
        \circuitikzbasekey/bipole/flow/x position = before,
        \circuitikzbasekey/bipole/flow/y position = below
    }
}

\ctikzset{f^</.style = {
        f = #1,
        \circuitikzbasekey/bipole/flow/direction = backward,
        \circuitikzbasekey/bipole/flow/x position = after,
        \circuitikzbasekey/bipole/flow/y position = above
    }
}

\ctikzset{f_</.style = {
        f = #1,
        \circuitikzbasekey/bipole/flow/direction = backward,
        \circuitikzbasekey/bipole/flow/x position = after,
        \circuitikzbasekey/bipole/flow/y position = below
    }
}

\ctikzset{f<^/.style = {
        f = #1,
        \circuitikzbasekey/bipole/flow/direction = backward,
        \circuitikzbasekey/bipole/flow/x position = before,
        \circuitikzbasekey/bipole/flow/y position = above
    }
}

\ctikzset{f<_/.style = {
        f = #1,
        \circuitikzbasekey/bipole/flow/direction = backward,
        \circuitikzbasekey/bipole/flow/x position = before,
        \circuitikzbasekey/bipole/flow/y position = below
    }
}

\ctikzset{f</.style = { f = #1, \circuitikzbasekey/bipole/flow/direction = backward} }
\ctikzset{f_/.style = { f = #1, \circuitikzbasekey/bipole/flow/y position = below } }
\ctikzset{f>/.style = { f = #1, \circuitikzbasekey/bipole/flow/direction = forward } }
\ctikzset{f^/.style = { f = #1, \circuitikzbasekey/bipole/flow/y position = above } }

\ctikzset{f/.code = {
        \pgfkeys{\circuitikzbasekey/bipole/flow/direction = forward,
            \circuitikzbasekey/bipole/flow/x position = after,
        \circuitikzbasekey/bipole/flow/y position = above }
        \pgfkeys{/tikz/circuitikz/bipole/flow/label/name=#1}
        \ctikzsetvalof{bipole/flow/label/unit}{}
        \ifpgf@circ@siunitx
            \pgf@circ@handleSI{#1}
            \ifpgf@circ@siunitx@res
                \edef\pgf@temp{\pgf@circ@handleSI@val}
                \pgfkeyslet{/tikz/circuitikz/bipole/flow/label/name}{\pgf@temp}
                \edef\pgf@temp{\pgf@circ@handleSI@unit}
                \pgfkeyslet{/tikz/circuitikz/bipole/flow/label/unit}{\pgf@temp}
            \else
        \fi
        \else
    \fi
}
}

%% Output routine
\def\pgf@circ@drawflow{
    \pgfextra{
        \edef\pgf@circ@ffffff{\pgf@circ@direction}
        \def\pgfcircmathresult{\expandafter\pgf@circ@stripdecimals\pgf@circ@ffffff\pgf@nil}

        \ifnum\pgfcircmathresult >4 \ifnum\pgfcircmathresult <86
                \ifpgf@circuit@bipole@flow@below
                        \def\pgf@circ@dir{north west} \else \def\pgf@circ@dir{south east}
                \fi
        \fi\fi
        \ifnum\pgfcircmathresult >85 \ifnum\pgfcircmathresult <95
                \ifpgf@circuit@bipole@flow@below
                        \def\pgf@circ@dir{west} \else \def\pgf@circ@dir{east}
                \fi
        \fi\fi
        \ifnum\pgfcircmathresult >94 \ifnum\pgfcircmathresult <176
                \ifpgf@circuit@bipole@flow@below
                         \def\pgf@circ@dir{south west}\else \def\pgf@circ@dir{north east}
                \fi
        \fi\fi
        \ifnum\pgfcircmathresult >175 \ifnum\pgfcircmathresult <185
                \ifpgf@circuit@bipole@flow@below
                          \def\pgf@circ@dir{south}\else\def\pgf@circ@dir{north}
                \fi
        \fi\fi
        \ifnum\pgfcircmathresult >184 \ifnum\pgfcircmathresult <266
                \ifpgf@circuit@bipole@flow@below
                         \def\pgf@circ@dir{south east}\else\def\pgf@circ@dir{north west}
                \fi
        \fi\fi
        \ifnum\pgfcircmathresult >265 \ifnum\pgfcircmathresult <275
                \ifpgf@circuit@bipole@flow@below
                         \def\pgf@circ@dir{east}\else \def\pgf@circ@dir{west}
                \fi
        \fi\fi
        \ifnum\pgfcircmathresult >274 \ifnum\pgfcircmathresult <356
                \ifpgf@circuit@bipole@flow@below
                          \def\pgf@circ@dir{north east}\else\def\pgf@circ@dir{south west}
                \fi
        \fi\fi
        \ifnum\pgfcircmathresult <5
                \ifpgf@circuit@bipole@flow@below
                         \def\pgf@circ@dir{north}\else\def\pgf@circ@dir{south}
                \fi
        \fi
        \ifnum\pgfcircmathresult >355
                \ifpgf@circuit@bipole@flow@below
                         \def\pgf@circ@dir{north}\else\def\pgf@circ@dir{south}
                \fi
        \fi

        \ifpgf@circuit@bipole@flow@below
                \def\pgf@circ@bipole@flow@label@where{-90}
        \else
                \def\pgf@circ@bipole@flow@label@where{+90}
        \fi
    }

    \pgfextra{\def\pgf@temp{short}\edef\pgf@circ@temp{\ctikzvalof{bipole/kind}}}
    \ifx\pgf@circ@temp\pgf@temp%draw flow at a short at middle of the line
        (\tikztostart)--(\tikztotarget)
    \else% normal bipole or source
        \ifpgf@circuit@bipole@flow@before
            (\tikztostart)--(anchorstartnode)
        \else
            (anchorendnode)--(\tikztotarget)
        \fi
    \fi
    \pgfextra{
        \ifpgf@circuit@bipole@flow@backward
            \ifpgf@circuit@bipole@flow@below
                \def\flow@offset{\ctikzvalof{flow/offset}\pgf@circ@Rlen}
            \else
                \def\flow@offset{-\ctikzvalof{flow/offset}\pgf@circ@Rlen}
            \fi
            \pgfmathsubtract{\pgf@circ@ffffff}{180}
            \edef\pgf@circ@ffffff{\expandafter\pgf@circ@stripdecimals\pgfmathresult\pgf@nil}
            \else
            \ifpgf@circuit@bipole@flow@below
                \def\flow@offset{-\ctikzvalof{flow/offset}\pgf@circ@Rlen}
            \else
                \def\flow@offset{\ctikzvalof{flow/offset}\pgf@circ@Rlen}
            \fi
        \fi
    }
    coordinate[flowarrow,pos=\ctikzvalof{flow/distance},rotate=\pgf@circ@ffffff,yshift=\flow@offset, anchor=center](Farrowpos)
    (Farrowpos.\pgf@circ@bipole@flow@label@where) node[anchor=\pgf@circ@dir, \circuitikzbasekey/bipole flow style]
    (\ctikzvalof{bipole/name}flow){\pgf@circ@finallabels{flow/label}}
}

%%%---------- close: tex/pgfcircflow

% notice that the default is nooldvoltagedirection; it's not explicitly set to allow for the warning
\ExecuteOptions{nofetbodydiode, nofetsolderdot, europeancurrents, europeanvoltages,americanports, americanresistors, cuteinductors ,europeangfsurgearrester, nosiunitx, noarrowmos, smartlabels}

\ProcessOptions\relax

%%%%%%%%%%% Springe nach tex/pgfcircpath
%%%---------- open: tex/pgfcircpath.tex
% Copyright 2018-2020 by Romano Giannetti
% Copyright 2015-2020 by Stefan Lindner
% Copyright 2013-2020 by Stefan Erhardt
% Copyright 2007-2020 by Massimo Redaelli
%
% This file may be distributed and/or modified
%
% 1. under the LaTeX Project Public License and/or
% 2. under the GNU Public License.
%
% See the files gpl-3.0_license.txt and lppl-1-3c_license.txt for more details.


\def\pgf@circ@direction{0.0}

% Names
\ctikzset{name/.style = { n=#1 } } %%%%%%%%%%%%%%%%%%%%%%%%%%%%%%%%%%%@@@
\ctikzset{n/.code = {
	\pgfkeys{/tikz/circuitikz/bipole/name=#1}
}}

% Reflect the node along
\ctikzset{mirrored/.is choice}
\ctikzset{mirror value/.initial=1}
\ctikzset{mirrored/true/.code = {\ctikzsetvalof{mirror value}{-1}} }
\ctikzset{mirrored/false/.code = {\ctikzsetvalof{mirror value}{1}} }
\ctikzset{mirror/.style = {/tikz/circuitikz/mirrored=true}}

% Invert node along path
\ctikzset{inverted/.is choice}
\ctikzset{invert value/.initial=1}
\ctikzset{inverted/true/.code = {\ctikzsetvalof{invert value}{-1}\pgf@circuit@bipole@invertedtrue}}
\ctikzset{inverted/false/.code = {\ctikzsetvalof{invert value}{1}\pgf@circuit@bipole@invertedfalse}}
\ctikzset{invert/.style = {/tikz/circuitikz/inverted=true}}
\newif\ifpgf@circuit@bipole@inverted
\ctikzset{bipole/inverted/.is if=pgf@circuit@bipole@inverted}

\newif\ifpgf@circuit@bipole@voltage@backward
\ctikzset{bipole/voltage/direction/.is choice}
\ctikzset{bipole/voltage/direction/forward/.code={\pgf@circuit@bipole@voltage@backwardfalse}}
\ctikzset{bipole/voltage/direction/backward/.code={\pgf@circuit@bipole@voltage@backwardtrue}}

% Initialize paths
\def\pgfcircresetpath{
    \ctikzset{bipole/name=, bipole/label/name=, bipole/label/position=90, ,bipole/annotation/name=, bipole/annotation/position=-90,
        bipole/inverted=false, bipole/kind=,
        bipole/voltage/direction=backward, bipole/voltage/label/name=, bipole/voltage/position=below,
        bipole/nodes/left=none, bipole/nodes/right=none, bipole/is voltage=false,bipole/is voltageoutsideofsymbol=false,bipole/is strokedsymbol=false,
        bipole/is current=false, bipole/current/label/name=, bipole/current/x position=after,
        bipole/current/y position=above, bipole/current/direction=forward,
        mirrored=false
    }
}

%% Helper function for path-function to ensure using anchors between nodes
\def\set@explicit@center@anchor#1{
    \pgfutil@ifundefined{pgf@sh@ns@#1}
    {
        %This coordinate is no node(but a relative position or a coordinate), no further handling needed
        }{
        \pgfutil@in@.{#1}
        \ifpgfutil@in@
            % Anchor is used, do nothing!
    \else%
        \let\tikz@moveto@waiting=\relax
        \pgfpathmoveto{\tikz@last@position}%force movement, because tikz@moveto@waiting
        \edef#1{#1.center}%ensure using center anchor
    \fi
}
}


%% Generic bipole path
%% I am not user what the second argument is needed for
\def\pgf@circ@bipole@path#1#2{

    \pgfextra{
        \set@explicit@center@anchor{\tikztostart}
        \set@explicit@center@anchor{\tikztotarget}
        \pgfsyssoftpath@getcurrentpath{\myp@th}%% save current path to extend after calculation of correct start/end coordinates
        \ctikzset{bipole/kind = #1}
        \edef\pgf@temp{\ctikzvalof{bipole/name}}
        \def\pgf@circ@temp{}
        \ifx\pgf@temp\pgf@circ@temp % if it has not a name
            \pgfmathrandominteger{\pgf@circ@rand}{1000}{9999}
            \ctikzset{bipole/name = #1\pgf@circ@rand} % create it (re-usage should not create problem, but...)
        \fi
    }

    (\tikztostart) coordinate (\ctikzvalof{bipole/name}start)%necessary to get correct coordinates in the case of relative start/end or constructions like ((node1)-|(node2))
    (\tikztotarget) coordinate (\ctikzvalof{bipole/name}end)
    \pgfextra{
        \pgfmathanglebetweenpoints{\pgfpointanchor{\ctikzvalof{bipole/name}start}{center}}
        {\pgfpointanchor{\ctikzvalof{bipole/name}end}{center}}
        % DO NOT ROUND HERE! pgfcirclabel.tex will do that when needed
        % \pgfmathround{\pgfmathresult}
        \edef\pgf@circ@direction{\pgfmathresult}%Calculate direction(angle) of path
        % \typeout{DIRECTION:\pgf@circ@direction}
        \pgfsyssoftpath@setcurrentpath{\myp@th}
    }
    \pgfextra{\def\pgf@temp{open}\def\pgf@circ@temp{#1}}
    \ifx\pgf@temp\pgf@circ@temp  % if it is an open
        \else
            --($(\ctikzvalof{bipole/name}start) ! .5\pgflinewidth ! (\ctikzvalof{bipole/name}end)$) %ugly workaround to get correct linejoins(node breaks path?)
        \fi
        ($(\tikztostart) ! .5 ! (\tikztotarget)$)%%positio of middle node
        node[#1shape, rotate=\pgf@circ@direction, yscale=\ctikzvalof{mirror value}, xscale=\ctikzvalof{invert value}]
        (\ctikzvalof{bipole/name}) {}
        \ifpgf@circuit@bipole@inverted
            \ifcsname pgf@anchor@#1shape@pathstart\endcsname%if special path-anchors are defined, use them!
                coordinate	(anchorstartnode) at (\ctikzvalof{bipole/name}.pathend)
                coordinate	(anchorendnode) at (\ctikzvalof{bipole/name}.pathstart)
            \else
                coordinate	(anchorstartnode) at (\ctikzvalof{bipole/name}.right)
                coordinate	(anchorendnode) at (\ctikzvalof{bipole/name}.left)
            \fi
            \else
            \ifcsname pgf@anchor@#1shape@pathstart\endcsname%if special path-anchors are defined, use them!
                coordinate	(anchorstartnode) at (\ctikzvalof{bipole/name}.pathstart)
                coordinate	(anchorendnode) at (\ctikzvalof{bipole/name}.pathend)
            \else
                coordinate	(anchorstartnode) at (\ctikzvalof{bipole/name}.left)
                coordinate	(anchorendnode) at (\ctikzvalof{bipole/name}.right)
            \fi
        \fi
        \pgfextra{\def\pgf@temp{open}\def\pgf@circ@temp{#1}}
        \ifx\pgf@temp\pgf@circ@temp  % if it is an open
        \else
            (\ctikzvalof{bipole/name}start.center) -- (anchorstartnode)
            (anchorendnode)  -- (\ctikzvalof{bipole/name}end.center)
        \fi

        \drawpoles
        \pgf@circ@ifkeyempty{bipole/label/name}\else\pgf@circ@drawlabels{label}\fi
        \pgf@circ@ifkeyempty{bipole/annotation/name}\else\pgf@circ@drawlabels{annotation}\fi
        \pgf@circ@ifkeyempty{bipole/voltage/label/name}\else\pgf@circ@drawvoltage\fi
        \pgf@circ@ifkeyempty{bipole/current/label/name}\else\pgf@circ@drawcurrent\fi
        \pgf@circ@ifkeyempty{bipole/flow/label/name}\else\pgf@circ@drawflow\fi
        \pgfextra{\def\pgf@temp{open}\def\pgf@circ@temp{#1}}
        \ifx\pgf@temp\pgf@circ@temp  % if it is an open
            (\ctikzvalof{bipole/name}end)%Move to end of path
        \else
            ($(\ctikzvalof{bipole/name}end) ! .5\pgflinewidth ! (\ctikzvalof{bipole/name}start)$) -- (\ctikzvalof{bipole/name}end)%ugly workaround to get correct linejoins(node breaks path?)
            %tikztostart and tikztotarget could not be used, because it would break if target coordinate is something like (node1-|node2)

        \fi
    % reset
    \pgfextra{
        \pgfcircresetpath
    }
    \tikztonodes%draw pending nodes an path
}

%%%%%%%%%%%%%%%%%%%%%%%%%%%%%
%% Handling of terminals

\ctikzset{bipole/nodes/.is family}
\ctikzset{bipole/nodes/left/.initial=none}
\ctikzset{bipole/nodes/right/.initial=none}
\tikzset{bipole nodes/.style n args={2}{%
    \circuitikzbasekey/bipole/nodes/left=#1,
    \circuitikzbasekey/bipole/nodes/right=#2
    }
}

%% Easily usable styles

\ctikzset{o-o/.style = {\circuitikzbasekey/bipole/nodes/left=ocirc, \circuitikzbasekey/bipole/nodes/right=ocirc}}
\ctikzset{-o/.style = {\circuitikzbasekey/bipole/nodes/left=none, \circuitikzbasekey/bipole/nodes/right=ocirc}}
\ctikzset{o-/.style = {\circuitikzbasekey/bipole/nodes/left=ocirc, \circuitikzbasekey/bipole/nodes/right=none}}
\ctikzset{*-o/.style = {\circuitikzbasekey/bipole/nodes/left=circ, \circuitikzbasekey/bipole/nodes/right=ocirc}}
\ctikzset{o-*/.style = {\circuitikzbasekey/bipole/nodes/left=ocirc, \circuitikzbasekey/bipole/nodes/right=circ}}
\ctikzset{d-o/.style = {\circuitikzbasekey/bipole/nodes/left=diamondpole, \circuitikzbasekey/bipole/nodes/right=ocirc}}
\ctikzset{o-d/.style = {\circuitikzbasekey/bipole/nodes/left=ocirc, \circuitikzbasekey/bipole/nodes/right=diamondpole}}
\ctikzset{*-/.style = {\circuitikzbasekey/bipole/nodes/left=circ, \circuitikzbasekey/bipole/nodes/right=none}}
\ctikzset{-*/.style = {\circuitikzbasekey/bipole/nodes/left=none, \circuitikzbasekey/bipole/nodes/right=circ}}
\ctikzset{d-/.style = {\circuitikzbasekey/bipole/nodes/left=diamondpole, \circuitikzbasekey/bipole/nodes/right=none}}
\ctikzset{-d/.style = {\circuitikzbasekey/bipole/nodes/left=none, \circuitikzbasekey/bipole/nodes/right=diamondpole}}
\ctikzset{*-*/.style = {\circuitikzbasekey/bipole/nodes/left=circ, \circuitikzbasekey/bipole/nodes/right=circ}}
\ctikzset{d-*/.style = {\circuitikzbasekey/bipole/nodes/left=diamondpole, \circuitikzbasekey/bipole/nodes/right=circ}}
\ctikzset{*-d/.style = {\circuitikzbasekey/bipole/nodes/left=circ, \circuitikzbasekey/bipole/nodes/right=diamondpole}}
\ctikzset{d-d/.style = {\circuitikzbasekey/bipole/nodes/left=diamondpole, \circuitikzbasekey/bipole/nodes/right=diamondpole}}

% rectjoinfill workarounds

\ctikzset{.-/.style = {\circuitikzbasekey/bipole/nodes/left=rectjoinfill, \circuitikzbasekey/bipole/nodes/right=none}}
\ctikzset{.-*/.style = {\circuitikzbasekey/bipole/nodes/left=rectjoinfill, \circuitikzbasekey/bipole/nodes/right=circ}}
\ctikzset{.-o/.style = {\circuitikzbasekey/bipole/nodes/left=rectjoinfill, \circuitikzbasekey/bipole/nodes/right=ocirc}}
\ctikzset{.-d/.style = {\circuitikzbasekey/bipole/nodes/left=rectjoinfill, \circuitikzbasekey/bipole/nodes/right=diamondpole}}
\ctikzset{-./.style = {\circuitikzbasekey/bipole/nodes/left=none, \circuitikzbasekey/bipole/nodes/right=rectjoinfill}}
\ctikzset{o-./.style = {\circuitikzbasekey/bipole/nodes/left=ocirc, \circuitikzbasekey/bipole/nodes/right=rectjoinfill}}
\ctikzset{*-./.style = {\circuitikzbasekey/bipole/nodes/left=circ, \circuitikzbasekey/bipole/nodes/right=rectjoinfill}}
\ctikzset{d-./.style = {\circuitikzbasekey/bipole/nodes/left=diamondpole, \circuitikzbasekey/bipole/nodes/right=rectjoinfill}}
\ctikzset{.-./.style = {\circuitikzbasekey/bipole/nodes/left=rectjoinfill, \circuitikzbasekey/bipole/nodes/right=rectjoinfill}}

\tikzset{reversed/.style = {\circuitikzbasekey/bipole/inverted=true}}

\def\drawpoles{
    \pgfextra{ \edef\pgf@circ@temp{\ctikzvalof{bipole/nodes/left}} \def\pgf@temp{none}}
    \ifx\pgf@temp\pgf@circ@temp\else(\tikztostart) node[\pgf@circ@temp] {}\fi
    \pgfextra{ \edef\pgf@circ@temp{\ctikzvalof{bipole/nodes/right}} }
    \ifx\pgf@temp\pgf@circ@temp\else(\tikztotarget) node[\pgf@circ@temp] {}\fi
}

%% Path definitions
\def\pgf@circ@resistor@path#1{\ifpgf@circuit@europeanresistor\pgf@circ@bipole@path{generic}{#1}\else\pgf@circ@bipole@path{resistor}{#1}\fi}
\def\pgf@circ@vresistor@path#1{\ifpgf@circuit@europeanresistor\pgf@circ@bipole@path{tgeneric}{#1}\else\pgf@circ@bipole@path{vresistor}{#1}\fi}
\def\pgf@circ@sresistor@path#1{\ifpgf@circuit@europeanresistor\pgf@circ@bipole@path{thermistor}{#1}\else\pgf@circ@bipole@path{resistivesens}{#1}\fi}
\def\pgf@circ@potentiometer@path#1{\ifpgf@circuit@europeanresistor\pgf@circ@bipole@path{genericpotentiometer}{#1}\else\pgf@circ@bipole@path{potentiometer}{#1}\fi}
\def\pgf@circ@thermistor@path#1{\pgf@circ@bipole@path{thermistor}{#1}}
\def\pgf@circ@thermistorptc@path#1{\pgf@circ@bipole@path{thermistorptc}{#1}}
\def\pgf@circ@thermistorntc@path#1{\pgf@circ@bipole@path{thermistorntc}{#1}}
\def\pgf@circ@varistor@path#1{\pgf@circ@bipole@path{varistor}{#1}}
\def\pgf@circ@capacitor@path#1{\pgf@circ@bipole@path{capacitor}{#1}}
\def\pgf@circ@capacitivesens@path#1{\pgf@circ@bipole@path{capacitivesens}{#1}}
\def\pgf@circ@ecapacitor@path#1{\pgf@circ@bipole@path{ecapacitor}{#1}}
%% polar capacitor is deprectaed, use curved capacitor instead
\def\pgf@circ@polarcapacitor@path#1{\pgf@circ@bipole@path{polarcapacitor}{#1}}
\def\pgf@circ@ccapacitor@path#1{\pgf@circ@bipole@path{ccapacitor}{#1}}
\def\pgf@circ@vcapacitor@path#1{\pgf@circ@bipole@path{vcapacitor}{#1}}
\def\pgf@circ@piezoelectric@path#1{\pgf@circ@bipole@path{piezoelectric}{#1}}
\def\pgf@circ@battery@path#1{\pgf@circ@bipole@path{battery}{#1}}
\def\pgf@circ@batteryone@path#1{\pgf@circ@bipole@path{battery1}{#1}}
\def\pgf@circ@batterytwo@path#1{\pgf@circ@bipole@path{battery2}{#1}}
\def\pgf@circ@europeaninductor@path#1{\pgf@circ@bipole@path{fullgeneric}{#1}}
\def\pgf@circ@americaninductor@path#1{\pgf@circ@bipole@path{americaninductor}{#1}}
\def\pgf@circ@cuteinductor@path#1{\pgf@circ@bipole@path{cuteinductor}{#1}}
\def\pgf@circ@cutechoke@path#1{\pgf@circ@bipole@path{cutechoke}{#1}}
\def\pgf@circ@inductor@path#1{%
    \pgfextra{
        \edef\pgf@circ@temp{\ctikzvalof{inductor}}%
        \def\pgf@temp{european}%
    }
    \ifx\pgf@temp\pgf@circ@temp%
        \pgf@circ@europeaninductor@path{#1}%
    \else%
        \pgfextra{	\def\pgf@temp{cute} }%
        \ifx\pgf@temp\pgf@circ@temp%
            \pgf@circ@cuteinductor@path{#1}%
        \else%
            \pgf@circ@americaninductor@path{#1}%
        \fi%
    \fi%
}
\def\pgf@circ@vinductor@path#1{
    \pgfextra{
        \edef\pgf@circ@temp{\ctikzvalof{inductor}}%
        \def\pgf@temp{european}%
    }
    \ifx\pgf@temp\pgf@circ@temp%
        \pgf@circ@veuropeaninductor@path{#1}%
    \else%
        \pgfextra{	\def\pgf@temp{cute} }%
        \ifx\pgf@temp\pgf@circ@temp%
            \pgf@circ@vcuteinductor@path{#1}%
        \else%
            \pgf@circ@vamericaninductor@path{#1}%
        \fi%
    \fi%
}
\def\pgf@circ@inductivesens@path#1{%
    \pgfextra{
        \edef\pgf@circ@temp{\ctikzvalof{inductor}}%
        \def\pgf@temp{european}%
    }
    \ifx\pgf@temp\pgf@circ@temp%
        \pgf@circ@europeaninductivesens@path{#1}%
    \else%
        \pgfextra{	\def\pgf@temp{cute} }%
        \ifx\pgf@temp\pgf@circ@temp%
            \pgf@circ@cuteinductivesens@path{#1}%
        \else%
            \pgf@circ@americaninductivesens@path{#1}%
        \fi%
    \fi%
}
\def\pgf@circ@veuropeaninductor@path#1{\pgf@circ@bipole@path{tfullgeneric}{#1}}
\def\pgf@circ@vamericaninductor@path#1{\pgf@circ@bipole@path{vamericaninductor}{#1}}
\def\pgf@circ@vcuteinductor@path#1{\pgf@circ@bipole@path{vcuteinductor}{#1}}
\def\pgf@circ@europeaninductivesens@path#1{\pgf@circ@bipole@path{sfullgeneric}{#1}}
\def\pgf@circ@americaninductivesens@path#1{\pgf@circ@bipole@path{samericaninductor}{#1}}
\def\pgf@circ@cuteinductivesens@path#1{\pgf@circ@bipole@path{scuteinductor}{#1}}
\def\pgf@circ@lamp@path#1{\pgf@circ@bipole@path{lamp}{#1}}
\def\pgf@circ@bulb@path#1{\pgf@circ@bipole@path{bulb}{#1}}
\def\pgf@circ@esource@path#1{\pgf@circ@bipole@path{esource}{#1}}
\def\pgf@circ@pvsource@path#1{\pgf@circ@bipole@path{pvsource}{#1}}
\def\pgf@circ@vsource@path#1{\pgf@circ@bipole@path{vsource}{#1}}
\def\pgf@circ@dcvsource@path#1{\pgf@circ@bipole@path{dcvsource}{#1}}
\def\pgf@circ@vsourceam@path#1{\pgf@circ@bipole@path{vsourceAM}{#1}}
\def\pgf@circ@vsourcesin@path#1{\pgf@circ@bipole@path{vsourcesin}{#1}}
\def\pgf@circ@isource@path#1{\pgf@circ@bipole@path{isource}{#1}}
\def\pgf@circ@oosource@path#1{\pgf@circ@bipole@path{oosource}{#1}}
\def\pgf@circ@oosourcetrans@path#1{\pgf@circ@bipole@path{oosourcetrans}{#1}}
\def\pgf@circ@ooosource@path#1{\pgf@circ@bipole@path{ooosource}{#1}}
\def\pgf@circ@dcisource@path#1{\pgf@circ@bipole@path{dcisource}{#1}}
\def\pgf@circ@isourcesin@path#1{\pgf@circ@bipole@path{isourcesin}{#1}}
\def\pgf@circ@vsourcesquare@path#1{\pgf@circ@bipole@path{vsourcesquare}{#1}}
\def\pgf@circ@vsourcetri@path#1{\pgf@circ@bipole@path{vsourcetri}{#1}}
\def\pgf@circ@isourceam@path#1{\pgf@circ@bipole@path{isourceAM}{#1}}
\def\pgf@circ@ecsource@path#1{\pgf@circ@bipole@path{ecsource}{#1}}
\def\pgf@circ@cvsource@path#1{\pgf@circ@bipole@path{cvsource}{#1}}
\def\pgf@circ@cvsourceam@path#1{\pgf@circ@bipole@path{cvsourceAM}{#1}}
\def\pgf@circ@cvsourcesin@path#1{\pgf@circ@bipole@path{cvsourcesin}{#1}}
\def\pgf@circ@cisource@path#1{\pgf@circ@bipole@path{cisource}{#1}}
\def\pgf@circ@cisourceam@path#1{\pgf@circ@bipole@path{cisourceAM}{#1}}
\def\pgf@circ@cisourcesin@path#1{\pgf@circ@bipole@path{cisourcesin}{#1}}
\def\pgf@circ@short@path#1{\pgf@circ@bipole@path{short}{#1}}
\def\pgf@circ@cspst@path#1{\pgf@circ@bipole@path{cspst}{#1}}
\def\pgf@circ@ospst@path#1{\pgf@circ@bipole@path{ospst}{#1}}
\def\pgf@circ@nos@path#1{\pgf@circ@bipole@path{nos}{#1}}
\def\pgf@circ@ncs@path#1{\pgf@circ@bipole@path{ncs}{#1}}
\def\pgf@circ@pushbutton@path#1{\pgf@circ@bipole@path{pushbutton}{#1}}
\def\pgf@circ@ncpushbutton@path#1{\pgf@circ@bipole@path{ncpushbutton}{#1}}
\def\pgf@circ@open@path#1{\pgf@circ@bipole@path{open}{#1}}
\def\pgf@circ@generic@path#1{\pgf@circ@bipole@path{generic}{#1}}
\def\pgf@circ@ageneric@path#1{\pgf@circ@bipole@path{ageneric}{#1}}
\def\pgf@circ@tgeneric@path#1{\pgf@circ@bipole@path{tgeneric}{#1}}
\def\pgf@circ@xgeneric@path#1{\pgf@circ@bipole@path{xgeneric}{#1}}
\def\pgf@circ@fullgeneric@path#1{\pgf@circ@bipole@path{fullgeneric}{#1}}
\def\pgf@circ@tfullgeneric@path#1{\pgf@circ@bipole@path{tfullgeneric}{#1}}
\def\pgf@circ@ammeter@path#1{\pgf@circ@bipole@path{ammeter}{#1}}
\def\pgf@circ@ohmmeter@path#1{\pgf@circ@bipole@path{ohmmeter}{#1}}
\def\pgf@circ@voltmeter@path#1{\pgf@circ@bipole@path{voltmeter}{#1}}
\def\pgf@circ@oscope@path#1{\pgf@circ@bipole@path{oscope}{#1}}
\def\pgf@circ@empty@path#1{}
\def\pgf@circ@photoresistor@path#1{\pgf@circ@bipole@path{photoresistor}{#1}}
\def\pgf@circ@emptythyristor@path#1{\pgf@circ@bipole@path{emptythyristor}{#1}}
\def\pgf@circ@fullthyristor@path#1{\pgf@circ@bipole@path{fullthyristor}{#1}}
\def\pgf@circ@toggleswitch@path#1{\pgf@circ@bipole@path{toggleswitch}{#1}}
\def\pgf@circ@memristor@path#1{\pgf@circ@bipole@path{memristor}{#1}}
\def\pgf@circ@emptytriac@path#1{\pgf@circ@bipole@path{emptytriac}{#1}}
\def\pgf@circ@fulltriac@path#1{\pgf@circ@bipole@path{fulltriac}{#1}}
\def\pgf@circ@tline@path#1{\pgf@circ@bipole@path{tline}{#1}}
\def\pgf@circ@mstline@path#1{\pgf@circ@bipole@path{mstline}{#1}}
\def\pgf@circ@squid@path#1{\pgf@circ@bipole@path{squid}{#1}}
\def\pgf@circ@barrier@path#1{\pgf@circ@bipole@path{barrier}{#1}}
\def\pgf@circ@openbarrier@path#1{\pgf@circ@bipole@path{openbarrier}{#1}}
\def\pgf@circ@thermocouple@path#1{\pgf@circ@bipole@path{thermocouple}{#1}}
\def\pgf@circ@fuse@path#1{\pgf@circ@bipole@path{fuse}{#1}}
\def\pgf@circ@afuse@path#1{\pgf@circ@bipole@path{afuse}{#1}}

\def\pgf@circ@gfsurgearrester@path#1{\ifpgf@circuit@europeangfsurgearrester\pgf@circ@europeangfsurgearrester@path{#1}\else\pgf@circ@americangfsurgearrester@path{#1}\fi}
\def\pgf@circ@europeangfsurgearrester@path#1{\pgf@circ@bipole@path{european gas filled surge arrester}{#1}}
\def\pgf@circ@americangfsurgearrester@path#1{\pgf@circ@bipole@path{american gas filled surge arrester}{#1}}

\def\pgf@circ@twoport@path#1{\pgf@circ@bipole@path{twoport}{#1}}
\def\pgf@circ@vco@path#1{\pgf@circ@bipole@path{vco}{#1}}
\def\pgf@circ@bandpass@path#1{\pgf@circ@bipole@path{bandpass}{#1}}
\def\pgf@circ@bandstop@path#1{\pgf@circ@bipole@path{bandstop}{#1}}
\def\pgf@circ@highpass@path#1{\pgf@circ@bipole@path{highpass}{#1}}
\def\pgf@circ@lowpass@path#1{\pgf@circ@bipole@path{lowpass}{#1}}
\def\pgf@circ@adc@path#1{\pgf@circ@bipole@path{adc}{#1}}
\def\pgf@circ@dac@path#1{\pgf@circ@bipole@path{dac}{#1}}
\def\pgf@circ@dsp@path#1{\pgf@circ@bipole@path{dsp}{#1}}
\def\pgf@circ@fft@path#1{\pgf@circ@bipole@path{fft}{#1}}
\def\pgf@circ@amp@path#1{\pgf@circ@bipole@path{amp}{#1}}
\def\pgf@circ@vamp@path#1{\pgf@circ@bipole@path{vamp}{#1}}
\def\pgf@circ@piattenuator@path#1{\pgf@circ@bipole@path{piattenuator}{#1}}
\def\pgf@circ@vpiattenuator@path#1{\pgf@circ@bipole@path{vpiattenuator}{#1}}
\def\pgf@circ@tattenuator@path#1{\pgf@circ@bipole@path{tattenuator}{#1}}
\def\pgf@circ@vtattenuator@path#1{\pgf@circ@bipole@path{vtattenuator}{#1}}
\def\pgf@circ@phaseshifter@path#1{\pgf@circ@bipole@path{phaseshifter}{#1}}
\def\pgf@circ@vphaseshifter@path#1{\pgf@circ@bipole@path{vphaseshifter}{#1}}
\def\pgf@circ@detector@path#1{\pgf@circ@bipole@path{detector}{#1}}
%
\def\pgf@circ@sacdc@path#1{\pgf@circ@bipole@path{sacdc}{#1}}
\def\pgf@circ@sdcac@path#1{\pgf@circ@bipole@path{sdcac}{#1}}
\def\pgf@circ@tacdc@path#1{\pgf@circ@bipole@path{tacdc}{#1}}
\def\pgf@circ@tdcac@path#1{\pgf@circ@bipole@path{tdcac}{#1}}

%%Mechanical
\def\pgf@circ@spring@path#1{\pgf@circ@bipole@path{spring}{#1}}
\def\pgf@circ@inerter@path#1{\pgf@circ@bipole@path{inerter}{#1}}
\def\pgf@circ@mass@path#1{\pgf@circ@bipole@path{mass}{#1}}
\def\pgf@circ@damper@path#1{\pgf@circ@bipole@path{damper}{#1}}
\def\pgf@circ@viscoe@path#1{\pgf@circ@bipole@path{viscoe}{#1}}

%% Styles

\def\comnpatname{\ifpgf@circuit@compat *\else\fi}
\def\compattikzset#1{\tikzset{\comnpatname#1}}

%\def\ctikzsetbipole#1#2{%
%	\tikzset{#1/.style= {to path=#2, \circuitikzbasekey, l=##1}}%
%}

\compattikzset{spring/.style = {\circuitikzbasekey, /tikz/to path=\pgf@circ@spring@path, l=#1}}
\compattikzset{inerter/.style = {\circuitikzbasekey, /tikz/to path=\pgf@circ@inerter@path, l=#1}}
\compattikzset{mass/.style = {\circuitikzbasekey, /tikz/to path=\pgf@circ@mass@path, l=#1}}
\compattikzset{damper/.style = {\circuitikzbasekey, /tikz/to path=\pgf@circ@damper@path, l=#1}}
\compattikzset{viscoe/.style = {\circuitikzbasekey, /tikz/to path=\pgf@circ@viscoe@path, l=#1}}
\compattikzset{resistor/.style= {\circuitikzbasekey, /tikz/to path=\pgf@circ@resistor@path, l=#1}}
\compattikzset{american resistor/.style= {\circuitikzbasekey, /tikz/to path=\pgf@circ@bipole@path{resistor}{#1}, l=#1}}
\compattikzset{european resistor/.style= {\circuitikzbasekey, /tikz/to path=\pgf@circ@bipole@path{generic}{#1}, l=#1}}
\compattikzset{potentiometer/.style = {\circuitikzbasekey, /tikz/to path=\pgf@circ@potentiometer@path, l=#1}}
\compattikzset{varistor/.style = {\circuitikzbasekey, /tikz/to path=\pgf@circ@varistor@path, l=#1}}
\compattikzset{photoresistor/.style = {\circuitikzbasekey, /tikz/to path=\pgf@circ@photoresistor@path, l=#1}}
\compattikzset{thermistor/.style = {\circuitikzbasekey, /tikz/to path=\pgf@circ@thermistor@path, l=#1}}
\compattikzset{thermistor ptc/.style = {\circuitikzbasekey, /tikz/to path=\pgf@circ@thermistorptc@path, l=#1}}
\compattikzset{thermistor ntc/.style = {\circuitikzbasekey, /tikz/to path=\pgf@circ@thermistorntc@path, l=#1}}
\compattikzset{american potentiometer/.style = {\circuitikzbasekey, /tikz/to path=\pgf@circ@bipole@path{potentiometer}{#1}, l=#1}}
\compattikzset{european potentiometer/.style = {\circuitikzbasekey, /tikz/to path=\pgf@circ@bipole@path{genericpotentiometer}{#1}, l=#1}}
\compattikzset{variable resistor/.style = {\circuitikzbasekey, /tikz/to path=\pgf@circ@vresistor@path, l=#1}}
\compattikzset{variable american resistor/.style= {\circuitikzbasekey, /tikz/to path=\pgf@circ@bipole@path{vresistor}{#1}, l=#1}}
\compattikzset{variable european resistor/.style= {\circuitikzbasekey, /tikz/to path=\pgf@circ@bipole@path{tgeneric}{#1}, l=#1}}
\compattikzset{capacitor/.style = {\circuitikzbasekey, /tikz/to path=\pgf@circ@capacitor@path, l=#1}}
\compattikzset{elko/.style = {\circuitikzbasekey, /tikz/to path=\pgf@circ@ecapacitor@path, l=#1}}
\compattikzset{ecapacitor/.style = {\circuitikzbasekey, /tikz/to path=\pgf@circ@ecapacitor@path, l=#1}}
% polar capacitor is deprecated, use curved capacitor instead
\compattikzset{polar capacitor/.style = {\circuitikzbasekey, /tikz/to path=\pgf@circ@polarcapacitor@path, l=#1}}
\compattikzset{curved capacitor/.style = {\circuitikzbasekey, /tikz/to path=\pgf@circ@ccapacitor@path, l=#1}}
\compattikzset{variable capacitor/.style = {\circuitikzbasekey, /tikz/to path=\pgf@circ@vcapacitor@path, l=#1}}
\compattikzset{piezoelectric/.style = {\circuitikzbasekey, /tikz/to path=\pgf@circ@piezoelectric@path, l=#1}}
\compattikzset{battery/.style = {\circuitikzbasekey, /tikz/to path=\pgf@circ@battery@path, \circuitikzbasekey/bipole/is voltage=true,  \circuitikzbasekey/bipole/is voltageoutsideofsymbol=true, v=#1}}
\compattikzset{battery1/.style = {\circuitikzbasekey, /tikz/to path=\pgf@circ@batteryone@path, \circuitikzbasekey/bipole/is voltage=true,  \circuitikzbasekey/bipole/is voltageoutsideofsymbol=true, v=#1}}
\compattikzset{battery2/.style = {\circuitikzbasekey, /tikz/to path=\pgf@circ@batterytwo@path, \circuitikzbasekey/bipole/is voltage=true,  \circuitikzbasekey/bipole/is voltageoutsideofsymbol=true, v=#1}}
\compattikzset{inductor/.style = {\circuitikzbasekey, /tikz/to path=\pgf@circ@inductor@path, l=#1}}
\compattikzset{gf surge arrester/.style = {\circuitikzbasekey, /tikz/to path=\pgf@circ@gfsurgearrester@path, l=#1}}
\compattikzset{american gas filled surge arrester/.style = {\circuitikzbasekey, /tikz/to path=\pgf@circ@americangfsurgearrester@path, l=#1}}
\compattikzset{european gas filled surge arrester/.style = {\circuitikzbasekey, /tikz/to path=\pgf@circ@europeangfsurgearrester@path, l=#1}}
\compattikzset{gas filled surge arrester/.style = {\circuitikzbasekey, /tikz/to path=\pgf@circ@gfsurgearrester@path, l=#1}}
\compattikzset{american inductor/.style = {\circuitikzbasekey, /tikz/to path=\pgf@circ@americaninductor@path, l=#1}}
\compattikzset{cute inductor/.style = {\circuitikzbasekey, /tikz/to path=\pgf@circ@cuteinductor@path, l=#1}}
\compattikzset{cute choke/.style = {\circuitikzbasekey, /tikz/to path=\pgf@circ@cutechoke@path, l=#1}}
\compattikzset{european inductor/.style = {\circuitikzbasekey, /tikz/to path=\pgf@circ@europeaninductor@path, l=#1}}
\compattikzset{variable inductor/.style = {\circuitikzbasekey, /tikz/to path=\pgf@circ@vinductor@path, l=#1}}
\compattikzset{variable european inductor/.style = {\circuitikzbasekey, /tikz/to path=\pgf@circ@veuropeaninductor@path, l=#1}}
\compattikzset{variable american inductor/.style = {\circuitikzbasekey, /tikz/to path=\pgf@circ@vamericaninductor@path, l=#1}}
\compattikzset{variable cute inductor/.style = {\circuitikzbasekey, /tikz/to path=\pgf@circ@vcuteinductor@path, l=#1}}
\compattikzset{tline/.style = {\circuitikzbasekey, /tikz/to path=\pgf@circ@tline@path, l=#1}}
\compattikzset{transmission line/.style = {tline = #1}}
\compattikzset{TL/.style = {tline = #1}}
\compattikzset{mstline/.style = {\circuitikzbasekey, /tikz/to path=\pgf@circ@mstline@path, l=#1}}
\compattikzset{european voltage source/.style = {\circuitikzbasekey, /tikz/to path=\pgf@circ@vsource@path, \circuitikzbasekey/bipole/is voltage=true, \circuitikzbasekey/bipole/is voltageoutsideofsymbol=true, v=#1}}
\compattikzset{american voltage source/.style = {\circuitikzbasekey, /tikz/to path=\pgf@circ@vsourceam@path, \circuitikzbasekey/bipole/is voltage=true, v=#1}}
\compattikzset{european current source/.style = {\circuitikzbasekey, /tikz/to path=\pgf@circ@isource@path, \circuitikzbasekey/bipole/is current=true, i=#1}}
\compattikzset{american current source/.style = {\circuitikzbasekey, /tikz/to path=\pgf@circ@isourceam@path, \circuitikzbasekey/bipole/is current=true, i=#1}}
\compattikzset{empty controlled source/.style = {\circuitikzbasekey, /tikz/to path=\pgf@circ@ecsource@path, \circuitikzbasekey/bipole/is voltage=true, \circuitikzbasekey/bipole/is voltageoutsideofsymbol=true, v=#1}}
\compattikzset{european controlled voltage source/.style = {\circuitikzbasekey, /tikz/to path=\pgf@circ@cvsource@path, \circuitikzbasekey/bipole/is voltage=true, \circuitikzbasekey/bipole/is voltageoutsideofsymbol=true, v=#1}}
\compattikzset{american controlled voltage source/.style = {\circuitikzbasekey, /tikz/to path=\pgf@circ@cvsourceam@path, \circuitikzbasekey/bipole/is voltage=true, v=#1}}
\compattikzset{european controlled current source/.style = {\circuitikzbasekey, /tikz/to path=\pgf@circ@cisource@path, \circuitikzbasekey/bipole/is current=true, i=#1}}
\compattikzset{american controlled current source/.style = {\circuitikzbasekey, /tikz/to path=\pgf@circ@cisourceam@path, \circuitikzbasekey/bipole/is current=true, i=#1}}
\compattikzset{sinusoidal voltage source/.style = {\circuitikzbasekey, /tikz/to path=\pgf@circ@vsourcesin@path, \circuitikzbasekey/bipole/is voltage=true, \circuitikzbasekey/bipole/is voltageoutsideofsymbol=true, v=#1 }}
\compattikzset{square voltage source/.style = {\circuitikzbasekey, /tikz/to path=\pgf@circ@vsourcesquare@path, \circuitikzbasekey/bipole/is voltage=true, \circuitikzbasekey/bipole/is voltageoutsideofsymbol=true, v=#1 }}
\compattikzset{triangle voltage source/.style = {\circuitikzbasekey, /tikz/to path=\pgf@circ@vsourcetri@path, \circuitikzbasekey/bipole/is voltage=true, \circuitikzbasekey/bipole/is voltageoutsideofsymbol=true, v=#1 }}
\compattikzset{sinusoidal current source/.style = {\circuitikzbasekey, /tikz/to path=\pgf@circ@isourcesin@path, \circuitikzbasekey/bipole/is current=true, i=#1}}
\compattikzset{controlled sinusoidal voltage source/.style = {\circuitikzbasekey, /tikz/to path=\pgf@circ@cvsourcesin@path, \circuitikzbasekey/bipole/is voltage=true, \circuitikzbasekey/bipole/is voltageoutsideofsymbol=true, v=#1}}
\compattikzset{controlled sinusoidal current source/.style = {\circuitikzbasekey, /tikz/to path=\pgf@circ@cisourcesin@path, \circuitikzbasekey/bipole/is current=true, i=#1}}

\compattikzset{voltage source/.style = {\comnpatname \ifpgf@circuit@europeanvoltage european \else american \fi voltage source, \circuitikzbasekey/bipole/is voltage=true, v=#1}}
\compattikzset{current source/.style = {\comnpatname \ifpgf@circuit@europeancurrent european \else american \fi current source = #1, \circuitikzbasekey/bipole/is current=true}}
\compattikzset{controlled voltage source/.style = {\comnpatname \ifpgf@circuit@europeanvoltage european \else american \fi controlled voltage source, \circuitikzbasekey/bipole/is voltage=true, v=#1}}
\compattikzset{controlled current source/.style = {\comnpatname \ifpgf@circuit@europeancurrent european \else american \fi controlled current source = #1, \circuitikzbasekey/bipole/is current=true}}


\compattikzset{generic/.style = {\circuitikzbasekey, /tikz/to path=\pgf@circ@generic@path, l=#1}}
\compattikzset{xgeneric/.style = {\circuitikzbasekey, /tikz/to path=\pgf@circ@xgeneric@path, l=#1}}
\compattikzset{ageneric/.style = {\circuitikzbasekey, /tikz/to path=\pgf@circ@ageneric@path, l=#1}}
\compattikzset{tgeneric/.style = {\circuitikzbasekey, /tikz/to path=\pgf@circ@tgeneric@path, l=#1}}
\compattikzset{fullgeneric/.style = {\circuitikzbasekey, /tikz/to path=\pgf@circ@fullgeneric@path, l=#1}}
\compattikzset{tfullgeneric/.style = {\circuitikzbasekey, /tikz/to path=\pgf@circ@tfullgeneric@path, l=#1}}
\compattikzset{short/.style = {\circuitikzbasekey, /tikz/to path=\pgf@circ@short@path}}
\compattikzset{open/.style = {\circuitikzbasekey, /tikz/to path=\pgf@circ@open@path}}

\compattikzset{lamp/.style = {\circuitikzbasekey, /tikz/to path=\pgf@circ@lamp@path}}
\compattikzset{bulb/.style = {\circuitikzbasekey, /tikz/to path=\pgf@circ@bulb@path, l=#1}}

\compattikzset{squid/.style = {\circuitikzbasekey, /tikz/to path=\pgf@circ@squid@path}}
\compattikzset{barrier/.style = {\circuitikzbasekey, /tikz/to path=\pgf@circ@barrier@path}}
\compattikzset{openbarrier/.style = {\circuitikzbasekey, /tikz/to path=\pgf@circ@openbarrier@path}}
\compattikzset{thermocouple/.style = {\circuitikzbasekey, /tikz/to path=\pgf@circ@thermocouple@path}}
\compattikzset{fuse/.style = {\circuitikzbasekey, /tikz/to path=\pgf@circ@fuse@path}}
\compattikzset{asymmetric fuse/.style = {\circuitikzbasekey, /tikz/to path=\pgf@circ@afuse@path}}

\compattikzset{twoport/.style = {\circuitikzbasekey, /tikz/to path=\pgf@circ@twoport@path}}
\compattikzset{vco/.style = {\circuitikzbasekey, /tikz/to path=\pgf@circ@vco@path}}
\compattikzset{bandpass/.style = {\circuitikzbasekey, /tikz/to path=\pgf@circ@bandpass@path}}
\compattikzset{bandstop/.style = {\circuitikzbasekey, /tikz/to path=\pgf@circ@bandstop@path}}
\compattikzset{highpass/.style = {\circuitikzbasekey, /tikz/to path=\pgf@circ@highpass@path}}
\compattikzset{lowpass/.style = {\circuitikzbasekey, /tikz/to path=\pgf@circ@lowpass@path}}
\compattikzset{adc/.style = {\circuitikzbasekey, /tikz/to path=\pgf@circ@adc@path}}
\compattikzset{dac/.style = {\circuitikzbasekey, /tikz/to path=\pgf@circ@dac@path}}
\compattikzset{dsp/.style = {\circuitikzbasekey, /tikz/to path=\pgf@circ@dsp@path}}
\compattikzset{fft/.style = {\circuitikzbasekey, /tikz/to path=\pgf@circ@fft@path}}
\compattikzset{amp/.style = {\circuitikzbasekey, /tikz/to path=\pgf@circ@amp@path}}
\compattikzset{vamp/.style = {\circuitikzbasekey, /tikz/to path=\pgf@circ@vamp@path}}
\compattikzset{piattenuator/.style = {\circuitikzbasekey, /tikz/to path=\pgf@circ@piattenuator@path}}
\compattikzset{vpiattenuator/.style = {\circuitikzbasekey, /tikz/to path=\pgf@circ@vpiattenuator@path}}
\compattikzset{tattenuator/.style = {\circuitikzbasekey, /tikz/to path=\pgf@circ@tattenuator@path}}
\compattikzset{vtattenuator/.style = {\circuitikzbasekey, /tikz/to path=\pgf@circ@vtattenuator@path}}
\compattikzset{phaseshifter/.style = {\circuitikzbasekey, /tikz/to path=\pgf@circ@phaseshifter@path}}
\compattikzset{vphaseshifter/.style = {\circuitikzbasekey, /tikz/to path=\pgf@circ@vphaseshifter@path}}
\compattikzset{detector/.style = {\circuitikzbasekey, /tikz/to path=\pgf@circ@detector@path}}
%
\compattikzset{sacdc/.style = {\circuitikzbasekey, /tikz/to path=\pgf@circ@sacdc@path, l=#1}}
\compattikzset{sdcac/.style = {\circuitikzbasekey, /tikz/to path=\pgf@circ@sdcac@path, l=#1}}
\compattikzset{tacdc/.style = {\circuitikzbasekey, /tikz/to path=\pgf@circ@tacdc@path, l=#1}}
\compattikzset{tdcac/.style = {\circuitikzbasekey, /tikz/to path=\pgf@circ@tdcac@path, l=#1}}

% % % % % %
% % Begin of Diodes
% % % % % %

\def\pgf@circ@fulldiode@path#1{\pgf@circ@bipole@path{fulldiode}{#1}}
\def\pgf@circ@fullzdiode@path#1{\pgf@circ@bipole@path{fullzdiode}{#1}}
\def\pgf@circ@fullzzdiode@path#1{\pgf@circ@bipole@path{fullzzdiode}{#1}}
\def\pgf@circ@fullsdiode@path#1{\pgf@circ@bipole@path{fullsdiode}{#1}}
\def\pgf@circ@fulltdiode@path#1{\pgf@circ@bipole@path{fulltdiode}{#1}}
\def\pgf@circ@fulllediode@path#1{\pgf@circ@bipole@path{fulllediode}{#1}}
\def\pgf@circ@fullpdiode@path#1{\pgf@circ@bipole@path{fullpdiode}{#1}}
\def\pgf@circ@fullvarcap@path#1{\pgf@circ@bipole@path{fullvarcap}{#1}}
\def\pgf@circ@fullbidirectionaldiode@path#1{\pgf@circ@bipole@path{fullbidirectionaldiode}{#1}}
\def\pgf@circ@emptydiode@path#1{\pgf@circ@bipole@path{emptydiode}{#1}}
\def\pgf@circ@emptyzdiode@path#1{\pgf@circ@bipole@path{emptyzdiode}{#1}}
\def\pgf@circ@emptyzzdiode@path#1{\pgf@circ@bipole@path{emptyzzdiode}{#1}}
\def\pgf@circ@emptysdiode@path#1{\pgf@circ@bipole@path{emptysdiode}{#1}}
\def\pgf@circ@emptytdiode@path#1{\pgf@circ@bipole@path{emptytdiode}{#1}}
\def\pgf@circ@emptylediode@path#1{\pgf@circ@bipole@path{emptylediode}{#1}}
\def\pgf@circ@emptypdiode@path#1{\pgf@circ@bipole@path{emptypdiode}{#1}}
\def\pgf@circ@emptyvarcap@path#1{\pgf@circ@bipole@path{emptyvarcap}{#1}}
\def\pgf@circ@emptybidirectionaldiode@path#1{\pgf@circ@bipole@path{emptybidirectionaldiode}{#1}}

\compattikzset{full diode/.style = {\circuitikzbasekey, /tikz/to path=\pgf@circ@fulldiode@path}}
\compattikzset{full Schottky diode/.style = {\circuitikzbasekey, /tikz/to path=\pgf@circ@fullsdiode@path}}
\compattikzset{full Zener diode/.style = {\circuitikzbasekey, /tikz/to path=\pgf@circ@fullzdiode@path}}
\compattikzset{full ZZener diode/.style = {\circuitikzbasekey, /tikz/to path=\pgf@circ@fullzzdiode@path}}
\compattikzset{full tunnel diode/.style = {\circuitikzbasekey, /tikz/to path=\pgf@circ@fulltdiode@path}}
\compattikzset{full photodiode/.style = {\circuitikzbasekey, /tikz/to path=\pgf@circ@fullpdiode@path}}
\compattikzset{full led/.style = {\circuitikzbasekey, /tikz/to path=\pgf@circ@fulllediode@path}}
\compattikzset{full varcap/.style = {\circuitikzbasekey, /tikz/to path=\pgf@circ@fullvarcap@path}}
\compattikzset{full bidirectionaldiode/.style = {\circuitikzbasekey, /tikz/to path=\pgf@circ@fullbidirectionaldiode@path}}
\compattikzset{full thyristor/.style =  {\circuitikzbasekey, /tikz/to path=\pgf@circ@fullthyristor@path}}
\compattikzset{full triac/.style =  {\circuitikzbasekey, /tikz/to path=\pgf@circ@fulltriac@path}}

\compattikzset{empty diode/.style = {\circuitikzbasekey, /tikz/to path=\pgf@circ@emptydiode@path}}
\compattikzset{empty Schottky diode/.style = {\circuitikzbasekey, /tikz/to path=\pgf@circ@emptysdiode@path}}
\compattikzset{empty Zener diode/.style = {\circuitikzbasekey, /tikz/to path=\pgf@circ@emptyzdiode@path}}
\compattikzset{empty ZZener diode/.style = {\circuitikzbasekey, /tikz/to path=\pgf@circ@emptyzzdiode@path}}
\compattikzset{empty tunnel diode/.style = {\circuitikzbasekey, /tikz/to path=\pgf@circ@emptytdiode@path}}
\compattikzset{empty photodiode/.style = {\circuitikzbasekey, /tikz/to path=\pgf@circ@emptypdiode@path}}
\compattikzset{empty led/.style = {\circuitikzbasekey, /tikz/to path=\pgf@circ@emptylediode@path}}
\compattikzset{empty varcap/.style = {\circuitikzbasekey, /tikz/to path=\pgf@circ@emptyvarcap@path}}
\compattikzset{empty bidirectionaldiode/.style = {\circuitikzbasekey, /tikz/to path=\pgf@circ@emptybidirectionaldiode@path}}
\compattikzset{empty thyristor/.style =  {\circuitikzbasekey, /tikz/to path=\pgf@circ@emptythyristor@path}}
\compattikzset{empty triac/.style =  {\circuitikzbasekey, /tikz/to path=\pgf@circ@emptytriac@path}}

\compattikzset{stroke diode/.style = {\circuitikzbasekey,\circuitikzbasekey/bipole/is strokedsymbol=true, /tikz/to path=\pgf@circ@emptydiode@path}}
\compattikzset{stroke Schottky diode/.style = {\circuitikzbasekey,\circuitikzbasekey/bipole/is strokedsymbol=true, /tikz/to path=\pgf@circ@emptysdiode@path}}
\compattikzset{stroke Zener diode/.style = {\circuitikzbasekey,\circuitikzbasekey/bipole/is strokedsymbol=true, /tikz/to path=\pgf@circ@emptyzdiode@path}}
\compattikzset{stroke ZZener diode/.style = {\circuitikzbasekey,\circuitikzbasekey/bipole/is strokedsymbol=true, /tikz/to path=\pgf@circ@emptyzzdiode@path}}
\compattikzset{stroke tunnel diode/.style = {\circuitikzbasekey,\circuitikzbasekey/bipole/is strokedsymbol=true, /tikz/to path=\pgf@circ@emptytdiode@path}}
\compattikzset{stroke photodiode/.style = {\circuitikzbasekey,\circuitikzbasekey/bipole/is strokedsymbol=true, /tikz/to path=\pgf@circ@emptypdiode@path}}
\compattikzset{stroke led/.style = {\circuitikzbasekey,\circuitikzbasekey/bipole/is strokedsymbol=true, /tikz/to path=\pgf@circ@emptylediode@path}}
\compattikzset{stroke varcap/.style = {\circuitikzbasekey,\circuitikzbasekey/bipole/is strokedsymbol=true, /tikz/to path=\pgf@circ@emptyvarcap@path}}
\compattikzset{stroke thyristor/.style =  {\circuitikzbasekey,\circuitikzbasekey/bipole/is strokedsymbol=true, /tikz/to path=\pgf@circ@emptythyristor@path}}
%\compattikzset{stroke triac/.style =  {\circuitikzbasekey,\circuitikzbasekey/bipole/is strokedsymbol=true, /tikz/to path=\pgf@circ@emptytriac@path}}

\def\pgfcircdiodestylemacro{\ifpgf@circuit@strokediode stroke \else\ifpgf@circuit@fulldiode full \else empty \fi\fi}

\compattikzset{Schottky diode/.style = {\comnpatname \pgfcircdiodestylemacro Schottky diode}}
\compattikzset{Zener diode/.style = {\comnpatname \pgfcircdiodestylemacro Zener diode}}
\compattikzset{ZZener diode/.style = {\comnpatname \pgfcircdiodestylemacro ZZener diode}}
\compattikzset{tunnel diode/.style = {\comnpatname \pgfcircdiodestylemacro tunnel diode}}
\compattikzset{photodiode/.style = {\comnpatname \pgfcircdiodestylemacro photodiode}}
\compattikzset{led/.style = {\comnpatname \pgfcircdiodestylemacro led}}
\compattikzset{varcap/.style = {\comnpatname \pgfcircdiodestylemacro varcap}}
\compattikzset{diode/.style = {\comnpatname \pgfcircdiodestylemacro diode}}
\compattikzset{thyristor/.style =  {\comnpatname \pgfcircdiodestylemacro thyristor}}
\compattikzset{triac/.style =  {\comnpatname \ifpgf@circuit@fulldiode full \else empty \fi triac}}%no stroke triac!
\compattikzset{bidirectionaldiode/.style =  {\comnpatname \ifpgf@circuit@fulldiode full \else empty \fi bidirectionaldiode}}%no stroke bidirectionaldiode! (based on triac)

%% Define Shortcuts
\compattikzset{Do/.style = {\comnpatname empty diode}}
\compattikzset{tDo/.style = {\comnpatname empty tunnel diode}}
\compattikzset{zDo/.style = {\comnpatname empty Zener diode}}
\compattikzset{zzDo/.style = {\comnpatname empty ZZener diode}}
\compattikzset{sDo/.style = {\comnpatname empty Schottky diode}}
\compattikzset{pDo/.style = {\comnpatname empty photodiode}}
\compattikzset{leDo/.style = {\comnpatname empty led}}
\compattikzset{VCo/.style = {\comnpatname empty varcap}}
\compattikzset{biDo/.style = {\comnpatname empty bidirectionaldiode}}
\compattikzset{Tyo/.style = {\comnpatname empty thyristor}}
\compattikzset{Tro/.style = {\comnpatname empty triac}}

\compattikzset{D*/.style = {\comnpatname full diode}}
\compattikzset{tD*/.style = {\comnpatname full tunnel diode}}
\compattikzset{zD*/.style = {\comnpatname full Zener diode}}
\compattikzset{zzD*/.style = {\comnpatname full ZZener diode}}
\compattikzset{sD*/.style = {\comnpatname full Schottky diode}}
\compattikzset{pD*/.style = {\comnpatname full photodiode}}
\compattikzset{leD*/.style = {\comnpatname full led}}
\compattikzset{VC*/.style = {\comnpatname full varcap}}
\compattikzset{biD*/.style = {\comnpatname full bidirectionaldiode}}
\compattikzset{Ty*/.style = {\comnpatname full thyristor}}
\compattikzset{Tr*/.style = {\comnpatname full triac}}

\compattikzset{D/.style = {\comnpatname diode}}
\compattikzset{tD/.style = {\comnpatname tunnel diode}}
\compattikzset{zD/.style = {\comnpatname Zener diode}}
\compattikzset{zzD/.style = {\comnpatname ZZener diode}}
\compattikzset{sD/.style = {\comnpatname Schottky diode}}
\compattikzset{pD/.style = {\comnpatname photodiode}}
\compattikzset{leD/.style = {\comnpatname led}}
\compattikzset{VC/.style = {\comnpatname varcap}}
\compattikzset{biD/.style = {\comnpatname bidirectionaldiode}}
\compattikzset{Ty/.style = {\comnpatname thyristor}}
\compattikzset{Tr/.style = {\comnpatname triac}}

\compattikzset{D-/.style = {\comnpatname stroke diode}}
\compattikzset{tD-/.style = {\comnpatname stroke tunnel diode}}
\compattikzset{zD-/.style = {\comnpatname stroke Zener diode}}
\compattikzset{zzD-/.style = {\comnpatname stroke ZZener diode}}
\compattikzset{sD-/.style = {\comnpatname stroke Schottky diode}}
\compattikzset{pD-/.style = {\comnpatname stroke photodiode}}
\compattikzset{leD-/.style = {\comnpatname stroke led}}
\compattikzset{VC-/.style = {\comnpatname stroke varcap}}
\compattikzset{Ty-/.style = {\comnpatname stroke thyristor}}
\compattikzset{Tr-/.style = {\comnpatname empty triac}}%no stroke triac!
\compattikzset{biD-/.style = {\comnpatname empty bidirectionaldiode}}%no stroke bidirectionaldiode! (based on triac)

% % % % % %
% % End of Diodes
% % % % % %


\compattikzset{memristor/.style =  {\circuitikzbasekey, /tikz/to path=\pgf@circ@memristor@path}}
\compattikzset{closing switch/.style = {\circuitikzbasekey, /tikz/to path=\pgf@circ@cspst@path, l=#1}}
\compattikzset{opening switch/.style = {\circuitikzbasekey, /tikz/to path=\pgf@circ@ospst@path, l=#1}}
\compattikzset{ncs/.style = {\circuitikzbasekey, /tikz/to path=\pgf@circ@ncs@path, l=#1}}
\compattikzset{nos/.style = {\circuitikzbasekey, /tikz/to path=\pgf@circ@nos@path, l=#1}}
\compattikzset{normal closed switch/.style = {\circuitikzbasekey, /tikz/to path=\pgf@circ@ncs@path, l=#1}}
\compattikzset{normal open switch/.style = {\circuitikzbasekey, /tikz/to path=\pgf@circ@nos@path, l=#1}}
\compattikzset{switch/.style = {\circuitikzbasekey, /tikz/to path=\pgf@circ@cspst@path, l=#1}}
\compattikzset{push button/.style = {\circuitikzbasekey, /tikz/to path=\pgf@circ@pushbutton@path, l=#1}}
\compattikzset{nopb/.style = {\circuitikzbasekey, /tikz/to path=\pgf@circ@pushbutton@path, l=#1}}
\compattikzset{normally open push button/.style = {\circuitikzbasekey, /tikz/to path=\pgf@circ@pushbutton@path, l=#1}}
\compattikzset{ncpb/.style = {\circuitikzbasekey, /tikz/to path=\pgf@circ@ncpushbutton@path, l=#1}}
\compattikzset{normally closed push button/.style = {\circuitikzbasekey, /tikz/to path=\pgf@circ@ncpushbutton@path, l=#1}}
\compattikzset{toggle switch/.style =  {\circuitikzbasekey, /tikz/to path=\pgf@circ@toggleswitch@path}}

\compattikzset{ammeter/.style = {\circuitikzbasekey, /tikz/to path=\pgf@circ@ammeter@path}}
\compattikzset{voltmeter/.style = {\circuitikzbasekey, /tikz/to path=\pgf@circ@voltmeter@path}}
\compattikzset{ohmmeter/.style = {\circuitikzbasekey, /tikz/to path=\pgf@circ@ohmmeter@path}}
\tikzset{oscope/.style = {\circuitikzbasekey, /tikz/to path=\pgf@circ@oscope@path, l=#1}}

% cute switches
%% closed cute switch
\def\pgf@circ@cuteclosedswitch@path#1{\pgf@circ@bipole@path{cuteclosedswitch}{#1}}
\compattikzset{cute closed switch/.style = {\circuitikzbasekey, /tikz/to path=\pgf@circ@cuteclosedswitch@path, l=#1}}
\compattikzset{ccsw/.style= {\comnpatname cute closed switch= #1}}

%% open cute switch
\def\pgf@circ@cuteopenswitch@path#1{\pgf@circ@bipole@path{cuteopenswitch}{#1}}
\compattikzset{cute open switch/.style = {\circuitikzbasekey, /tikz/to path=\pgf@circ@cuteopenswitch@path, l=#1}}
\compattikzset{cosw/.style= {\comnpatname cute open switch= #1}}

%% closing cute switch
\def\pgf@circ@cuteclosingswitch@path#1{\pgf@circ@bipole@path{cuteclosingswitch}{#1}}
\compattikzset{cute closing switch/.style = {\circuitikzbasekey, /tikz/to path=\pgf@circ@cuteclosingswitch@path, l=#1}}
\compattikzset{ccgsw/.style= {\comnpatname cute closing switch= #1}}

%% opening cute switch
\def\pgf@circ@cuteopeningswitch@path#1{\pgf@circ@bipole@path{cuteopeningswitch}{#1}}
\compattikzset{cute opening switch/.style = {\circuitikzbasekey, /tikz/to path=\pgf@circ@cuteopeningswitch@path, l=#1}}
\compattikzset{cogsw/.style= {\comnpatname cute opening switch= #1}}


% short forms
\compattikzset{esource/.style = {\circuitikzbasekey, /tikz/to path=\pgf@circ@esource@path, \circuitikzbasekey/bipole/is voltage=true,\circuitikzbasekey/bipole/is voltageoutsideofsymbol=true,  v=#1}}
\compattikzset{pvsource/.style = {\circuitikzbasekey, /tikz/to path=\pgf@circ@pvsource@path, \circuitikzbasekey/bipole/is voltage=true,\circuitikzbasekey/bipole/is voltageoutsideofsymbol=true,  v=#1}}
\compattikzset{dcvsource/.style = {\circuitikzbasekey, /tikz/to path=\pgf@circ@dcvsource@path, \circuitikzbasekey/bipole/is voltage=true,\circuitikzbasekey/bipole/is voltageoutsideofsymbol=true, v=#1}}
\compattikzset{dcisource/.style = {\circuitikzbasekey, /tikz/to path=\pgf@circ@dcisource@path, \circuitikzbasekey/bipole/is current=true, l=#1}}

\compattikzset{ioosource/.style = {\circuitikzbasekey, \circuitikzbasekey/bipole/is current=true,/tikz/to path=\pgf@circ@oosource@path, i=#1}}
\compattikzset{voosource/.style = {\circuitikzbasekey, \circuitikzbasekey/bipole/is voltage=true,/tikz/to path=\pgf@circ@oosource@path, v=#1}}
\compattikzset{oosourcetrans/.style = {\circuitikzbasekey, \circuitikzbasekey/bipole/is voltage=true,/tikz/to path=\pgf@circ@oosourcetrans@path, v=#1}}
\compattikzset{ooosource/.style = {\circuitikzbasekey, \circuitikzbasekey/bipole/is voltage=true,/tikz/to path=\pgf@circ@ooosource@path, v=#1}}

\compattikzset{vsource/.style = {\comnpatname voltage source = #1}}
\compattikzset{isource/.style = {\comnpatname current source = #1}}
\compattikzset{cisource/.style = {\comnpatname controlled current source = #1}}
\compattikzset{ecsource/.style = {\comnpatname empty controlled source = #1}}
\compattikzset{cvsource/.style = {\comnpatname controlled voltage source = #1}}
\compattikzset{vsourcesin/.style = {\comnpatname sinusoidal voltage source = #1}}
\compattikzset{vsourcesquare/.style = {\comnpatname square voltage source = #1}}
\compattikzset{vsourcetri/.style = {triangle voltage source = #1}}
\compattikzset{isourcesin/.style = {\comnpatname sinusoidal current source = #1}}
\compattikzset{cisourcesin/.style = {\comnpatname controlled sinusoidal current source = #1}}
\compattikzset{cvsourcesin/.style = {\comnpatname controlled sinusoidal  voltage source = #1}}

\compattikzset{controlled vsource/.style = {\comnpatname controlled voltage source = #1}}
\compattikzset{controlled isource/.style = {\comnpatname controlled current source = #1}}
\compattikzset{controlled vsourcesin/.style = {\comnpatname controlled sinusoidal  voltage source = #1}}
\compattikzset{controlled isourcesin/.style = {\comnpatname controlled sinusoidal current source = #1}}

\compattikzset{R/.style= {\comnpatname resistor = #1}}
\compattikzset{vR/.style= {\comnpatname variable resistor = #1}}
\compattikzset{phR/.style= {\comnpatname photoresistor = #1}}
\compattikzset{thR/.style= {\comnpatname thermistor = #1}}
\compattikzset{thRp/.style= {\comnpatname thermistor ptc= #1}}
\compattikzset{thRn/.style= {\comnpatname thermistor ntc= #1}}
\compattikzset{pR/.style= {\comnpatname potentiometer = #1}}
\compattikzset{C/.style = {\comnpatname capacitor = #1}}
\compattikzset{eC/.style = {\comnpatname ecapacitor = #1}}
%% pC is deprecated, use cC instead
\compattikzset{pC/.style = {\comnpatname polar capacitor = #1}}
\compattikzset{cC/.style = {\comnpatname curved capacitor = #1}}
\compattikzset{vC/.style = {\comnpatname variable capacitor = #1}}
\compattikzset{PZ/.style = {\comnpatname piezoelectric = #1}}
\compattikzset{L/.style = {\comnpatname inductor = #1}}
\compattikzset{vL/.style = {\comnpatname variable inductor = #1}}
\compattikzset{V/.style = {\comnpatname voltage source = #1}}
\compattikzset{cV/.style = {\comnpatname controlled voltage source = #1}}
\compattikzset{sV/.style = {\comnpatname sinusoidal voltage source = #1}}
\compattikzset{sqV/.style = {\comnpatname square voltage source = #1}}
\compattikzset{tV/.style = {\comnpatname triangle voltage source = #1}}
\compattikzset{csV/.style = {\comnpatname controlled sinusoidal voltage source = #1}}
\def\pgf@temp#1{
    \compattikzset{V#1/.style = {\comnpatname voltage source, v#1=##1} }
    \compattikzset{cV#1/.style = {\comnpatname controlled  voltage source, v#1=##1} }
    \compattikzset{sV#1/.style = {\comnpatname sinusoidal  voltage source, v#1=##1} }
    \compattikzset{csV#1/.style = {\comnpatname controlled sinusoidal voltage source, v#1=##1} }
}
\pgf@temp{_>} \pgf@temp{_<} \pgf@temp{^>} \pgf@temp{^<}
\pgf@temp{>} \pgf@temp{<} \pgf@temp{^} \pgf@temp{_}

%current sources
\compattikzset{I/.style = {\comnpatname current source = #1}}
\compattikzset{cI/.style = {\comnpatname controlled current source = #1}}
\compattikzset{sI/.style = {\comnpatname sinusoidal current source = #1}}
\compattikzset{csI/.style = {\comnpatname controlled sinusoidal current source = #1}}
\def\pgf@temp#1{
    \compattikzset{I#1/.style = {\comnpatname current source, i#1=##1} }
    \compattikzset{cI#1/.style = {\comnpatname controlled current source, i#1=##1} }
    \compattikzset{sI#1/.style = {\comnpatname sinusoidal current source, i#1=##1} }
    \compattikzset{csI#1/.style = {\comnpatname controlled sinusoidal current source, i#1=##1} }
}
\pgf@temp{_>} \pgf@temp{_<} \pgf@temp{^>} \pgf@temp{^<}
\pgf@temp{>_} \pgf@temp{<_} \pgf@temp{>^} \pgf@temp{<^}
\pgf@temp{>} \pgf@temp{<} \pgf@temp{^} \pgf@temp{_}

% cute sources
\def\pgf@circ@vsourceC@path#1{\pgf@circ@bipole@path{vsourceC}{#1}}
\compattikzset{cute european voltage source/.style = {\circuitikzbasekey, /tikz/to path=\pgf@circ@vsourceC@path, \circuitikzbasekey/bipole/is voltage=true, \circuitikzbasekey/bipole/is voltageoutsideofsymbol=true, v=#1}}
\compattikzset{vsourceC/.style= {\comnpatname cute european voltage source= #1}}
\compattikzset{ceV/.style= {\comnpatname cute european voltage source= #1}}

\def\pgf@circ@isourceC@path#1{\pgf@circ@bipole@path{isourceC}{#1}}
\compattikzset{cute european current source/.style = {\circuitikzbasekey, /tikz/to path=\pgf@circ@isourceC@path, \circuitikzbasekey/bipole/is current=true, i=#1}}
\compattikzset{isourceC/.style= {\comnpatname cute european current source= #1}}
\compattikzset{ceI/.style= {\comnpatname cute european current source= #1}}

\def\pgf@circ@cvsourceC@path#1{\pgf@circ@bipole@path{cvsourceC}{#1}}
\compattikzset{cute european controlled voltage source/.style = {\circuitikzbasekey, /tikz/to path=\pgf@circ@cvsourceC@path, \circuitikzbasekey/bipole/is voltage=true, \circuitikzbasekey/bipole/is voltageoutsideofsymbol=true, v=#1}}
\compattikzset{cvsourceC/.style= {\comnpatname cute european controlled voltage source= #1}}
\compattikzset{cceV/.style= {\comnpatname cute european controlled voltage source= #1}}

\def\pgf@circ@cisourceC@path#1{\pgf@circ@bipole@path{cisourceC}{#1}}
\compattikzset{cute european controlled current source/.style = {\circuitikzbasekey, /tikz/to path=\pgf@circ@cisourceC@path,  \circuitikzbasekey/bipole/is current=true, i=#1}}
\compattikzset{cisourceC/.style= {\comnpatname cute european controlled current source= #1}}
\compattikzset{cceI/.style= {\comnpatname cute european controlled current source= #1}}

% noise sources
\def\pgf@circ@vsourceN@path#1{\pgf@circ@bipole@path{vsourceN}{#1}}
\compattikzset{noise voltage source/.style = {\circuitikzbasekey, /tikz/to path=\pgf@circ@vsourceN@path, \circuitikzbasekey/bipole/is voltage=true, \circuitikzbasekey/bipole/is voltageoutsideofsymbol=true, v=#1}}
\compattikzset{vsourceN/.style= {\comnpatname noise voltage source= #1}}
\compattikzset{nV/.style= {\comnpatname noise voltage source= #1}}

\def\pgf@circ@isourceN@path#1{\pgf@circ@bipole@path{isourceN}{#1}}
\compattikzset{noise current source/.style = {\circuitikzbasekey, /tikz/to path=\pgf@circ@isourceN@path, \circuitikzbasekey/bipole/is current=true, i=#1}}
\compattikzset{isourceN/.style= {\comnpatname noise current source= #1}}
\compattikzset{nI/.style= {\comnpatname noise current source= #1}}

% resistive sensor american style
\def\pgf@circ@resistivesens@path#1{\pgf@circ@bipole@path{resistivesens}{#1}}
\compattikzset{american resistive sensor/.style = {\circuitikzbasekey, /tikz/to path=\pgf@circ@resistivesens@path, l=#1}}
\compattikzset{european resistive sensor/.style = {\circuitikzbasekey, /tikz/to path=\pgf@circ@thermistorntc@path, l=#1}}
\compattikzset{resistive sensor/.style= {\circuitikzbasekey, /tikz/to path=\pgf@circ@sresistor@path, l=#1}}
\compattikzset{sR/.style= {\comnpatname resistive sensor= #1}}

\compattikzset{capacitive sensor/.style= {\circuitikzbasekey, /tikz/to path=\pgf@circ@capacitivesens@path, l=#1}}
\compattikzset{sC/.style= {\comnpatname capacitive sensor= #1}}

\compattikzset{cute inductive sensor/.style = {\circuitikzbasekey, /tikz/to path=\pgf@circ@cuteinductivesens@path, l=#1}}
\compattikzset{european inductive sensor/.style = {\circuitikzbasekey, /tikz/to path=\pgf@circ@europeaninductivesens@path, l=#1}}
\compattikzset{american inductive sensor/.style = {\circuitikzbasekey, /tikz/to path=\pgf@circ@americaninductivesens@path, l=#1}}
\compattikzset{inductive sensor/.style= {\circuitikzbasekey, /tikz/to path=\pgf@circ@inductivesens@path, l=#1}}
\compattikzset{sL/.style= {\comnpatname inductive sensor= #1}}

\compattikzset{Mr/.style = {\comnpatname memristor}}

\compattikzset{cspst/.style = {\comnpatname closing switch = #1}}
\compattikzset{ospst/.style = {\comnpatname opening switch = #1}}
\compattikzset{spst/.style = {\comnpatname switch = #1}}

\compattikzset{afuse/.style = {\comnpatname asymmetric fuse=#1}}

\compattikzset{vdd/.style = {\comnpatname vcc = #1}}
\compattikzset{vss/.style = {\comnpatname vee = #1}}

% activate the to-style crossing
\def\pgf@circ@crossing@path#1{\pgf@circ@bipole@path{crossing}{#1}}
\compattikzset{crossing/.style = {\circuitikzbasekey, /tikz/to path=\pgf@circ@crossing@path, l=#1}}
\compattikzset{xing/.style= {\comnpatname crossing= #1}}

%% loudspeaker and microphone

\def\pgf@circ@loudspeaker@path#1{\pgf@circ@bipole@path{loudspeaker}{#1}}
\compattikzset{loudspeaker/.style = {\circuitikzbasekey, /tikz/to path=\pgf@circ@loudspeaker@path, l=#1}}

\def\pgf@circ@mic@path#1{\pgf@circ@bipole@path{mic}{#1}}
\compattikzset{mic/.style = {\circuitikzbasekey, /tikz/to path=\pgf@circ@mic@path, l=#1}}

% more instrument
\def\pgf@circ@rmeter@path#1{\pgf@circ@bipole@path{rmeter}{#1}}
\compattikzset{rmeter/.style = {\circuitikzbasekey, /tikz/to path=\pgf@circ@rmeter@path, l=#1}}
\def\pgf@circ@rmeterwa@path#1{\pgf@circ@bipole@path{rmeterwa}{#1}}
\compattikzset{rmeterwa/.style = {\circuitikzbasekey, /tikz/to path=\pgf@circ@rmeterwa@path, l=#1}}
\def\pgf@circ@smeter@path#1{\pgf@circ@bipole@path{smeter}{#1}}
\compattikzset{smeter/.style = {\circuitikzbasekey, /tikz/to path=\pgf@circ@smeter@path, l=#1}}
% current loop probes
\def\pgf@circ@iloop@path#1{\pgf@circ@bipole@path{iloop}{#1}}
\compattikzset{iloop/.style = {\circuitikzbasekey, /tikz/to path=\pgf@circ@iloop@path, l=#1}}
\def\pgf@circ@iloop2@path#1{\pgf@circ@bipole@path{iloop2}{#1}}
\compattikzset{iloop2/.style = {\circuitikzbasekey, /tikz/to path=\pgf@circ@iloop2@path, l=#1}}
% qucs-style probes
\def\pgf@circ@qiprobe@path#1{\pgf@circ@bipole@path{qiprobe}{#1}}
\compattikzset{qiprobe/.style = {\circuitikzbasekey, /tikz/to path=\pgf@circ@qiprobe@path, l=#1}}
\def\pgf@circ@qvprobe@path#1{\pgf@circ@bipole@path{qvprobe}{#1}}
\compattikzset{qvprobe/.style = {\circuitikzbasekey, /tikz/to path=\pgf@circ@qvprobe@path, l=#1}}
\def\pgf@circ@qpprobe@path#1{\pgf@circ@bipole@path{qpprobe}{#1}}
\compattikzset{qpprobe/.style = {\circuitikzbasekey, /tikz/to path=\pgf@circ@qpprobe@path, l=#1}}
% multiwire(s)
\def\pgf@circ@bmultiwire@path#1{\pgf@circ@bipole@path{bmultiwire}{#1}}
\compattikzset{bmultiwire/.style = {\circuitikzbasekey,
/tikz/to path=\pgf@circ@bmultiwire@path, l=#1}}
\def\pgf@circ@multiwire@path#1{\pgf@circ@bipole@path{multiwire}{#1}}
\compattikzset{multiwire/.style = {\circuitikzbasekey,
/tikz/to path=\pgf@circ@multiwire@path, l=#1}}
\def\pgf@circ@tmultiwire@path#1{\pgf@circ@bipole@path{tmultiwire}{#1}}
\compattikzset{tmultiwire/.style = {\circuitikzbasekey,
/tikz/to path=\pgf@circ@tmultiwire@path, l=#1}}

% reed switches
\def\pgf@circ@reed@path#1{\pgf@circ@bipole@path{reed}{#1}}
\compattikzset{reed/.style = {\circuitikzbasekey, /tikz/to path=\pgf@circ@reed@path, l=#1}}

% Transistor like bipoles

\def\pgf@circ@trans@path#1#2{
    \pgfextra{
        \edef\pgf@temp{\ctikzvalof{bipole/name}}
        \def\pgf@circ@temp{#2}
        \ifx\pgf@temp\pgf@circ@temp % if it has not a name
            \pgfmathrandominteger{\pgf@circ@rand}{1000}{9999}
            \ctikzset{bipole/name = trans\pgf@circ@rand} % create it
        \fi
    }
    \ifpgf@circuit@bipole@inverted
        (\tikztostart) node[coordinate] (\ctikzvalof{bipole/name}end) {}
        (\tikztotarget) node[coordinate] (\ctikzvalof{bipole/name}start) {}
    \else
        (\tikztostart) node[coordinate] (\ctikzvalof{bipole/name}start) {}
        (\tikztotarget) node[coordinate] (\ctikzvalof{bipole/name}end) {}
    \fi
    \pgfextra{
        \pgfmathanglebetweenpoints{\pgfpointanchor{\ctikzvalof{bipole/name}start}{center}}
        {\pgfpointanchor{\ctikzvalof{bipole/name}end}{center}}
        \pgfmathadd{\pgfmathresult}{-90}
        \pgfmathround{\pgfmathresult}
        \edef\pgf@circ@direction{\pgfmathresult}
    }
    ($(\tikztostart) ! .5 ! (\tikztotarget)$)
    node[#1, /tikz/rotate=\pgf@circ@direction, xscale=\ctikzvalof{mirror value}]
    (\ctikzvalof{bipole/name}) {}
    node {\ctikzvalof{bipole/label/name}}
    \ifcsname pgf@anchor@#1@pathstart\endcsname%if special path-anchors are defined, use them!
        (\ctikzvalof{bipole/name}start.center) --(\ctikzvalof{bipole/name}.pathstart)
        (\ctikzvalof{bipole/name}.pathend)  -- (\ctikzvalof{bipole/name}end.center)
    \else
        (\ctikzvalof{bipole/name}start.center) --(\ctikzvalof{bipole/name}.left)
        (\ctikzvalof{bipole/name}.right)  -- (\ctikzvalof{bipole/name}end.center)
    \fi
    \drawpoles
    \pgfextra{
        \pgfcircresetpath
    }
    (\tikztotarget) 	\tikztonodes  % e si continua
}


\def\pgf@circ@definetranspath#1{
	\compattikzset{T#1/.style = {\circuitikzbasekey, /tikz/to path=\pgf@circ@trans@path{#1}{}, l=##1}}
}

\pgf@circ@definetranspath{elmech}
\pgf@circ@definetranspath{nmos}
\pgf@circ@definetranspath{pmos}
\pgf@circ@definetranspath{hemt}
\pgf@circ@definetranspath{npn}
\pgf@circ@definetranspath{pnp}
\pgf@circ@definetranspath{nfet}
\pgf@circ@definetranspath{nigfete}
\pgf@circ@definetranspath{nigfetd}
\pgf@circ@definetranspath{nigfetebulk}
\pgf@circ@definetranspath{pfet}
\pgf@circ@definetranspath{pigfete}
\pgf@circ@definetranspath{pigfetd}
\pgf@circ@definetranspath{pigfetebulk}
\pgf@circ@definetranspath{njfet}
\pgf@circ@definetranspath{pjfet}
\pgf@circ@definetranspath{pigbt}
\pgf@circ@definetranspath{nigbt}
\pgf@circ@definetranspath{Lpigbt}
\pgf@circ@definetranspath{Lnigbt}
%%%---------- close: tex/pgfcircpath

\ifpgf@circ@siunitx
    \RequirePackage{xstring}[2009/03/13]
    %\expandafter\let\csname angstrom\endcsname\relax
    \RequirePackage{siunitx}
\fi

\ifpgf@circ@explicitvdir\else
    \PackageWarningNoLine{circuitikz}{%
        You did not specify one of the voltage directions:\MessageBreak
        \space\space    oldvoltagedirection, nooldvoltagedirection, \MessageBreak
        \space\space    RPvoltages or EFvoltages \MessageBreak
        Default directions may have changed, \MessageBreak
        please check the manual%
    }
\fi

\newenvironment{circuitikz}{\begin{tikzpicture}}{\end{tikzpicture}}
%override (unused) circuitikz environment for compability to externalization)
\let\circuitikz\tikzpicture
\let\endcircuitikz\endtikzpicture

\endinput
