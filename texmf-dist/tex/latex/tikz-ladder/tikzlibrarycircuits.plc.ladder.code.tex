%% This is file `tikzlibrarycircuits.plc.ladder.code.tex'
%% Version: 1.3
%% Version date: 2022-04-10
%% 
%% 
%% Copyright (C) 2018 by Luis Paulo Laus, laus@utfpr.edu.br
%%
%% This package can be redistributed and/or modified under the terms
%% of the LaTeX Project Public License distributed from CTAN
%% archives in directory macros/latex/base/lppl.txt; either
%% version 1 of the License, or (at your option) any later version,
%% with `The Package' referring to the software
%% `tikzlibrarycircuits.plc.ladder.code.tex'
%% and its accompanying documentation and `The Copyright Holder'
%% referring to the person Luis Paulo Laus.
%% 
%% 
%% IMPORTANT NOTICE: 
%% 
%% For error reports, comments or suggestions in case of UNCHANGED 
%% versions send mail to:
%% laus@utfpr.edu.br
%% 
%%
%% Change History:
%% 1.0: 10 January 2018:  Original Version
%% 1.1: 06 February 2018: Macro file renamed to proper TikZ library file name.
%%                        Parent library changed from `circuit.ee' to `circuits'.
%%                        Bug fixes: added \pgfusepathqstroke where it was missing,
%%                        block anchors reworked.
%% 1.2: 05 December 2020: Bugfix: corrected block height when the symbol is empty ({}).
%% 1.3: 10 April 2022: new macro for typesetting power rails, fix the annoying bug that prints
%%                       `Missing character....'

\usetikzlibrary{circuits}
%
% Length and macros
%
\makeatletter
\newlength{\ladderskip}
\def\ladderrungsep@{0.2}
\def\ladderrungend#1{\pgftransformyshift{-#1\ladderskip-\ladderrungsep@\ladderskip}}% mark the end of a rung, #1 is the number of rows
\def\ladderpowerrails{{% draws power rails
  \draw let \p1=(laddertopright) in%
    (0,\y1+0.7\ladderskip) -- (0,0.8\ladderskip)%
    (\x1,\y1+0.7\ladderskip) -- (\x1,0.8\ladderskip);}}

%
% Setting up a ladder
%
\tikzset{
  ladderrungsep/.store in=\ladderrungsep@,
  circuit plc ladder/.style={
    circuit,
    inputs/.style={block ladder inputs={##1}},
    outputs/.style={block ladder outputs={##1}},
    input sep/.style={block ladder input sep={##1}},
    output sep/.style={block ladder output sep={##1}},
    clksize/.style={block ladder clksize={##1}},
    symbol/.style={ladder symbol={##1}},
    symbol color/.style={ladder symbol color={##1}},
    x=5\tikzcircuitssizeunit,
    y=5\tikzcircuitssizeunit,
    execute at begin picture={\ladderskip=\pgf@yy\coordinate (laddertopright) at (0,0);}, % sets \ladderskip and defines a point for power rails
    every circuit plc ladder/.try
  }
}       

\newif\ifpgfnormallyclosednegated%
\pgfkeys{
  /pgf/crossed ladder/.is if=pgfnormallyclosednegated,%
}

\pgfkeys{/pgf/.cd,
  contact ladder symbol/.initial=, % no value
  contact NC ladder symbol/.initial=, % no value
  var contact NC ladder symbol/.initial=$/$,
  contact N ladder symbol/.initial=N,
  contact P ladder symbol/.initial=P,
  coil ladder symbol/.initial=, % no value
  coil NA ladder symbol/.initial=, % no value
  var coil NA ladder symbol/.initial=$/$,
  coil S ladder symbol/.initial=S,
  coil R ladder symbol/.initial=R,
  coil ladder curvature/.initial=0.2, % parameter for ( and ) in coils
  ladder symbol color/.initial=, % no value
  ladder symbol/.initial=,
  block ladder input sep/.initial=1.6\tikzcircuitssizeunit,
  block ladder output sep/.initial=1.6\tikzcircuitssizeunit,
  block ladder inputs/.initial={IN},
  block ladder outputs/.initial={Q},
  block ladder clksize/.initial=0.8\tikzcircuitssizeunit,
  block ladder symbol/.initial=TON, % blocks like TON, TOF, TP, ADD, CDU, CDUD\_*
}

% Internal macro for parsing inputs, outputs and clock inputs (no short version).
%
\expandafter\ifx\csname pgf@lib@sh@ladder@parseinout\endcsname\relax%
\def\pgf@lib@sh@ladder@parseinout#1{%
	\edef\pgf@lib@sh@temp{\pgfkeysvalueof{/pgf/block ladder \pgf@lib@sh@ladder@inout}}%
	\c@pgf@counta#1\relax%
	\c@pgf@countb0\relax%
	\expandafter\pgfutil@in@\expandafter,\expandafter{\pgf@lib@sh@temp}%
	\let\pgf@lib@sh@next\pgf@lib@sh@ladder@parseinout@long%
	\pgf@lib@sh@next%
}	
\def\pgf@lib@sh@ladder@parseinout@long{%
	\expandafter\pgf@lib@sh@ladder@parseinout@@long\pgf@lib@sh@temp,\pgf@stop,%
}
\def\pgf@lib@sh@ladder@parseinout@@long#1,{%
	\ifx#1\pgf@stop%
		\expandafter\edef\csname pgf@lib@sh@ladder@num\pgf@lib@sh@ladder@inout\endcsname{\the\c@pgf@countb}%
		\let\pgf@lib@sh@next\relax%
	\else%
		\ifnum\c@pgf@countb=\c@pgf@counta%
			\expandafter\edef\csname pgf@lib@sh@ladder@num\pgf@lib@sh@ladder@inout\endcsname{\the\c@pgf@countb}%
			\let\pgf@lib@sh@next\relax%
		\else%
			\advance\c@pgf@countb1\relax%
			\def\pgf@lib@sh@temp{#1}%
			\expandafter\pgf@sh@resavedmacro\expandafter{\csname \pgf@lib@sh@ladder@inout-\the\c@pgf@countb\endcsname}{%
				\expandafter\def\csname \pgf@lib@sh@ladder@inout-\the\c@pgf@countb\endcsname{\expandafter\pgf@lib@sh@temp}}%
			\let\pgf@lib@sh@next\pgf@lib@sh@ladder@parseinout@@@long%
		\fi%
	\fi%
	\pgf@lib@sh@next%
}
\def\pgf@lib@sh@ladder@parseinout@@@long{%
	\pgfutil@ifnextchar x{\pgf@lib@sh@ladder@parseinout@@long}%
		{\pgf@lib@sh@ladder@parseinout@@long}%
}
\fi%


% Adjust the width and height of the ladder block  symbol.
\def\pgf@lib@sh@ladder@dimensions{
  \pgf@x.5\wd\pgfnodeparttextbox
  \pgf@y.5\ht\pgfnodeparttextbox
  \advance\pgf@y.5\dp\pgfnodeparttextbox

  %
  % Adjust the height for the number of inputs and the width and height of the ladder block symbol.
  %
  \c@pgf@counta\pgf@lib@sh@ladder@numinputs
  \pgfmathsetlength\pgf@yb{\pgfkeysvalueof{/pgf/block ladder input sep}}
  \edef\inputsep{\the\pgf@yb}
  \addtosavedmacro\inputsep
  \advance\c@pgf@counta-1\relax
  \multiply\pgf@yb\c@pgf@counta
   \c@pgf@counta\pgf@lib@sh@ladder@numoutputs
  \pgfmathsetlength\pgf@yc{\pgfkeysvalueof{/pgf/block ladder output sep}}
  \edef\outputsep{\the\pgf@yc}
  \addtosavedmacro\outputsep
  \advance\c@pgf@counta-1\relax
  \multiply\pgf@yc\c@pgf@counta
  \ifdim\pgf@yb<\pgf@yc% the highest between inputs and outputs
    \pgf@yb\pgf@yc
  \fi
  % now the symbol width and height
  \begingroup% keep font setting local
    \pgfutil@selectfont% restore font
    \global\setbox\pgf@hbox\hbox{{\pgfkeysvalueof{/pgf/ladder symbol}}}% add braces as color-aftergroup-fix TT
  \endgroup%
% \typeout{Test \the\wd\pgf@hbox  \the\ht\pgf@hbox  \the\dp\pgf@hbox}
  \pgf@xc.5\wd\pgf@hbox
  \pgf@yc\ht\pgf@hbox
  \advance\pgf@yc\dp\pgf@hbox
  \ifdim\pgf@yc=0pt
    \advance\pgf@yb4pt
  \fi
  \advance\pgf@yc1pt
  \advance\pgf@yb0.5\pgf@yc
  \advance\pgf@yc2pt
  \edef\symbolypos{\the\pgf@yc}
  \addtosavedmacro\symbolypos
  \advance\pgf@yc0.5\pgf@yc
  \advance\pgf@yc\pgf@yb
  \pgf@yc.5\pgf@yc
  \ifdim\pgf@x<\pgf@xc
    \pgf@x\pgf@xc
  \fi
  \ifdim\pgf@y<\pgf@yc
    \pgf@y\pgf@yc
  \fi
  \pgfmathaddtolength\pgf@x{\pgfkeysvalueof{/pgf/inner xsep}}
  \pgfmathaddtolength\pgf@y{\pgfkeysvalueof{/pgf/inner ysep}}

  %
  % Adjust for minimum height and width.
  %
  \pgfmathsetlength\pgf@xa{\pgfkeysvalueof{/pgf/minimum width}}%
  \pgfmathsetlength\pgf@ya{\pgfkeysvalueof{/pgf/minimum height}}%
  \ifdim\pgf@x<.5\pgf@xa%
    \pgf@x.5\pgf@xa%
  \fi%
  \ifdim\pgf@y<.5\pgf@ya%
    \pgf@y.5\pgf@ya%
  \fi%
  %
  % Take into account the outer sep.
  %
  \pgfmathaddtolength\pgf@x{\pgfkeysvalueof{/pgf/outer xsep}}%
  \pgfmathaddtolength\pgf@y{\pgfkeysvalueof{/pgf/outer ysep}}%
  %
  \edef\halfwidth{\the\pgf@x}%
  \edef\halfheight{\the\pgf@y}%
  \addtosavedmacro\halfwidth%
  \addtosavedmacro\halfheight%

  \advance\pgf@yb\pgf@y
  \advance\pgf@yb-\pgf@yc
  \edef\boxdepth{\the\pgf@yb}% below 1st input
  \addtosavedmacro\boxdepth
}%


\def\pgf@lib@sh@ladder@inputanchor#1{%
  \dimensions%
  \pgfpointorigin%
  \advance\pgf@x-\halfwidth%
  %
  \pgfutil@tempdima\inputsep%
  \multiply\pgfutil@tempdima#1\relax%
  \advance\pgf@y-\pgfutil@tempdima%
  \advance\pgf@y\inputsep%
}

\def\pgf@lib@sh@ladder@outputanchor#1{%
  \dimensions%
  \pgfpointorigin%
  \advance\pgf@x\halfwidth%
  %
  \pgfutil@tempdima\outputsep%
  \multiply\pgfutil@tempdima#1\relax%
  \advance\pgf@y-\pgfutil@tempdima%
  \advance\pgf@y\outputsep%
}

\def\pgfutil@first@until@relax#1#2\relax{#1}

% Shapes
% block
\pgfdeclareshape{block ladder}{
	\expandafter\pgfutil@g@addto@macro\csname pgf@sh@s@block ladder\endcsname{%
	     \def\pgf@lib@sh@ladder@inout{inputs}
		\pgf@lib@sh@ladder@parseinout{1024}% Maximum 1024 (!) inputs.
		%
		\pgfmathloop%
			\ifnum\pgfmathcounter>\pgf@lib@sh@ladder@numinputs%
			\else%
				\pgfutil@ifundefined{pgf@anchor@block ladder@input \pgfmathcounter}{%
					\expandafter\xdef\csname pgf@anchor@block ladder@input \pgfmathcounter\endcsname{%
						\noexpand\pgf@lib@sh@ladder@inputanchor{\pgfmathcounter}%
					}%
				}{}%
		\repeatpgfmathloop%
		\ifnum\pgf@lib@sh@ladder@numinputs<1\relax%
			\pgferror{A block must have at least one input}%
		\fi%
		%
	     \def\pgf@lib@sh@ladder@inout{outputs}
		\pgf@lib@sh@ladder@parseinout{1024}% Maximum 1024 (!) outputs.
		%
		\pgfmathloop%
			\ifnum\pgfmathcounter>\pgf@lib@sh@ladder@numoutputs%
			\else%
				\pgfutil@ifundefined{pgf@anchor@block ladder@output \pgfmathcounter}{%
					\expandafter\xdef\csname pgf@anchor@block ladder@output \pgfmathcounter\endcsname{%
						\noexpand\pgf@lib@sh@ladder@outputanchor{\pgfmathcounter}%
					}%
				}{}%
		\repeatpgfmathloop%
		\ifnum\pgf@lib@sh@ladder@numoutputs<1\relax%
			\pgferror{A block must have at least one output}%
		\fi%
	}%

	\savedmacro\numinputs{\let\numinputs\pgf@lib@sh@ladder@numinputs}%
	\savedmacro\numoutputs{\let\numoutputs\pgf@lib@sh@ladder@numoutputs}%
	\savedmacro\dimensions{%
		\pgf@lib@sh@ladder@dimensions%
	}
	\savedanchor\centerpoint{%
		\pgfpointorigin
	}
	\savedanchor\midpoint{%
      \pgfpointorigin
      \advance\pgf@y-.5\ht\pgfnodeparttextbox
      \advance\pgf@y.5\dp\pgfnodeparttextbox
		\pgfmathaddtolength\pgf@y{+0.5ex}%
	}
   \savedanchor\basepoint{%
      \pgfpointorigin
      \advance\pgf@y-.5\ht\pgfnodeparttextbox
      \advance\pgf@y.5\dp\pgfnodeparttextbox
   }
	\savedanchor\textpoint{%
      \pgfpointorigin
      \advance\pgf@x-.5\wd\pgfnodeparttextbox
      \advance\pgf@y-.5\ht\pgfnodeparttextbox
      \advance\pgf@y.5\dp\pgfnodeparttextbox
   }
	\anchor{center}{\pgfpointorigin}%
	\anchor{block center}{\dimensions\pgfpointorigin\advance\pgf@y\halfheight\advance\pgf@y-\boxdepth}%
	\anchor{text}{\dimensions\textpoint\advance\pgf@y\halfheight\advance\pgf@y-\boxdepth}%
	\anchor{mid}{\dimensions\midpoint\advance\pgf@y\halfheight\advance\pgf@y-\boxdepth}%
	\anchor{mid west}{\dimensions\midpoint\advance\pgf@y\halfheight\advance\pgf@y-\boxdepth\advance\pgf@x-\halfwidth}%
	\anchor{mid east}{\dimensions\midpoint\advance\pgf@y\halfheight\advance\pgf@y-\boxdepth\advance\pgf@x\halfwidth}%
	\anchor{base}{\dimensions\basepoint\advance\pgf@y\halfheight\advance\pgf@y-\boxdepth}%
	\anchor{base west}{\dimensions\basepoint\advance\pgf@y\halfheight\advance\pgf@y-\boxdepth\advance\pgf@x-\halfwidth}%
	\anchor{base east}{\dimensions\basepoint\advance\pgf@y\halfheight\advance\pgf@y-\boxdepth\advance\pgf@x\halfwidth}%
	\anchor{north}{\dimensions\pgfpointorigin\advance\pgf@y\halfheight\advance\pgf@y\halfheight\advance\pgf@y-\boxdepth}%
	\anchor{south}{\dimensions\pgfpointorigin\advance\pgf@y-\boxdepth}%
	\anchor{east}{\dimensions\pgfpointorigin\advance\pgf@x\halfwidth\advance\pgf@y\halfheight\advance\pgf@y-\boxdepth}%
	\anchor{west}{\dimensions\pgfpointorigin\advance\pgf@x-\halfwidth\advance\pgf@y\halfheight\advance\pgf@y-\boxdepth}%
	\anchor{north east}{\dimensions\pgfpointorigin\advance\pgf@x\halfwidth\advance\pgf@y\halfheight\advance\pgf@y\halfheight\advance\pgf@y-\boxdepth}%
	\anchor{north west}{\dimensions\pgfpointorigin\advance\pgf@x-\halfwidth\advance\pgf@y\halfheight\advance\pgf@y\halfheight\advance\pgf@y-\boxdepth}%
	\anchor{south east}{\dimensions\pgfpointorigin\advance\pgf@x\halfwidth\advance\pgf@y-\boxdepth}%
	\anchor{south west}{\dimensions\pgfpointorigin\advance\pgf@x-\halfwidth\advance\pgf@y-\boxdepth}%
	\backgroundpath{%
		\dimensions%
		\pgf@xc\halfwidth%
		\pgf@yc\halfheight%
		\pgfmathaddtolength\pgf@xc{-\pgfkeysvalueof{/pgf/outer xsep}}%
		\pgfmathaddtolength\pgf@yc{-\pgfkeysvalueof{/pgf/outer ysep}}%
		      
		\pgf@yb\boxdepth%
		\pgfmathaddtolength\pgf@yb{-\pgfkeysvalueof{/pgf/outer ysep}}%
		{%
			\pgfpathrectangle{\pgfqpoint{-\pgf@xc}{-\pgf@yb}}%
			{%
				\pgf@x2.0\pgf@xc%
				\pgf@y2.0\pgf@yc%
			}%
			\pgfpathclose%
		}%
	}
	\foregroundpath{
		\dimensions%
		{%
			%
			% Write block type (symbol)
			%
			\pgf@yc\symbolypos
			\expandafter\pgftext\expandafter[/pgf/text/at=\pgfqpoint{0pt}{\pgf@yc},/pgf/text/center]{%
				\pgfkeysgetvalue{/pgf/ladder symbol color}{\pgf@lib@sh@temp}%
				\ifx\pgf@lib@sh@temp\pgfutil@empty%
				\else%
					\expandafter\color\expandafter{\pgf@lib@sh@temp}%
				\fi%
				\pgfkeysvalueof{/pgf/ladder symbol}%
			}%
			% Draw inputs.
			%
			\pgf@xc-\halfwidth%
			\pgfmathaddtolength\pgf@xc{\pgfkeysvalueof{/pgf/outer xsep}}
			\pgf@xa\pgf@xc
			\pgfmathaddtolength\pgf@xc{\pgfkeysvalueof{/pgf/inner xsep}}
  			\pgf@yc\inputsep%
			\pgfmathloop%
			\ifnum\pgfmathcounter>\numinputs%
			\else%
				\advance\pgf@yc-\inputsep%
				\edef\pgf@lib@sh@temp{\csname inputs-\pgfmathcounter\endcsname}
				\pgf@xb\pgf@xc
				\if>\expandafter\pgfutil@first@until@relax\pgf@lib@sh@temp\relax
                           \edef\pgf@lib@sh@temp{\expandafter\expandafter\expandafter\pgfutil@gobble\pgf@lib@sh@temp}
					\pgf@ya\pgf@yc
					\pgfmathsetlength\pgf@yb{\pgfkeysvalueof{/pgf/block ladder clksize}}
					\pgfmathaddtolength\pgf@ya{0.5\pgf@yb}
					\pgfpathmoveto{\pgfqpoint{\pgf@xa}{\pgf@ya}}
					\pgfmathaddtolength\pgf@ya{-0.5\pgf@yb}
					\pgfmathaddtolength\pgf@xa{\pgf@yb}
					\pgfpathlineto{\pgfqpoint{\pgf@xa}{\pgf@ya}}
					\pgfmathaddtolength\pgf@ya{-0.5\pgf@yb}
					\pgfmathaddtolength\pgf@xa{-\pgf@yb}
					\pgfpathlineto{\pgfqpoint{\pgf@xa}{\pgf@ya}}
					\pgfmathaddtolength\pgf@xb{\pgf@yb}
					\pgfpathclose
				\fi
				\expandafter\pgftext\expandafter[/pgf/text/at=\pgfqpoint{\pgf@xb}{\pgf@yc},/pgf/text/left]{%
					\pgfkeysgetvalue{/pgf/ladder symbol color}{\pgf@lib@sh@@temp}%
					\ifx\pgf@lib@sh@@temp\pgfutil@empty%
					\else%
						\expandafter\color\expandafter{\pgf@lib@sh@@temp}%
					\fi%
					{\pgf@lib@sh@temp}%
				}%
			\repeatpgfmathloop%
			% Draw outputs.
			%
			\pgf@xc\halfwidth%
			\pgfmathaddtolength\pgf@xc{-\pgfkeysvalueof{/pgf/inner xsep}}
			\pgfmathaddtolength\pgf@xc{-\pgfkeysvalueof{/pgf/outer xsep}}
  			\pgf@yc\outputsep%
			\pgfmathloop%
			\ifnum\pgfmathcounter>\numoutputs%
			\else%
				\advance\pgf@yc-\outputsep%
				\expandafter\pgftext\expandafter[/pgf/text/at=\pgfqpoint{\pgf@xc}{\pgf@yc},/pgf/text/right]{%
					\pgfkeysgetvalue{/pgf/ladder symbol color}{\pgf@lib@sh@temp}%
					\ifx\pgf@lib@sh@temp\pgfutil@empty%
					\else%
						\expandafter\color\expandafter{\pgf@lib@sh@temp}%
					\fi%
					{\csname outputs-\pgfmathcounter\endcsname}%
				}%
			\repeatpgfmathloop%
		}%
	} 
	\anchorborder{%
		\dimensions%
		\pgfextract@process\externalpoint{}%
		\ifdim\pgf@y<0pt%
			\pgf@process{\pgfpointborderrectangle{\externalpoint}{\pgfqpoint{\halfwidth}{\boxdepth}}}%
		\else%
			\pgf@yc\halfheight
			\advance\pgf@yc\halfheight\advance\pgf@yc-\boxdepth
			\pgf@process{\pgfpointborderrectangle{\externalpoint}{\pgfqpoint{\halfwidth}{\pgf@yc}}}%
		\fi%
	}%  
}




% contact
\pgfdeclareshape{contact ladder}
{
  \inheritsavedanchors[from=rectangle]
  \inheritanchor[from=rectangle]{center}
  \inheritanchor[from=rectangle]{north}
  \inheritanchor[from=rectangle]{south}
  \inheritanchor[from=rectangle]{east}
  \inheritanchor[from=rectangle]{west}
  \inheritanchor[from=rectangle]{north east}
  \inheritanchor[from=rectangle]{north west}
  \inheritanchor[from=rectangle]{south east}
  \inheritanchor[from=rectangle]{south west}
  \inheritanchor[from=rectangle]{input}
  \inheritanchor[from=rectangle]{output}
  \inheritanchorborder[from=rectangle]  % New:
  \anchor{input}{
    \pgf@process{\northeast}%
    \pgf@ya=.5\pgf@y%
    \pgf@process{\southwest}%
    \pgf@y=.5\pgf@y%
    \advance\pgf@y by \pgf@ya%
  }
  \anchor{output}{%
    \pgf@process{\southwest}%
    \pgf@ya=.5\pgf@y%
    \pgf@process{\northeast}%
    \pgf@y=.5\pgf@y%
    \advance\pgf@y by \pgf@ya%
  }
  \savedanchor\centerpoint{%
    \pgf@x.5\wd\pgfnodeparttextbox%
    \pgf@y.5\ht\pgfnodeparttextbox%
    \advance\pgf@y-.5\dp\pgfnodeparttextbox%
  }

  \backgroundpath{
    \pgf@process{\pgfpointadd{\southwest}{\pgfpoint{\pgfkeysvalueof{/pgf/outer xsep}}{\pgfkeysvalueof{/pgf/outer ysep}}}}
    \pgf@xa=\pgf@x \pgf@ya=\pgf@y
    \pgf@process{\pgfpointadd{\northeast}{\pgfpointscale{-1}{\pgfpoint{\pgfkeysvalueof{/pgf/outer xsep}}{\pgfkeysvalueof{/pgf/outer ysep}}}}}
    \pgf@xb=\pgf@x \pgf@yb=\pgf@y
    \pgfpathmoveto{\pgfqpoint{\pgf@xa}{\pgf@ya}}
    \pgfpathmoveto{\pgfqpoint{\pgf@xa}{\pgf@yb}}
    \pgfpathlineto{\pgfqpoint{\pgf@xa}{\pgf@ya}}
    \ifpgfnormallyclosednegated
       \pgfsetbeveljoin
       \pgfpathlineto{\pgfqpoint{\pgf@xb}{\pgf@yb}}
    \else
       \pgfpathmoveto{\pgfqpoint{\pgf@xb}{\pgf@yb}}
    \fi
    \pgfpathlineto{\pgfqpoint{\pgf@xb}{\pgf@ya}}
    \pgfusepathqstroke
  }
  \foregroundpath{
    \pgftransformshift{\centerpoint}
    \pgftext{\pgfkeysvalueof{/pgf/ladder symbol}}
  }
}

% coil
\pgfdeclareshape{coil ladder}
{
  \inheritsavedanchors[from=rectangle]
  \inheritanchor[from=rectangle]{center}
  \inheritanchor[from=rectangle]{north}
  \inheritanchor[from=rectangle]{south}
  \inheritanchor[from=rectangle]{east}
  \inheritanchor[from=rectangle]{west}
  \inheritanchor[from=rectangle]{north east}
  \inheritanchor[from=rectangle]{north west}
  \inheritanchor[from=rectangle]{south east}
  \inheritanchor[from=rectangle]{south west}
  \inheritanchor[from=rectangle]{input}
  \inheritanchor[from=rectangle]{output}
  \inheritanchorborder[from=rectangle]  % New:
  \anchor{input}{
    \pgf@process{\northeast}%
    \pgf@ya=.5\pgf@y%
    \pgf@process{\southwest}%
    \pgf@y=.5\pgf@y%
    \advance\pgf@y by \pgf@ya%
  }
  \anchor{output}{%
    \pgf@process{\southwest}%
    \pgf@ya=.5\pgf@y%
    \pgf@process{\northeast}%
    \pgf@y=.5\pgf@y%
    \advance\pgf@y by \pgf@ya%
  }
  \savedanchor\centerpoint{%
    \pgf@x.5\wd\pgfnodeparttextbox%
    \pgf@y.5\ht\pgfnodeparttextbox%
    \advance\pgf@y-.5\dp\pgfnodeparttextbox%
  }

  \backgroundpath{
    \pgf@process{\pgfpointadd{\southwest}{\pgfpoint{\pgfkeysvalueof{/pgf/outer xsep}}{\pgfkeysvalueof{/pgf/outer ysep}}}}
    \pgf@xa=\pgf@x \pgf@ya=\pgf@y
    \pgf@process{\pgfpointadd{\northeast}{\pgfpointscale{-1}{\pgfpoint{\pgfkeysvalueof{/pgf/outer xsep}}{\pgfkeysvalueof{/pgf/outer ysep}}}}}
    \pgf@xb=\pgf@x \pgf@yb=\pgf@y
    
    \pgfmathsetlength\pgf@xc{\pgfkeysvalueof{/pgf/coil ladder curvature}\pgf@xb-\pgfkeysvalueof{/pgf/coil ladder curvature}\pgf@xa} % uses a reduction of 20% (by default) for beginning of the arcs, if it is 0.5 and width=height we got a circle, do not set to zero
    \pgfmathsetlength\pgf@yc{.5\pgf@yb-.5\pgf@ya}
    \pgfmathsetlength\pgf@ya{0.5\pgf@xc + 0.5\pgf@yc*\pgf@yc / \pgf@xc} % radius
    \advance\pgf@xa by \pgf@xc
    \advance\pgf@xb by-\pgf@xc
    \pgfmathsetlength\pgf@yc{\pgf@yc / \pgf@ya}
    \pgfmathasin{\pgf@yc}
    \pgfmathsetlength\pgf@yc{\pgfmathresult} % angle
    \pgfpathmoveto{\pgfqpoint{\pgf@xa}{\pgf@yb}}
    \pgfpatharc{180-\pgf@yc}{180+\pgf@yc}{\pgf@ya}
    \ifpgfnormallyclosednegated
       \pgfsetbeveljoin
       \pgfpathlineto{\pgfqpoint{\pgf@xb}{\pgf@yb}}
    \else
       \pgfpathmoveto{\pgfqpoint{\pgf@xb}{\pgf@yb}}
    \fi
    \pgfpatharc{\pgf@yc}{-\pgf@yc}{\pgf@ya}
    \pgfusepathqstroke
  }
  \foregroundpath{
    \pgftransformshift{\centerpoint}
    \pgftext{\pgfkeysvalueof{/pgf/ladder symbol}}
  }
}

\makeatother

\tikzset{
  circuit declare symbol=contact NO,
  circuit declare symbol=contact NC,
  circuit declare symbol=var contact NC,
  circuit declare symbol=contact P,
  circuit declare symbol=contact N,
  circuit declare symbol=coil,
  circuit declare symbol=coil NA,
  circuit declare symbol=var coil NA,
  circuit declare symbol=coil S,
  circuit declare symbol=coil R,
  circuit declare symbol=block,
  set contact NO graphic={
    circuit symbol lines,
    circuit symbol size=width 1.5 height 2,
    transform shape,
    shape=contact ladder
  },
  set contact NC graphic={
    circuit symbol lines,
    circuit symbol size=width 1.5 height 2,
    transform shape,
    crossed ladder=true,
    shape=contact ladder
  },
  set var contact NC graphic={
    circuit symbol lines,
    circuit symbol size=width 1.5 height 2,
    transform shape,
    ladder symbol=\pgfkeysvalueof{/pgf/var contact NC ladder symbol},
    shape=contact ladder
  },
  set contact P graphic={
    circuit symbol lines,
    circuit symbol size=width 1.5 height 2,
    transform shape,
    ladder symbol=\pgfkeysvalueof{/pgf/contact P ladder symbol},
    shape=contact ladder
  },
  set contact N graphic={
    circuit symbol lines,
    circuit symbol size=width 1.5 height 2,
    transform shape,
    ladder symbol=\pgfkeysvalueof{/pgf/contact N ladder symbol},
    shape=contact ladder
  },
  set coil graphic={
    circuit symbol lines,
    fill=none,
    circuit symbol size=width 2.4 height 2,
    transform shape,
    shape=coil ladder
  },
  set coil NA graphic={
    circuit symbol lines,
    circuit symbol size=width 2.4 height 2,
    crossed ladder=true,
    transform shape,
    shape=coil ladder
  },
  set var coil NA graphic={
    circuit symbol lines,
    circuit symbol size=width 2.4 height 2,
    ladder symbol=\pgfkeysvalueof{/pgf/var coil NA ladder symbol},
    transform shape,
    shape=coil ladder
  },
  set coil S graphic={
    circuit symbol lines,
    circuit symbol size=width 2.4 height 2,
    ladder symbol=\pgfkeysvalueof{/pgf/coil S ladder symbol},
    transform shape,
    shape=coil ladder
  },
  set coil R graphic={
    circuit symbol lines,
    circuit symbol size=width 2.4 height 2,
    ladder symbol=\pgfkeysvalueof{/pgf/coil R ladder symbol},
    transform shape,
    shape=coil ladder
  },
  set block graphic={
    circuit symbol lines,
    circuit symbol size=width 2.4 height 2,
    ladder symbol=\pgfkeysvalueof{/pgf/block ladder symbol},
    transform shape,
    inner sep=.5ex,
    shape=block ladder,
    every ladder block/.try
  }
}

\endinput