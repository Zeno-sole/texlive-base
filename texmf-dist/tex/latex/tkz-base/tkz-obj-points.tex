 % tkz-obj-eu-points.tex
% Copyright 2022  Alain Matthes
% This work may be distributed and/or modified under the
% conditions of the LaTeX Project Public License, either version 1.3
% of this license or (at your option) any later version.
% The latest version of this license is in
%   http://www.latex-project.org/lppl.txt
% and version 1.3 or later is part of all distributions of LaTeX
% version 2005/12/01 or later.
% This work has the LPPL maintenance status “maintained”. 
% The Current Maintainer of this work is Alain Matthes.

\def\fileversion{4.2c}
\def\filedate{2022/07/14 } 
\typeout{2022/07/14  4.2c  tkz-obj-points.tex} 
\makeatletter
%<--------------------------------------------------------------------------->
%                             init def point 
%<--------------------------------------------------------------------------->
%                            tkzDefPoint
%<--------------------------------------------------------------------------->
%                     macros complémentaires  pour def point
%<--------------------------------------------------------------------------->
\def\tkz@parsecoordinate#1{%
\tkz@getseparator#1,\@nil
\iftkz@polar 
   \tkz@getfrompolar#1\@nil
\else
   \tkz@getfromcart#1\@nil
\fi
}
\def\tkz@getseparator#1,#2\@nil{%
\ifx\tkzempty#2\tkzempty%
   \tkz@polartrue
\else
   \tkz@polarfalse
\fi
}      
\def\tkz@getfrompolar#1:#2\@nil{%
\edef\tkz@a{\fpeval{#1}}
\edef\tkz@r{\fpeval{#2}}
\xdef\tkz@polarrad{\tkz@r}
\xdef\tkz@polarangle{\tkz@a}
} 
\def\tkz@getfromcart#1,#2\@nil{%
\edef\tkz@x{\fpeval{#1}}
\edef\tkz@y{\fpeval{#2}}
\xdef\tkz@absc{\tkz@x}
\xdef\tkz@ord{\tkz@y}
}
%<--------------------------------------------------------------------------->
%           options pour point definition
%<--------------------------------------------------------------------------->
\pgfkeys{/tkzPt/.cd,  
         polar/.is if       = tkz@pt@polar,
         polar/.default     = true,
         polar              = false  
}
%<--------------------------------------------------------------------------->
%        tkz@ptStar  
%<--------------------------------------------------------------------------->
\def\tkz@ptStar[#1](#2,#3)#{%
\pgfkeys{/tkzPt/.cd, polar     = false} 
\pgfqkeys{/tkzPt}{#1}  
\iftkz@pt@polar%
  \edef\tkz@x{\fpeval{#3*cosd(#2)}}
  \edef\tkz@y{\fpeval{#3*sind(#2)}}
\else
  \edef\tkz@x{\fpeval{#2}}
  \edef\tkz@y{\fpeval{#3}}
\fi
\edef\tkzpt@xa{\fpeval{(\tkz@x-\tkz@init@xorigine)/\tkz@init@xstep}}
\edef\tkzpt@ya{\fpeval{(\tkz@y-\tkz@init@yorigine)/\tkz@init@ystep}}
\tkz@node
}
%<--------------------------------------------------------------------------->
% la macro defpoint les coordonnées cartésiennes ou polaires sont  traitées
% afin de pouvoir être adaptées avec les unités choisies
%<--------------------------------------------------------------------------->
\def\tkzDefPoint{\pgfutil@ifnextchar[{\tkz@DefPoint}{\tkz@DefPoint[]}}

\def\tkz@DefPoint[#1](#2)#{% 
\tkz@parsecoordinate{#2}
\iftkz@polar
   \tkz@ptStar[polar](\tkz@polarangle,\tkz@polarrad){tkz@coord@temp}
\else  
    \tkz@ptStar[](#2){tkz@coord@temp}
\fi
\tkz@DefPointEnd[#1]
}% 
\def\tkz@DefPointEnd[#1]#2{\coordinate[#1] (#2) at (tkz@coord@temp);}
%<--------------------------------------------------------------------------->
%                            tkzDefPoints
%<--------------------------------------------------------------------------->
\def\tkzDefPoints{\pgfutil@ifnextchar[{\tkz@DefPoints}{\tkz@DefPoints[]}}
\def\tkz@DefPoints[#1]#2{%
  \begingroup
  \def\tkz@parenthesis@for@err{(}%
  \def\tkz@maybe@parenthesis{#2}%
  \ifx\tkz@maybe@parenthesis\tkz@parenthesis@for@err
    \PackageError{tkz-base}{Syntax error  use {} instead of ()}{You need to use {} instead of ()}%
    \expandafter\@firstoftwo
  \else
    \expandafter\@secondoftwo
  \fi
  {\tkz@absorb@bad@arg}% 
  {\foreach \ptx/\pty/\name in {#2}{\tkzDefPoint[#1](\ptx,\pty){\name}}%
   \endgroup}%
}
\def\tkz@absorb@bad@arg#1){\endgroup}
%<--------------------------------------------------------------------------->
\def\tkz@node#1{ \pgfinterruptboundingbox
\path[coordinate](\tkzpt@xa,\tkzpt@ya) coordinate(#1);
 \endpgfinterruptboundingbox}  
% \def\tkz@drawnode#1{\path[coordinate](\tkzpt@xa,\tkzpt@ya) coordinate(#1);
%              \tkz@DrawPt{#1}
%              }
%<--------------------------------------------------------------------------->
% 
%<--------------------------------------------------------------------------->
\def\tkzRenamePoint(#1)#2{\coordinate (#2) at (#1);}
\def\tkz@RenamePointEnd[#1]#2{\coordinate[#1] (#2) at (\tkz@ax,\tkz@ay);}
\def\tkzGetPoint#1{\coordinate  (#1) at (tkzPointResult);} 
\def\tkzGetPoints#1#2{\coordinate  (#1) at (tkzFirstPointResult);
                      \coordinate  (#2) at (tkzSecondPointResult);}
\def\tkzGetFirstPoint#1{\coordinate  (#1) at (tkzFirstPointResult);}
\def\tkzGetSecondPoint#1{\coordinate  (#1) at (tkzSecondPointResult);}
\def\tkzDefShiftPointCoord[#1](#2)#3{%
                \begin{scope}[shift={(#1)}]
                     \coordinate  (#3) at (#2);
                \end{scope}
                           }%
%<--------------------------------------------------------------------------->
%            tkzDefShiftPoint
%<--------------------------------------------------------------------------->
\def\tkzDefShiftPoint[#1](#2)#3{%
   \tkz@@extractxy{#1}
   \tkz@ax\pgf@x %
   \tkz@ay\pgf@y %   
\begin{scope}[shift={(\tkz@ax,\tkz@ay)}]
   \coordinate  (#3) at (#2);
\end{scope}
}
%<--------------------------------------------------------------------------->
%                    Coordonnées d'un point 
%    result in #2x et #2y    #1 est le point et on récupère ses coordonnées
% usage soit A un point \tkzGetPointCoord(A){V} alors \Vx = xA et \Vy = yA
% en cm 
% tkzGetPointCoord avec [#1] cm ou bien pt ?? todo
%<--------------------------------------------------------------------------->
\def\tkzGetPointCoord(#1)#2{%
\begingroup
   \pgfextractx{\pgf@x}{\pgfpointanchor{#1}{center}}
   \pgfmathparse{\pgf@sys@tonumber{\pgf@x}/28.45274}
   \global\let\tkzresultx\pgfmathresult
   \global\expandafter\edef\csname #2x\endcsname{\tkzresultx}% 
   \pgfextracty{\pgf@y}{\pgfpointanchor{#1}{center}}
   \pgfmathparse{\pgf@sys@tonumber{\pgf@y}/28.45274}
   \global\let\tkzresulty\pgfmathresult
   \global\expandafter\edef\csname #2y\endcsname{\tkzresulty}
\endgroup
}  
\def\tkz@@extractxy#1{%
\pgfextractx{\pgf@x}{\pgfpointanchor{#1}{center}}
\pgfextracty{\pgf@y}{\pgfpointanchor{#1}{center}} 
}
\let\tkzGetPointxy\tkzGetPointCoord
%<--------------------------------------------------------------------------->
%%%%%%%%%. Label,draw
%<-------------------------------------------------------------------------->
%                   tkzLabelPoint          Affichage des LABELS pour un point
%<-------------------------------------------------------------------------->  
\def\tkzLabelPoint{\pgfutil@ifnextchar[{\tkz@LabelPoint}{\tkz@LabelPoint[]}} 
\def\tkz@LabelPoint[#1](#2)#3{%
      \node[label style,#1] at (#2) {#3};}%
%<--------------------------------------------------------------------------->

\def\tkzLabelPoints{\pgfutil@ifnextchar[{\tkz@LabelPoints}{\tkz@LabelPoints[]}}% 
\def\tkz@LabelPoints[#1](#2){%
 \foreach \point in {#2}{
      \node[label style,#1] at (\point) {$\point$};}
}%
%<--------------------------------------------------------------------------->
\pgfkeys{/tkzsetuppt/.cd,
      size/.store in       =  \tkz@pt@size,
      color/.store in      =  \tkz@pt@color,
      fill/.store in       =  \tkz@pt@fill, 
      shape/.store in      =  \tkz@pt@shape,
      size                 =  \tkz@euc@pointsize,
      color                =  \tkz@euc@pointcolor,
      fill                 =  \tkz@euc@pointcolor,
      shape                =  \tkz@euc@pointshape,
   /tkzsetuppt/.search also  = {/tikz},
 } 
\def\tkzSetUpPoint{\pgfutil@ifnextchar[{\tkz@SetUpPoint}{%
                                        \tkz@SetUpPoint[]}}
\def\tkz@SetUpPoint[#1]{%
\pgfqkeys{/tkzsetuppt}{#1}
% redefine point style with new values
\tikzset{point style/.style={draw         = \tkz@pt@color,
                             inner sep    = 0pt,
                             shape        = \tkz@pt@shape,
                             minimum size = \tkz@pt@size,
                             fill         = \tkz@pt@fill}}
}% end setup
%<--------------------------------------------------------------------------->
%
%<--------------------------------------------------------------------------->
\def\tkzSetUpLabel{\pgfutil@ifnextchar[{\tkz@SetUpLabel}{%
                                        \tkz@SetUpLabel[]}}
\def\tkz@SetUpLabel[#1]{%
\tikzset{label style/.style={#1}}
}% end setup
%<--------------------------------------------------------------------------->

\pgfkeys{/tkzautolab/.cd,
    center/.store in           = \tkz@center,
    dist/.store in             = \tkz@dist,
    dist                       = 0.15,
   /tkzautolab/.search also   = {/tikz},
}
\def\tkzAutoLabelPoints{\pgfutil@ifnextchar[{\tkz@AutoLabelPoints}{\tkz@AutoLabelPoints[]}}% 
\def\tkz@AutoLabelPoints[#1](#2){%
\begingroup 
\pgfqkeys{/tkzautolab}{#1} 
 \foreach \point in {#2}{
 \path (\tkz@center) -- ($ (\point) + \tkz@dist*($(\point)-(\tkz@center)$) $) node[/tkzautolab/.cd,label style,#1]{$\point$};}
\endgroup
}%
%<--------------------------------------------------------------------------->
%                    tkzSetUpPoint  définit la forme d'un point
%<--------------------------------------------------------------------------->
\pgfkeys{/tkzsetuppt/.cd,
      size/.store in       =  \tkz@pt@size,
      color/.store in      =  \tkz@pt@color,
      fill/.store in       =  \tkz@pt@fill, 
      shape/.store in      =  \tkz@pt@shape,
      size                 =  \tkz@euc@pointsize,
      color                =  \tkz@euc@pointcolor,
      fill                 =  \tkz@euc@pointcolor,
      shape                =  \tkz@euc@pointshape,
   /tkzsetuppt/.search also  = {/tikz},
 } 
\def\tkzSetUpPoint{\pgfutil@ifnextchar[{\tkz@SetUpPoint}{%
                                        \tkz@SetUpPoint[]}}
\def\tkz@SetUpPoint[#1]{%
\pgfqkeys{/tkzsetuppt}{#1}
% redefine point style with new values
\tikzset{point style/.style={draw         = \tkz@pt@color,
                             inner sep    = 0pt,
                             shape        = \tkz@pt@shape,
                             minimum size = \tkz@pt@size,
                             fill         = \tkz@pt@fill}}
}% end setup
%<--------------------------------------------------------------------------->
%                       Draw Point   
%<--------------------------------------------------------------------------->
\pgfkeys{/tkzdrawpt/.cd, 
  size/.code   = {\tikzset{point style/.append style={minimum size = #1}}},
  size         = \tkz@euc@pointsize,
  /tkzdrawpt/.search also  = {/tikz},
}
%<--------------------------------------------------------------------------    
\def\tkzDrawPoint{\pgfutil@ifnextchar[{\tkz@DrawPoint}{\tkz@DrawPoint[]}} 
\def\tkz@DrawPoint[#1](#2){%
\begingroup 
    \pgfqkeys{/tkzdrawpt}{#1}    
    \node[point style,/tkzdrawpt/.cd,#1] at (#2) {};%2016
\endgroup
}
%<---------------------------------------------------------------------------> 
\def\tkzDrawPoints{\pgfutil@ifnextchar[{\tkz@drawpts}{\tkz@drawpts[]}} 
%<---------------------------------------------------------------------------> 
\def\tkz@drawpts[#1](#2){%
\begingroup
   \pgfqkeys{/tkzdrawpt}{#1}        
   \foreach \point in {#2}{\node[point style,/tkzdrawpt/.cd,#1] at (\point) {};} %2016
\endgroup 
}
%<--------------------------------------------------------------------------->
%                                 PointShowCoord  
%<--------------------------------------------------------------------------->
\pgfkeys{/tkzprcoord/.cd,
          xlabel/.store in    = \tkz@xlabel,
          ylabel/.store in    = \tkz@ylabel, 
          xstyle/.code        = {\tikzset{xcoord style/.append style={#1}}},
          ystyle/.code        = {\tikzset{ycoord style/.append style={#1}}},
          noxdraw/.is if      = tkz@coord@noxdraw,
          noxdraw/.default    = true,
          noydraw/.is if      = tkz@coord@noydraw,
          noydraw/.default    = true,
          xlabel              = {},
          ylabel              = {},
          xstyle              = {},
          ystyle              = {},
          noxdraw             = false,
          noydraw             = false,    
  /tkzprcoord/.search also   = {/tikz},
}
\def\tkzPointShowCoord{\pgfutil@ifnextchar[{\tkz@PointShowCoord}{\tkz@PointShowCoord[]}}     
\def\tkz@PointShowCoord[#1](#2){%
\begingroup 
\pgfqkeys{/tkzprcoord}{#1} 
% 2019 for showcoord
 \iftkznodedefined{tkz@xline}{}{%
 \path (0,0) --(1,0) node(tkz@xline){};
 \path (0,0) --(0,1) node(tkz@yline){};
 }
\iftkz@coord@noxdraw\else\draw[arrow coord style,/tkzprcoord/.cd,#1] (#2)--(#2 |- tkz@xline);\fi 
\iftkz@coord@noydraw\else\draw[arrow coord style,/tkzprcoord/.cd,#1] (#2)--(#2 -| tkz@yline);\fi   
\ifx\tkzutil@empty\tkz@xlabel
\else
\path (#2)--(#2 |- tkz@xline)
  node[xcoord style] {\tkz@xlabel};
\fi
\ifx\tkzutil@empty\tkz@ylabel
\else
 \path (#2)--(#2 -| tkz@yline)
   node[ycoord style] {\tkz@ylabel};
\fi    
\endgroup
}
\let\tkzShowPointCoord\tkzPointShowCoord
\makeatother  
\endinput