% tkz-obj-eu-draw-circles.tex
% Copyright 2022  Alain Matthes
% This work may be distributed and/or modified under the
% conditions of the LaTeX Project Public License, either version 1.3
% of this license or (at your option) any later version.
% The latest version of this license is in
%   http://www.latex-project.org/lppl.txt
% and version 1.3 or later is part of all distributions of LaTeX
% version 2005/12/01 or later.
% This work has the LPPL maintenance status “maintained”.
% The Current Maintainer of this work is Alain Matthes.

\def\fileversion{4.25c}
\def\filedate{2022/09/23} 
\typeout{2022/09/23 4.25c tkz-obj-eu-draw-circles.tex} 
\makeatletter 
%<--------------------------------------------------------------------------–>
%                    tkzSetUpCircle  
%<--------------------------------------------------------------------------–>
\pgfkeys{%
    /tkzsetupcirc/.cd,
    color/.code              =   \def\tkz@circle@color{#1},
    line width/.code         =   \def\tkz@circle@linewidth{#1},
    style/.code              =   \def\tkz@circle@style{#1},
  /tkzsetupcirc/.search also =   {/tikz}
 } 
 %<--------------------------------------------------------------------------–>
 
\def\tkzSetUpCircle{\pgfutil@ifnextchar[{\tkz@SetUpCircle}{\tkz@SetUpCircle[]}}
\def\tkz@SetUpCircle[#1]{%
\pgfkeys{%
    tkzsetupcirc/.cd,
    line width                  = \tkz@euc@circlelw,
    color                       = \tkz@euc@circlecolor,
    style                       = \tkz@euc@circlestyle
}
\pgfqkeys{/tkzsetupcirc}{#1}
\tikzset{%
    circle style/.append style   = { %
    color                        = \tkz@circle@color,
    line width                   = \tkz@circle@linewidth,
    style                        = \tkz@circle@style,
    #1}
    }
}% end setup 
 %<--------------------------------------------------------------------------–>

\def\tkzDrawCircle{\pgfutil@ifnextchar[{\tkz@DrawCircle}{\tkz@DrawCircle[]}}
\def\tkz@DrawCircle[#1](#2,#3){%  
\begingroup 
\node [draw,circle through=(#3), circle style,#1] at (#2) {};  
\endgroup
}
%<--------------------------------------------------------------------------–> 
\def\tkz@multicircles#1 #2\@nil{% 
\protected@edef\tkz@temp{
\noexpand \tkzDrawCircle[\tkz@optcircle](#1)}\tkz@temp% 
\def\tkz@nextArg{#2}%
\ifx\tkzutil@empty\tkz@nextArg
     \let\next\@gobble
\fi
\next#2\@nil
}%
%<--------------------------------------------------------------------------–>
\def\tkzDrawCircles{\pgfutil@ifnextchar[{\tkz@DrawCircles}{\tkz@DrawCircles[]}} 
\def\tkz@DrawCircles[#1](#2){%
\xdef\tkz@optcircle{#1} 
\begingroup
   \let\next\tkz@multicircles
   \next#2 \@nil %    
\endgroup     
}% 
%<--------------------------------------------------------------------------–>
% #2 #3 rayon

\def\tkzDrawSemiCircle{\pgfutil@ifnextchar[{\tkz@DrawSemiCircle}{%
                                            \tkz@DrawSemiCircle[]}}
\def\tkz@DrawSemiCircle[#1](#2,#3){% 
\begingroup 
    \pgfpointdiff{\pgfpointanchor{#2}{center}}%
                 {\pgfpointanchor{#3}{center}}%
    \tkz@ax=\pgf@x%
    \tkz@ay=\pgf@y%
   \path(#2)--++(-\tkz@ax,-\tkz@ay)coordinate (tkz@pt); 
 \tkzDrawArc[#1,delta=0](#2,#3)(tkz@pt)
\endgroup
}%
%<--------------------------------------------------------------------------–> 
\def\tkz@multisemicircles#1 #2\@nil{% 
\protected@edef\tkz@temp{
\noexpand \tkzDrawSemiCircle[\tkz@optsemicircle](#1)}\tkz@temp% 
\def\tkz@nextArg{#2}%
\ifx\tkzutil@empty\tkz@nextArg
     \let\next\@gobble
\fi
\next#2\@nil
}%
%<--------------------------------------------------------------------------–>
\def\tkzDrawSemiCircles{\pgfutil@ifnextchar[{\tkz@DrawSemiCircles}{%
\tkz@DrawSemiCircles[]}} 
\def\tkz@DrawSemiCircles[#1](#2){%
\xdef\tkz@optsemicircle{#1} 
\begingroup
   \let\next\tkz@multisemicircles
   \next#2 \@nil %    
\endgroup     
}% 
%<---------------------------- Fill Circle  --------------------------------–>
\def\tkzFillCircle{\pgfutil@ifnextchar[{\tkz@FillCircle}{\tkz@FillCircle[]}}
\def\tkz@FillCircle[#1](#2,#3){%
\begingroup      
   \node [fill,circle through=(#3),#1] at (#2) {};   
\endgroup
}%
\def\tkz@multifillcircles#1 #2\@nil{% 
\protected@edef\tkz@temp{
\noexpand \tkzFillCircle[\tkz@optfillcircle](#1)}\tkz@temp% 
\def\tkz@nextArg{#2}%
\ifx\tkzutil@empty\tkz@nextArg
     \let\next\@gobble
\fi
\next#2\@nil
}%
\def\tkzFillCircles{\pgfutil@ifnextchar[{\tkz@FillCircles}{%
\tkz@FillCircles[]}} 
\def\tkz@FillCircles[#1](#2){%
\xdef\tkz@optfillcircle{#1} 
\begingroup
   \let\next\tkz@multifillcircles
   \next#2 \@nil %    
\endgroup     
}% 

%<--------------------------- Clip Circle  ---------------------------------–>
\pgfkeys{/tkzclipc/.cd,    
         out code/.is if         =  tkzClipOutCircle,
         out/.code               =  \tkzClipOutCirclefalse
}%
%<--------------------------------------------------------------------------–>
\def\tkzClipCircle{\pgfutil@ifnextchar[{\tkz@ClipCircle}{\tkz@ClipCircle[]}}
\def\tkz@ClipCircle[#1](#2,#3){%    
\tkzClipOutCircletrue
\pgfqkeys{/tkzclipc}{#1}
   \tkz@@CalcLength(#2,#3){tkzLengthResult}
   \iftkzClipOutCircle
     \clip (#2) circle (\tkzLengthResult pt);
   \else
     \clip (#2) circle (\tkzLengthResult pt) [tkzreverseclip] ;
    \fi
} 
%<--------------------------- Label Circle  --------------------------------–>
\def\tkzLabelCircle{\pgfutil@ifnextchar[{\tkz@LabelCircle}{%
                                         \tkz@LabelCircle[]}}
% [option]  (#2,#3) #2 center #3  un point du cercle  #4 angle #5 the label
\def\tkz@LabelCircle[#1](#2,#3)(#4)#5{%
\begingroup      
   \tkzURotateAngle(#2,#4)(#3)
   \node[label style,#1] at (tkzPointResult) {#5};        
\endgroup
}
\makeatother 
\endinput