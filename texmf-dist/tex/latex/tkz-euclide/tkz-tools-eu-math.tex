% tkz-tools-eu-math.tex
% Copyright 2022  Alain Matthes
% This work may be distributed and/or modified under the
% conditions of the LaTeX Project Public License, either version 1.3
% of this license or (at your option) any later version.
% The latest version of this license is in
%   http://www.latex-project.org/lppl.txt
% and version 1.3 or later is part of all distributions of LaTeX
% version 2005/12/01 or later.
% This work has the LPPL maintenance status “maintained”.
% The Current Maintainer of this work is Alain Matthes.

\def\fileversion{4.25c}
\def\filedate{2022/09/23} 
\typeout{2022/09/23 4.25c tkz-tools-eu-math.tex}     
\makeatletter
%<-------------------------------------------------------------------------->
% \tkzpointnormalised#
% tkzCalcLength
% \tkzGetLength
% \tkzpttocm
% \tkzcmtopt
% \tkzFindSlope
% option xfp

%  \tkzpointnormalised    normalise un point A-->A' tq ||v(OA')=1||
% équivalent de \pgfpointnormalised avec fp
% example
% \tkzpointnormalised{%
% \pgfpointdiff{\pgfpointanchor{A}{center}}
%              {\pgfpointanchor{B}{center}}}

% or
% \pgf@x=1 cm
% \pgf@y=12 cm 
% \tkzpointnormalised{}
%<--------------------------------------------------------------------------
\def\tkzpointnormalised#1{%
\pgf@process{#1}%
\edef\tkz@den{\fpeval{sqrt((\pgf@x)^2+(\pgf@y)^2)}}
\edef\tkz@coordx{\fpeval{\pgf@x/\tkz@den}}
\edef\tkz@coordx{\fpeval{round(\tkz@coordx,5)}}
\edef\tkz@coordy{\fpeval{\pgf@y/\tkz@den}}
\edef\tkz@coordy{\fpeval{round(\tkz@coordy,5)}}
\pgf@x = \tkz@coordx pt
\pgf@y = \tkz@coordy pt
}
%<-------------------------------------------------------------------------->
% restaure and save length
\def\tkz@save@length{\global\let\tkz@temp@length\tkzLengthResult}%
\def\tkz@restore@length{\global\let\tkzLengthResult\tkz@temp@length }% 
%<-------------------------------------------------------------------------->
%    \tkzCalcLength      Distance entre deux points en pt ou en cm  avec xfp 
% \veclen mais avec fp 
%  option cm le résultat est en cm sinon en pt with cm=false
%<-------------------------------------------------------------------------->
\pgfkeys{tkzcalclen/.cd,
       cm/.is if         = tkzLengthIncm,
       cm/.default       = true,
       cm                = true}   

\def\tkzCalcLength{\pgfutil@ifnextchar[{\tkz@CalcLength}{\tkz@CalcLength[]}}  
\def\tkz@CalcLength[#1](#2,#3){%
%\pgfkeys{tkzcalclen/.cd, cm = true}
\pgfqkeys{/tkzcalclen}{#1}%   
\begingroup
\tkz@@CalcLength(#2,#3){tkzLengthResult}
\iftkzLengthIncm 
   \pgfmathparse{\tkzLengthResult pt/1cm}
   \edef\tkz@xfpMathLen{\fpeval{round(\pgfmathresult,6)}}
   \global\let\tkzLengthResult\tkz@xfpMathLen  
\fi 
\endgroup
}%
\def\tkz@@CalcLength(#1,#2)#3{%
\pgfpointdiff{\pgfpointanchor{#1}{center}}%
             {\pgfpointanchor{#2}{center}}%
\edef\tkz@xfpMathLen{\fpeval{sqrt((\pgf@x)^2+(\pgf@y)^2)}}
\edef\tkz@xfpMathLen{\fpeval{round(\tkz@xfpMathLen,6)}}
\global\expandafter\edef\csname #3\endcsname{\tkz@xfpMathLen}
}
\def\tkz@@CalcLengthcm(#1,#2)#3{%
\pgfpointdiff{\pgfpointanchor{#1}{center}}%
             {\pgfpointanchor{#2}{center}}%
\edef\tkz@xfpMathLen{\fpeval{sqrt((\pgf@x)^2+(\pgf@y)^2)}}
\edef\tkz@xfpMathLen{\fpeval{round(\tkz@xfpMathLen/28.45274,6)}}
\global\expandafter\edef\csname #3\endcsname{\tkz@xfpMathLen}
}
%<-------------------------------------------------------------------------->
\def\tkzGetLength#1{%
\global\expandafter\edef\csname #1\endcsname{\tkzLengthResult}}  
%<-------------------------------------------------------------------------->
%     \tkzpttocm  passage de pt   cm div par 28.45274
%<-------------------------------------------------------------------------->
\def\tkzpttocm(#1)#2{%
\begingroup  
  \pgfmathparse{#1/1cm}
  \edef\tkz@mathresult{\fpeval{round(\pgfmathresult,5)}}
  \global\let\tkz@mathresult\tkz@mathresult
  \global\expandafter\edef\csname #2\endcsname{\tkz@mathresult}% 
\endgroup
}%
%<-------------------------------------------------------------------------->
%     \tkzcmtopt  passage de cm   pt mul par 28.45274
%<--------------------------------------------------------------------------
\def\tkzcmtopt(#1)#2{%
\begingroup 
   \pgfmathparse{#1*1cm}
  \edef\tkz@mathresult{\fpeval{round(\pgfmathresult,5)}}
  \global\let\tkz@mathresult\tkz@mathresult
  \global\expandafter\edef\csname #2\endcsname{\tkz@mathresult}% 
\endgroup  
}% 

%  Schrodinger's cat idea 03/01/20
\tikzset{xfp/.code={%
\pgfmathdeclarefunction*{veclen}{2}{%
\begingroup%
    \pgfmath@x##1pt\relax%
    \pgfmath@y##2pt\relax%
    \edef\tkz@xfpMathLen{\fpeval{sqrt((\pgf@x)^2+(\pgf@y)^2)}}   
    \pgfmath@returnone\tkz@xfpMathLen pt%
\endgroup%
}}}%
%<---------------------------------------------------------–>
\def\tkzSwapPoints(#1,#2){
  \pgfnodealias{tkzPointTmp}{#2}
   \pgfnodealias{#2}{#1}
  \pgfnodealias{#1}{tkzPointTmp}}
%<---------------------------------------------------------–>
\def\tkzPermute(#1,#2,#3){
\tkzURotateWithNodes(#1,#3,#2)(#3)  \tkzGetPoint{tkzpt}
\tkzURotateWithNodes(#1,#2,#3)(#2) \tkzGetPoint{#2}
\tkzSwapPoints(tkzpt,#3)}
%<---------------------------------------------------------–>
\def\tkzDotProduct(#1,#2,#3){%
\begingroup
\pgfextractx{\pgf@x}{\pgfpointanchor{#1}{center}}%
\pgfextracty{\pgf@y}{\pgfpointanchor{#1}{center}}% 
\tkz@ax\pgf@x%
\tkz@ay\pgf@y%
\pgfextractx{\pgf@x}{\pgfpointanchor{#2}{center}}%
\pgfextracty{\pgf@y}{\pgfpointanchor{#2}{center}} 
\tkz@bx\pgf@x%
\tkz@by\pgf@y%
\pgfextractx{\pgf@x}{\pgfpointanchor{#3}{center}}%
\pgfextracty{\pgf@y}{\pgfpointanchor{#3}{center}}% 
\tkz@cx\pgf@x%
\tkz@cy\pgf@y%
\edef\tkz@@dotprod{\fpeval{round(abs((\tkz@bx-\tkz@ax)*(\tkz@cx-\tkz@ax)+(\tkz@by-\tkz@ay)*(\tkz@cy-\tkz@ay))/(28.45274*28.45274),5)}}
\global\let\tkzMathResult\tkz@@dotprod
\endgroup}
%<---------------------------------------------------------–>
 \def\tkzGetResult#1{%
   \global\expandafter\edef\csname #1\endcsname{\tkzMathResult}}
%<---------------------------------------------------------–>
% #1,#2 and #3 aligned
\def\tkzIsLinear(#1,#2,#3){%
\begingroup
\tkz@@CalcLengthcm(#1,#2){tkz@la}
\tkz@@CalcLengthcm(#1,#3){tkz@lb}
\tkzDotProduct(#1,#2,#3)
\edef\tkzResult{\fpeval{abs((\tkzMathResult)-(\tkz@la)*(\tkz@lb))}}
\ifdim \tkzResult pt < 0.01 pt\relax%
\global\tkzLineartrue
\else
\global\tkzLinearfalse
\fi
\endgroup
}
%<---------------------------------------------------------–>
% syntax : vec(#2,#1) ortho vec(#3,#1)
\def\tkzIsOrtho(#1,#2,#3){%
\begingroup
\tkzDotProduct(#1,#2,#3)
\edef\tkzResult{\fpeval{abs(\tkzMathResult)}}
\ifdim \tkzResult pt < 1 pt\relax%
\global\tkzOrthotrue
\else
\global\tkzOrthofalse
\fi
\endgroup
}
%<---------------------------------------------------------–>
%<---------------------------------------------------------–>
% \tkzPowerCircle(M)(O,A) --> OM^2-OA^2
\def\tkzPowerCircle(#1)(#2,#3){%     
\begingroup 
\tkz@@CalcLengthcm(#2,#3){tkz@ra}
\tkz@@CalcLengthcm(#1,#2){tkz@om}
\gdef\tkzMathResult{\fpeval{round(\tkz@om*\tkz@om -\tkz@ra*\tkz@ra,5)}}
\endgroup
}
%<---------------------------------------------------------–>
\def\tkzDefRadicalAxis(#1,#2)(#3,#4){%
\begingroup
\tkz@@CalcLengthcm(#1,#3){tkz@d}
\tkz@@CalcLengthcm(#1,#2){tkz@ra}
\tkz@@CalcLengthcm(#3,#4){tkz@rb}
\edef\tkzMathResult{\fpeval{\tkz@d-(\tkz@ra+\tkz@rb)}}
\edef\tkzMathResultb{\fpeval{abs(\tkz@d-(\tkz@ra+\tkz@rb))}}
\edef\tkzMathResultc{\fpeval{abs(\tkz@d-abs(\tkz@ra-\tkz@rb))}}
\ifdim \tkzMathResultc pt < 0.1 pt\relax%
 \tkzURotateAngle(#2,90)(#3) \tkzGetPoint{tkzFirstPointResult}
  \tkzURotateAngle(#2,-90)(#3) \tkzGetPoint{tkzSecondPointResult}
\else
\ifdim \tkzMathResultb pt < 0.1 pt\relax%
 \tkzURotateAngle(#2,90)(#3) \tkzGetPoint{tkzFirstPointResult}
  \tkzURotateAngle(#2,-90)(#3) \tkzGetPoint{tkzSecondPointResult}
  \else
\ifdim \tkzMathResult pt > 1 pt\relax%
  \tkzURotateAngle(#1,60)(#3)    \tkzGetPoint{tkz@aux}
  \tkzInterCC(#1,#2)(tkz@aux,#1) \tkzGetPoints{tkz@pta}{tkz@ptb}
  \tkzInterCC(#3,#4)(tkz@aux,#1) \tkzGetPoints{tkz@ptc}{tkz@ptd}
  \tkzInterLL(tkz@pta,tkz@ptb)(tkz@ptc,tkz@ptd) \tkzGetPoint{tkz@pta}
  \tkzUProjection(#1,#3)(tkz@pta)  \tkzGetPoint{tkz@ptb}
  \pgfnodealias{tkzSecondPointResult}{tkz@ptb}
  \pgfnodealias{tkzFirstPointResult}{tkz@pta}
\else
\tkzInterCCR(#1,\tkz@ra)(#3,\tkz@rb){tkzFirstPointResult}{tkzSecondPointResult}
\fi
\fi
\fi
\endgroup
} 
\makeatother 
\endinput